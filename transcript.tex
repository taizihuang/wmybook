
\documentclass[twocolumn]{ctexart}
\usepackage{ctex}
\usepackage{graphicx}
\usepackage{titlesec}
\usepackage[hyphens]{url}
\usepackage[colorlinks=true, urlcolor=blue, linkcolor=black]{hyperref}
\usepackage{geometry}
\usepackage{cancel}
\setCJKmainfont{SimSun}
\newCJKfontfamily\Kai{KaiTi}
\newCJKfontfamily\Hei{SimHei} 
\setcounter{secnumdepth}{0}
\setcounter{tocdepth}{1}
\titleformat*{\section}{\centering\Large\bfseries }
\titleformat*{\subsection}{\centering}
\titlespacing*{\subsection} {0pt}{0pt}{10ex}
\geometry{a4paper, scale=0.85}
\begin{document}
\begin{titlepage}\vspace*{8cm}\begin{center}{\Huge\bfseries\heiti 王孟源访谈}(v0.1.0)\end{center}\end{titlepage}\newpage
\tableofcontents
\newpage
\pagestyle{plain}
\twocolumn[\begin{@twocolumnfalse}
\section{霸权交替、民选体制、武统、台湾的未来、}
\subsection{20170906}
\end{@twocolumnfalse}]*全文语句有轻微调整,使文字更流畅



史东 00:38

各位朋友你好,我是史东。再一次地说一声:谢谢您按时地收听以及收看八方论谈这个节目。我首先要跟你打声招呼。非常地抱歉,因为我们器材有点问题,所以说现场节目晚了五分钟,到现在为止晚了六分钟开始,希望不会给你造成任何的困扰。



在今天节目中,我为您准备的是一个相当重要的题目,也是一个相当大的题目。在今天节目中,我非常高兴为您请到的这位贵宾叫做王孟源王先生。我想对于他,你可能不会陌生,因为他的这个部落格,他的博文被很多人赞赏。那么事实上也是一位听众以及很多观众和我联系,希望能够在节目中有机会请王先生来和我们谈谈他专长的一些事情。这也是今天我们节目中为您请到王孟源先生来和我们见面的一个非常主要的原因。当然在今天节目中,我非常高兴能够通过他,来和我们谈谈他对今天我们的这个议题,他有一些什么样的看法、什么样的想法。



今天的议题我再一次地重复,就是《当台湾被统一》。我做这个节目,我一个主要的思维就是说:我希望在我们都知道,在一个不可避免的两岸关系的结局之中,能够找出一个最好的结局,或者最好的结果。当然了,这个和我自己生性比较乐观的个性有点关系。所以说在今天节目中,我们和王孟源王先生聊的事情,包括:是不是已经武统了,以及统一之后他对于台湾以后所面临的种种的政治、军事、外交、教育,还有各种文官制度方面的事情,是什么样的看法。我们想听听他的意见。



现在我们就把王先生带进我们的画面之中。王先生,首先说一声谢谢,再说一声欢迎。



王孟源 02:55

很荣幸能够上你的节目。



史东 02:57

非常好。今天这个议题《当台湾被统一》,我们在节目之前稍微聊了一下。你似乎认为,武统是一个已经不可避免的一种命运、一种现象,是吗?



王孟源 03:16

我们在开始谈这个问题之前,我事先讲一下:我本身是在台湾长大的,我全家包括我的父母跟我的弟弟都是深绿,他们甚至觉得民进党还不够激进。我今天所讲的一切话题,不是我自己私心想要怎么样,而是客观地来评估最可能的未来会怎么发生,所以希望大家不要有任何的误解。



史东 03:51

我觉得你这个点的非常的好。这也是我从一个主持人、制作人的角度,我觉得非常重要的一点,就是你所提出来的,是一个客观的观察跟评论,甚至说是一种英文叫做 Educated Guess,一种具有常识的推理。我觉得这才是一种更重要的知识,对不对?



王孟源 04:19

因为我自己所受的训练是纯粹科学的,而且我真的相信用现实和逻辑来做推论,能够用绝对的理性来分析。所以如果习惯看我的部落格的人应该都知道,我坚持客观的理性的态度。所以今天所谈的一切,都是客观的立场来看,最可能会发生的未来是什么样子,而不是我们自己主观认为什么是最好的。我在部落格上面也会偶尔谈到我自己主观、我希望的结局是怎么样的。但是我都会特别地声明——这是主观的期望,而不是客观的事实。



那么言归正传,来谈你刚刚问的问题。目前台湾政经和媒体方面的人物在谈到未来展望的时候,我觉得都有一个很严重的误区,就是他们认为自己还有相当的角色可以扮演,这其实是非常非常错误的。台湾未来的归属,是世界国际局势的大棋盘里面的一个小棋子,它这个棋子将来会怎么样动,完全是必须遵循世界国际变化大趋势来进行的。过去这十五、二十年——我在部落格上面常说,世界上有三个大的问题。



第一大的问题是贫富不均。这是因为资本主义市场经济,自然就会造成贫富不均。唯一能够打破这个惯例的是全球性的灾难。上一次有全球性的灾难,已经是七十多年前了,所以资本主义本身、市场经济本身的缺陷,就变得很明显化。



史东 06:46

七十年前那个灾难,是哪一个灾难?



王孟源 06:52

二战。20世纪刚好是全世界全人类多灾多难的一个世纪,尤其是在前半。所以我们在20世纪之后,并没有贫富不均的问题。因为一战跟二战把这个问题给解决了。但是一旦长期和平以后,自然就会贫富不均。法国最近一个经济学家有一本很有名的书——就是《21世纪资本论》。那里面提到一个很简单的逻辑:就是他去看资本报酬率(资本聚集之后成长的速度)跟GDP成长的速度(整个经济大小的成长速度)一相比,发现前者永远都是比后者大。就是在和平时期,都是前者比后者大。



再重复一遍:前者是资本成长率(也就是资本累积的速率),然后后一个是全球的GDP成长率(也就是整个人类经济的规模)。所以这一旦确立,也就是说,在和平时期是前者比后者快。那么你用简单的逻辑就可以知道:一旦资本累积之后,资本所占世界经济的比例会越来越大,指数性地增长。我们在过去七十多年,就是目击到这个结果。然后我们现在看到很多热钱流来流去,都是这样的结果。这个跟台湾没有太大的关系,因为台湾不是国际资本运作的一个主要战场。



第二个问题——我刚刚说21世纪有三大问题——这第二个问题是全球暖化。这个跟台湾也没有太大的关系。当然,全球暖化的结果就是有很多比较强的风暴,未来水灾、旱灾都会更严重,但这是一个长期的问题。我们现在讲台湾,讲的是十年、二十年的未来了。



真正跟台湾有最大关系的,是21世纪面临的第三个问题——就是全球霸主交接的问题。全球霸主在一百年前,由英国转交到美国的手里。我们这一代人很有幸,很有运气。我常常在博客上讲:我们是很幸运的一代,因为我们现在所处的是人类历史上一个很有趣的时间——就是霸主交接。这基本上是每一百年一次,这个霸主从美国要转交到中国去。台湾基本上是中美博弈之间的一个棋子。从 1949 年蒋家国民政府退守到台湾之后,之所以会不统不独保持现状,就是因为美国要预留一手。它认为台湾是对付中国的一个很有用的棋子。这个你用比拟的话,比如说,现在俄国占据东乌(乌克兰的东部),也是让它不统不独,而且也不直接占领,为什么?占领以后就跟乌克兰成了不共戴天之仇,还给乌克兰之后,乌克兰还是跟你捣鬼,而且可以专心地统一建设力量强大。但是如果把中国的这么一块不能够切割的领土,这样不统不独地留着,就是有把柄在手里,高兴了就可以去扯一下。



史东 11:16

所以很多人说,台湾是美国用的非常巧妙而且非常有效的一个杠杆。



王孟源 11:24

对。其实如果大家去看看中国古代历史也是有例子的。不过我想在现代,因为这种霸权的思想更是普遍。像俄国基本上是一个区域霸权,它玩美国(全球霸主)同样的把戏。将来在霸主交接之后,对美国最好的、最有利益的是台湾不统不独。对中国最有利益的——无论是意识形态,还有对内宣传的传统——都不可能接受台湾无限地不统不独,所以这个问题必须要解决,尤其是有一个绝对的大限,就是建国100周年。不可能在建国100周年之后,台湾还不是中国的一个行政区。如果他们接受这样的话,他们的整个政权的正统性完全失去。所以我们一旦看到 2049 年是一个大限之后,你就会知道,它必须要在那之前解决。而且由于近来的发展,解决是越早越好。但是由于霸权交接的过程,不是马上可以发生的,所以不一定能够马上,必须要把它推迟一段时间。换句话说:中共要强行武统,其实在10年、15年之前,在军事上就有底子。但是当时以它们的实力,如果强行地武统的话,在国际外交舆论上所付出的代价极为严重,基本上会完全打断它们现代化的进程。所以在两者相权之下,统一台湾跟中国本身现代化建设相比,还是一个相当次要的议题,所以就一直拖下去。拖到什么时候?必须要拖到它的实力,已经能够完全排除美国在西太平洋的干涉。



而且最重要的是:美国即使要利用武统之后,在国际舆论上或者国际贸易规则上打击中国,也不会一呼百应。这里面最重要的,就是像澳洲跟欧洲。我们知道日本是绝对——除非政权更迭——否则一定是跟着美国来搞。所以中国最近搞这个一带一路,固然是有经济上产业升级,然后输出产能的这些考虑,但实际上在大战略上面也是很有意。一带一路基本上是要把中国跟欧洲连在一起,那么一旦欧洲跟中国的利益绑在一块之后,中国在区域上进行军事或者其他外交行动,就有很大的自由。因为美国不管说什么,欧洲都不会参与。欧洲不参与,美国的经济体量比中国还小,那么自然就没有没有什么好搞的。所以我认为:台湾必然会统一,而且这个统一的时间很容易猜,大概就是10-20年的未来。十年之后,中国基本上就会控制整个东亚跟西太平洋...然后是跟欧洲贸易结合的紧密度,然后是否有导火线...



台湾在这个过程中有什么话语权?它其实只有两个很次要的话语权。第一个是,台湾如果愿意文统的话,在过去 15-20 年有很多机会。台湾直接文统的话,就可以绕过美国的阻挠,其实长期来看是台湾最好的未来。但是那个机会已经过了,现在你台湾就算(想)——现实是不可能完全文统的,台湾的政治现实是不可能文统的——但是现在,即使台湾说我要文统,大陆也不会接受,为什么?就是香港占中(在两年多前)。香港原本就是...



中共很谦虚的。中共不像欧美那样子,自认为自己的政治体系是世界第一,完美的不会有缺失。它其实永远都是摸着石头过河,所以他们一直都觉得目前的这个有点——用欧美的话语来说——专制独裁的体制,不一定是永远能够适用的。当初二十年前,香港被重新吸收进去。回归之后,它就把香港故意留着,作为一个政治的实验室、一个样板。一方面是给台湾看,一国两制可以怎么样。但更重要的,一般台湾人没有注意到的是,它本身也是要敞开心胸,来看看香港能搞出什么好东西来。那么搞出什么好东西来了?没有,乱七八糟的,验证了欧美的所谓的民选民主制度,事实上是一个奢侈品,必须要有非常高明而且深刻的文化跟人民素质,才能够搞得像样。香港不够格,台湾更不够格。中共对自己人民品质的考虑是什么?大部分还是农民嘛,对不对?至少这一代是绝对不够格搞这种民选。因为你搞市场经济本身都已经有这么多的问题。



市场经济就是假设每一个市场的参与人都是一个理性的所谓的 Homo economicus。美国的经济系的人,真正有这么一个名词,就是 Homo economicus——不是 Homo Sapien。它假设每一个人都是理性的,而且是全知全能,就是几亿个上帝来做经济。当然,实际上搞出来的结果,就是有那个大衰退了,还有金融次贷危机这些东西。但是至少在经济上,你有一个很特别的机制叫做 Arbitrage,中文翻译是套利。我觉得翻译的不太好,这不是真的你把东西放到了一个套子里面去,弄出利益了。Arbitrage 就是价格不对的时候,即使大多数人都在吹捧一个泡沫,但是如果有少数的人,他确定自己是对的,他可以卖空,那就可以把价格压回合适的阶段。你看看世界上搞出泡沫的东西,在现代来说,大部分都是房地产这样的东西。为什么?因为房地产没办法卖空。最近这个比特币搞成几万几万这样一直涨上去,涨了几百几千倍,为什么?没办法卖空。你没办法卖空,就没办法套利,没办法套利,就没有经济效率可言,市场就完全失效。



史东 20:33

特别是对那些玩金融的财团。



王孟源 20:39

对。套利的人其实是市场经济有效率的一个最重要的机制。套利是市场经济最重要的机制。我可以再说两遍。



史东 20:58

我在想一个事情,你讲的套利这个事情,和我们老中讲的一句话叫买空卖空有多大的关系?



王孟源 21:11

买空卖空纯粹是杠杆,是 Leverage。套利的人当然必须要加杠杆,但是它本身跟杠杆没有绝对的关系。这个套利纯粹只是让少数的理性人,能够纠正大多数的非理性。但是民主政治其实上它是一个把市场经济的机制,搬到政府政策跟政客做决定的一个机制。



史东 21:47

所以说王先生,我现在想把这个话题再带回来一点。因为你讲到 21 世纪的三个重要的事情。你第三点就是讲到了这个霸主的交接。然后从霸主的交接里,谈到了经济的这方面的现象、跟经济这方面的问题,对不对?然后你就比较了这个中国的状况和西方的状况,西方的文化条件——美国,以欧洲为主。



王孟源 22:23

我现在讲的,事实上这都是有关系的。民主基本上就是市场经济搬到政治领域里面去。但是民主是一人一票,所以你没办法套利。所以它必然是非理性的多数会淹没理性的少数。我们现在看到,不止在欧美,更多的是在这种开发中国家,例如台湾或者是香港这种有民选制度(的地区)。但是事实上,民选制度只是把事情搞得越来越乱。



这其实都跟我们现在讲这个统一的话题非常有关系。因为中共在过去的二十年,他们从邓小平刚刚去世的时候——1997 年,原本是认为民主民选制度,有可能成为中国未来这个体制的一部分。但是过去这二十年的历史,完全否定了。因为他们是以实践为主,他们是尊重现实,而不是迷信一个理念。那么既然民主的缺陷这么多,而且台湾这个社会已经证实了,它非常不够格搞民主。那么在统一之后,就不能够再继续搞民主——就是所谓的地方自治的民选。



这就是为什么绕了这么一大圈,虽然你问的是有关台湾统一的问题,但是他们必须要民主跟市场经济这些东西(联系到一起)。因为你不考虑这些事情的话,你没办法解释为什么大陆...就是中共政权会怎么看。你把自己放到中共政权的角度上来看这个问题的话——台湾必须要统一,但是台湾搞民选一定会乱七八糟。其实我认为,这是因为台湾没有经过真正的土改,所以原本的乡村的地主,后来都变成所谓的企业家,去搞那个房地产去了。然后他们也就变成地方政客。所以不管你怎么选,都是这票人出来。民进党也好,绿的、蓝的、国民党也好,都是这票人。都是他们的利益霸占著国家、社会。地方势力,就是土豪势力。



那么中共很聪明的,他们是有几千几百个智囊,一天到晚就是有组织、有系统地在分析这些事情。他不可能让台湾在统一之后,还继续搞这样的制度跟这样的现象。那么你如果文统的话,不能够改,必须要一国两制。一国两制就是维持现状。那维持现状是不能接受的,那么就不可能文统。你现在台湾去求大陆文统,他们都不会接受,唯一的能够强制放入一国一制来搞的,就是武统。所以我们现在终于回到你原先问的问题,因为你问的问题太大了,所以我需要...



史东 26:10

对,我觉得你回答的非常好,有关于这个文统、武统的问题。当然我们都(看到)在各种的网上网民的反应,特别是中国大陆网民的反应,也都认为武统的这个呼声越来越高。可是在最近有一个消息,就是国台办的前副主任王在希先生,他在接受记者访问的时候他说:和统可能性渐渐在消失,这就符合你刚刚讲的这些事情。但是他说,他还是不排除以战逼和的这种北平模式。以战逼和的,也就是武统的一部分,对不对?



王孟源 27:04

因为你真正武统——虽然台湾国军士气已经低到这个地步——基本上也就是五六天的事情。如果有超过三个旅没有马上投降,我就觉得很了不起了。但是,虽然是五六天的事情,还是会死伤一两千人的,而且有一部分会是平民。这种死伤能够避免还是最好。但这是我们个人主观的偏好。客观来说,如果以战逼和不能够达到长治久安的目的,不能够让台湾无条件地接受中共未来的计划,那么他们还是必须要打。因为长痛不如短痛,对不对?你如果接收了一个地方,结果闹成像香港这个样子,出了一大堆民主斗士。其实台湾的民主斗士,已经比香港还多了,这些都是被英美洗脑过的人。美其名是民主斗士,事实上是被外国洗脑过的间谍。他们的主张,即使是在完全控制自己的土地上的时候,仍然是搞得民不聊生,而且消耗了极大的财富、人民的福祉。那么将来如果是中共要来,这些人成事不足,败事有余。所以必须要能够有...中共必然会坚持要有能够处理他们的一些自由。



所以不管是武统或以战逼和,它的差别,也就是在有没有那五六天真正的开枪开炮,然后有没有一两千人死亡,我个人就希望能够避免。但是必然的结果是——中共不可能再搞一国两制,不可能再搞地方自治。这里的意思是民选。所以现在像我爸爸妈妈,还有我弟弟,他们讲到说你们这些人都对中共说好话,是不是准备要当特首。



你们还在做梦,这个台湾文统的时机过去了,早就过去了,现在已经过了,台湾现在唯一能够影响这个进程的,就是把这个国债看看能够搞得有多大。基本上国民党还有点节制,马英九虽然在政治、社会的政策上很差劲,至少在财政上还有点节制。但是蔡英文基本上就是借钱,这个就是老鼠会嘛,对吧?拿未来下一代的钱...



史东 30:31

基本上你觉得现在他们这个台湾政治人物的想法是把这个窟窿,把这个洞捅的越大越好。因为不管谁来接收之后,这个洞就是就是他们的事情,并不是我们的事情。



王孟源 30:46 

是。其实是为什么大陆可能会越早统一越好。就是在 2025 年能够统一就赶快统一吧,要不然,它这个洞一年就增长 10\% 的话,这个八年就加倍,特别是你再拖 16 年变成 4 倍。而且这些人捞了钱以后...因为台湾现在的政客基本上不管蓝绿,都是这些土豪,或者是土豪台前的代言人。像蔡英文是土豪兼代言人,她自己本身就是土豪出身。那他们如果是统一之后,一定是跑到美国或者日本,去当他们的流亡政府,或者是去拿他们的诺贝尔和平奖——这可能不行,因为挪威最近知道他们要卖三文鱼给中国,所以不能够再搞这些东西,拿不到诺贝尔奖。



史东 32:06

对,这个流亡的日子,也有流亡的日子可以过。所以现在如果我是蔡英文的话,我也会说今天有这个机会在台上,我能够捞多少就捞多少。我想这个也是很容易理解的一种心态。我想把话题再稍微往回带一点。您刚刚谈到了这个——就从北京这条路来看,我们可以揣测,这也是合理的揣测:在没有统一台湾之前,北京的想法一定是希望能用一个成本最低的方式。这个成本是——你可以用各种各方面的角度来看——成本最低的方式来统一台湾。统一之后,您刚刚也提到了,他希望能够有一个长治久安的台湾,而不是一个动乱的台湾。那么这两个前提,如果我们都同意的话,你觉得譬如说,我今天举出一个(比较)稍微确实一点的可能性,你觉得把台湾列为一个中华人民共和国行特别行政区的可能性大不大?



王孟源 33:18

我想行省或者特别行政区其实就是一个名字嘛,实际上必须要特别处理的。因为台湾人民从李登辉开始到民进党,被洗脑了二十年。所谓的民主化跟多元化,基本上就是非理性化。这是李登辉搞日系台独弄出来的。台湾人如果没有变蠢变笨,变的极度的非理性变疯,他这个日系台独就搞不起来嘛。因为无论如何,这不但是违反理性...像用爱发电这种东西,一听就是非理性的。



史东 34:08

你说发电这个事情,不但是非理性,还是怎么样?



王孟源 34:13

用爱发电。这就是我在我的博客上常常遇到台湾年轻这一代来说:就是你说这些事情错的,但是人家总是有权利发言。我说发言的权利,意思是他不犯法,不犯法并不代表他是对的。你可以拿你自己的屎,往你自己的脸上抹。但是这不代表是正确的。李登辉的意思就是,把大家搞成这个态度,就是大家都拿屎、拿大便往自己的脸上涂,或者是互相丢来丢去。这个台湾,居然接受说这是合理的、应该的、正常的。



你这样不但是愚蠢,而且是有严重的经济代价。就是企业,即使是没有意识形态的企业主,一看也知道这对他们的生意是有害的。那企业在一般的民主体制里面都会有很大的发言权,尤其像美国这样,基本上是财团垄断,他们才是幕后的权力中心。但是台湾李登辉为了要搞日系台独,必须把企业的发言权也阉割掉。那阉割掉唯一的办法就是搞这些什么民主化、多元化制度,自我麻醉的这些理念,基本上可笑到极点了,完全跟事实脱节。而且不但是他们的信仰与事实脱节,而且他们的态度就是鼓励与事实脱节,实在是很可笑的一件事情。



史东 36:33

我想趁这个时间,比较具体地把一些我所想到的统一之前跟统一之后的这种可能性,或者这种现象的改变,我想一个一个提出来。我想听听你对这些事情的改变的看法,以及你说会不会改变,如何改变,或者是根本就毫无问题的是没办法改变的事情。第一个,我想宪法,中华民国宪法就是早早就丢出窗外了,这根本没有什么话说了,对不对?跟着宪法走的就是中华民国的中央的政治体制,也就也就失去了,对不对?根本不用再谈了。另外就是跟着中华民国的军队,你觉得会有一个什么样的情况?



王孟源 37:27

军事、外交,这些都是中央的权限,绝对都是一笔抹清。就是中共一旦进来以后,就是人民解放军进驻。台湾其实在战略上对中共只有两个有意义的东西。至少在目前来看,第一个是台积电在经济上有点价值。第二个是花莲那一带的海军基地,因为大陆一个很大的问题是它被美国用第一岛链封锁之后,它的潜艇很难进入太平洋。但是花莲是一个相当好的军港,而且直接面对太平洋一出几公里就是很深的水,很适合核潜艇,一出动就不知所踪,真的找不到。而且作为空军基地用来包围和威胁日本也是一个相当合适的。这是台湾对中共的两个小小的战略价值。那么除了这些,台湾没有什么,你全部完全就是一个负担。在目前来说,所以为了长治久安,必须要把省级以上的机关基本上完全抹杀掉重建。县级以下,当然县长不能够再选,但是必须要官派。但是民政系统必须要保存,法律系统保存,但是那个法官的选拔必须要重来。



史东 39:14

重来是什么意思?也是官派吗?还是怎么样?



王孟源 39:22

可能要重新的考虑。我想他们一定有很多议案,而且不会只有一个议案。所以我一个旁观者,实在是不能够确定这些细节,他们会怎么选择。



史东 39:35

对,我对这个完全同意。但是我们今天谈这个问题,就希望至少能够带起所有关心海峡两岸未来的情势的人的一些讨论。那么这些讨论当然是和北京的决定会有多大的关系,我不敢说,但至少是一种讨论。我觉得这也是应可能会有会有益处的。那么您刚刚提到了这些中央的事情...那根本就是要丢出窗外了,就是 throw out of the window ,根本不用谈的事情。你刚才提到一国两制是根本不用了,特别行政区你觉得是没有什么特别,也许叫做一个特别行政区,但是基本上就是一国一制了,对不对?



王孟源 40:30

说是一国一制,也不一样,因为一般的行省没有这么多的头痛麻烦的问题嘛。他必须有很多外来的干部来做这个交接转换。你说是一个行省,也不是行省,所以说是一个特别行政区,也不是特别行政区。它比较像是当初占领新疆以后,可能会有一些新疆兵团。



史东 40:56

你觉得最好的。你如果从一个法律的位阶上来看,台湾被统一之后的台湾,就是你觉得对北京来来讲,在法律的位阶上,他应该是和中华人民共和国其他的行省是同样的位阶,还是比较高?还是比较低。



王孟源 41:15

当然会变成一个行省。长期来说他们当然是希望能够做一个行省,但是至少 5-10 年必须要有一个过渡的阶段。这个过渡的阶段是整个行政跟社会结构都必须要逐步适应,而且是阶段性的,可能每一年都不一样,渐进性的。一开始的时候必须要有几万甚至几十万的干部来官派、来接收。然后慢慢地本地化。这个本地化:一方面你把本土的人才...,另一个方面是外来的共产党干部也逐步地就在台湾生根,这是一个双向的过程。而且我刚刚说过,大概是 5-10 年。



台湾目前的人物如果有眼光,而且是真正关心台湾的未来的话,可以做的基本接近是零。但是这个在这个零上面有一点零头,有两件事:第一个是你不要把国债搞得太快,国债搞得越快。大陆接收的那个时间会越快;第二个是不要再担心什么选举了。因为台湾非理性的老百姓被洗脑到这个程度,你有理性的人根本不可能选择这样嘛。除非是像宋楚瑜那样子,为了要搞到三个立法委员,才能够有个党团,这样才能够有政治资本来游说捞钱。



你如果是真正关心台湾的人,像是洪秀柱或者是郁慕明这样的人,我还不懂他们为什么还在关心选举,他们跟选举没有关系。因为再选举上也就是10-15年,一定是都是胡搞。只是他们又不是像宋楚瑜那样子想要捞钱的人,他们真正应该关心的是台湾在过渡的这段时间,怎么样跟共产党派来的这几万或者几十万干部合作。从一个社会的观点来说,就是社团之类的。以民间或者是半官方的社团,帮助他们稳定台湾,然后让台湾的经济能够有一个和缓地过渡,能够很快的再进入一个增长期。



你要注意到大陆的现在做脱贫,在2020年前要把所有的贫民都完全提升到非贫穷的状态。台湾在过去这20年,其实赤贫的人口是在增加的,这是可怜的一件事情。你说将来大陆过来以后,交接的过程中,如果有民间组织或者半官方的组织,能够帮助他们做一些统计或者资料的事情——实际上地执行,大陆根本不需要资金,也不需要你的帮助——但是对地区的细节跟现在的生活,你这个原本就住在那里的人是可以帮助的。我觉得洪秀柱——你前两个月访问她的时候,她似乎还在想着选举这种事这种事情——我觉得是:你有什么意义吗?你第一个选不上,第二个在10年、15年之后,统一之后就完全没有意义了。



史东 45:26

我在节目里面也也曾经说过一句话,我觉得现在我观察台湾的现象,就像在一个快要沉的船上,还有很多人要争着做船长。



王孟源 45:39

争船长是因为船长还有一些资源,还可以卖票什么的,还可以让乘客花钱。你如果是已经明显选不上船长,也连那个第五助理船长都选不上...



史东 46:03

我了解,你的意思就是说,如果这些政治人物他是真心是关心台湾老百姓的福祉的话,他应该把他的精力放在一些更有效率的事情上,而不是一个把精力放在一个快沉的船上。



王孟源 46:23

对呀。反正非理性的民众你要去教育他们是不可能的。我自己知道嘛,因为我跟我爸妈,跟我弟弟讨论这些事情是不可能的。一般人到了35岁之后,既有的迷信就已经没办法抹去了。我自己都还是每天必须要提醒自己,要用理性看待的事情。



史东 46:44

因为这里面有一个心态是这样,如果你过了35岁,到了35岁、40岁,如果你发觉你一向的思维是错误的话,你已经在无形之中承认你过去的30年、35年、45年是白活的,对不对?这个心态是很让人家很害怕的。



王孟源 47:05

对,一般人就没办法做这个转变。许历农是相当的了不起,因为他九十几岁的老人居然说我不反共,这个不是一般人能做到的。因为他不是找一个立场,然后盯着那个立场,一辈子死守不放,百分之九十多的台湾人是这样的。但是许历农是说,我关心的是对人民的生活水准好不好。



中共自从邓小平上台之后,你如果看看联合国的统计,2012年联合国的一个资料吧,它说,在过去30年,全世界的脱贫人口中共占 80\%,也有另外一种算法说 100\%。这两种算法怎么会不一样呢?是因为除了从贫穷线以下到贫穷线以上的转变之外,还有从贫穷线以上掉到贫穷线下(就是台湾这样子)。你如果只算全部的那些从贫穷线以下升到贫穷线以上的话,中国占全世界的80\%。但是从贫穷线以上掉到贫穷线以下,像是伊拉克那些人,或者是叙利亚、利比亚 ,或者是台湾的一些赤贫民众,低收入民众,这又占了五分之一。所以中国占了全世界脱贫的80\%,净脱贫人口的100\%。就是如果中国没有遵循邓小平的政策的话,那么全世界的脱贫基本上是停滞不前。过去三十年...那许历农能够看到这一点,然后说我现在不反共了,所以他做到了。



史东 49:22

这个我觉得就是没有私心了。这种情操,我觉得是相当相当值得人家尊敬的。



王孟源 49:33

我觉得是态度的问题,他客观嘛,他有一个更高的绝对的客观的标准。而不是对主观的立场死守不放。



史东 49:44

我想趁着节目剩下的时间跟你谈一谈经济方面的改变或者可能的改变。第一个,你觉得新台币还会存在吗?



王孟源 50:01

绝对不会。新台币也成为历史,然后一定会换成人民币。这是很好的,因为人民币永远都在升值。



史东 50:07

另外你觉得台湾未来的经济状况在统一之后会有一个什么样的情况?



王孟源 50:15

两年多之前,我说台湾真正有技术实力的企业只有一大一小两个。一个是台积电,另外一个是联发科,然后就说联发科在五年之内就会出毛病。因为它会被美国的高通跟中国的展讯上下夹击。最新的财报,比我想象的还快,两年就出问题。台积电的累积就是很大,台积电的营收占台湾GDP的 5\%。台湾是一个工业化地区,它一家公司占 5\%,而且它只在一个很长的产业链里面占一层。它的上下游,像是“封测”一大堆的上下游的产业。你全部算下去的话,对台湾的经济的贡献至少是 15\%,有可能高到 20\%。这样的一个企业,它对台湾要求什么?就是你给我足够的电力,它目前用的大概是个 1GW 的电。他说未来十年,我可能需要另外 1GW。



这是什么时候讲的?两年多前讲,然后不但两年多前这样讲,而且他说我们可能因为台湾不能够再建新电厂,所以可能要自己建电厂。那时候我一听,我就说这个绝对是要到大陆去,为什么?它是搞半导体的,它就不会有可能跑去建火力电厂,这只是一个借口。这个口是用来解释自己为什么要到大陆去。台湾的顶尖企业台积电都跑到大陆去了。在 2020 年之后,大陆的制程开始威胁台积电之后,你想他会怎么样?那到 2025 年台湾统一之后,台积电应该还是会有相当的价值。但是我想基本上他所有的新工厂都会建在大陆,即使现在从两年前开始,新工厂建在大陆,不会在台湾去建厂,因为没办法保证供电的稳定。



史东 53:04

我们接一个电话,这位朋友,谢谢你久等,贵姓。



张先生 53:08

史东你好,我姓张。我很简短。第一个就是这个王先生他讲的非常精彩,谢谢。那第二个就是说我就是趁这个机会——因为我再差两天就来美国三十年——所以当你们谈到许老爹的这个想法,我就感慨非常深刻。因为我也慢慢有这种想法。我要趁这个机会提一下,我就讲这一点,就是说再过五天嘛,就 911 这个周年。我就想趁这个机会,就是在这个在您的节目上面,谢谢我们的美帝国主义。他这个自导自演的 911 能够让我们中国在 2001 年到现在,有的是十几年...在他重回亚洲之前有个十几年能够平平安安的...就是建设经济,这个强国强兵。所以我就在这里,提出来这一点,谢谢。



史东 54:27

好,谢谢你张先生。王先生有什么反应没有?



王孟源 54:33

我个人觉得 911 对美国的整体战略来说是一个大灾难。尤其是刚好小布什当总统,他的反应非常的糟糕。基本上是拿它当做一个空白支票来满足个人的私欲,他的私欲是什么?他是一个公子哥,他不需要借执政来捞钱的,他是借执政来满足他的一个家族荣誉。原本他的父亲在第一次海湾战争的时候,没有把萨达姆搞掉...这其实是很明智的事情,因为老布什本身是一个老油条,考虑事情非常的周到。但是事后——当然是在民主制度下——非理性简单的谩骂声量是最高的。结果小布什就觉得这是他们家族名誉上的一个污点,他必须要把萨达姆给解决掉。



因为第二次伊拉克战争的原因,的确是中国得到了另外 10-15 年的喘息时间。要不然美国帝国主义绝对不会容许任何一个国家的GDP涨到他的70\%以上。我说这个GDP 是以PPP来算的,就是Purchasing Power Parity 来算的。在冷战时间,有两个国家达到 70\%。然后在 1985 年两个国家一起都被一杆子打死了。一个是苏联,一个是日本。苏联是用意识形态来把它解体,宣传来解体;日本因为是一个被美国控制的国家,所以直接就强迫他采用错误的财政政策,然后就把他的财政给搞垮了。中国在什么时候达到 PPP 70\%?那是 2005 年吧。那你如果 2001 年没有搞出这个事情,2004年、2005年美国就会想个办法把你搞掉了。现在(中国)已经达到了一百(多percent)



史东 57:13

我想节目时间剩下还有大概五分钟左右,我再把话题再带到这个统一之后的台湾。现在在台湾的社会上有一些事情,我不知道北京方面有没有兴趣,或者他觉得这是不是对他有利的一种行为。这个事情在我们今天所有的谈的话题里面,相形之下是件小事情,就是这个军公教人员的待遇的事情,退休待遇的事情。我在想,可能我很天真,如果统一之后,北京能够继续地实行中华民国政府时候的这军公教人员退休的待遇的话,我觉得对台湾的岛内的安定会是一个相当大的一个factor,一个因素。你觉得是吗?



王孟源 58:16

如果军公教这个退休待遇不是太离谱,18\% 当然是有点高。所以事实上蔡英文现在把这个军公教的待遇砍下去,其实是替中国做轿子,因为一旦是比较合理负担得起的话,他直接接受,然后稳定民心。如果是我,觉得马上就说好,没问题,因为你要烦的事情还有很多。



史东 58:47

对,因为这一块你把它搞定之后,这是社会安定的一个很大的一块,不是吗?



王孟源 58:56

对。所以蔡英文基本上搞这些事情并不是为了她关心财政,她如果关心财政就不会搞那个八百亿的前瞻计划,她搞这个完全就是政治斗争嘛,因为这批人是支持这个国民党的。但是这个她的这个新政策,这个财政政策有一半是对统一有利,一半对统一不利。对统一有利的是把军公教的那个退休负担弄下来的,不利的是你搞八百亿的前瞻计划。你知道这八百亿至少有一半都会不见了。



这种钱不见对政权交替是很重要的。一般人不知道——我们现在没有什么时间,我就两句话解决——你如果去看看为什么辛亥革命会成功?是因为在辛亥革命之前,那个保路同志会铁路私有化失败。铁路私有化的过程中,满清政府私有化弄来的钱不见了(相当于现在的几百亿前瞻计划),所以辛亥革命才会成功。这个我觉得是很有趣的。



史东 01:00:21

最后还有一点,我们结束之前,就是教育权的问题。我觉得这也是,怎么讲,也是很容易理解的事情嘛。就是我觉得统一之后,台湾的教育权一定回归到要符合中华人民共和国的宪法,对不对?这也是毫无疑问的事情。



王孟源 01:00:45

所以会有失落的一代,有二十年左右的学生,他们学的就是台独跟民进党搞的这个莫名其妙的这一套历史观,将来对他们来说会是很痛苦的。这些年轻人基本上——我想大概 1990 年出生,一直到2010年出生的这一代——他们一辈子都必须要背负。他们的价值观、历史观和人生观、世界观全都不对。



史东 01:01:26

我在节目中也提过,我说这件事情是——我这一代的这些在台湾从事教育事业的——是在作孽。



王孟源 01:01:45

我觉得美国的这个教育系统本身问题很大,台湾照搬之后又把它政治化。你说如果这些人是在作孽,还太客气了。一个人一辈子的三观都毁掉了,而且不是一个人,而是一个 Generation,一个世代,比文革还要惨。文革只搞了十年,这个一搞就是二十年。



史东 01:02:23

好的,从你这么说来,我更意识到这个事情的严重性.王先生,我今天要再次地说一声,非常谢谢您的时间,谢谢您的智慧。我希望我以后能够有很多的机会,能够就你感兴趣的事情,我们可以尽量的谈一谈,谢谢。



王孟源 01:02:42

你也看出来我兴趣来了,话匣子就打开



史东 01:02:46

很好很好,那是我们的福气。谢谢你,王先生,谢谢。



王孟源 01:02:50

好吧,我们有空再聊。好好,拜拜。 







\twocolumn[\begin{@twocolumnfalse}
\section{石油人民币}
\subsection{20170927}
\end{@twocolumnfalse}]史东 00:00:29

各位朋友你好,我是史东。再一次地说一声,谢谢您按时地收看我们八方论谈这个节目。



在这个月的月初,九月初,一个叫做《日经亚洲评论》的刊物,它的英文名字叫做 Nikkei Asia Review,它发了一个消息。这个消息的标题说:中国准备推出人民币计价原油的期货,可以兑换成黄金。然后在里面有些文字,我很快地为您介绍一下。它是这么说的。它说:根据《日经亚洲评论》,世界第一的石油进口国中国,准备发行以人民币计价,可以转换成黄金的原油期货的合约。这个合约希望能够成为一个重要的石油基准,并且允许出口商绕过美元计价的基准,用人民币来做交易。下一段它说:这个原油的期货,将是中国第一次向外国的投资者、交易所以及石油公司开放的一个商品合约。这种绕过美元的交易,可以允许比如说像俄罗斯、伊朗这样的石油出口国,通过人民币的交易来避开美国的制裁。最后一段他说:为了令这款人民币计价的期货合约更有吸引力,中国还允许、还计划人民币在上海和香港的证券交易所兑换成黄金。



今天的节目中,我们再度地为你请到了王孟源王先生。他是金融方面的专家,来和我们谈谈这个非常大、非常重要的这个题目。那么首先我们要把王先生带进我们的画面之中,然后说一声,王先生非常谢谢,非常欢迎。



王孟源 00:02:33

非常荣幸,这么快又上你的节目。



史东 00:02:36

是的。我想这个我一开始念的这三段,特别是第一段,对于像我这种对于金融不是很懂,对石油期货交易不是很熟的人,可能需要你进一步地为我们解释它的内容。我们就从标题来讲,说中国拟推出人民币计价的原油的期货,而且(人民币)可以兑换成黄金。你的反应是什么?



王孟源 00:03:04

好,我们先讲一讲这个所谓的原油期货。期货英文叫做 Future。 它是一种提早几个月先进行买卖的契约。但是这个契约是标准化的,所以可以像股票一样,类似于股票市场,也可以做交易——就是说它已经被完全金融化。比如说你进去卖期货的时候,你不需要自己真的有,你只是纯粹做空;你进去买的时候,也不是真的要有提货的能力——就是说你自己有管道,到时候可以接收一万吨的石油。(买期货)的意思,只是说你是做多,所以它让本身不是(石油)持有者或使用者的跟踪者,也可以参与石油交易。



期货有两个功能:它第一个是让不是使用者跟提供者的人也可以参与买卖交易;第二个是它能够提前几个月事先预估届时未来的需求跟供应之间的平衡。比如说如果我要去炒作,我要做多石油,我可以进去买期货。这个期货如果是三个月前的,在这三个月里面我们就直接买卖,像是股票一样。三个月到期的时候,(我)要是真的是要石油,那就选择兑现。那个石油真的送到同一个事先约定好的港口,交货的时候,(我)必须要有能力提货。所以在(期货到期)最后一天,通常就会转手。这些真正纯粹去买卖的人,会把他的位置(仓位)清零,让给真正的供应商跟真正的使用者。他们有他们的基础设施,能够真正地提供石油或者是接收石油。(石油期货)它是一种让金融界跟资源的使用者能够交汇流通的管道。目前我们说石油的价钱多少多少,事实上指的是石油期货。



大部分的石油...除了是长期合约之外,还有一大部分短期的期货——就是大概三个月之内的。那真正我今天生产了,但是我不知道谁要买的,这叫现货石油。其实这个石油期货已经算是短期(的合约)。大部分的石油提供...因为石油生产出来以后是占地方的,(提供商)要知道你这个油轮要开到哪里。所以绝大多数的石油大家都事先就知道它值多少钱,你要准备多少。这是所谓石油期货,不要把它看作是一个很坏的东西,它其实是所谓的石油现货市场中的一个内建机制。



史东 07:31

据我了解是您刚刚也提到了,石油也有现货的买卖,但是很多是期货的买卖,对不对?期货我现在重复一下你刚刚讲讲的这个事情,看我听的是不是正确。期货大部分是以三个月为期,对不对?你可以卖长,可以卖短,你可以预估这个石油会涨价或者是会跌价。然后你不用立即付钱,但是三个月到期之后你要付钱。所以说我回想到那个时候石油暴涨的时候,很多航空公司他们赚了很多钱。因为他们就是事先在石油还没有涨以前...



王孟源 08:17

三个月是不够的,要比那个更长,但基本上道理就是这样子。Future 是一种标准化的,所以它就直接在市场上——像股票市场——那样子的交易。但是如果你做对应的特定时间的合约,叫做 Forward 。它的那个好处就是很有弹性——这个时间多少,或者它的规则都可以改变。所以航空公司他们做 Hedge 对冲——因为他们是很大的使用者——他们要对冲掉他们的风险的话,其实是去投资银行 Investment bank,或者到对冲基金去跟他们签 Forward。Forward 可以是一年、两年、三年。然后如果他们的头脑不是太灵光的话,说不定还会被骗签 Option。这个 Option 又是有很多 Option,也有 Strike 在公开市场上面交易,(航空公司)他们这个 Option 是很复杂的特别的Option。



史东 08:15

这样就好了。王先生,很复杂的话,我们就不要继续再追下去。基本上你把原则给我们交代了一下。然后我想知道的是,就您刚刚谈到的这些这些石油期货的交易原则,你把它套在今天我们谈到的这个主题上,他的 Significance 在哪里?



王孟源 10:31

我想中国人、美国人还有行内的人都知道,这就是一大盘棋里面的一步。这盘棋叫做什么?叫做金融霸权。美国的霸权是建立在三个角上。第一个是军事,美军;第二个是他的宣传,基本上世界上什么事情是对的,什么事情是政治道德高尚,价值宣传我说了算。民主自由、普世价值。很搞笑,民主自由制、西方选举制、真的一人一票是美国一直到 1968 年才有的。那 1968 年以前这个世界不存在了?



那第三个就是美元。就像美国的整体的霸权是从英国人手里经过两次大战拿下来的。可是英美之间没有战争了,没有打起来。这个导火线是因为一战之前英国的执政者没有眼光,没有看到真正的威胁是美国,而被近在咫尺的德国给迷惑了,所以被拖进一次大战。一次大战打起来英国人至少有一半的原因。打下来以后他们精疲力竭。这时候英镑这个国际货币,它的地位开始动摇。但是虽然当时美国的 GDP 已经比英国大,而且英国本身的财政在一次大战之后就非常的糟糕。但是因为市场惯性,所有的市场机制还有大家使用的习惯传统都是用英镑。英镑跟美元在一战跟二战之间十八年,基本上拖拖拉拉打成(平手)。



在后期...因为大萧条在 1929 年开始,它的起源是美国,美国受创最深,反而是让英镑(重新强势)。就是过去看看英镑跟美元的历史的使用比例,其实在接近二战的时候,英镑又变成比较有优势了。但是二战又打起来以后,美国本土没有受打击,欧洲被打成稀巴烂,到了 1944 年的时候,他们已经在诺曼底登陆。这个时候德国的败亡指日可待了。这时候美国召集了所有主要的盟国在 New Hampshire 这个叫做 Bretton Woods 一个度假胜地,开了一个财政会议。这个财政会议要决定的——趁着仗还没有打完,你英国还是全靠美国的支撑,先把丑话说在前面——我们要美元取代英镑。



史东 14:22

这个就是有名的,叫做 Bretton Woods 森林会议对不对?



王孟源 14:30

对。那时候英国的代表是 20 世纪最伟大的经济学家——Keynes。他知道英国的国力——我想一般翻译成凯恩斯——他知道英国的国力不足以再支撑英镑。所以他其实建议使用一种新的国际货币,那时候还来不及命名。美国人回答很简单,不接受这个方案。这个是鸿门宴,你来了就...那这个 Bretton Woods 会议开完,国际的货币结构成为每一个货币都是跟着美元走。就是我跟美元有固定的汇率,大家都跟美元,美元跟黄金。那个时候是 35 美元一盎司,那现在是 1300 美元。这个 35 美元一盎司一直弄到 1971 年。这个时候是美国总统 Nixon…



经过越战之后,美国持续二十几年的经济发展撑不下去的时候,尤其是政府的财政非常糟糕。所以 Nixon 就说,好吧,那我们开印钞机。开了印钞机以后,这个 Bretton Woods 协议的核心,就是美元跟黄金的兑换比例,就撑不下去。因为你印了美钞,美钞越来越多,黄金却没有越来越多。隔年,其他的国家就注意到不对劲,就是说你的美金越来越多,我的英镑却没有变得越来越多。我的德国马克没有,对不对?那黄金也没有越来越多,那我们还是用你的汇率就不对劲。(史东:就是英镑和就其他国家吃亏了嘛,因为美国一直在印钞票嘛)。而且他拿这个钞票去换黄金,去换你其他国家的钞票的比例是固定的,所以其他的国家就给他抗议。这时候美国的财政部长说了一句很有名的话:The dollar is our currency,your problem.



史东 17:35

中文翻译就是美元是我的货币,但是你们的问题,应该说是你们的难题。从这句话,这是很有名的话,就是看出来那种当时美国的那种霸道,对不对?



王孟源 17:55

美国一向就是这样霸道,对。但是我刚刚说过,美国人也很在乎宣传,放上台他们就讲究要遮遮掩掩,要找借口。所以从一开始共和党是要霸道就霸道,不会在那边假装圣母。但是那些欧洲人说你这个是真金白银地跟我们抢,不接受,所以他就又签了一个协议叫做 Smithsonian Agreement,在 DC 签了。这个协议说:好,我们回归到以前的 Bretton Woods——就是把美元贬值 10\%,黄金的价格变成 38 美元一盎司...那大家说好吧,让你们占了 10\% 的便宜。结果十一月还是十二月,1971 年的十一还是十二月他们签了,结果他们回去以后发现根本印钞机从来就没有关过。到了 1973 年,大家都不干了。你们这样子的,这样来的话,我不能够让你用固定的汇率跟我换我的货币,所以我怎么办?我也只好跟着印,这样至少你要用固定的比例跟我换的时候,没有白占便宜。



史东 19:45

换句话说就是美国你开始是稀释你的金融,我也一定要相对地要稀释我自己的金融。不然的话,这个这个比例就就不对了



王孟源 19:57

这是不对称的。因为美元稀释自己的金融,其他的国家都必须拿美元当外汇,美国却没有拿其他国家货币当外汇的。所以美国人印钞票,别人必须收,别人印钞票,美国可以完全不管。你印了钞票,你就准备自己家通货膨胀。美国印的钞票是全世界通货膨胀,美国现在的GDP不到全世界的 20\%,他只负担不到 20\%,这就是前所未有的国际货币的 Exorbitant privilege,就是昂贵的特权。因为英镑在 19 世纪就是国际储备货币,但是对英国来说没有什么太大的利益。因为这时候还是金本位,英国人如果如果印钱印多了,他那个英镑兑黄金就要贬值。事实上他是没办法拿这个印钞机兑换上有价值的东西。它的好处在于拿着英镑去跟人家借钱,因为它是国际货币,所以大家都愿意用英镑。所以用英镑借钱的利率特别低。那打仗是打钱的,所以你要是能够用低利率去借钱的话,你打仗打起来就资源丰富(手头松一点)。十九世纪能够持续保持的一个全球性帝国的关键是这个财政关键。另外一个关键是因为他当时有印度——印度等于是一个次大陆,整个次大陆的资源都让我用,整个次大陆的人口都帮你消费。



史东 22:17

我们的这个音讯断的很很厉害。我希望能够能够把这个谈话继续地往前推。你刚刚讲的基本上就是 Bretton Woods 这个会议,讲到了 Smithsonian 这个协定,讲到了美国原来答应要以 35 美元兑一个盎司的黄金,后来变成了 38 美元一盎司。换句话说就是贬值了。然后你刚刚谈到英国,在当时英镑在称霸的时候,因为英国是遵循的金本位的这个货币,所以英国并没有当时像美国后来的这样子的为所欲为。我可以用这四个字来形容,为所欲为。



王孟源 23:17

Bretton Woods 的制度基本上是持续以前的英镑霸权。就是说美国你要当霸主,你至少接受我们英镑当霸主的游戏规则。现在变成说:钞票是我印的,但是责任你们负。所以到了 1973 年,这时候大家就开始讨论把美元丢到垃圾筒,改用另外一个合成货币。事实上也有也有 Candidate 合成货币。



Bretton Woods 除了确定这个货币兑换的机制之外,它还建立了两个国际性的机构,一个是世界银行,一个是国际货币基金组织。世界银行的责任是国家之间的扶贫,到贫穷的国家去借钱给他们(贷款给他们)。但是这个国际货币基金会更为重要,货币基金会它的责任其实是就像中央银行——一个国家里面的中央银行。当你的金融出了问题,中央银行必须出面作为所谓的 Lender of the last resort,最后一个解决问题的人。当时美国人还假装客气说好,我们美国人执掌世界银行,国际货币基金会由欧洲人任命行长。但是实际上两个机构都建在华盛顿地区DC,距离大概一个半 Block,距离美国财政部只有一个 Block,实际上等于是美国财政部的附属机构。



史东 25:29

我想(我们的谈话)继续地再往往前走几步,能不能趁这个机会给我们解释一下。因为我们也知道美国的这个金融财政,美国没有中央银行,美国有个 Federal Reserve 联邦储蓄银行。我听说联邦储蓄银行它的存在是很有意思的,它是一个私人的机构,而不是一个国家级的机构。对于这方面你能不能给我们解释解释它的存在跟它可能造成的影响是什么。



王孟源 26:13

因为我这些都是即兴表演,所以很多细节可能(不准确)。1907 年的时候...美国在 19 世纪末期有好几个金融危机,到了 20 世纪在 1907 年的时候又有了一个很大的危机,这时候出面的是JP Morgan——Mr. JP Morgan 先生,(当时)那个人还在世——他的公司后来拆分成两份 Morgan Stanley 和 JP Morgan,JP Morgan 是美国最大的银行——他出面。基本上他做了中央银行后来该做的事情。在这件事之后,美国国家联邦政府就觉得你总不能老是靠着私人来搞,所以他们才建了法案,然后有了 Federal Reserve。它事实上就是现代中央银行的滥觞,大家都是拿它当做模型来模仿。它的责任一般也负责控制货币的供应量,就是决定...这个所谓的印钞机就是由他们把关的。那在出了大毛病的时候、有金融危机的时候,他们作为那个 Lender of the last resort。



史东 27:53

我现在插一句话,您刚刚提到了这个 Morgan Stanley 他们代表的这个利益,就变成了美国现在的这个 Federal Reserve,就是联邦储蓄银行(国会授权)。我的问题就是说:因为这个非常明显的会有一个利益上的 Conflict,利益上的冲突。从美国国家的角度来看这个联邦储蓄,跟一个由私人的集团控制的联邦储蓄。私人集团控制的,他会用他自己的利益为第一优先,而不是美国国家的利益为第一优先来做决策。



王孟源 28:42

就是因为担心这样子,所以不愿意让 JP Morgan 当一个中央银行的角色,所以变成了一个公家的机构,另外设立的一个公家机构。我觉得这是 20 世纪初很重要的一个金融发明。所以 Bretton Woods当然也是这个经验。比如说我们国际上也应该有这样一个,可以创建一个国际的货币。那到了 1956 年...其实 1956 年的时候,那时候英国跟法国还在假想着殖民地帝国。虽然那个时候印度已经独立了,但是大部分非洲的国家还没有独立。那时候他们又出了另外一件国际事件,就是他们出兵去保护苏伊士运河,这是第二次苏伊士危机。当时埃及开始威胁着要把运河收为国有——以前是由一个英国公司控制。英国跟法国人当然是不能够接受,我们的殖民地,为什么可以抢我们的东西?这个有一部电影说 You are trying to take what I've rightfully stolen, 所以英法就出兵埃及。结果当时的美国总统艾森豪威尔就下令 IMF 要收回给英国的贷款。因为英国从二战之后借了一笔债,1956 年还是...IMF说要把你的贷款收回来了,要收银根了。这时候英国的那个市场马上就崩溃了,所以英国不得已地从这个埃及撤军。这个是美国霸权正式开始。



史东 31:10

这个很有意思。我们因为时间的关系,我再跳过一次,然后因为我现在很快,我想我想谈一谈,就是说您刚刚谈到的,今天我们在谈的主题就是关于人民币。因为原油计价的这些事情是美国和中国在金融棋盘这个大棋盘上的一小步,对不对?



王孟源 31:33

这个棋局就是到了 1973 年的时候,美国又把前一年签的Smithsonian给撕掉了。那这时候大家都不高兴了嘛,对不对?舆论大哗,结果美国人就说不要紧,还是得用美元,为什么?因为他一九七三年的时候跟沙特阿拉伯当时的国王签了个协议,说从此以后沙特的石油只能够用美元来,所以美元从黄金定位,这是Bretton Woods system,用了29年。从一九七三年到现在四十三年,我们说石油美元,就是从那里来



史东 32:25

我可以这样说吗?从美元就是从一个原来从以黄金为基准的,现在变成以石油为基准的货币。



王孟源 32:35

对,那就是一九七三年做的转换。所以接下去。原本一九五六年的时候,英国跟法国去打埃及,是跟以色列联盟的。虽然美国很支持、一直都很支持以色列,但是他是到一九七三年之后才一天到晚在中东用兵,然后支持以色列跟沙特,这个政策的转折也是一九七三年。



史东 33:01

就是说沙特同意美国以石油作为这个交易的期货的基本货币。然后美国对沙特阿拉伯提供保护,是这个意思吗?



王孟源 33:22

没想到他们刚开始搞,到六年又出了个大娄子,那个这样的这真的是很大很大的,伊朗的政变的事情。一九七九年,那因为以前伊朗的国王是美国的卒子,那个cia 历史上第一次在国外搞政变就是在伊朗。就是一九五三年把当时的民选出来的首相给搞掉了,让国王()。所以在那之后,那个国王当然是乖的不得了。不只是亲美,这个是我们说的Poodle 贵宾狗,特别的乖。所以我们过去这四十三年,美元虽然没他把Bretton Woods协议里面所有的责任都丢到了垃圾桶,他的权利确实还在的。他这个权利还在,而其他国家愿意接受,就是靠的石油美元。所以现在我现在要跟你讲的这个很重要。这个我的博客上有讨论过,但是博客上讨论的东西太多了,一些重要的东西我没有说三遍。



史东 34:51

你博客上所有的链接都会在我们的网站上都会都会放上去。



王孟源 34:56

我对我这个人很懒的,所以即使是很重要的事情,我从来也不会说三遍。但是我现在要讲的这个东西很重要很重要。金融最基本的就是货币有什么功能?我今天讲这个,我一辈子没有进金融系修过课,我一辈子没有读完一本金融课本,所有的东西都是我自己悟出来的。所以你不要跟我讲说,讲的东西我在金融学的课本里面找不到,或者是找到的东西跟你的有抵触。这是我的自己的看法,自己的见解,不是从书上抄下来的。一个货币从下到上有四个功能。最早货币出来的功能是作为贸易的使用。因为你以货易货很不方便,(货币)是最基本的贸易的手段。第二个功能是保值。你想想原始社会大家开始交易之后,然后有些人有钱了买了地,买了地以后,他那时候还没有复杂的观念,他不能够把它租出去。所以那个地多到他一个人耕不了了以后怎么办?存起来。英文说Under the mattress藏在床底下。它的第三个功能,这不能算是功能是特点。那就是后来就出现了专门的机构银行,中央银行,国际货币基金组织



史东 36:57

银行存在的价值是什么?



王孟源 37:02

他们是作为货币跟金钱流通的阀门跟渠道。



史东 37:17

我为什么问这个问题,有一个原因,银行这个概念,这个观念是西方来的,对不对?



王孟源 37:29

中国其实叫做钱庄嘛



史东 37:32

钱庄那个是银行吗?



王孟源 37:40

银行最基本的就是从空间跟时间里面帮你做货币的兑换、交流。比如说这个农民,我冬天的时候要种地的这个种子要到哪里买,要肥料,要到哪里买,他需要贷款,这个贷款跟银行贷了款,然后秋天收成了卖掉了,我有钱,我还你。这个是时间上的货币。空间呢?那我这个富商,想想看明朝的时候或者是宋朝的时候,我这个山西的富商要到哪里哪里江南去买一批货,然后要运到北京去。这个我到江南的时候需要有钱,到了北京的时候货卖掉了突然又有钱,这就是钱庄跟他的银票之间的交流。这是空间转移。所以银行最基本的也是最早的功能,就是对货币的时间和空间的转移。他们是这种货币交易的闸门跟渠道,而且经济跟金融发展的越先进越复杂,他们也越来越重要。所以我才把这个放在第三位。就是说你先第一步货币是用来取代以货易货,就是促进实体交易。第二个是保值的手段。第三个这时候就有机制了,帮助这个货币流通。但是这个货币流通之后,又帮助本身扩大影响,所以是就是在时间上跟地域上能够充分的发挥他的影响力。然后到最后,这也是到一九七三年才出现的国际性定价权。我们现在中国或者是印度要去跟沙特买石油,不关美国什么事嘛,但是他必须用美元。两个第三方在做大笔的交易的时候,却必须用你的货币。这是到一九七三年以后,就是就是所谓的定价权,最高级的货币的形式。你看世界上有几个主要的国际货币,美元当然是第一,欧元是第二位。人民币我想现在已经算到前五。因为很多这些指标是稍微滞后,稍微晚一点。美元是从一九七三年开始开创了,就是他这个神功练到了第四层,我们用武侠小说的话。这个欧元现在到什么?欧元现在已经变成外汇。就是你跟欧洲做生意,你自然就用欧元定价。所以第一步是可以做到的。保值的话,国与国之间的所谓的保值就是外汇,对不对?你任何一条的经济现在绝对都会有百分之三十左右的外汇是放在欧元里面。欧元已经在第二层的的货币这个层次里面做到了和美元旗鼓相当。你再看第三层,这个就有点有意思了,因为这个机构方面有两种,一种是国际性的官方机构,另外一种是非官方性的,也就是银行。,我们先讲官方的,官方的一个是IMF,一个是世界银行。世界银行由美国人当过行长,IMF由欧洲人当行长,看来好像是平等的,但是他两个都在两个都在华盛顿DC,而且只有美国有否决权。



史东 42:38

在你文章中也提到这两个机构。基本上美国财政部的下属机构所以表面上好像是是独立的。



王孟源 42:53

对,欧洲人对IMF的防控是很有用的。从Keynes 开始在一九四四年就设想建立一个国际性的合成货币。我们现在最接近这个理想的东西是就是IMF的SDR,也是一篮子货币。他这个是IMF定义的一个货币,这些会员国捐钱,然后他把这个放到一篮子里面,然后用一个单位把它重新定义单位。每个单位里面包括了,比如说每一元SDR里面包含了每个不同国家的币



史东 43:41

主要的好像五个国家,对不对?是五种货币,不同国家的比例在上面每一个比重不一样。



王孟源 43:55

人民币在二零一二年就就应该进入了SDR,但美国把它否决了。对,就是美国用他的否决权,也就是美国的投票,它的占比百分之十六,这个IMF的规则是投票必须要到百分之八十五,才可以通过。投票的结果是百分之八十二,就是美国加上一个小国。不愿意让人民币,不愿意让人民币进SDR。结果拖了几年,拖到什么时候?中国建立了亚投行,这个是两三年前我在我的博客那个部落格上就已经预言了。他们建立的亚投行,基本上就是你们不让我玩,我另起炉灶。这个亚投行就是针锋相对的,就基本上架空了世界银行嘛。对,而且反正这种扶贫做投资贷款,谁有钱,谁的声量大,这个跟你的历史悠久没有关系。他一出来以后,世界银行被架空了,就是IMF看了吓一跳,世界银行不是我们管的也罢了。要是搞一个IMF的替代性组织,我们就完蛋了,国际上的金融货币的话语权就靠着IMF这一点点东西,你要是把IMF给搞掉了,那人真的就变成过气的帝国。所以他们跟美国国会这样的吵来吵去,国会说什么都不过。这时候IMF拿出杀手锏说要把那个百分之八十五的投票门槛要降低,是这个意思。这个东西取消的话,美国没有否决权的,这个真的是,这都是所谓的nuclear option。在那个二零一六年去年。



史东 46:18

所以这一路上,王先生,你讲的这个故事,也多多少少我想大家也都也都知道,这一路上怎么过来,只是你把它非常精简的把它叙述出来。我想这个任何人都知道人民币和美元之间的这种竞争或者说斗争。你刚刚也说过,就是说今天我们谈的这个题目是他整个大棋盘之中的一小步。我现在是有个问题,你觉得人民币希望能够取代美元的霸权,还是人民币希望和美元平起平坐,这是两个不同的思维。你觉得你觉得从北京的角度看这个事情,他们他们的目的是什么?



王孟源 47:08

我觉得他们有两个目的,十五年是一个中期来看。我说他们有两个阶段的,十五年之后大概在二零三零年左右希望能够和美元并驾齐驱。至少在同一个层级上的,就是同一个class, 稍微小一点稍微弱一点没有关系,但是至少可以望其项背。到了2050年,我想他们是希望能够完全超越,二零五零年,就是再过二十年。那我刚刚讲的那四个货币的特性,我们现在再重新复习一下,不过这次不是用欧元,因为欧元只是两个半。四个层他只练到第三层练了一半。现在人民币从两三年前开始,大概在二零一四年开始,就是习近平上台之后,二零一二年大概酝酿了一年,大概做内部讨论。然后到了二零一三年年底,二零一四年年初开始动手。这时候第一步很简单,因为你这个交易的话,中国的对外贸易额跟美国差不多嘛,基本上还高了一点,大约就是占世界的百分之十一百分之十二。那你说我这百分之十一十二如果是要卖给美国或者欧洲的话,你用美元用欧元。但是你跟其他的国家的话,我就可以一定要想办法用人民币。但是世界贸易是很分散的一件事情,你这样子说来说去,除掉跟美国贸易跟欧洲贸易之后,剩下的就是个位数字的百分比。事实上美元在这个贸易上的使用广泛,也不是因为美国自己跟别人的贸易额有多大,而是练到第四层的那个定价权的时候,大家自然就用它了。所以第一层不是特别重要,也是特别容易那做了以后也没有什么太大的效果。而且这个是出台政策之后,一年两年你就可以看到很明显的成果,其实也没有什么成果。



第二步是要当外汇。那当外汇我们刚刚给他进入SDR,这个很严肃而且事实上内容就是很复杂的。各位有兴趣能到我的部落格。这个困难的地方在哪里?在美元跟英镑跟欧元都是自由交易,就是没有管制,开放的。对。我现在要会汇1 trillion,这不是开玩笑,在货币市场真的是(可以)。我我三个礼拜之前跟你跟你聊的时候,我就说现在的热钱很多,热钱多到什么地步?世界GDP的十倍二十倍。对,这是世界上的热钱有这么多。反正我进来以后再杠杆再借更多,这样翻个几遍,一个billion就变成几个trillion。我记得当时这是二十五年前的,我那个时候我刚从哈佛毕业的时候,我是物理系的,那个全班就只有我一个人(全班的意思是说高能物理的七个人)就只有我一个人留下来念物理。我又做了一个postdoc,后来我去找一个同学,他说他在citi bank,做了不到一年,他是在那个currency就是货币部门。我去找他吃午饭,半年的员工,然后我们吃饭的时候,我说你怎么今天看起来一副心不在焉很忙的样子。他说这边有一个\$14 Billion dollar order。我那时候还在学术界搞了一辈子,从来没有见过这么大的数字。我吓了一跳,你一个刚进银行半年的员工,一个人就搞\$14 Billion,跟他说你这个是不是人家很赏识你,准备要让你。他说哪有,随便几个Trillion的交易额都有。



史东 52:50

代表他们那个行业到处都是黄金是吧?



王孟源 52:55

这些都是无中生有的热钱的,实际上不是真正生产出来的产品。因为gdp 是世界资产累积的每年增加的上限吧。因为GDP基本上它的意义是说我们这一年生产的所得。生产所得除了我们真的用掉花掉的最后的储蓄下来这个才会变成资本。所以GDP是这个所谓的wealth 的增长率的上限。那你的这个钱增加到GDP的十倍、二十倍,显然就已经跟世界资本财累积的到了同一个数量的。而世界资本财大部分是像公路、机场这些固定的设施嘛,那怎么会突然变成热钱流来流去?那这个基本上就是杠杆,翻来翻去。



史东 53:58

你说的这个钱是不是基本上也是美国印钞机印出来的,印钞机印出来以后,然后再把它撒撒到世界上去。



王孟源 54:12

对,那个银行可以有信用嘛,对不对?这个一开始原来只是对货币的时空的转换,但是后来以后有了这个信用的观念,美国人特别喜欢credit这个观念,就是拿我这个担保,我的资产只需要保证说我未来比如说要借一年期,一年期可能会我的资产可能会损失多少,比如说我的资产最多,你认为可能损失百分之五。那我的这个margin就只需要5\%,你这个钱就翻了二十倍。这是现代金融,就基本上是这样搞出这些热钱



史东 55:04

所以事实上这个这个事情你这样一说出来,我觉得我为什么自己早一点没意识到。这世界上这个来来去去来来往往这么多的热钱,炒作金融这些东西,完全都是主要是美国几个国家印钞机印出来的。这些多出来的钱嘛。



王孟源 55:25

印钞机印出来的。然后你借我借,先跟这个银行这样翻了十倍吧到另外银行,这时候再跟另外一家,那就翻



史东 55:39

那我们现在再把话题再带到这个这个石油这件事情,就是是中国人民币



王孟源 55:49

刚刚要讲的是金融的四个存在的价值。所以这个人民币在外汇这方面初有小成。就是目前世界上拿人民币当储存的话,比例还很小,大概占不到美元的十分之一。但是万事开头难,从零到百分之二,比从百分之二到百分之二十要难太多了。对对对,所以人民币现在就是已经到百分之二。那我们刚刚会得到这个热钱,这个话题是因为人民币不是自由交易的。他们他有资本管制,就是说你这个钱汇出汇入怎么样,有那种很严厉的管制。原因是因为美国利用美元霸权可以随时入侵。防止美元的入侵了,基本上就是这样的,对他这个是一个提防。但是你这样一来的时候,就不能够被其他的国家要把你的钱当做外汇就不方便。我这个中央银行说我现在要百分之十的钱在人民币,我们的全部外汇存底是一千亿美元,百分之十是一百亿。这一百亿是我们出了事情的时候,要随时拿出来用。就是基本上是这个国家的储备金。那如果到时候还要等你人民银行批准的话,这个不像话嘛,这也不可能在政策上就不可能,远水救不了近火。所以你必须要让他的中央银行能够绕过人民币一般的资本的管制,能够让他们安心说。也许一般的银行不能够这样做,但是我是中央银行,我跟人民银行有特别的关系。



这个你也许在过去几年有看到,也是从三年前开始,都是同一个时期开始做的,就是所谓的swap交换。我们刚刚提过几种那个银行金融衍生品,提过forward 这些都是很简单的,当然还有option。那还有另外一种叫做swap交换。这个swap的意思就是说,我们现在我出一百亿美元的人民币,你出一百亿美元,我跟你交换。人民银行说你现在一百亿美元worth 的人民币,随便你怎么搞都可以。但是我们原则上是当做你的外汇。做贸易的时候提供存储的服务。这个你如果注意去看的时候,过去这三年人民银行基本上跟世界上主要的经济都签了这个swap。自己看到的就是十几个,实际上说不定有二十多个。世界上有一百九十个国家的,一百八十个,你这样子把几个主要的经济体都照顾到了。但是还有另外的一百多个国家。这时候你就必须要让他们能够,那些小国家你必须要做第一流的服务,但是可以慢一点,间接一点。但是你还是要有让他们必要的时候,能够把他们的人民币换成美金或者美金换成人民币。那这时候他就建立了一个这个机制,忘记名字。他基本上是在取代欧美的一个swift,就是国际货币交易。人民币事实上也是练到第二层。然后他也是练到三层的这个机制。



我们刚刚讲了一大堆,也就是说这个亚投行它是自己是完全控制的,就您可以有自己的世界银行。IMF的SDR它现在排第三位。因为比这个美元欧元要小很多。然后这个国际货币贸易的那个那个通报系统,他自己也建立类似的替代。这些官方的组织可以建。就是虽然说人民币的体量比起欧元来说还差一些,尤其是在外汇方面,欧元现在平均大概占世界外汇的百分之二十多吧,那人民币绝对还是到百分之五,还差很多。但是事实上每个国家都有人民币外汇,重要的国家都有人民币外汇。然后这随时间慢慢就会涨。然后到第三层建机制,在官方方面中国做的非常积极,而且中国的这个体制的优点就在于他能够动员官方的协调整个官方的动作。因为他在官方这方面做的特别已经超越欧元。但这个swift 就是欧洲跟美国共管,但是中国已经有自己的swift,然后中国有他自己的世界银行,中国已经参与了IMF,所以事实上比欧洲还要多一点点。但问题是,国际的金融机构除了官方以外,还有非官方,非官方的是哪些国家,哪些东西?就是Goldman Sachs,Morgan Stanley,这些银行全部都还是美国的,尤其是两千零九年之后,因为上一次金融危机,那些被骗的金融机构都是欧洲的机构。你那个美国次贷危机这个他的mortgage 是佛罗里达的,包装它的是美国的银行,但是到最后接手的接盘侠都是欧洲的,而且是德国的中小企业。



史东 01:03:55

所以说买这些债券的,或者买这些金融的这些东西的人,都是欧洲的人,对不对?



王孟源 01:04:10

我在华尔街的时候就常常看到嘛,大家都知道这些欧洲人是凯子傻子,我自己是不知道,但是我的同僚跟他们聊天的时候都可以看得出来,你那个美国的这个信用评比,信用评估机构有两个是美国的,一个是英国的。他们都是已经被买通了嘛,



史东 01:04:33

所以我这又是一个我们应该可以谈,但是今天没有时间谈的,就是信用评比这个事情根本上也是一个破产对的



王孟源 01:04:45

那是骗局,只要弄个AAA的标志。美国人都知道这是买来的,德国人他们就相信,但是因为他们的这个民族性。你这个上面贴的AAA的标志,这个一定就是AAA的,那我们就可以买。到时候爆炸的时候都是他们被炸成稀巴烂了



史东 01:05:06

有一点,王先生,我想在今天节目最后的大概我想有十分钟,能不能讲一讲。因为我今天想谈的几个方向,一个是这个事情对中国的意义是什么?这个非常明显。第二个是这个事情对美国的意义。我想我们也可以知道这个事情对美国的意义什么。事实上在当初新闻的出来的那一天,就是九月初新闻出来那天,在台湾的联合报有一个标题说,假如全世界都用人民币结算石油的话,美国百年的霸权或将毁于一旦。我相信这个标题可能过于耸动一点,可能要经过一段时间。



王孟源 01:06:24

中国的这个石油期货很聪明,它最后可以用黄金结算。它的妙处是说,既然我们人民币对还是一个很新兴的货币,这个接受的程度很低,基本上就是过去三年才开始国际化,那你大概不太放心,一下子就从美元换成人民币。所以我就让你有那个选择,你可以选择换成黄金嘛。世界上如果有一个金融货币,比美元还受信任,那就是黄金。这个很聪明,但是不要太乐观。为什么我这么说?我们刚刚提到这个非官方机构,两年前中国也推出一个跟欧元之间的的期货。就是这个期货不是石油,而是从人民币跟欧元兑换之间的一个期货。我刚刚说我那个同学,刚进银行就搞\$14 billion,就是搞的那个货币外汇兑换的。这也是你这个currency 方面最常用的一个最流通的一个手段。那两年前人民币推出兑欧元,那这个是世界第二号跟第三号的货币之间的兑换,我们应该觉得应该很好,事实上不是呀,就是他这个真正的这种市场必须要有庄家。就是我们刚刚讲到有庄家,它基本上一开始这里都有时间跟空间的转移。这就是说,你不需要直接找另外一个交易对手,你先跟银行做交易,银行再另外换一个时间换一个空间去找交易的另一方,就是银行的基本职责。银行的一个重点就是坐庄的方式 bridge the space time gap 就是做买卖双方的桥梁,结果那个期货没搞起来,没搞起来,为什么?因为只有一家中国银行来做庄家,那这个它的这个供应的流通量就是liquidity不够。所以到最后到一直到现在,如果要从人民币换成欧元,你还是先去买人民币换美元的future,然后再从美元换到欧元,这是世界第二跟第三的货币之间的交换都如此。那你可以想象。大家都是仍然是继续先换(美元)。当初Bretton Woods设计成这个样子,就是也知道你直接从一个货币换到另外一个货币之间,不会有那么多庄家。如果大家都先换成美元的话,这时候流通性反应是比较有效率。就好像那个所有的飞机先飞到这个大站,然后再从这个大站飞到(其他地方)。



史东 01:09:57

我想最后一个问题,这个问题可能比较这个不着边际。因为我们都知道这个不以美元计价石油这件事情,世界上也不是头一次发生的。几个国家曾经说过要这么做。第一个是伊拉克,我记得第二个是这个利比亚,结果这两个国家都被都被美国给灭了。你觉得这个会发生战争吗?因为的确我觉得是一个很可能会发生战争的一个原因。但是因为中国看起来他是一步一步的有备而来,并不是很唐突的。就说我要这么做,而且中国老实讲块头相当的大,并不是像伊拉克,像利比亚可以容易的被美国的军队灭掉。但是你觉得这个可能性怎么样?



王孟源 01:11:06

我有一个嗜好,我会去看看地质历史,还有演化的历史。一般人的观念是说好,我们现在这个山变成了河谷,然后或者是一个新的物种突然演化出来,这个是一个随机的事件。而且是从时间上是很平均的分布。实际上不是这样,实际上是这个地貌很可能持续几千年几万年都没有什么变化,然后突然有一个极为少见的大风暴把那个山头给扫平,或者一个大地震。这个生物的演化也是这样,也许这个物种会很稳定的存在几万年甚至几十万年。但是因为某个环境或一个很小的关系(e.g.)一个病毒,突然之间他就演化出好几个新的物种。你如果看看这种美元金融霸权的交替,必然也是。就是中方一定会布局,把所有的门都开通。就是你们如果要从美元走到金人民币,中间的路我们就过河搭桥,有墙就开个门,这条路是通的。但是你开通以后大家还是不会走,为什么?因为network effect 就是网络效应。用这个方便嘛,而且有一点习惯性。我刚刚讲过,过去这十年,欧洲的银行都被打垮了。所以现在国际性的银行只剩下美国的。所以这个期货现在推出来,我觉得短期内不会有什么,因为基本上交易量不会特别大。但是中国长期来看必须要一步一步的布局,把所有要走的路都()。等到未来十年、十五年,美国出了什么大动乱,大家都急着要从美元跑出来的时候,这可能未来十五年发生,也可能要三十年四十年才发生。但是它发生的时候中国要准备好,这些要从美元逃出来,逃亡的人不会逃到欧元,而是逃到人民币。



史东 01:13:32

我非常同意你的看法,所谓万事俱备,只欠东风,是不是这个意思?这个东风我们不确定,东风不知道什么时候来,但是来的时候你万事一定要先具备才行。你说的就是这意思,对不对?王先生,非常谢谢。今天我们谈的题目很大,谢谢谢谢你王先生



史东 01:14:31

谢谢你给我的一个机会



\twocolumn[\begin{@twocolumnfalse}
\section{赌城枪击、比特币}
\subsection{20171004}
\end{@twocolumnfalse}]史东 00:29 

各位朋友你好,我是史东。再次的说一声,谢谢您按时收看八方论坛这个节目。今天的节目中,我们再度的为您请到了王孟源王先生,他是连续三次在我们的节目之中出现,我再次要说一声谢谢,谢谢王先生。同时我再一次的说声谢谢谢谢你按时的收看我们这个节目。今天的节目中我们这个要谈的主题是这个在十月一号刚刚发生的Las Vegas的枪击案的事情,另外还有一个事情,我们想跟你谈一谈的就是这个bitcoin 就是比特币的事情。事实上在我们的制作的过程之中,我是一直很早就跟王先生约好了,想谈谈这是个bitcoin 的事情。但是十一月十月一号这天,这个发生的这件枪击案使我们不得不花花点时间来关注一下这个非常重要的事情。那么在今天节目中,再度的为您请到的王孟源王先生。现在把王先生请到我们的现场之中,谢谢王先生,谢谢。



王孟源 01:38 

非常荣幸能够再上你的节目,我先事先道歉一下,我发现我的internet service provider 有问题,所以要更换。但是因为连我的电话也要换,所以必须要等十天之后,今天可能这个视频仍然不是很好。那我先向你读者致歉。



史东 02:02 

我想这个事情,我们两个都要,谢谢一位这个在youtube 上给我们提供了一些线索,解决的问题线索的一位先生,姓胡胡先生,他指出这个上一次我们音讯非常不理想的情况,可能是出于这个你的麦克风那边的的调整事情,我们的确是朝那个他所建议的方向去找了一下,的确我们好像找到了一点问题。那么然后我们改正了问题,希望今天我们的谈话至少能够比上一次我们谈话的这个情况要好很多。虽然这我也知道internet 不是一个很稳定的一种平台。当然我们也都知道这件事情。我想我们先谈谈这个在十月一号发生的这个Las Vegas的枪击案的事情。我首先想了解一下王先生,你对于这个事情的感触是什么?



王孟源 03:06 

目前没有足够的资讯。现在大家不知道枪手的动机。很明显的,他不是穆斯林极端主义者。所以最可能的是他可能心理跟生理上都有重病,比如说他已经有绝症了,然后心理上又决定要找人陪他去死。那很不幸在美国要找几十个人陪你死,是蛮容易的——他这次是找了一个音乐会。其实美国像是高速公路上堵车的地时候,你如果拿了一把步枪,在公路旁边开枪也是很简单,就可以杀死,这是很不幸的。不过我认为我自己对美国国内的政治没有什么太大的见解。我之所以会同意谈一下这个问题,是因为我们上一次做节目,最后你问的那个问题,我没有好好的回答。



如果你不记得的话,你问的那个问题是:中国的发展这么快,又在金融上想要用人民币来取代美元,那么美国会不会想办法用他的另外霸权的支柱——也就是军事,来遏制中国的发展。当时因为我急着要收尾,把整个讨论做两分钟的总结,所以就没有注意到你说的细节,而直接是自说自话了。所以我想要谈一下,回答一下你这个问题。



这个问题的解答很简单,就是一般人认为中美对峙的话,美国对中国仍然会有军事上的优势。不止今天如此,即使在十年之后,可能还是会有一些优势。比如说中国很显然永远不会弄到两三千个核子弹头。美国是不但核弹的数目比中国大概在高十倍左右,而且他们才刚刚更新了所谓的B61弹头,就是极端小型化,而且能够穿透碉堡的弹头,能够穿入地底下很深很深。所以你不管地底下工事是多么的深,但一样都能够穿透。像这种东西中国还没有消息说(有)。所以一般人看到这个局势,他就会说:那为什么中国在经济贸易的总体国力,工业上不断地挑战(美国),那为什么美国不干脆就发动战争?



其实美国发动战争的最好时机已经过去了。我个人认为 Rumsfeld 在 2001 年当国防部长的时候,就计划在 2004 年发动对华战争。如果你去回看一下当时美国 Joint Chiefs of Staff 联邦参谋会议的几个高级将领的回忆录,你会注意到在 2001 年年初他刚刚上任的时候,基本上是天天都跟这些高级将领讲,我们要准备跟中国打。而且同时他加强投资海空,基本上很明显的是要在台海或者南海打一场海空战争,把中国的气焰打下去。但是因为 911 发生了。那发生之后,George W. Bush 是准备要对伊拉克动手,所以这个事情不了了之。



当时中国的经济实力——就是 GDP,不到美国的 50\%。美国在 80 年代后期,把日本跟苏联一个是用金融手段,一个用宣传手段做掉的时候,他们的 GDP 都是美国的百分之 70\%。所以当时Rumsfeld 在未雨绸缪,准备三年后把中国的气焰打下去,时机也是相当地遵循历史规律。但是因为 911 ——现在美国在阿富汗已经搞了 16 年,看来还要再继续呆 10 年——他们根本就无从动手。所以到现在,中国实际上用 PPP 来算的话,他的 GDP 已经是大概美国的 110\% 多,那这时候再动手已经是有点晚。



而且我们在考虑(美国)这个动手之前有三个主要的因素。第一个就是我们从这次枪击案上,可以看美国的内部矛盾非常的深。除了我以前说的贫富不均之外,还有种族、文化、区域,性别、年龄等等。这一次这个人如果是因为重病,那显然是他没有得到合适的心理治疗。美国的医疗保险没有全民保险,是第一世界中独一无二的——就是全世界都知道的笑话。而它的医疗花费在 2014 年是年度 GDP 的 18\%,同样是比所有的西欧国家都高出几倍。西欧国家的医疗费用大概是 6\% 到 9\%,而且每个国家都有全民保险。所以这一次这个枪击案当然是有特别的(地方)。这个人到底是怎么发疯的,这是属于八卦新闻。



史东 09:40 

你觉得我们有一天能够真的知道真正的原因吗?



王孟源 09:46 

至少这个人他对社会的不满一点迹象都没有。很明显的,他没有受到足够的医疗关怀,而且一个和谐的发展向上的社会,大家不会这样动手。就是其实这种事情会不会发生,最简单的就是看族群有没有被撕裂。美国在世界各地搞族群撕裂,台湾这个族群撕裂他也是乐观其成。但是他自己内部的族群撕裂,其实是靠着过去几十年,尤其是二战以后,尤其 50 年代跟 60 年代,它的经济发展独霸全球——那是因为欧洲跟日本全部都被打烂了,它的工业基础全部被打断,必须重新建。等到到了 60 年代后期、70 年代初期,欧洲跟日本开始恢复了,他的好日子就过不下去。



我们我们上次谈过,那时候它因为越战,所以财政有问题——其实另外一个原因是因为它的工业独霸开始被推翻了。那时候像是丰田或者大众汽车开始在全球挑战通用汽车。所以我们看到这个赌城枪击,我觉得...我喜欢从大局观来看这些小事件。大局观(来看)这个的背景,就是反映出美国内部的矛盾非常的多重。而且由于它的经济发展到目前已经开始——不是说衰退——而是相对的中国有严重的落后。当有这种现象出现的时候,一个国家的社会就开始不安定。



你如果只读英美的媒体,比如说纽约时报、华盛顿邮报,你会觉得他们老是说中国不但没有改革,而且在习近平任内还退步了。我上个月才刚刚看到卫报——英国的卫报这样写,那个纽约时报也是的,几乎每个月都会这样写,这是非常非常奇怪的事。五年前中国没有全民保险,那现在他的全民医保做到非常好,这样不算改革,你美国能做得到吗?就是基本上英美的改革,所谓的改革是向着英美的制度模仿着改,其他的真正改革,他视而不见。这就好像一个小孩子跟他的爸爸说,你根本就没有努力。你拿回来这个诺贝尔奖一点用都没有,对不对?你根本就没有陪我去玩沙子,那所以你是一个非常不努力,没有改革的家长。



史东 12:57 

对不起,我重复一下,你是不是说英国和美国他们在观察中国或者其他的这个发展中国家的改革过程,一定要先符合他们的价值观念,或者他们的做事的步骤,才会觉得你的改革是正确的。



王孟源 13:19 

不管你的什么改革,如果你是朝着他们的方向走,就是好的;但是朝着好的方向走,只要不是朝着他们的方向走,那就算是坏的,会批评的。这是我们在看英美这些媒体——事实上他们有国际的话语。全世界的主要媒体都是英文的,英文的主流还是英国跟美国来控制,这是毫无疑问的事情。Russia Today 只不过是在欧洲有少数人开始看,美国跟英国就受不了,因为这是开始挑战他们的霸权。



所以我扯了这么远,基本上是回答你原先的问题说会不会有惊喜。第一点是其实美国是内外交逼,把他以前过去一个世纪,社会经济文化上的很多的问题掩盖掉的——就是它的经济发展现在没有了以后,这些问题会呈现出来。



史东 14:25 

换句话说,你说的是不是因为在那个时候,美国的经济发展比较好,所以说他有这个能力去掩盖这些社会上的不公或者不满。



王孟源 14:38 

是。你想想看,在 50 年代、60 年代平均增长率是 6\%。世界第一的 GDP,6\% 的年增率来增长——每个人都有工作,大家抢着要雇人。然后你随便...买汽车以后,就想着再买第二部汽车;洋房三个卧房的大家(觉得还不够)... 至少我是在新英格兰地区住。你如果看的是 1940 年代盖的房子,很小很小的,那个房间很小,房间数量也不多。但是到了 50 年代、60 年代你可以看得出来,越往后盖的房子越大。等到到 80 年代就变成 mansion,就是有五千平方英尺已经起来了。现在真正的豪宅,2 万平方英尺都不算什么。2 万平方英尺的话就是基本上 2 千平方米。大家可以——如果不是在美国住的人——可以想想看那有多大。



所以第一个是美国本身内忧外患,第二个是中国基本上是一个战略守势。在这个战略守势上面有两个很大的优势。第一个是他不需要打赢你嘛,打成平手了是算他赢。因为双方打成平手的话,就不会打起来,不会打起来就算他赢,所以这是他的第一个优势。第二个是他的战场在西太平洋。西太平洋跟冷战时期的中东欧那个战场完全不一样。中东欧是全世界文明人口、工业文化、工业人口最密(的地区)。基本上是十几二十公里就有机场、雷达站。那个地方打起来,你根本就不需要什么航程。所以你想想看,当初 F22——就是美国现在最新锐的战机——发展出来的时候,根本就没有考虑航程。他那个出了问题以后,第一个牺牲的就是航程和载油量。



那它在太平洋就完全不一样,西太平洋你的那个航程有 3 千公里,作战半径 1200 公里不算什么。因为你看看美国能够在西太平洋起飞战机的是什么?除了琉球之外,下一个就是关岛。关岛距离战区 2 千多公里。即使是长程战机都不够,那不用说 F22 这种短腿。美国在这种地方作战,基本上就只能够送航母上去。航母的话不能够载最新的战机,而且航母本身也是很容易被打,也是一个大目标。中共在过去十年要么就是打航母,打美国在西太平洋的一线海空基地。你现在可以看到美军在过去几年不断往关岛撤,那个琉球的基地跟日本人吵得一塌糊涂。最后虽然他赢了,政治上赢了,但是在军事上反而是减少人员,为什么?因为中共的小飞弹——就是短程飞弹——都可以把它轻松地打到,所以就必须往后撤。后来发现,中共又有 DF-21,DF-26 这种东西,可以打到关岛。后来它在三年前居然往澳洲的达尔文撤,达尔文有多远?五千多公里。到了这个时候,你打一个战术性战役,必须用上战略武器。我说战略武器就是像 B2 这种战略轰炸机,它有足够的航程。那像这样子的话就是完全的一个守方的优势。所以有了这两个考虑,你就可以知道:即使在政治外交上美国要打,在军事上——尤其中国有针对性的发展了反航母跟反地的武器,然后在飞机上又有歼 20 这种隐身战机出来——基本上在今日美国要打中国,打不赢。你除非是打全面核战,可是谁会傻到为了争夺一个霸主的位置打全面核战的。所以回答你上次的问题:不会。中国已经有了准备,所以不会有战争。



史东 19:28 

我再顺着上次的问题以及你的思路再往前走下去,我提出来的问题主要的一个背景就是:在过去有一些国家,这些国家都是中型或者比较小型的国家,他们都是卖油的国家。他们希望能够舍弃美元,用欧元或者用一个所谓的中东自己创出来的一个钱币来作为计算的单位。一个是伊拉克;一个是卡扎菲的那个国家叫利比亚。利比亚基本上这个国家被用不同的名义,都被美国灭了。所以说,我在这个前提上,我说中国要做这个事情非常明显的,他们是有备而来。所谓他们——就是北京,是有备而来,而且北京的块头实在太大,而且北京有能力,有武力去保护自己。那么让美国觉得这这个仗不是一个软柿子,就是他们不能够予取予求。可是在这个前提下,我相信美国也不会坐视说,中国有一天人民币会取代美元。那我现在问题来了:第一点,你觉得中国有这个兴趣或者有这个野心去取代美国吗,还是说他基本上只是希望能够和美国平起平坐而已?这是第一点;第二点,美国如果——你刚刚解释了——不会用军事或者是——我不要说敢不敢的问题——就是不会用军事来解决这个问题的话,美国有没有什么其他的方法——你可以想到的可以压制中国。因为还是回到我刚刚的那句话,美国绝对不会坐视这种事情。虽然他能够控制的这个可能性越来越小,会越来越少,但是我相信他们,他们不会坐视这种事情发生的。你的答案是什么?



王孟源 21:28 

永远会有他们的 Neo Conservative 想打起来,像那个 Bannon 他本身就是很喜欢、很想要跟他们打架。不过不是军事,而是贸易战。所以我们还是回到我以前讲过的美国的霸权三个支柱:军事、金融跟宣传。我们来看看,它基本上军事是用来打小国的、中小国家的金融。金融上次成功的是把日本搞掉了。不过那是因为日本是他们的半殖民地,所以他们自己的财政金融政策必须听美国的,美国告诉他们做傻事要做下去。宣传上次成功的机会是对苏联,这个是非常非常成功的一个案例,把一个两亿多人全世界第一大国家肢解了。但是你如果看看,美国目前除了他的内忧外患之外,他这三个方面都在衰退。中国已经有足够的力量能够打一个首次作战,或者能够吓阻美国为了小事情而挑起战争——你要么你就打全面核战,要么就不要打。那到这个局面下,中国已经算是稳定了。



在金融上——中国在前几年,尤其是两年前,这个很大的错——基本上这种东西中国自己没有做错的话,就不会有机可乘。两年前他们为了疏导国内的热钱,让股票市场狂飙,这个是非常错误的政策。有人迷信美国的自由市场主义,所以以为一切自由化就可以。其实自由化只是为财阀好,而且不是为国内的财阀好——目前世界上最大的财阀是美国的财阀(就是大鳄)——而是让美国的财阀得利。美国国家层面的这些 Neocon 有可能也会趁机地来把你搞乱。资本管制绝对不能够完全裁撤。我想两年前那个股票让它狂涨,然后最后泡沫爆炸,这是一个很大的失策。不过好在这是他们管经济的人搞的,这个管人民币的是人民银行的周小川,没有跟着一起发癫,所以人民币没有没有动到根本。那么这是金融方面。



我想经过两年前的一件事,他们现在学乖了,而且也发现内部有一些蛀虫。你看到什么证监会的前主席被抓起来了,然后某某秘书是第一个被抓了。这些上层有迷思,中层有腐败,这些问题都在慢慢地(在解决)...所以我想中国在往前走,他的金融方向会比较审慎。我个人觉得他们完全不必管美国这个所谓的自由金融,这种东西是利于财阀的东西。中国是搞实业。搞实业的话——我们上次提到一个以实业为重的经济,它的一切为稳定,尤其是汇率。比如说是以实业为重的经济,是以稳定为第一要求——就是你的汇率不能够大幅地涨跌,你的股票资产价钱也不应该大幅波动。凡是有大幅的涨跌,就是你的管理人员没有做好。所以你看到前一阵子(一年前)因为大陆严重地做去产能,结果大宗货品像是煤矿或者是铁矿的价钱都掉了一半还多。最近又恢复了,这个其实是很好的事情,但是是没办法必须要做去产能。最好的目标是让它能够缓缓地降下来然后不在那边震荡。



史东 26:22 

不懂你说你的意思,就是软着陆是吗?



王孟源 26:25 

对,软着陆。一切的价钱都应该是和缓地变动。凡是你看到有这种重要的工业用的原料或者是成品,它的那个价钱、或者是金融的指标有大幅震荡的时候,就代表管理的人员没有做到完美。没有做到完美是没有关系的,人总是人嘛,人不是神。但是最重要的还是货币。货币比这些大宗货币的价钱还要重要。这个货币绝对不能够让它大幅震荡。因为中国的贸易在过去两年进入了进口替代。就是原本在 2001 年加入 WTO 之后,这 15 年基本上都是在外销为主。但是到了习近平上台之后,经济已经发展到一个程度,可以开始把进口的东西(做国产替代)...



就比如说他们以前做一个锅——举个例子好了:一个锅很简单的东西——我们现在买的锅都是大陆做的——好像是在浙江那里有一个生产基地。他们都是一个镇,然后全国 90\% 的厂商都聚集在那里。那因为它有那个网络效应:就是供应商生产产业链的关系,产业链就集中起来,有自然的集中的趋势。美国用的锅都是有那个防粘的涂料——有特氟龙,现在换成其他的涂料。当然刚开始的时候,那个不只是这个特氟龙,其实锅的金属有的时候都需要进口。后来第一个就是先把金属用国内的替代。替代以后,最高端的是什么?就是那个涂料。那个涂料先是特氟龙,然后后来换成更新一代的东西,结果在过去两三年也开始变成国产。



这时候,像这种进口替代的东西,从巨观贸易的资料来看会发生什么?就是中国的外贸反而停滞不前。你说过去这两三年外贸的数量反而没有提高很多,为什么?因为一开始的时候中国是做组装嘛,组装的话你所有那些高价值零件全都是从境外进口。90\% 的价值是从国外进口,然后你 100\% 组装成之后再输出,你一共有 190\% 的出口贸易。现在一切进口的原件都是自己做的,那就没有所谓的进口,只有出口的那个 100\%,所以你看进出口的总额没有在增加。尤其是进口的总额,在过去这几年基本上就是平稳震荡。虽然它每年还是 6\%、7\% 的 GDP 成长,但是它的进口额就不行。尤其是大宗原料价钱下去之后,就更是低迷不振。我看到这些资料就觉得一般的评论员说:中国的进出口没有没有动能,我就觉得他们没有把事情看得深入精准。事实上是在产业升级之后,这些表面的症状看起来好像是负面的,其实是一个正面的。



我们这样讲的是比较远了一点,不过我们现在已经开始讲到有关这个热钱还有货币这些东西,我们可以开始谈比特币。



史东 30:34 

今天我在节目之前看到一则消息——这是凤凰国际的报道——说这个 Bitcoin 经过三个礼拜之后,现在又触及到 4400 美元一个 Bitcoin。那么它继续讲一下,他说:这个情况是抹平了 (Jamie Dimon)JP Morgan 对它的批评,以及中国关闭了 Bitcoin 交易平台而造成的这个跌价,现在又回来了。所以说比特币的这个价格并没有花费太多的时间,就从 (Jamie Dimon)JP morgan 的批评、以及这个中国关闭交易平台的这个有负面的情况之中回来了。我想这个 Bitcoin 这个事情——我刚刚节目一开始也讲到——我有一直很想跟你谈谈这个事情,因为这是一个新的东西。这是不是代表了一种新的 Concept,一种新的思维。在整个的所谓的钱币的发展过程之中,这是不是一个划时代的东西,我想透过您的知识来了解一下,也为我们的观众了解一下,你对 Bitcoin 这个事情你的看法是什么?



王孟源 32:15 

首先我们来看他是不是货币,它名义上是一个货币。我们上一次讨论过一个货币有四个主要的特点。第一个是它必须要用能够用来以货易货,就是用来购买,来做贸易,比特币很明显的目前基本上只是用来买毒品,或者是网络犯罪的勒索勒索。那长期来看事实上也不可能超越,因为他你最大的问题就是什么?除了没有官方的承认之外,最大的问题是你如果要做贸易的支柱的话,就是我刚刚提过的,他的那个汇率必须是稳定的,要不然买方卖方都不方便。我现在跟你讲好这个价钱,结果等到后天你这个东西汇过来的时候,价钱已经变动了百分之二十。哪有人会用这种货币?不是买方吃亏就是卖方吃亏,对不对?货币的要做贸易的支柱,贸易的根本流通的一个手段。一个很基本的就是它必须要有稳定性。那比特币的特性就是它非常的不稳定,那既然它非常的不稳定,就永远不会有人想要真正用到正当的用,必须要有非常的理由才会用它,一定是非法的,就是犯罪才会用。



那么第二是他能不能保值,跟投机不一样。保值的意思是说,保本。所以要保值的话,同样也是必须要稳定。你如果不稳定的话,那个你的本钱有可能会亏掉。虽然有可能会赚一倍。但是很多人他们关心的是保值,而不是投机赚钱。那么一个货币的价值在于,所有你如果学过经济学的话,经济学会跟你讲所谓的risk-free interest rate就是没有风险的利率,就是你只拿货币最基本的东西。货币本身在经济学里面来说就是提供零风险。你如果是这个国家的货币,那你要在这个国家里面拿一个零风险的资产。用从定义上来说,就必须要拿他的货币。那这个比特币能够称得上是零风险吗?我想就是刚好相反,对不对?高风险,第三个是机构,要有银行跟国际机构或者是中央银行来管理。比特币本身发展就是特别没有管理。



史东 35:21 

这个管理是不是一种担保,还是超过了一种担保。



王孟源 35:26 

就是到时候出了问题,我保证我还会在这里,保证中央银行不会把你禁掉,因为我就是中央银行发的,毕竟我们想真想看,中央银行会把人民银行会把人民币禁掉,这是不可能的。



史东 35:44 

这提到这儿有一个题外话,印度政府前一阵子不是把印度的这个钱币废掉了。



王孟源 35:50 

印度人是印度人。



史东 35:52 

这是另外一个话题。再回来谈担保,



王孟源 36:01 

you have a good point。倒不是说全世界绝无仅有。但是我只是说像样的国家像样的国家不会搞那种,像在中国,就是你这边比特币交易的平台,今天还在收你的钱,明天就是非法的准备要关门了。像这样子就不是一个像样的货币应该有的。最后的是要有定价有,前面三个都做不到,所以才不会有人把原油用比特币来定价。



史东 36:39 

能不能为我们解释一下定价权这个概念



王孟源 36:43 

定价权的概念就是不是你自己的货币,两个交易方买方跟卖方都是有本身的货币。但是他还是要用第三方的货币,这个就是叫做以定价权。就是美元这么如此的强势。虽然比如说中国去跟沙特买,它既不是用中国的货币,也不是用沙特的货币,而是用美元来交易,这叫做定价权。当然比特币也没有定价的个性在里面。所以比特币不是货币,那他的这个特点我们刚刚提到好好多次了,有两个很大的特点。第一个是他没有管理,他从设计出来的时候,就是用所谓的block chain,这种是分布式的,没有任何一个中央机构来管理。我说个题外话,这个比特币我等会儿会说一些比较负面的。但是他对人类还是有一个很大的正面贡献,就是这个blockchain 的技术,将来一定会有很大的应用。这是一个某一个特定的一个词,它能够它的特点是说一旦分布式的储存起来以后,好,除非所有分布的每一个人都同意。否则这个资料不可能再被更改。所以您只要分布出去有几千个几万个节点,除非每个节点的人都花时间来跟你合作,来去改资料,否则这个资料就是定了。这是一个非常非常在金融上一定有很重要的,就是不一定用在比特币。但是在其他的。因为金融最大的一个问题就是比如说上个月中国跟印度差点要开打。在我的部落格上,有人问说中国可能会怎么把握?就是说如果真的要狠起来,要把那个印度的损失从几百亿美元打到几千亿美元。很简单,你就要他的轰炸机上印度的那些资料中心data center,这个人员伤亡很小,但是那个资料中心都毁掉以后,印度的经济就完蛋了。就可以很简单的把他的经济损失。你因为你在前线怎么打,不管怎么打土地什么的,然后基建什么的,打的顶多就是一百亿美元。但你轻松的炸几个data center的,就可以让他的经济损失到几千亿美元,甚至几千甚至几万亿。所以我那个时候说,如果你真的要往狠里打,也不需要去打到新德里了,最好是派轰炸机炸几个。因为尤其印度的经济没有什么实体经济,全部都是虚拟经济,不是搞金融,就是搞那个什么也带好,就是帮美国人做会计这些东西,或者是IT的这种东西。你如果把资料毁了,那资料是没有办法重建的,就算两年三年来搞更没办法。



史东 40:13 

所以说再回到头来你刚刚谈到这个这个block chain



王孟源 40:19 

block chain 就能够抵抗这个,它是一个全新的技术。如果我们所有一切重要的资料都是用block chain 来来记载的话,它可以同时分布到好几百个,会有好几千个节点。除非你把这些节点全部都炸光,他是永远可以再找回来的,而且是不会被篡改。



史东 40:44 

王先生你讲这个概念,让我想到最早的internet 是不是也是这种概念相似?



王孟源 40:50 

这个就是继承了internet,internet 本身设计就是没有节点。要分布式的对,所以他这个是把资料储存也用internet 这个做实时联系的这个概念。



史东 41:07 

中国有一句话叫做化整为零,是不是这个意思?



王孟源 41:13 

就是这样的,就是而且它的妙处在化整为零之后,不但每一个人有一个备份,而且个人没办法去改这个备份,这个是很重要的。你化成为零之后,每个人都有,还有办法改就不行了。所以他这个是一个很重要的。我我觉得bitcoin他发明人当然是纯粹就是想骗人捞一笔。但是他做出这个技术就代表他对人类有贡献



史东 41:38 

你刚刚说这个技术很新吗?



王孟源 41:44 

bitcoin 的发明人是谁?我们还是不知道,现在还不知道。假名。但是他很显然是一个it 的专家,显然是一个电脑的专家,所以他才能够发明这个block chain ,他如果会搞出bitcoin,而是光明正大的搞出black chain 的话,它有专利,它可能还是会赚很多钱的。而且名利双双这种。



史东 42:10 

所以black chain 这个东西,我要问一句几句外行话。第一个它的专业性很强吗?是不是只有少数的人有这个能力搞,还是基本上是一个, 如果你是个programmer,或者你在internet 做的话,就容易做的。这是第一件事。第二件我的问题就是说black chain 这种东西,是不是就是每一个国家,他们的我不知道这款法币,就是他们自己国家的金融的钱币,也可以用blockchain 这种方式来保护他们他们的资源。



王孟源 42:50 

这种技术跟IT的技术一样,一般就是一样,是说穿了就不值钱。发明的时候很困难,但是说一旦他写下来以后,大家随便拿一个像样的程序员,是程序员看一下都会懂。所以现在很多银行已经开始做了,不过他是不能够用在货币上,他用在其他的资产上。比如说你现在在银行有开支票的账户,这个记录以往是在一个统一的资料中心存在硬碟上,然后有另外的备份是在磁带上面,这个是很麻烦的。因为你顶多就是只有少数几个,这一份,要是还是有可能出了事情,比如说火灾或者是意外,或者比如说程序病毒,有可能全部都。但是像这种你如果用block chain 的话,会很简单的把它分布到好几百个、几十个或几百个。那这样子用用internet 来做这种资料,就是基本上就没有。除非 internet 都完蛋。否则你这个资料永远都不会给你。



史东 44:06 

你说分布出去,它每一个点分布出去的点都是一个服务器吗?



王孟源 44:12 

对,你通过amazon 或者是像microsoft 的那种云,它本身就已经有terrabyte terrabyte 现在都不小心到哪个byte了。对,你就是说你帮我存一份,然后那个帮我这样就可以了。然后它的保密性非常好,也不能够随便更改。



史东 44:34 

所以说前一阵子这个那个公司Equifax 他的那个客户的资讯全部被被盗



王孟源 44:50 

那个不是被更改的问题,是保密的问题。不过他的确是做自己的资料中心嘛。Equifax 是做专门做那个信用评比,就是个人的信用。我们每一个人在美国大家都很熟悉,都有所谓的credit score。



史东 45:12 

我想这个事情,你再给我们讲一下这个bitcoin。听你这么介绍的话,bitcoin不是一个好东西。不是货币,风险很高,它的fluctuation 就是它的这个风险蛮大的。但是另外我看到一些资料,就是说譬如说他他说在二零二一四零年、二一四零年,今年是二零一七年,二一四零年的时候,他说要达到两千一百万个,这个bitcoin 那个是一个极限。换句话说,他好像是有一个我不知道他这个两千一百万的这个数字是怎么来的,我就不知道了。他似乎有一个界限,那么很多人认为这是基本上很多人用bitcoin 是对于各政府的这个QE的一种反抗。



王孟源 46:23 

不要把它跟黄金来比,黄金我们每年全球生产量非常的小,世界上挖黄金挖了大概三四千年的这个所有的黄金放在一起,所有人类提出的黄金放在一起还不够填满两个标准游泳池?那但是你仔细想想就会到这里,这没有什么道理。第一个黄金是非常特殊的实际金属,它每一个原子都是在星球核心所建立的。你说bitcoin 虽然本身稀缺,但这种虚拟货币的概念会不会稀缺?没有嘛。因为我们前两天私下聊天的时候,我们也提到目前就有六百多个ICO。bitcoin 本身的限制的数量,怎么别人可以照抄嘛,因为你这个实际bitcoin本身的价值多少,他的intrinsic value 是零,跟黄金不一样。黄金的intrinsic value,黄金是有用途的,至少可以当做首饰,每年开采的每年开采的黄金绝大多数都可以用在首饰上面,它是有价值的。你想想看,你即使是拿其他高价值有用途的金属,都没办法取代黄金了。而且这种真正的保值,就是那个价值储备,它有东西嘛,就是大家不会说有其他的仿效的东西就会跟着去弄。贵金属有好几种,还是有十几二十种,而且他们都是有真正内建的价值。但是事实上大家还是不用。那你说要把bitcoin来跟黄金比,就是比喻不对。因为他第一个没有内建价值,第二个他自己人为的设限,他自要发几个bitcoin,但是别人可以再做另外一个相似的。事实上有无限多个,现在已经有六百多个ico,对不对?



史东 48:48 

这个ICO它的concept,它的这个概念是跟bitcoin是一样的吗?



王孟源 48:55 

完全一样的,bitcoin 是第一个,然后后来有一个Ethereum 第二个。这些他可以就直接让赌人来吸金。但是现在等到你做到第三个第四个的时候,就必须要再搞出一些新的花样。第一个他就给你跟bitcoin 同样的东西,再加上一个option。就是他们这个公司的股票期权,就是基本上把bitcoin这种虚拟货币,所谓的货币。但是它虚拟是真的货币不是真的,跟他们公司的期权结合在一起,就叫做一个ICO。这是一个公司吸取资金的方式。,就是我的这个公司什么都没有,就是我们可以去抄bitcoin 的那个科技跟concept。但是我说过那个随便哪一个像样的程序员是都可以搞的。基本上我们公司是资金零,人力资源,就是一个程序员,一样可以跟你吸金,这就是bitcoin必须面临的竞争。你说bitcoin 他只有两千一百万个,但是跟他类似的ICO现在就有六百多。你说我我回头来说吧,这个bitcoin本身就是一个人为设计的工具,你把它的除了这些进来炒作的人之外,他的本身的价值多少?是零。比我们在三百多年前,将近四百年前,在荷兰有一个Tulip mania 就是当时把荷兰的国花,炒作到一朵花,一颗种子比一栋房子还要贵,这是什么时候的事儿?十七世纪。是因为他们当时在海外贸易,他们当时荷兰掌控海外贸易热钱太多了。对,热钱太多了就开始。你任何一个能够容纳这些热钱,而且那个价钱不不稳定。你注意到有有炒作的空间,真正的经济的货币或者其他的要素都是稳定的。而且我刚刚一再强调,这些价钱越稳定越好,但是也有东西是越不稳定会越好,价钱越不稳定越好的就是投机工具,他这个比起tulip, tulip至少你还有欣赏的价值。我不晓得bitcoin本身有什么价值。我并不是说这个到了四千美元一个他就会跌下去。我们在我在做金融的时候,我们常常有一句老话,the market can stay irrational longer than you can stay solvant 市场疯狂的程度会比你的资本能够撑的时间要长。所以换句话说,你不管你有多少钱,你是比不过市场的风险。对,就是不管你有多少钱,不管你的多么确定说五年十年之后这个价钱会跌到什么地,你千万不要下去做空,因为这种投机市场,它不管这个价钱多么的疯狂,还是可能再增倍再加四倍八倍。那这样子你的资本再怎么多,还是斗不过整个全世界的热钱。



史东 53:15 

好,我们接个电话,这位朋友谢谢您的电话,王先生,你好,请讲。



王先生(53:19 

是这样的,我刚刚赶上你直播在谈到比特币,然后我之前是因为比特币影响到我的生活了,所以说我对他有点忧伤。我看我想说一点我的看法。刚刚王孟源先生说比特币没有价值,然后我觉得这个是不对的。因为首先比特币本身它虽然只是一串代码,但是它确实是存在,确实它是一个确实存在的一个商品,那么它的价值来源是由于他在挖掘比特币的过程当中,他消耗了电力,消耗了人工。也就是说比特币首先是它本身确定是有价值的,而且这个价值虽然它只是一串代码,但是这个价值是有有两个方面所构成的。第一个方面是说它由它的算力,就是我在生产的过程当中,我付出了劳动,所以它是生产力之后获得一个商品,这是第一点。第二点,使用比特币的人认为比特币有价值,它就有价值。就好像我们一开始使用贝壳,使用黄金作为货币一样。,那么我认为我个人认为比特币并不能取代目前的法币。理由是因为从马克思主义经济学角度上,比特币它不能够衡量社会生产力。也就是说比特币的增发,它不能根据经济社会的增量来决定它增发的速度。他有可能解决解决通货膨胀问题,但他不一定能解决通货紧缩的问题。比如说如果是现在的央行法币的话,它是中央银行中央银行来调控它发货币的发行量,来调控整个市场。但是比特币没有这个功能,所以这是我认为比特币不能代替货币的一点,但是比特币本身是有价值的,而且比特币本身我认为它可以把它当做一个黄金期货,这是我的个人看法,我不知道。所以我刚才听到王孟源先生说这个比特币本身没有的价值。然后我想反驳这一点,仅此而已。



王孟源 55:55 

首先这个所谓的内建价值跟过去花费在它上面的那个资源没有关系。经济学里面有一个有关GDP的老笑话,两个经济学的学生在路上走路讨论GDP,然后在路上看到有团狗屎,那个经济学的学生a 跟b 说,你如果吃了那团狗屎,我给你一亿美元。那那个学生B就吃了狗屎。那个他说,你欠我一亿美元,然后那个学生b 跟a 说,你吃那团狗屎,我也给你一亿美元,那也给你个狗屎。那这样子两人不相欠,是不是就增加了两亿美元的gdp ?是不是?你这个用在比特币上,这两个人吃了至少吃了两团狗屎,这是他们的很特别的努力,对不对?那个他们是不是那这样子,他们对gdp 是不是有贡献?如果用这位王先生的逻辑来说,他们至少两个人都吃了狗屎了,那这是不是就是有两亿美元gdp 的贡献,没有的。这个所谓的内建价值不是看过去而是未来。拿一个例子来说,股票也有内建价值,不是光是市场去来决定的。这个股票的内建价值。就是这个公司从现在一直到地球的毁灭,就是这个公司所发的所有的红利,给股东所有的红利换算回来。比如说二十年后发的红利,同样一块钱就不止现在的一块钱,它是只值几毛钱。因为你这个有利率是正常。那么用利率这样累计下去到二十年之后,可能就是二十年后的一块钱。你把所有这些未来的红利加起来,然后用这种换算回来,现在的价值就应该是这个它的股票价值。你把未来所有的利率红利加起来,所有的红利,然后换算成现在的红利。把二零一八年的红利,换算成二零一七年的币值。二零一九年的红利换算成二零一七年的红利,把所有这些的红利全都换算成二零一七年的币值。然后加起来,这就是他这个股票的内建价值。这个股票市场当然是有非理性的因素在里面。但是实际上它有一个真正的理性的内建价值。就是大家都在想办法估计这个内建价值,那有买有卖是因为大家都没办法估计。因为我们讲的是未来嘛,所以未来永远不可能有百分之百的。不同的估计就有不同的估计值。那估计的比较低的人就会想要卖。估计的比较高的人就会想要买。这个基本上是用一种投票来取得中间估计。所以股票市场其实是一个民主的机构,用你的钱来当选票,来选他的那个实际上的内建价值。你看这个比特币的内建价值是什么?你必须要考虑到他的竞争者进入这个的竞争门门栏,基本上就能感觉门槛基本上就是十万美元嘛。十万美元雇一个程序员帮你写一个,对不对?这种门槛我我我觉得是很好笑的。现在只有现在有六百多个,光是美国就有六百多个ICO。



史东 01:00:11 

这个ICO今天我们没有花很多时间谈的,但但是基本上从你的介绍,我们也是说ICO事实上是比这个bitcoin 比特币的概念之中的一个分支。我可以这么做吗?



王孟源 01:00:24 

对,第二就是复制它,然后再加上加上他那个发行公司的那个股票期权。



史东 01:00:32 

从一个公司想要集资的公司的角度来看,ICO是不是要比IPO更容易一点。可能我在想,ICO它限制公司集资的这种框框架架,没有像ipo 那么多,



王孟源 01:00:50 

对你讲到核心了,它有两个很大的影响。第一个是因为比特币的这个炒作工具。所以一般的投资人会除了在公司的那个股票期权之外,还会再加上一个价值,所以它吸金会更多。第二个更重要的是,你如果做IPO的话,这个是因为历史比较悠久,所以监管的非常的严,你这个公司不能够是一个空壳子。然后他那个那个会计要求也是非常的严。ICO的话,因为是全新的东西,监管是一片空白,随便拿了钱以后,你要不用都可以。



史东 01:01:27 

所以说现在中国政府就是ICO跟bitcoin全部全部禁止。



王孟源 01:01:35 

我不觉得奇怪,奇怪的是怎么会等到现在才做这样子。 因为全中国是全世界热钱最多的国家,它这个投资管道因为因为必须怕被美国财阀把门放开,所以他的那个投资资金的管制是比较严的。所以热钱一直是一个很大的问题。我说过两年前,他让股票市场炒作上去是一个,但是事实上他要处理这么多的热钱,也是一个很大的问题。因为黄金是绝对没有足够的容量,我们说到这个就是满足不了这些找地方要投资的热钱的需要。全世界的热钱大概有一百万亿。光是中国的那个热钱就有大概十万亿以上,这个十万是美元,不是人民币。像这样的钱你进哪里就炒哪里,这个中国地方这么大,房地产都能够被炒起来。然后股票市场其实容量也很大,你拿这些热钱来炒黄金,黄金都可以炒上十万美元一个。当初其实三四年前人民银行就有官员公开的说,我们不是不想投资黄金了,而是黄金市场实在太小了,连我们的外汇都不够。他们的外汇只有多少,十万亿,这比起热钱来说还算是小case。那你想想看,连黄金都容纳不了这比特币这种东西,你只要这钱里面有百分之一的人愿意花百分之一的这个钱在里面,就是万分之一就可以把比特币炒到几千美元,对不对?



史东 01:03:27 

所以说所以说王先生看样子你是不同意我们去玩bitcoin,那你要去玩可以,

王孟源 01:03:33 

你要知道这个是inaggregate 就是在它里面已经吸了十亿几百亿美元在里面了,到最后都会变成零嘛,对不对?那在这之间,在过未来的三五年,它还可能在翻倍。你还是有可能在赚钱。但是整体来说会是。这就好像你到赌场去,你如果全部的



史东 01:03:59 

就是看你什么时候进,什么时候出嘛,对不对?从从一个从一个大的这个框架来看,基本上是一个losing proposition。

王孟源 01:04:11 

对,赌客会可能会赚钱。但是所有的赌客加起来就是赔钱的



史东 01:04:16 

不然的话赌场盖不起高楼嘛,对不对?我最后一个问题,事实上我们已经超时,我想最后一个问题就是这bitcoin 这种形态或者这种模式,在整个的金融的框架之中,您刚刚谈到了这个叫做blockchain,这是一种贡献,也没有其他的贡献。换一句话说,bitcoin 以及像ICO这种东西的这种数码的存在,对于未来钱币的走向会不会有什么影响?



王孟源 01:04:54 

我觉得不会,这种热钱投机,世界上有两样东西很像,我不晓得大家有没有投机的东西。因为金融都是监管的嘛,虚拟的东西,虚拟的资产都是全靠监管的。所以最新的东西绝对没有监管,没有监管的话就有会从中混水摸鱼。同样的,在那个奥林匹克运动会上的,他也有监管,就是你不能够用药品。所以最新的药品还没有到药的例子,就会有人抢着要这两个市场,我觉得是很有意思的。有人应该也写一个论文来研究两个类似的。我是因为学术界出身,所以你可以看到这个是可以写论文的东西有相似的。我跟你讲比特币这个东西,我刚刚已经提过,就是基本上是完全为投机而设计。那本身没有内建价值,他什么时候会发红利,比特币如果会发红利的话,我觉得你到时候可以你可以算一算他这个红利加起来。你去买股票是要本上是要他的红利。那个你去买债券,主要是他会给你们利息,再加上本金。这些都是可以用那个利率换算回来,换算成当今的币值,然后算一算他的内建价值。这个同样的东西也可以用在比特币上。内建价值很简单,就是零。所以长期来说,而且我觉得这个长期也不会是太长的。三到五年为什么?因为这个他的竞争者已经来了,他们没有门槛,对不对,所以基本上这种这种asset mania,就是冲一直冲到热钱进来的。热钱不足以支持以前的前一代的热钱,就是老鼠会的观点,前一代的热钱要提出去的那个速度是。所以现在各地还在往上升的那个路上。那你看说一下就涨回来看看为什么还是有很多这个钱。因为它的容量其实不大,只需要全世界热钱的万分之万分之一,就是亿分之一就足以把它炒上去了。所以随着广告出去,还会有其他的傻子来进入这个事。但是由于他有竞争者,现在有六百多个,将来有一千多个只是时间的问题。然后开始分散,这种东西就是因为你没办法做空嘛,所以而且事实上你就算有办法做空,我也不建议你做。就是我前面讲的the market can stay irrational longer than you can stay sober。,但是一旦这个由于竞争的其他的虚拟货币够多,然后这种愿意冒险的热钱被用完了之后,这个一旦他的momentum停了以后,这种东西先天就不能够稳定。一旦他不再往上涨,可以开始,而且下降的速度会比上涨的速度要快。,所以这就是它必然会崩溃。所以我们常说泡沫,泡沫就是涨上去看起来很快,但是它崩溃的时候更是快非常的快,完全一眨眼的功夫。所以我估计大概就是三到五年吧。那这三到五年之间去赌博的话,可以。但是这个总体来说,你就是如你说我不但是一个平均的投资者,而且我是超过平均的投资者。那这样子你就有可并不是说个人不可能的。



史东 01:08:46 

换句话说,这个什么时候进,什么时候出。但是你要知道到后来的结果是什么,对不对?



王孟源 01:08:55 

我觉得你还是到赌场去了,至少还有buffet还有show。



史东 01:09:00 

我听你的意见,然后对今天节目的内容,我想对于所有看这个节目的观众,以及听这个节目的听众来讲,我应该都有不少不小的意义了。我觉得我是因为我在在跟你谈这个事情,节目之前就跟你谈这个事情。我觉得我们今天的节目是造一种所谓的the public service,就是给中文怎么翻译?就是像公众服务的这种这种形式来做这个节目。所以说我再一次的说一声孟源,非常谢谢谢谢你的支持,谢谢,高兴的好,谢谢。



王孟源 

哪里哪里,非常荣幸\twocolumn[\begin{@twocolumnfalse}
\section{中兴}
\subsection{20180420}
\end{@twocolumnfalse}]史东 00:12

各位朋友你好,我是史东。再一次地说一声,谢谢您按时地收看八方论谈这个节目。今天的节目中跟你探讨的是一个我相信您相当关心的,也是可能一定很熟悉的议题了,那就是:从中兴之挫来看一看中国和美国之间贸易和技术之间的战争。在今天节目中,你已经看到了,我们为您请到的是王孟源——王先生在节目中都会有一些非常精辟的看法和见解——所以我们今天非常高兴地、再次地为您请到了王先生来和我们谈谈这件事情。王先生,我首先跟你打声招呼,说一声欢迎,说一声谢谢。



然后我要借你几分钟的时间,很快地为我们观众把这个今天的谈论的题目点出来,好不好?今天我用这种方法,是把我手边的所有 Download 下来的媒体上的消息,我用他的一个一个的标题按照顺序讲过去,用这个方式来把这今天的主题把它点出来。然后我们就可以开始请您来谈谈您对这个事情主要的一些看法和见解。



(他首先回顾了一下他的分析)这个事情从 3 月 23 号开始——说美国计划对进口的钢铁要加征 25\% 的税率,为铝产品加征 15\% 的税率,然后中国迅速回击,中国计划(还没有实施)对美国的水果以及其制品等 120 项进口的商品要增加关税,税率是 15\%;对猪肉及其制品等八项进口的,要增加关税为 25\%。然后到了 4 月 4 号,美国政府发表了一个叫做加征关税的商品清单——将对中国输美的 1333 项,一共总计有 500 亿美元的商品加征 25\% 的关税。接着下来在同一天 4 月 4 号这一天,中国的国务院的关税税收委员会的决定——对于原产于美国的大豆、汽车、化工品等 14 类 106 项的商品加征 25\% 的关税,实施的日期另行公布。接着下来,4 月 16 号这天,美国的商务部宣布——就是今天我们要谈的这个主题——未来 7 年将禁止美国公司向中兴通讯销售零部件、商品、软件和技术。然后到了第二天 4 月 17 号,中国的商务部公布对原产于美国进口的高粱实施反倾销调查。他的初步的裁定,说决定要对原产于美国的进口高粱采取临时的反倾销的措施。换句话说,就是从今年 4 月 18 号起,进口原产于美国的高粱的进口商或者美国的商人,将要向中国的海关提供将近 178.6\% 的保证金——178\%就是就是一倍多了,一倍半还多的这个保证金。这是最新的有关于中国和美国之间的贸易战的情况。



然后我就根据我手边的几个头条,把今天的题目点出来。这个头条他说:中兴被禁,华为以及海康告急。下一个新闻说:中兴高层慌了,说美国这一拳打到中国通信业的要害。接着下来 Forbes 他说:软硬件全部都遭断货,中兴或者会在几个礼拜之内破产。接下来一个新闻:中兴的董事长他说,美国这次采用的(制裁)使公司立即进入休克状态。中兴发表最新的声明,他说这极不公平,不能接受。另外,中兴通讯的董事长他说,认真反思,加大研发投入,中兴旗帜将永远地飘扬。另外,中兴通讯的董事长他说,有能力、有决心渡过难关,绝不放弃。下一则新闻的标题是:中兴、华为之后,美国的下一个目标会是阿里。接下来这是一篇文章,我把这篇文章的第一段很快地介绍一下,然后我们就开始今天的讨论。这篇文章的题目说:这是中国和美国的技术战争。他是这么说的:他说美国处罚中兴的(措施是)不卖芯片以及元器件的,这根本不是贸易战,而是技术战争。美国发动了贸易战,代表了美国的觉醒,开始正视中国对美国的挑战。而美国发动的技术战争却相对地引发了中国的决心。这是我今天点题目的过程。



孟源,你对整个这个事情,你的反应是什么?你的想法是什么?



王孟源 06:01 

事实上,我当然是很关心这件事情了。我在上个月已经写了一篇部落格的文章来讨论。那时候基本上还是写关税战争的这个范畴,那现在又继续升级了。就像你刚刚讲的,目前真正中美双方,除了中兴这件事情是已经实行以外,关税其实都还没有加上去。那中方已经实践的步骤,也就是只有对高粱的那个——因为它并不是一个对等的 25\% 关税的措施,而是所谓的反倾销,所以马上就实现。而且又是 100\% 多。所以当天一宣布,当时有三十多艘船正从美国载着高粱往中国去。根据今天 Reuters 的报道,在接到消息后,所有的船都掉头往里头回去了。因为他不可能付出 100\% 多的保证金。所以目前真正已经实现的步骤——就是已经动手,已经打到肉痛了——就是美国对中兴的制裁。中国反过来是对高粱动手。其实高粱的贸易并不是很大,一年大概十几亿。中国之所以对高粱先动手,是因为真正的重手是下在大豆上面,但是现在正是南美的收成季节,所以中国对北美的大豆进口现在不是高潮期。而且他说也要等到 6 月才生效,所以先对高粱动手,证明说自己不是空口说白话。



我们先退几步来看看这个大的背景。Trump 是一个很不典型的政客,也是史无前例的美国总统。他在竞选期间到现在当选之后一年多,说了很多很多话,很多是自相矛盾的。但是我们如果整理一下真正实践的结果——在一年多前他刚当选的时候,那时候我对他头一年的预测是:他会专注在拆除 Obama 的政治遗产。这是因为很明显的在 2011 年有一个事件。就是 Obama 当选之后,共和党有一大堆莫名其妙的 Conspiracy 说他不是美国人,说他是回教徒,说他的出生证明是一个很大的问题。从 2008 年到 2011 年,这些问题政经的政客已经不再提了,只有一个人还在那边放炮,那就是 Trump。到了 2011 年 4 月,Obama 也被他烦的受不了,就跟另外两家联手——一个是 Washington Post,另外一个是 Seth Meyers。那一年刚好白宫记者年度晚会是 Seth Meyers 当嘉宾,在台上开玩笑。Trump 是从来没有去过的,但是由 Washington Post 出面,特别请他去出席。出席之后,Seth Meyers 火力全开,整个晚上就是拿 Trump 开玩笑。我想 Trump一辈子没有受过这样的歧视。经过那一次以后,Trump 说要选总统。以前说要选总统并不是没有,但是经过这一次之后才认真的。他这个竞选其实就是为了找一个借口来骂 Obama。所以一年多前他刚当选。我就说当然共和党重掌政权,所以他的建制派——也就是国会那些人——会想办法推行共和党那一方财团的利益,立法方面仍然是有利于共和党方面的。但是 Trump 个人的决心、做总统的个人的政治理念,就只有一项——就是为了报仇。报仇的话就是 Obama 有什么政绩他就反什么。我们看看他头一年一直到今年 1 月 2 月,的确是这样子。他发布了几十个行政命令,可以说无一例外都是跟奥巴马对着干。连奥巴马所创立的国家公园,他也要把它砍掉。如此 Petty 的地步。最近他又开始对着...谈话,对着 Amazon 的老板 Bezos 在说 Bezos 的坏话。那为什么 Bezos 变成他的敌人?因为 Bezos 就是 Washington Post 的老板。所以你可以看出这都是有脉络可循的。



不过在大概在今年 2 月、3月的时候,忽然有一个很大的转变。就是当初他的那个内阁形成的时候,建制派掌握了绝大多数的内阁成员,从国务卿一直到财政部长。国务卿是石油派的,那财政部长是高盛出身,这些都是典型的建制派,唯一的例外:一个是Bannon——他以前的智囊 Bannon;那另外一个就是商务部长 Ross。那时候我就说:因为 Bannon 是真正有理想——不管这个理想是不是错的——但是他是唯一一个有理想的人,所以我一年多前就说他大概做不满一年。那他的确也做不满一年。但是那时候我没有想到的是,Ross 在一年之后会忽然变成内阁里面最强势的部长。我个人认为是 Trump 经过一年的建制派的掌控之下——因为他们常常开玩笑说是得到 Day care,就是这些人把 Trump 当做一个小孩子一样来看待。经过一年之后,他已经熟门熟路了,就不再受得了这样的控制。然后第一个就是清理门户,最重要的就是 Tillerson 被一脚踢了出去,这个对外的政策就完全由他自己掌控。最近的消息是,他的白宫幕僚长 Kelly 也已经被打入冷宫。这样一来内政外交都完全在他自己的掌握之下。



Trump这个人没有什么才能,他唯一的一辈子的经验就是撒谎吹牛,骗人讹诈。在政治方面,他唯一的成功的经历就是选举。所以他现在真正掌权以后,回过头来把他以前选举时期的口号拿出来实践,其实并不是奇怪的。就是刚好在一年之后,建制派被他一脚踢开,到期中选举目前还有六个半月,这个半年多的时间刚好是一个空档期。因为上一次选举是他带领共和党大胜的——共和党的议员虽然自己的良心跟专业知识都知道 Trump 是在胡搞,对国家、社会、经济有严重的损害,但是他们不敢出面。因为他们这些民主的政客最终的评价就是选举。上一次真正的选举是 Trump 引领着大胜,到现在的所有的民意调查,在坚定的共和党选民之中,他还有 80\% 的支持率。他的总支持率是美国近代史上最低的——就是有民意调查以来最低的——只有大概 40\% 左右,基本上除了共和党人之外没有人支持,但是坚定的共和党人里面有 80\% 支持。那这样一来,这些期中选举还是主要先经过初选,这些国会议员如果初选的时候跟 Trump 作对的就会被刷下去。所以目前,也就是说在期中选举之前(他们不会作对)。期中选举是一个很重要的关口,到了期中选举之后,整个政治生态就会完全洗牌。我们回想一下 8 年前,也是一样的。Obama 上台也是同时掌控总统白宫跟两院的国会,在头两年也是推行了很大的变动。不过他们的变动是真正的改革,也就是 Obama Care。那时候民主党人也知道风向不太对,但是因为 2008 年的选举是民主党大胜,他们也不能够出面反对。现在也是这样子,共和党人知道不太对,但是上次的选举余威还在,他们不能够出面顶撞总统。



所以因为选举周期的关系,今年刚好是 Trump 可以自由发挥的时间,然后他自由发挥的结果就是把建制派推到一边去。建制派也推到一边以后,有谁愿意依着他呢?就是独立财阀。Ross 刚好是一个独立财阀,它是不属于任何一个大财团,他自己就是一个 Billionaire,跟 Trump 一模一样。那他们两个人凑在一起一谈,就会把一些经不起真正经济学考验的观念拿出来实践。他们真正以为以前的那些政客跟经济学家学者所说的是假话。什么样子的话他们认为是假话?就是中美互相经济依赖,你中有我,我中有你,所以如果打贸易战的话,会杀敌一千,自损八百。这些话其实很浅显的道理,而且是完全正确的。但是 Trump 不相信,因为他不学无术;Ross 也不相信,他不学无术;Navarro——他的贸易政策主任——他是个疯子。他虽然是学术界,但是他是真正一辈子讲我们要跟中国做生死斗争的。



所以现在到未来这半年多,我想 Trump 的政策里面,跟中国做斗争是一个主轴。那这个斗争不是军事方面的,主要就是在贸易跟经济上。所以他们第一步是做关税战争。这个关税是第一步是你刚刚讲过的,对 500 亿进口做 25\% 的关税。500 亿是什么意思?去年中美的进出口贸易,中国对美国的出口是大概 5000 多亿美元。那美国对中国的出口是大概 1300 多亿,这个比例大概是 4:1。所以你可以想象 Navarro 和 Ross 在跟 Trump 商议这件事的时候,他就是看我们进口的数量是出口的 4 倍,对不对?我如果加关税,我只要加到我们进口的总额的四分之一,那么中国就没有东西可以再加关税了,对不对?所以他第一步是先加 500 多个亿。然后看到习近平居然敢对等的,也对 500 亿的进口加 25\% 的关税——就是完全对等的——他这自然的第二步就是要加 1000 亿的关税。所以这些数字都是有意义的。我觉得他一旦开动,后面这两三步都是全自动的。就是一开始 500 亿选的是什么?是工业用品。就是 1300 多亿里面有 500 亿工业产品,就是铁跟铝半成品或者是原料这些东西,剩下的是消费品。他先对工业品动手,因为这样子选民开始不会感觉到东西打上去了。但是一旦中国不退让,下面这个 1000 亿是很自然的,这样子总共 1500 亿。全部美国对中国出口,也就 1300 多亿,你怎么可能有对等关税呢,对不对?所以在开打之前,Trump 就说——他这是公开对记者说——贸易战很容易赢。为什么很容易赢?因为我们进口 5000 多亿,我们出口只有 1300 多亿。我只要对 1500 亿美元的进口弄关税,对方照理说没有办法做对等报复。



但是你如果看的深一点,这个就是似是而非了。我在我的博客上已经解释过了,就是事实上美国人在过去这十几二十年中国经济崛起之后,对中国在美国购并投资非常有警惕。四五年前有一个很明显的案例:在 Alabama,中国的一个钢铁公司要买 Rebar,就是那个钢筋。这个东西是一百多年前就有的,可以说是八竿子打不着的“高科技”。但是就是要买一个小小的 Rebar 厂,都不让它来。所谓的投资是什么?投资是你到一个外国去设工厂设公司,然后可以再给你卖东西。就是说你不需要直接从国内出口,你可以直接从在外国境内生产到销售到赚钱。中国虽然对美国的公司进来是有限制的——比如说汽车就必须要对半出资,找一个本土的公司。GM 通用现在有超过一半的利润是来自中国,他必须要跟上海汽车——上汽合作,不能够自己开一家公司,这是有限制的。但是不管你有没有限制的,至少美国人是可以去投资的。但是中国人不能够对美国投资。你如果真正看看美国公司在中国的销售额,跟中国公司在美国的销售额,事实上那个比率不是 4:1,而是 4:3。就是美国公司事实上不愿意从美国直接出口,因为美国的生产成本很高嘛,对不对?他们如果要在中国卖东西,他们本来就是要到中国去生产(在当地生产,在当地销售)。然后你再考虑一下中国卖给美国的这些东西,很多是组装或者是最后的简单加工。比如说 iPhone :iPhone 的销售、设计,这些真正的利润的大头都是在 Apple。现在一个 iPhone 卖到 1000 多美元,这里面中国的附加价值有多少?还不到 40 美元,这 40 美元里面它的利润有多少?不到 1 美元。就是 Apple 的利润有 400 多美元...



史东 24:42 

我趁这个机会澄清一下。换句话说,从美国的立场,从 Trump 的立场,每 1 个 1000 块钱的Iphone 卖回到美国,都是中国卖到美国的,



王孟源 24:56 

算是几千美元的利差,但是实际上中国只赚不到一块钱。



史东 25:05 

可是问题说到这儿了,这么浅显的事情,我们一谈就知道了。难道川普为什么不知道?你说他是装傻,还是真傻?



王孟源 25:17 

他不是装傻,他是不学无术嘛。你看他当初去那个宾大的 Wharton School——这个是靠关系进去的,不是靠他自己自己的本领进去。因为他爸爸爸已经是财阀,对不对?你也知道美国的大学高等教育并不是看学生的水准,而是看家长的水准,看家长的口袋。所以他真的是不懂。你像 Trump 一旦宣布要打这个贸易战,Gary Cohn 就辞职了。Gary Cohn 是谁?是高盛的前总经理,他代表的是良心,这方面跟高盛没有关系。但是他因为是懂这件事情,所以他一看说你这个太离谱了,我不愿意再为你买单。



史东 26:20 

我们要节省一下时间,有很多的问题我想请教你。就一个大的层面,你刚刚介绍了这个 Trump 他为什么上台,他上台之后他的这个 Motivation 很大一部分是要打垮他的前任 Obama 总统的种种措施。然后他现在的这个对中国的贸易的这种看法和他的做法——你刚刚也提出来了——是非常不靠谱的。我现在问题就是说:你觉得他要到什么时候才会知道他这么做是错?因为这是一个很重要的问题。



王孟源 27:03 

期中选举。你在期中选举之前,要指望他迷途知返是不可能的。这个就是要看期中选举共和党被血洗的程度。目前,我想连共和党的众议院领袖都已经要退休了,你指望这个形势是怎么样,我想是很明显。是不是真的会这样子,还是有疑虑的。但是真正的内行的人基本上都知道,共和党会被血洗的机会很大,所以才会有退休的。



我刚刚讲到 iPhone 这个例子,所以你看一下 1000 美元的进口额,中国的利润还不到 1 块钱。我刚刚讲说你如果只看进出口的那个数额的话,是 4:1;如果看销售额,那个中商对美商的销售额是 4:3;但是如果看看中商对美商的利润的话,反过来是 1:2 还不到,就是还不到一半。那你看看,如果真正把纯粹的贸易战打到最后,看谁的损失大。损失不是看你的那个进出口。进出口只决定雇佣,就是这个工作、生产数据。但是真正这些厂商承受的损失是用利润来计算的,这个是机会成本来算的。所以事实上美国如果要打这个贸易战争,没有底气打到最后,所以这个东西就出来了。



为什么他会打中兴?这件事其实是在 2016 年的时候...那个时候美国因为发现中兴违反对伊朗贸易制裁,所以他被罚了 8 亿多美元,而且有 35 个雇员要被处置。第一个是中兴本身的纪律差——中国经过 30 多年的飞速经济发展,国人急功近利的态度是很普遍的,而且守法守纪律的态度(失控),那么中兴自己是有问题。但是事实上两年前美国商务部去抓中兴就是很可疑的,为什么?因为刚刚在中国对 Qualcomm 实行了 10 亿美元的罚款—— Qualcomm 臭名昭彰。就是 Qualcomm 他在 10 年前左右,在 CDMA 的技术上用专利讹诈了几百亿美元的利润,就是专利权。因为他那时候掌握了 Cellphone的 Spectrum,所以每一个人都要申请一个专利。他在韩国、在欧洲、在日本都被罚,后来中国在 2015 年也罚了他 10 亿美元,罚之后不到一年中兴就被罚了8亿美元,所以说这都是政治上的。不过那个时候,美国人还知道自己有点理亏,而且他是在 Obama 底下,所以出手还是有点分寸——就是八亿美元,然后说你要处置这 35 个。现在的中兴的事情到底是怎么样?很简单。没有新的犯错,同样一个案子,唯一的问题是什么?35 个人里面有 4 个是高管,这四个人必须要开除。剩下的那 30 几个人是比较低级的职员,他们是说必须要罚钱,就是你的奖金要扣,这个是写在你们当初跟商务部的协议上。那么到今年年初,中兴内部换了律师事务所,律师事务所一看不对,你那 4 个人是开除了,但是另外那 30 几个人的奖金没有扣掉。这个是一个细节,违反了一个细节,他就说那你弥补一下,赶快去补扣,然后赶快跟美国商务部认罪,说我们不小心弄错了,而且他的前任的那个 Chief Compliance Officer 也因此被开除。道理说已经是诚意满满了。这个律师事务所是美国的律师事务所,他之所以会这样建议,是因为事实上在美国就是这样的。你内部犯的错,自己赶快招认,然后赶快弥补,应该就没事了。可是这么小的、芝麻绿豆点的小事,这个 Ross 刚好正要找中国的麻烦,于是他就把它放大了一万倍,



史东 32:43 

就做今天我们所知道的这种处置。我想有一点,就您刚刚提到这个 8 亿的罚款,我一直搞不清楚。这 8 亿的罚款还是在进行中吗?还是说已经交了,对不对?换句话说,这是在 8 亿罚款之上,另外再做新的处置。



王孟源 33:10

就是两年前的协议是你交 8 亿的罚款,然后对这 30 几个人处罚,那就没事了。那他交了 8 亿的罚款, 30 几个人里面最严重的 4 个也真的开除了,就是后面的奖金...为什么没扣,没扣是因为他内部纪律差。



史东 33:26 

我们进一步的了解——就是这 8 亿已经付了,而且已经是过去式了。那么现在中方自己发现有些还没有完全补全的一些处理过程,现在发现了之后,把它补全。在美国方面,就利用这个机会,另外就再狠狠地加了一把处罚



王孟源 33:52 

这个非常的狠。因为中兴不是华为,中兴是世界上四大那个通讯供应商里面第四号,华为是第一号,后面两个是爱立信跟 Nokia——就是两个欧洲的厂商——然后第四号是中兴,那从第五名开始就无足轻重了,所以基本上世界上是四个大头,他是最小的一个。但是它跟华为不一样,华为是有自己的 CPU 然后很多零部件,还有软体支持来开发——即使还没有用上,也都还在。中兴一向是造不如买,买不如租的。他都是跟美国买芯片,然后跟美国 License 那个软件。一旦美国断绝供应,中兴的生意基本做不下去,是非常的严重。Ross 不是要再罚钱了,这个现在已经不是 100 亿的问题了,其实就是要搞垮你。而且中兴我觉得是要垮,如果我是中兴的话,我现在不会再浪费时间要在内部重整,什么激发士气。赶快做金融方式的处理,就是把那个公司拆分卖掉。因为你这样的生意,做不下去了。



史东 35:36 

我想又是一个问题,美国政府的这种做法。因为在我的观察——我很久以前的时候,也在 Silicon Valley 做过事情——所以说至少在 80 年代、90 年代的时候,那个时候PC、PC Compatible 那些东西,基本上的观念就是今天中兴的观念吧,就是你刚才讲的做不如买,买不如租。就是说我把你的硬碟,把你的 CPU,把你的什么这个 Memory 弄来之后,我自己就把它装起来,然后这是我的品牌,有我的 Knowhow、一点附加的价值在里面,说不定有我的一些软件。这些工作就是说和它的上一代,所谓的上一代就是美国的这种 Mainframe computer 那种,就是从 CPU 一直到下来,全部都是一个公司自己的的东西



王孟源 36:38 

是这样的,全球化分工本来就是这样子。那你从一个低端的经济,低端的技术要进入,你当然是从组装开始,组装的是人工



史东 36:48 

就是这个。我的意思就是今天美国对于这个中兴的这种处置,全世界有多少像中兴这种商业模式的这种公司。这种工厂在他们自己国家之中,或者他们的国家知道自己的国家之中,有很多这种样子的这个operation,会有些什么样子的反应,以及会有些什么样的影响。基本上美国独霸这个半导体工业嘛,还有软体工业嘛,作业系统,还有就是软硬体这两方面,它独霸了上游的工艺。我记得二十年前有人说,中国最羡慕的美国的两样东西,一个是华尔街,一个是硅谷。那后来十年前那个金融危机之后,大家知道那个华尔街没有什么羡慕的,应该要避免。所以你剩下就是硅谷,中兴这样的公司,其实全世界都是这样子的。除了美国之外,就是这样子,



史东 38:03 

他这样子基本上就摧毁了这整个的企业形态嘛。



王孟源 38:07 

而且这是史无前例的,美国从来没有对苏联之外的任何一家公司像这样地动手。中兴不是一家小公司,这个是全球第四大通讯设备公司。它真的是一刀致命,这个是头已经砍下来了,嘴巴还在讲话,但是头已经掉下来了,这已经没有救了。但是如果局限在中兴的话,对中国其实不痛不痒,为什么?这四家通讯企业,华为要比中兴大两倍还多。他们的(产品)价钱和品质定位的基本一样,就是比欧洲要物美价廉。所以中兴一旦这样倒闭,替代中兴的主要是华为,不是 Nokia 和爱立信,所以反而很好。就像几年前那时候中国的铁路出口(铁轨火车的出口),有中车的南车跟北车在竞争嘛,结果是自己杀的一塌糊涂,血本无归。所以它们还特意把南车北车合并为中车。因为你以前在内销的时候是有两家公司竞争,这对自己的国内的买家比较容易;等你都是要做外销的时候,你再拼命地竞争,买家都是国外的,那就不需要对他有如此的照顾了。



事实上美国这件事情,如果只做到中兴为止,其实是在替中国做他们自己没有来得及下手的,就是他们整并两家公司成为一个所谓的 Champion——我们那个英文里面叫做 Industrial champion,就是工业里面的冠军。Champion 这个字并不是代表第一名的意思,它原本是代表中世纪的时候,有所谓的 Tournament。就是那时候你有一个代表你的武士,就是所有这个家族的希望都放在这个武士身上,这个是 Champion 的意思。这个 Industry champion 就是用这个中世纪的古典的含义。中国事实上没有必要有两家在同一个工业通讯工业的都同时外销互相竞争。华为既然比较大、比较健康,而且技术层次也比较高,把中兴弄倒其实是一件好事。所以你看中国现在虽然是对这件事情很恼怒,但其实并没有马上动手反击。他们正在合计说下一步要怎么样,对不对?



那如果我越俎代庖,当中国的领导阶层,考虑一下他们这个面临的局势是什么。刚刚今天美国又传出来,大概在下个礼拜会宣布下一步 1000 亿美元的关税。这个时候就超过 1500 亿进口的商品。而且这次是有对消费品的——因为那个工业产品就只有就这么多,大部分都是消费产品像 iPhone 这样的。不晓得他这些消费产品会选什么,因为全部有 5000 多亿,他要选 1500 亿并不难。那中国对第一波的那个 500 亿的对等反应里面,主要选的是农产品大豆。这是因为 Trump 跟共和党在最近的几年的支持者,主要是在农村里面那些没有见识的乡下人,所以容易被他哄。Trump 我想可能是美国历史上第一个在他自己出生的州里面——在总统大选的时候在自己出生的那个洲大败的。你如果看看 2016 年的选举,纽约州基本上是一面倒地一个选择,因为纽约人知道 Trump 是什么样的,但是既然 Trump 的支持者是来自乡下,所以先对大豆动手。这个是让他在期中选举的时候更加的困难,这是很合理的,但是下一步要怎么办?美国这样一来 1500 亿...(中国)再下一个 1000亿,你根本就没有那么多进口可以再加关税。这时候我们必须要打击在中国内部生产销售的美商。这时候最好的对手就是像通用汽车。



史东 43:50 

基本上从中国的角度要打击美国在中国的这些生意,对不对呀?



王孟源 43:59 

你应该挑利润最高,而且一定要有两个选择的要点:第一个,他的利润很高;第二个他的替代性很好,就是中方有可以替代的产品。那么我认为有两个工业很适合,一个是通用汽车。通用基本在过去的十年是靠在中国的利润来补贴他在美国的生意,他连那个欧洲的生意跟南美的生意都已经越做越小了。现在基本上就是中国跟美国生意。那美国他不怎么赚钱,



史东 44:38 

我这是有一个我自己写给自己的一句话,那就你回应你讲的这些事情,就是如果你是这个中方的主政人。我这一句话就是说在所有在中国的美国企业我要中国列出一个清单。这个清单是按照他伤以及自伤的比率列出一个清单,然后照表实施。换句话说,因为你有这种反应的话,一定也会伤到人也会伤到自己,对不对?这个这个是美方好像没有想到,但是中国很容易就想到的一件事情。所以你这个清单下来是按照自伤跟他伤的比率排下来的。你要开始找自伤最高自伤最低的这个企业去下手。然后一个一个照表实施。这是我给自己写了一句



王孟源 45:42 

通用动力的那个雇佣的人刚好是在中西部乡镇,还有 Ohio, Illinois, Michigan,这些也都是最近这几年从民主党转共和党的州。那如果你把通用的那个中国市场拿掉的话,通用马上就会有问题,马上就必须要裁员。在中国本身,中国最近这几年本土的车商开始正在往上崛起,完全可以进来(替代)。还有第三个考虑就是:必须对欧洲杀鸡儆猴。我在我的文章里面一再强调,真正中方跟美国这样对干,其实都是只是战术上的一个细节,战略上实际上没有太大的选择。因为你不可能无条件投降,对这件事情,如果无条件投降,美方并不只是真正要平衡贸易,他是真正要扼杀你的那个决心,为什么?



美国在二战之后,因为其他的先进国家被打成一个稀巴烂,只剩下美国基本上没有受到战争的威力。所以他在二战之后非常喜欢自由贸易。为什么?如果是自由竞争的话,有谁跟竞争的过,你们都是废墟,对不对?但是在冷战时期,他必须要有附庸,附庸国有小弟来帮他跟苏联对抗。所以他就容许像德国跟日本,跟英国,跟法国这些国家扩充到什么程度,就是 PPP 就是那个Purchasing Power Parity 用购买力平价来算,每个人的 GDP 到美国的 70\% 的程度。如果你是在这以下,美国会用心地帮助你复苏。但是你如果超过到 70\%,我就要打你。这个在 1985 年的那个 Plaza Accord,这个是很明显的开始准备要出手打英国、法国、日本跟德国。其中对日本最为成功。五年之后(即 1990 年)日本的那个经济就整个爆掉,然后到现在还没有恢复。德国是他失败的最大的,为什么?因为刚好引发的那个金融[xxx]被德国两国统一给对冲掉了。那一对冲掉,德国也知道这个形势不对,因为那个在 1990 年日本就已经爆掉了。德国制造这个是暂时的[xxx]。所以他马上在一九九二年签了 Maastricht Treaty,这是什么东西?——就是欧元的开始。



所以我们现在看这个所谓欧元是什么?欧元其实是德国讲出来避免被美国用 1985 年的 Plaza Accord 那种金融手段...不只是剪羊毛。因为美国在冷战时间,他让给这些小弟的东西——工业经济、贸易,但是他掌控的是什么?金融、外交跟军事。德国是没办法跟他们[xxx],但是你的贸易跟经济要继续发展的话,必须要让自己的货币不受美元波动的打击,那最好就是把整个欧洲拉起来,做成一个欧元,自己躲在里面。这个体量够大,你就不必怕美元。



我说这么多,你可以想想看...所以我们现在看过去的十几年,中国真正崛起之前,世界的那个领导机制是什么样子。你有一个第一世界,第一世界是美国做投资,他的经济实业工业是慢慢地被掏空,不过真正的高科技还是掌握在他自己手里。然后有一部分分散在他的小弟手里。这些小弟因为在军事外交上面必须要完全依赖他,在金融方面也不能够放开,所以是完全可控,对他来说完全可控。那这些国家加起来有多少人口?美国是三亿人口,欧盟是五亿人口,日本是一亿,然后我们再加上零零星星的澳洲、加拿大。但是你要记住,这些国家人口很少,你加起来还不到十亿——世界人口已经快要到八十亿了——就是全部加起来第一世界,你连东欧那些小国,像那个匈牙利都算进去还不到十亿,还不到 12\% 的人站住了技术跟经济的第一梯队,他们就有无限的红利。因为比如说你要买一个核电厂跟谁买,你必须要跟第一梯队买;你要买芯片跟谁买,跟他买,你要买手机,你要买无线电通讯,这个谁买?必须要跟这些第一梯队。中国的兴起是什么意思?13 亿人要加入第一梯队.他这个野心在 1949 年建国的时候就有,到邓小平掌权的时候他说韬光养晦,意思是:我们还是有这个野心,但是我们装作是人畜无害的小白兔。但是事实上中国的人口超过目前第一世界整个军队的总人口,美、欧、日什么加起来。中国加进去,不但第一梯队一下子从人类的总社会的 12\% 变成 30\%。而且中国的价钱特别低,他的效率特别高,你这样一来,这些现有的收租的 Rent seeking 的第一梯队以后要吃什么,我觉得这才是欧美目前在过去的几年对中国敌意很快越来越深的原因。



史东 52:34 

我想这个您说的这个事情,比如说中国加入第一梯队的这些国家,事实上我觉得你的意思是中国变成第一梯队这些国家的竞争者



王孟源 52:48 

就是他马上要加入第一梯队。为什么中国制造 2025 三年前一出来,我就知道这是很重要的事。我说这是中国的习近平政权的核心任务——一带一路只是为他创造外部条件,这个内部反腐是为他创造内部条件,这是什么意思?就是中国要加入世界第一梯队。美国跟欧洲反应比较慢一点,但是三年之后他已经看清楚了。



为什么德国明明是对中国有很大的顺差——德国最近的经济过去这十年顺风顺水,就是因为卖东西给中国——这样的一个国家,对中国现在都敌意满满,为什么?就是因为他知道这个不能能够持续。中国的意愿就是在 2025 年的时候就加入第一梯队,而加入第一梯队之后,他们这些利润都没有了。所以你现在看他们这些宣传说什么我们当初认为对中国开放的 WTO 自由贸易,他们会变成民主化什么的,这个都是借口。其实他们在不在乎中国民主化,中国如果不民主化,但是也不进步,他在乎吗?不在乎。墨西哥、巴西他们的经济发展进一步退两步,这些欧美在不在乎?真正在乎是因为中国威胁到他们的利润。所以现在美国像是 Trump Ross Navarro 他们针对中国在做这些东西,其实都是发自于这个动力。而欧洲尤其是德国也有这个同样的担心。所以明明德国是一个主要出口国,它去年的出口顺差达到 GDP的 7\%,中国的出口顺差只有百分二点几,那这样的一个国家到现在还在那边观望着,他们的下个月 Merkel 会到 DC 跟 Trump 会面。我想他主要是说请你不要这样子搞这个贸易战争,我们可以在WTO的层面更改WTO规则,让中国的经济有困难。改什么规则?就是规定比如说像国营企业就会被歧视,就可以处罚。你可以单方面地处罚国有企业这样的东西。也就是说:中国的制度是怎么样,就是非法的;美国的制度跟德国的制度是什么样,就是合法的。这样很简单地定这种罪。所以我觉得在中美贸易战争中,中国其实长期来看最重要的应该是争取欧洲。欧洲的重要性在于什么?因为欧洲如果不参与,美国就没办法重新制定改变国际贸易制度、贸易规则。



史东 56:08 

所以我们可以说欧洲是一个所谓的关键少数



王孟源 56:12 

对,举足轻重。因为日本跟澳洲基本上是很乖的,这个美国说什么马上点头的。但是美国跟日本跟澳洲合起来,能不能改变国际贸易制度?不能,你非要有欧洲加入。



史东 56:33 

现在很自然的一个问题,你觉得中国应该用什么方式来把欧洲拉进自己的阵营,或者拉近自己的怀抱之中。



王孟源 56:47 

这就是我刚刚已经开始准备要提的,就是又拉又打,这个很奇怪的是,中共几十年前一百年前是统战高手。现在好像都忘光了,我刚刚说的是 GM,另外一个很适合打的 P\&G,这些都是利润很高,中方有替代性,然后对美国伤害极大。还有另一个特点就是欧洲也有对应性的企业。那个VW,在对中国市场的依赖跟GM一样的,联合利华跟P\&G在中国一样,是每年百亿百亿地赚钱



史东 57:36 

所以说这个应该很好做嘛。就是说我把欧洲的品牌来取代美国的中国市场的,



王孟源 57:45 

对,我现在去打GM,去打P\&G,欧洲先占便宜,但是也杀鸡儆猴。因为你如果不乖的话,我同样可以处理。而我有内部国内的厂商。如果你们两家都不好,我就由我们国内自己的厂商来取代。所以那个中方有些人在说,你在这个对波音动手,我觉得你对波音动手非常不合适。因为你相对来说波音动手唯一的替代厂商就是airbus。那这样子他就有恃无恐。因为欧洲这时候就知道,你只有糖果没有大棒,对不对?所以要动手,必须是有糖果有大棒。两个兼有,中国目前的态度似乎是要以拖待变的,就是在战术上先美国做什么,我们就说做对等的,但是不主动升级。然后等到十一月中旬以来,这是合理的,但是我觉得必须要把预案做出来。因为事实上美国在下个礼拜可能就会宣布下一阶段的一千亿美金的关税。这个时候你要怎么办?你没办法对等做关税,你必须要开始准备要宰GM和P\&G,这个宰下去以后,刚好时机也对,在merkel跟trump会面之前,就先震慑他们一下。先给欧洲一些甜也让他们知道后续是有杀手锏。



史东 59:25 

那句话怎么说的,糖果跟棍棒是在一起,是side by side在一起出现的,对不对?



王孟源 59:31 

对,你跟trump没有什么糖果可言嘛,他现在就是一心要杀了你。那这个你除了跟他决斗到死之外,你有什么好讲的?但是欧洲目前是鹬蚌相争,渔翁得利,他们看的很高兴。但是实际上他的心里是想要加入美国。为什么,他们知道长期来看,中国是最大的威胁,因为美国已经去工业化了。但是对美国来说,他跟中国认为是生死搏斗,为什么?因为中国威胁他的不只是经济跟贸易,而且在外交跟军事上也在威胁他。欧洲从来就没有外交跟军事的优势,他唯一的优势就是科技跟经济贸易上的优势。这个可以用贸易上的利益,用贸易上的糖果加大棒来改变他的心。但是我觉得中国,习近平当然是一个很英明的领导。但是他的中级官僚有很多不适任的,就是因循苟且,这个不可能在五年之内就全部换洗了。那么他最弱的几个方面就在宣传外交,还有高科技管理这几个方面,这都是专业性很强的,那他们都是做的不好。那我们目前看到的就是因为美国打击的这些事情,刚好就是需要这几方面的官僚来做()。比如说美国人是不讲国际规则,在他们违反规定在犯贱耍下三滥的时候,他必须要先做宣传。他现在宣传的是什么?他说中国要求技术转移是一种偷窃,就是那个用theft。这个你跟美国人没有什么好讲的,就是要蛮不讲理,但是我看到了欧洲的媒体也开始讲了,也开始说要求技术转移也是,但是在英国跟德国,他们是有毁谤法的,跟美国不一样。这种东西你很明显的,你说偷窃,偷窃是一个法律名词,你可以到他们那个法院上面请问你们这个法院历史上有没有对任何一个人定了偷窃罪,他是因为要求加技术转让的,没有。



史东 01:02:26 

孟源,我们接一个电话,有一个电话进来,这位朋友,谢谢你的电话。贵姓,

张先生(01:02:31 

我姓张,史东你好,我很简单的讲一下,我们刚才谈的就是说中美就是在贸易对抗的这个一些方式。但是有一点我想要补充一点,我觉得中国有一个很大的利器,在这个关键点的时候可以拿出来,就是释放美国的国债。中国有大概百分之二十的美国国债,这一杀下去可以把美国经济杀垮。所以我就提到这点。

王孟源 01:03:04 

我是做金融的,正好是我的本行,我觉得一般人太高估这个国债的力量。因为你的外汇如果是美元的话,你就必须要买国债。为什么?没有任何其他的资产,包括黄金在内能够储存三万亿的美元资产。那除非你就是改成美元现金。改成美元现金的话,你就没有利息,第一个你就没有利息。第二个,你还是没有杀伤美国嘛,因为还是美元嘛,还是他们凭空印出来的钞票,所以只要美元还是国际储备货币,你这个买国债这种东西,是没有办法完全避免。而且事实上即使好吧我们假设你真的有办法把国债全卖了,两万亿美元的美国国债全部卖掉然后全部换成欧元,你这么一转,你必须要损失几千亿的损失价值。但是我们假设你愿意这样做,对美国有什么损失?没有。为什么?其他国家一样购买。尤其是现在的时机非常的糟糕。美国的那个经济正在复苏当中,他那个现在正要加息,事实上今年的那个联邦联储会要准备要加息,在加息的时候就不怕没有人买你的国债。而中国所持有的国债是美国国债总额的多少?百分之十一,你把这个百分之十一拿出来卖剩下的百分之八十九的人,愿不愿意每个人再多买九分之一。这个我觉得,一般的外行人总是觉得我有这么多的国债在网上卖了,对美国有什么损失?其实美国政府巴不得你卖,因为现在是他不在乎自己货币贬值,他们现在担心美元升值,因为他们正要加息,他也不怕别人要卖他的国债。所以现在你要卖他的国债是最好最好的时机,刚好帮他们manage 他们的currency跟他们的debt。所以你真正要卖国债是在什么时候?是在他们有通货膨胀的时候,但是美国已经有十几年没有什么通货膨胀。而现在通货膨胀反而还在往下降,所以这一招很。对不起。所以我们内行人一看就只好摇头,这是你运气不好。



史东 01:06:02 

好,我们先把这个话题就把它摆在一边,然后我们再回过头来,你告诉我们中国应该做一些什么样的事。你刚刚提到了对于欧盟的这种糖果跟棍棒的使用,还有没有些其他的基本上的招式。

王孟源 01:06:21 

我觉得打铁还得自身硬,中国在短期上面对美国没有什么好讲的,没有什么战术好讲,就是由美国出招,我们见招还招。那对欧洲,我觉得他们不够重视这个宣传外交。还有在贸易方面的手段都不够好,层次不够高,手段不够巧,我刚刚讲过,你应该是对GM跟P\&G 动手。这时候你可以私底下跟Merkel讲,您的大众,这个VW对德国经济有多重要。我想只要是对德国稍有了解的人都知道,他们事实上是德国一个州的州立的公司,而且是德国最大的雇主,就是如果大众那个中国的利润也没有了,对德国是多大的打击了,这种事情私下要拿出来讲。可是我们也看到中国说什么这个反而是德国的媒体现在一天到晚都说中国人在这个要求技术转移。他们对高铁的技术被中国学去了,还是耿耿于怀。这个东西你必须要出来证明说,我们当初可是付了几十亿几百亿欧元给你买来的,不能够说我们是偷窃。



史东 01:07:59 

有几点我想跟您确认一下,有一篇文章是台湾的一个部落格。这个部落格的作者叫荀子。我想你可能没听过他。他这篇文章他写的主要是引起我的注意,是说美国如果想阻止中国崛起的话,基本上就是大家都知道美国的这次是阻止中国崛起的一种手段之一了。他说如果想阻止的话,已经来不及了。他其实有一段话,呼应了你刚刚讲的事情,他说中国在强大之后,也不可以掉以轻心。因为等着看好戏的国家太大太多了。我认为除了中国,要不断的强调他们自己要做一个负责任的大国之外,也要做一点大国的公关。就是完全是您刚刚讲的这个事情。他说千万不要小看公关。同样的一件事情,公关如果做得好,即使是坏事,也有人会帮你解释成好事。

王孟源 01:09:11 

对,就是这样子。这个明明是他们耍无赖,可是中方没有人出来据理力争。那美国跟欧洲做事情就是这个样子,他们先要占领舆论上面的高地,不是道德上的高地,而是舆论上面的假道德的高地。那你如果能够在这个阶段就先遏制他,就省了后面的麻烦。那这方面先放一下,我觉得最重要最长期的是中国内部的整顿,像刚刚讲的这个问题,其实是他内部的宣传跟外交人员的素质不够,管理不够严。但是另外一方面更重要的就是它的高科技的发展浪费了很多资源。比如说他们在过去这十年其实已经投入了几千亿美元来做芯片。可是大部分都是变成去买美国的专利,买美国的技术。而这些技术都是有string attached。就是三年之后就expire或者到这一代为止,下一代新技术出来,你又要重新付钱。你这样子刚好是play into their hand。就等于是这几千亿美元浪不但是浪费掉了,而且真正国内的champion也被扼杀。就是刚好帮助美商来扼杀你自己的那个襁褓中的婴儿。



史东 01:10:44 

这个事情应该是中国的这个梦,自己的梦也能够醒一醒。

王孟源 01:10:50 

对,是他们内部的管理官僚实在水准不够,放任地方官员自己去跟美国的厂商做交易。那做交易的结果是反而自己扼杀自己的襁褓工业,这是非常糟糕的一件事情,习近平必须要派专人来整顿。这个如果继续放任这样子下去的话,下面还会有类似中兴的事情



史东 01:11:18 

我们谈到这个中国可能可以出手的几个要点,其中有一个今天我们还没有提到,是中国好像现在正在审这个高通,就是高通要并购一个叫做恩智浦半导体公司

王孟源 01:11:43 

你知道这个NXP对对对,NXP是以前飞利浦半导体,就是十几年前叫做Phillip,是欧洲最大的半导体公司。中国要不要对这件事情来动手?我觉得你如果动手了,美国人也不会觉得有什么吃惊,因为这是典型的美国式式手法。是不是要以其人之道还治其人之身,这是可以考虑的。但是我觉得并没有打到痛处。你这样子虽然是高通会觉得很烦,但是他并没有真正损失什么利润嘛,对不对?那你而且名不正言不顺的,我觉得是可以做,但是不要推到极点,还是真正打造他真正的痛处比较好。因为事实上高通不是一个好的目标,因为高通掌握了上游的技术哦,还有专利什么,我不会推荐动高通。因为目前中国跟高通在同一个方面的企业公司是展讯,展讯他不在同一个制程,他是那个低端的那个工艺,那没办法做一对一的替代。所以我不认为高通是合适的对象对象,而且你说真的要怎么样,你不准他,那就说好,那我就这个可能



史东 01:13:34 

很大一部分的心理上的投射。是因为川普政府否定了新加坡的那个博通去买高通这件事情。所以说有就有心理上的这种反射。就是如果今天中中国政府正在看这个有关于高通要并购这个恩智浦的这个事情。我不懂的,就是我这有一个岔开有一句话不懂得NXP为什么它的合并必须要经过中国政府来核准。因为这两家公司在中国公司吗?

王孟源 01:14:11 

是欧洲公司,事实上中国买了恩智浦的一部分嘛,这个是



史东 01:14:18 

所以我我再把这个话题带回来。所以说很多人就认为说既然川普你否定了博通买高通,我今天就否定你高通买恩智浦。第一个博通并不是中国企业,它甚至不是新加坡企业,它是一个跨国企业。它的主要资产其实是在美国,中国没有必要为他出头了。第二个是美方对中兴的打击是致命的,出手就把他的头砍掉了。你对这个是人家要结婚,你说我不让你结婚,这个这个是可以绕得过去的。对,你知道为什么可以绕得过去。



史东 01:15:02 

我了解你的意思,我了解,所以你说要这种让对方觉得痛感到痛的这种致命性的打击,那就是你刚刚讲的一个关于在gm在通用汽车方面上面,在这个P\&G上面对不对?这种样子行为才能构成对方的

王孟源 01:15:20 

这个是一个骚扰,没办法说要他交钱嘛,你就只是拖下去而已。你要拖到什么时候,要骚扰也可以,但是我觉得不是一个主战的手段,所以不是太重要,另外还有你刚刚提到的这个VW,就是关于大众汽车的这个事情,我相信你一定还记得这个事情就是在什么时候,在几年以前的VW被美国的这个也是整的很惨嘛,就是赔了多少钱。整得很惨,而且中国对他真的是很好很客气,比对自己国产的那个汽车公司还要客气。其实你把GM跟大众在国内养的这么大,养的这么肥,现在不宰更待何时。现在欧洲站队的方向非常非常重要。你现在不把这个这个牌拿出来打,你要等到什么时候?我觉得很奇怪,就是为什么到现在还没有动手,我觉得很奇怪。



史东 01:16:23 

好,这这也是我们的未来的观察点,观察的角度。有关于这一次我们谈的事情,我们这个问题很大,所以我们谈的比较杂一点。你还有没有什么其他还想告诉我们的一些事情,有没有?

王孟源 01:16:45 

我觉得很不幸的是,三四个月前trump开始亲自掌权,准备要发动贸易战争的时候,其实全世界都在观望,看他会对谁先动手。一开始好像他不会对中国动手,后来转过来,现在已经打到头上了,这个事情已经打到头上,你只有正面跟他对回击,这个没有办法。但从战略上来看,其他的国家就只能就占了便宜了,在看好戏。所以你不能够让欧洲置身事外。我觉得这件事情是我过去这两三个月看的最不懂的一件事情。为什么中国就是一副觉得欧洲没有事的样子。但是所有对欧洲的牌都没有打,也没有去打宣传战,也没有把大众这些公司拿出来当做leverage,是非常非常奇怪的事,我真的不懂他们内部在想什么,事实上跟美国这个的斗争,有百分之八十以上的机会是到十一月之后就会缓解。实际上就是期中选举,就是我觉得民主党胜选的机会大概是百分之九十,但是即使民主党胜选,仍然会有大概百分之十的机会,他们会继续打对中国的贸易战,为什么?因为我刚刚解释过,这个美国对中国的这个疑虑是长期性、战略性、根本性的。就是他怕他的那个霸权全面的被取代了。这是一山不容二虎。



史东 01:18:40 

在西方的意识形态里面是完全不能接受的,不能坐视

王孟源 01:18:46 

完全不能接受。所以你共和党trump只不过是他不要脸,肯跳出来直接做。民主党这些建制派只不过是喜欢故作高尚,这个要做肮脏的事情,然后先甩脱一下他们的恶名。所以如果共和党已经帮他做了,不见得他们就会再



史东 01:19:18 

这也是我的看法跟你很相似,孟源,事实上今天我们节目做了大概是有九十分钟的节目,最后几分钟了,我想我想发表一下,我对这个整个事情的观察,以及我不能说我的问句了。就是我觉得这个事情是这样子,就是说中国重新成为世界强国这条路了,你可以看出来是有多么的凶险。我们这个事情没有打到我们头上之前我们是有多么的naive。所以认为中国在复苏跟复兴的这条路会有一些可能崎岖不平的地方,但是基本上都是可以跨过去的一些坎。但是我看到最近发生的事情,我看到发生在叙利亚的事情,我看到发生在中东的事情,那是多么残忍的一种事情。他就是没有理跟你讲理,然后制造出理由,我老子就是打你怎么样,对不对?你怎么样。然后中国好在是块头很大,他一下子啃不下来,所以他一定要用这这种其他的方式。



王孟源 

我觉得其实是一个短暂的挑战。但是中国应该能够克服的。我们现在看到这些问题,其实是中国内部官僚本身的弱点刚好被打痛了。我在我的部落格上说是刚好中国所有的共产党人一次都被打痛了。但是整体来说并没有动到元气。事实上美国要动到中国的元气还差得很远。这个中国我想应该也是沉得住气。现在我有些读者在问我说,是不是刚好趁机武统台湾。这种事情是由内部需要来决定时间,而不是逼急了以后心血来潮。因为习近平不是trump,不会搞这种疯狂的事情。



史东 01:21:36 

另外节目最后我想提出一点,就是我觉得身为中国人可以非常骄傲的一种精神,因为在准备今天节目的时候,我不断的想到了这个故事,然后上网去查一下。我说这个故事是真的是假的。在我查到的网上的资料,基本上是真的。中国在造原子弹的时候,有很大一部分是中国的算盘的,一个珠子一个珠子算这个计算出来。

王孟源 01:22:08 

是,我也听说过,



史东 01:22:09 

你也听说过这个事情,然后我回去查一下,的确是这样,不是全部了。但是有一部分因为那个时候中国没有那种计算机。而且其实在当时有的话,那计算机的计算能力也也也很差。我要说这个事情吧,就中国当时的那种精神,我我在我在感叹,共产党人的当时的那种精神,难怪把国民党打到台湾去



王孟源 01:22:37 

卧薪尝胆。我觉得卧薪尝胆真正一个很好的态度。因为中国晚清末年吃了那么多的罪受了那么多苦,只要能够记得住那些历史的,自然这个态度应接正确。我觉得最近这三四十年富了之后,反而是有危险,这个像台湾也是八十年代九十年代暴富之后,一下子心态就完全变了。大家就只想着要享受,这个我想一两个多月前我看到网上有一个陈平教授,他讲了一件事情,我觉得跟你讲的很有道理。他说犹太人过节都是在纪念历史上自己祖先的苦难。而我们现在过年过节都是在那边想着发财什么享受的,这样是很不好的。因为你这样必须要记着,过去的苦难,才会有光明的未来。我觉得你现在讲的这些,当初在发展原子弹的六十年代、七十年代的那些苦楚,靠的都是愿意吃苦奋发的那种精神。那这个精神都是从以前苦难的的记忆而发出来的。你如果现在一天到晚就只想着我要买iphone,我要买什么的话,那你很快的就跟西方人没有什么差别。



史东 01:24:11 

所以我的结论就是我希望我们不要忘记当初中国造原子弹的那种精神,那种一个算盘一个珠子打出来的这种计算,这是非常感动人的一件事情。

王孟源 01:24:28 

所以现在美国打压中兴,就像你刚刚讲的,其实是很好的提醒中国说我们现在这个道路还不是完全平的,还是有苦难的。可能大家不要忘记初心



史东 01:24:45 

非常好,孟源再次说声谢谢你,今天是耽误你九十分钟的时间,再次的说非常感谢

王孟源 01:24:51 

每次跟你聊天都是很大的愉快,谢谢。



史东 01:24:55 史东 

谢谢我们保持联系好不好?谢谢。好,拜拜。\twocolumn[\begin{@twocolumnfalse}
\section{中美国力、贸易战交锋、智库、群众和民主制度、产业升级、期中选举}
\subsection{20180809}
\end{@twocolumnfalse}]史东(00:30 )

各位朋友你好,我是史东。再一次的说一声,谢谢您按时的收听与收看八方论坛这个节目。今天的节目中想跟你谈谈有关于中美贸易的事情。我想这件事情一直在过去一个月以来一直可能占据的你很大的这个注意力。那么今天节目中主要是谈这个事情,因为也是大概一个月过去了,我们想和您非常欣赏喜爱的一位评论员,就是王孟源王先生来谈谈他对这个事情的一个月以来他的看法,关系方面的事情也会有一个什么样的发展,或者什么样的进展,或者什么样的结果。这是今天的节目中跟您要谈的事情。首先我们就把王先生带进我们的画面之中。孟源你好说一声谢谢,说声欢迎



王孟源(01:21 )

非常荣幸能够再上你的节目



史东(01:24 )

我想节目一开始要为我们的观众求个情,这就是你大概最近很久一段时间没有什么动静,所以我这边已经接到了不少的信息,说这个王先生最近怎么样,我们很关心。我在想,如果我这边接到了不少信息,你那边接到了一定更是不少信息。我想趁这个机会怎么样谈谈给大家报个平安吧



王孟源(01:52 )

没有什么大碍。不过基本的问题就是上了年纪嘛,上了年纪一些小问题,身体精神都不好。尤其我过去二十年一直都有失眠的这个毛病,最近这段时间失眠特别厉害。那因为我写作是呕心沥血型的,我想我也跟你提过,所以必须要有一整天专心下来,把逻辑想清楚,沉淀下来,然后每字每句字斟句酌的这样写。所以如果精神不好的话就不想写。其实常常我在闲空下来的时候,心里想可以写一篇文章,这样这样子。但是一直都没有时间,没有这个精神去把它写下来。那另外一个比较次要的原因是你知道我来自于台南的一个家族,那他们的政治观点跟我基本上是南辕北辙,那最近似乎他们都注意到我写作还有上这个节目。所以我妈妈接连打了几个电话(这样子会出现这相当程度的困扰)不是特别鼓励,就是少说几句。我想我的个性是实话实说,我觉得事实跟逻辑胜于一切。就是当然明朝的时候,于谦有一首诗,他说浑身碎骨浑不怕,要留清白在人间。我没有到那个地步了,是但是我现在我的卧房还挂着一个画轴,那个画轴上面有两个红柿子,还有一个白菜。然后上面的题字是世事清白,这是我当年我祖父的同事画给他的,那是我留下来。现在我这个房间里面,两件祖父的遗物里面的一件,算是我的座右铭之一吧。所以我的人生态度就是对就是对,错就是错,该说的就是就说,那家人不高兴,那也没有办法了。说实话哪有不得罪人的,对不对?



史东(04:33 )

我也不知道该如何劝你。因为我相信,就是你走进了这一行,就像我走进了这一行,我就要有准备,要懂得会得罪人的这种心理准备。不管我这个得罪人是正确的还是不正确的,这是英文叫come with the territory。这个就是他会进来一些状况,不能说是一些无意识的状况。



王孟源(05:13 )

我想公众人物最重要的是对自己的良心负责,其他的一切都是次要的。我了解你说的意思。



史东(05:25 )

谢谢你,谢谢你这个和我们大家解释一下你最近的情况,同时我也希望你能够尽快的恢复你的精力,跟你的这种专业的精神,大家都等着你的作品,好不好?



王孟源(05:46 )

而且另外一点是我儿子再过一年要上大学了,所以现在正在申请大学的途中。他这个暑假也是很忙的,然后美国的小孩子你也知道到了十六岁就开始学开车,然后一大堆准备要成年的事情,所以我也的确是为了家事比较忙一点。



史东(06:13 )

知道你一切都好就好了,大家就放心了(没有什么大碍)。回过头来谈谈主题,就是今天的我们要麻烦您给我们解释一下中国和美国之间的贸易之争,我不敢说贸易之战,可能目前发展大概也有一个月了吧。那么我想你也看到了不少的消息。关于现在中国说中国目前为止状况不错,美国说是美国目前到为止这个状况也不错。你的想法是什么?你的看法是什么?



王孟源(06:49 )

我想我们三四个月前谈过这个问题。对那时候我所评论的基本上没有什么要修改的,就是现在补充几点。我们那个时候,我的重点就是这个中美的贸易战争,其实是欧美这些先进国家要遏制中国崛起的一个战略决定。因为他们不能够容许他们霸占世界政治、经济、军事霸权的任何正式的挑战。所以在本质上我觉得是无可避免的,这只是时间问题。事实上他拖到2018年才真正明显化,已经算是中国的福气了。而且由 Trump 这个小丑来执行,更是不幸中的大幸,事实上欧洲跟日本都已经在去年就已经准备好要修改 WTO 的的规程,要把中国的国营企业划作是非法的。这个国际会议连时间都已经定了。但是因为 Trump 上台就被中断了。所以最近中国有一个叫胡鞍钢的教授,他说中国的国力已经超过美国。我个人不觉得,因为这个当然是有主观的观点,就是你这个所谓综合国力是怎么定义的,但是这个他的那个科技实力和综合国力,我想以一般人能够接受的定义来看的话,中国还没有到完全超过美国的地步。就是它的经济规模已经大于美国了,工业产量几乎已经到达美国的两倍。但是你如果看到这个品质,尤其是高科技方面的技术,尤其是半导体跟软体这两方面,中国还是落后的,很明显的落后,所以才会有中兴的那个事件嘛,对不对?不过这种人文社会学的评论原本就不可能完全精确,所以他稍微乐观了点,就乐观了点嘛,结果他变成众矢之的。有一大堆清华的校友,要把它开除掉,这真的是很可笑的。美国99\%的学者都是对美国自己的制度过度乐观。如果按这个标准的话,美国只剩下不到1\%的教授。所以我觉得这反映的其实不是胡鞍钢本人有什么过错,而是中国的整个知识界自信非常的不够。你中国就算综合国力以客观来说,还没有真正赶上美国。但是事实上也就是望其项背嘛。在很多方面比较低端的一些技术,已经事实上已经赶上了。还有一些比较新的方面比如说像人工智能,像是无人机,这些过去十年十五年才出现的东西,中国完全是可以跟欧美匹敌的。欧美就是过去有一百年的累积,有一些的有一些技术,像是化工、冶金这些东西,还有半导体,这些东西,是不可能说十年十五年就一下子赶上了。那我们不能自高自大,但是也不需要妄自菲薄,对不对?一个教授写一篇文章稍微乐观了就有几百个校友签名写信要他要把他开除。我觉得这非常可笑。真正可笑的是这些校友,而不是胡鞍钢教授。



史东(11:13 )

其实谈到这个事情,我也有一种感觉,当然我我先声明,我无从证明我要说的话是正确的,我只是这种感觉。而且我可以告诉大家,我为什么会有这种感觉?就是这个中国和美国的贸易争执开始之后,有很多人开始说习近平先生可能在国内面临到一个什么样的政治上的压力。然后有各种各样的声音来指责习近平的很多的不是。我为什么会有这种感觉?因为这是一个客观的事实,就是这些事情都是所谓的不约而同的发生,不约而同对我来讲就造成我的疑惑,就是从 RFA,从美国之音这边开始,同样的然后就想到这个事情,是不是?如果是真的话,我也觉得并不是十分的意外,因为这也是美国对付中国的这个手法之一。这个事情延伸到你刚才讲的胡鞍钢老师的这个事情,我是同意你的看法,但是不是可以延伸到那个地步,我就不敢说了。但是基本上我觉得美国一定有这个意向或有这个行为来鼓动一些所谓的这些反动派,来造成对习先生的压力。



王孟源(12:49 )

我对这个宣传这方面的看法是很实用性的,就是它是一种软实力里面最强的。一方面最重要的一方。所以你说综合国力,你就必须要算进对外宣传。中美之间这个以宣传来比的话,根本就是实力悬殊。所以光凭这一点,你就不能够说,中国的综合国力超过了美国。所以胡鞍钢的论点其实不是完全正确的。但是他之后是他成为众矢之的,其实反映的不是他有什么大错,而是这个综合国力的差别。就是尤其在宣传上面,我一再的说过,美国的霸权是建筑在三个角上面的。一个是它的军事实力军队;第二个是它的金融霸权,也就是美元;第三个就是宣传。所以这个他的当时在冷战,在八零年代,他有两个大的敌手,一个是日本,一个是苏联。苏联就是靠宣传把它从内部搞垮的,完全不费一分钱。然后在九零年代有无限的红利。所以我们我们在美国都知道,在九零年代那个日子是好过的,一年一年都是那个经济百分之三点五,百分之四这样子,这都是从那个苏联垮台,从东欧那边所获得来的红利。那另外一方面日本垮台,他们也搜刮了很多,但是日本的垮台是靠金融实力把它打垮的。那我在一年前写了一篇文章,那里面我说他其实现在搞来搞去,还是这两点嘛。因为他现在要搞欧盟,欧盟还不知道他们已经成了美国的目标了。不过那个他们还一心想着等 Trump 下台之后,可以跟下一任的民主党总统言归于好。但是实际上他们现在美国对欧盟的大战略是靠宣传,然后对中国而是宣传跟金融并用。



所以我们言归正传好了,这个中美贸易过去这半年多的交锋,我觉得在战术上面有些可以讨论的是。但是整个战略来说,习近平跟刘鹤这个团队基本上没有做任何的错,我这句话是怎么说?我上一次上节目的时候,我说我已经解释过了:美国之所以要打击中国,不是因为中国做错了什么,而是他根本就不容许一个人口这么大的后进国家来挑战他的霸权。这挑战霸权之后,他的政治跟经济的红利会完全消失,那么他们以前过去这一百多年所过的好日子就完蛋了。所以任何那些人说那个像是胡鞍钢这样子,因为自夸所以招惹来别人的打击,这根本连时间顺序都没有搞清楚。然后像是他们认为习近平态度过于强硬,其实是因为习近平的态度足够强硬所以才撑到现在,要不然他们会提早好几年动手。因为习近平自己站得稳,而且有一带一路这些东西,所以他们只能够用一些比较缓慢长期的。那最重要的,我以前在我的博客上面讨论过好多次,就是就是TTP跟那个TTIP,TPP是那个环太平洋的自贸协议,至于TTIP 只是跟欧盟之间的自贸协议。所以我们来看一看这个贸易战争打起来之后,习近平的反应,其实习近平是已经是比我个人的喜好还要息息事宁人的态度,那个刘鹤从一开始就是希望能够购买天然气、石油跟大豆,以几百亿美元甚至上千亿美元的订单来收买美国,这个我在上次上节目的时候,我没有讨论到这个可能性。那原因是因为像你像我这样的在美国住了二十三十年的人,知道是根本不可能的。你不管拿出多少的现金,他们的目的不是要赚钱,他们的目的是要打死你。所以那根本不是问题的重心。对,所以你不可能收买这东西,人家就是存心要置你于死地,你怎么可能拿出一些黄金来就买你自己的命?但是反过来说,我们是旁观者,我们可以说风凉话,说这个不可能。但是,他们习近平跟刘鹤是负责十三亿人未来命运的实际掌权者,他们的责任非常的重大。所以不管成功的几率多么小,只要他不是完全是零,他们就必须要尝试一下。他如果能够,而且像川普这样子疯疯癫癫的人,你要是运气真的很好,他发疯然后就说好这个红利非常的大。



你想想看,如果我们只付出一千亿美金去买一些原本我们需要的东西,中国原本就需要的东西。然后这个贸易战就结束了。那反过来,美国就必须专注在日本跟欧洲上面,这真的是最美最美的战略形式。而且也想一般人没有注意到,这个 Trump 这样的一个白痴来当权,其实是所有对手的最梦中的最理想的结局。所以才会有普京去在选举中动手,直接动手让川普上台。中国当然不会干这种事了。但是如果花几千亿美元买了大豆,买了石油,让美国的选民高高兴兴,美国的经济能够撑到下一次大选2020年,然后仍然是繁荣茂盛,让 Trump 连任的话,那对中国来说又多了四年的战略红利。所以这可以说是一举三得。一你避免了贸易战,你把贸易战的压力弄到欧洲跟日本上。然后第三个是美国的这个白痴总统可以再多做四年。因为你看现在美国的这个本季的成长率到了四点多,这个是什么?美国的其实现在我们是乘着二零一七年世界的经济有了明显的复苏。所以这个是美国受这个影响,然后再加上大幅的减税。这个减税其实是把子孙的钱拿来现在花掉。那花掉的话,那当然你就可以让成长率冲上去。那光是这个减税在今年就至少会增加百分之一的那个 GDP 年增长率。所以就是这样弄到四的。但是这种东西是饮鸩止渴,而且它不能够持久,绝对是在明年就会开始消退。消退之后,你因为这种强力刺激所造成的经济扩张,会反而有崩盘的危险。到了2020年 Trump选总统的时候,基本上我认为绝对是已经有一个 Recession going on,有一个经济衰退。那到时候川普要连任基本上是无望的。所以当初刘鹤想办法要跟 Trump做交涉其实是合理的。你看后来这个Juncker就是欧盟的主席,跑来跟 Trump搞的也就是完全是这一套嘛,对不对?他们也就是说要买你的大豆,买你的石油,买你的天然气,而且他们的规模比中国承诺的小了太多了,不到十分之一。但是美国还是愿意跟欧盟打交道,而不是跟中国打交道。这就是回到我刚刚讲过的他整个的战略目的不是要占你的便宜,而是要置你于死地。几千亿美元要收买他还是没有用的。这个刘鹤先生,我对他是很尊敬的,他的那个经济跟金融的专业知识非常的强。但是他没有在美国住过二十几年、三十年,后来他跟欧盟,他吃在美国吃了闭门羹以后,跟 Juncker 谈话的时候,又提出跟欧盟结盟。后来这也吃了闭门羹。但是这同样的我刚刚讲过,我们这种旁观者可以说这个不可能的。因为欧盟他也是嫉妒中国,也是希望置中国于死地,只不过是手段选择不同。他愿意要修改 WTO 的规则,而不是直接做那个关税壁垒。那你要跟他结盟去打美国,当然是会吃闭门羹。那个后来我在观察者网上面看到,宋鲁郑先生有一篇有一篇评论,他说千万不可以去跟欧洲提议结盟,否则徒然会惹人讪笑。



史东(24:20 )

他的立论是什么?就是欧盟不会理你,是这个意思吗?



王孟源(24:25 )

就是欧盟根本就有太强的种族主义,而且他也是在大战略上面跟美国站在同一条战线上。他也是希望能够置中国于死地。他也不希望有十三亿人成为先进国家,成为先进的经济体。所以而且以欧盟的这种文化和心理习惯,你越是提议跟他亲近,他越是会看不起你。我完全同意他讲的是对的,你在美国我不晓得住了二十年三十年,我在美国住了三十年。我完全同意他的这句话,(他说的是一种是一种优越感嘛)你如果提议跟美国人好,他把你看成奴才的。他对你就会像对日本跟韩国这样子,对日本韩国对他们基本上是附庸国家,这附庸国家只是拿来用的卒子,不是可以跟你平起平坐的这种朋友,这个中国也是一样的,中国现在因为太大了,不可能真为真正成为附庸,所以必须要被打倒。那你这个时候要说跟他成为盟友,他们绝对是不会接受的。



史东(25:56 )

你刚才说的这个饮鸩止渴,我觉得是一个非常恰当的形容,我觉得是非常恰当的。另外我的看法就是您说这个刘先生到美国来想买很多的这个油跟跟黄豆、大豆这些东西,以及和欧盟之间的尝试的接触。这会不会是一种明知不可为而为之必须要做到的一些步骤。从他们的角度来看



王孟源(26:25 )

我在前面已经提过了,他们虽然有可能是因为刘鹤先生没有在美国住过二三十年,所以他对美国人的这个民族,种族优越感跟那个文化劣根性没有足够的认识。那事实上任何一个国家的高层领导都不可能说在在外国住过二三十年嘛,对不对?一个大国的领导。所以一方面我的评论,第一这个问题不在于刘鹤,而在于中国的智库。中国的智库水准太低了,他的没有办法提出的对美国跟欧洲这种很简单的认识,宋鲁郑先生可以很简单的在观察者网上面讨论,我我们一看就知道这是常识。但是中国的智库没有一个人能够提得出来,这是很可笑的。因为领导在跟外国打交道的时候,要知道外国的文化跟心理习惯是要靠智库提供的嘛,对不对?不是领导自己知道了。那反过来说,我刚刚也提到另外一个点,就是我们是旁观者。我们可以说这个成功的几率百分之一百分之二不用算了。但是对他们来说事关重大,要是成功的话,这个红利太大,你必须要去尝试,顶多就是吃个闭门羹吧,反正也没有什么损伤。那我想一年多前,那个中国跟印度在洞朗对峙的时候,那时候我在我的博客上评论说,那时候也是因为中国的智库太差劲,对印度人的那个心理完全不了解。所以中国的政府在一开始就高调的威胁说,你如果不退,我就要把你打垮。当然中国是有这个军事实力可以打垮,但是他的战略形态是前有狼后有虎。既要担心美国、欧盟、日本跟澳洲对不对,你印度这种东西,他的那个战略的优先顺序非常非常低的。你跟他打起来干什么?



所以那时候我一开始就说实在是不应该威胁,因为你无论如何都是要跟他妥协才是为才是上策。那你一开始就把话说绝了,反而是最后要丢自己丢脸,果然就是这样子。那这一次不太一样,这一次同样是那个中国的智库,没有提供对对手的那个正确的分析。但是跟上次不一样,上一次是你已经提出要威胁了,然后必须要 Climb down the ladder,对不对?我们那个英文是要自己爬下来,那这次是我们知道来软的没有用,但是他还是尝试来软的,尝试来软的,失败了,没有什么大不了的嘛,对不对?反正反正就是吃闭门羹,这个没有什么好丢脸的。那现在美国跟欧洲能够怎么样?事实上欧洲欧盟基本上就是一盘散沙,他们说要买美国的石油、天然气跟大豆市场也买不了多少。而且这些进出口的政策不是欧盟的职责,是个别国家的职责。所以光是 Juncker 根本就是空口说白话了,而且现在只是一个停战协议而已。那所以我们经过半年的折冲,于是跟我三四个月前在上节目所讲的完全一样。就是我们必须要等到期中选举。那在期中选举之前,现在美国那个已经实现了大概五百亿,这个是第一批嘛,第一批的还分两次两个机制,然后接下去他要把他要再提高,这个时候就超过那个美国对中国全部一千三百亿一千四百亿的出口。那你对所有美国来的东西都提关税,也没办法做到,对不对?我上次提过这个 Trump 有恃无恐的,就是凭着这个。但是反过来说,因为美国不让中国在美国境内投资,美国公司在中国的资产要比资产还有营收,还有利润这个事情



史东(31:08 )

我跟你讲这个事情,当然我们现在都知道了。川普这边去证明他讲的就是说我们卖给中国的,我们卖给中国的要比中国卖给我们的要少很多。一比四。因此如果有这个贸易之间的这个争执或者贸易战争的话,我们一定赢。川普自己也讲过。而而且我相信川普的这种这种理论,这种歪理是来自他旁边的一些军师(navarro 就是这样讲)那么我我跟你讲一件很有意思的一个比喻。因为很多人都知道,如果两国真的是发生了贸易战的话,他怎么会止于贸易差额这这个部分去有动作?一定会牵涉到全局的吗?有一个加拿大的可能,如果没记错,是一个加拿大的经济学家。在一次访问中,他说我跟你我跟你打个比方,就好比说我说我今天要把你房子烧掉,你听到说那你也要我把我的房子烧掉,那我就说哈哈我赢了。因为我的房子比你的房子小,这个mentality,这种是很可笑,但是他这个也是也是一针见血,他这种比喻,不是吗?



王孟源(32:29 )

对,但是你的房子虽然小,但是你还有个工厂,人家对方那边没有的工厂,他如果愿意的话,可以把你的工厂烧了。



史东(32:37 )

这当然这是另外一个层次层面的事情了。单就面积,你说我说我占便宜,因为我的房子比你的房子小,因此我们互相烧掉对方的房子,我占便宜了。



王孟源(32:51 )

对,所以我们来看一看,这个整体来说,我觉得中国目前的战略是保持尊严,但是尽量忍让。就是不认输,不投降,但是能不升级就不升级。他们上一次最近他们又威胁要两千亿嘛这就超过完全超过美国对中国的出口。但是中国只对等了说我要再加六百亿,目前都还没有实施,这只是威胁而已。他可以说我马上就对GM通用汽车或者是apple 在中国的销售,所有的营收都增加百分之二十五的营收税。这马上这些公司就完蛋了



史东(33:50 )

我跟你讲,还有一件事情,对不起,我又打断你一下。还有一件事,到目前为止,没有人提到,至少我没有看到有人提到任何的媒体或者其他的评论。就是这个中美方双方的贸易,中国是世界的工厂,至少在那段时间是世界的工厂。由于工厂由于生产带来的这种环境,地方的污染这个价值没有被算进去。



王孟源(34:15 )

事实上中国在去年开始禁止垃圾进口,这个对欧美已经产生了很大的影响。



史东(34:25 )

我的意思就是说,你看这个双方打算盘的时候,这方面的这种我可不可以叫无形的这种价值在中国方面所承受的这种价值,这种这种负担是没有被计算进去的



王孟源(34:40 )

没有被计算进去,事实上中国本身就就在去污染的过程。所以这些高污染的工业,如果能够趁机把它丢到越南跟印度的话,其实他们是乐观其成。这其实我觉得这是长期的战略,这个从短期来看,中国目前是不想升级,反正可以等到其中选举。然后再看结果。(我一猜就是这样子)尤其你对 Apple 跟 GM 这些有投资的公司动手的话,比这个贸易加关税还要惨很多。如果是对大豆加关税,他那个大豆田还是永远都在那里。你明年要是这个关税取消了以后,他很容易再过一年就那个大豆又种回来又卖给你。但是你如果对 Apple 跟 GM 加税,尤其对 GM 加了税,然后 Apple  的那个 GM 的销售额降到零,他那个工厂就必须要关门。然后一年之后没办法再回来了,这个损失是非常非常大的。但是问题在于这些工厂 GM 还有 Apple,这些是民主党的势力。所以你现在打民主党的势力没有用。所以他们现在不升级其实有很多详细的考虑。所以我说来说去,觉得习近平跟刘鹤的处置,我没有什么可挑剔的。可能我自己来的话,基本上也就是这样子。他们其实已经是比我认为最简单直接的手段还要忍让很多了,还要还要客气忍让很多。所以中国的那个舆论里面有很多批评他们招惹来美国,这个这个根本就是反因为果,然后颠倒黑白的论调。这是在为美国的宣传战做服务,因为美国最希望的就是你从内部...,(这也是他非常擅长做的事情)事实上苏联就是这样垮的嘛。不过苏联垮过以后,现在俄国这个是完全学乖了。那中国也可以把它做一个借鉴嘛,对不对?那我想习近平过去这几年来整顿舆论,这是必要的。



我可能现在要扯得远了一点。不过我以后会写一篇文章来这么写,就是我刚刚我稍早提过说我有一些想过说要写什么文章。有一点是说中国人的素质平均素质其实是还是不如欧美的,甚至还不如台湾。我想中国的读者听众可能会很不高兴。事实上是这样子,你这个中国的人口有十三亿,所以你最顶尖的那些知识分子素质是是蛮高的。你可以提出来可以跟任何一个民族来比较。但是你不要忘了,还有十三亿里面,百分之九十几都是一般的老百姓。那而且这些老百姓比起欧美这种已经有几百年富强的经验。而且高等教育的这种文化来比,事实上你的素质比不上是很正常的。我举个例子,前一阵子有一个有一个女孩子因为被强暴所以他准备要跳楼自杀。那跳楼自杀的时候就有一大堆人来观望。第一点我个人绝对不会去观望这种事情,这个是莫名其妙,你看怎么热闹。但是在美国、在欧洲、在台湾都有可能会有一大堆人在观望,都没有关系。但是在中国观望的人在那喊着跳跳,你跳。而且这是多数,在观众中的多数。光凭这一点你就可以看出中国人的平均素质是不行的。那我的结论我提这一点,不是说我要莫名其妙的要侮辱中国人,而是我作为一个华裔的知识分子,我要强调不要自高自大,要了解到自己的民族的弱点,就是如果一般的民众的平均素质比美国的平均素质还要低,美国的人民的平均素质已经很低了,这个有百分之二十四的美国人以为这个太阳绕着地球转。真的,我不是开玩笑,这个是每年都有那个做调查的,这个过去这十几年平均是从 21\%-24\%。我这个我记得2014年的数据是24\%。那个正是因为这样子,你绝对不能够搞这种为反对而反对,不能够搞一人一票这种制度。所以我对习近平所做的过去这五年六年所做的制度,他的政策完全是支持的。中国之所以没有垮,就是因为有一个精英政治的传统,从孔孟时代一直到现在士大夫的专政,有一个强大的中央政府,这个中央政府能够有全国最优秀的人才来为人民做决断。那这时候人民的平均素质就不重要。你看看欧美跟日本台湾,现在所面临的这个问题是什么?因为你的人民的平均素质很差,然后你由他们来做决定,越是由他们来公投,结果越是越乱,让人掉下巴,对不对?如果连这些先进国家都已经富了好几代的,都还没有办法搞这种制度。那我想中国就不要搞这种制度。那些所谓的公共知识分子,一天到晚就是为了批评政府,就是觉得他们专制。专制本身只是制度的一种选择,你在客观的环境下,不同的制度有不同的好处。你如果只有两个人,一个是孔子跟一个孟子,他们要搞投票选举,我完全赞成了。一个当总统,一个当副总统。但是如果百分之九十的人连太阳绕着地球,什么这种很基本的问题都不懂的话,连经济学都没有学过的话,那你要求他们做正确,正确的决断是缘木求鱼。由精英政治是正确的,所以目前这些人他们批评那个习近平跟刘鹤,基本上不是因为他们做错了什么,是不是?你如果真正客观的分析他们所做的每一步都是最优的方案。他们所这个批评完全就是因为不满他们是所谓的专制。专制是什么?就是他们为人民做决定,可是他们本来就比人民聪明的多,而且只要他们是能够为公为理想来做决定,而不是为自己的私利。那么这反而是一个最优的制度,一个最优的情况。那这个反对的人反而就成为民族的罪人。不过我想我扯得太远了



史东(42:33 )

没关系,因为这个基本上也关系到中国和美国这一次的在贸易上的这个争执的一个很重要的基点。因为我刚刚也提到了这个我的感觉就是有关于这个忽然有一批人不约而同的在几乎在同一个时间开始说同样的事情,他的目的好像都也是一样的。这个就像我这种可能我疑心重,也是因为我是新闻人,所以我对这方面的特别敏感。我就有这种这种反应,基基本上也是也是回应了你刚才讲的这个有关于这个中国政治方面的事情。有一点,我我想请教一下,你觉得到目前为止,这一个月以来,中国有哪些在你觉得因做而未做的事情有没有?



王孟源(43:28 )

我觉得从中国的观点来看,在未来这半年再到一年,这个贸易战持续下去,他们有短期跟长期的两个重点,短期的重点是金融。因为我不认为美国的宣传战能够彻底,能够真正有效的那个影响共产党的控制。所以他们要打击中国,只能够像八零年代打击日本那样子,用金融的手段。那这个所谓的金融手段就是货币。那我们最近的一个例子,我们可以看看就是2014年他们开始制裁俄国之后,有超过一兆美元就跑出去了。结果那个俄国的卢布就一下垮了,掉了超过三分之二的价值,这个不到六个月,中国短期最大的危险就在这一点。很好很幸运的是中国的人民币跟外界没有完全流通,所以他们这个危险是小很多。(史东:可以说有一个防火墙吗?)有一个防火墙。就是事实上中国的资本家并没有比俄国的资本家更为爱国。资本家很多有机会的,那些都是资本家,有机会的话绝对是几兆几兆的往外面汇。我上个月注意到那个中共的国务院部门又多了一个小组,就是金融管理小组什么的,刘鹤当组长,你猜这是为什么?就是因为他是经济跟金融的专家,他也知道短期来看最大的危险不在于这个贸易。因为你对美国的出口就算完全没有了,影响gdp 也就是百分之一左右嘛,对不对?最大的危险就是杠杆效应。这个金融尤其货币方面的杠杆可以超过一百倍甚至上千倍。



史东(45:36 )

就是这个钱币的外流,是这个意思吧?



王孟源(45:39 )

对,而且外流不是不是外国人的资产占比中国的资产,而是中国人自己的资产,就是那个资本家是不爱国的,他们为了保护自己人的资产,百分之一百分之二贬值的危险,他们可以把百分之百的资产都跑出去。那如果每一个人都这么想的话,百分之一的危险就会变成百分之百的 GDP 都跑掉了。这个2014年、2015年那个卢布贬值,就是我所说。对所以当然刘鹤知道中国不是俄国,但是我猜是他也注意到这个危险。所以他跟习近平讨论过之后,就成立一个金融小组,由他来统筹规划,因为他是目前这个中美贸易战的主将嘛,对不对?所以他必须要从这个贸易战的观点来管理金融,不能够让不能够让低层次的。因为中共老实说,地方跟低层次的官员其实水准不怎么样,这只有在中央最高层才是真正的人才荟萃的地方。所以上个月他变成国务院新的金融小组的组长,我觉得很有意思,这是正确的步骤。那这是短期的。



长期来说中国的问题在于什么?我一再的在我的博客,过去这四年五年,我一再的说,中国必须要在外贸上面,必须要以欧洲为重点。那这时候就有很多中国的读者说,欧洲这种衰退的老人,连美国都远远比不上,那有什么好重要的,欧洲的重要就在于国际贸易规则。如果没有欧洲的参与就不能够更改。就是WTO的这个规则。所以美国不管要怎么胡扯,如果欧洲没有跟他同步协力来更改这些国际贸易规则的话,他就不能够把中国的国营企业全部定为非法。这个说的更简单一点,这个的差别就在于中国不怕改革。中国事实上是希望要改革,但是这个改革必须是出于自己的需求,根据自己的情况来做决定,由自己来定这个步骤,而不是由美国来跟你说,你这样这样的改革。因为你知道美国所给你的改革都是要致你于死命的。他那个药不是要治你的病,而是要毒死你的。所以这个跟欧洲处理好,跟欧洲的关系,这个是很重要。所以我提到这个问题是因为长期来说,中国说来说去这个贸易战的长期问题,美国人自己也提出来了,就是那个中国制造2025的计划。这个计划可以用四个字来简单的总结,就是产业升级。那产业升级里面重中之重,这它的十大工业里面的第一工业就是半导体,就是说这次吃了大亏,在中兴这个事件里面吃了大亏。这个我最近看到网络上有一个博主叫做铁流,他以前是写工业的铁路工业开始的。但是他后来在这个半导体方面有很多文章,我觉得真的写的很好。我很同意他的看法,这个看法是什么?就是中国的地方跟基层官员有很多不称职或者是甚至自私自利。我们在两年前曾经指责过那个中国的高能所,为了自己的私利,要浪费国家一千亿美元的投资。这个结果有秦皇岛地方政府也正配合。事实上那个他们要投资半导体,也是投资的钱更多,这是几千亿,一千五百亿还是两千五百亿。这个投资下去以后,钱到哪去了?地方政府都跟台湾公司,跟美国公司,跟欧洲公司,跟日本公司,甚至韩国公司引进他们的技术,引进他们的技术你其实是为人作嫁,这个细节我想大家去看铁流的文章好了。一个很典型的,就是比如说他们从 IBM 里面弄,他弄来他的CPU这个中央处理器叫做 Power CPU,事实上这个 Power CPU 已经落伍了。这个 IBM 收了一亿美金的这个版权费,把一个落伍的核心设计给你。你要是做不出来,完全就是在浪费时间跟金钱。你要是做出来的是落伍的东西,而且要升级这个权利仍然在 IBM 手里。



史东(51:29 )

这有点像台湾买美国军队嘛。



王孟源(51:33 )

对,就是要是真正做出一点市场来,这个市场未来要继续做下去。赚钱是 IBM 不是你。这个就叫做自带狗粮,你当人家的狗还要自己出钱买狗粮。中国的那个 CPU 投资就是这样子,几千亿通通被浪费掉了。那个他们有没有像是联想像是华为这种公司其实都是这样子的,他们都是租用外国的那个cpu 的设计。中国有没有自主的cpu ?有的,有两家,一个叫龙芯一个叫申威,但是连公家政府机关买CPU的时候,都没有说专门买这个。因为这些地方政府也说他们是自主可控。为什么?因为我们已经引进了,其实他这个自主可控被他们这样扭曲,就是很可笑的。你记不记得,一年前你跟我谈过那个bitcoin,bitcoin大家都已经看出来,是我当初没有讲错吧,我那时候讲说这个最重要的重点在于不在于你过去放进多少,而在于未来他会付出多少。这里也是一样的,你这个自主可控不是说过去这个技术是哪里来的,因为他们就是说你这个龙芯,你这个申威的这个cpu 也有用过外国的核心。这个差别在于他们引进外国的技术以后,已经完全变成自己所有。未来不需要付任何版权费,不受任何限制。



史东(53:17 )

所以说你说这才是一个正当的途径,是吗?



王孟源(53:20 )

不是,我的意思是说这个所谓的自主可控,可以很简单的定义成你未来需不需要付版权费,你未来需不需要得到外国公司的允许?你未来有没有可能受 Trump 的制裁。这样你如果做出这种正确的定义的话,很简单就可以看出只有两家是合乎规则的。像是华为、联想这些都是在骗国家的钱。所以我觉得中共在半导体要鼓励自己的那个产业升级,到目前为止做的非常差劲。而这才是你这个贸易战打的这么辛苦,这一年半载这样子花了多大的代价买下来的。结果你把它都浪费掉了,不但是浪费掉,你那个投资在半导体上面的国家资金产业资金,而且你把花了几千亿几千亿美元代价的贸易战所买来的这一年半载两三年的时间也浪费掉了,这才是真正中美贸易战最大的长期问题。你邀请我上这个节目是上个礼拜的事情,所以我我有一个礼拜来想这些事情。本来讲到这里已经把我要想讲的讲完了,但是昨天刚好有一个新闻出来,我觉得很有意思,中共又在国务院新成立了一个机构,你猜猜是什么?国家科技领导小组,这个组长是个李克强,副组长是谁?你说不知道是谁,刘鹤。这个科技领导小组我猜就是因为刘鹤也终于注意到这一点,所以也必须要通过中美贸易战的观点来去管理中共对这些先进技术的投资跟管理,不能够放任地方政府。比如说你那个贵州有一个小小的铜仁,他居然说要投资几十亿几百亿美金要引进所谓的超高速铁路。这很明显的就是骗钱的东西。



史东(55:46 )

这个消息我也看到,我搞不懂这个他是到底他是什么样一个道理



王孟源(55:51 )

他是说把那个火车放到一个真空管里,面对这样就没有空气阻力。那你想想看,这火车要是出了毛病,大家把窗子打开,要逃命的时候外面是什么?外面是真空。像这种东西,你能够让消费者使用吗?那根本就是骗钱的。而且中国自己也有预研的技术,他不感恩中国的研究机构,你就算要浪费钱,你至少浪费给中国的科研机构吧,他偏偏要把钱拿给那个elon musk。elon musk 还要到火星去殖民,你是不是也要投资去火星,是不是哪一个小小的县政府也说我们要直接在火星建殖民地。这种事情必须要管了。这是中共的问题,不在于太专制,或者是管的太严,而在于管的不够。他们那个地方政府,还有那个企业资本家,太自私自利了,胡说八道。那个残民以自肥的事情太多了。



史东(57:13 )

我想这个节目时间还剩下了最后几分钟,这是一个也是很重要的问题。就是你个人觉得中国和美国这一次的贸易争执将会如何的解决,包括有没有发生热战的可能?



王孟源(57:40 )

不可能。因为trump不喜欢打仗,这种打仗的事情,他交给他的将军们,他的将军们是有理性的。那习近平连在关税,扯到这个在华企业,他都不愿意升级,当然更不会升级到这个热战。我认为,十一月的期中选举是一个转折点。那个昨天Ohio的第十二选区,这个是一个铁杆的共和党选区,他们传统上是百分之十百分之十一投给共和党的优势,超过民主党。那结果昨天的特别补选里面,他只超过了百分之一,就是共和党



史东(58:41 )

好像是开票过程中发现了一些问题,是不是这个



王孟源(58:44 )

对,就差一点就被反转了,这个也就是说他的这个优势降低了百分之十。你如果再看看其他几个州过去,这一年的补选,基本上就是共和党的那个优势的投票率,获票率比传统上要降低了百分之十,而全国性都是这样子。那你如果用这个比例去算的话,在年底民主党会在众议院多得大概七十几个席位。(这么多)这么多,就是你如果有百分之十的反转的话,那目前共和党在众议院只领先三十四票。所以你可以看出来这个是很危险的。他们基本上到了十一月的时候,有很大的机会,就是超过一半的机会



史东(59:53 )

这样子就好了。如果到那个时候,就如你的projection,如你的这个判断期中选举之后翻盘,共和党和民主党之间的翻盘,对于中美贸易又是一个什么样的意义?



王孟源(01:00:07 )

我觉得的很难说,因为有很多疯子是那个player ,很多player尤其包括trump 在内,他是最主要的。所以你用理性来推测是非常困难的。但是我你既然问了,我就尽可能的猜。我觉得年底的选举大概有三分之二的可能,民主党人会获得众议院,有三分之一的可能会获得参议院。那我们假设他得到众议院了,没有得到参议院好了。那这样下来以后,川普还是总统,但是民主党人就可以真正的阻挠他的所有的法案。就是基本上他就没办法任何需要立法的东西,他都没有办法做。那另外一件事,就是这个muller investigation ,目前已经开始真正的发酵出来了。到了年底的时候,民主党人控制众议院之后,我想因为在这个比较大的良好的环境下,会开始直接攻击trump,这个时候trump应该就不利继续升级这些关税。这并不代表民主党人对中国友好。其实民主党人对中国的态度(这一点很重要。我觉得这个观点很重要),但是它的差别在于党争胜于一切,这个就是美国制度了。英国政治制度的特征,党争胜于一切,这个即使是他们支持的政策,只要是trump出来的,他一定会唱反调。所以这时候中国就有机会至少不会再继续升级,那也就打击像apple 或者是gm,或者是P\&G 这些公司在中国的生意。这种事情只需要暗地里私下威胁,然后再让他们转递给他们的民主党的代言人。这件事情慢慢的就会大事化小,小事化了。因为他毕竟不能够每年都一百二十亿美元的替农民做补助,这个说不过去了。



史东(01:02:56 )

那么在这个在这个假想的情况之下,那就代表着美国想要打垮中国的这个原意就就失败了嘛。



王孟源(01:03:08 )

就主要是因为美国的政府会完全瘫痪。瘫痪之后,一直到2020年的总统大选之前,中国又得到两年喘息的机会。这个关税可能就大家你退一步,我退一步,然后就放下来了。那同时民主党在二零二一年很有可能会夺回总统的席位,到时候一定会重新提一定会加入TTP,然后重新提TTIP。(那就是另外一场,另外一场)到现在还有三年嘛。你这三年如果能够好好的搞半导体的话,你这个有很多事情可以完成。而且在谈自由贸易协定方面,你这个中国的那个协议叫什么RCEP的样子。你加把劲儿,这个未来三年把它谈成不是一件难事嘛,对不对?所以我觉得有超过一半的可能是明年到这个时候,贸易战已经是昨日黄花,大家不太理会了,美国的这个内斗变成他们的主轴,又变成未来两年的内政主轴。然后同时中国有机会把那个RCEP好好的搞起来。这个日本搞的那个TPP基本上就是空的位置。等下一任民主党总统来加入,但是那至少是三年之后的事情。你这期间中国在外面搞与他竞争的自由贸易协定是时间是很多的,那千万不要松懈掉了。所以我的几点就是,短期来说要注意你的货币,好好的管理你的货币。就是刘鹤已经当了组长了,应该没有什么问题。长期来说,内政要好好的搞你的半导体发展,千万不要再让这些自私自利的地方政府跟那个企业,那个资本家把国家的命运拿出去卖掉,



史东(01:05:37 )

这个钱我觉得浪费钱,我觉得还是一个小事,浪费时间就是更要命的事儿。



王孟源(01:05:45 )

那个时间跟时机是国家花了多大的代价弄来的。就这样子把它浪费掉了,真的是国家的罪人。第三点就是这个自由贸易好好的去谈。这个尤其是东南亚的这个协议,还可以好好的谈。那么在军事方面,这个习近平是专家,他自己是专家,所以他整顿的很好,这没有什么话好讲。我想我这个周末大概会写一篇评论,他们陆军装甲部队编制的改进的文章。我时间也不太不太够,我自己精力不好,时间也不太够,所以可能不会再到我的博客上面一个一个读者的问题我回答。但是写文章倒还可以,偶尔发个一整



史东(01:06:46 )

不管怎么样,这是我的经验,这是我的经验。不管怎么先把身体弄好,其他的都是其次,好不好?我们今天就谈到这儿,非常谢谢你的时间,我们下回见。\twocolumn[\begin{@twocolumnfalse}
\section{华为}
\subsection{20190129}
\end{@twocolumnfalse}]

Credit: arbitrage



史东(00:15)

各位朋友你好,我是史东。最近发生的连续的几件事情,其中有一件事情我想一定是让你非常的揪心。什么事情?就是关于华为的事情。我们在今天节目中想跟你探讨的还有这个事情。今天我们节目的名称定为从华为案来认识美国,我觉得这是一个很有趣的角度。在今天节目之中的时候,非常高兴为您请到的您非常喜欢的一位评论员,他是一位非常有名的部落格的格主,那就是王孟源王 先生来和我们谈谈这个事情。事实上王孟源王先生他,我想在一个礼拜以前吧,还是不到一个礼拜以前,他有一篇部落格的文章出来,叫做域外管辖权。他从那个角度来看这个华为案。那么我现在首先借用了您以及王先生的几分钟时间,把今天我们的主题稍微为您介绍一下。我首先为您介绍的是,这是在这个大陆网上的一则消息。他是说北京时间的一月二十九号的凌晨,美国的司法部宣布了对华为公司,对有关的子公司以及他的副董事长,以及他的首席财务官叫做孟晚舟女士的指控,并且声称的即将向加拿大提出对孟晚舟女士的引渡请求。对这件事情,华为公司表示他说,对美国政府提出的指控感到非常的失望。华为公司他回应他说。他说孟女士被捕之后,华为试图与司法部就是美国司法部纽约东区的调查进行讨论,但是被拒绝,而且没有给出任何的理由。华盛顿西区法院关于华为的商业秘密案件的相关的民事诉讼早已经和解。和解前的西雅图陪审团也对商业秘密的相关诉请做出了没有赔偿的这个决定。另外,华为公司他又否认了关于华为公司以及他的子公司,还有他的附属机构犯有起诉之中所指出的违反美国法律的各项罪名。同时华为公司他也说,他说不知道孟女士在行为上有任何的不当。那么华为相信美国的法院,相信最终美国的法院会得出相同的结论。这件事情在中国的环球时报,他是这么说的,他讽刺说全世界都能看懂,美国在假借法律在搞政治的鬼,华盛顿其实心虚的很。他说美国司法就起诉华为举行的记者会,居然还请来了商务部的部长,联邦调查局的局长,还有国土安全部的部长等多名高官出席,并且助阵。他说,不做亏心事,何须搞这么大的阵仗来对付一家外国的企业,环球他的视频指出,他从美国司法部正式提出刑事诉讼,是在其全球范围内打压华为行动的一部分。然后我们就带到今天,我们在节目中为您请到的贵宾,我们把它请出来王先生,非常谢谢,非常欢迎。

王孟源(03:27)

史先生,非常荣幸能够再上你的节目

史东(03:34)

--在带入主题之前,我听说你有两件事情要宣布,能不能够趁这个机会向大家宣布一下。

王孟源(03:42)

好的,第一件事是我在blog 上也宣布了,我儿子现在已经完成申请大学,在今年夏天要进大学,他已经不需要我照顾了。那我的爸爸妈妈身体不好,年纪也大了。我觉得在他们最后几年的生命里面,我有责任在他们身边。那尤其我现在并没有上班,所以我准备在暑假之后搬回台湾去,那可能需要找工作。如果有读者或者观众认为可以帮忙的话,非常欢迎到我的部落格,然后留下你的联络方法。我的想法是我可以做一些顾问或者是智囊,或者是adjunct professor 这样的工作,当然最最理想的是能够在从我妈妈住的地方,我跟她住在一起,这样我方便开车,送她到医院里面去看我爸爸,我妈妈的视力非常的不好。所以我觉得我应该在她身边,这是第一件事,其实也就是作为一个人子的基本责任了。我想中国人嘛传统都是这样子的。如果如果是已经退休了,没有理由继续留在美国。我想大部分人都会想要这样做。我第二件事情是比较轻松一点,大概几个月前我上你的节目,我们访问的时候,我提到我们全家都是深绿,除了我个人在内,那没想到我弟弟也是你节目的观众之一。他看完以后就打电话给我,我一辈子从来没有收到他打过来的电话。第一次,然后打过来,是抱怨说我侮辱了他的智商,因为他说他从二零零六年陈水扁贪腐案爆发之后,就没有再投过民进党的票了,没有那么笨,所以我在这里也特别道歉,我们全家除了我是认为我是超脱蓝绿之外,我弟弟也早已不是绿营的支持者。

史东(06:09)

好好,就是应该这件事情可以去传为美谈。你弟弟现在在台湾还是在哪里?

王孟源(06:15)

他在台湾云林,他虽然有哈佛的教育硕士学位。但是他自愿到穷乡僻壤去教人家没有人愿意去教的国中,这个情怀我是很佩服。

史东(06:30)

这个我也很佩服,我一直有这种,谈到这个事情,有点谈到私人的领域了。我一直有这种浪漫的幻想,希望我能够做到。但是现实把我又拉回来,每次想到这个又把我拉回来。所以我听到这个消息,我特别佩服你弟弟、

王孟源(06:52)

钟鼎山林,各有天性,我弟弟是真的是以他的学生为第一。所以他一个礼拜只有周日才回去看我妈妈一次,其他的时间他都是完全奉献给他的学生。(因为他的距离跟他的工作的关系,是这个意思)是这个意思。所以我想我回去的话,我妈妈天天都有人照顾是比较合适的。你要回去的话,能不能透露你希望在哪一个城市找到工作是你的第一优先。我妈妈住在新营,那我爸爸现在在台南被看养,因为他是植物人的状态,所以我基本上能够开车去的地方还不少,从嘉义以南一直到高雄其实都可以。就是南台湾那一带了,台中就稍微远了一点,不过也可以的。如果真的需要的话,我就在那边租个公寓嘛,然后一个礼拜回去三四次这样子。

史东(08:00)

好的,这个事情我想透过您的部落格,以及透过我们的节目和网站,我希望能够很快的带给你一些你想得到的好的消息

王孟源(08:14)

已经有些人自愿要介绍。不过还没有什么确定的结果就是了。所以我还是希望能够广播一下,谢谢你

史东(08:25)

很好很好,这个事情凡事都有一个起头嘛,对不对?只要抱着一个乐观的心理,总会有一个非常好的结果,在这祝福你

王孟源(08:34)

是,我在部落格上写了四年半,跟几千个读者算是神交了,那有很多朋友我自己都不知道我有的,所以也是一个很大的荣幸。

史东(08:48)

对对,这也是一种福气嘛,我觉得对不对?是人心中的一种福气。然后我也为我自己请教一下,或者请问一下,也是为我的观众或者您的读者去。您的搬家从美国搬到台湾去,并不会影响你写部落格以及影响你出现在我们的节目之中,对不对?

王孟源(09:11)

不会的不会的。好,我想台湾的internet service 还不错吧。

史东(09:16)

好,那这样的话我们大家就放心了。我想要这个话题带回到今天节目的主题。我定的这个主题叫做从华为案来认识美国。我为什么这个主题这么定?主要的和你的这篇文章有很大的关系。因为在你的这个部落格叫域外管辖权。这个博客你谈到很多这个美国的不同势力的运作,我觉得这是一个很好的教育我们的一个机会。从美国对华为案的反应以及他处理的状况,我们可以反观出这个国家的某一个层面,或者我们以前不知道的一个层面的一些内容。

王孟源(10:01)

其实我一直有点疑心,不过我的个性是光是有疑心不够。就是即使逻辑上很合理,如果没有证据的话,是不能够讲出来的。有一分证据说一分话,这是我个人的态度。我们去年讨论中兴那个案子的时候,我不晓得大家记不记得,当时中兴那个案子其实是好几年前的,然后他已经,已经认罪了,而且已经罚了好几亿了。然后,事后又在爆发是因为一件小事,就是他们没有真正扣薪水,他们的协议认罪协议里面同意要把几个相关人员的薪水扣掉。那当时没有扣薪水,没有扣薪水,这件事情又被报到司法部,是由中心的律师报上去的,ok。报上去以后,司法部立刻就下了杀手。那我记得那时候是要了八亿多还是九亿吧。那时候我就有点疑心,因为这个做法不像是在正当执法,这个像是反而是我们平常,我不晓得你自己有没有这个经验。但是我曾经在德州住过两年,后来搬到这边以后也有机会到纽约州那种乡下去开车经过,你开车经过那种比较乡下的地方的时候,会有所谓的speed trap,就是超速陷阱。这个超速陷阱是什么意思?就是他有一个小镇,那边基本没有什么钱也没有什么人。但是他有一条高速公路经过,那么那个警长就会专门在那段路上开超速的票。我那时候在德州记得有看到说有一个小镇原本连一个警官都养不活。那后来换了一个警长上来之后,两年之内变成有十五个警官,而且负担了整个镇三分之二的经费。就是他那个开罚单赚的钱比整个税收还多一倍,像这样的做法,其实在美国是司空见惯。美国纽约州上周那里也有一段所谓的parkway,就是州来经营的高速公路,不是州际的,就只是在纽约州里面的。那个parkway它跟那个interstate就是州际的正式的高速公路是平行,它的那个时速限制是五十五英里。但是跟它平行的那一条是时速限制是五十五或六十五,基本上是一样的。但是我就注意到,基本上那条路上没有人去开车。但是那个interstate上面是人山人海,常常挤得不得了。我一开始不太懂,为什么大家不去开这段,后来我在那边拿了超速罚单以后,才知道那一段路是恶名昭彰的speed trap超速陷阱,就是他你即使开五十六mile,他也会他也会开一张票给你。那一张票是三百五十块钱。他就是靠这个来支持镇的财政。所以这种事情在美国其实是司空见惯,我以前在我的部落格上也写过一篇文章说过,只要法律允许的,他们这些司法人员就会想办法来自肥。自肥的时候,当然是要看有没有力量能够限制他们。那个像这种超速的话,基本没有任何限制的力量,因为你是完全合法的嘛,对不对?所以唯一能够,能够影响他的就是大家不去开那条路。

史东(13:59)

我想到老中有一句话叫做靠山吃山,靠水吃水,这个就是靠高速公路吃高速公路。

王孟源(14:06)

对,我在三四年前在部落格上写了一篇文章讨论,美国后来有了一个法条,说容许警察局扣押车里收到的现金跟财产,为什么?因为他们认为只有毒品犯才会有一大笔现金在车上

史东(14:30)

对不起,你再讲一遍,你刚刚说什么警察有权扣押

王孟源(14:36)

对,你们大概都不知道。美国现在的法律是说地方警察如果把你停车检查,即使你没有犯任何的法,但是他如果发现你车上有什么有钱的东西,他们可以扣押,(不只是现钞,有任何值钱的东西都可以)有任何值钱的东西,他们都可以扣押。扣押之后,你当然是可以到他法庭申诉。他的意思就是说,反正毒贩的钱你先扣下来,那他敢申诉的话,我们再来处理。所以,这跟一般人认为美国是个法治国家的想法完全不同,因为这完全跟保障财产是相反的。可是问题就在于,你如果是当地的住民的话,他不敢这样做。他如果知道你是当地的住民的话,你可以投票选警长嘛,对不对?所以通常是倒霉的呢。除了是那个真正犯法的人之外,倒霉的通常就是外地人。那个外地人我看到有很多例子,比如说有些人他刚好有卖了什么东西,卖了一辆车,然后身上有两万块现金,两万现金被扣了。但是他住在几百英里外,他的这个制度是你要到当地,那个镇的地方法院出席五六次才能够把钱要回来。要回来之前,还有保管费,保管费可以高达三分之一,你们大概都不知道,这个都是网上可以查得到的东西。但是美国的体制之黑之污烂之腐败,不是一般人了解的。因为他这个,他这个腐败是欺负弱势群体,这些就是你外来人,你没有本地的那个关系的话,他就可以欺负你。你是中产阶级,平常在像我住的这个镇,这个警察客气的要死,连我的儿子都说,他根本就不怕镇里的警察只有开出镇的时候才会紧张,ok。因为这个镇的警察是拿我们的薪水嘛,对不对?你有出了什么事情去闹一下那个镇长会处理的。但是你到了别的镇,或者是州警的话,他就根本不理你这套,你就要小心他拔枪打你。这个,美国的制度就是这样子,那个没有在美国住过的人,真的是没有办法想象的。

史东(17:16)

其实我顺便提一下,据我了解,你住的区域还是高级区。(算是非常高级的)所以这个这个事情如果如果是这样子的话,那在特别在南方,在比较贫穷的地区,那那个事情是怎么样搞?那真的是没有人知道,

王孟源(17:32)

没有人知道,因为他们也没有管道可以抱怨,对不对?我们这边出了什么事情,说不定我有朋友或者朋友的朋友在哪一个电视台,或者是在哪一个报社,我可以讲。那些南方人都是穷人嘛,他们他们的莫名其妙被扣了两万块,是他们半年的收入。那就不见了。但是你网络上可以找得到这些消息,但是一般人不会去看,那主流媒体也不会去报道。所以大家就不知道,那这一次这个华为的事情,我一直有疑心,认为这个很可能就像这个超速陷阱这样子。但是一直到上周,我看到这上一期的就是这本经济学人,就是这一期,里面有两篇专栏报道讲到他们用FCPA就是反海外腐败法,怎么样来整欧洲的企业,这个让我忽然想起来,我在瑞银的瑞士信贷做事的时候,就听到我的上级那些高管开玩笑说他们要来美国要小心。那后来我退休之后也注意到他们被罚了几次都是几亿,尤其是德银deutsche bank被罚的最长,几乎每年都有一个新的花样被罚了。那德银在我们这些做金融的人都知道是管理最松散的一家世界大型银行,他的那个内部管理最松散,基本上就是那些德国人很信任他们的手下。那这些手下是美国人英国人,他们根本就不在乎,都是在底下为了赚自己的红利跟胡搞。我认识好几个,不好意思的,所以想一想,然后你再想想这个除了银行之外,还有好几个被罚。那然后经济学人那两篇文章里面讲到阿尔斯通,阿尔斯通那个这可是一个法国的pride,法国的骄傲,一个法国留下来最后的重工企业,真的是高科技重工企业。中国要是有这样的燃气轮机的制造技术,都是大概都会去偷笑了。但是他就莫名其妙的被GE并吞,并吞以后也是头几年我也不知道是为什么。但是看了那两篇文章以后,我恍然大悟,我这几年来一直心里的疑心就被验证了,这很明显的是,司法部已经形成了一个产业链,这个产业链是干什么?专门去敲诈外国公司,就是找一个名目,找一个借口。那这个敲诈,我又回想到过去三十年在美国看到的一些他们处理美国的白领犯罪的事情。我突然想起来,这其实是渐渐演进的。因为你一开始的时候,在九零年代、八零年代九零年代如果有白领犯罪的话,这些高管还有机会被关起来,其实那是最有效的处理方法,但是到了九零年代clinton 之后,就越来越少看到这种刑事处罚。通常出了事情就是由公司认罚。那认罚的话,当然是罚款要越来越多了,对不对?你这个CEO问他说,你要选进牢去关十年,还是你要付一亿,当然那个随便哪一个CEO都会说我付你一亿,反正是股东的钱,不是我的钱。对,就是因为这么容易,所以那个罚金很快就在九零年代这样上去了,上去到两千年之后,就是好几亿都是司空见惯。我这我就一想,几亿的罚金,这就代表他们在雇律师的时候,还有请顾问的时候,也会同样愿意出相当数量级的钱。而且我也记得我们的公司,那个时候我还在瑞联银跟那个瑞士信贷的时候,那个时候公司有要教育我们几件事情,一个是是compliance,另外一个是专利。就是在十五、二十年前的时候,这个他的这个美国的法规开始越来越广泛推广,那这个我们一个外国公司在美国做生意,要注意到的有两点。第一点是专利法忽然变得非常广义了。所以公司说叫我们要去查专利,有什么莫名其妙的改变、改动就要申请专利,因为那个专利腐烂到一个地步,就是根本没有人仔细去看你的专利有没有什么内涵了,他们这个专利法要讨论的时候,就是两边的律师到会议桌,然后把那个卷宗往桌上一摆,看谁的卷宗高,谁的底气就足,对不对?那你这个卷宗里面的专利是滥竽充数的,根本没有人在乎,没有人有这个时间,哪有那个专门技术去讨论。所以我们所受的训练,第一个是要尽量发专利,第二个就是一些小事也必须小心,就是像洗钱。我们在银行里面特别小心洗钱这种事情。然后这就让我想起来我以前的几个那个真正的高管,就是我们从瑞士来的高管那边开玩笑说要到美国要小心了。他们并不是很喜欢来美国出差,这是我就忽然恍然大悟了,这个他们美国人司法部去罚那些达到几亿的话,那么他们那些退休人员,还有像周边的律师事务所,一定也是几千万几亿的赚。这样赚了以后,他们一定会形成一个产业链,就是,跟那个固定的法院法官,还有那个司法部的执行人员都有默契,都会产生默契。那为什么呢?因为司法部里面管这个管这方面的一定只是一个很小的部门,他们的薪水一年多少?还不到十万美金。你如果退休之后,可以去当个顾问的话,一年几百万几百万的赚,这个是绝对你人人都会认的,那很快的就整个形成一个默契,他也没有任何写下来的规则。就是大家都知道我们这个是为了帮哪一家律师事务所赚钱,A事务所负责去假装是为这家外国公司辩护,然后B事务所假装为他的其他的第三者代表。反正大家有利益均沾。然后那个美国的这种法院,通常是专门的法院,不是说你随便指定一个法院的。就是比如说你专利法的话,他们有专利的法庭,就是少数几个法官专门管这件事情,那参与诉讼的律师,还有司法部的官员也可以还可以再仔细的选择哪一个法官,在少数几个法院的选择里面选择哪一个法官。像这种非常容易形成一个闭环的close circuit 产业链。

史东(25:43)

对不起,我了解一下你刚才讲的事情,就是提起诉讼的基本上就是司法部门,提起诉讼的可以选择他们被哪一个法官来审这个案子

王孟源(25:59)

因为他们就是检察官嘛,他们选择这个案子提到哪里去嘛

史东(26:03)

这是不是就是外神通内鬼吗?这不是上下其手吗?

王孟源(26:09)

对呀,那个法官他的他的表亲就是这里面专门的律师事务所的所长,我这是假设了,然后他的助手也全都是跟那个律师事务所专门这一行的事务所的律师这样轮替的旋转门的。这样子,每个人每年几百万,那个律师事务所长,每年几千万几亿的赚。像这样的利益链都比他们的薪水高十倍,你说他们怎么有可能拒绝?这个只要一两年就会形成一个产业链。这种事情我看的太多了。比如说美国的这个insider trading 内幕消息,这个专门做内幕消息的这个hedge fund 这个对冲基金,最大的这个就在Connecticut距离我这边还不到十英里。那每一个行内的人都知道,他就是这里面的专家。然后他一年赚大概是六十亿美元大家都知道,但是SEC不会去查它,为什么?SEC的那个律师薪水很少,你说四万块,五万块在DC那里,在纽约住,根本就是没办法还清他的那个法学院的贷款。对不对,你要还清贷款必须要做了四五年之后有了人脉,退休下来就是专门到律师事务所去为这种对冲基金服务,他们怎么可能去找他们麻烦,对不对?然后这还是二十年前,后来大概二十年前,他开始建立了一个完整的产业链,就是有中间人开始做consulting firm。这个顾问公司他就到那个硅谷那边,反正你是一个低级的或中级的产品经理,你有什么内幕消息?比如说apple 的下一季卖的不好,但是我是一个低级经理。我知道我们在砍我们的采购量,那么砍采购量就代表我们的销路不好嘛,对不对?这个消息你就可以去到那个顾问公司,一次卖大概是卖十万到五十万不等。同样的也是比你的年薪还要高,这种人怎么不做白不做,因为你这个卖给那个顾问公司,顾问公司抽百分之百卖给下一个顾问公司。这个下一个顾问公司最后卖给这个对冲基金,对冲基金一次可以赚一亿,对不对?那最后大家利润均沾,然后平常自然没有人会去找他们,因为他们这个钱都分给那个律师跟法官去了,早就分的好好的。五年前我还写了那个部落格,也专门写了一篇文章报道这件事情。五年前当时纽约的联邦检查处有了一个新的检察官,是真的想要办这个,他就真的去办这一个案子。结果他起诉了几十个那个交易员,就是没办法把最大的那个对冲基金的老板定罪。为什么?因为他的白手套太多了。等到这个检察官追的飞得很近的时候,忽然最高法院裁定说,你这个定罪必须从泄露内幕消息的到第一个顾问,到第二个顾问,到第三个顾问,到最后卖给这个对冲基金,环环都定罪,才能够定到最后一环。你说这么一定下来,还有什么可以玩的,这个检察官根本不可能定到这个对冲基金,他也就放弃了。放弃以后,他到那个trump上台之后,他就被开除了嘛。因为那个trump 耳朵软,倒不是说trump 跟他有什么仇,而是trump 耳朵软。所以这些对冲基金的人老是想要把它整下去,等到trump上来以后他们就有办法把它整下去了。所以像这些事情,我是早有疑心了。但是你真的必须要有证据,有没有证据的话,我不敢公开讲。这个证据是什么?就是经济学人上详细报道,为什么?我再复述一下,经济学人的文章里面有,我的那个部落格的文章也摘要讲了一下。二零一二年那个Alstom的主业有两个,一个是卖燃气轮机,这个就是用天然气来发电

史东(30:56)

你说的就是那个法国的国宝公司

王孟源(31:02)

国宝公司,对。燃气轮机这个真的是高科技,它这个中国做不出来了。我现在说点题外话,世界上最先进的燃气轮机,每十年一代这个一每一代都换一个英文字。现在是最新的是到H级了。十年前最先进的是G级,再十年前是F几,中国连F级都做不出来。那个上个月我才看到他那个一个大新闻,说做出来一个F级的部件,这还大肆宣传,说是大突破,这个就是这个科技有多么困难。燃气轮机这个东西,一个燃气轮机几百吨,这是世界上最先进。。。

史东(31:46)

对不起,我问一句外行话,这个燃气轮机是在什么地方被使用,用在什么地方?

王孟源(31:53)

就是发电厂。发电厂现在台湾不是缺电吗?缺电就要建天然气发电厂吗?天然气发电厂的核心就是燃气轮机,所谓的燃气就是烧天然气的意思。燃气轮机比蒸汽轮机还要先进,还要困难,ok。 蒸汽轮机是煤电厂用的,烧煤电厂用的。那个煤的话,因为它没办法直接把那个煤一块一块的丢到轮机里面去。所以你是先烧热水,热水烧成蒸汽以后再用蒸汽来推动这个。这个蒸汽轮机也是高科技,但是比燃气轮机还差一级。那世界上燃气轮机最先进的有两家公司,就是美国的GE,还有德国的西门子,这个Alstom算是第三了,不过算是小三,他这个是比前两大要小很多,但是技术还算是相当先进的,比那个中国要先进多了。那二零一二年的时候,他在印尼要卖一个燃气轮机的时候,他有两个生意,Alstom这种跟Siemens一样,跟GE也一样,它都是有燃气轮机的生意,然后也有铁路的生意。这个都是一百年前传统的重工业一直流传到现在,这个底子非常深厚。燃气轮机是至少占他一半的,那他这个燃气轮机的生意在二零一二年到印尼去卖,付了回扣。你也知道印尼嘛,印尼这种地方你不付回扣,根本就没办法做生意,是吧?事实上后来查到,付回扣的地方还有台湾,我觉得台湾有很多绿营的人说台湾政治文化比中国高明。Alstom在中国也卖了很多燃气轮机,但是他没有在中国付回扣,在台湾只卖了一两个,就必须要付回扣。那我请问你哪一个政治体制比较先进?不过美国司法部去查,他这个导火线不是台湾,导火线是印尼。

史东(34:05)

对不起,美国司法部凭什么去查,就是你说的这个FCPA这个法条。

王孟源(34:11)

对,原本这个法条是在一九七零年代订立的,他的原本的意义,原本的主旨是为了要惩罚美国的公司在国外行贿,这个用意是好的。但是,因为这个法条不可能写的很详细嘛,对不对?他这个只说他里面并没有特别说只有美国的公司,他的意思。因为这种事情照理说你没有域外管辖权嘛,域外管辖权这种东西是美国人最近这十几年为了为了中饱私囊搞出来的。所以当初立法的时候没有想到这些细节,所以你这个法条直接看上去好像是只要能够扯得上就可以。那他这个扯上是什么关系?因为阿尔斯通在美国有分公司,然后他在那个印尼付钱的时候是付的是美金嘛。你行贿用美金,我们美国就管得到。你这笔钱有可能用的是跟美国银行借的,我们就管得到。

史东(35:15)

即使你不是美国公司,我也可以管得到

王孟源(35:17)

不是美国公司。但是他在美国有分部,对不对?

史东(35:24)

你意思是说这是一个法律的漏洞。这个法律的漏洞就是来自这个FCPA这个法律,就是你所谓的域外管辖权的这个法条。当初你讲了,当初在定这个法条的时候并没有特殊的恶意,而是在执行上的时候被人找出了这些漏洞,是这个意思吗?

王孟源(35:45)

 你也知道美国不是大陆法系,他的那个法律的条文原版就是写的很笼统松散,是因人施法的。这个给检察官的那个空间非常的大。那现在这个检察官原本就想要中饱私囊,你想他是往哪一个极端去推?这个美国的域外管辖权就是这样子搞出来的。对中兴,对华为,他们用的是对外制裁,这个也是最近十几年才会去追这些外国公司嘛,对不对?原本对外制裁,美国对外制裁从二战之后就搞了七十几年了,头几十年有这样的吗?没有嘛,也是最近这十几年才开始的,那为什么?这个产业链形成了。这个产业链形成之后就自然会去找生意。而他们有极大的自由裁量权,对不对?你说哎你这个不合理呀。我虽然Alstom在美国有分部,但是这个行贿的不是美国分部,是我们美国总部,这是分开的一个公司。你说你要辩,是不是来我们美国法院辩,请你的CEO到美国来。那在事情没扯清楚之前,请你先到重刑犯的牢里面坐一坐。你去看那个这个详细的故事。为什么?美国人发现Alstom有点行贿之后,马上就进入high gear,就开始准备要绑架阿尔斯通高管。

史东(37:19)

这个现象就和今天我们讲的这个华为的现象就很相似了

王孟源(37:25)

一模一样。我那时候一看,就这下子还有什么可以怀疑的。我们一开始以为他们针对华为是因为跟中国斗争,但是这个不太合理嘛。因为跟中国斗争的话,这个主要是trump跟人家斗争的时候,第一个不搞军事,第二个不搞法律对不对?他搞的是贸易,他搞的是命令,所以不太对劲。那你一看那个他们是怎么搞阿尔斯通的,就跟这次一模一样,而且手段之毒辣,有过之而无不及。法国还算是他们的盟友。如果都能够这样子的话,那你就可以知道,这其实一开始并不是因为他们什么爱国了。当然这个爱国是政治正确,所以他们知道他们不能够去搞国内的企业嘛,搞国内的企业,他们哪一个大企业口袋里面没有几个国会议员,对不对?你们这几个搞的是那个中级的法官,还有那个司法部的中级官员,他们斗不过国会议员,国会议员真的狠起来的话,就叫你们把他全部开除了。

史东(38:41)

我想到一件事情,趁这个机会提出来,照这么说的话,当初trump说他在逮捕有华为的CFO的这件事情,他事先不知情

王孟源(38:56)

我觉得是完全合理的,我觉得他在说实话,这个,他们可能是到逮捕之前一天才照会trump的团队。那trump的幕僚也没有觉得说是什么大不了的事儿。大不了事情是司法部内部的事情,他们也不太了解,因为他们是管的是共和党斗争,而律师是民主党的人嘛,对不对?而且这些不是真正的大律师团。就是说虽然他们在过去几年几千万几亿的赚,我想他们都是低着头赚钱,没有真正有名。我在这个美国商业界待了十几二十年了。我的感觉是当然有那种你们在加州那些人,他们是拼命要提高自己知名度,像是elon musk这种人,基本上都是吹牛、圈钱的。但是也有另外一种,就是他们找到自己的一个niche一个利基之后,闷声大发财很多的。其实我自己也是,我自己以前是做program trading,high-frequency trading, 我是世界上第一个做出全自动交易程序的人,而且独霸那个市场独霸了两年。但是我从来没有上过报。事实上我们最怕曝光,因为我们赚钱,第一不需要客户,第二不需要投资人。那你为什么要(违反)这个财不露白的原则了嘛,对不对?你为什么要去广告?对不对?你出去广告第一个就是要骗客户,第二个就是要骗投资人。如果这两个你都不需要,绝对都是财不露白,就是闷声大发财。真正赚钱的这些人都是闷声大发财,不是你在纽约时报上看到的那些有名的律师。

史东(40:57)

对不起,我要插一句话,因为你刚刚讲到,您刚刚在这个这个从事金融工作的时候,这个成就是不是很多你的朋友都知道。因为如果不知道的话,觉得这是一个很容易吸引人来拜访你的一个项目,对不对?

王孟源(41:20)

我其实在我的部落格也只是很简单的提一提,这有两个原因。第一个是知道的人都是我行业的人。那我现在跟这些行业的人很少有交往的。第二个是我其实我觉得也没什么光彩的。因为这种东西,金融跟律师业一样,我都觉得都是寄生虫,所以我当初是一个大号的寄生虫,我觉得这种事情不好意思。如果提起来我也不会否认。但是出去出去吹嘘没有什么意义嘛。因为这个当初是为了赚钱,所以我惮精竭智为我的银行赚钱,对不对?我没有骗人,我没有像那个,所以我那时候跟跟圈内的朋友聊起来。我说我觉得我们是bed bug,为什么是bed bug?就是那个床上吸血虫,就是我们是靠吸血为生,但是至少我们不传播传染病,我们不影响经济,就是基本上我们这个所谓的program trading 就是用程序去交易嘛。交易的时候你就这样一分一分钱的这样一点一点赚。

史东(42:46)

但是积少成多嘛。这个我想把话题再带回到这个域外管辖权这个事情。然后你从这个事情讲到你的观察,对华为,美国对华为的惩治,然后你的想法就是说这个事情不一定是trump政府或者白宫所做的事情,而是司法部他们已经形成了一种产业链。这种产业链基本上就是一个国家级的一个shake down 的一种行为模式嘛,对不对?就是一种敲诈

王孟源(43:24)

我在我的文章里面就是提到一个比拟,就是我觉得很跟十七世纪英国的海盗很类似。海盗的英文是pirate,对不对?但是那个当时的海盗,其实你不能够不分青红皂白的看到商船就抢,你必须要依附哪一个国家,这样子你才有港口可以靠,要不然到了哪里都是过街的老鼠。所以他们其实都是国家的代理人。就是比如说像英国的话,他就会开一大堆Letter of marque 给那些海盗。这些海盗专门去抢那个西班牙的商船。那这时候他们就不叫海盗了,他们叫privateer,这个其实是一种佣兵的形式。这些人我觉得司法部这些人就是外神通内鬼嘛,那他基本上已经变成一种雇佣兵了。就是他们基本上是为自己的利益而办这些事情。

史东(44:25)

这个事情的好处,我觉得基本上还是回归在美国的国家在世界上的威力跟他的势力作为后盾,不然的话他没有办法达到这个地步。第一个立法是他在立,审判是他在审判,执行是他在执行,宣传是他在宣传,这件事情。这完全是一条龙的作业嘛,甚至完完整整的产业链嘛,对不对?

王孟源(44:51)

 这一件事情如果不是那个吃亏的法国高管自己写了一本书出来呀,根本就没人知道。alstom被那个GE这样子莫名其妙的吃下来,当时的法国总统叫做奥朗德,他根本就不在乎,你法国的工业的皇冠上的珠宝被美国人吃掉,他根本不在乎。但是现在Pierucci是那个高管,我刚刚准备要讲,就是在二零一二年,美国司法部听到风声说有在印尼行贿之后,他把这个Pierucci在二零一三年四月到纽约办公的时候,在纽约国际机场当场逮捕。逮捕以后马上丢到那个重刑犯监狱去,跟那个杀人犯住在一起,然后威胁要判他十九年。那司法部有什么证据?唯一的证据就是他们拿到alstom内部的那个电子信函,里面有讨论要对印尼的中间人付钱

史东(46:01)

王先生,为什么这个事情这么这么耳熟?听起来又是这个email跟美国这次处理华为是一回事嘛

王孟源(46:12)

他们有办法弄到这些email 嘛。因为他们这个他们全世界到处去到处去偷听。那不管怎么样,他就是这个email 上面,有cc就是抄送给这个Pierucci,就是Pierucci根本就只是旁观者,他根本就没有参与这件事情。他不管三七二十一,因为他的那个阶级够高,他这个是算是前十大主管之一,对吧?从美国角度就是大肥牛,只要是美国人能够把手放在他身上,就把他抓起来。抓起来以后就不管三七二十一,要折磨你,折磨你以后他没办法了,他就说你要么你就继续争辩,争辩的话,这个律师费很容易就是几千万,而且摆明了就是已经整个制度,整个检察官兼法庭,就是要整Alstom,这根本就没有什么所谓公平审判可言。那他就只好认罪,认罪了也没有用,也没有被放出来。什么时候被放出来呢?他是四月被关进去的,大概七八月的时候,那个Alstom的总裁就知道这个事情是怎么回事。他主动去跟GE联络,说要把燃气轮机卖给他。他的希望是说GE口袋里面有一大堆国会议员嘛,对不对?也有很多法官了。这件事如果扯上GE的话,自然就这些司法部人就不敢再继续闹大。果然是如此,他们这个美国司法部继续逮人,一共逮了四个,逮到第二年的二零一四年的四月,然后,最后一个人被逮的第二天,他们就公开了,要卖给GE,公开之后,就没有人被逮了。到了二零一四年六月,他们签了约,正式签约把这个燃气轮机部门卖给GE。然后就真的那个Pierucci就被保释出来了,他被关了十四个月。你说这种很明显,他是一个旁观者,没有什么重犯,居然在重刑犯监狱被关了十四个月,一直到GE现在这个他所以前所工作中的燃气轮机部门,现在变成GE的,他才被保释出来。这个你说他不黑,我想没有人会相信啦。然后当然这些司法部的人,还有律师,他必须要继续拖下去一面拖,每个月就是几千万的律师费嘛,对不对?要拖到那个二零一七年十月,他们要结案了,结案了说,我们决定这个罚金,这个罚金还是要给的。因为你毕竟是用了国家机器这个壳子嘛,对不对?消耗了国家的公信力。所以那个美国联邦政府总是要分抽成,对不对,要不然你这样说不过去,罚钱当然罚不到GE的头上。因为这种事情你不能够在太岁爷上动土嘛,对不对?所以他虽然Alstom跟GE的那个买卖的协约上面说,这个事后要由GE负责。但是司法部就是不让他这么办。一定要Alstom负责,那Alstom说你这个不行,这不合理嘛,我们有契约,你这个不尊重我们的契约。那Pierucci就又被关进去了,其他的高管也都被关进去了。你说就是这样子吧,这跟那个海盗掳人勒索有什么差别?所以那个Alstom只好只好认栽。这下这样一来一去。再加上律师费什么的又是十亿,这样的Alstom就没有,真的没有现金了。没有现金了之后,只好把那个把那个铁路的生意也打包起来,要卖给西门子。这个案子现在我们还在审查之中,不过Alstom这个一百多年的老公司就这样完蛋了,就是这样子。不是因为他们那个经营不善,不是因为他们的总裁忽然要不爱国了,这件事情Pierucci的那个书是几个月前出版之后,在法国闹得很大。如果不是因为他们有那个黄背心的事件的话,他们的总统马克龙据说在国会抱怨了好几次,原本是可能会有些行动的。我把这件事情写出来,当然对我自己也是有点危险的。不过既然经济学人已经把它在英文世界里传播开了,我想大概还不至于。

史东(51:07)

我跟你讲,有一个很好的,就美国司法部应该请你去做顾问。

王孟源(51:14)

我是觉得在中国的外交部,他们在对外宣传,还有对外了解上面,真的是做的(让人摇头)

史东(51:22)

我跟你讲,这是我的习惯。我在访问之前我不会跟我的受访者说要说些什么,或者不说些什么,我会给一个大纲说,我可能会问一些什么,但是我不需要你马上给我答案。我喜欢这种现场的即兴式的这种来往跟问答。我在今天访问之前,我虽然看过你的部落格,我知道你讲的故事有很多在你的部落格中写了,但是我没有意识到这件事情它黑暗的程度,一个很严重的程度。在国内所谓中国国内谈到华为跟美国之间的这个案子的时候,大部分都说华为在5G方面的成就令美国担心,令美国不放心,而被美国有这种要消灭华为的这种心态跟行为。但是你说的这个并不亚于在中国所讲的这一套理由,就是美国为什么对华为如此这般

王孟源(52:36)

这个不是trump的modus operandi(行为模式、作案惯例),他是商人嘛,他就是商业手段嘛,对不对?这种用司法手段去搞,他从来没有这种权利过,所以他也不会想着这样去做的。这完全是因为这批海盗为了自己赚钱,那必须要找那个钱多那又在美国没有势力的。那现在钱最多又在美国又是人人喊打的是什么?就是中国企业嘛,对不对?那中国企业在海外销售靠海外销售最多的就是中兴跟华为。中兴是华为是这样被盯上的。不是因为这些人爱国,说我要帮助美国打击华为,而是因为他们要赚钱,必须要找软柿子捏。而软柿子就是指在国会那边没有人会反对的,对不对?那你说美国国会有哪一个议员会反对说他们去抓华为的人了,没有了吗?对不对?这个政治正确是这样子。所以我之所以冒一点危险,把这件事情用中文讲出来,是因为觉得中国政府如果不了解幕后的真相,他处理的手段就不会正确。因为上次去年中兴那一件事情发生之后,他们处理的方法是什么?就是去找这些始作俑者,这些律师然后几千万几千万几亿的付给他们,希望他们能够减轻罚则。那我觉得因为他们掏钱比那些法国的企业还要爽快。你如果回忆一下去年那个样子,这个很可能现在他们会这样子变本加厉去抓华为。就是因为上次中兴两次被抓,掏钱掏的太爽快了。他们觉得这个与其去找其他的法国企业,七亿他们掏的这么不快活,弄了五年才弄出来,才弄到七亿,不如抓中兴、华为,几个月九亿就很爽快的支票开来,而且还高兴的不得了,还以为自己捡到便宜了,对不对?你说如果你是司法部那一帮那一撮人,那个集团里面的人,你会觉得说你下次要找法国企业,还是要找中国企业。

史东(55:27)

其实这个话讲到这儿,我必须要问这一句非常重要的。我们如果知道这个事情是这样子,就知道美国的司法部或美国的政府是这么一个玩这种这样的游戏的话,中国政府应该怎么办?

王孟源(55:44)

他们这些人,第一个是怕见光,对不对?我刚刚说过,他们这个是闷声大发财的,对不对?现在有可能有几个那个大律师已经身价都是好几亿了,但是你绝对不会在报纸上看到他们的名字,他们不喜欢见光。其次是他们怕真正有权利的那个中央政府监管,尤其是国会。中国是没办法在美国国会拿到任何影响力的,所以这件事情只能够公开来办,公开来办的话,现在法国人正在回忆Alstom这件事情,应该利用这个热头,由中国中央政府出面跟马克龙讨论这件事情,而且闹得越大越好,你就算是专门为这个来办一次高峰会议都可以的,对不对?如果你这个中国跟法国的总统来办高峰会议,然后理由是因为他们要讨论美国域外管辖权,可能是一小撮腐败官员在里面乱搞,这个美国的主流媒体即使想要把它盖住也盖不住,对不对?中法的高峰会他们怎么能够忽略掉呢?所以这种事情要办是可以的。但是中国的那个对外的理解,对美国的理解实在是让人叹息。居然去年中兴的事情他们的处理方法就是拿出九亿,然后再加上不晓得多少律师费,就给这票人,高兴的不得了了。事后还洋洋得意认为挽救了中兴。

史东(57:34)

其实这种行为我觉得也蛮自然的。就是中国人讲的叫做破财消灾嘛,

王孟源(57:42)

不行啊,真的,你把他们养肥了以后,他们下次会再来

史东(57:52)

问题就是这个。所以说我们必须要讨论出一个比较具体的建议给有关当局

王孟源(58:00)

所以如果我讲的话,习近平政府能够听得到。我的第一个建议就是现在立刻去跟马克龙开高峰会议讨论这件事。

史东(58:13)

我想世界上我想除了马克龙之外,我想世界上还有其他的国家的大公司也深受其害过

王孟源(58:21)

你看看我列的那张表嘛,siemens 德国的siemens 也是高高在上,第一名是巴西的那个petrol bus,对不对?连那个英国都有一家公司在上面,ok。这种东西你要到欧盟跟他说,你们不是要改革WTO吗?世界贸易组织吗?把这种单个国家的敲诈限制一下嘛。你一个欧洲的国家在亚洲行贿,关美国什么事情,要罚也不能美国一个国家纳入他的那个口袋里面,对不对?你美国只占了这里面利益关系的百分之五,那你就只能够罚百分之五,你不能够罚百分之百嘛。这种长期来说,你这种规则要订立,订立以后这些人无利可图了,他才不会去找你的麻烦。只要他有利可图,但是他现在是躲在美国国会后面,美国的国家机器后面自肥,你除了把它揭穿然后让他难看。他必须要有代价。美国的国会不在乎你中国,但是他会在乎法国人,对不对?很难看嘛,对不对?你自己是盟友,你这样子把人家

史东(59:40)

在那个单子上,我顺便也提一下,这张单子上几个国家,巴西、德国、法国还有一个以色列。

王孟源(59:50)

你想想看,这还只是那个FCPA,其中的那个比较小的那个手段。那个那个经济制裁禁运那个才真正是大笔。现在为了伊朗,整个欧盟也是说我不想让美国禁运嘛,对不对?你光为了这件事情,你说要跟欧盟的高官开高峰会,要跟merkel德国的总理开高峰会,他们都会愿意谈的。因为他们都也是头痛的,他都站在你这边的。你不要指望说美国人会为你发声。但是你如果是他们自己也想要谈的话题,你去跟他们约来一起来推动,这就是顺水推舟。这个我就不懂为什么中国人不去做。其实我懂了,就是因为他们完全没有看穿美国人这套把戏嘛。他们根本其实是在资助去年中兴的那件事,根本就是在资助这个团队,让他们继续壮大,而且吸引他们来整华为。所以你说现在华为出了这件事情,为什么?就是因为去年中兴那件事情处理不当,没有真正了解到美国这个黑幕,内幕到底是怎么回事。

史东(1:01:05)

其实即使像你这样专才,你也是在在后来才恍然大悟,你当时也没看出来吧,对不对?

王孟源(1:01:13)

不是。但是这个经济学人是一个很普遍的英文杂志,他这个专栏上的两篇文章出来了。我还等了三天,我才写出来,这三天有哪一个中国人

史东(1:01:27)

我相信了,因为我知道第一个前提是我自己智商不够高。第二个我自己的经验,我看了你的文章之后,我知道有这么回事。我知道这件事儿还蛮有意思的,可以继续追查下去。但是我一直到今天跟你谈话,我才知道这个事情的重要性跟它的严重性,这中间有差距的吧。

王孟源(1:01:50)

这个华为这件事情你绝对不能认了,你如果认了的话,下次变本加厉啊。不只是华为的,连什么那个中铁、中船,这些通通都完蛋了。中国的那个国营企业的贸易那个主管基本上都不用出国了。而且基本上如果是出国的话,就是在帮美国赚钱嘛,所以他们都跑到美国那边去了。而且你想想看,我很担心孟晚舟,因为她如果被抓到美国的话,这个可能是关进黑牢,什么单独禁闭这种事情都会出现。为了施压,这些人是无所不用其极,他们连法国人都可以这样整,中国人他们根本不在乎的。

史东(1:02:36)

对,其实我最近一连串发生的事情,像这个事情,华为的事情、这个委内瑞拉的事情,还是世界上美国的其他的世界上的各地的折腾,跟他们对于这些某些国家,特别是小国家,他们可以这个吃定的这些小国家的这种蹂躏的程度,每一个都是一个例子。中国人看到就是说,如果中国不强的话,中国就是会被美国这样的处理。

王孟源(1:03:08)

但是你这个长期来说是这样的。但是凯恩斯说过,in the long run, we are all dead. 你如果看长期的话,我们都死了。我们还是活着。我们今天还是必须要活下去,明天还是必须要活下去。你今天明天怎么活下去,你要处理,而且我们希望明天活的能够比今天更好,对不对?,所以这个去年好几年前中兴认罪,就是已经入了他们圈套。去年被他们这么一整,急急忙忙的把钱掏出来,这个还是由中央政府出面,中国的中央政府出面去跟trump套热乎,才把这个司法部这件事情解决了。你觉得解决的很好吗?我们现在回头看看,其实基本上就是鼓励他们继续去抓华为的人。我的建言希望能够传到中国的中央政府去,或者是至少一个能够有人听的话可以带过去。就是这一件事的黑幕是这个样子。那这些人他们怕的就是见光死。我刚刚已经讲过,他们也知道自己就干的这一套是见不得人的事情。所以如果报纸上有报出来,他们这个至少财路就断了。所以你当然最好是能够如果是美国国内的企业,他们有那个国会议员的话,就不怕这件事情。因为他们十几年前就解决了这件事情。有国会议员有法官会保护他们。但是你中国的企业或者甚至法国的企业,你没有办法靠国会、美国国会来保护,没有办法靠美国法官来保护。你这个时候就必须公开的揭穿他们的把戏,闹得越大越好。Pierucci 的这本书中国应该马上翻译过来。现在加拿大在审判孟晚舟的这个法官,你应该送给他,马上送给他这一期的这个economists,然后要求他把那两篇文章看完了,可以好好的看完才能了解。说你把那个孟晚舟引渡到美国会是什么样的命运?这么简单的事情,我真是不懂,为什么十三亿人没有一个人才能够看得出来。我在这边干着急也是没有办法,所以才会写文章,才会赶快要上你的节目。

史东(1:05:44)

因为这个事情的确实不是在一般人的生活范围以及想象范围之内的一种状况

王孟源(1:05:55)

我想我是天生对这些盗贼的花样就很敏感了。所以司法部搞这个玩意儿,我也是一直有疑心了。但是一旦一有了证据,我就觉得可以讲了。尤其现在孟晚舟她的这个案子还没有确定,只要人还没有进美国,你还有挽回的余地。你如果真的是进了美国,她开始被人家折磨了,你这真的只能够由习近平出面才能够解决了。习近平要么必须拜托trump去。可是这个trump他本身就被那个司法部在整,对不对?他并没有对司法部有什么特别的影响力嘛。那最好的办法就只好跟马克龙去国际的那个媒体上,国际的去搞国际舆论。但是这个就是搞得非常大。你最好是能够现在立刻出手。在那个加拿大的法官面前提出这些证据做这些议论。我不知道中方在加拿大雇的那些律师是什么。如果有美方的律师的话,说不定那个美方的律师就是这个团队的同伙。

史东(1:07:08)

我想这个事情的解决,非常明显的就是要多管齐下。你讲的这个合马克龙联系和欧洲领袖们的联系,跟在加拿大的这些事情要同时的去做才行

王孟源(1:07:22)

我三天前写的那篇文章,我马上就联络我认识的观察者网的编辑。但是他有一个首页,这是他的正式版。这个看的人大概比那个博客栏要多七八倍了。三天前我那个文章一写出来,我写这篇文章是有用意的,就是希望能够帮助孟晚舟。结果我拜托那个那个编辑,那编辑说他他马上就去推荐,推荐的拖了两天多,他们才放在首页。为什么?因为这个编辑是科技编辑,他是不管这种外交的事情,他只能够推荐,他不能够决定。我想那个管外交的那个编辑,管首页的那个编辑,自身也没有认识到这件事情有多严重、有多么紧急。他这个文章直到不到一天前才被移到首页上,然后才有人注意到这。那你要等那个大陆的舆论再推上去,然后再层层上转。我真的不知道这个几率多大。我现在今天在你的节目上,我是把话讲明白了,就是我写这篇文章是真的是在做建议,不是我自己一个人在写的高兴。这真的是这件事情有时间性,这个不能不尽快正确处理的话,会有很严重的后果的。所以我是希望那个中方能够注意到这件事情,因为这些人是贼呀,他们是在消耗你。你从美国的观点来说,他们也不是爱国者,他们是在消耗美国国家的公信力,来中饱私囊。他们是从美国的观点来说,他们也是卖国贼。

史东(1:09:14)

我同意

王孟源(1:09:16)

所以我希望我这一次站出来,希望能够那弄点那个舆论的风声。也是因为三年多前,我在那个大对撞机那件事情,最后有点影响。原本开始写blog 的时候,没有想到说我会有任何实际的影响。因为你想台湾那种状态有多少高人现在还在台湾也是埋没着,有多少人有正确的认识结果也是没有用,对不对?,但是在大对撞机的一件事情上,在中国大陆的舆论就有正面的影响。你想想看,这大陆十三亿人,比台湾要多多少倍。我一个人微言轻的外人讲的话,能够凭理性跟证据,就凭那个逻辑的力量能够有影响。这个让我对大陆的观感就好很多了,对不对?一个国家他的未来,他的那个未来的前途,其实是取决于他能不能够理性的听取合理的建议,逻辑的建议,然后采取最佳的方案。这个大陆在这一点上做的比台湾高明不知道到哪里去了。我希望这一次这批美国司法部跟律师这样子,为了私利来危害中美关系,来危害这些无辜的人,我是希望能够尽一点微薄的力量。然后因为我知道他们在玩什么花样。但是很奇怪的是很奇怪的,我好像是唯一一个注意到的人,所以希望能够有人听得到,刚好能有机会上你的节目。所以我赶快借这个机会也大声疾呼,希望能够有人转达到有权利执行的人。

史东(1:11:23)

我想只要这个消息正确,一定会转达到的。不是我们,一定有别人,一定有人会转达到,只要这个消息正确。最后三十秒还有什么还有什么嘱咐叮咛没有

王孟源(1:11:39)

现在这个跟trump的贸易冲突,我讲几句话好了。trump是一个商人,他现在最关心的是他的民意调查,原因是因为他被司法部追杀,他能够处理他目前现在这些头疼最好的办法就是连任。他如果连任的话,就代表选民说,我不在乎你司法部起诉什么。这在美国这种政体下,你在被起诉之后还胜选这本身就是正当性,就代表选民不在乎这件事情。那选民不在乎就代表整个体制不应该在乎。所以trump现在最担心的就是要连任,要连任就必须要讨好他的选民,讨好他的选民,所以他才会为了那那堵墙闹了一个多月,不让那个预算过关,为什么?因为这对他很重要,他甚至连度假都没去度,这是他第一次没有去高尔夫场度假。然后他接着第一次到海外劳军,这个的代表的不是他的强势,而是他的弱势。正是因为他头两年共和党掌控了国会的那个局面已经不在了。他了解到第一个在政治方面,他已经会被国会掣肘。第二个在司法方面,mueller 的调查已经快要出来。所以在这种局面下,他必须要跟中国达成一个协议。他能够跟他的选民吹牛的,ok。那中国因为在大豆禁运上,也导致了一些媒体的报道。所以如果他能够把大豆的销路弄开,这当然是一点,但那应该是不够的。应该中国还需要再给他其他能够吹牛的东西。那我讲这个是说trump本身是希望能够有一个协议,他不想这个无限的拖下去。现在不幸的是,中国的经济在这个经济周期之后,衰退的要比美国早半年到一年左右。而且这并不是因为这个贸易战的关系。你如果看过去这十二个月的那个贸易消息,中国对美的出超反而创了记录。这个很简单嘛,因为几个月前就有我一个朋友写信来问我说怎么搞的。这个出超反而还每个月每个月都在创纪录。我说这很简单嘛,因为那个他们叫做front running, 就是囤积。对它关税还没有真正还行。那大家先赶快进货。对,所以那个整个一直到十二月都还在创纪录。所以如果这个你是看这个贸易战的影响的话,照理说他对那个中国的出口是有帮助的,一直到现在都是有帮助的。对中国贸易经济的现在这个衰退是本身的问题,就是因为他去杠杆跟那个地方债。这个在二零零八年之后,他们刺激经济,这个经济有浮肿,到现在这个后果还在消化之中,他是因为这样子,所以才会自行衰退。那这个时机很不利,因为这个习近平跟川普同意的就是三个月的缓冲期,到三月一日必须要谈一个决定来,当然也可以继续拖下去了。中国是希望能够解决,但是中国不一定要马上解决,所以如果能够延期到三月,再延三个月,或者再延六个月,我觉得是一个最好的办法,就是拖到美国的经济也开始下行了。美国为什么他的股票市场在过去一两个月会出问题?因为股票市场是前瞻六到九个月的,他已经看出六到九个月,就是今年下半年美国经济会出问题。那中国要是能够拖到美国的经济也出问题之后,再跟Trump谈判,他的那个底牌就会硬很多。所以我的看法是,中国最好是能够再继续谈下去,然后要求美国再继续把这个谈判期延展半年,延展到比如说九月一号,然后希望在七月或者八月再跟他谈判。因为中国这个经济缓慢,现在已经是很明显了嘛。再过半年的话,大概大概已经触底开始反弹了。那时候底气就比较够,美国的刚好是他们的经济开始进入衰退,这是我个人的想法,这也不一定就是这样子。如果说非要现在就有一个协议的话,只要这个协议不是伤筋动骨,不要不是有制度上的改变,制度上的限制,不要有长远的例子就还可以。因为你总是要记得这个trump 在二零二零年当选连任的机会其实并不是太大了。很可能下一任的民主党政府根本不在乎你跟trump有什么协议。所以你不应该有什么在两年之后还会绑住你手脚的事情。这如果没有这样的事情,跟他现在达成一个协议,就是花钱买消灾,其实也还可以了。我是希望最好是能够拖到那个trump就是时间谈判的这个形式更有利的时候,这是我在对trump的这个看法。你不论你是用什么手段来花钱买消灾,你都要担心会不会像对中兴那样子,反而让他们变本加厉。

史东(1:18:08)

贪婪这个事情是没有底的,没有人会有一个底线的。

王孟源(1:18:14)

所以你现在跟他谈判,他把这个关税从百分之十到百分之二十五,这个加上去他暂缓,你再把他拖六个月的话,你拖到夏天或者秋天才定案。这时候很快再过半年,他们就开始大选了。因为那个美国的选举是在选举年的一月基本上就开始了。那到时候他就根本没这个功夫再来找你麻烦。所以我的看法是这样子的。至于这个司法部,这个是另外一回事,就因为这个司法部这个跟川普独立,我觉得不一定川普能够帮你解决。因为川普自己就搞不定司法部,(泥菩萨过河)而且这司法部他们这些人,他们的标准答案就是司法独立,这个美国的政治正确里面,包括司法独立。这里面他们怕你把这个底牌完全揭开。但是你如果不完全揭开了底牌,你比如说你走trump那一条路的话,他就会说司法独立,那就是你也没办法了

史东(1:19:24)

了解了解。孟源,非常谢谢,祝你一切顺利,好不好?一切顺心。然后随时我有什么要求,我会跟你联系,我希望你随时有什么想法,你跟我联系。

王孟源(1:19:44)

好,很荣幸上你的节目,谢谢。

\twocolumn[\begin{@twocolumnfalse}
\section{中美贸易战}
\subsection{20190512}
\end{@twocolumnfalse}]Credit: anonymous



史东 00:12

各位朋友你好,我是史东。我想这几天大家都会关心同样的一件事情,就是有关于中美贸易谈判的事情。那么今天的节目中,我借用几天以前,5 月 9 号中评社在香港的一则新闻,来把今天的话匣子打开。它说:在中美贸易谈判渐渐现曙光的时候,美国总统 Trump 周日突然变脸,威胁说周五起(这是几天以前的周五起)要对 2000 亿美元的中国货品加征关税。这件事情在环球股市上激起了千层浪。



当然这个事情也是最近几天以来大家都关心的。那么这件事情发生之后,我收到很多的信息,很多的朋友都不断地告诉我,希望听听——今天我们为您请到的来宾就是王孟源——王先生他对这个事情的看法。我跟王先生联系了,王先生就非常大方地答应了,问我什么时候方便,我说你说什么时候方便我就什么时候方便。于是我就造成了今天我们为您做的这个访问。



首先把王先生带进我们的画面之中,谢谢您能参加,谢谢您的参与。



王孟源 01:27 

非常荣幸再上你的节目。



史东 01:29 

有关于这件事情,我想先从一个大大的框架来切入。总的来说,你对这个事情的反应如何?



王孟源 01:41 

我们节目开始之前我已经跟你稍微提过,就是去年我在去年 4 月, 2018 年 4 月的时候已经做过一个全面的分析,到去年 8 月的时候,也是上你的节目之后有了一些更详细的讨论,然后我又把它写成另外一篇我博客上的论文。里面其实把过去这一年多所发生的事情都完全预料到了,而且描述得非常详细。



我们做这种时事的分析基本上有三个层次:第一个是 Diagnosis,你先要了解事情的来龙去脉;第二个是 Prognosis,你必须要预期未来会怎么样继续发展;最后一个是 Prescription,你在各个不同的方案之中选择最优的方案。我那两篇文章其实这三个层面全都解释得非常的详细了,我鼓励有兴趣的读者回去复习一下。我在这里当然也会把它重新解释一下,为了大家的方便。



基本上,美国对中国的出手并不是纯粹追求自己贸易的利益,所以你不能够用理性的、共赢的那种态度来做谈判。最基本的谈判态度就是双方求共赢,那中间的那个线划在哪里可以互相推。最后可能一边赢得多,那另外一边赢得少一点,但是你至少两边都要赢,这才是一个正常的谈判,但是...



史东 03:36 

各取所需,是吧?



王孟源 03:38 

对。但是这次的贸易战,从一开始就不是这样的一个理性的谈判。美国之所以会挑起这次贸易战,其实是在 2008 年金融危机之后,他认识到过去在那之前 20 年用宣传跟金融的方法来打击中国,已经完全失效。这是因为中共对内部的金融、货币、还有媒体跟宣传的管制非常的严格。所以美国在 80 年代用宣传搞垮苏联,用金融打垮日本这两个霸权挑战者的这个手段,他在冷战结束之后用了将近 20 年在中国身上,一点用都没有。



到了 2008 年之后,美国自顾不暇,眼看着中国在 2010 年它的工业产值超过了美国,在 2014 年它的 PPP  ——就是 Purchasing power parity,根据购买力的平衡之后实际的 GDP 产值——中国其实已经超过了美国。现在到 2019 年,中国的工业产值已经比高出美国 80\% 了,已经是美国的 1.8 倍。日本跟苏联的工业产值从来没有超过美国的 70\%。所以基本上,他要遏制一个霸权的挑战者,已经是为时已晚。



所以在 2008 年金融危机之后,Obama 当选总统。他花了两年来获得喘息的机会,把危机最大的那个风头给解决掉了以后,马上就有动作。这就是一个跟冷战时期的宣传、跟(对日本的)金融的手段不一样的(动作),就是 Pivot to Asia。Pivot to Asia 只不过是一个态度,实际上的手段细节是有三方面:第一个是它的那个TPP,这是在亚洲的自贸协定;然后对欧洲是TTIP;还有就是在 WTO(World Trade Organization)的章程上,它要求把中国永远定义为非市场经济。这是三管齐下。



我在去年 4 月的时候就已经解释过它基本的思路。虽然整个的策略叫做 Soft Power——这是哈佛大学的 Joseph Nye 搞出来的——但是实际上说穿了,他的思路就是,我没办法用宣传来忽悠你自废武功,那我就只好用国际关系、国际机构来强迫你自废武功,强迫你放弃你的公营企业,强迫你放弃对经济的管制,强迫你放弃对金融的管制,强迫你放弃对内部宣传媒体的管制。然后等到这些让中国高速增长的背景、政策被强迫放弃之后,它可以再试图颠覆。



因为事实上中国的原罪在于它的体量太大了,它的体量超过了所有第一世界(就是美欧日澳)人口的总和。你不可能成为既有国际秩序的 Follower,它自己必须是一个颠覆性的崛起,这对美国来说尤其的糟糕。因为它在过去这四五十年(基本上在 70、80 年代之后)已经习惯利用霸权来攫取非法的利益。我们今年年初讨论的域外管辖权,或者是在稍早前我在我的博客上详细讨论的 Manafort 案。Manafort 是 Trump 前任的竞选经理,他是一个游说的掮客。他基本上就是介绍美国的政商精英接收外国银行或者客户的贿赂,来获得管理世界的一些后门。



我们现在看到美国的系统里面是充满了各种后门的,你只要有钱有权就是有办法。



史东 09:05 

基本上美国是借这个机会去出卖美国自己在世界之中的影响力。



王孟源 09:14 

对。这个美国的政商精英来说,这是完全没有损耗到美国国内的利益,因为损坏损害的是世界的秩序,对不对?像 Manafort 一个典型的经历,是他把中亚的一个负责洗钱的银行把它洗白了。洗白了的话,这是中亚的银行,并不影响美国的银行界的健康,对不对?他们只收了几千万美金,对黑银行来说是很便宜的代价就可以继续做生意,但是对美国来说没有损失。损失的是谁?是中亚的国家,是整个世界的金融体系。



因为美国是世界的霸主,所以他不只是世界的警察,他也是世界的主管。尤其是在金融界、政界、外交界、军工交易等等都是美国说了算。这里面就有很多决定,它影响到的纯粹是外国人的事情,他们收贿也就是收的外国人的贿赂,这不影响美国,所以他们连犯法都没有,甚至连对美国的道义责任——他们是美国的政商精英,他们对自己国家的道义责任——都没有受到影响。他们牺牲的是世界的权益、世界的秩序,但是他这样能够赚容易钱。像 Dole,他一年就只是到纽约出席一次大概一个多小时的会议就可以拿几百万。Bob Dole 这个事我去年写了一篇文章详细解释了。这样的好生意它的前提是美国不受任何挑战,不受怀疑的霸权。



所以美国的霸权对美国的政商精英来说非常非常的重要。美国在 80 年代打倒日本跟苏联对他的霸权挑战的时候,还只是从整个国家的观点来看,必须要保护自己国家的利益。现在是美国这些政商精英要保护自身个人的私利,这个更是切身,所以也就更加不择手段。那我必须要先把这些话解释清楚,因为这对我们现在这个贸易战有很重要的影响。



首先是美国的任务不是——像我刚刚一开始讲的——要做一个理性的谈判,达成双赢:美国要赢多一点,让中国赢少一点,这不是。美国的任务是要把中国的崛起扼杀掉。我刚刚提到 Nye 有这一套 Soft Power 软实力的策略,Trump 一上台就把它全部砍掉了。现在 TPP 变成日本搞了,没有美国的加入,然后 TTIP 现在已经是无限期的冻结,因为它跟欧盟闹翻了,他根本就不让WTO任命新的法官,所以现在 WTO 根本没办法做任何的裁判。但并不是 Trump 认为 Joseph Nye 的那一套策略是错误的方向,他只是认为他的作用太慢。Trump 这些人,他的 advisor 是Navarro、 Bolton 还有 Banner 这些人都是极右派的,这些人都是很直接而且很不耐心的。他们觉得不需要经过国际组织来搞中国,可以直接跟中国来对抗,就是直接压迫中国来吞这些穿肠毒药。



史东 13:49 

我再插一句话,美国这么做是为了美国的利益,还是为了美国后面的财团的利益?还是两者之间是 intertwined,是没有办法分割的?



王孟源 14:03 

对,都有。所以现在美国打击中国的这个势力是左右两党、财团精英、政商精英以及蓝领阶级的老百姓有志一同。你在美国,这个内斗非常的严重,它基本上找不到像这样子全国一致的,有一致共识的东西,但是打击中国是唯一的例外。因为我刚刚提到,它在 80 年代打击日本跟苏联的时候,还是为了国家整体的利益,现在已经变成国家利益刚好跟政商精英的利益是同一个方向,一致的,所以它做起来更加的不择手段。这些背景我们必须要先搞清楚。



史东 14:57 

照你这么解释的话,就是这个事情就不是一个 Trump 的 issue,而是美国的 issue。换句话说,下一个美国总统谁上来,不管是不是Trump,跟整个事情的发展不会有太大的关系,不会有太大的影响。



王孟源 15:14 

手段会不一样,但是那个目标跟方向是绝对一致的。



事实上 Trump 本身是不学无术,而且他没有什么政治信念,他也不属于美国传统的政商精英,他对中国其实没有个人的利害关系。他搞这个完全就是为了选举,为了讨好民粹,他在国内有很多问题,比如说他逃税,比如说他干涉司法。前一阵子通俄门最后不了了之。事实上 Mueller 虽然没有明说——我觉得 Mueller 毕竟不是玩政治的,他是法律人出身,他没有理解到这种事情他必须要明讲——但是你如果看那个 Mueller Report,他的意思是说:通俄门查无实据,但是 Trump 干涉司法是可以起诉的,只不过因为他是总统,所以不能起诉。



像这些莫名其妙的问题很多很多。Trump 要解决这个最好的方法就是获得连任。因为美国跟英国这种所谓西方民主体制的传统就是:不管再大的问题,民意最大。你只要赢了选举,就代表国民原谅你了,不在乎这些事情。一切的这些政治甚至法律问题都可以推到一边。所以 Trump 现在也不择手段地来对中国打击,并不是因为他属于美国的政商精英集团的一部分,基本上他是一个政商精英集团的局外人。但是因为美国的蓝领民粹,也就是他的基本盘是也是仇视中国的,所以他必须要讨好他们。



我们把这个大背景先讲清楚以后,然后又讨论到 Trump 本人不学无术,还非常的虚荣。他以推特治国,然后他获得消息就基本上是看 Fox News。这些我们都先讲清楚,然后我们接下来才能够讨论细节。



这个贸易战一开始的关税,这是因为 Navarro 建议这样子。那至于 Navarro 会为什么会建议从关税战争来压迫中国,我去年也在节目上提到过了:中国对美国的货物出口每年是 5000 亿美金,美国对中国的货物出口每年是 1000 多亿美金,大约是 4:1 的比例。所以和你如果纯粹只打关税的话,很明显的是中国会吃亏,因为你有 4 倍的货物可以加关税。



我在去年那两篇文章已经讲得很清楚,中国在这四年(2017-2021 年,Trump的任期)上有 Trump 当对手,而不是 Obama 或者跟他类似的一个总统来做对手,其实是很幸运的。因为你如果是继续 Obama 的那个政策的话,这是是要美欧日澳(整体西方的第一世界)来围殴中国,那基本上是没有好的方法来解决 ,这是非常严重的问题。但是 Trump 选择跟中国单挑,然后同时还跟欧洲日本也都宣战,这真的是天上掉下来的礼物。



但是有了这样的幸运的背景,你自身也必须要认清楚这个局势。所以我在去年一年多前就已经一再地强调,你必须要尽快地把 Trump 打到痛,然后尽快解决。刚才节目一开始你问我说这个谈判有什么好处?谈判的好处就在于这是美中单挑,你如果跟他解决了,至少到 2021 年,甚至到 2022 年——就是即使是下一任的是民主党总统要重回 Obama 的政策的话,他也需要一段时间重新沉淀,然后跟欧洲跟日本重新建立关系,对不对?所以这又是好几年的缓冲期。所以你跟Trump 单挑的时候,只要没有伤筋动骨,就是没有真正去改变自己的体制,丧失自己的竞争力,能够尽快和解是一个很大的好处。



我那时候就解释了有上策、中策、下策。



下下策当然就是投降,你对它要求的那个体制变革照单全收。因为这个体制变革是 Navarro 开出来的,然后由他的贸易谈判代表 Lighthizer 来负责跟你交涉,对不对?它这个背后是充满了毒药的,就是要求你放弃自己的竞争优势,你当然不能够接受,这个就是全面投降,这是下下策。



我那时候说有 3 个策略。第一个是你真的要做到对等。你的出口是它的 4 倍,你怎么做到对等?重点是在美国没有让中国的财团进入美国做投资、做生意。但是美国的财团却在中国有很大的市场。中国市场的整体规模其实跟美国是同一个数量级的,在某些方面甚至更大。比如说汽车,它比美国大大概40\%,所以像 GM 它有超过一半、 60\% 的利润来自中国。这是不是美国的出口?不是。因为它是在中国生产的,但是你能不能够打击它?当然可以。因为事实上美商在中国所赚的利润高于中国商人,对美出口跟投资所赚的利润大概是 2:1。



所以你如果回顾去年这个贸易战的那个过程,一开始是 Trump 先跟你 1000 多亿的商品做关税,这个 1000 多亿不是随便乱定的,他这定的就是中国对美国一年的进口额差不多是那么多。但是等你做对等关税以后,第二步就超过你能够对等的范围了。然后在 8 月的时候,我就详细地说:上策是马上就对 GM 或者是宝洁这样的公司开始征税,开始打击他的利润。因为中国没有同样的大厂商在美国市场赚钱,美国这方面没办法做对等的反应,这是上策。



中策是私下对抗施压,但是不要公开地加税。下策是你在口头上说对等,但是你的一般的关税加满了之后就没办法再做什么东西了。



王孟源 23:23 

那时候我说除非采用上策——这个是白纸黑字的,我真的昨天晚上又回去看看我去年写的,白纸黑字——我去年写的是你如果不采用这个上策的话,那么在一年之内跟美国签订协约的机会大概只有 30\%- 40\%,结果中国采用了下策。



就是他口头上说是要对等,但实际上没有真打痛 Trump。第二个没有提醒 Trump,说这个贸易战就是贸易战,没有人规定说只能够做贸易关税。你的美商在我国内所卖的东西,我一样也可以加税,对不对?你这一点要提醒Trump的,否则他不把你当回事的话,实际执行细节的是 Navarro跟 Lighthizer,那他们这两个都是想要全赢的,要杀到底。



我已经刚刚解释过,这个目的不是抢劫,而是谋杀。所以中方在去年没有照我的建议来做真正的对等的打击,这已经是非常失策了。我认为是因为中国的智库系统还没有好好地建立起来。就是因为当世界霸主,当世界第一的国家,它那个智库必须要有我刚刚讲的,很准确的 Diagnosis、 Prognosis 跟 Prescription。但是显然中共内部的智库没有提供正确的资讯,所以他们还是做非常保守的策略,就是怕事。他没有看清楚整个事情的本质,也不了解Trump 他的个性跟背景。



事实上我刚刚讲过,Trump 他对中国本身没有切身的利害关系,他也没有什么仇恨,他这个完全就是为了讨好民粹。那为什么会这样?拖了半天,到这个月忽然又出了变故,好像过去这 6 个月好像一切都很好,就是谈判一步一步地来,然后忽然到了一个礼拜之前,忽然就是风向一转。为什么会这样?第一个就是它没有采用上策,真的没有把它打痛,所以那个美国方面姿态还是非常的高,它这个还是想要全拿全赢,那么一旦碰到中方的底线,中方不愿意再退了,美方又不愿意让中方留着裤子回家,真的是要把它剥光。这个时候 Lighthizer 就回去跟 Trump 说谈判不了。



那现在有报道说是美方指控中方那个 Roll back the commitment 。就是反悔、反复。我觉得这是美国人在讲话,你如果当真的话,那伊拉克还有那个核武器。所以有三个可能。



史东 26:45 

这个事情当然非常明显,美国是 Rollback 的,世界上的大王,他指责别人Roll back……而且谈判就是这么回事。



王孟源 26:56 

谈判就是这么回事嘛,对不对?不管是中国在反复,还是美国在反复..



史东 27:01 

美国反复是已经谈了谈好之后的反复;中国如果有反复的话是在谈判之中的反复,这两个意义是完全不一样的。



王孟源 27:11 

对。而且不管有没有,美国人一定会这么说嘛,对不对?所以有三个可能:第一个是刘鹤真的承诺了什么,然后习近平把它否决了。我觉得这个可能性非常的低;第二个是刘鹤跟他们谈判的时候,可能是因为语气上没有解释清楚。这是有可能的;不过我觉得最可能的还是美方为了施压,所以 Lighthizer 就编出这个故事来,但是真正翻脸的是 Trump,因为 Lighthizer 没有权利决定是不是要跟中国翻脸。



为什么 Trump 现在忽然要翻脸?其实最主要的原因很可笑——不过 Trump 本来就是一个非常可笑的总统——原因是因为在过去这两个礼拜里面,民主党有两个名列前列的候选人——他的民主党目前有大概三十几个候选人了——排名前三名的第一个是 Biden,第二个是 Bernie Sanders,第三个是一个名不见经传的同性恋市长,叫做 Pete Buttigieg。过去这两个礼拜,排名第二的 Sanders 跟排名第三的 Buttigieg 两个人都出来指控 Trump 对中国太软弱,这是很可笑的。



其实美国每 4 年的大选,你如果回去看的话,永远都有人指控现任的总统对中国太软弱,从二十几年前, 30 年前就是这样子.这是因为候选人在当选之前是可以随便放炮,当选了之后就必须要务实。所以这个规则是...比如说很一个很明显的例子是 George W. Bush 小布什。他在竞选的时候也是指责克林顿对中国太软弱,但是一上台就跟中国签了,让中国进了WTO。



这是很好笑的,这就是西方民主制度的特性,你在选举的时候说一套,上台之后你就必须要照顾现实,必须要真正地做出有实际影响的政策决定,这时候就会有不同的看法。但是 Trump 不是普通的政客, Trump 是例外。他是美国过去 100 多年第一个民粹总统, 根本不在乎实际的政策影响是什么,所以他也就不知道这个逻辑。既然民主党的这两个候选人批评他对中国不够强硬,他的第一个反应就是,那我强硬一点点看,所以忽然他就翻脸了。



那从长远来看我认为Trump 的这个态度其实是对中国有利的,因为美国的本职是要谋杀中国的崛起,OK,会把这个目标这个大方向放下的只有一种政客,就是Trump这样子的疯疯癫癫、不学无术的政客。那他现在可以忽然跟中国强硬起来,他风向转的时候也可以忽然就跟中国妥协,所以这不是问题。我倒我觉得,这一次这个Trump忽然又加关税,这不是太大的问题,真正的问题是在于中国的本身的应对不对,就是我这个长期的处置、这个长期的因素。我刚刚一开始所解释的没有采取我去年所说的上策,就是没有把美国打痛,没有让美国了解这个它的关税因为进出口有 4: 1 的关系。这个是一个假象,似是而非的假象。实际上美商在中国所赚的利润是比中商在美国做赚的利润要高一倍。OK,他没有考虑到这一点,你必须要提醒他们。OK,因为你不提醒他们的话,Navarro 跟 Lighthizer 这些人他不会帮你考虑这些事情。我也没办法,因为我不在庙堂之上,我这个是一个山野闲人,我事情看得清楚,但是我除了在博客上讲两句,我也没有办法。你看现在我在中国的舆论界看到的还是都是隔靴搔痒,没有摸到实质。比如说基本上十几篇、二十几篇评论讲的都是这违反贸易战,违反经济理论,是杀敌一千自损八百……这都没有摸到事情的实质嘛?因为事情的体不是贸易利益,而是霸权竞争。那至于它的表层实际在执行的细节是Trump的这个民粹总统要追求他的选举票。OK,这两者都跟经济利益没有关系。我不晓得他们为什么十几个……你现在去看那个中国舆论上的评论,都是讲的经济利益,我们有损失,但是那个美国也损失也很大,他们这个是非常不智,Irrelevant!



史东 33:00 

在这个前提之下,你觉得中国应该怎么办?你刚刚提出来的那几个上策、中策、下策、下下策那 4四种策略还是 relevant 吗?还是可以用。



王孟源 33:15 

还是relevant。因为你除了这样子还能怎么办呢?现在的问题是你浪费了一年,以后这个Trump的任期就只有 4 年。



史东 33:25 

而且你刚刚也讲到这个事情,就是不是说这个事情的严重性,跟它的巨大性已经超出了Trump这个总统任期对不对?所以说下任期来也不会有什么……



王孟源 33:44 

你现在跟他签约,只能够解决到2021/2022,对不对?



史东 33:47 

对对对,我就是这意思。



王孟源 33:50 

你 2018 年能解决的事情,能够买三年的和平,现在顶多就只能买两年的和平,不是很聪明的事情。我觉得,当然中共内部没有蠢到说会选择下下策了。我觉得他们的问题是在于一方面他们是很理性的,他们这个他很理性,所以他们就整个文化是非常理性的,政治文化都是讲理性的,计算利害关系,所以他们也假设外国人会这样子。可是美国不是这样的嘛。你要了解到他的那个厉害,美国不是一个卡通里面的坏蛋,他是,他是一个很大的国家,很复杂的组织,里面有很多不同的人阶层,很多不同的玩家,在用一个市场化的那个方式,来决定国家的政策,这是一个很复杂的过程。那中国并没有类似的经验。所以他们纯从理性的观点来看的时候,就没办法看出这些很复杂,往往还包含非理性因素的事情。那事实上对外国的理解你不能够要求国家领导人知道嘛。因为国家领导人绝对几乎不可能是在外国住过的,你必须要有智库,里面的人是有真正工作政商经验,而且真正在外国住过。



史东 35:23 

对,所谓一个国家的,我不知道中文怎么翻译一个国家的Psyche?对,就是他的心理状况跟他的思绪状况是怎么一回事?



王孟源 35:37 

是,你必须要有专家能够理解这些事情。那中国的智库都是学术界出来的,这个学术界发展到 21 世纪,除了几个科目是可以继续做实验的,比如说像工程或者是生医这种是很明显的可以做实验的东西,其他的都已经开始钻牛角尖了, 99\% 的那个论文都是没有意义的。那你这个研究社会学的,研究人文学的那个,他们的那些论文,研究美国的论文,就好像用中国的古画来研究中国的地理一样,对不对?你怎么能够凭着古画去找石油呢?这你的智库里面全都是人文学系的教授,全都是不食人间烟火,所知道的是就是怎么写论文,怎么出版论文,他们遇到这种事情就没办法把这些细节,甚至连大方向都(没办法)掌握好。那至少他们,中国这次是因为他们的基本原则就是我不会投降,他们了解到他们知道自身的那个制度是有优势的,不能够放弃。



史东 36:53 

你觉得他没有谈判到最后,今天这个节骨眼上没有办法继续往前走,是不是因为中国政府意识到这个事情,一直到今天,他们才意识到这个事情的严重性?我所谓的严重性,就是说有一点符合你刚刚所说的美国所提出来的表面的要求,和对中国的意义是两回事。中国今天才意识到这件事情的意义是什么?



王孟源 37:21 

这就是我刚刚提的刘鹤去跟他们做了承诺,然后被习近平否决。我说过这个可能性不太大,我觉得刘鹤是个很有经验的学者。他不会应该不会犯这样错误。



史东 37:38 

对,几乎我看我这任何人看都是几乎是不可能的事情了。



王孟源 37:42 

对不对?几乎不可能的事情。对,这最可能的是中国一向就知道就有他的底线,他真正的失误在于他没有了解到美国人永远不会尊重你的底线,你除非把美国打痛了,否则没有办法得到一个合理的妥协。



史东 38:00 

所以说今天我们从开始谈论这个事情到现在,最重要的一句话就是要把美国打痛。



王孟源 38:08 

而且事实上是去年8月的时候,我在上一次上你的节目之后,我就已经讲过了。他那个时候他从 1000 多亿加到 2000 多亿,就是基本上有一半对中国的进口都加了关税,然后这已经是中国能加关税的 2 倍了,这个时候你没有利用那个好机会去把它打统。OK,因为名正言顺的,你对等 2000 多亿,那我也 2000 多亿,这一次,现在也就是前天,前天 10 号Trump,这次是再加倍。 OK 加到 5000 亿,所以基本上是全部都加关税了,基本上 90\% 以上都加了关税了。



王孟源 38:53 

你这个时候再做已经晚了,浪费了从去年8月到现在5月, 9 个月的时间,而且这期间让美国人赚了便宜,因为他加的关税比你高,而且他在谈判的时候趾高气扬的,这基本上最重要的是浪费了这 9 个月的时间,而且这个是可以预见的。



史东 39:14 

就是说他们在做这个工作,作为一个智库的参与者、研究者他们的任务就是要提出一个很客观的、很现实的、真正的所谓的 一个 Picture、 一个图画,对不对?今天你讲的就是他没有提出这个非常真正的、真实的、而且残酷的这个图画。



王孟源 39:40 

刘鹤从来没有在美国长期住过,习近平从来没有在美国长期住过。你不可能要求他们对美国有这么深刻的了解。但是你中国有 13 多亿人,你总是要有一些智库,里面总是要有些人能够懂美国内部的结构,它的那个作用,各方作用的那个机制跟方向。没有!



很明显的,从这个结果来看,是没有人出来说我们必须要对等,必须要在他们已经超过我们能够加关税的范围之后,我们就必须要打击 GM 跟宝洁这些公司。现在我第一次看到中国国内的人提出这个,是昨天在观察者网上看到金灿荣教授的文章,他才提到说这是我们的手中的王牌,我一年多前我已经说这必须要做了,结果他拖了一年多,现在才开始讨论。这已经就是你现在解决,已经是最好,也就是 2019 年年底解决,很可能要到 2020 年年初解决,但是你这个,你这个解决的和平顶多就是到 Trump 任期结束,你这个没有多少时间。



史东 41:00 

我顺便在这个项目,我和观众交代一下,就是你刚提到金灿荣的三个主张。金灿荣是人民大学国际关系学院的副院长,大家都知道他的三个,他说三张王牌,他这第一个是禁止稀土出口美国,第二个是中国有2万亿的美国国债,可以在上面做文章。第三个他说美国公司在中国的市场的利益。其实你特别点到了这一点。



王孟源 41:30 

他说最后一张是大王牌,这是对的。我也一直说这才是正确的。至于前面两张王牌,我在我的博客里面也有讨论过,我为什么他们没有什么作用,我认为没什么作用。第一个中国的稀土产量高,是因为开采稀土是一个高污染产业,并不是因为中国的存量占世界的90\%。OK,中国的产量占世界90\%,因为在美国它的这个高污染产业的成本高,事实上美国的存量非常高。OK,你如果他这,你如果禁止出口,不但他们有很多已经精炼过的那个库存,如果他要重启这些工业,也就是三五年的事情,这没有什么太大的。对他们的损失多少?几亿美元。有什么了不起?不痛不痒。至于那个他们的国债,这也是一般人常常会提起来说中国可以打的牌。我也一再地解释,这种金融的东西,只要美元还是国际的储备货币,没有什么不能够用印钞票解决的。你不买我的国债没关系,美联储会来买,反正他就印钞票来买,对不对?这个我本身是做金融的,我完全知道一点用都没有,你除非是在一个很大的金融危机,像 2008 年那种金融危机,它这种有象征性的心理因素可以把整个市场的那个心理完全打垮,那才有点



史东 43:10 

你说中国不像 2008 那样子一样,而是购入美国的国债。



王孟源 43:13 

美国没有 2008 年那样的那个信心危机,对不对?所以你现在再去卖那个卖他的国债,事实上你自己损失可能还比他大。损失大,因为他大不了就是再打开印钞机,对不对?他那个这叫做 QE Quantitative easing,很简单的事情。对,真正的办法就是,在中国的美商,事实上除了几个芯片之外,没有一个美商是不能替代的。中国发展到现在这个地步,基本上每一个工业都可以自主,事实上反而有些工业可能是好处,比如说你把那个GM打下去以后,得利的除了欧洲厂商之外,还有就是那个中国国内的厂商,对不对?是那宝洁这种更加是没有什么问题,那个化妆品这种东西就基本上就是纯粹是卖品牌的,没有什么实质的内容,没有技术内容。所以我从去年一年多来一直强调这是不但是适合的反应手段,而且是一个有利无害的手段。OK,美国即使没有这个贸易战,你都要考虑是否要做;现在美国把这个借口双手捧上来给你,你当面犹豫着不做,真是搞不懂。



史东 44:51 

我觉得我我们今天像看看一个人下棋,中国和美国下棋,如果中国出手打其美国公司在中国市场的利润的话,美国会有什么反应?



王孟源 45:02 

他能怎么反应?他根本不让中国公司进去?我记得很清楚的是大概六七年前中国有一个公司要去买一个 Rebar factory,你知道 Rebar 是什么吗?就是钢筋,就是那个我们那个钢筋混凝土里面那个钢筋,这是这算哪门子高科技?他都不让他买。你的美国这样处处防着。它的结果就是中国没有。在美国没有什么资产可以当人质的,但是反过来美国有一大堆资产在中国是可以当人质。



史东 45:43 

我的意思,倒并不是说像一对一的这种,美国也可以向中国,譬如说你刚刚讲的有关于收税的事情,有关于关税的事情,所以说相对来讲你要做关税战争的话,美国占上风这个事情就是中国占上风,但是美国是不是也有像中国类似的这种?这样我就不在这个环境里跟你打,我在另外一个地方来来照顾你。



王孟源 46:11 

他可以继续地再扯下去,但是每扯一步他都是要被打痛的,对对,对不对?那你那个 4: 1 的关税,现在他那个大豆的农民,美国的农民在 30 年前 80 年代的时候走,经过十年非常痛苦的职级有很多,有一个有很大的比例破产了,那破产之后就被公司收购,就变成财团化。那现在在过去这几年,又有一波破产潮,如果这个再拖一两年的话,很可能规模会比 80 年代的破产潮还大。那但是问题是你说这些农业州是放的铁票区,但实际上美国的农民的数目已经很少了,他一个农民可以种几百亩,甚至上千亩的土地,你真的真正数一数,他们的人头没有多少个,你再怎么打击他们也只不过是在外面叫苦,他们可能会真的破产,但是会让 Trump 失去选举吗?也不一定,因为他这个是政治正确的。



史东 47:26 

换句话就是说,农民并不一进来是美国的痛处,美国企业在中国的盈利才是最大的痛,才是真正的痛处。



王孟源 47:34 

对,因为这些农民,他们在政治上选影响选举的能力还是比不上那些大企业。



史东 47:43 

微乎其微。可以这么说吗?哈哈哈,和这些大公司来讲是微乎其微的对不对?来相较的话。



王孟源 47:51 

如果宝洁一下子损失了全世界 1/ 3 的市场,你想它这个股票会怎么样?对不对?



史东 47:58 

对对,有一个现象,我注意到有不少其他的评论,对于中国和美国这次贸易谈判受挫的事情,他们预测着中国和美国的关系从此开始将会分道扬镳,就是中国和美国的关系将会重新的洗牌,这个洗牌就是从上到下,从头到尾的重新洗牌。你的看法怎么样?



王孟源 48:27 

这我刚刚已经解释过了,从 11 年前开始就已经很明显。



史东 48:32 

就已经在洗牌了,是吗?



王孟源 48:34 

2008 年金融危机的时候,美国人就已经开始积极地做内部检讨了。嗯,Joseph Nye写的那些论文,难道中国都没有人在看吗?



史东 48:47 

所以说这个洗牌已经从美国开始了。中国还……



王孟源 48:51 

美国决定要扼杀你的,他只不过是因为那个金融危机,所以他自顾不暇,所以在 2011 年才开始动手。OK?你如果看那个什么Pivot to Asia 什么都是 2011 年左右开始动手。嗯,那动手还没有结果呢,就 Trump 就当选了,当选以后又换了一套来人马,换了一套手段来。



王孟源 49:15 

这个是我觉得这种事情已经是 11 年前的旧闻了,我现在还在讨论这个没有什么意义嘛?这个唯一的真正的意义是这次贸易战谈判,他们过去这一年中方真的是判断失准,就是过于乐观,没有真正没有像我所说的去年一再的强调这个是,这是生死存亡的斗争,必须要没有什么退让的余地。他没有认识到这一点,那这一次被打痛之后,或许中方,吸取这个教训,然后会有真正比较合适的应对。那这是这一次这个礼拜所出现这件事情的真正的长期的意义,要不然这个所谓分道扬镳,从 2008 年开始就已经是那个, the writing is on the wall, i mean。美国的有关讨论这种事情和论文是非常公开的,中国的那些智库的人在干什么啊?



史东 50:23 

还有一种一厢情愿。



王孟源 50:27 

一厢情愿!对,就是他们写学术论文,就说“照利润来看的话,就算我们谈判让他们全赢,我们不赢不输,我们也还是可以活得下去”,顶多也就是这样的,根本就不是这么回事!人家根本就不是要来跟你共赢,也不是要来跟你抢劫,人家是要来跟谋杀。你的意义是因为这个,不是因为世界的利润是零和游戏。经济本身就是加成的,OK,大家合作的话就会有财富的创出,所以永远不是零和的游戏。但是霸权政治的权力是零和游戏。如果中国在世界上的那个话语权增加了,美国的话语权就必须降低,这是零和的,对不对?所以就是美国,美国对中国的忌惮,是政治方面的忌惮,不是经济上的忌惮。他们如果一直中方在过去,一直到过去这一年,甚至到这个月,甚至到这个礼拜,我看到的文章还是在讨论经济上的双赢,怎么样创造最大的财富,就是他们还认为说这是美国单方财富最大化跟世界财富最大化之间的争执,不是的,美国不是这样想的。



史东 52:03 

所以说我想从另外一个角度来谈,同样的,你讲这个事情,你是不是中国政府过于低估了他对美国政府造成的压力?



王孟源 52:16 

我想中国还有一大票人,尤其在这个智库里面不了解这种霸权红利的零和本质,他们不了解这是你死我活的斗争。像这样子你在基本这个基本的定调就错误的话,你如果只是从经济层面去看问题,你这个永远都不是零和的,永远都是合作就互利双赢,他们的思路就永远都是靠在那边,那就不能够理解为什么美国人要这样做。我刚刚提到prognosis,他们错就是diagnosis,就是错的,所以 prognosis 就错得更离谱。



史东 53:03 

所以说诊断错误所以是开药方,也开错了就是开药方,他自然就是错的



王孟源 53:07 

对,你在我昨天晚上回去看我去年写的东西,完全就预言了一切的发展,这一年来一年多来的发展。这个他们就是有盲点呐。哈哈哈。



史东 53:24 

对,我个人觉得我这些你提这个盲点我都同意。我个人觉得这个盲点存在有一个,有几个元素。从中国的这个角度来,第一个中国有一点太过于乐观,还是我刚刚提到了,他不知道从美国的角度看,中国已经对美国造成了多大的威胁,这是第一点。第二点就是中国还希望也是过于乐观,还希望有一个机会可以跟美国创造一个所谓的双赢的局面,对不对?



王孟源 53:55 

他们就认为现代经济就是分工合作。第二我这边讲一下这个市场经济,还有这个经济的发展基本上创造财富最主要的因素就是分工合作。OK,你那个分工之后,大家就可以专注在一个很小的领域里面,然后精益求精。对,这是绝大多数市场效率、经济效率的来源,改进的来源。但是你如果只从这个角度,只从经济的角度来看大国外交,你你就会有错误的结论,因为是中国……



史东 54:40 

其实这种现象中国人应该很容易理解。中国人有一句很有名的话就是卧榻之侧,岂容他人酣睡?哈哈哈,不但睡觉,还打呼了。



王孟源 54:50 

哈哈哈,这还不是心理问题,这是真正的真金白银呐。因为你那些美国的政经权贵到全世界搜刮,靠的就是美国的治理权。世界治理权,嗯,那你这个中国兴起以后,已经很明显地在侵蚀美国的世界治理。



史东 55:13 

所以这也是我觉得他们误判,所谓他们对北京误判,是他们低估了自己对美国造成的威胁性,对不对?



王孟源 55:20 

对,然后他们先想的,我们只不过是要自己富起来,不威胁你,因为你也可以跟着富,大家都可以跟着富。但是这是经济,就是我刚才一直讲的,经济上不是零和的,是可以合作创造新财富的。但是在政治权力上,国际政治权力上是零和的,绝对。



史东 55:40 

是零和的,绝对是零和的。我想讲到这,嗯,我把话题稍微往回拉一点,稍微谈得比较微观一点,我想谈谈你,对于刘鹤,他谈他在记者招待会之中所谈到的美国中美团贸易谈判的时候的所谓的,我把它叫做三座大山,它并没有用这个词。第一个他说是关税是否取消,是个issue。第二个他说采购的数字有多少,换句话说采购的量有多大,这是第二个issue。第三个是文本的是否平衡。对于他讲的这三点你有什么反应吗?



王孟源 56:21 

我觉得这是细节,就基本上刘鹤所扮演的角色就是在中方是等同于美方的 Lighthizer,就是他是负责谈判细节跟文字的,这个大方向不是他们,不是由他们的个人。对,所以你看他所讨论的这些事情也都是细节。那这个细节,有几个可能?第一个是他知道明知这个是细节,不是真正的症结所在,他还是讲出来,因为你公开讲的话不是不一定,不一定是完全准确的,对不对?那第二个是也有可能是因为他的确这就是他的权责所在,所以他的他也就只讲这些事情。但是你不能够因为他公开的讲了,讲了这三点,就把这 三点当做是唯一的……



史东 57:13 

真正的症结



王孟源 57:14 

对,不是真正的症结,因为它很明显不是真正的症结,真正的症结在于美国要中国做体制改变,到它的那个竞争能力降低,中国绝对不愿意。OK,所以中国的中国有两方面,第一方面是让利,这一点是他们在做的。但是我从去年开始一再地讲让利,光是让利是不够的,因为美国自己很明显的已经知道光是让利不够的,就是说这一次不是来抢劫,我一再讲不是来抢劫,而是来谋杀。OK,你能够唯一让能谋杀犯能够知难而退,你要把它打痛,先打痛它才行,对不对?你不能够说你要来谋杀我,我口袋里面的这个皮包都给你,他本来就是要拿你的皮包,只不过是要杀了你,又得拿你的皮包。



史东 58:14 

对,这个比喻很好,哈哈。



王孟源 58:17 

他必须要能够把他打痛。



史东 58:18 

他的最终目的并不是在谋财,而是在害命,对,对不对?所以说你要从害命这个角度去看他,去解释这件事情来做出反应。我想对我来讲孟源最大的一个收获。今天我们的谈话就是我们在观念上不能再有双赢的观念,这是一个你死我活的一些情况,对,对不对?这个双赢的观念。



王孟源 58:50 

你跟欧洲可以谈,你跟日本都可以谈,你跟澳洲都可以谈,日本跟澳洲已经是美国最忠实的。但是你跟他们都还可以谈双赢,但是跟美国不能谈。为什么呢?现任霸主的红利是美国独家专有的,他之所以要打你,就是为了保障这个零和的那个红利。



史东 59:16 

我们谈了很多,我想最后一个问题,谈到这,我想很多观众在看这个节目,在看我们的谈话,一定会有一个疑问。这个疑问我记得我以前好像也问过你,就是中国和美国之间发生战争的可能性有多大。



王孟源 59:34 

我个人认为不大。那为什么呢?习近平未雨绸缪,他从那个一上台就积极军改,他这个,他的那个战略眼光是相当好的。这个他事先要知道,你要有一个稳定的中美关系,要能够不让美国冒进,你必须要本身有足够的吓阻力,那这个吓阻的力量,这同样的我现在也是讲先讲体,然后再讲那个细节。它的这个本体长期的因素是中国在军工跟军改,尤其是训练跟组织方面在过去这 5 年有很大的进步。OK,我们一般人只会注意到那个新的武器,实际上怎样使用这些武器,这些组织才是更重要的事情。因为我们做军事研究的人都知道,武器决定你战力的上限,但是人士兵跟组织决定你战力的下限。



王孟源 01:00:40 

就是你可以有极很好的武器,但是上限很高,但是你如果那个士兵的素质不够的话,那你还是一样战五渣。那中国在训练教程组织已经完全现代化了,基本上就是跟上美国,在过去这 5 年。那在武器上面,也基本上……美国它的主要作战武器是航母,那他打击航母已经有两个新的武器,一个是弹道飞弹,另外一个是他的新的隐形隐身战机。有了这两样之后,基本上美国的航母要在像二十几年前、 1996 年那样到台海附近来耀武扬威,基本上已经是不可能的,所以美国如果要跟中国打呢?他还是可以赢的,但是这必须是一个全球战争,而且是长期的贸易封锁,那这个就是,这是世界大战的格局了,就是除了没有用核弹之外,基本上就是世界战争。我觉得美国不会,尤其是Trump,我们现在回头谈谈Trump, Trump 其实色厉内荏,他就是一个商人,地产商,他本身是啊,不习惯他的思维模式,没有使用武力。所以我像这次他这个对伊朗,这也是那个 Bolton在背后怂恿的。事实上Bolton连在委内瑞拉都没办法动手,对人家的委内瑞拉拖了这么久,近在咫尺,而且还有一大堆带路的,他都没办法下定决心动手。这你说要真正跟中国开打,而且开打不是像二十几年前那样,派两艘航母到台海绕一圈就结束了,因为你的航母现在已经没办法进入关岛,你这个就是 3000 公里外之内,没办法进入的话你的那个打击半径只有 1000 多公里,那你怎么打?哈哈哈,那唯一的办法就是中国的那个商船要到印度洋,要到大西洋,这些它就鞭长莫及,那美国就可以很轻松地利用它的全球的基地,还有那个航母来封锁你的贸易,但是这个这对美国没有利益,因为你这样中国的贸易损失,贸易是两边在做的,你拦截一个商船,这个商船从中国出来要到另外一个地方去,另外那个地方答不答应?对不对?所以兹事体大,我觉得习近平已经在军事上做到很好,就是美国已经没办法打一个中小型的局部战争,快速的解决中国。那么这么一来这个大型战争兹事体大,它可能这个几率就变得非常的小。



史东 01:03:47 

那么换句话说,如果我们的前提是正确的话,换句话说我们的诊断是正确的话,嗯,中国和美国之间只有靠现在这个方式来周旋下去了。



王孟源 01:04:01 

就是在未来几年,当然Trump能不能当再度当选还是一个很大的疑问。我个人认为。



史东 01:04:08 

我的意思是即使是Trump落选,下一个总统上来,对不对?



王孟源 01:04:12 

下一个总统一样,



史东 01:04:14 

这个大环境没有变。



王孟源 01:04:16 

没有变,没有变,他继续进逼,这个会逼到什么程度?逼到中国的国力全面凌驾美国。但是这个我们不要太乐观,这至少要到 2035 年才会那么明显化,就说至少还有十五六年的时间,那这段时间里面,你可能换三四任的总统,但是他们绝对的、政策一定都是同一个方向,就是打击中国,对于打击的手段那就会有各式各样不同的方法。但是Trump其实是很帮忙的,因为它第一个就是……欧盟已经很明显地是要跟美国联手,在 2016 年的时候已经很明显地要签TTIP,然后跟美国合作把 WTO 的章程改掉,结果他们一下撕破脸了。现在欧盟反过来置身事外。我不是说中国要跟欧盟合作,这是不可能的,但是因为Trump的关系使欧盟置身事外,成为一个旁观者,这已经是对中国的大大的机会。对,现在连日本都已经知道 Trump 不可靠,他们愿意跟中国重新接触,重新合作,因为中国的态度就是你只要愿意来合作双赢,我永远都在等你。嗯,这个态度是很好的,他对欧洲work,对日本work,那你连澳洲现在这个月底再跟,再过一个礼拜要选举,选举完之后他如果是 Labor 上台的话,说不定就会有政策上的改变,所以很可能连澳洲都会work,但是唯一不可能work就是美国。



史东 01:06:03 

对,因为美国的要求,美国的目的不一样,所以美国不吃这一套。



王孟源 01:06:07 

因为世界霸主只有一个,那就是美国对。



史东 01:06:12 

孟源,这我们我要麻烦你做一件可能不太容易做到的事情,就是为我们今天的讲话做一个结论。



王孟源 01:06:21 

我的结论是,这个礼拜贸易战所发生的事情,是合理的,甚至是必然的。是事实上我去年一年多前就已经预见了这些,我觉得它真正的意义在于对中国的决策阶层有一个教育性的意义,他们这在这之后不会再有什么幻想,能够彻底了解中美冲突的本质所在,然后希望他们在未来能够有比较合适的对应。



史东 01:07:01 

顺便在这儿跟您也跟所有的观众打声招呼,就是孟源他在今天节目之中所提到的几个其在之前写的一些文章,我我们会把这些 link 全部在这个节目之下会贴上,这样的话观众就可以很快的很自然的走到孟源的文章之中。去去去,我想。



王孟源 01:07:26 

我想8月的那份文章叫做再谈美国贸易,中美贸易战,然后 4 月份的那篇文章是在谈Trump的执政理念,这样子就是基本上在我那个联合报的博客上面可以找得到。



史东 01:07:44 

OK,这样这也是一种方法。另一种方法也可以直接到我们这个这 video 的下面的这个链接,就可以直接连上了,就直接到你的部落格去了。非常感谢孟源,然后我们当然是保持联系,联系。同时我在这里祝你这个一切顺利,一切顺心,好不好?谢谢,谢谢,好,谢谢,好。



王孟源 01:08:06 

再见。再见。



编注:



文中提到的链接如下:



【美國】【經濟】再談Trump的政策



https://wmyblog.site/html/111444506.html



【戰略】【經濟】再談中美貿易戰



https://wmyblog.site/html/114356703.html \twocolumn[\begin{@twocolumnfalse}
\section{香港动乱、}
\subsection{20190812}
\end{@twocolumnfalse}]史东(0:00)

各位朋友你好,我是史东。香港的这件事情,香港的动乱在最近一个礼拜、两个礼拜甚至三个礼拜以来,一直都是揪在我们的心里面。那么今天的节目中,我们还是会继续的跟你谈谈有关于香港的一些事情。不过角度可能会有一点不一样。今天的节目中为您请到的是我们节目中的好朋友,那就是王孟源王先生。他对于香港动乱这件事情,有一些属于他自己的观察,那么或者是直接或者说间接的和香港的这些情况都有关系。他今天想谈的几个要点,第一个是国泰航空,第二个是汇丰银行,第三个是一些有名的人,他们在国外的账户的情况,第四个是载人登月的事情,最后一个大家谈的是CIA在叙利亚的一些情况。当然,这些事情都是和香港有直接有间接的关系。首先我们把王孟源王先生请入我们的画面之中,孟源非常谢谢,非常欢迎。



王孟源(1:07)

一直都是我的荣幸,今天非常高兴能够再上你的节目。我想跟听众先解释一下,这是我们第一次事先交换大纲,所以我一向都是随意发挥。不过这次因为你问我,事先问我,大概会谈些什么,所以我写下五个点,希望我能够照这些大纲来讲。



史东(1:29)

不是,我是我这个人做事即兴的成分很大,所以说即兴对我来讲是一种快乐,即兴方式的这种是一种快乐,也是一种挑战。



王孟源(1:42)

很高兴你能容许一向即兴而作的我。此外在我们开始谈正题之前,我想做一个个人的宣布,就是有些读者因为我在半年多前提到,我准备回台湾找工作,所以他们一直会问说我的工作找的怎么样了。我大致解释一下,我找工作并不是因为我想要开辟一个新的生涯,而是在家里照顾我妈妈,然后每天去看望我医院里面的爸爸的时候,我希望还是能够有一些有意义的事情来充实我的生活。那这里面就有一个很大的限制,就是因为医院每天只有早上一个小时的访问时间。那我就是固定每个每天上午必须要开车带我妈妈去。那这样一来,我觉得基本上只有两种工作可以做,一个就是只上下午班的那种,或者是我可以从家里工作的。后来有有很多读者很热情的贡献了他们的connection ideas。那最后是我几个大学老同学把我牵线牵着过去。那目前我原本我已经接受了一个offer,就是要在我家附近最近的那个国立大学开课。但是很不幸的,我儿子今年没有申请到如意的学校,所以他准备要明年重新申请。那这样一来,我就晚了一年。所以我想那个教职大概还是会等着我。所以大概明年这个时候我会回台湾,然后一方面在家里照顾双亲,另外下午就会到附近的一家国立大学当金融系的教授,这是我目前的计划。不过我还没有确定,就是我那还有就是另外一个朋友,我原本是希望觉得智库的工作,台湾智库的工作,我可以在家里面做。反正我现在写博客基本上就是做智库的工作嘛,对不对?只不过是没有正式的管道。那后来我去看了一下,结果发现台湾的智库还是让马英九跟龙应台那些人把持这些人的意见跟我是,所以后来我就有一个朋友说他可以另外开一个智库。不过他今年还要去选举,他选举上了就不会有时间搞这些事情。如果他没选举上的话,他可能会跟我合伙开一个智库,那这是可能是在教职之外另外兼的。所以这个都还没有确定,就是因为还有一年嘛,所以这边提一提,让读者了解一下我目前的规划。



王孟源(4:43)

那我们现在可以言归正传来谈了香港的问题。我想大家都很关心这个香港的动乱,而且越闹越大。那我觉得,一般来说我看人口的分布,我通常是把它分成,很多事情不不一定永远都是这样。但是很多时候我遇到一个人口的特性的时候,通常是三分的,就是有一个极端,然后有一个相反的极端,然后中间有一个比较大的板块中间派。从香港来看,它也是有这样子的,一个是被英美洗脑的那一派,就是闹事的那一派,那另外一个极端是亲中派,那中间就有中间派,中间派通常是比较非政治性。他们也许是专业人士,像医生或者是或者是白领阶级日常的公务人员。他们平常没有注重这些事情,也没有太强烈的感觉。那我觉得香港他的被英美洗脑的那些闹事派,比一般的社会要多,就是一般的社会,也许这种极端人士大概是百分之二十,但是香港可能高达百分之,三十或甚至四十,就是他们或许是就是不管他们愿不愿意上街,但是他们其实是支持这个想法的。就是他们认为英国跟美国的宣传的那一套,西方式民主自由是真正的真理。中国所代表的就是刚好相反的,什么专制集权,压迫这种,就是这其实是美国在冷战时期发展出来的那一套宣传系统,用来瓦解苏联用的,后来也成功了嘛,对不对?但是香港的话,我觉得他是一个比较特别的社会。因为他的文化根底很差。你如果比如说你看那个媒体的话,台湾的媒体我已经是诟病已久。因为他的既不专业也不中立,更不诚实。但是事实上台湾的新闻业媒体会衰落到这个样子。十几二十年前,受香港人的影响,香港人是始作俑者。我想苹果日报,所以即使到现在,像台湾的媒体都还没有堕落到香港那样子。就是我这个所谓堕落是什么意思?就是自由时报或许是每天都需要必须要篡改事实,或者是假造假造论点。但是我觉得中国时报还没有到那个地步,我我有一篇专门的博文就讨论这些媒体他们的诚实的程度就是可靠的程度。我也谈了这些事情,但是在香港,不不管你是亲中派还是还是那个亲美派,他们他们的所谓的媒体就是撒谎,就是撒谎,经济就是他们的文化就是这样子。那这是很不幸的,因为我之所以会想到这一点,是因为在这个访问之前,你问我有关汇丰银行,我要讲什么?我们讨论下来你送我一篇香港的那个新闻截图那个什么,结果那个是我一看就觉得他是很明显是撒谎的那这就让我联想起来,香港的媒体文化相当的糟糕。那因为正因为这样子,所以他们一个没有一个很强的文化根基,所以特别容易受鼓动。



王孟源(8:59)

那另外一个最大的问题是香港在二十多年前回归中国之后,原本第一任的特首知道在那个环节必须要产业升级,因为香港传统上在从四九年一直到九七年,它所一直发展的经济根基就是作为大陆的门户嘛,对不对?因为它跟大陆有特别的联络关系。不管在文化服务业跟制造业都是这样子,一直到现在它还是一个物流的中心,整个广东地地区有很多那个集装箱是必须要从香港转运出口,空运的更多。但是当时的董建华就看出来,这个不能够长久依靠下去,因为大陆一定会逐渐发展嘛,那发展下去很快的,他们就会到处建机场,建港口,那到时候香港的这些传统的优势就不见了,所以不见了,你就必须要往前走一步。其实这他也不是独家的观点,那新加坡李家的那些,李家那父子两位也是早早就看到这样子。但是正因为香港回归中国之后,反而在往英美的民主制度上往前走了几步,这使得他的执行政效率立即大幅下降。大幅下降的结果,那就是董建华的那几个计划没有一个实现了。那就是如果你后来看到在九零年代末期,两千年初期,新加坡还建了一些晶圆厂,可能新加坡也一直都是国际上船舶服务业的领头羊之一。他在海水淡化场上的技术是跟以色列并列的头两名。在这些方面,你可以看到新加坡都有高科技,可以依赖,就是实体经济可以依赖。那香港什么都没有,那么的二十多年下来,你既没有以前的那些转口贸易的的利润,然后又没有自己产业升级所产生的高科技实业发展。那这时候他的经济是靠什么?他的经济就变成靠中国的让利。那这个让利是怎么让利?你说来说去就是靠制度的不同,就是以往是以往也是靠制度的不同。不过靠的是比较直接的那种低阶的制造业转口跟服务业的的转口。但是在最近二十年,他这些东西中国已经发展出来,赶上了他可以直接替代了。那么香港的经济就越来越依据虚拟经济,就是金融、会计,法律服务、公证这些东西;就是比较高端,比较虚拟的服务业。好,那这些服务业高端服务业,高端的虚拟服务业,它有一个共同的特征,就是他所依赖的根本在于香港的法治是独立于大陆的。要不然你没有理由在香港做嘛,对不对?香港又挤又贵。我之所以会想要提到那个大纲里面会提到有些名人的私人账户,就是想要用这个主题的例子。



王孟源(12:37)

所以请容许我自由发挥一下,这个这生意叫做私人银行,这是一个很典型的香港的生意,私人银行是什么?就是你存一大笔钱去给他。但是在我们这个金融的专业里面,同样这笔生意又分成两类。看起来好像一样嘛,你把钱拿去给那个银行,它帮你管理,给你放到mutual fund或者是hedge fund。他们有的是自己的fund,有些是外面的人的fund,他帮你保管,然后希望帮你投资创造利息。可是,这个生意在行业里面却分为两个完全不同的部门:一个叫做private banking,一个叫做asset management。这个的差别在什么?只有一点:asset management 本身要跟IRS 报告的,而private banking是保密的。就是你们传统想象中瑞士银行是搞什么,就是那种银行机密。所以他的这种最大的客户,第一个是要逃税,第二个就是贪官污吏的脏钱。然后一般人又不了解,世界上搞私人银行,最保密对私人银行最友善的那个辖区是谁?前三名大家都没听过,不是瑞士,不是Cayman island,不是Bahamas,这些都是都是被英美的媒体拿出来当锣鼓敲打的。实际上他们不是最严厉,最糟糕的。



史东(14:37)

就换句话说,你就说这些这些地方都是外行人讲的话



王孟源(14:41)

外行人讲的。内行人都知道,保密法律最宽松的,也就是对这个私人银行最宽松,然后对他们的保密是保护的最严格的,头一位是美国。但是问题是,只要你是美国公民,你就不能够用他的,他们的私人银行只供外国人专用,不是给美国人用。所以美国人要用就得要跑到加勒比海的Cayman island 跟Bahamas去。这是为什么Cayman island跟Bahamas才会有名?因为对美国人来说,他没办法用本地的私人银行。



史东(15:27)

这可不可以把它解释成美国政府希望吸收外来的脏钱的一种方式?



王孟源(15:33)

对,因为你吸收脏钱的话,这个私人银行跟asset management虽然只有这么一个小小的差别,但是在实际运作上,asset management是在竞争你这个投资报酬率哪一个高嘛,对不对?私人银行的投资报酬率(即使)是零,人家都不在乎,只要能够保护这个本金。那这个生意是不是一本万利?是,当然了,对不对?因为你的钱拿了来,你只要付出零利率,人家也一样愿意存在你这边。那你可以转过头去就拿4\% 5\%,这个价差不是很好赚吗?所以我说排行第一的是美国,排行第二的是谁?就是香港。然后第三个是新加坡。我很抱歉在这边公开地讲这些行内的机密,等于是教贪官污吏怎么藏钱。我事先其实考虑了一下要不要这样讲。



史东(16:32)

你应该写一本书,你应该写一本书,孟源,你说《贪官污吏须知》。



王孟源(16:38)

所以十几年前当陈水扁的事情闹出来的时候,就有人说他外行,因为他的钱是偷偷弄到南太平洋,然后从南太平洋弄到瑞士银行的。其实瑞士银行没什么不好,我刚刚说(去)瑞士银行不行,不是说瑞士银行不行,瑞士银行还是最老资格的。我说的是瑞士(这个)国家。因为他在欧盟旁边,他吸收的客户大部分都是欧盟的。那欧盟在近年来非常看不顺眼他们的有钱人富豪去逃税,所以跟美国联手对瑞士施压。那瑞士是一个正当的国家嘛,对不对?他这个承受不了,而且他被整个欧盟环绕着。虽然不是欧盟共同市场的一个成员,但是他跟欧盟有几百个条约,所以他没办法拒绝这些要求。所以他们最近这十几年来,真的是,他们的那个银行保密法已经千疮百孔。所以真抱歉,我这个又是泄露给贪官污吏,你千万不要飞到日内瓦或Zurich去藏钱。最好的地方,其实是哪里?就是纽约的瑞士银行分部。你找瑞士银行的时候,还必须不能够找我以前干过的那两家,我干以前干过的那两家是谁?就是UBS跟credit suisse,因为这两家是所谓的全职银行。它除了private banking 以外,其实它的主要生意是investment banking, commercial banking 跟asset management。他们的private banking 只是小小的一个附带的地方,所以他们完全承受不起像IRS这种机构来跟他要机密。因为你的私人银行如果出了什么事情,他可以一下就罚你几十亿美金。



史东(19:05)

所以说,我这个知道就是你做一个private banking,你没有几把刷子是没办法做的了。



王孟源(19:12)

你这个private banking,要做private banking 就必须是专业的private banking。就是如果当地的那个政治跟法律变了,你可以马上走路(跑路)的。让客户为先,你的这个支部不重要,UBS跟credit suisse没办法嘛,他们在美国的支部不可能关掉的,对不对?所以当初人家说陈水扁是外行,外行就是因为他是找这两家大银行。真正内行的会去找只做private banking。那瑞士只做private banking 的最有名的是哪一家?最老牌的就叫做Julius baer。你大概没有听说过。但是你如果是亿万富翁,就是Billionaire,一定听说过Julius baer,。我如果有一个biggest new friend 要藏钱的话,问我说去哪里藏钱,我一定说你的钱是哪里来的。如果是在美国之外的话,你就到纽约的Julius baer。你如果是在美国国内的话,你就飞到香港去找香港的Julius baer。然后在陈水扁那一段时间在闹的时候,刚好我有一个朋友在Julius baer做事。有一天下午我们在聊天喝茶,聊天的时候,他说你是台湾人吗?我说是,他说你们台湾有没有一个总统,前总统姓李的,我说有,呵呵。他说我们今天Julius baer来了一家的人,他们都是这个李总统的家人,他们那个好像李总统的身体不好,他们准备把那个藏的钱分一分。那个我说:“李登辉,这下子严重了。”



史东(21:06)

你这是什么时候的事情?多长时间以前的事情?



王孟源(21:08)

就是十五年前嘛,陈水扁任内,然后那一件事在闹的时候,2005,2006年的时候,或许是再早一两年吧,反正就是在那段时间。然后我说,这下这个是很严重的内幕消息。其实我四年前在我博客已经提到了,但是没有详细讲。怎么说,愿闻其详。然后他就开始讲讲讲讲。哎,我一听,这个家人是李庆安跟李庆华嘛,所以他搞错了,不是总统,而是总理。就是......(笑声),不管怎么样,你可以看得出李焕他就比陈水扁要高明很多,他找的是全世界最好的藏钱所在。那这个我那个朋友跟我讲,负责你家账户的那个主管,就是香港人,就是从香港的支部,然后转到纽约支部来的。因为你在香港的服务,一些客户当然是可以。但是这些客户还是不太放心,他觉得太近了,像他们不了解香港的法制是独立的,或者是他们就是担心现在会有这个引渡条约这种事情。所以后来他们就把这个香港人从香港调到纽约,负专门负责跟台湾或者中国,或者是其他华人的客户,他的客户当然还包括一大堆台湾的富商,反正基本上都是那些那些财富都是不愿意露白的财富嘛。不愿意露白的财富,不能够露白的财富。 然后,他们也不在乎那个投资报酬率,反正就是机密为要。



王孟源(22:40)

我扯了这么远,其实就是跟你讲,这个私人银行其实是香港的一大支柱。他这个一年几十亿美金这样进来的。那一般你看不到,看像那个我刚刚说的那个香港主管,他基本上就是负责招待这些客户嘛。对,像这样子一年也是几百万美金的赚。那这种人他在香港生活下去,就是要靠着法治独立。你如果有了现在的这个引渡条约,就完全威胁到他们的生计。所以,这为什么这次因为引渡条约所引起的这个暴乱,它的规模和程度会超过四五年前的占中,就在于这一点。我刚刚说过,你有一个极端派,另一个极端派,还有一大堆中间派。上一次占中的时候,这些中间派没有参与嘛,他们只是在旁观。那这一次这些中间派有的是会计师,有的是金融业者从业人员,他们有的是建筑师什么的,他们通通进去了。因为香港如果必须要尊重大陆的法律,像是藏钱,这个很简单嘛,你如果要,你如果香港去收大陆来的赃钱,然后大陆能够追溯的话,这还得了,这个行业全都跑光了,没有人会把钱拿到香港去嘛,对不对?同样的其他的生意也是一样。你为什么什么法律咨询,什么,会计这些东西。如果是大陆能够问责的话,人家选择的地方太多了,不需要到你香港去,这些都是生计问题。所以这一次会闹得这么厉害,就是因为我在两个多礼拜前看到有一堆银行业的白领阶级,金融业的也上街游行。我就知道这一次会闹的比上次占中厉害很多,要更为长久。至于像是亲美派的那些暴乱分子,在香港占的比例相当大。但是他们还不是多数,他们事实上上次占中也已经闹过了,这一次是因为有中间派的支持。那这些人基本上他们嗯没有理性的思考能力嘛,所以他们基本上是依习惯性或者是周边的回声,那这些中间派的也参与,那对他们来说就是很大的鼓励。



王孟源(25:18)

那当然有英美这次英国的媒体,还有美国的媒体,还有美国的官方非官方的那些颠覆工具,尤其是CIA,这个他们都参与了。你送给我的一份资料里面有一个文章说是因为香港引渡条约会威胁CIA在香港的在香港。当然CIA当然在香港有一个很大的站,所以CIA特别全力反扑,我觉得这是太高估CIA了,CIA要出去给人家搞颠覆,哪还需要自己先受威胁的,对不对?他们他们到处都,这就是他们的本行,他们到处都搞这个。你有没有威胁到他的自己的安全都是一样。你看像在syria, ISIS是怎么起来的?就是CIA支持起来了,这不是因为为了美国的利益,而是为了CIA自己的利益。因为他在那有人脉,有资源,有关系,那但是没有钱,就是国内不发经费,那不发经费,你把这个东西搞起来以后,经费就来了嘛,资源就来了,对不对?,所以一开始的时候,Obama说要去把isis 解决了,他送了国防部,,派了几个将军去。那将军第一步是先武装当地的武装分子,地方武装了。那结果这些这些武器到了那边,那些将军们也不晓得东南西北嘛,对不对?他们就问当地的那些CIA,CIA就说好吧,你把那些武器交给我,我发给这些武装分子。对,那个时候syria有一百多个两百多个武装部队,那其实都是isis 的名头。他然后他拿了那些武器以后,就转头去加入isis 去了,所以isis 才会有那么一大堆,每次都会有。那后来搞到一六年还是一七年,那个国防部发现不对劲了,我们的武器老是出现在我们的对手手里,这才改变做法。他有一段时间在Syria,那个pentagon跟CIA两头是对着干的,就是CIA仍然想要支持那些所谓的民族武装。但是国防部说我谁也不相信,我,只相信库德族人。所以后来isis 被剿灭,就是因为由国防部完全接手,他把那个所有的资源都投到库德族人这个人里面去。然后刚好那个叙利亚政府又接到收到俄国的资源,这边两头夹击才把ISIS解决了嘛,对不对?那对CIA来说,他不管世界的利益,他也不管美国国家的利益,他只管自己的利益。他的自己利益就是天下越乱越好,越乱的话,他们的工作越多,资源越多,自由越大。所以在香港绝对是他们有参与的。那参与的话,我以前也在我的博客上解释,他们很简单的就是最基本的一个就是找钱。那在这一次上面大概不需要,因为香港有很多富豪主动的掏钱。那其次就是发教程告诉这些暴徒怎么搞。



史东(29:26)

你说交战守则是吗?



王孟源(29:29)

交战守则,对。他们那个当初在乌克兰的时候,乌克兰是二零一四年搞出来的,人家发现那个用乌克兰文传播的一大堆那个传单,交战守则跟两三年前在阿拉伯之春用阿拉伯文文写的完全一样,只不过是翻译过去。那你想谁在两这两个地方都有共同的利益,想要搞颠覆,然后又有这个资源能够搞翻译的,除了CIA以外,我很难想到。



史东(30:07)

因为你从他们的行为的模式看来也是非常相似的嘛。



王孟源(30:11)

对,完全一模一样。那个后来是一个美国记者把他揭发出来,揭发出来以后,没有一个美国媒体敢登。后来一个是一个小网站,小网站把它登出来了。那俄国人那个RT就是russian today 才把它转发出来。所以像是在这个国际话语权的斗争中,RT是一个很重要的玩家。那当然他最近这几年也因为有了成就,所以有了成果,所以才变成英美媒体共同打击的对象。所以说人话就是RT 跟Al Jazeera。RT或许还会为了服务俄国的利益,所以扭曲事实。Al Jazeera可是真正都是趁那个英国的媒体。因为layoff 很多,就是解雇了很多,所以他们雇了一大票人,全世界去报,完全靠他们的良心来报。就是除了半岛阿拉伯国家这些内务事,他们可能会有一些立场之外,否则他们别的国家那个不一定报的对,他至少是没有存心去污蔑什么人的。但是他一样也被英美的媒体围起来打嘛,对不对?就是因为他们打破了英美媒体传统在英文话语权上的独占。那你看像这次香港的暴动,他们这个话语权的独占怎么行使?最基本的一个形式就是双方都在讲理的时候,他只报那些亲美派的。那亲中派的要讲理,他们根本就不管你。但是双方打起来的时候,他们就只照亲中派的这种选择性报道是最基本的。所以我现在担心的是,接下来当初在乌克兰,后来他乌克兰的那个暴动,2014年会很快的激化,有一个很大的原因是,那些示威者开始受挨冷枪了,死了十几个,后来俄国人因为他们有监听嘛,俄国的KGB有监听,他们监听到两个欧盟的外交官之间的电话,然后把它放到youtube上面去了。那个这个电话是说,据我们的消息,这些冷枪发冷枪的是CIA安排的,打自己人的。



史东(32:31)

CIA安排打谁,自己人是谁?



王孟源(32:36)

CIA安排让那个反对派,打自己的反对派,这样来栽赃给政府。



史东(32:47)

这个是一个这个很常见的一种,他们叫false flag 嘛。就是制造一些假的事态来来转嫁这个罪名



王孟源(32:54)

在香港还没有发生。不过我很担心,因为也许他不是那么容易,因为这种狙击枪在乌克兰很容易弄到。但是在香港大概不是那么容易,所以我这个就绕回来了,说那个在处理这件事情上,我对中国政府的建议有短期跟长期,小跟大两两个方面来看,短期跟小的来看,是你千万不要直接干预。从现在的这个势头来看,中国政府有这个智慧,因为目前这个暴动基本上就是地方暴动,你交给地方政府处理,你越是直接出手,越是给这些英美媒体闹大的机会。而且他们现在就是只等着你越闹越大。那你如果不闹大的话,他们会栽赃。栽赃的话,你如果比如说有武警进去的话,那一定会有人忽然就中了中那个,然后就洗不清了。对,即使你武警自己没有开枪,他们也会想办法栽你的赃,对不对?那所以你千万不要进去,进去了,就是帮忙他们栽赃。这有一个例外,有例外的情形就是如果他们把林郑逼下台了,那就只好直接预制管制,那这时候出手就必须要一块求稳,那直接坦克开进去了没办法。因为你这已经是颠覆了政府,那没有办法。那我觉得在短期来说,正确的做法,我给你的大纲说,我会提到国泰航空跟汇丰银行。那我先讲汇丰银行,这个汇丰银行的消息是刚好一个礼拜之前在八月五号消息传出,就是他们的总裁是个英国人,不然就辞职了。这个当然你也晓得在商界所谓的辞职就是被开除了,就莫名其妙的被开除了。开除的话非常的奇怪,因为他们的利润虽然不是大涨,就是不是全垒打,那至少是一个一垒吧。还不错,还成长了一些。然后,你这个董事长也没有说宣布什么新的战略方向,也不是什么大的这个策略上的不同意。那你后来那个《经济学人》去采访的时候我念给你听,这个董事长的回应是,the board had decided that a change was needed to make the most of the significant opportunities ahead of us. 他的回应就只有这么一句,董事会决定万人有利于我们未来赚钱的机会。这是非常奇怪的,我在过去这十几年二十几年,我还不记得有这么大的一家银行总裁是这么换的,就是没有任何理由就换



史东(36:24)

就是不着边际的一些话



王孟源(36:27)

不着边际这些话。就是他真实的理由不能讲。但是你知道美国的这个英美的这些大公司的总裁,他基本就相当于封建时代的男爵Baron,对不对?就是地方势力他们是很强的贵族,会互相保护的。从我过去二三十年在华尔街干活,从来没有看到一家汇丰这种前十名,全球前十名的大银行是这样子换领导人的。但是这背后有一个很好的理由,有些人是不是已经讲了好几个月,就是汇丰,我们半年多前不是谈过华为的事吗?对不对?美国想要陷害华为,基本上是司法部想要趁全国都在打华为,包括他们都在打华为的时候,落井下石,想要分杯羹,所以就起诉华为。他们起诉华为所用的那个证据所谓的证据,其实也就是欲加之罪,何患无辞,就是找茬。但是他总是要有一些物理证据,这个所谓的证据,是孟晚舟当初做presentation 的时候用的一个powerpoint slide。那这个问题就是这个powerpoint slide哪里来的?后来几个月后,人家发现是汇丰拿去交给司法部的,这是两三个月前出的事情。所以那时候就有人说中国政府应该出手惩治汇丰。等到这个汇丰出事以后,到了礼拜五,礼拜六,他们连汇丰的大中华区分部的总裁也辞职了,越来越明显。这显然是因为中国的事情,所以汇丰才会出这些人事变动,因为原先还可以说这或许是个巧合,但是你连怎么大中华部的总裁是第二个辞职的吗?那这很明显是因为中国业务的关系嘛,对不对?那整个事件就必须是跟一个中国业务的关系。那我刚刚提到那个香港的媒体上面的那些揣测我认为是不可信的。那个揣测是怎么说?他说是那个汇丰的第一大股东,是平安集团,就是平安保险集团。他说是因为平安保险集团施压,所以才不会这样,因为这是不可能的。你如果对那个全球的大银行有点概念,就知道这是不可能。因为汇丰所谓的第一大股东,他占股份是多少?百分之七,而且没有投票权。股票最重要分成两类,有收益权跟投票权。那收益权是一般的股票,有投票权的是优先的股票。那没有投票权的话,你根本就就是考验做不了人性上的决定了。平安连一席董事都没有,你说要施什么压,我见过很多那个很多英美的企业,它有百分之七十八十的股东,股权的股东。但是因为他们的投票权只有百分之四十五或者百分之四十九,他们就没办法把那个总裁搞下台。你就凭平安这个百分之七还没有投票权的股票,怎么可能把总裁搞下来,这不可能的。所以唯一的可能就是中国官方施压,直接跟他们讲,你们要么换人,要么你们就扫地出门,不用想做任何中国的生意。那我认为这个是正解。那如果这是正解的话,这是我第一次看到中国出手对外国企业做对等反应,合理对等的反应,我认为是一个非常让我乐观其成的正面发展。



史东(40:58)

谈到这儿,孟源我想有有有一个问题想插进来跟您探讨一下,就是关于这个中国平安银行还和这个汇丰银行的一个事情。在这个你刚刚指出来不是正确的报道里面,其中他们讲这个事情是个好消息。因为这样的话,就又切断了汇丰银行的资助这些动乱者的金元,就是他的财源。他说他断了黎智英的财源,这个可能吗?



王孟源(41:34)

我觉得不会。黎智英的财源是哪家银行?我觉得这些人不见得知道,我自己也不知道,但是黎智英不可能只跟一家银行打交道。这这一点我知道。如果黎智英拿不出钱来,CIA也会想办法拿别的到别的地方拿钱。然后黎智英也不是唯一在背后的富豪。你像很这背后有很多英国的钱,有英国企业的钱。另外一个企业就是国泰航空。国泰航空,他们基本整个企业文化就是我是英国人,我有优越感,你们中国什么东西都是落后的。所以这一次一出来,他们那些他们这个这些上等英国人手下的香港奴才们也都热烈参与,成千上万的参与。那现在那个中国民航局刚刚出手,我觉得目前出手还不太够,现在已经正式的出手的是他要求所有飞过中国领空的国泰航空,他的机组必须事先经过民航局认可。那你说这是不是一种形式上的?不是。昨天八月十一号,那个国泰从纽约飞香港的班机CX899,就因为没有通过,所以必须要在大阪迫降。



史东(42:55)

就是不能进入中国领空,是不是这个意思?



王孟源(42:58)

不能进。因为你如果不进入中国领空的话,他那个就超过747或者777直飞的距离,他那个飞机就没办法一次飞到香港去,那他只好在大阪迫降,然后加油以后再飞往香港。所以这个是认真的,不过我觉得还没有做到正确的执行点,就是你必须把他们打痛。我一直都说你跟英国美国这些强盗交易,没有什么善良的,他们是存心要把你搞死,不是为了要抢钱。那你只有把他们打痛,打痛是到什么地步?就是最起码CEO要换人,在汇丰银行,这是第一次。我看到他们中国出手回击,是真正打痛了,对不对?你那个CEO换人以后,新的CEO敢不敢再搞,绝对不敢,对不对?这种东西上行下效嘛,上梁不正下梁歪,一定他那个不止领导人一定有责任,而且他也是最好的杀鸡儆猴的办法。我觉得在国泰航空,最起码最起码要让他们的CEO,还有他们的那个香港的执行总长,两个都换掉,就像汇丰这样子,这是最起码。其实最好就是把国泰航空搞成像阿尔斯通那样子,把它搞掉。因为整个企业的文化已经都完全变成亲英派,就是暴乱派,你事实上就没办法保证飞航安全嘛,对不对?他们将来要是有一个人想要弄炸弹上去,谁能保证他们那些后勤不会不会睁一只眼闭一只眼合作,这这是真正的一个危险。所以正确的方法是应该让国泰航空破产,破产以后,然后再由中资收购重组。而且这有正正当当的安全理由。我刚刚说的那个,你想想看,他们现在已经在泄露消息了。然后然后开始闹机场了,这些事情已经是真正的安全意义。在美国有谁敢去敢去机场示威,你还没进机场,就已经全部被那个来福枪撂倒了。M16把你撂倒了,对不对?在香港那个香港的警察,我以前在那个香港的警匪片,你看香港的警察一副威武雄壮的样子,其实都是骗人的。他们那个跟美国的警察根本没得比。我跟你保证在美国要去机场示威,你申请绝对申请不到,你还没有进机场。你如果一个集团要去冲击机场,还没有进机场,在那个路上就已经死倒一片。就是你在美国住,你应该知道这事实上就是这样子。



史东(45:40)

对,因为这些事情不是闹着玩的,不是开玩笑的



王孟源(45:56)

不是开玩笑的。那这个国泰你将来真正有人要国泰的包机,要是他们知道中国哪一个代表团,哪个公司包机,保不准就会有炸弹。这个是真正合理的一个可能性,可以担心的事。最正确的合理方法是民航局必须要像中国的不晓哪一个单位处理汇丰这样子,最起码最起码要CEO要换人。那我觉得他们这个汇丰还没有算到整个腐败层反中的一个窝点。国泰已经是这样子了,所以必须要更进一步,把整个国泰弄到破产为止。那这就是我今天要讲的,就是短期的处理,基本上就是这个原则,就是你不直接出手。但是因为暴民之所以难处理,是因为他们。冲在最前面的都是那些小卒嘛,就是在社会边缘精神有问题的那些人,真的黎智英绝对不会冲在最前面的,所以你如果直接出手派军警出手的话,你打这些小卒刚好就是他们所希望的。因为他们就是希望能够制造宣传的闹的热点嘛。



史东(47:33)

对,因为就是就像你刚刚讲的,在在闹事的媒体的那些camera 都已经等在那儿了,拍镜头了。



王孟源(47:42)

你发射几个烟雾弹,那个gas 他们就在那边大闹,美国自己一年用几万颗几十万颗,他们自己为什么不讲tear gas 这种很standard 的东西嘛,对不对?但是这个就可以上国际新闻,你光看那个就觉得很好笑。香港警察大概用了十几颗,美国自己用了十几万,二十万一颗。



史东(48:07)

所以我又把话题带回你刚刚提到RT,RT在美国为什么受欢迎,就是这个道理。他讲了很多主流媒体不讲的事情嘛。



王孟源(48:17)

对对,所以RT有的时候会帮中国人讲话,但是最好还是中国自己有这样的机构。但是中宣部,我想是扶不起的阿斗不指望了。



史东(48:31)

对,因为因为我自己干这一行,我也有相当深刻的感触,你知道吗?因为因为我我甚至怀疑这是不是跟中国的文化有点关系。



王孟源(48:40)

不是,我觉得现在中国的中高层干部,那很多是经过文革,所以经过文革你就失去信仰,他们基本上就是混一口饭吃,为了个人的地位跟利益,你必须要把这个整个阶层换掉。新的没有经过文革的那一代,只在邓小平的治下,他们才相信新中国,他们才相信中国应该主动出击,他们才会有那个信心跟动力。所以说来说去,这件事短期上没有什么好处理的,因为你的没有话语权嘛,所以根本最大的问题就在于你没有国际话语权。



史东(49:22)

我觉得有时候你看有一些话语权在酝酿之中,在散播之中。但是你会发现他还是只是在老中自己的圈子里面散步。



王孟源(49:33)

对,并没有跳出那个圈子。你必须要跳出中文圈子,中文圈子这几年我也进了一份薄力。就是你看我的博文上面,因为美国那一套宣传本身就是一面一张薄纸,一戳就穿嘛,对不对?你只要把美国的真相报道出去,就没有人会相信他们自由民主这一套。所以你看我的博文上面就有一系列有关美国真相的文章嘛。那接下去他的那个理论也是千疮百孔,完全不合逻辑。所以我也写了一大堆有关政治理论东西。虽然我这不是我的专业,但是逻辑是我的专业。不过我们回归正题了。所以我觉得短期上处理这次香港报案,就是在政军警都不要出手。但是你要切断他们的财路,要杀鸡儆猴。那些企业该打的就打,该杀的就杀,尤其是国泰航空,那简直是自愿出来当杀鸡儆猴的对象,这个真是天赐良机,他自己自愿要要让你打成残废,那你还不好意思,对不对?不打对不起他是吗?哈哈哈。但这个问题是中国的那些智库,还有幕僚,他没有这个魄力,习近平有这个魄力。你看汇丰这个东西拖两三个月,为什么?因为他必须要层层上报上去



史东(51:20)

其实没有人没有人愿意,也没有人敢负责任嘛,就是这样,



王孟源(51:24)

没有人敢说,问题是习近平敢做,但是你总是要有人建议才行,这个最高领导人不能够坐在那边关心一切的细节。你下面的人必须要把这个合理的建议送上去。上次汇丰这件事情不急嘛,所以两三个月之后报上去了。国泰这件事情我我就很担心他会不会是因为没有人敢有那个担当,把这个建议送上去,结果就没有人敢,没有人。



史东(51:55)

而且我想到另外还有一件事情,孟源就是譬如说你是国泰这个事情如果要处理的话,这个timing 也是非常重要。你要在时间点上要打七寸,这个这个效果才会最大。如果你拖了一个月两个月过去,这个效果,即使你做了这个效果,也失去很多。



王孟源(52:15)

就没有名正言顺了,我去年就说你应该对等反击也是这样子。因为你他刚刚抽关税的时候,你做对等反击,你有人说你不对,那你如果拖了六个月以后才说我要对等反击,最后就没有办法,就就很奇怪了,对不对?对对,那国泰这东西很容易嘛,你说你也不需要什么新的政策,就是他现在这个交名单的政策。但是现在国泰是交的名单,就是把名单交上去,他没有还没有被要求说证明这些人没有参加示威什么的。那你如果只要把这个标准改成说,你必须要证明。最起码所有参与飞航管理的人都必须要经过测谎器的测验。而且这不包括这不是是飞航组员,连他们的经理什么的,经理到后勤每一个只要有一个人还没有通过测谎器测验,你就不能够通过中国领空。你要知道,香港虽然是在中国的海岸线上的,但是他只要离开香港海岸,大概五公里十公里就进入中国领空,四周都是连南边都是。你不要看南边,好像是海南边的,另外那些小岛都是中国的领土,所以它的中国领空其实把它团团围住。你如果说必须要通过测谎器测验,全公司十万个员工必须要通过测谎器测验通过了。我才让任何一个扳机通过周末领空,马上所有的国泰班机就全部没办法飞。一个月之内他就要就要破产。这很简单的事情,只看他敢不敢做。我觉得国泰已经是自己出头,准备要当出头鸟了。那你不做白不做,而且事实上你在军警政都不能够出手,只有在经济上能出手,那这个是最合适的。好吧,短期上的事情我已经讲的差不多了,我讲一下长期的。我想我没有详细解释为什么这个谈到香港会跟载人登月有关的。你我只是把这个大纲标题送给你,你大概也在纳闷



史东(54:18)

我对你有信心,你一定会做到的



王孟源(54:24)

我倒不是说要把香港当成发射台,或者香港有什么科学家可以参与,工程师可以参与载人登月。,是这样子的,你如果看历史上这个历史上的那个霸权交替,其实中国现在是在一个很尴尬的时期,就是这个霸权交替大概是三十年的过程。从中美的这个霸权交替大概是2008年开始的嘛,就是金融危机开始。那美国也刚好是从2008年才开始,真正全力火力全开。国内国外的宣传完全丑化中国,就是把中国当成头号敌人。在那之前他们还认为isis 不是不是isis,就是al-Qaida是头号敌人。就是从2001年到2008年之间,他们认为al-Qaida是头号敌人。但是在2008年那个金融危机之后,他们看出美国国力衰弱了。这下中国可以赶上。这时候他担心的就是中国。所以美国的对中国的丑化敌化,其实是2008年进入全速前进。那这种过程大概是五年之后就可以有很强的成效。所以大概在2013年2014年的时候,我就注意到,这时候就像我刚刚讲的,香港有中间派,美国也有中间派,有很多非政治性的人。,就是比如说你去理发的时候,那个理发师那也许不在乎这个政治,平常不在乎这个政治,或者你去看牙医,那个牙医摆弄牙齿,聊聊天。当然平常不在乎这些政治,但是你一旦这种铺天盖地的洗脑搞了五年之后,连这些人都知道中国就是坏蛋,中国就是邪恶的化身。我们出的什么问题都是中国的责任,就是已经到达这个地步。那时候我就观察到这个地步。所以川普一上台搞贸易战,在一年半,两年前我就知道这种事情不可能善了。那时候只有我一个人在讲,说不可能善了,每一个人都说,哪一个像样的智库,都说我们应该设法跟他们和谈,赶快弄个合约。我跟他说这机会不大,对不对?你记不记得那个时候我大概全世界只有我一个人这样子。我敢这样讲的那个底气就是在于我人住美国。我观察他的这些宣传,我天生就对虚假虚伪的宣传很敏感,所以我就注意到这个宣传的那个火力在2008年之后已经上来了。那上来到一个程度,就是他五到十年之后,基本上谎话重复三遍以后就变成真理。那这个已经被重复过三百万遍了。那你想想看,对一般美国人他们的印象是怎么样?现在trump上台,他搞这个贸易战,然后又撕毁跟国际盟友,然后又侮辱很多第三世界的国家,对中国是一个很好很好的机会。就是我们现在在2019年,才从2008年开始的霸权转换。那目前才过了三分之一,就是还没有到过了那个dividing point 。所以这时候中国会有些很尴尬的处境是很自然的,因为中国的国力总体来说还是适应美国的。那这个旧霸主占有国际话语权也是很自然的事情,对不对?那他利用国际话语权来打击你,一定也是理所当然的事情。为什么?因为我我以前说过很多次,美国的霸权建筑在三个这三个角上面。第一个就是宣传,第二个是美元,第三个是美军。但是美军,用美军去打一个强国是非常昂贵的,风险极大的,所以它一定是不得已才去用的。那其次是美元。但是中国的那个金融不是对外开放,所以你那个美元没有着力点,那宣传是耗费最低,花费最低,功效最高的。你像苏联就是这样搞垮的嘛。所以所以中国在受美国的宣传打击,像包括这种在香港的颠覆,这都是其中之一。那你要你要争取国际话语权,除了跟他们讲理。可是事实上我们在国际上怎么讲理?你在那个香港,这次我看到有些有些暴乱分子,他们那个真正是生活在好莱坞的创造的世界里面。他们那个你看他们所说的话,你会以为真的这里有所谓的justice league 正义联盟,或者是avengers。美国的超级英雄马上就会飞过来解救他们。因为比如说他讲的是说,我们应该计划怎么样抵抗解放军,支持到让联合国军跟美军来支援为止。你说这个脱离现实到什么地步,我真的是很难解释,这比台湾的那些绿卫兵还要离谱了。基本上这就是看那个好莱坞电影看的太多了,不知道他们是指处在哪一个地球上面的。但是你这些已经完全被洗脑的或许没有办法。但是有很多中间派,有很多第三世界的国家,美国的力量的来源其实是因为他一呼百应。整个国际结构,很多组织,你想想看,军事、政治、金融、外交,连那个慈善什么都是美国设立的,对不对?所以他们就可以挟挟持这些欧洲、非洲、亚洲的所谓盟中东所谓的盟国或者是友善国家,更别提中南美洲来来造势。美国最喜欢的说的就是international consensus,国际共识。对英国美国说出来的,那些人不敢说不的。



史东(1:00:56)

就是呼拥而上。



王孟源(1:01:00)

对,你要打破这种国际共识的格局,创造一个新的格局,就是大家公认有权利搞一些事情。这个在历史上都是靠战争过程的,不一定是全面战争,那个希特勒搞成全面战争是很笨的。但是你看像日本当初他会成为东亚霸权是怎么样先打了甲午战争,然后再打了日俄战争,这样他就成为东亚霸权。要不然他一个黄种人的国家,你想十九世纪的欧洲跟美国怎么会把它当一回事,对不对?你战争打赢了,人家就把你当一回事。那重点是必须要打,只打局部战争,你不能打全面战,全面战争就很可能会打输了。而且在核战争核子武器的那个背景之下,其实无所谓打赢了,对不对?大家两边都是会输的。所以我个人是希望不要扯到战争。因为事实上中国最早可能战争也就是2026年,我我最近写了一篇博文,说中共最早可能会主动武统,也就是2026年。在那之前条件不存在,为什么?这个三十年的霸权交替到2024年2025年大概是到一半吧,对不对?在那之后中国才会有足够的实力,和国际话语权,也就是国际正当性。就是你有权利去处理你地区内的一些事情。人家把你当一回事,那可是如果你想要有这个,这就变成鸡与蛋的问题。你要有这个话语权,要有这个行动权利,你就必须要先靠赢得战争。可是你要先赢一个战争又没有太大的后果。你要先有这个行动权,有一个解套的方法,就是你完成一个大家公认非常困难的事情。尤其是载人登月,载人登月是极少数可以做这种立竿见影。就是说我是世界头号强权的。我能够载人登月,美国都没办法,你凭什么跟我说,我不能够干这个。那刚好美国是因为美国是因为波音公司最近这二十年腐烂的很快,然后又想办法独霸了NASA的预算。所以他把那个多快好省的一个方案解决掉了,然后把钱投入那个很浪费,效率很低很困难的一个方案。这就给中国一个很大的机会,就是能够用那个比较好的方案来超过美国。那但是这里面有很多专业的技术细节,所以我不在这里详谈了。我的博文上面已经解释的很清楚,就是有很多专业上的考虑。不过搞载人登月,它本身没有经济上的实际价值,它完全就是一种广告。那这个广告平常是不值得那几百亿美元投下去的。但是对中国来说,它是值得的。因为你如果做了这个广告,香港人一闹,一般的老百姓就会说我们为什么要跟美国美国连载人登月都拼不过中国了。美国过去这五十年吹了多少次载人登月,是多么多么的了不起,对不对?



史东(1:04:48)

解释一下载人登月是怎么样一个情况。



王孟源(1:04:52)

载人登月是基本上是一种炫耀,是一种面子工程,但是这个是一个很国际性公认的面子工程。因为美国人已经帮你吹嘘了五十年,说他有多么重要。所以美国现在的计划是在未来这个decade 要重新载人登月。我是看到有一个机会,中国如果采取正确的技术方案的话,有可能在美国之前,也就在十年之后就载人登月。



史东(1:05:25)

你说的这个载人,你说的这个载人登月和当年的armstrong 登陆月球有什么不一样?



王孟源(1:05:32)

一样,只不过是现在的技术更好。你可以到月球背面或者是月球的南极。美国是因为它本身那个NASA有很多问题,他这个官僚官僚体系的腐化很严重。然后波音他的主承包商,波音的腐化更严重,所以基本上给中国一个可乘之机了。要不然美国在火箭技术上的累积还是胜过中国的,所以他目前也不把中国当回事。他不认为中国能够在十年之内搞出来。那中国如果是要依循美国的那个技术方案的话,别说是十年了,就是二十年都不一定能够搞错。因为它的技术累积不如美国。但是它有一个可以超进入的捷径。而这个你超捷径以后,你想想看,如果中国今年在2028年登陆月球,结果美国搞了十几年,投入了三千亿美元,什么都没有搞出来。这个会不会是多么大的震撼全世界的



史东(1:06:30)

我们可不可以把它看成是一个二十一世纪的郑和下西洋



王孟源(1:06:35)

对,就是基本上就是宣扬国威了。而这个这个对你对那个非洲印度什么中东这些人说,哎,我们为什么还要跟美国?美国连载人登月都打不过,都没办法干过中国了,对不对?



史东(1:06:53)

这个我了解一下,我觉得这个重要性这个不言可喻



王孟源(1:06:58)

所以我个人的看法是,中国这里有一个机会,他不一定是能够成功,但是中国可以偷偷摸摸的做嘛,他不需要像美国这样子高调的说,我就是要哪一年哪一年做成,然后失败了,也就在后推,他可以偷偷摸摸做。因为事实上没有人期望中国做得到,那中国不也没有关系,就是我在那个博客上有跟读者讨论这件事情。上次是能够抢先在美国之前登月,这个收获的广告价值是以万亿来计的。如果中策是你就不要理他,因为这件事情事实上没有经济效益。让美国人要浪费钱,就让他们浪费钱嘛,对不对?反正他们已经登月过了,再登一次,我也没有什么太了不起的,广告效益也没有多大。下策则是照抄美国的结果,你比美国晚十年登月,那这个意义没有什么变化。所以我之所以会把香港跟载人登月扯在一起,其实我那两篇载人登月的文章是上个礼拜才写的。就是因为其实就是因为我想到中国很多很多的问题,是因为他在当前国际社会没有话语权。他讲出来的话,没有人管他的立场,人家认为你不是世界霸主,对不对?我为什么要跟你?人家第一个反应就是这样。你如果有一个标杆,能够有一个广告能够告诉全世界说,我现在是真正可以跟美国平起平坐。那这时候那些对美国有点疑心的人,就会正视这些真相,正视一些新的可能。否则大部分的人没有那个知识,也没有那个理性逻辑去分析这么深刻的事情。我们这里谈论这些事情,连观众都必须是高级知识分子才能够吸收这些事情嘛,对不对?我在这边补充一下,知识分子是什么,我自己也有定义。我的坏习惯。因为我发现这这些搞宣传的人,他们一个很常见的伎俩,就是他们利用语义学学上面的含糊不清来困惑读者,困扰听众。所以我很喜欢讲一件事情,就一定要明确定义我现在讲的这个是什么东西。那知识分子这个东西也是一个很含糊的一个包包,我觉得知识分子是有能力,这个能力就是逻辑跟能够有知道足够的事实,然后又有足够的能逻辑能力去分辨一般的宣传文稿



史东(1:10:15)

能够独立思考



王孟源(1:10:20)

不一定是独立思考。就是能够别人教给你一套说法的时候,你能够做独立,会有足够的质疑能力来判断。高级知识分子是有能力不只是质疑,而且能够引申来来推演来想到没有被直接提到的话题。那我们要知道知识分子一定是人口中的少数,对高级知识分子更是少数中的少数。那这就是为什么宣传很重要?美国人为什么这么重视宣传?因为大部分的人口是没有理性质疑能力的。但是这些世界人口他们的态度还是决定你在国际上的话语权,话语权决定你的行动权。所以中国如果要保卫自己的权利,必须要做一些广告。那我认为载人登月是一个在这个阶段是一个很好很合适的广告。几百亿美元也不是太贵,因为你的收益可能是几百倍。这就是我我对解决香港问题这个一个看法。因为你如果有了国际话语权,更强的国际话语权,国际行动权这种类似香港的这件事情就不会发生,或者不会有这么广泛的支持。至于collateral effect,就是你这个香港的这个问题对台湾有什么影响,我想这也是你说你你有兴趣的一个话题。我觉得中国应该在吸收这次经验,就是你这个让利真的不是一个好的策略。你想想看,这次香港的中间派会支持这些战争的暴动派,就是因为他习惯你让利了,所以你这个引渡条约稍微拿走一点,有可能拿走一点让利的那个皮毛,开了一个窗子,他们就非常的不高兴。人性就是这样子,他们已经拿到手的东西,他们就认为是自己,他们不会感激你,给他们让利不是一个长久之计。那当初邓小平有这个智慧要拿香港来做实验,他不只是要拿样板,而且也是来做实验。因为他也不确定这个策略,对不对?那我我觉得你一直演化到2019年,现在可以确定的是,他的这个对香港的这种一国两制让利的策略已经是完全破产了。



史东(1:13:10)

就在这个节骨眼上,我也要插一个问题,因为这也是我想跟您探讨的问题。所以你刚你已经把这个头起来了,就是五十年是一个不长不短的时间,二十多年已经过去了,还有剩下来大概多少?二十八年吧,二十五年加几年,这二十五年在历史的长河之中,更是一个很快就过去的时间。五十年过去之后,你觉得香港会变成一个一国一制的地方吗?你觉得这样子做对北京是不是有利?有是不是有最大的利



王孟源(1:14:04)

我觉得香港是很小的弹丸之地,那中国应该也一定会等到五十年之后再搞一国一制。不过届时再怎么处理其实不重要了。他们那个如果到时候。觉得还是有需要一个实验地的话,也可能就继续搞一国两制。不过我觉得可能性不大,因为到时候香港一定已经烂的一塌糊涂。你看这一次他们我在这边说,应该把国泰航空搞成破产,他们在他那些暴动人那边讲的,要把香港机场封锁了,他们自己还想要搞这个,看自己的经济,他们想要想要把它搞烂,说这些人有多蠢。他们基本上就是在搞颜色革命。我觉得目前的处理是对的,你就是不要有军方,有不要有武警,不要有升级,不要让他们去制造事件。因为你只要只要军人出现在那里,他们就有办法制造事件。哪一个人挨着冷枪,根本不会跟你讲理,对不对?这个在乌克兰殷鉴不远,真的真的不能出手。那香港基本上就这边让他烂嘛,哪一个企业敢支持这些暴动,你把这个企业搞掉了,就是经济可以出手。但是在军警政不要出手



史东(1:15:03)

我问这个问题有一个点,就是到了五十年之后,北京再重新审视香港这个地方,以及香港这个地方,能够对于中国或者给中华民族提供一些什么样子的利益。这个利益或者金融上的,或者是这个不管是金融或者情报上的,或者什么样子的。会不会有这个可能性。北京政府说,我还是照这个样走下去,因为基本上对于北京来讲,照这个样子,而且还是有利的。还是说北京就是说五十年到了,我给你的promise 给你的承诺的时间到了,然后我就重新洗牌,重新来过。那就一国一制吧,在那个时候就符合你所讲的,就是那个时候就代表香港对于北京的意义已经没有什么了。香港说不定他在金融界的那个这都已经可以被被中国其他的内地其他地方取代了。然后香港就从一个历史的一个小渔村变成一个很灿烂的钻石,然后慢慢的又变成一个历史的边缘地带。



王孟源(1:16:28)

我觉得这是必然的。因为香港的这些虚拟经济不只是上海想要拿,当然上海没办法全拿,因为美元还是国际储备货币,所以中国没办法全面金融开放。那会是自杀性的,但是新加坡也在抢,新加坡也知道这是easy money,是这样子。二十八年之后,美元绝对不是国际储备货币。那到时候中国很可能就会金融开放的程度比现在高很多。因为真正金融不开放的最主要原因就是因为美元,怕美国会用美元来作为打击的手段。那你如果二十八年之后,美元不可能是国际唯一的储备货币的话,那就没有这个忧虑。那中国的金融业会比现在还要更大的多。那对香港就有完全取代的本钱。那所以我的意见是二十八年之后,香港基本上没有什么产业值得一提了。就是他已经没有任何的战略价值,它本身是一个容积高度,贫富不均。然后,由于这个暴动本身的秩序稳定性也不可靠。就是你有一大堆企业和将将近一半的民众想要把你的经济搞垮,那这个经济怎么可能搞得好?这样一个无关紧要的地方,你说他将来要做什么?我觉得没有必要做预言,因为either way 那怎么办?台湾也是一样,我很我很怕台湾,其实台湾如果在七年之后就被武统是一件好事。因为至少那个时候台湾还是relevant。你如果台湾再拖个二三十年,说到2040年的时候,他很可能就是totally irrelevant。



史东(1:18:16)

对对对,因为因为我为什么问香港的问题,我这个问题的来源是从台湾来的。因为我在和另外一个受访者讨论这个事情,他就断言他说台湾经过这个长时间的繁荣之后,慢慢的他又会回归到他历史的原点。这就是是中华民族的一个边陲地带



王孟源(1:18:54)

一个边缘省份,一个比较贫困的幕后的边缘省份。对,那事实上就是台湾继续在现在这个trajectory 下去必然是如此,而且很快就会如此。我上个礼拜才又复习了一下台湾的这个GDP。因为一般你在发展的时候,做这种很长期的几十年的国民生产毛额的比较的时候,都是相对美国来说的。就是你工业化的过程中,美国除了既有的工业基础,它还是国际霸主,有一大堆有国际储备货币这种优势嘛,对不对?然后他又是国际行政上的,跟宣传上,跟法律上都是他说了算的。所以你说要平均国民生产毛额,我要追上美国是很困难的,但是这些四小龙,还有日本,这些新加坡追上了,那个日本曾经追上,现在又掉下来了。但他掉下来到目前为止,至少名义上还在百分之五十,百分之六十左右。那另外一个,所有这些小龙里面就是,香港就是上到百分之四十五十左右就停在那边了。然后,韩国还在努力的往上一点一点的挤。那另外一个最可怜的是台湾,台湾是很高速的,冲到了百分之三十四十左右。然后到了他是1990年,到1996年,他到达顶点就是美国的百分之五十。然后从那之后,又往下掉了,现在大概是百分之三十五左右吧。我不太确定,大概要什么样的。就是说你过去的那二十几年,从李登辉开始,基本上就是往下掉,你即使有后发优势,人家的那些先进科技,你不需要投资去尝试,你可以直接引进你,结果还是越走越远。在后面吃灰尘。这个趋势我是没有看到停止的的希望。



王孟源(1:21:12)

所以反正我觉得香港这个事情是两岸关系的一个很好的明镜。但是在台湾已经没有人有这个能力或者意愿去参考这个这个例子。所以只能够希望他能够对中共有一个很好的借鉴的价值,就是了解到让利还有一国两制是已经完全破了,在实践上完全破产的方案。你必须要有更好的方案来谋求全中国人民的最大化。那这个全中国就是包括还没有在这政治权上面统一的管辖区里面,只能说到这里了。这个哎谈到台湾的事,就是只能给我摇头,真的是没办法。你看像这次选举明明就是最后的机会了。你如果要扭转一个文化,即使是最初始的一些苗头哦,最初步的也必须要至少五年。那你如果要到2026年,2027年,中共可能武统的时候能够证明给他看,说台湾有改善的迹象的话,你必须从今年明年选举上台的那个总统就开始做。那要做,至少从哪里做?从教育上做,就是从陈水扁开始改的那些课纲,必须要改回去。但是没有一个候选人让我有希望会做到这一点的,所以台湾基本上是没有希望



史东(1:22:58)

我同意我同意,这个非常遗憾。这个今天的访问是在一个非常令人depressed 的这个气氛之中的结束。不过我非常谢谢孟源再一次的说一声,非常谢谢您非常精辟的一些看法和一些见解。我们会继续观察香港的事情。但是我们关心香港的事情,同时我们也关心台湾的事情,同时也非常高兴知道你现在回到美国的家中,然后一切都在继续进行之中。有关于其他的一些事情的安排?所谓一句话叫做什么,好事多磨,对不对?反正好事总会来到。



王孟源(1:23:47)

我想我们都是关心国家民族的的人,(知识分子是吗?)高级知识分子,所以台湾这个局势一副末代朝廷的样子,真的是让人沮丧。明明知道他该怎么改,但是我们就是没有力量去改。那个我我写了五年的稿子,哎,该说的都已经说了,但是没有人愿意听。



史东(1:24:18)

我想我想不不止你一个人有这种感想了,不止有你一个人,有这种感想,很多人很多人有这种感想



王孟源(1:24:25)

我们只能够把国家民族的希望放在大陆上面了。这个我觉得大陆老一代有一些问题,那新一代还是有希望的。我觉得他们蛮有冲劲的,是但是需要一些智慧。大概年轻人只需要智。



史东(1:24:51)

对孟源非常谢谢,谢谢你,我们下回见。

\twocolumn[\begin{@twocolumnfalse}
\section{、NBA和美式宣传、美国经济周期、量化宽松、Repo}
\subsection{20191008 }
\end{@twocolumnfalse}]Credit: arbitrage



史东(0:00)

各位朋友你好,我是史东。在过去几个礼拜,我和我的好朋友也是你的好朋友就是王孟源王先生有一次节目内容的可能性的交谈,因为我看了点事情,我觉得这个事情非常值得我们注意。我把这个email 这个就传给了王先生,王先生说他也注意这个事情,他可能会在不久之后就会写一篇文章,我觉得很好。我说等你文章发出来之后,我们就在节目上来谈谈这个事情。那么今天我们非常高兴为您请到了王孟源先生在节目中和我们谈这个事情。今天我们节目的名字定的是《一叶知秋——看美国回购利率窜升的现象》。同时我也非常高兴向您报告:王先生在带入主题之前,也会和我们谈谈他最近观察的一些其他事情的看法。首先我们就把王先生带进我们的画面之中。



王孟源(01:16)

非常荣幸能够再上你的节目



史东(01:19)

我想这个话题怎么打开呢?在你的部落格里面,你是这么开头的。有关于这个议题,你是说九月十五号十七号两天的美国银行之间的回购利率,也就是Repo rate 突然从年利率的百分之二窜升到百分之十,这件事情也是引起我的注意。因为我是一个非常外行的人对这个方面,但是以我的直觉,我知道这个事情非比寻常。那么我们今天就从这个角度切入。然后如果你要在这带入主题之前谈一些其他的问题,我觉得这也是一个很好的时间。



王孟源(02:01)

主要是因为刚刚发生的有关NBA的一个火箭队的经理,他支持香港的事情。因为香港是我们的上一个议题,那到现在也还是占据新闻。而且欧美,尤其是在美国支援香港的声浪越来越大,所以这件事情大概不会马上平息。所以我想就最新发生的事件做简单的评估,事实上也跟今天的主题有一点点关系。是这样子的,那个火箭队的经理,他是MIT出身的,他是一个受过高等教育的。所以不管怎么样,在这一种顶尖的大学里面,他们是一个非常白左,就是非常讲究政治正确的环境。所以我感觉很自然的,他纯粹只是因为他自己受过洗脑,他的周边的家人朋友,他的价值观就都是受白左的那一套政治正确所影响。所以他一时冲动就把它写下来。写下来之后,中方开始抵制这点,我就很佩服。因为国民党政府从来是没有一点骨气的。我一直觉得他们需要去订购一套钛合金的脊梁骨,要不然好像真的是无脊椎动物。



不管怎么样,中共有这个骨气,我觉得是很好。但是也不要有指望说NBA会跟你妥协,因为这是不可能的。因为他在中国每年赚几亿顶多十亿美元,他在美国是以几十亿上百亿这样赚。那这个支持香港这件事情是美国左右两派共同支持的政治正确,所以NBA如果在这方面这一点让步的话,他一定会受到美国从上到下,从左到右各方面的谴责跟抵制,这是不可能的。他们不可能这样做,也就是纯从生意人的观点来看,就不可能这样做。所以这并不是说我说这样做不好,像两三年前有关洞朗的那件事,我就说你跟印度这种疯子没有什么好吵的,你就低调地在幕后跟他们协调,然后要威胁也是幕后私下的威胁。这一次不一样,我觉得高调的反应是对,我怕的不是他那个高调反应会有什么副作用。我怕的是你一开始雷声打得很响,到最后如果雨点越来越小,到最后这些制裁就撤掉了以后,反而是一个负面的示范。就是说你们这些美国人,不管要怎么样冒犯中国人,要搞种族歧视,或者是要搞政治正确、要支持香港,甚至要支持什么独立台独什么的。随便你们,反正我们抗议也就是一个月的事情,你们过了一个月以后生意照做。如果是这样子的话,那还不如一开始就不要讲话,对不对?



我在我的博客里面详细解释过了,美国从来不容许有任何一个国家威胁他们霸权。从来没有任何一个国家的工业产值到达他的 70\% 以上。但是中国运气很好,他在到70\%的时候,刚好是911。所以那十年美国没有办法旁顾。然后到了2008年,有了那个金融危机。那个金融危机以后——我等一下会跟你讲那个金融危机对美国的打击有多大——在那之后的六年基本上它完全喘不过气来,一直到2014年才稍微喘过气来,这个待会我会在经济上给出确实的数据。所以中国在2014年的工业产值,那时候已经是120\%,今年已经到了美国的170\%。Purchasing Price Parity,就是你平衡物价不同来算GDP的话,2014年中国是超过美国的,所以基本上现在已经是反过来。美国的工业产值只有中国的70\%不到。美国现在没办法直接动手,经济战这件事情打的时候,他们还挑的专注是进出口不平衡这件事情来做。但是事实上两边所受的伤大概是差不多,都是5000亿到1万亿左右,就是过去这一年半到未来一两年。



我为什么先谈这些事情?因为这些事实上的数目,我待会儿都会实际上给出。这个美国的经济它的底气到底有多少,在节目稍后都会详细的给出数字。在2008年,美国原本已经急着要出手对付中国的时候,就是他从911喘息,从伊拉克开始撤军。原本准备要对中国动手的时候发生了那个金融危机,对他们是非常不利的时机。在当时很明显的,他们就在对内的宣传、对外的宣传上面,有了口径一致的转向,就是全部转过来对准中国。在我的博客上有一个图片显示:纽约时报在2008年4月之前,基本不管中国;在2008年5月之后,中国基本上是他专注的外国,而且讲的都不是好话。



另外一点是大概四五年前(这个我在那篇文章也解释过)。原本我们讨论全球暖化(就是气候变化)的时候,我们讲温室气体有两种:二氧化碳跟甲烷。原本大家都是把它们合起来算。因为你分开来算没有意思嘛。你看全球暖化的效应的话,这两种主要温室气体合起来的效果是多少。那时候因为二氧化碳的贡献比较大,你就把它们加起来,算成等效二氧化碳。几十年来都是用等效二氧化碳来算。但是在五六年前(也许是四年前)忽然之间一个月之内,从美国的洛杉矶、纽约、芝加哥,到英国的伦敦,所有的英文主要媒体一个月之内全部都停止使用等效二氧化碳,而只算二氧化碳。你能不能猜一猜为什么?(就是他们不算甲烷了)



那个月刚好就是中国的二氧化碳排放量超过美国,成为世界第一大二氧化碳排放国。为了能够说中国是世界第一污染大国那句话,必须要改变标志。事实上,美国的甲烷排放量比全世界其他国家加起来还多。如果连同甲烷一起算下去的话,中国就不是世界第一大的温室气体排放国。所以我讲这些事情是跟你讲一下,从2008年开始,美国有一个有系统,有体系的,有组织的对内宣传洗脑。我们现在看到的这个NBA这个这个事情是这个洗脑的一个后果。不是他们预先设计的,而是他们就知道,只要有组织的去洗脑,自然就会有无数的民众自愿地来做。这是美国式的动员。



像这种政治正确,比如说现在很有名的那个瑞典小女孩Greta Thunberg。她搞这个,这也是政治正确,这是左派的政治正确。那为什么政治正确在过去这四五十年会这么流行呢?因为美国的精英、美国的财团知道,政治正确先天就是非理性的。如果是有理性的讨论就不是政治正确,而是真正的深刻的讨论。只要是非理性的,你就能够要么把它导向来到自己有利的地方,要么即使原本是针对自己的,比如说occupy wall street占领华尔街,你也能够把它乾坤大挪移地把它拨开。讲了这么多,我要强调的是:NBA这件事情是美国的宣传体系在过去这十年来,在他们国民的心目中要种下仇中反中的心态的一个结果。你能唯一的正确的反应就是要准备长期抗战。你要长期抗战的话,像是NBA的这个事情,你要抵制就要抵制到底。你如果不能抵制到底的话,一开始就干脆不要做。要不然反而是丢脸给他们做负面的教材、负面的示范。那你看看,这一件事看起来好像跟金融没有关系,可是实际上它的源头就是金融,就是2008年的那个金融危机。



王孟源:(12:40)

我们现在这个世界是有70多年的和平,70多年的和平的意思就是资本可以无限地成指数地成长,这基本上是《21世纪资本论》里面最主要的论点。除非有世界性的天灾或人祸,像世界大战那样子,否则资本的成长率一定高于经济本身的成长率。经济的成长率就是GDP 成长率,资本的成长率就是投资报酬率。投资报酬率一般是5\%,有的时候更高一点,可以高到8\%。世界的GDP的成长率基本上3\%-4\%。所以你可以想象到,资本跟世界经济规模的比率,在和平时期一定是越涨越高的。所以在1970年代,资本受不了president Johnson 的那一套Great Society 社会主义的做法。他们开始反扑的时候就建立了一整套的宣传体系,就包括美国的经济系,美国的金融系、美国的智库,全部都转化成讲财团的那套假话,骗普通老百姓的。但是这种谎话说久了以后,他们连自己也相信。从这里就可以讲到为什么在冷战之后会有一系列的金融危机。在冷战期间,一般的经济周期是五年、六年,顶多就是七年,这为什么?这是因为他们那个时候规模很小。所谓的自由世界就是欧洲跟美国,大概是六亿人口的经济规模。在这个经济规模里面开始景气复苏以后,大家开始投资,开始雇人买机器,增加生产。到了五年之后,好的生产的机会已经都投资完了,这时候投资进去的就是报酬率会不好的,这是第一个原因。第二个是人工会不足,所以这个时候工资会上涨。工资上涨以后,成本就提高。所以到最后,第一个是投资报酬不能够继续维持过去这四五年的水准。第二个是人工成本开始往上窜,这两个加起来的结果就是获利率开始下降。你获利率一下降了,经济就开始衰退。这是一个正反馈的系统,因为越来越糟糕,滚雪球一样。然后这样衰退了大概是一年到一年半左右,再开始慢慢复苏。这是一个典型的从二战后到1990年(到冷战结束以后)的经济周期。



史东(15:53)

对不起,我打个岔。你刚才说是二战后的这种周期,二战前难道不服从这种定律吗?



王孟源(16:03)

二战前因为刚刚好有经济大衰退Depression,在Depression之前也是有这种规律(当然也被一战给打断了一段时间)。但是到了1929年之后,大家全部都拿出关税壁垒,那时候世界的经济就完全碎片化了。所以从1929年到1945年之间,没有所谓的全球经济,基本上是非常碎片化的。比如说那个时候很多经济之间的贸易甚至是以物易物的,像德国要跟中国做生意的话,就是把大炮卖给中国,然后从中国换钨矿回去,这是以物易物的。他们要跟南美做生意也是从那边弄矿产,然后送工业机器过去。那时候可能也没有用到什么国际储备货币,就是纯粹的以物易物。这种当然效率很低了。不过不论如何,当时还是西方世界,也就是世界有一半在那边。那时候中国也还没兴起,印度什么的第三世界都还没有兴起,土耳其也还是第三世界国家,基本上就是先进国家,就顶多就是有日本。但是在冷战结束之后(就是1990年之后),这个周期一下子加倍了,变成十年一次,为什么呢?



其实有点不太一样,他是这样的,第一个十年是这样子。第一个十年是因为美国变成一超独霸,所以他可以在全球到处掠夺。尤其是在东欧苏联,他的那个垮台以后,人才跟资源完全被他掠夺了。就是那个以前国有化的企业全部都被私有化。私有化的过程中,那些那些新的巨富其实都跟美国人有关,有勾结的,都是拿了美国人的钱,然后跟美国人合作的。那这边就有注入了一大堆资源。然后在一九九四年墨西哥有一个金融危机,然后南美后来有金融危机。到一九九七年,俄国跟东南亚的一系列国家,包括韩国、泰国和印尼,那一年出现了所谓的亚洲金融危机。



那一次正是美国抄底的不不亦乐乎,那十年是最典型的所谓的那个美元的霸权的收割周期。就是他在景气的刚开始的时候,因为刚刚还在复苏嘛,所以那个联储会还在降息。然后降息的意思就是把美元的现金送出去给大家,送出去给大家以后,这个他送出去不是送到老百姓手里,他送到那个银行的手里,银行拿到这边手里就要交给对冲基金。对冲基金拿着这笔钱跑到亚洲或者是到中南美洲去收买有有价值的资产。人家如果不卖你的话,他就放贷贷款给你。然后到一九九七年金融危机很简单嘛,就是有恐慌,那这个恐慌其实也是美国人开始的,就是那个对冲基金开始卖,然后跟他们的银行讲好了,说你们把他们到期的贷款不能 roll over,不能够延期,那就是受影响还不出来嘛,对不对?还不出来以后就有骨牌下面再一个一个倒下去。倒下去以后你看像三星就被当成美国的基金廉价拿去了,三星这样的世界一流的企业就被他们整个拿去了,那个三星的那个李家其实占的股权很少了,他只是有经营权,他没有那个利润分利润的权利,这是第一方面。第一方面就是他可以从外面掠夺,利用美元这个一进一出,就是你这个美元这样放出去,人家跟你借的时候,他稍微有点问题,你就逼他们马上还,他们还不出来,你就可以把他的资产买回来,这样子大赚一票,这个是当然就是他的景气原因。



另外一个原因是我刚刚提到,那个从七零年代的那些智库还有那些经济系被收买之后,他们讲那个绝对自由主义。绝对自由主义的意思就是说你这个企业没有社会责任嘛,但是企业是无国界的。无国界的话,那你就可以把你的工作全部都 outsourcing 出去,外包出去。外包出去以后,这很好,那个工资大幅下降,工资大幅下降以后,欧美日的那个工资就涨不上去了嘛。涨不上去以后,这下子你以前那个经济衰退的起因都没有了。但是他就有一个新的经济衰退的那个起因,就是从两千年开始的,就是过去这三十年是一个新的经济阶段。他那个因为你这个钱这样泛滥,然后又赚钱又有这么多投资工具,你又可以到国外去收割,利用美元的那个国际储备货币地位去收割外国的劳动。结果这样一来,美元在美国内部也是泛滥。到最后美国。。泛滥,然后你又不投资到生产工具上面去,怎么办?就全部都挤到金融资产去了。两千年的时候,哪一个金融资产是它最热门的?就是股票。我不知道你记不记得。那时候有一个股市的泡沫,这个那时候的greenspan 就是联储会的主席,还在那边说什么irrational exuberance大概是二十年前,就是两千年那个泡沫爆炸之前一段时间,好像两年,九八年还是九七年,他说的irrational exuberance (史东:怎么翻译?)非理性的狂欢。其实他没有意识到你这个是以前经济周期的标准来看,这个股票的价钱,实际上的动力是整个国际的地缘政治的背景不一样了。整个全球经济的盘面都不一样了,就是整个经济开始全球化了。美国国内的政治也从左派走向右派,然后可以把经济把生产工具往外包。然后金融巨鳄可以在全球到处掠夺,这几点统统在九零年代发生。发生了之后,所以长期景气的结果不再是在那个生产工具上的,不(智?)投资,而是大家都把钱弄到金融工具去了。那金融资产的价钱就直线上涨,指数上涨,那上涨到一个地步就是泡沫,泡沫被破坏以后就有后果,这个后果是什么?就是联储会就必须降低利率,降低利率来刺激这个经济。这个是以前凯恩斯你来所讲的联储会的基本功能,就是调整经济周期。



史东(24:01)

我看到一个文章,解释这个为什么需要,调整利息这个事情,我想跟你印证一下,就是联储会这样的机构或者任何一个国家的这个这个银行,它降息是希望能够刺激这些这些钱流向市场,而不是流向储蓄。这个说法对不对?



王孟源(24:22)

对,就是这样子,他们那个联储会都只能够跟大银行来往,他不能跟普通老百姓来往,所以他们这个金融其实就是现金的管道。联储会的降息升息都是基本上就是控制现金的流动的方向,降息就是要把现金往外推。外就是市场,银行再把它往外推,当然他希望是能够银行把它推到实体企业里面去。实际上银行常常是把它推到对冲基金的手里。但那个联储会没办法控制那种事,至少美国体系之下没办法,那个中国的体系可以,是中国的体制体系的优越性之一。因为他们的中央银行可以直接告诉银行说我要你借给哪些人,你不可以借给哪些。美国的联储会就是做梦都不敢想象有这种好事。不管怎么样,到了两千年之后,那个金融泡沫就是股市的泡沫爆掉了以后,那联储会必须执行他的责任嘛。就是这时候经济下行,那大家没有工作了,你必须要出来挽救增值经济。这时候你就必须要放现金。放现金的话就是降息。那这次降息。一降就是降了好几年了,但我记得好像是降到二零零五年然后才开始停止,然后到二零零六年才慢慢的升,这时候已经太晚了。你的那个现金放的太多,为什么美国能够这么放心一直往外放现金?因为美国是国际储备货币,它不管怎么印,你印的太多,自然有外国人拿去用。因为全球都要用嘛,全球的那个贸易都要用你的,他的外汇储备也要用你的。



但是这个外汇储储备这件事情就很好玩。一九九七年他们在东亚大收割了一顿之后,东南亚跟东亚的国家全都学乖了,一下子他们的那个美元储备通通是往上涨。就是你如果去看中国的现在的那个美元储备,外汇储备有三万亿美元,什么时候弄出来的?就是从九七年开始之后,一直到我记得大概2006年,一共大概是八九年之间,累积了三万亿。不只是中国,哪一个东亚国家?韩国当然自己吃的亏很大,那东南亚那些国家,什么印尼、越南,通通是累积了一大堆美元。发生的事情就是,联储会把这个利息降低了以后,他发现怎么经济的反应没有以前那么强烈了。以前在上一个周期就是九零年代初期,你稍微降息,他就反应的很快,这个经济就复苏起来了。因为你这个以前借钱,在一九八零年那个利息高到什么15 percent 8 percent。但是那个到了,两千年,那个利息是百分之一。那你想想看,一般的企业如果能够以百分之一去借钱,他怎么会不借?对不对?所以他们这样子,放松银根应该放松了五六年,觉得奇怪,就是经济就是不起来。但是实体经济不起来,另一个金融泡沫又起来了。他这些放出去的这些钱被银行拿去给对冲基金,对冲基金又去炒作。然后,只不过是人性就是这样子,上一次炒作股票刚刚被弄了一脸灰,他们就不会再去炒股票。



他们这次炒的是什么房地产,你应该也记得。这房地产,而且一旦形成了一个产业链之后,就是他们发现这些mortgage backed securities(MBS),就是由这个房贷来背书的的债券,一旦这些生意做起来有了产业链之后,他们就想办法拼命要扩充,因为你的那个利润这么厚嘛,你只要房贷越多,你就赚的越多。就想办法去增加房贷,想办法增加房贷的办法就是次贷,次等房贷。就是那些从路上随便抓一个行人来说,你要不要这栋房子,这栋房子给我不需要你什么存款,你这个全部的那个百分之百都贷款,甚至可以超过百分之百的,就是就是不惜一切手段的把钱贷出去。你看这一次联储会就没办法收割了,他这个没办法到国外去收割了,他还是有收割但是不太多了,为什么?第一个是那些真正有钱的新工业国家,就是开发中国家他们都有外汇储备,他的外汇储备都比十年前要多出来五倍甚至十倍。另外一个是他那个放水放的很久,放到二零零五年。到二零零六年才开始,升利率开始回收,还来不及去挤压那些开发中国家,当然像拉丁美洲那种体质非常弱的,你随便有点风吹草动,他们都会倒,但是他没有什么油水嘛。所以这一次真正被他们收割的是谁?反而是德国。很好玩的,因为他们这些MBS有一半卖给了德国的银行。所以到时候那个2008年金融危机很快就传到欧洲去。因为很多欧洲的银行买了一大堆这些MBS,就是mortgage backed securities,这些次贷的证券,但是美国自己本身还是非常受伤的。受伤之后这一次的经济衰退比两千年那次还要严重许多。



我这边有这个数据,刚好这个他们的政权更替。所以奥巴马上台以后,收到是史无前例的一个烂摊子,上一次有这么严重,是一九二九年。当时美国的财政赤字,本来在clinton 的任内已经消除了,已经是有那个surplus,就是没有赤字的。就是他这个每年有盈余了,他们的那个财政有盈余,就是开始把国债开始削减了,已经削减了。那这个小Bush上台以后,当然共和党人都是都说他们是什么保守,那些人都骗人,一上台就开始乱花钱,然后一面替富豪减税,然后一面乱花钱。到那个小Bush要下台之前,他们的每年的财政赤字大概是三千四千亿。奥巴马一上台,两千零九年美国的经济真的是惨不忍睹。那个时候就业率到达二战之后最低点。你不要看trump说失业率现在很低,那个那个他们一天到晚讲说失业率是二战之后最低的,那是骗人的。你因为美国的失业率是指在找工作的人之中,没有工作的人,ok。你如果六个月还没找到工作以后,你就不算了。他把衡量失业率的标准改掉了。所以实际上要看这种就业失业的问题,要看就业率不是看失业率,因为就业率是用全部的人口来做那个分母的,不是在找工作的人做分母的。因为他们你只要六个月找到工作,就自动被排除出这个分母。那你当然这个分母跟分子,所以你自然这个失业率就上不去嘛,对不对?实际上美国的就业率现在还是处于历史的低端。Obama没有办法,他一次就把那个赤字从三千四千多亿提到一万四千多亿。



一万四千多亿当然那是创历史记录的赤字。然后接下来那三年,一直到我看一下,2012年还是一万多亿。光是这四年他的赤字就有五万亿。与此同时在联储会也开始所谓的量化宽松quantitative easing,为什么?因为他们在前一轮的那个两千年之后,发现降息已经没有什么效果了。所以必须要做更强力、更直接的手段,更强力、更直接的手段就是印钞票了。反正他们印钞票没有代价的嘛,对不对?因为他是国际储备货币,你不管怎么样,什么东西都是美元定价的。这个美元,你印的再多,他感觉上也没有贬值,所以那你就印。那他印了多少?印了四万多亿,从一直印到二零一五年。所以你可以看出,从他的财政上他丢了五万亿进去,从那个从货币上丢了四万亿,这样子美国的经济到什么时候才真正缓过气来了?到二零一四年、二零一五年才缓过气来,就是五、六年之后才缓过气来。然后到二零一六年又稍微有一点,全世界的那个经济又稍微掉下去了一点。Obama当然是很努力的,他不愿意让那个财政赤字这样子无限的涨下去。因为你这个国债累积到一个程度,就会有很严重的问题。



什么样的问题?全世界的国债最高的是日本,它有两百GDP的240\%,排名第二的是希腊180\%,排名第三的是意大利130\%。这个问题是什么?你如果超过了一百percent 就有问题,为什么?因为你的gdp 成长率是大约百分之二、百分之三,对这些国家来说,顶多百分之二点多。那你借钱的利率是多少,大概也是百分之二。所以你如果那个债务到达百分之百以上的,你那光利息呀就可以吃掉你的那个经济成长,就是利息大概是百分之二,你的经济成长是百分之二,你今年的经济是这么大,债务也是这么多,明年的经济大了百分之二,但是你的债务也多了百分之二,因为利息就是百分之二,所以你超过百分之百就会有问题。所以为什么日本到现在还可以这样子?因为日本人他们的那个债务都是内债,他们那个债不是跟外国人借的,所以他可以跟他可以挤自己的国民。但是为什么希腊就在一百八十percent 就垮了?因为他的那个债是外债,他都是用欧元来跟外国人借的,外国人不吃你这一套。



然后 Trump 上台以后,它马上又是大幅减税,然后大幅增加开支。你知道今年的那个财政赤字是多少?又是一万一千亿,又回到十年前的水准,就是一万多亿的水准。那你财政赤字怎么办?财政赤字你这个钱没有了,你就要借嘛,借就是所谓的treasury security, 就是联邦债券。联邦债券你如果赤字是一万多亿,你联邦债券就要发一万多亿,要不然你没有钱嘛,对不对?但是问题是量化宽松的那四万亿现在变成三万多亿,这些钱到哪里去了呀?这些钱被世界吸收去了。照理说,联储会在这一轮,他还是想向二十年前那样子去收割这些第三世界国家。但是问题是现在除了他们都有很强的外汇存底之外,有一个更大的问题,就是中国有三万亿的外汇储备。你要到哪一个国家用十分之一的价钱去收买他们的优质资产的时候,中国就过来说,我愿意出百分之九十的价钱。那这下子你还怎么收割?对不对,就是中国抢在美国的前面收割。就是中国愿意出接近合理价钱合理的价钱来买这些资产,对不对?美国人去收割这些东西是以十分之一的价钱去买,中国是以十分之九的价钱去买。那你想这还有什么好玩的?所以美国最近这几年那个酸葡萄酸的不得了。一天到晚就说你这个是什么所谓的贷款陷阱



史东(38:04)

我想光就这一点,我想光就这一点,美国就要把中国打趴才行,对不对?不然他这游戏玩不下去了。(他的这个经济周期用美元霸权去收割世界的这个游戏玩不下去了)。网上有两个字,我觉得非常传神,就描述你刚刚描述的,那个叫做截胡,我觉得非常棒



王孟源(38:29)

对,我自己不打麻将,不过我知道这是什么意思。(请继续讲下去,我只是觉得这个是很有意思,这也是再表示美国为什么不能坐视中国继续再往上走,一定要把中国打趴。)因为你当世界霸主有太多的好处,这只是金融跟经济上的好处。你还有政治、军事、外交上的好处,对不对?不过我们今天就是讲经济跟金融,所以我们继续讲这方面的,我们今天的标题是美国回购率,这个回购率是怎么回事?在九月十六号跟十七号两天,忽然一般来说是以年利率两percent的美国回购率呢,忽然一下涨到10\%



史东(39:16)

要麻烦孟源先跟我们解释一下什么是回购率



王孟源(39:22)

回购率是什么?回购叫做Repo。其实它是一个缩写,我也不晓得为什么会缩成这样的,其实它的意思是Repurchase Agreement。它这个没有什么学问,它很简单,就像当铺一样。你去贷款的话,一般的贷款你会拿一个抵押品,对不对?那但是你在贷款的过程中,这个抵押品还是你的。他们只不过是有一个lien, 英文叫做lien。 就是在法律上,他们可以说可以到法院去声明要扣押你的这个财产,如果你还不出钱来。但是这个财产还是你的,他们要拿到这个财产,必须要到法庭去走一个程序。这个程序还不一定走得过,你的手续要是有点问题的话,说不定就会拖延时间,甚至完全不过关。那这样当然是很不方便的,很不方便的话,那你看像当铺就有一个办法,就是你这个东西直接就卖给这个当铺。这个东西已经是当铺了,你到时候赎不回来,没有什么手续,就是当铺就会直接处置。Repo回购就是这么回事,他的意思就是说贷款,但是不算是贷款,而是直接这个抵押品就卖给你了。但是我们两个签一个契约,到时候我要用本金加利息的价钱把它买回来。这个好处就在于,你到时候如果没有钱去把它买回来的话,那个东西已经是借款,(他就省了一道手续嘛),就省了一道法庭这个麻烦、手续、风险跟时间。但是,他美国只用在一件事上面,就是大银行或者大金融机构之间彼此调度现金。因为这个是每天都要做的,而且用的那个金额非常大,是几万亿来计的。



他整个那个repo 的规模是两万亿,今年的规模是两万亿,因为他每天都要做,而且规模这么大,所以他有必要让律师来帮你弄这些什么契约,把它统统搞定。然后大家来做,否则的话一般还是贷款,房贷没有所谓的repo所谓的东西。所以说来说去,我们一讲到回购或者repo的话,就是大银行之间或大金融机构之间用现金互通有无。我前面已经讲到整个金融的基本目的,它的基本作用就是让现金流动。流动原本来说是从储蓄流动到银行,然后银行再贷款出去给企业。这样的这个流动是银行基本存在的意义,也是唯一对实体经济有贡献的的功能,这个功能非常的重要。就是你那个在实体经济里面,企业需要贷款融资才能够投资嘛,能投资才能够扩展产能。我们刚刚讲到这个回购率,在半个月前也半个多月前忽然就窜升,窜升的意思是什么?就是有现金的大银行不愿意借钱的。那需要现金的那些金融机构就借不到钱,他必须要出五倍的利率才能够借到钱。那接下来接下来彼此之间的那个现金的流动就断掉了。那这是很严重的。因为在二零零八年的时候,那个金融危机正式爆发的时候,第一个问题就是回购市场的那个回购利率窜升。然后整个就停顿。



史东(43:13)

我给你打一下岔,孟源。你刚刚说这个有些银行它不愿意借钱出去,他不愿意借钱出去的原因是什么?这就是最大的谜题了,对不对?



王孟源(43:25)

你问到关键点了,为什么他会不愿意借出去?那你现在如果去看新闻的话,他们事后就是说,因为有两个原因,一个是那两天刚好联邦政府拍卖了一批那个联邦债券,总额是多少?一千一百多亿,ok。那你这所谓卖债券是什么?我把这个债券给你,然后你的现金给我嘛,因为你帮政府要这个现金来花钱嘛。那也就是说美国的金融界这些银行吸血被那个联邦政府吸血吸了一千一百多亿。另外的原因,是因为九月是第三季度的最后一个月嘛,到九月的最后一天的话,他们必须要公开自己的账户,银行必须要公开自己的账户。那所以有一个传统,就是在最后那两个礼拜,通常是减少现金的外流。这样子你的账面比较好看。那他们还说另外还有一个原因是因为九月十五号也是大公司预付企业税的时间。以前都是因为美国的税,都这种税都是要预付的,而且是在每个月的十五号预付。那又有几百亿。我觉得实际上这些都是小问题,为什么?美国的赤字是多少?一万多亿。以前有没有一万多亿的赤字?有。那你只不过是一千亿的那个债券卖出去,怎么可能一下子这个休克了?因为你这个回购对外行人来说,好像第一次听到。可是你在内行人来说,你知道这个是最基本的,最重要的主干道。就好像你的那个动脉,是从心脏出来的那条主动脉,这个回购是这样子的的地位。那你这个东西一下子利率升了五倍,他的意思就是说你的这主动脉已经堵塞了,基本上打不出来了,很严重很严重的,怎么可能会这样子?美国的GDP是将近二十万亿了,一个二十万亿规模的经济,怎么可能它的金融业。而且我刚刚讲过,经过七十多年和平的资本累积,怎么可能为了一千多亿就休克,这是没有道理的。



但是你如果看长期的(背景?),而且那个什么企业税那几百亿, 然后后来有人讲说是因为有些大银行,尤其是JP Morgan ,JP Morgan其实目前是美国最大银行,他为了要让账面好看,所以留了几百亿,这种事情都是几百亿,几百亿加起来顶多连那个联邦债券两千亿,两千亿怎么可能就让你美国的金融体系完全休克了。我们算一算账好了,刚好有一个指标可以告诉你,那个银行手头上有多少现金,这个叫做 Excess Reserve。这个东西是11年前才发明的,就是在2008年的时候。那个时候,财政部跟年初会忙的手忙脚乱的时候,他们也做了一个小小的政策调整,就是他们后来觉得认为当时让银行的那个储备金要求的不够高。所以他们除了提高这个要求提高了一点之外,他们还鼓励银行保留超过法定准备金的额外现金,这个额外现金就叫做excess reserve,额外准备金,但是额外准备就是这个现金必须要放到联储会的账户里面去,就是基本上联储会就变成这些银行的银行,叫这些银行把你这个现金存在联储会的那个账户里面。以前没有这个要求,也没有这个习惯,为什么?因为你这样存的话没有利息嘛,对不对?他贷款出去,他可以拿百分之二百分之三的利息,你这个放在放在联储会。这可是我们讲的可是几千亿及几万亿的美元的的事情,这个一年的利息就是几百亿,这个不是开玩笑的。所以联储会在二零零八年的十月,为了鼓励这些大银行多留现金,决定开始向这些银行的所存的准备金付利息付百大约百分之二的利息。就是随着他的那个利率调整,就是那个短期的利息不是太多。可是至少,你不是做完全亏本的生意。那这样一来,从2008年完全没有所谓的excess reserve就是额外准备金,当时是零,一直涨涨到。我刚刚讲到那个美国从那个上次金金融危机开始恢复元气到元气最足的时候,是2014年。



到了2014年的时候,我去看这个excess reserve,你会发现他那时候到达顶点,顶点是多少?两万七千。这就是为什么我刚刚说这个两千亿真的应该是小钱。你就会看这些银行,他们的额外准备金,就是不是法律规定必须要有的现金就有两万一千多亿。那你怎么可能为两千亿而活不下去?从那之后,我刚刚说到,联储会开始回收了九千亿,然后 Trump又把他的财政赤字从四千多亿一下冲到一万一千多亿。然后你这些发的这些联邦债券,他发了大概,必须比以前的这个步骤就是奥巴马的步骤,大概要多花大概一万三四千,就是过去这两三年多发了一万三四千亿的的债券。这其中有大概三分之一是外国银行,外国的中央银行就像日本跟中国,他从这里面大概拿了六千亿。然后剩下的那些,我刚才说一万三千亿,剩下的大概有七千亿,这个就必须从美国国内的那个银行吸了,然后你再加上联储会的九千亿,一共吸了一万六千亿。从美国国内的银行业,总之是过去在过去这两三年,美国的联储会再加财政部一共是向美国的银行界吸了大概一万六千亿左右,就是九千亿再加上七千亿,一万六千亿左右。对,然后我们去看看那个美国银行界的那个excess researve,就是他们手头上的现金、闲钱,现金从两万七千亿,现在你知道是多少吗?今年是一万两千亿,刚好掉了一万五千亿,几乎是完全一样。



也就是说联储会一起,然后那个美国财政部发债券,除了外国中央银行认购去当做他们的外汇储备之外,基本没有在国外收割的效益。那些剩下的那一一万六千多亿,基本上是从美国银行借账吸走,美国银行界损失了超过一半的闲钱。就是因为这样子,所以他们才不愿意互相借贷,为什么?因为他们看出苗头不对,这个经济衰退的苗头已经已经来了。那在2008年的时候,美国的银行界倒了一片了。但是有一家银行大赚特赚,你知道记不记得是哪一家了?高盛在2008年的时候,他提前半年预见了这个经济衰退,就是泡沫要爆炸。所以他把他的那个资产重新调整,调整之后反而大赚了一点。现在还有一部电影叫the big short,你如果有兴趣的话,去看一下。(这个电影不错)就是讲那个高盛怎么跟这些对冲基金合谋来从那个经济衰退泡沫爆破里面赚钱。我的感觉是,因为美国的联储会跟财政部从美国的银行界过去这三年吸了一万六千亿美元现金,然后这些大银行又看到经济衰退的苗头不对,所以他们都想要学十几年前高盛那个样,所以最起码你要准备经济衰退,最起码是什么?保护手头的现金,不要再借贷出去。因为你到时候你不晓得能不能收的回来,对不对?那个企业可能垮掉,那个对手的银行可能像雷曼兄弟或者其他的那些小银行那样子破产。所以讲来讲去这一叶知秋,你可以看得出从这个回购率窜升,而且联储会马上下手去挽救,他挽救到目前为止大概投入了五千亿,但你这个新闻看不到的,因为他不敢讲。你要去间接的去计算,估计大约是五千亿。



史东(53:36)

我这儿有个消息,他是华尔街日报在十月八号这一天,他说Fed Adds \$76.4 Billion,这个是不是印证了你说的那句话?



王孟源(53:53)

对,那是七百多亿,那是一天。他已经搞了三个礼拜了,我估计是大概五千亿的那个全部投入。照理说我们刚刚提到那些短期的因素里面,有两个是季节性的。一个是因为他到第三季度尾部,他会必须要 window dressing,必须要把账面做的好看一点。另外一点是那个企业要交税。这些照理说你到了第四季就是十月了,就应该不是一个问题。可是你你看联储会下手去挽救的,他不敢讲他那个一共买了多少。但是他的那个挽救的规模基本上没有大幅的减小。到了十月已经三个礼拜了,还没有大幅减小。他现在最新的是说预估会一直到十一月,就是说这个问题远比联储会愿意承认的还要严重。为什么?他实际的问题严重到就是我刚刚讲的一万六千亿,不是那两千亿的问题,而是一万六千亿的问题。那也就是说美国的金融体系已经没有足够的现金来借贷了。换句话说,现在纯粹就看经济本身,就是金融,已经没有办法去帮助经济放贷来成长。那全看你经济本身的这个惯性往前冲。



但事实上美国的经济本身的一些指数,也都是已经开始连续下降。当然这个中美的贸易战我刚刚讲过,两边大概都损失超过五千亿以上,这个也是一个很大的问题。但是这些东西累加起来,基本上美国的那个经济衰退是躲不掉了躲不掉。但是它不会像2008年那么严重,为什么?因为二零零八年是很多金融机构破产,实值破产。那这一次没有,这一次主要不是债券的问题,这次也不是消费者的问题。那个二零零八年一个很大的问题是那个房贷,房贷主要是消费者的房贷嘛,那这一次的问题主要是企业,企业它是那个股市跟那个垃圾债券junk bond,他发的太多了,发的太多以后,他只要那个盈利,它的利润下降,它就会有问题。这个股市的泡沫就会爆破,它的那个junk bond就必须要就发不出去,那你就没办法roll over,没办法延展。所以我的预见是在明年头半年就会进入经济衰退。然后这个经济衰退的严重程度,它的进程大概是跟两千年那个一样。就是没有两千零八年的那一种,这是短期来看。



但是长期来看,问题是这样子,我们看这个经济的很多人不了解,连很多经济学家都不了解这个现代金融对现代经济的这个影响力之大,尤其对美国这个经济。因为主要的原因就是我刚刚讲的,我在节目刚开始讲过两点。第一个是有七十多年累积这个资本,这个资本跟经济规模的比例已经是很大,然后再加上过去的这三十年的金融创新有一大堆杠杆,而且不是明显的杠杆,你看不出来。但实际上它让你十亿美元可以买一千亿美元的资产,这种杠杆到处都是。所以实际上那个金融对经济的影响,尤其是对美国这种自由主义经济的影响非常非常的大。



金融说来说去就是现金流。那现金有两个来源,一个是政府行政就是财政部,另外一个是发货币的就是中央银行或者是联储会。我稍早提到,在两千年泡沫爆破之后,联储会就已经感觉到奇怪。我必须要放松银根好久那个景气才慢慢上升出来。到两千零八年之后更惨,足足有六年这个经济才复苏变成真正景气。而在这其中这个联储会的账面从零一直增加到两千零八年的八千亿,到现在的大概四万亿,越来越多。这个东西看起来好像是没有边际成本嘛,就是印钞票好像没有边际成本。实际上你到达某一个程度以后,它是那种突变性的。就是现在没有成本,然后一下动就上去了,这个就是美元不再被世界尊重,而 Trump现在搞的这些外交跟那个地缘政治都是到处得罪人嘛。那个俄国跟土耳其,昨天俄国跟土耳其才签约互相的贸易停用美元。你这样一点一点的累积到最后,大家都要改用欧元了,或者改用自己的货币了。你美元的地位慢慢被侵蚀。



财政部搞赤字,联储会搞量化宽松,这些都是在自掘坟墓,就是饮鸩止渴,只不过是这个这个毒药是很慢性的毒药。那个现在你想想看,他从四万五千亿降到现在三万六千亿,然后又涨回到,可能到年底可能是三万八千亿元。然后明年经济衰退他能怎么办?绝对光降息是不够的。你看现在它的利息多少,一点七五,今这个月月底一定会降到一点五,到十二月还有另外一次很可能会降到一点二五,现在那个美国财经新闻还不能面对现实,还在说不一定会降息。我跟你讲绝对会降,到年底绝对是一点二五,他们还估计明年年底是大概一点零或者一点二五之间,这都是骗人的不可能的。明年年底绝对到零点五以下,而且这个你降到零点五以下,这个利息就不能再降了。为什么利息不能降到零以下嘛?降到零以下很麻烦的。我借给你钱,借你一百块给你,你到时候只要还我九十九块。最大的问题是,你就没办法吸收那个散户的存款了,所以你这个利息一旦降到大概零点五左右,他就必须要开始做量化宽松。所以明年的年终大概就会开始做量化宽松。那这个量化宽松一做,上一轮做了四万亿,这次至少也要做个两三万亿才够。然后trump这个赤字今年是一万一千多亿,这个绝对还会继续上涨,那你说这两方面这样一挤,他的这两三万亿真的能够刺激经济吗?不行的,为什么? trump的赤字一年就是一万多亿?刚好两年下来就把你这些你放水的这两万亿通通吸走了。



王孟源(01:01:42)

吸走是什么意思?你印了两万多亿的钞票,全部被联邦银行拿去用赤字把它花掉,这就是我所说的饮鸩止渴。而且时机非常的糟糕。我们现在已经到了一个临界点了,就是在金融上,联储会已经是无计可施,只能够继续的加药,加码加量,这样不断的增加,不断印钞票下去。那个trump 这个赤字主要还是他乱花,他减税乱花钱。但是你如果去看美国真正的财政炸弹,都不是这些,美国真正的财政炸弹是那三个社会福利,就是social security,medicare and medicaid 这三项在未来这二十年,就是刚好从明年开始会从远远慢慢的这样上升,会突然这样增速。就是从二零二零年到二零四零年之间,这三项花费增加之快你会吓坏了。就是在十年之内这个他们就会增加超过一万亿,就是每年的超过一万亿。就是你如果现在有一万亿的盈余,十年之后就没有了。但是美国现在的财政有一万块亿的盈余吗?没有,他有一万一千亿的赤字。那你想想看,十年之后就是三万亿,然后你再看看他现在的国债是多少,百分之一百零八。我刚刚已经讲到,这还是没有算到他那个利息,他的那个国债已经到一百零八,就是已经接近意大利的水准了。那你再经过十年这样子,他的这个国债就会到达百分之一百三十到一百四十左右,就是完全到现在意大利的水准。所以十年之后,我刚刚说过这个周期大概现在变成十年了。十年之后的下一个经济衰退会非常的惨。



史东(01:03:58)

十年之后是从现在开始看十年,还是从



王孟源(01:04:00)

从现在开始十年。我的意思是说,从冷战之后,现在这个经济周期大概是十年一次嘛,2000年一次经济危机,2008。然后2020年眼看着有一次经济衰退,再下一个应该是大概2030年。2020年这个他们还可以蒙混过关,因为这个时候美元还是有那个世界头号货币的这个头衔,他这个还可以从四万亿再多印两三万亿,没有什么太大的问题。但是在这期间,他的国债会增长的非常的快。就是国债会从现在的108\% 增长到130\%,甚至140\%。到了那个地步,到了二零三零年再有经济衰退的时候,他们那个时候每年的利息本身就增加到一万亿。因为那个时候,不但本金增加,利率也必须增加。因为现在的利率会这么低,还是因为美元的那个国际储备货货币的那个的地位嘛。现在的那个十年期的财政部的借款利率是多少,一点五,那吓死人了,低的不得了,不可能再往下掉。十年之后,等到美元的地位被动摇的时候,一定会往上涨。只要稍微涨到比如说二点多,他的那个国债的利息就会超过一万亿。那超过一万亿以后,你这个就是滚雪球了指数成长了。美国的这个经济跟金融的衰退也会非常的快。当然他还是有军事的优势,这个军事永远都是最后一个衰退掉的。在老霸权里面永远都是最后一个衰退掉的。所以我们到时候再看。我觉得中国的这个长期兴起,已经是胜利在望了嘛。就是现在这个中美贸易战,你就是必须要扎稳脚步,准备长期抗战,美国水来土掩兵来将挡,他们要干涉香港,你就杯葛他们的企业,反正没有撕破脸,反正你来对我搞宣传颠覆这一套,我就不让你赚钱,对不对?我不花钱给你总可以的



史东(01:06:48)

那就是看谁的气长了



王孟源(01:06:52)

对,我我今天讲了这么多数字,就是要跟你讲,美国的气没有那么长了。美国到二零三零年他就会有断气的那个感觉了。因为你的这个国债不能够这样子无限的增加上去,所以你只要看看美国金融跟经济的这个现况,然后延展十年之后,你就知道。再十年,眼看着明年的这个衰退,然后在十年后的下一个衰退他之间就会有一个很大很大的落差。就是从目前还可以勉强维持,反正就是继续借债。然后到十年之后他们就没办法了。十年之后很明显的,他们就必须要face the music。



史东(01:07:41)

你现在看这个事情,在美国的情况,你觉得这个情况还是可以有reversible,就是有回转的机会吗?还是说这个基本上已经已经是已经是过了这个桥了,没有这个店了,已经就是这样子下去了



王孟源(01:08:01)

如果四年之前是Hillary 当选,或许还可能做一点事情来避免国债上升,但是 Trump的这么一胡搞,真的下一任总统会很难很难做。然后要征税,你看看每任的共和党,那个民主党人的那个总统总是设法要征税来弥补国债,但是共和党减的税永远都比他征的税多,而且征税是征到中产阶级头上,减税减的是巨富。所以你那个然后在这个过程之中,中产阶级又萎缩。因为你那个我刚刚说过工作外包了嘛,所以中产阶级越来越少,他的那个资产越来越少,这个没办法延续下去。而且他这个政治体制,两党的恶斗,然后整个宣传体系被财阀控制,都对我来说,他能够扭转能够面对这个现实来做所需的改革的机遇,几乎是零近乎是零。最大的可能是他们会继续转移矛盾,就是把这些经济问题,为什么当初会找中国做替罪羔羊?那些精英是知道中国会威胁他的霸权,但是川普不在乎这些事情,川普什么都不懂。川普中国做替罪羔羊的唯一理由就是哗众取宠。而他自己本身对经济的了解,就跟他们手底下那些农民一样。他就是觉得这些问题,他们的经济问题都是中国竞争带来的。



史东(01:09:56)

找一个代罪羔羊嘛。



王孟源(01:09:58)

你看上个月我说最近这一轮的中美贸易和谈不是很乐观。最好最好的希望就是能够把时间调回到六月,就是从六月到九月之间所新加的那些关税通通免掉,那是最理想的结果。你看现在刘鹤到了到了华盛顿,他昨天才到,他连特使这个名义都没有拿。就是这一次非常的低调,他自己很明显的,他的确就是没有什么太大的指望。为什么?trump已经到最后一年了,你跟他第一个不可能,因为他就是搞民粹的他不可能。我想中国的领导阶级现在终于认识。在经过一年半两年之后,终于认识到美国的仇中情绪是根深蒂固。有十几年的有系统的鼓动,根本没有办法不可能跟他们做真正的和解。然后,第二就是trump到明年很可能就会落选。那你现在跟他谈有什么好谈的,反正明年民主党的人上台以后,第一件事情就是把关税免掉。并不是说他们对中国有什么好感,而是他们因为这是共和党搞出来的事情。他们为了表示他们不一样,而且在竞选的过程中一定会拿出来抨击,说你你危害到农民的利益。所以他作为象征性的一定会把它搞掉。那你现在就没有必要做太大的牺牲去来谈这些事情。



史东(01:11:45)

我们今天谈了很多,我想现在收个网。我们今天谈的所谓的一叶知秋。我们谈了一叶,我们也谈了一下秋,然后能不能给我讲解一下这些事情。整个这个大环境的转变,因为这是个很大很大的事情。对于一个普通小老百姓,像我这种小老百姓,以及看这个节目的小老百姓来,说到底有什么样的意义?



王孟源(01:12:19)

没有办法,因为美国的整个政治经济体系被财阀把持了以后,中产阶级被压榨是没有办法的事情。你在政治上、经济上都没有办法还手,因为他们这个所谓自由主义经济跟政治的,他的真正目的就是要消除弱者的还手的能力。因为你如果是自由竞争的话,一定是谁的钱多谁赢了,对不对?美国吃他的老本,过去这三十年吃他的老本,这个这种生活是无法无限延迟下去的。就是十年之后就是到二零三零年的那个经济衰退,就会很明显的看出来,到了二零四零年,真的大概到时候就无暇外顾了。中国在这段时间,未来二十年,基本上只要管好自己的事情,因为中国自己本身面临的老龄化问题比美国还要严重。所以自己应该未雨绸缪,赶快的做产业升级。事实上中国政府一直都知道,产业升级是第一优先嘛。像是什么祖国统一或者其他这些事情都是次要的。最重要的是产业升级。产业升级以后才有经济的底气,才有财政的底气来处理老龄化跟城市化的这些问题。你看日本现在有多惨,美国基本上就是在步日本的后尘。美国其实在政策上没有比日本好。他的唯一能够比日本有这个底气,是因为他是世界霸主,他有美元跟美军可以靠,没有人能搞他,只有他自己能够把自己搞死,而且他还可以大印钞票。你想想看,从九零年一直到现在,大印钞票已经三十年了,然后未来十年会印的更多,印了四十年的钞票。其实从那个尼克森开始就开始印钞票,就是他取消那个金本位bretton woods。但是他基本上只是在七零年代印了一些,到了这个七零年代八零年代所印的,比起现在来说真的是小巫见大巫了。



史东(01:14:41)

所以就是好比一个人不断的在吃兴奋剂嘛。吃久了就没有效了。



王孟源(01:14:50)

因为他本身就是drug dealer毒贩,所以他这个兴奋剂是没有成本的。但是,这并不代表你可以无限地吃下去。



史东(01:15:05)

那么在这个情况之下,相对而言,在世界的舞台上能够取代,我想取代这个词肯定用的不太对。能够怎么讲还是用取代好了,取代美国地位在经济上的、金融上的只有中国了。我看不到第二国家



王孟源(01:15:20)

只有中国。印度很多人都在捧他,我觉得是很好笑的。这个一年之前,刚好一年之前我写了一篇文章,你知道我写的东西很杂,但是刚好去年十月我写了一篇文章,说印度有一个很大的影子银行倒闭了,他们的印度政府必须出手。我这照理说是一个小新闻。我的博客到现在五年也才只不过两百多篇文章,为什么会这样去写?因为我知道这种金融的波动,很可能就是一个经济衰退的前兆。就好像我们今天讲的对这个回购率窜升,就代表现金不足,就代表美国的金融体系已经没有能力去帮助经济继续高速成长。印度也是一样,刚好一年之前我看到这个影子银行出了问题,我就知道印度这几年也是虚胖。然后他的一旦金融管道的那个现金流动出了问题,就很可能经济衰退。果然今年这个现在。本季的那个成长率已经降到百分之三点多了,就是从原本的七点多降到三点多,这还是那种印度式的统计的结果,所以实际上可能是更惨。



基本上就是中国,我希望中国继续这个一带一路,就是用双赢的这种态度。因为英国跟美国这些霸权国家,他们的老本是哪里来的?基本上是全球掠夺来的。那中国如果真正在未来十年成为一个新的世界霸主,它可能是世界上第一个不是靠对外掠掠夺的一个霸主。那你想想看,这对全世界人类是多大的一个好事,对不对?



我一直说全世界在二十一世纪有三个大的问题。第一个是贫富不均,第二个是全球气候变化,第三个是霸权交替。霸权交替刚好是解决前两个更严重的问题的一个前提。因为美国这么自私,他连巴黎条约都不愿意尊重你,怎么能够解决全球转化的问题?必须要有中国这样子,有公平的天下为公的这种态度的的新霸主,才能够领导世界来解决共同的问题。至于贫富不均,你看美国本身就是贫富不均的始作俑者,他是全球富国和贫国之间差距的创造者。所以你指望他们解决这个问题,就是缘木求鱼,只有你只能够希望中国将来能够为人类社会解决这些问题。



史东(01:18:12)

是的,说的非常好。孟源再一次的谢谢你,谢谢你。(王孟源:今天讲的数值太多)我觉得很好,因为这是一个非常不容易表达的题目,但是我又觉得非常重要的一个题目。所以说这个这个节目我喜欢做我现在要的节目,就是说我不用去向广告商磕头,我只向我的观众负责,所以就会有这种节目出现,你懂我意思吗?(王孟源:技术性太高了) 就真的这是一个必须谈的事情。但是观众能够吸收多少,好在这个节目会永远放在外面。你能够吸收多少就吸收多少。



王孟源(01:19:01)

重点是美国现在是虚胖,很严重的虚胖,而这个虚胖你从金融上就可以看得出来,然后所以它的经济会有很大的问题,



史东(01:19:10)

非常好。孟源,谢谢



\twocolumn[\begin{@twocolumnfalse}
\section{英国脱欧}
\subsection{20191027}
\end{@twocolumnfalse}]史东 00:12 

各位朋友,你好,我是史东,在今天节目中,我想跟您谈谈脱欧这件事情,我稍微把这个事情很快地给您介绍一下,我想你一定知道了,而且你已经可能听得很烦了,因为这个事情弄了现在已经大概 3 年多了。这是在 2016 年6月 23 号,这一天,英国有一个公投叫做脱欧公投,投票率很多很高,有 72\%。 投票的结果赞成脱欧的是52\%,事实上 51. 9\%,但是一般来说他们讲是52\%,然后赞成留欧的是48\%。从那个时候到现在刚刚谈到已经三年四个月了,在这三年四个月里面,英国消耗了可以说是两个首相,现在是第三位首相。



史东 01:00 

那么这个事情到底是怎么回事?它的前因是什么?它的后果会是如何?它对于全世界会有什么样的影响?今天我们的这个来宾,我们的好朋友,他在他的这个部落格上曾经写过一篇文章,不过在今天节目中会有一些更新的信息会加进来和我们介绍。这个问题。我们的朋友就是王孟源王先生。首先跟王先生打一声招呼,梦圆说一声谢谢,说一声欢迎。



王孟源 01:26 

很荣幸再上你的节目。



史东 01:28 

是的,有关于这个议题,主要的要看你来跟我们解释解释这个事情的前因和后果了。



王孟源 01:36 

其实我会从更早讲起。这件事情不是凭空冒出来的,所以在我开始之前,我想提一下我们两个多礼拜前做的有关 Repo 回购的利率上升的那一件事情。最新的消息是两天前,美国的联储会把每天的 repo operation……我不晓你记不记得,在我们节目中你有一个printout,那说他们当时每一天的 repo operation 是 800 多亿……那现在刚刚两天前增加到 1200 亿。



史东 02:17 

对,那个消息我看到了。



王孟源 02:19 

换句话说事情不但没有缓解,而且是越来越糟糕,最新的消息是欧盟,欧元银行,跟英国中央银行,还有中国的人民银行都已经开始做他们的 QE quantitative Easing。对,所以这件事发生的基本上就是我两个多礼拜前讨论的。



王孟源 02:47 

对,但是发生得非常快,就是。



史东 02:50 

这个 QE的动作是不是他们做最坏打算的动作?



王孟源 02:54 

我想基本上就是在过去这两个礼拜,所有的中央银行都感觉到苗头不对了,就是他们大概在内部也做了个跟我所讲的同样的分析,然后也都知道大概全球性的经济衰退大概明年上半年是跑不掉了,所以他们都希望能够赶快动手来缓解这个那个impact。所以我想基本上这个反映的就是我在两个多礼拜前已经说过,那时候我认为有部分的原因是那些银行他们觉得苗头不对,他们不但手头上现金很紧,而且觉得苗头不对,这个会有一个经济衰退,所以他们不愿意放弃他们的现金。实际上的原因大家所有的中央银行大概都感觉到了,就是现在有现金短缺,还有经济衰退的问题。所以我先利用 5 分钟从上一集到现在两个多礼拜的时间的新发展讨论一下,我想常常看我博客或者是看我视频的读者,听众会注意到,我通常只会列举两三个重点,然后就从这两三个重点来做预测。我觉得这是科学的真谛,因为科学就是所谓的要能够证伪,就是你要能够做一个假设,这个是假设,在我们这种时事评论上面就是所谓的预测,你做出一个预测,然后对不对?然后如果是对的,就对了,如果是错的话,你回过来反省说你做了哪些错误的假设,这是一个科学的态度。但是很不幸的,一般的那个时评的或者是人文社会的研究者,他的态度不是这样子。



王孟源 04:47 

那个其实两个月前大陆有一个很有名的张文木教授,他是战略方面的专家,他写了一篇文章批评大陆有很多亡国的学者,他说这些亡国学者他们就是写 八股文,他们有什么事情,比如说像上个上次讨论的那个回购的问题,或者是我们这次在讨论的脱欧的问题,他们就把所有的可能性十几个、二十几个可能性全部列举出来,然后也不说哪一个比较可能,也不说哪一个会有什么样的影响。这样子他们完全没有任何预测,也就是说不可能是错的,他们都是在说废话。



王孟源 05:28 

那所以他把他叫做亡国的学者,张文木教授不是理工科学出身,他不是学科学的,但是因为他是一个出色的学者,所以他本能地就了解到你这样做是没有意义的。这事实上是过去这半个多世纪,在美国它这整个社会腐化之下,那英美的学术界出现的现象,中国照理说是一个新兴的国家,就新兴的工业国,它应该能够有这种,自省的能力,所以我是希望自己也能够树立一个榜样。就是说一天到晚说废话对社会是有腐蚀效应的,因为你社会必须要能够讨论真实的事情,而这些人偏偏,如果你愿意出来说实话做实际的预测,他还说你以偏概全。他们的全,就是所有的可能性都讨论一遍,然后不做任何实际的预测,这他们叫做全,我觉得这是很可笑的。所以我想,今天的讨论其实是一个很好的示范。我同样也会只抓英国脱欧背后的 两个原因,这两个原因就逐一解释我们现在看到的几百个、几千个现象,所以这就是科学的精神,你用一个最简单的理论,抓住最要紧的重点,然后能解释最多的现象。



王孟源 06:59 

其实在我开始之前,我先举一个更简单的例子,两年前巴拿马决定跟中华民国断交,跟台湾断交,然后承认中共的中华人民共和国的政权。那时候其实很多人很吃惊,因为巴拿马不但是美国控制得很紧的, 100 多年来控制,从他19 世纪,就是因为要建巴马运河,所以从美国硬是把它从哥伦比亚割出去的,从那一开始就会控制得紧紧的。而且在中南美洲,在中美洲台湾的邦交国之中,其实尼加拉瓜是跟中共向来最友善的。但很奇怪的是,为什么?为什么两年前第一个变节的是巴拿马?那当然你可以说整体的环境是中大陆的一带一路战略非常的成功,然后它的经济的那个规模越来越大,所以吸引力也越来越大。但是你这样没办法解释为什么同样的吸引力对尼加拉瓜没有效,而对巴拿马反而更先有效。



王孟源 08:13 

我跟你讲一个小故事,你也许就会同意,这是背后的重点:在 50 年代、 60 年代的时候,那时候全球化刚刚开始,那时候然后又有标准的集装箱开始运,所以全球海运有一段很高的成长期,在成长出来的时候,那些货船以前原本是英国的货船就挂英国旗,美国的货船挂美国旗,但是那时候开始流行一个办法,就是所谓的 flag up convenience,就是你挂方便旗,这个就是一个小国家,然后把主权出租出去。你可以挂一个旗,然后因为这个国家跟人家没有冲突,所以他要到苏联那边去,也可以去,美国那里去,也可以,很方便;然后花的钱不多。那后来很快地发展出来,领先第一位的就是Liberia,OK,赖比瑞亚(编注:大陆翻译为利比里亚)。但是很不幸地到了 1997 年,赖比瑞亚的内战产生了一个总统,叫做Taylor。



王孟源 09:21 

我不晓得你记不记得,他后来因为违反人权被欧盟跟联合国追讨了一阵子,所以他在位的那六年,从 1997 年到 2003 年, Liberia 的国际身份就非常的尴尬,因为它是被考虑到是一个违反人权的政权。它的这个在货船宗主国的那个登记上面的生意就一落千丈,结果六年之间他就从以前占一半多掉到只剩下 1/ 10 的生意,这个生意是被谁抢走了?就是巴拿马,巴拿马一下子变成世界排名第一的,这个这门生意那每年大概就是 5000 万,但是这不是 5000 万的经济,而是 5000 万的财政收入。就是你想想看这些国家,他们通常税收的效率不是很高,你要一下子产生 5000 万,而且两三个人、一台电脑就可以办好的事情,这样的生意你等于是经济上要发展好几亿,甚至上 10 亿才能够有这样的……



史东 10:34 

就是纯盈余。



王孟源 10:39 

纯盈余,对,是非常有吸引力的钱。那到了 2003 年, Liberia 终于稳定下来,他们做的第一件事情就是跟大陆复交,那时候这已经是十几年前的事了,大家也许不记得,他复交以后马上就第一件事就是签一条约,就是让 Liberia 挂旗的货船进大陆的时候有特别的方便,就是费用还有程序上面都有特别方便,这样子他才能够慢慢地有点竞争力。到现在他每年进账大概 2000 万,但是巴拿马还是排名第一的。巴拿马跟台湾断交的时候,刚好是Liberia,正在那个条约已经快要到齐了, 15 年快要到齐了。正在重新签约啊,我想我讲到这里你应该可以了解我要说什么,哈哈哈,因为大陆这几年它的那个经济地位,还有在国际贸易上的地位实在是已经首屈一指了,对不对?那你这个如果要干这种货传中主国的生意的话,你第一个要讨好的政权就是大陆,所以巴拿马就我想就很自然的那时候承受到比宁加拉瓜要大很多的压力。好,讲到这里,顺便再谈一下那个现在排名还是巴拿马第一,Liberia第二,那第三的一个非常非常小的国家,Marshall Island,马绍尔,那个太平洋中的一个小国,上个月刚刚跟大陆复交,跟台湾断交的是Solomon Islands,他们倒没有这个生意。那这些Solomon Islands跟 Marshall Island 都是等于是澳洲的后院,因为他们基本上都是还是执行帝国主义的外交政策,这也都有自己的外交后院。



王孟源 12:34 

我的感觉是如果 Marshall island 在明年跟台湾断交,然后跑去跟大陆建交的话,大家不要太吃惊,因为事实上从当选开始到现在,大陆已经不再搞这种外交战了,所以他没有再去特意地追求这种事情,都是愿者上钩,姜太公钓鱼的方式,那他们这些鱼哪一个愿意上钩?就看他的经济,哪一个对大陆的依赖最深了,对不对?所以你看我讨论这件事,讨论巴拿马跟尼加拉瓜的比较之间,我只讲了一条,但是在经济跟政治上,你可以讨论几百甚至几千起的,但是我觉得,我的贡献就是我去思考这些问题,研究这些问题,我把最重要的一条抓出来,然然后看看这一条能不能够对未来的事情做出预测。这就是科学的方法,也是我一辈子的原则。那我想如果有人觉得我的博客做的预测比较精确的话,那这是一个重要的原因,那我今天讨论英国脱欧也是从这里开始。。



王孟源 13:49 

我觉得,脱欧有两个背景,一长一短。我们先讲长的,长的是英国从大概 18 世纪,甚至可以上溯到 17 世纪,就是 17、18 世纪的英国跟荷兰先争夺海权,后来跟那个西班牙争夺海权,然后最后跟法国争夺海权。那个时候他在欧洲大陆上就采用所谓的 Continental balance of power 大陆平权战略。这个意思就是如果欧洲大陆去上有哪一个强权开始挑一枝独秀的话,那英国就去支持挑战他的人,要确立这个大陆是分裂的,这对英国非常的好。因为等到整个西欧都工业革命之后,欧洲欧陆始终没有办法整合成一个强大的整体。来挑战英国的领先,因为英国是第一个进入工位革命,当然到了一战的时候,很不幸的这个策略玩过火了,变成他去压制德国,结果打得两败俱伤,后来让美国捡了便宜。



王孟源 15:15 

到了二战之后,到 1956 年发生了苏伊斯的运河危机,那时候英国跟法国去入侵埃及,就是搞以前的老把戏,结果艾森豪不答应了。他倒不是说他喜欢埃及,而是他觉得你们动手做这种事情,没有经过我的批准是很不敬的,所以他就出手教训了一下,就是叫那个 IMF 国际货币基金会把给英国的贷款召回。那英国还不出钱来,还能说什么呢?对不对?就只能乖乖地听命,从那开始,美国才真正确立是在西方世界一超独霸,但是这个英国虽然放弃了他的霸权,那 1956 年之后正式放弃他的霸权了,但是他的这个民族记忆,他们的历史还是一直搞那个 continental balance of power 他们一直记着,就是他们没有完全忘记往日过往帝国的荣光。



王孟源 16:19 

到了 1970 年代,欧盟在酝酿的时候,他们就一副扭扭捏捏,不太心甘情愿的样子,后来最后还是比较有理性,因为那时候的政客还是愿意为国家做正确的大决定,那最后他们还是加入了欧盟。但是他在欧盟里面的地位一直都是很特别的,都一直都有很多特别的条款,就是有一些欧盟的法律对他不适适不适用之类的。其实英国到现在已经没有理由再去拆开欧盟了,但是因为这些历史上的关系。



史东 17:00 

欧盟什么?是和欧盟拆开?



王孟源 17:02 

就是像Trumo他想要搞的就是在欧盟内部搞革命,去分裂肢解它,但是因为他们有这个历史传承,所以还是有一股惯性,尤其是老一辈的人。那这个是不是一个很重要的因素?就是如果短期内,就是最近没有什么导火线的话,其实大家淡忘了,老一代慢慢凋零,新一代的英国人其实都是认为自己是欧洲人,尤其是你欧盟,其实大家都讲英文了,已经给你这么客气了,用你的语文了,你还能要说什么?对不对?



王孟源 17:55

这个真正短期的问题其实是我在我的那两篇文章里面详细讨论了,就是真正的是在过去这十几年,因为欧盟开始查税查得紧了,OK,查得紧以后,他们针对的主要目标其实不是英国人,他们主要目标有两个,一个是欧洲大陆上的富豪,他们一般逃税的方法就是把钱藏到瑞士去。所以、欧盟在过去十几年,对瑞士施加了很大的压力,让瑞士的银行秘密法、银行保密法变得千疮百孔。就是基本上欧盟如果发一个正式公文的话,他们必须要把消息透露出来。另外,一个抓逃税的是针对像 apple 这样的,OK, apple 的那个 iPhone 的利润非常非常的高,那照理说这个利润是你在哪里零售就是在哪里赚的,对不对?你在法国卖了几百万台,那每一台的利润有,比如说三四百美元,那你这样的利润就是三四亿了,对不对?一年的利润就有好几亿,但是他不在法国交税。apples 是这样的,他们在税率很低的国家或地域,比如说像爱尔兰,或者是甚至更远,比如说那个避税天堂,那基本上是 0 税率的。那你在那边设立一个法人,然后让这个把你的那个智慧财产权统统交到这个法人上面。然后你的在欧盟卖东西赚钱的时候,你要求当地的分销商或者总经销付很高的专利费,或者是智慧财产权的使用权、使用费,那这样一来在当地的利润至少在会计账面上就几乎是0。 apple 的右手付给左手,他的右手是法国的分布,然后比如说爱尔兰的税率是10\%,但是法国的税率是百分之 40 或者30,那你这个他的那个法国的分布就必须要把基本上所有的利润都名义上转到爱尔兰去,那他会计上的那个利润就是记录在爱尔兰去。或者更过分的是你可以放在那些避税天堂,那个税率是那里的企业税是0,对不对?那你根本就……



史东 20:30 

有时候我们看到有媒体上报道说欧盟,对,譬如说 apple 罚钱,主要是这个原因吗?



王孟源 20:37 

这是其中一个原因,他们有两个原因,一个是避税,另外一个是他的托拉斯,就是比如说他搞鬼,让人家不能够跟你光明竞争这样子。所以欧盟在过去这十几年也对apple,还有刚好这些新的高科技公司、软件跟那个互联网的公司都是美国的,所以他们都搞这种玩意,那欧盟就非常的不满。那欧盟在过去这十几年其实步步紧逼,开始的时候动作还不是很大,但是英国人已经觉得不太对劲了,为什么呢?英国的那个土豪,历史特别长,特别多,然后他们因为有很多海外领地,就是 19 世纪的时候那个帝国的余音,有很多很多的海外领地,这些海外领地几乎全部都变成避税天堂。那这个避税天堂主要不是让西欧大陆上的国家用的,而是英国自己的土豪在用的,OK。当然最近这几年也有各国的富豪跑去分一杯羹,那伦敦的那个房地产会涨上去很多,是因为那个俄国富豪去那边买高级地产。



王孟源 21:59 

现在的问题是说欧盟原来原本的目标是像 apple 这样子的外国企业或者美国企业,但是问题是他们搞避税的那些手段是一样的。比如说在英国有一个土豪,他是 Daily Mail 的老板,他是一个子爵,他在英国赚的钱每年赚几千万几亿,他这英国报税是零,因为他那个利润全部都付给他的左手,就是他在英国的企业,是他的右手,他的那个左手是在像直布罗陀或者其他避税天堂的那些法人,那钱送过去以后到那边就不用报税了,对不对?那这样非常的方便,而且一切合法就是你,虽然他们的身份跟外来的企业,像 apple, 这样子不一样,那实际上所用的避税的方法是一样。



王孟源 22:57 

所以在很巧的是,在欧盟在什么时候搞这个反避税搞到最认真。 2016 年 1 月 28 号,他们的那个立法机构通过了一个发一条叫做 ATAD,就是 Anti-Tax Avoidance Directive,OK。这里面有 5 大项,我想因为听众都不是会计或者金融或者经济学哈哈的专家,我就不要去讲那 5 大项,主要就是针对像我刚刚讲的那样,用那个在避税天堂设立法人的那个手段。而且他们真实的目标其实是Apple。这个条例是 2016 年 1 月立法,他准备在 2019 年1月实行。



王孟源 23:48 

但是英国的这些土豪一看这个法条,哎,不对劲了,这一下子我一年要多交 1/ 3 的税了,那我的这个利润掉了 1/ 3,这还得了?你刚刚开始的时候也讲那个 Brexit是 2016 年搞的,你知道那个  Brexit的这个公投是什么时候通过英国国会的?你知道吗? 2016 年 2 月 22 日,距离那个 ATAD跌通过还不到4个礼拜。



史东 24:19 

所以说那个通过之后就不到4 个礼拜。



王孟源 24:22 

Cameron 就被迫要做公投。他为什么要被迫做公投?因为当时保守党里面有四十几个议员中,那个下议院的议员参加一个党派组织,就是一个小组,叫做 ERG, European research group。这个这个组织就是专门由英国土豪鼓动起来,要搞脱欧的。它其实是一个老的组织,以前就有了,但是在 2016 年忽然地壮大起来,而且忽然地换了组长,换了领导人,OK,然后一下子他们的身势就忽然变得很大,然后对 Cameron 做了很大的压力。那Cameron。也没有仔细想清楚,他也不知道,不明就里啊。他只觉得说,诶,奇怪,怎么以前是一些乡巴佬在那边胡扯聊天,说我们要脱欧,怎么现在忽然有好大的政治压力,经济压力就是一下子所有的媒体基本上除了Guardians,就是卫报之外,一夕之间——就是在 2015 年底 2016 年初——忽然全部都开始鼓吹脱欧,声势一下做得就很大。那在那个,保守党内是这个ERG,然后在保守党外是 Nigel Ferrage,它当时有一个党,后来它在前年还是去年又组了另外一个党,就是有一个小档,就是这两个,ERG跟 Nigel ferrage 那个党在推。其实算一算不多,你看看英国下议院有多少议员? 650 个!



史东 26:11 

对,我记得那时候媒体的报道, Cameron 他把这个事情诉诸公投,因为他以为公投一定不会过。



王孟源 26:21 

因为他已经搞了两次公投,一次是苏格兰,另外一个是国内的法条。 OK 都很轻松地过了,他觉得,然后他食髓知味。他以为这是一个很简单的事,他没有想到,这件事背后在推动的那些土豪,他的蛋糕被直接冻到。他的那个利害关系有多么的严重,所以他们做起来会不择手段。我们在过去这 3 年多脱欧看来看到的英国政坛真的是史无前例的,他们以前都是讲绅士风度的,也对很多是不成文的规定,大家彼此要尊重。但是在过去这 3 年,基本上是撕下脸,就是撕破脸,大家各种阴狠的手段都使出来了,它的原因就是因为这些土豪必须要这样做,否则他们会损失 1/ 3 到一半的那个未来的收入。



史东 27:16 

我有个问题,这些在英国的土豪和在欧盟之内的土豪,他们的利益应该也是相关的,不是吗?



王孟源 27:24 

是相关的,但是那个欧盟之内的土豪一般没有,因为他们比如说你法国人对不对?法国他们的那个他们防逃税是有很久的历史的,就是欧洲大陆是一个中央集权式的,那英国则是比较资本主义分权的社会,所以他们的那个富人权力比较大,所以历史上又有那么很多海外领地,所以……



史东 27:54 

对不起我不要我再打断一下,因为这个是很重要的一个观点,就是在欧洲的这些土豪说没有办法用在英国的土豪的那些手段去逃税的,是不是这个意思?OK。



王孟源 28:07 

因为就是有国内法,欧盟的法律,一般来说所谓欧盟的法律是最低限度,就是你国家可以比它做得更严,你只是不能比它更宽这样子。所以事实上在那个像意大利,我以前有一个同事就是专门做意大利富豪的生意,就是把那个钱偷偷摸摸地弄出去,但你在国内赚的就没有办法,国内赚的利润就没办法这样。



史东 28:38 

那个就一个萝卜一个坑,那个更跑。



王孟源 28:40 

资产可以转移出去,利润就转移不出去。美国人也是这样子,美国的这个国税局也是很厉害那,但是英国就不一样,英国它的那个利润可以连你现在在国内今年赚的利润都可以转移出去,所以这是为什么?英国的土豪跟欧陆的土豪不太一样。噢,我再我继续讲下去。我补充一下,后来那个欧盟发现它这个 ATAD 里面有一个漏洞,这个漏洞叫做 hybrid mismatch。它为什么叫 hybrid mismatch?我也不要解释了,这是会计上的的细节,所以它的重点是,所以隔了一年到 2017 年2月,它又有一个 ATAD 2,二阶段,OK,这就是要补这个漏洞,所以刚好是晚一年。所以它的那个执行的期限是 2020 年1月1日。嗯,OK,所以刚好要晚一年。那这个就很重要,为什么?因为你看过去这两三个月,那个脱欧派跟留欧派斗得你死我活,每天都有新闻,就是因为这个大限快到了,它最基本。



史东 29:51 

说是 2020 这个大限,对吧?



王孟源 29:53 

2020 年 1 月 1 号的这个大限快到,这就是到 2020 年1月1日,他就会把这个反避税条例的第二阶段,也就是把漏洞全部补起来。那这下子……



史东 30:09 

除非英国脱欧。



王孟源 30:11 

除非英国脱欧,你那个你 ERG 被拿到很多资金以后,它涨到五十几个议员,但是这是保守党里面的,保守党是执政党,然后那个他们土豪他们也可以收买保守党之外的,他们又收买了大概四十几个,也有工党的、也有那个 Nigel Frog 这些小党的,那其另外其中有 10 票是来自所谓的DUP,就是 Democratic Unionist Party of Northern Ireland,就是北爱尔兰民主统一党,OK,这个爱尔兰的问题是因为英国当时是霸主,他到爱尔兰去殖民,结果大部分的英国人是去的是北爱尔兰。那后来那个宗教也不一样,对国家的认同也不一样,爱尔兰一直到上个世纪才真正独立,但是北爱尔兰就没有收回来,那之后搞了恐怖分子,他那个,所以英国其实在 60 年代 70 年代 80 年代有很多国内的恐怖袭击的。到目前到现在整个英国伦敦也是整个欧洲公共摄影机最多的地方。就是他们的警察局,基本上可以每一个街角都看得到,就是因为那时候留下来。后来他们有了和解,这个和解就是靠着欧盟来做的。和解就是,既然你都是在欧盟,然后大家也都是免关税,然后那基本上北爱尔兰跟爱尔兰就是流通,就是名义上不是同一个国家,但是实际上就像同一个国家一样可以自由来往,那这样就可以和解,对不对?所以他们这个爱尔兰和解是一个是基于欧盟的,那在这个上面。北爱尔兰那些亲英国的那些激进派,因为他们的那个宗教也是英国国教,那而不是爱尔兰的天主教,那这个差别使得他们就很害怕那个英国如果再继续在欧盟待下去的话,那个北爱尔兰因为跟爱尔兰自由流动,他们怕日子长久下去到最后就会统一了。



史东 32:36 

你说谁在怕被统一。



王孟源 32:39 

北爱尔兰的精英派怕北爱尔兰被爱尔兰统一了。所以他们也想脱欧,他们想脱欧的原因是脱欧以后他们希望能够在北爱尔兰跟爱尔兰之间有一个边界,这样子北爱尔兰就不会跟爱尔兰扯到一块。OK,所以我刚刚讲了,在一大堆是基本上真正想脱 欧的一员,也就是 100 票左右,这是在 650 个人中算是很少很少的少数。嗯,那他们把那个 Cameron 哄的去公投,公投之后他们无所不用其极,撒谎撒得一样很离谱。这些脱欧派的人在公投之前都是指天画地说我们只不过是想要变成挪威,但是一旦公投胜利以后就全部忘记了。这是我说我们要无协议公投。



王孟源 33:36 

事实上无协议公投是他们真正要真正的目的,因为他们真正的目的是保障土豪的权利,那保障土豪权力就是那个 ATAD不能够适用到英国,而唯一能够让 ATAD不适用到英国的的办法,不但英国要脱欧,而且英国不能够跟欧盟有紧密的自贸条约。



史东 33:59 

换句话说就是彻底的脱欧。是吗?。



王孟源 34:02 

彻底的脱欧,嗯,就是必须要彻底脱。所以他们这边讲什么什么协议,都是骗人的,那个实际上他们的目的是无协议脱欧,所以我们先把这一点确立了以后。这为什么 100 多个议员能够闹成这个样?是因为它虽然是只是议员里下议院里面的 1/ 6,但是它占有大众媒体的九成。OK,我以前在讨论香港的时候我说过,你要这种民主制度要有效地落下去,有两个事情你一定要抓稳,一个是基础教育,另外一个是大众媒体,因为大部分的民众不懂事,非理性、愚蠢的。你可以很简单的哄骗他们,把他们领到错误的方向。



史东 34:54 

然后其实我就顺便插一句孟源,我在观察这个脱欧的时候有,我有很多的招式看不懂,有一种招式我是看得懂,有一种招式不但看得懂还是看得叹为观止,那就是他们在媒体上的运作。对,那简直是不得了。是不?但是传统的媒体,还有在这个 social network 上面的这种运作。



王孟源 35:20 

那是不得了的。非常对, four part 也是完全利用那个俄国人在美国搞的那一套 social network 造谣的办法来哄骗民众。所以事实上脱欧跟留欧在在英国的那个民众的支持率是五五波,我看到有一个比喻很好玩,他说那些托欧派的民众其实都是 Turkey voting for Christmas。为什么说 Turkey Voting for Christmas?我们在美国吃火鸡是感恩节吃的,英国人的那个圣诞节的大餐就是一只火鸡。所以你基本上是,那些民众是火鸡,投票要求,赶快有圣诞节,因为这你看到处都是,台湾那些部也是这样子,也是火鸡投票。



史东 36:15 

哈哈哈,太棒了,这句话说得太好,太揣神了。



王孟源 36:22 

这这是一个英国的评论家他讲的,他说那些脱欧派的民众都是 Turkey voting for Christmas,所以这是他的底气。因为那个土豪他们那边占了90\% 的公众媒体跟那个跟 social network,所以说这个媒体是抓在他们手上。他在下议员,为什么没办法抓超过 1/ 6?是因为他的对手包括了实业家,你现在还必须要做进出口贸易的那些人,还也包括了金融业。伦敦的那个金融业是英国对金融业的依赖比美国还要深,因为它的那个去工业化比美国还要更进一步。那你如果跟欧盟断绝关系以后,那金融业一定都统统跑掉了,整个伦敦。嗯,那百万个高薪的工作,就不只是你直接被雇用了……



史东 37:21 

就比我们曾经谈到的香港面临的状况更要严峻。



王孟源 37:25 

对,所以在那个英国脱欧的这件事情,土豪跟媒体固然是站在脱欧的那一边的,但是在留欧这边也不是,也不只是理想派,或者是或者是学者,OK?他其实是也是有实力派的。实力派就是那些做进出口贸易的实业家,不多了,但是还是有。那另外就是金融,金融也放在那,伦敦到目前为止还是一整个欧洲首屈一指的金融中心,就是全球仅次于纽约的金融中心。那如果脱欧的话,那基本上就完蛋了。我估计是会被那个巴黎跟法兰克福瓜分掉。这些实力派他们掌握上下议院的议院,所以他们那个托欧派在下议院的那个实力就不够,就没有办法像展现出像他在民众的……



史东 38:23 

他们不是现在正在炒着再做一次公投吗?



王孟源 38:27 

好,我现在就解释一下,他们那边两边过招实在是精彩。我们先从Theresa May的那个协议。我现在再解释一个细节,先讲清楚他们的这个所谓的协议。我们中文就说协议,其实他们在那里,在英文里面有两部分,有两个不同的意思。第一个是 withdraw agreement,就是脱欧协议,OK,我们直接翻译叫脱欧协议。但是还有另外一个,意思是 the deal,我们也翻成脱欧协议,其实这是不对的,这个 deal 是自贸协议的意思,就是你在脱欧的时候之后是不是含有自贸关系,OK,所以我会我说脱欧协议,我指的是 withdraw agreement。



史东 39:27 

我这么理解你看对不对?脱欧协议讲的是如何脱欧的条文。第二个你讲了这是 the deal,是脱欧之后如何和欧盟打交道的嫌疑的。



王孟源 39:44 

脱欧协议。Agreement, withdraw agreement 就像是离婚协议。这个 deal 呢?是你的赡养费。



史东 39:51 

对,就离婚之后怎么过日子的问题,对不对?



王孟源 39:54 

你同样的离婚,可能有赡养费,也可能没有赡养费,对不对?



史东 39:58 

就是它的目的是不一样的。



王孟源 40:00 

然后在英国的这个下议院的操作,这个脱欧协议,这个离婚过程本身又可以分成两个同的意义。我这因为你如果不懂这些的话,你就没办法把这些招数看懂,OK,所以我必须要花时间来解释一下, withdraw agreement 是一条约,是英国跟欧盟之间的条约,OK,但是这个在这个条约是一个国外条约,在英国国内是没有法律的力量的。你要让它变成国内法,必须要立法。这个立法叫做 withdraw agreement bill WAB,不是只有 WA OK,那我要先把这两个讲清楚。然后我们在过去这两个月的过招才能够弄清楚。



王孟源 40:58 

OK,因为这些细节都是很重要的,因为英国人他们的传统就是这种民事法,他们搞得很精,他们搞了三四百年,所以已经登峰造极了。这听起来像是在律师在钻牛角尖,不,事实上就是这样子,他们就是一大堆律师在钻牛角尖。话说回来到Theresa May, Theresa May在那个 Cameron 辞职之后,辞职负责之后她上来,她也不懂这个土豪在背后搞的这个,她也只是一心一意想说他原本是留欧派,然后他说既然我当了这个首相,她还是一个传统的英国政客,就是说我固然要顾及自己的政治生涯,但是我的阻止还是要为国家做出最好的服务。那最好的服务就是既然公投已经决定要脱欧了,我就在脱欧的前提下有一个最好的 withdraw agreement,这个 withdraw agreement 里面要有最好的deal,OK,那这个但是他这个 deal 连续送了三次,头两次被否决;第三次那个议长说你已经试了两次了,不能够再试第三次,然后她最后在这个今年夏天黯然下台,就是知道这她已经不可能把这个决议通过。为什么不通?不过他这个基本的问题就在北爱尔兰,为什么?我刚刚特别去提 DUP跟北爱了,就是先让大家有这个背景的知识。



王孟源 42:34 

欧盟的意思是说,因为爱尔兰是欧盟的一员,所以你如果在北爱尔兰跟爱尔兰之间出现了海关跟边境,那爱尔兰不接受,爱尔兰不接受,欧盟就不能接受。所以欧盟的态度是说,我不管你这个协议说什么,反正得爱尔兰跟爱尔兰之间不可以有出现边境,不可以出现海关,OK?那你说为什么欧盟不在乎损失因果?因为它不能够去挽留英国。你如果对英国退让的话,欧盟一共有 27 个国家,那每一个国家都会威胁着我要退让,然后我要脱欧,然后你给我特殊条件。不可能这样子。所以你为大局着想,英国要脱欧,它只能够让它脱欧,OK,但是因为爱尔兰的关系,它唯一坚持的条件就是北爱尔兰跟爱尔兰不能够分裂开,但是你看英国本身的考虑,怎么可能说我自己的国土分成两半?你说大陆跟台湾是一国两制,它是因为内战的结果,这个英国跟北爱尔兰是统一的国土,已经有几百年了,怎么可能说一下子就把它分裂成两半?OK,所以英国国内的政治会有很大的反对力。但是你想想看,如果北爱尔兰跟爱尔兰之间没有海关,那北爱尔兰就必须执行欧盟的法律。如果北爱尔兰跟英国之间没有海关,那英国就必须执行北爱尔兰的法律,也就是欧盟的法律。那这下一来土豪的脱欧还搞什么脱欧?就是不要执行欧盟的法律,也就是ATAD,所以你看这就是一个三角的矛盾,三个要求不可能同时被满足。



史东 44:35 

这个问题其实就因为我前两天我还看了一个图,你想到北爱尔兰和英国和欧洲之间的,他们的这个海湾的问题,就是物流,那个物流的那个,那个流向。



王孟源 44:48 

那个不要紧,我我会把重点抓出来。OK,那些烟幕,你可以忽略。



史东 44:55 

对,那些都是烟幕。



王孟源 44:56 

即使这个重点,主流也是很复杂的,因为,因为他们是根据法律来做斗争的,所以政治斗争跟法律斗争在同时在发生,所以是很复杂的事情。刚刚讲到这是那个基础的矛盾。那Theresa may 的那个条例,为了解决这个矛盾,她所用的方案叫做backstop,这 Backstop 是什么意思?承认北爱兰跟爱尔兰之间不能够有边界海关,然后承认英国跟北爱兰之间也没有边界海关,OK,所以我们现在他就说我们会脱欧,我们会谈自贸协议,这个自贸协议不晓得要多久才能谈得出来,但是在谈出来之前,在有永久性的协议之前,也就是deal(之前),在我说的协议,就是deal (之前),就是我们先离婚,但是这个离婚分财产、赡养费这些事情以后再谈,还没有弹出来之前暂时采用 Backstop。什么是Backstop?就是维持现状,就是名义上离婚了,但是实际上,还是维持现状,那维持现状的意思就是他还必须要遵从欧盟的法律,那这下土豪当然不可能同意。



史东 46:18 

那就是他原来的原意就没有办法达到了,没。



王孟源 46:22 

办法达成。所以土豪 那边自然就不答应了。对,不答应的话他就发动民意,然后利用保守派内部的那个ERG,然后刚好这个 DUP 就是北爱人的那个党,那个民族统一党。他也笨了,他们的目的是要在北爱尔兰跟爱尔兰之间建一个边界。事实上那是不可能,因为欧盟不可能同意。 OK 那但是他们为了这个原因也反对 Theresa May这个 deal, 这个协议。那DUP 为什么这么重要?是因为Theresa May在 2017 年召开大选,结果没有达到多数,必须要组织联合政府。那个联合政府就是靠 DUP 的那 10 票,他才刚刚好过关超过半数。所以这个 Theresa May的的协议其实是已经人力所可及、在先天限制之下所能够做的对英国最好的协议了,但是一年被否决了三次。那她就黯然下台。好,那黯然下台以后,到八月七八月这个他们重新选举首相,这个英国的党派跟美国的党派不一样,美国的党派是没有党费的,而且你不是真正的党人,比如说我现在认同,我在那个登记的时候,我登记我是民主党人,我其实不是民主党员。OK,我也不需要交党费,我只是我自己决定,我要认同民主党,所以我就自称是民主党。也就是说用英文来说是 affiliate party, affiliate 。那 party member 是党员对不对?英国的这个政党就是党员式的政党,所以他这个美国共和党跟民主党各有大概 1/ 3 的人口,选民人口是共和党跟民主党,然后其他的中间派在英国的话,保守党是第一大党,但是你猜猜看他的党员占全部人口是多少?0.3\%,所以 Theresa May这么一辞职,他们选这个新党魁,新首相就是这0.3\%来选的,那这百分明显上刚好就是大部分被土豪洗脑过的,所以选出来的这个首相就是从从大学毕业,牛津大学毕业一开始就是脱欧的干将,因为他不是一开始就是土豪的人。但是他刚好一开始当记者, Boris Johnson一开始当记者,二十几岁当记者,又被派驻到欧洲去,然后他发现一个哗众取宠的办法,就是编,编造夸张欧盟的坏处,所以它的整个政治生涯,就是建筑在撒谎来抹黑欧盟上面。后来他运气很好,那土豪刚好要搞这个脱欧,那他就脱颖而出了,对不对?因为他已经搞了一辈子了,他当了首相以后,一连几个法案都不被国会接受,然后他就搞出了一个妙招。我先再提醒你一下,他的目的当然就是要无协议脱欧,而且是当时最后的期限一直到现在还是就是本月底、 10 月底,那你如果 10 月底没有脱 欧 的话,再延期至少 3 个月、 4 个月,那你就会过了那个 ATAD2。



史东 50:10 

那就是  ATAD2就生效了。



王孟源 50:13 

就生效了,对,那就不好了,对不对?所以这就是为什么,在过去这两个月他花样百出,就是非要英国在那之前脱欧,那他玩的第一个重要的花样就是Prorogate,所谓的 Prorogate是暂停国会,不是解散国会。解散国会的话是说你要重新选举。



史东 50:39 

是休会吗?



王孟源 50:41 

就是休会,不是解散国会。因为历史上的原因,英国的国会一开始的是每隔三四年不定期才会召开一次,每次只有是一两个礼拜。那后来到现代当然是不一样,但是那个传统还是留下来,因为英国一直没有成文,宪法一直都是靠传统来治理。但是休会这个东西理论上传统上一直是首相来决定,一个人可以决定说我现在就要休会,但是传统上休会就是一个礼拜 5 天的时间,然后还有一大堆仪式,但是他这是因为他的幕僚还有一大堆律师在那边想办法要绕过国会,因为他知道,国会不会同意无协议脱欧。我刚刚已经解释了,他们这些无协议真正强硬的脱欧派占的是少数,所以他们就想出一个办法来,就是休会 5 个礼拜,从9月中休到 10 月底。那这样子,因为英国国会的效率不是特别高,你平常一个法案都要讨论几个礼拜,那你如果从9月中一直休到 10 月底,那最后剩下一个礼拜就脱欧了,他们大概来不及做出什么反应了。这就是他们的如意算盘。



王孟源 52:01 

跟台湾或者大陆的连续剧不一样,这个这里的那个好人不是傻白甜,这里不是这里的那个留欧派,也就是 Johnson 的对手,是我刚刚讲过实业家跟金融派,这些人智商不低,他们有必要的时候也能够整合力量来做出反应,所以他们很快的。Johnson 在9月初做这个宣布的,他们很快地在一个礼拜之内就通过了一个法案,这个法案叫做 Ben act,这是9月初通过的方案。这个法案很重要,他说到了 10 月 19 号,那是一个礼拜六,就上个礼拜六,如果国会还没有同意脱欧协议的话,首相就必须向欧盟要求延期。OK,这里的协议指的是WA,就是 withdraw agreement。英国跟欧盟之间的条约。



史东 52:55 

就是离婚协议,。



王孟源 52:56 

就是离婚协议。对,离婚协议,但是不是后面的那个赡养费跟分财产的。对,其实他们写 Ben act 的时候已经是请了专家来了,由法律专家来执笔,其中一个是以往的司法大大臣,但,但是他们百密一疏,就是他们没有想到。我刚刚大概在 15 分钟前解释成一个差别, withdraw agreement 跟 withdraw agreement bill 还是不一样的。 withdraw agreement 是那一纸条约,然后 withdraw agreement bill 是英国把这个条约变成国内法的那个法案。它的那个 Ben act 里面指的只要求你通过那个条约,没有要求你通过那个法案。但是你如果没有通过那个法案的话,到了 10 月 31 日脱欧期限之后,你还是没有办法合法地有那个协议,那么你就自动无协议脱欧。所以他们国会休会之后,一直闹到最高法院,结果只休会了一个多礼拜,然后国会又回来。之后没有多久,留欧派就发现风声流出,就是 Johnson 的那些幕僚已经发现了这个漏洞,那他们由那个前首相 John Major, 90 年代的一个保守派前首相派出来,把他公开指责说你们这是用到违反法律的精神,这是拿条文来玩法弄权。所以接下来,Johnson 过去这一个多月的努力,完全就是为了要钻这个漏洞,他怎么钻这个漏洞?他必须要让一个协议通过,但是又让那个就是 withdraw agreement 通过国会,那这样子他就不必要求延期,但是在那之后他只要想办法让那个 withdraw agreement bill 没有在 10 月底前通过,他就可以无协议脱欧。所以他在玩这个花样,你要有一个 withdraw agreement 被通过,就必须要先有一个草案,对不对?那个他就提出的那个草案。



王孟源 55:12 

这个草案一开始是在 10 月初的时候,是说我们不是要设立一个边界,而是设立两个边界,北爱尔兰跟爱尔兰之间有一个边界,北爱尔兰跟英国之间有一个边界,OK?好。这时候 DUP 还高高兴兴地说,好,至少北爱尔兰跟爱尔兰之间有边界了,但是这显然是行不通的,对不对?你一个边界人家都不接受,你怎么可能同时有两个边界那?所以他这个 withdraw agreement 在不交国会去投票之前,必须要先跟欧盟同意。欧盟我刚刚已经讲了很多次了,他欧盟要同意的条件就是北爱跟爱尔兰之间没有边界,没有海关,所以上个礼拜,就是 10 天之前,当然这个Johnson 的这个协议就一下子变成只有一个边界了,这个边界就是北爱跟英国之间的边界。换句话说, Theresa May 以前讲的那个backstop,也就是整个英国,包括北爱尔兰。在所有的自贸协议谈成之前,都必须要暂时实质留在欧盟;它(编注,指Johnson的草案)变成北爱兰无限期地留在欧盟的法律。然后英国在一年之内,就是在 2020 年底要谈成一个自贸协议,而且这个自贸协议必须在7月 1 号之前谈成。如如果不谈成的话,到 2020 年年底就正式的无协议脱欧。所以搞来搞去就是英国本土可以无协议脱欧,而且是马上无协议脱欧,那北爱尔兰就被割掉了,那这代表什么?他们狗急跳墙到什么地步?狗急跳墙到他愿意割让领土,因为你这样一来爱尔兰高兴得不得了,因为北爱尔兰基本上就是实质跟爱尔兰统一了,现在边界变成北爱尔兰跟英国之间,爱尔兰本来想说要再等 100 年才能够真的统一,这看起来十年之后就可以。知道他高兴的不得了。你如果去看那个爱尔兰的反应,那当然这下子 DUP 就知道他们被骗了对不对?他们的目的是要让北爱尔兰留在英国,而不是让北爱尔兰跟爱尔兰统一。所以这一下就进展到上个礼拜六,就是那个 BEN ACT的期限, 10 月 19 号他说你必须要通过这个 withdraw agreement,要不然你就必须要求延期,所以当天就有很热烈的讨论。Johnson 的如意算盘是说你通过了以后, you know,就算北爱尔兰被割让出去也不要紧,反正我的土豪老板们都很高兴。但是这时候留欧派我刚刚讲过,他们都是懂事的人,他们很聪明,他们就通过了一个amendment,就是他把这个 withdraw agreement 送到国会的时候,国会在他表决之前可以加附加条例,他强加了一个附加条例,这个条例叫做 Letwin Amendment。这Letwin Amendment说什么呢?他说在国会对 withdraw agreement 做投票之前,必须要先通过Withdraw Agreement Bill,就是那个国内法要先通过,然后才能够对国外法做投票。那这下子这个Withdraw Agreement 过不过都没有关系了,都因为事实上不能过,因为你那个 Withdraw Agreement Bill还没有通过。事实上在上个礼拜六还没有,甚至还没有公开。就是那个 bill 的草案,是 Theresa May就在今年2 月还是3月的时候就已经拟好了,但是它保密,不敢公开,因为 Johnson的这个协议跟 Theresa May的协议差不多,只不过是更加丧权辱国而已。所以其实那他那个 Bill版本也是差不多,但是你像这样一来他就没有办法了,他就是在礼拜六晚上,上个礼拜六晚上的他就只好送出那封信,然后要求欧盟延期。那现在欧盟还在审查之中,他们还在讨论是延 3 个月还是延其他的期限?



史东 59:37 

最主要的就是那个 2020 年的1月 1 号的那个ATAD的这个期限,这个是最重要的一个日子,大家都在绕了这个日子,在在尔虞我诈是不对?



王孟源 59:53 

那到了礼拜一,这个礼拜一,今天是礼拜四,三天前他们 Johnson那一派,虽然已经要求延期了,他还不死心了,他又重新了上一次。这次他在礼拜一搞的是什么?他把那个 Withdraw Agreement  又重新送进一次,OK,就是礼拜六送进去讨论的时候,他说被人家加了一个附加条例,那个 Letwin Amendment,然后他就觉得那这样没有意思了,我不搞了。然后到了礼拜一,他说我们再送一次,但是这是同样的条例,但是同样的条约也已,但是是重新送,所以就没有那个附加条例。那这时候那个议长又出来说没有这回事,我们的英国的传统是同一条法律,被否决之后不能够重新送。OK?所以礼拜一的新闻就是说议长把他否决掉了,他就送不进去,以后。这下他说好了,那我们唯一合法的办法能够还在这个月底脱欧呢,就是先把 WAB 通过了,然后你才能够通过WA,所以到了礼拜二他是又送进协议。对,你也奇怪,明明礼拜一已经议长说同样的法案不能够再送了,怎么礼拜二又才送了一次?哈哈,而且还投票表决了那个 Johnson 的还第一次十几个方案,他提出了十几个方案,第一次通过,这是怎么一回事?因为他这次送的不是那个withdraw agreement,不是WA,而是 WAB 了,因为那个 letwin agree amendment 已经说你必须要先通过 WAB 才能够审查WA。他送的是WAB,那这个当天有两个投票,第一个投票是可不可以进入二读,就是那个国内法,他们是要三读才能通过,就是你要不要进入二读?进入二读的意思就是说一读就只是宣布这个法案,然后二读的话就是可以讨论辩论,然后加那个附加条款。 amendment ,好,他通过了,但是这个是无关紧要的,你从一读进入二读,谁理你对不对?反正还是会有其他的。



王孟源 01:02:26 

真正重要的是第二个投票,第二投票就是说Johnson要求国会在三天内审查出这个WAB,非常非常的离谱,因为你是一个非常复杂的法案,然后前一天才刚刚公开,事实上英国的法律是 2010 年刚刚修订的法律,说这种资历的国内法必须要讨论两个礼拜。那所以 Johnson 送进去的第二个法案就是说我们要Supersede, 不管的既有的法律。然后这里创造一个特例,就是要 3 天之内就讨论出来。结果这个就被否决了,被留欧派否决了。我觉得以后Johnson知道这下没得玩了,他就放弃了,从礼拜二他是彻底输了,这这基本上他是不可能在 10 月底脱欧了,这时候我必须要退而求其次了。



王孟源 01:03:17 

其实我在9月底 10 月初我那时候已经写了文章说不可能在 10 月底拖了,因为那个Ben Act 像个铁桶似的,唯一的漏洞,那个留欧派也知道也能够补助,那过去这一个多礼拜也的确就是为了针对的那个漏洞,双方在做攻防。然后最后是留欧派胜利,因为留欧派不但控制了国会,而且有最高法院的支持,所以我在一个月前就知道这是不可能在 10 月底脱的。那现在的形式就是这样子, Johnson 基本上已经不可能在 10 月底拖了。那他退而求其次,就是希望在今年年底脱,这很奇怪啊,那个欧盟大概会给你至少三个月的延期,那就是到1月底,你怎么能够在 12 月底脱欧呢?他要在 12 月底脱欧呢?



史东 01:04:13 

不就是这个 a t a d 的事情吗?



王孟源 01:04:17 

他这个时候要在 12 月底脱欧,唯一的办法就是重新大选,他赢得多数,控制了国会。那你可以重新立法,什么样什么都可以了。所以他现在过去这两三天,他高调要求的就是在 12 月初做大选。他知道虽然他在现任的国会是少数派,但是土豪已经是狗急跳墙,愿意投入比 3 年前更多的资源来操弄那个舆论,操弄大众媒体,他们就有很大的机会能够控制国会。那控制国会以后到 12 月初选举完, 12 月中重组国会,他就可以随便立法,甚至可以无协议脱欧。



史东 01:05:04 

如果他没有办法达到这个目的的。



王孟源 01:05:07 

原本我觉得他有希望成功,为什么呢?最大的反对党是劳工党的那个,他主要的力量真的就是在工会,他现在的这个主席工党的党魁也是工会出身,他就一个——你从那个劳工运动的角度来看——他是一个真正的工人运动的领袖,所以他从政这三十几年以来已经被这个资本家控制的媒体抹黑到名声很臭了。就是不管你本人实际上怎样,其实是我觉得他是一个有理想的人但是,因为他从一出道就是搞劳工运动,所以资本家的媒体从来就没有对他客气过。



史东 01:05:55 

所以说他的公共形象有问题,对不对?



王孟源 01:05:57 

他公共形象非常糟糕。现在那个留欧派中的实力派就是那些实业家跟金融家,他们下注,是下在那个 Liberal Democrat 第三个档,那他们是绝对要留欧的,但是他们也不喜欢工党,所以他们如果留欧派这样内讧的话,进去大选是很危险的,再加上那个脱欧派又控制了大众媒体,原本工党还是同意的。我过去这几天其实还蛮担心这个大选会真的在年底之前就发生,然后那个保守党会大获全胜。不过今天消息传来,工党忽然吃了理性药了,忽然明白他们在大选的机会不大,因为他们还要跟那个Liberal Democrat分一分选票,然后大众媒体也很可能会对他们非常的恶劣。现在听说他们正在考虑拒绝在今年大选,就是要求先公投,再大选,那如果是明年年初,先公投再大选的话,那就有可能是留欧派胜利了。对,那如果是先大选再公投的话,先大选然后保守党胜利的话,那就直接脱了。对,OK,所以对目前的情况我觉得还是五五波,因为你到最后还是看那个大选的结果嘛,对不对?但是有协议脱欧,基本上已经名存实亡,就是你有可能有那个离婚协议,但是实际上后面紧接的那个自贸协议是不可能的。



史东 01:07:35 

好像看这个样子,看到这刚刚我讲的这个 2020 年的1月 1 号,这个期限呢?即使过了也可以无限制脱欧,只不过是。



王孟源 01:07:47 

只不过是他们有几个月的利润。对。



史东 01:07:50 

就是说这次也是一个,对托欧派来讲也是一个退而求其次的状况。



王孟源 01:07:59 

对,他们今年的目的是第一个要求是到本月底脱欧,这个目前已经不可能。在两天前已经不可能了,然后现在他们是要求 12 月大旬,然后看看能不能在12 月底。到这为止,他们都完全不需要担心这个 ATAD。对对,但是如果是没办法成功,那个大选必须到明年初才举行,他们可以忍受几个月,他的损失也不是太大,但是拖得越久,当然他们的损失就越大。当然英国的损失,英国国家整体的损失是非常大的。首先。



史东 01:08:37 

要,我正要问你这个问题,我正要问你的这个问题。



王孟源 01:08:41 

北爱尔兰基本上已经是完蛋了,就是国土会分裂;然后接下去苏格兰一定会要求,因为苏格兰也留欧派,苏格兰的那个最大政党也是,已经说他们就是要趁这个机会再重新公投一次,这一次的这个机会就很大,因为他们的民调已经超过 50\% 支持了,上一次是大概只有 40 出头,如果就要通,这次已经是超过 50\% 了。然后如果无协议脱欧之后,英国的经济会马上崩溃,就是因为它的那个经济跟欧盟的联系是像脐带式的关系,你如果一下子,就好像堕胎一样这样冲出来,这个英国的经济会马上崩溃,以后捡便宜的是谁?当然就是美国了,美国就会进来跟他说这里是一个自贸协定,那基本上是 Trump 提出什么条件他们就必须要接受了,对不对?那这个英国就可以尝尝当人家殖民地的味道,哈哈哈哈哈哈哈哈哈。当然那些土豪不在乎,反正他只要是。



史东 01:09:54 

对,因为这个跟他们没有关系的,国家的这个兴亡跟他们没有关系,他们是到处有钱,到处都可以跑的。



王孟源 01:10:04 

对对,资本没有国界,对,那。但是现在也有人说,就是在观察者这个网上有一个评论家说是英国的来这种专家,他说这个脱欧是美国人搞的。其实不是美国人只是、渔翁得利而已,这是趁势接盘,当然美国会得到很大的利益,但是真正的动机真正控制这个脱欧背后的动机是英国本地的土豪。真正在阻止他们脱欧的,不是那些青年人,因为青年人人数虽然多,但是你还是比不上那些容易被洗脑的老年人或低教育程度的人,最终能够拿出力量整合,像我刚刚讲那些什么Ben Act, Letwin amendment,那些都是必须要有聪明人拟定,然后必须要有政治运作,把国会的那个反对派拧成一股力量才能够通过。那都是破纪录的速度通过的。这个是实力派,就是那个实业家跟金融业的那个实力派推动的。但是他们能不能掌控大众媒体,能不能赢得大选,这个就是很难说,因为民众先天就是非理性的。对,我在我的博客上已经也提过很多。



史东 01:11:29 

我想讲到这里,我在一面听你在讲述这些来龙去脉的时候,我一面有很多的感想。我们在谈到对全世界各个地方的影响之前,我想有一个问题,你觉得这个事情的发生演变到现在,以及看到它可能会发生的结果,甚至是可以说有一种自残的这种自状况存在,这个和英国的民族性有没有关系?换句话说,如果这个问题和发生在世界上其他国家会如何的解决?



王孟源 01:12:06 

其实你如果是 5 年前问我说,哪一个先进工业国会发生这种事?英国是我最后会想到的。就是英国跟北欧国家是我认为最不可能发生的,因为英国的民主传统最久,它这个民主传统为什么能够这么久?一方面是它能够有一大堆殖民地,能够对外掠夺,所以满足国内的民众,但是满足国内民众以后,你实际上运作还是必须让那些真正理性的国会实力派,能够尊重传统,然后有以国家优先的那个考虑,就是他们能够做出理性的考虑,然后不要对民众做出过分的剥削。他们有几百年这样的传统,突然一夜之间,在三年前做了第一个转向,但是那个时候首相还是传统式的人物,到了今年,也就是 3 个月前,才忽然一下子变成台湾式的那种所谓的民主,这个是从一流的西方式民主跳到最低级的西方式民主,这一跳这个幅度之大真是非常惊人的。



史东 01:13:19 

这个,我又延伸出来的第二个问题,你觉得这个事情基本上和民主制度这四个字有多大的关系?还是说这显示出来民主制度的一个非常大的问题。



王孟源 01:13:34 

就是民主制度一定是越来越糟糕的,因为它本身有一个内建的,就是非理性趋势,你这个制度越久,走得越久,那个资本家就越有机会去钻漏洞来掌控大众媒体,从大众媒体来掌控那个群众的意见,然后把群众都变成投票选圣诞节的火鸡就是Turkey Voting For Chrsitmas



史东 01:14:04 

绝妙的比喻。



王孟源 01:14:07 

台湾也是选,就是选那个民进党,就是把自己送到他们的那个桌上当火鸡,给人家宰割对不对?但是他们过去这二十几年还不是一次又一次地自愿当火鸡?



史东 01:14:22 

还有一个很重要的问题就是你觉得这个脱,如果脱欧的话,你讲到对英国的影响,对世界上其他国家。



王孟源 01:14:35 

对欧盟的影响很小很很小。欧盟的英国是,欧盟现在第二大经济体,就是次于德国,可是其实它的体量跟法国差不多,跟意大利也差不多,欧盟比它大大概六七倍,你这样一来对欧盟的影响就很小,而且英国会有很快的经济危机,这个一脱欧以后经济连那个药品都拿不出来。



王孟源 01:15:04 

对,我今天受时间所限,其实已经跳过了我要讲的一个路线,就是在 2008 年那个金融危机之后,那时候保守党在 2010 年选上来之后, Cameron 的政府在 2012 年通过了一系列的法案,就是所谓的财政紧缩法案。在那之后他们的那个福利社会有了很大的回转,在过去这六七年,他们的生活水准急需急度下降,贫富差距极度增大。我,我事先送给你的几篇文章有一篇是这样讲的,他说在 2011 年,就是那个法案通过之前,当年冬天英国冻死了 7000 人,就是在自己家里冻死了 7000 人,到了 2017 年——也就是那个研究是 2018 年做的——最后的一年冻死的是 17000 人,就是多冻死了1万人。为什么呢?因为他们付不起那个暖气了,OK?然后另外一篇文章是说他们估计在那个财政紧缩法案之后,每年多死3万人,就是其中 1/ 3 是冻死的,另外2万人是因为没办法去做医疗,没办法。



史东 01:16:28 

看就是饥寒交迫。



王孟源 01:16:30 

饥寒交迫?对,你为什么现在这个脱欧派这么容易哄骗民众?这其实是一个很重要的原因。就是,1。这些低收入民众他们不懂,你越是跟他们讲你们现在穷,就是因为欧盟害的,那你投脱欧,OK,其实你越是脱欧死的越快。



史东  01:17:10

对中国呢?

王孟源 01:17:12

中国不在乎了,因为英国到最后就会变成乞丐一样了,这个他在美国面前是一个乞丐,那美国除了在那个自贸协定里面能够塞进去,说不准他跟中国来往的条文之外,他面对中国的时候也是没办法挺直腰板的,因为体量相差太大了,所以中国可以还是可以专心跟欧盟。这英国脱欧不脱欧对世界影响都不大,真正影响到最大的是欧盟那,但是欧盟已经觉得可以承受了,那(收益)最大的一个国家是爱尔兰,爱尔兰非常的高兴,因为你这一脱欧北爱尔兰就受不了。



史东 01:17:35 

所以我看你看来今天我们谈话最大的这个脱欧之后最大的受害者还是英国本身。



王孟源 01:17:43 

英国的那些火鸡。



史东 01:17:44 

对,英国的那些火鸡。



王孟源 01:17:46 

欧洲,其实欧盟其实是很高兴因为,这一次英国脱欧这样子逮细脱膨,弄得这么乱这么难看,以前闹着要脱欧的那些国家现在都不提了,都不敢提了。



史东 01:18:03 

也有点道理。



王孟源 01:18:06 

他们很高兴的,损失了一个英国,其实换来长久的安定,是很好的交易。所以3 年下来……



史东 01:18:14 

我想全世界没有人会想到。



王孟源 01:18:17 

没有一个人,我真的没有想到看到任何一个人会预测会拖到最后。



史东 01:18:22 

所以我刚刚问你,这和英国人的个性是不是有关系?



王孟源 01:18:26 

不是,实在是因为那个时候土豪的那个真面目还没有全部露出来,我也是最近这一年多才明白真相,因为就是看到他们的那个说法转变,变脸变得这么快,我才去深刻地想想,这不太对劲。好像不可能是因为某种理念啊,这种态度是完全是为了切身利益,而且是非常大的切身利益才会做出来的生死搏斗,我从那个地方才会进一步研究,然后才会发现到有这些资料。



史东 01:19:04 

这也让我们在座的认清了这些财阀人的真面目,就是完全是。



王孟源 01:19:12 

那是 5 年前,你问我说英国的那个民主运作好不好?我会说还不错,因为他们有那个,也就是英国有几百年的历史,所以他们的那些政客政治学的传统就是,我们哄着民众,但是我们背着民众在国会里面,我们做我们应该做的事情,但是现在这个根本不是这样子的,现在是我们哄着民众,我们做土豪要做的事情。那这就变成台湾式的。



史东 01:19:44 

然后慢慢地连民众都不哄了,。



王孟源 01:19:47 

哄都不哄了,对不对?英国跟美国同时,这个他们的政治素质一下子在 2016 年往下掉成这个样子,是一个巧合,但是也真的是对全世界的当头棒喝。



史东 01:20:02 

晚上好,孟源。谢谢谢谢,谢谢你。非常精辟的 90 分钟的这个讲评,我觉得很有价值,因为我在一面听你讲这些,另外一方面我也在想这些资讯,在其他地方是不是能够如此有系统地得到?我想不出来。我觉得只有在你这里可以有如此有系统的得到这些资讯,我觉得这是就是你对大家的贡献了。我觉得这是很好的。



王孟源 01:20:36 

我一开始讲的是我受的科学训练。就是要抓出主轴嘛,因为你要能够做预测的话,就必须要先简化这个世界,而不是把它复杂化。嗯,对,你如果,哈哈,全部都几百个几千条,那个因素跟动力都考虑的话……



史东 01:20:53 

,我跟你讲一个我的经验以及我的感觉,一个有能力的人是会把事情简化的,一个没有能力的人是把事情简单的事情复杂化的。对不对?



王孟源 01:21:06 

我觉得现在太多假专家。



史东 01:21:11 

他就是把事情复杂化,这样表示“他很懂”的人,很多。



王孟源 01:21:15 

就是用专业术语来哄你,事实上他的什么都不懂,什么都不,没办法预测,这很可怜的事情。



史东 01:21:22 

非常好,非常好。谢谢谢谢,谢谢。



王孟源 01:21:24 

高兴能够有你这个平台能大家分享这些。很好。



史东 01:21:28 

很好,这个平台永远为大家存在,永远为你存在。



\twocolumn[\begin{@twocolumnfalse}
\section{2019年的回顾与前瞻}
\subsection{20191221}
\end{@twocolumnfalse}]5月14日勘误完成



史东:(0:00)

各位朋友你好,我是始终在今天节目中我们跟您谈的题目是2019年的回顾与前瞻。今天的节目中为您请到的是一位您非常欣赏的来宾,也是我们节目中的好朋友,那就是王孟源,王先生来和我们谈谈这个事情。在2019年,他在我们节目之中以及他自己的这个博客之中呢,谈了很多他对世界的种种问题或者局势的观察或者判断。那么现在是2019年的年底了,我觉得应该请他来和我们谈一谈他对过去这一年他的一些回顾,他的判断的结果,以及他对于未来,就是未来一年的事情的发展会有一些什么前瞻性的看法。现在说到这儿,就把我们今天的来宾请到我们的画面之中。孟源,你好,非常谢谢,非常欢迎。



王孟源:(1:06)

很荣幸再上你的节目。



史东:(1:09)

是的,是的。我想这次这个节目之前呢,我们商量一下,你提到了四点或者五点的,正好也是我想提到的。所以说我就很快地,在你打开话匣子之前,就把这四点或者五点,向我们观众很快地介绍一下,好吧。第一点当然就是关于香港的问题,第二点是有关于这个脱欧的问题,第三点是有关于美国经济的问题,第四点是有关于这个中美贸易协定,首轮协定的问题。那么然后我再加了,就是有一点就是川普这个被弹劾的事情。这个事情呢和以上几点呢都可能会有一点多多少少的关系。所以在我们谈话之中呢,您随时可以把它掺加进去。孟源,我就把麦克风交给你了。



王孟源:(2:00)

这些话题刚好就是我们今年谈过的。然后,我在我的博客上有也反复地讨论过好几次,所以这次只是做一个回顾,然后做一些补充啊。首先在香港,我想我在我的博客上已经讨论过。就是我认为现在基本上就是会慢慢的沉积下来。这个风头已经过了。事实上中央很有智慧啊,中共中央很有智慧的,没有派武警进去。那这样CIA 跟美国的国务院看到他没有上当,因为他希望就是你去进去把事情闹大,然后可以对中国采行制裁一类的东西。在目前为止呢,他就只能够在媒体上这样子造造谣啊,夸大其词,然后吹吹风。这种事情呢,因为大众媒体面对的是一般民众,而一般民众在二十一世纪,他的那个注意力是很短的,就是attention span 非常的短。所以你既然已经吹过风了以后,都已经超过半年了,没有必要继续下去,就你继续撑下去,反而会让他们的目标听众,也就是欧美的民众呢,起到反感。所以呢我我认为这件事情大家不必太担心啊,因为就是中共处理得很有智慧啊,就是我其实在今年年中也讨论过,你千万不要派武警进去,你派了武警进去以后呢,他一定会有人开黑枪来制造、升级事端。本周才刚刚在香港截获了一只AR15, 就是M16的民用型。那这很明显的,不是一般的民众,我想香港的那个对枪支的管制大概跟台湾相似,你要拿到一支这种全威力的军用步枪,是非常不容易的。那这背后应该会有很多的黑幕。我想交由香港的警方去处理啊,就是整件事情,就是大家只需要支持香港的政府跟警方,依法处理就可以了。那这件事情我认为会慢慢的淡化啊。



不过我要做一个Side Note,就是大概在一个礼拜之前,美国的一个小网站,他统计了一下,他说今年全世界有七八十个国家有示威活动,其中有二十几个国家的示威活动是长期的,而且激烈的,而且广泛的。但是你纽约时报跟CNN 对香港的报道比其他国家加起来还多,但是香港的示威呢,既不是人数最多的,也不是时间最长的,也不是死伤最多。那你这不是很明显的就是偏颇嘛?这一点呢我觉得就切中了他们这些白左媒体他们自身的弱点。你如果能够针对这一点来在国际舆论上好好地发挥的话,就是一个很好的反击。很可惜的是,这种事情应该是由新华社做研究,然后交给外交部的那个发言人。在这些纽约时报或CNN的记者提出这个问题的时候,当场羞辱他们。你看看我这里就有统计,今年二十个主要的demonstration ,有包括呃法国、印度、苏丹、利比里亚、俄国、西班牙、伊朗、伊拉克, 一共有二十多个。像这种事情呢,应该是主动出击的。我想基本上就是我们上次提过的,中宣部是扶不起的阿斗。我现在就是再伸一把手,再扶他们一把,提醒他们这种事情其实是可以主动地反击美国这些偏颇的媒体。



史东:(6:28)

呃,中国方面对这些事情的反应都是很被动。 



王孟源:(6:32)

我个人的感觉啊,因为我对大陆的事情似乎不是很熟,但是我个人的猜测是觉得有些中级的专业官僚,他们本身对体制的信心就不够。那所以事实上我在博客上也常常去写政治理论的东西。虽然我不是政治理论出身的啊,但是事实上英美的现在这一套政治理论本身就不合逻辑了。所以我可以从逻辑的方面来批评他。而在历史上呢,他也是很短的。就是在1970年之前呢,其实他们不是搞的这一套,是在1970年以后,新一代的那Neocon 智库才搞出来这样子。所以我们如果……这些都是有记录的啊,事实上学术界都知道,那我不太懂,为什么中国人不去研究美国的这些历史,还有他们本身在自相矛盾的事情。所以香港这件事情,当然there is silver lining to the cloud,就是在负面的事件之下,还是有点正面的影响。就是海外的华人跟大陆的大陆的知识分子群众,他们能够了解这些,他们中国所目前的处境是非常困难的。而且事实上美国的打压并不是有什么正当性的,他纯粹只是一种霸权争夺的手段。我想大家看到香港暴徒在街上这样子,然后听他们的言论非常的无知、愚蠢、幼稚,海外的华人跟大陆的看到自然就知道这些人是被利用,为什么呢?因为我们都看过红卫兵的事情,我们都还记得红卫兵是怎么回事,对不对?所以你看到香港的这些年轻的暴徒,看他们自己往自己的脚上开枪,损坏香港自己的经济跟前途。



史东(08:42 )

我们在旁边只能看。我想我说在香港这些年轻人学,你觉得是他们知道自己的愚蠢吗?还是说他们根本不知道他们的愚蠢?



王孟源(08:51 )

我自己是台中一中毕业的。但是几年前台中一中的学生跑去搞太阳花,那一看就知道他们是他们的智商是在五岁以下,至少政治方面的,就是完全不懂历史,完全不懂理论,完全不懂现实。他们是完全吃了英美的那一套宣传,而英美的这一套宣传呢,原本是在冷战时期针对苏联的发展的。但是到了七十年代之后呢,完全就变成一种虚伪的工具了。在冷战时期还可以说是因为事实上是一种战争形态,你可以不择手段。但是到冷战结束之后,他们还继续用这种来做颠覆,基本上是一方面打击任何可能挑战他霸权的国家。另一方面呢在无中生有的挑起事端,让这个国际财阀国际财团呢才能够趁着火中取栗,对这个



史东(09:48 )

提起这个无中生有的挑起事端呢,今天我们讲的现在我们讲的是香港。呃,当然你也知道,除了香港之外,美国对中国四周的其他地方也没闲着,对不对?呃,台湾是其中之一,新疆包括在里面,西藏前两天也出来了一个什么特别法或者支持法。



王孟源(10:10 )

其实真正造成中美决裂的是Trump,但是呢,他只针对着贸易跟经济。那么这些指指点点的,这是白左的那一套,也就是建制派的那一套。那他们用这一套其实就是我讲的很长的历史了。最早是从苏联开始的,在冷战之前,其实英国就用这套来抹黑德国。从一战到二战,你回去看看他们当时的那个宣传资料,基本上就是同一套宣传。那英国为什么会讨厌德国呢?很简单,不是因为他什么制度有什么不好。呃,人家德国自己搞自己的制度,关你英国人什么事?这主要是因为德国的工业越来越先进,他开始挑战英国的制造业了。



史东(11:05 )

这不是很耳熟吗?



王孟源(11:09 )

所以你你对历史熟悉历史的人就知道,美国人这一套过去一百年,已经先是英国,然后美国用,用在六七个不同的对手上的。你不能够太单纯地就以为,他真的是个字面上所说的那一套什么什么道德高地。不过我觉得刚好,因为Trump现在搞中美的分割。那这么一来呢,中国更不需要在乎那美国人这些立法,什么样的东西。而且Trump 到处得罪人,已经连欧洲都快要受不了他了。那这一点呢,就是他们这种软实力的手段啊,是必须要一呼百应。你光是一个人在那指指点点,一点都没有用,对不对?就是要打群架才行。对,要打群架。现在要上去打群架的,基本上就是像瑞典啊、德国啊跟捷克啊,这刚好是欧洲那个欧洲对中国最敌视的三个国家,是瑞典、捷克,然后德国因为他们就是被白左洗脑的最厉害。但是欧洲是一个很复杂的地方,他们有很多比较脚踏实地的政客。所以我觉得这些少数国家对中国的莫名其妙的敌视,不会有什么太大的影响。所以就是回答你刚刚说的问题。我觉得其实是因为Trump 的关系,使得白左的这一套花样,我想会越来越没有效果。那当初在2016年选举之前啊,Hillary 其实已经内定,由Joseph Nye来当她的National Security Adviser。Joseph Nye,就是软实力背后的发明人跟最大推动。所以现在他们搞的。这些啊香港啊、新疆啊、台湾的这一套,西藏的这一套,其实是Joseph Nye那一套。那你可以想想看,二零一六年啊,Trump当选,而不是Hillary ,那真的是中国逃过了一劫,这真的是一个战略机遇。



史东(13:27 )

不幸中的万幸。有关于这个你刚谈到欧洲啊,我们现在就谈谈这个脱欧的事情。最新的消息我想你也知道,就是这个英国看起来好像他们已经通过了,好像一月底一月三十一号就正式脱欧了。



王孟源 (13:45 )

这个消息啊,我们上一次聊天的时候,其实我已经把这个列为可能的结果。就是我那时候已经说这个呃,工党同意去选举,其实是非常的不智的。因为你这个大环境就是百分之九十的媒体还是在鼓吹脱欧,并没有改变。而且因为工党本身现在的那个政策呢,他是希望能够国有化。这个呢,被那些资本控制的媒体抹黑的很厉害。所以你看看这次的选举,其实工党也是被赶鸭子上架了。就是我后来在我博客上也提过,他之所以非得同意选举,有可能是因为Liberal Dem, LibDems,这个是第三大党。他的那个领袖叫Jo Swinson,她自己主动跑去跟Johnson合作,说我们要选举。那这两个党合起来就足够让这个选举发生,他们的那个选举法是说要有三分之二同意才能选举。但是国会可以以二分之一的多数通过特别的法案来办特别选举。我觉得工党是应该跟他斗争下去的,但是他也不是特别的聪明。一想到啊你看LibDem 已经要跟保守党合作,要通过特别法。那我干脆也合作好了,省得点面子不好看,那当然是很不智的啊。结果LibDem的背后就是那些实业派,就是国际资本,不想要脱欧的财阀。



史东:15:26 

是想要脱欧的还是不想要脱欧的?



王孟源:15:29 

是不想要脱欧的。想要脱欧的是支持conservative 的,那个保守党。那Jo Swinson就以为她可以在这一次,这基本上就是为了党的势力而牺牲国家。她知道那个名义上是说要反对脱欧,就是Libdems 的政策,是说我们绝对反对脱欧。可是你自己去主动提议要支持选举,事实上就是把脱欧的危险无限的放大。那这个呢,就是嘴上说一套,实际身体做另外一套。那这个做的另外一套就是为了自己的利益,党的利益。结果那个Libdems 这一次他的那个得票率的确是上升了百分之四,超过了超过了百分之十。这一次。但是呢,他的席位反而降低了一些。嗯,那最好笑的是Jo Swinson自己的那个选举都落选了,你看这就是报应这个事情。





史东 16:26 

这个事情,如果我们可以看脱欧这个事情,当然现在已经就是一月三十一号就脱欧了嘛,对不对?



王孟源:16:33 

就是很可能还会提早,就是因为一月三十一号只是最后的期限,他们现在是说希望一月九号就正式开始。那因为我在上一次讨论的时候,我已经指出整个英国土豪要脱欧的原因,其实是为了要避免那个反避税法。他们的最重要的是不要遵循欧洲的法案。刚好昨天Johnson 就出来说,不管我们有没有这个自由贸易协定,就是他们这个现在要还要谈半年的这个协议,我们绝对不会必须执行欧洲的法律。这个就是把那个底牌亮出来了嘛。



史东:17:12 

这就是就是无条件脱欧了嘛。



王孟源:17:15 

他基本上是说不管是有条件脱,还是无条件脱,反正我们不能够遵循欧洲法律。那意思就是说要么无条件脱,要么是一个很弱很弱的自由贸易协定。



史东:17:27 

就是他们的目的达到了嘛,对不对?



王孟源:17:29 

目的达到了,目的达到了



史东:

然后下一步你可预期的这个脱欧之后的一些情况是什么?



王孟源:17:39 

Well,脱欧之后的,也是我上次就提过了啊。那第一个就是北爱尔兰的身份非常的尴尬,因为这下子北爱尔兰必须遵循欧盟的法律,也就是爱尔兰的法律,反而不是英国的法律。那也就是说你海关必须设在北爱尔兰跟英国之间,所以你名义上北爱尔兰是英国的一部分,实际上是爱尔兰的一部。所以我上次已经提过,爱尔兰高兴地不得了啊。那个苏格兰当然是必须要脱欧,而且这次是有,法理上,就是他们五年前吧,是五年前,五六年前上一次公投的时候呢,那个时候支持独立的人呢只有百分之四十左右。所以投票出来也的确就是百分之四十几支持独立,然后就失败了。但是呢现在因为这个脱欧这件事情闹成这个样子,苏格兰可以独立之后加入欧盟,所以现在的支持比率已经超过百分之五十了。所以那苏格兰国家党当然现在是全力要推行进行公投。那Johnson你也知道,一旦让他公投,那大不列颠就变成小不列颠了。



史东:19:00 

因为现在苏格兰保不住了,爱尔兰也保不住了。我记得上一次访问你的时候,你谈到,你讲了一句话,大家印象都非常深刻。那到底说慢慢的英国就变成美国的殖民地了。



王孟源:19:15 

对,现在已经有很多端倪,就是因为Johnson 不只是搞脱欧,而且他把英国政坛上的很多的传统跟道德规范都扔到一边去了。所以,现在我的猜测是他们会大幅削减劳工的福利。然后对他们的医疗系统做半私有化。其实很好笑的是美国的大选现在正在考虑要把他们的医疗系统国有化。那英国反而是要私有化,那这下一来就变成,我上一次已经提过那个英国每年已经因为贫困交加要死好几万人了,我想这等保守党搞个两三年,这个数目会再加倍。但是我想提一下啊,即使是保守党在这次选举这样的大胜,他们获得了好像是百分之六十多的席次啊,你如果看他们那个选票来算选票的话,其实支持脱欧的党,就是保守党跟脱欧党,加起来只有百分之四十六的选票;公开反对脱欧的政党呢,加起来有百分之五十二的选票,那问题是他们分成了四五个政党,所以他们力分则散。所以你这个选票换算成议员席位的时候,就反而变少了。



史东(20:50 )

所以说这这我我这么理解你说的这句话,你看对不对啊,所以这一次会有这种四十六比五十二的这种差距。但是还是希望脱欧的通过了。主要就是因为这次选举并没有把脱欧和非脱欧做成投票的主题,是吗?



王孟源(21:09 )

对,因为它是一个普通的国会选举,而且国会选举本身有它的规则。这个规则本身就不是完全合理的。然后所以会产生这个现象呢,我有两个评论。第一个是造成这个脱欧派的分裂呢,这个第一个罪魁祸首就是LibDem 的那个Jo Swinson 什么。因为她为了自己党的利益,而嘴巴上说我们是反脱欧,最极端反脱欧。可实际上只是用这个借口来想要增加自己的席位,结果弄巧成拙,这种这是一种以私害公的行为。那第二个呢,我想提出来的是,你看看2016年的美国选举啊,也是Trump比Hillary 少了百分之三的选票,但是他还是大胜,那待会儿我会谈一下明年选举,我明年大选我的预计。但事实上英国跟美国那一年出现这种少数的选票,造成了多数执政,这样的结果,这不是巧合,这不是巧合。因为你这个土豪要利用媒体啊,要啊收买那些……不是收买,欺骗忽悠那些智商,政治智商很低的选民。他们当然是希望有杠杆效应。那他们选择的当然是这个选举制度里面内建的不合理的地方。也就是说你怎么样能够最以最小的代价来夺取政权。所以因为有这样的背景,所以才会有这样的现象,就是一而再,再而三地少数选票得到了多数的政权,而这个多数的政权要实行愚蠢的政策。对,我想这是我们必须要理解。那另外一个我已经讲了很多年了,就是这个选民,绝大多数是群众,绝大多数是非理性的。然后我上次也提过,就是呃turkey voting for christmas,他们这次真的是一大堆turkey voting for christmas。工党这次会死的这么惨,是因为在英格兰的北部啊,以前的那个老工业区啊,有很多失业的工人,他以前是工党的选民,这次呢全部跑去投保守党。因为他们真的相信说脱欧以后他们就会有工作。你想想看,这明明是镜中花水中月的事情,他们居然放弃了要国有化,然后给他们发福利的工党,而去支持这个很明显的会把他们搞死的保守党,这真的是没有办法。



史东(23:59 )

我想这个事情,我们观察这个事情很久的人都会同意这句话,就是“老百姓是容易受欺骗的,而且老百姓是健忘的。”嗯,然后美国不是有一句话很有名吗?第一次骗我是你的错,第二次骗我就是我的错。然后然后接下去还有一句话,就是你要问这些老百姓,你要被骗几次,你才知道你自己被骗。



王孟源:(24:29 )

其实丘吉尔自己说,反对民主制度的最佳证据,就是你随便找个选民跟他聊五分钟。(哈哈哈哈,丘吉尔厉害)我在博客上最新写的一篇文章其实就讨论过了。一直到丘吉尔,一直到其实到最近啊,英国他们的政政坛的那个政治哲学,其实是所谓的托管式代表。就是你投票给我之后,我并不是要照你的意见,而是以我的意见来决定什么是对你最好的。现在这一套这个所谓的delegate model application,就是我刚刚讲的,在一九七零年代初期啊,因为美国Johnson总统搞了great society 那一套社会主义的东西。所以财阀大幅的反馈,你如果去看那个美国贫富不均的指数,就是在一九七零年到达最低点。然后从七零年,一九七五年开始就开始慢慢的往上爬,有两个转折点,一个是雷根,雷根那八年很糟糕,然后后来到小布希在巴黎养大那。现在Trump出来,他这个统计资料还没有出来,不过我相信也是很糟糕的。你去看看美国现在的那个新保守派的智库,都是一九七零年代初年初期创立的,或者是扩张的。就是他们有些是一九二零年代一九三零年代创立,但是一直很小,一直没有什么影响力。到一九七零年代,忽然就拿到一大笔钱,然后就扩张了十几倍。



史东(26:09 )

有关于这个脱欧这个事情,你上次访问的时候,你结果你做的结论,我想再跟你印证一下:英国从此将这件式微,欧洲的金融中心,已将从伦敦移到可能德国,可能法国。



王孟源(26:24 )

巴黎跟法兰克福都在争啊,我想巴黎可能会得到多一点。就是你如果是像是美国的这些投资银行啊,他们的欧洲总部以往都是设在伦敦,对不对?那你现在如果跟欧盟一刀两断的话,你在这个金融本身就是一种虚拟的产业,它完全是由法律来决定你的那个怎么样交易。那如果你不能够在欧盟里面自由交易,你总部设在伦敦就没有意义了,那这我相信就会有大批的流失。那事实上金融现在是英国的第一产业。那伦敦的那个地产也会受到很大的影响。不过我觉得他们自作孽不可活,那也没有办法了,对不对?欧盟其实我上次已经说过是很高兴的,就是你这一次就等于是为其他的一些国家做错误示范。真的他们都不敢再提要脱欧,一提要脱欧都吓得不得了。英国所创下的例子实在是太可怕了啊。



史东:27:30 

那么这个这个情况啊,就是关于这个英国的衰落,金融从伦敦转到或者巴黎,或者是德国。对于中国来讲,我上次也问你同样的问题,这个大趋势的转变对中国有没有什么利害关系?





王孟源:27:48 

因为欧洲有种族歧视,还有那个白左的思想,所以他你中国不管怎么样去争取,他都是热脸贴到冷屁股上面。但是我觉得因为Trump的关系,所以欧洲决定自己独立出来做第三者。而英国脱欧呢,是有利于这个进程的,就是德国跟法国之间,法国一向是比较有独立精神的;那你看最近那个马克龙所说的,他甚至认为呃,(有关于北约的事情吗),对,北约都已经脑死了。他的意思就是说我们不必要再跟美国搞什么同盟了,我们欧洲人自己搞自己。那这个呢反对的,主要是德国,德国到现在还不愿意,但是川普却不断的在逼他们上梁山,这两年才又对他的那个北溪计划做出了制裁。那这个对德国是不得了,他们已经投资了百亿美元,要建那个北溪计划。



史东:28:57 

你说的是哪个油气管?



王孟源:29:00 

天然气管,这是他们能源安全的关键啊。这个美国会这样子,当然一方面是建制派,觉得从战略上来看啊,必须要继续让欧洲依附于美国。那Trump他个人不是这些这种想法,他根本不在乎这种这些态度啊,这种战略考虑,他考虑的完全是贸易利益,考虑的呢,其实是因为美国现在刚刚从油气进口国变成油气出口国,所以他忽然想到。(他要拿生意,他要抢生意),他是为生意而搞的。你这个德国如果不能够跟俄国进口天然气,你就必须要跟我买天然气,那不是很好吗?不过讲到天然气,我顺便讲两句,就是美国这个天然气可能明后年会暂时停顿一下。



史东:29:57 

什么原因?生产过剩?



王孟源:29:59 

原因是因为这个他们的这个页岩油,页岩气生产的不是传统的那些大石油公司,那些大石油公司都是市值几千亿的公司。他们这些页岩油,页岩气是十几年前一大堆小公司,(个体户嘛),对个体户搞出来的。那这些小公司主要都是发公司债,那我已经说过了,这次的这个经济的问题。就是因为他们的那个corporate debt 公司债发的太多,那所有公司在里面发的最离谱的就是这些页岩油,页岩气的企业。刚好呃,我们现在进入一个那个世界的那个经济成长稍微慢一点,所以消费没有成长的那么快。那然后生产反而是过量了,因为过去几年俄国还有美国都在拼命地增加产量。那结果现在这个价钱上不去。那再加上我们前几个月讨论过的就是那个REPO 的问题,就是在金融界本身有现金短缺的问题。所以呢,忽然之间这些这个这些搞页岩油页岩气的小公司呢,他们没办法roll over。没办法借新债来还旧债。那所以明年我想,应该会有一两家中大型的页岩油气的公司破产或者合并,那与此同时他们也会消减新投资的数额。所以呢,你可以估计就是页岩油气的产量在过去的几年一直是指数上升,但是明年可能是会停顿下来,就是不会上涨太多,当然也没有理由要掉下去了。因为他们如果不生产的话,他们就没办法还债,他们必须要解决。



史东 31:57 

谈到这个油气,你对这个中国和俄国油气管开通有一些什么什么想法没有?



王孟源 32:03 

这是战略安全。OK,那对俄国很好,因为俄国这样子对在欧洲那方面如果被美国搞掉了,它至少还有东方可以卖,这是合则两利的事情。那我觉得美国这过去这几十年这样胡搞,一个战略上很大的失利,就是即使在 Obama 期间,——Obama 其实还是很勤政的一个总统,而且也愿意听专家的建议,但是他就没有想得足够深刻——没有想到说你去跟俄国对着战干,就是一直扩充北约,然后一直支持像乌克兰的这种事件,你到最后一定是把俄国推到中国的怀抱里面,那现在已经基本上确定了,这个是板上钉钉。你看过去这一两年,俄国从上到下,从Putin一直到它的助手所讲的话都是中国是我们最好的朋友,我们是……这基本上是不正式结盟,但是实际上就是结盟了。



史东 33:16 

回头谈谈美国的经济。



王孟源 33:20 

美国的经济我们刚刚已经谈了一下那个Repo,这件事情我上次提过了,是因为它的有远因跟近因。我这远因我再重复一下,几个月前谈的就是,他的远因是因为Trump把赤字又搞大了一倍以上,然后在过去这两三年的那个联储会又反其道而其行,他把那个以前的量化宽松的钱收回来一些。我记得是收了大概1万亿,你这两个加起来这就是将近两万亿被收回来,那所以他就开始资金有短缺,然后资金短缺以后又刚好大银行觉得现在这个经济态势不是很好,他们为了保全自己,把这个不愿意再借贷出去,不再愿意再扩张他们的借贷。那么这个结果就造成了那些影子银行,影子银行在美国基本上就是Hedge Fund,他们就借不到钱,以后就变成联储会要进来。



我上次说大概是 5000 亿,的确是,现在算一算的确是 5000 亿。就是他们隔了要隔一段时间以后才可以公布他们自己的账单,要不然他们都是……他们很喜欢放烟幕的,就是不给你最重要的数字,但是只给你那个就片面的数字。所以实际上重要的东西你必须要自己进行另行估值。



王孟源 34:58 

现在有一个转机,就是上个礼拜那个他们的 unemployment 出来了,这个就是失业人口统计结果的,比预期的失业人口要少,也就是说这代表什么呢?这不是他制造业复苏了,因为你去看那个就业人口,就业人口并没有扩张超过预期,就基本上有一些临时工这种东西,还有零售业,这样子使得这个失业率的账面数字比较好看。



王孟源 35:36 

那但是这代表着它的那个美国的消费仍然是很强劲的,就是消费者靠借贷来消费,那这我上次也提过,就是因为从 2008 年, 2008 基本上是一个消费者的危机,消费者也参与了那个危机,消费者借贷过度,但是过去这十年的这个经济成长,消费者并没有参加到这个盛宴。



王孟源 36:01 

这个狂欢里面主要的问题是企业的借贷太多,然后那些我说过那些影子银行,他们为了追寻投资,他们把它放到金融市场,也就是股市,那这些公司他们发公司债也主要是用来诟病,或者是用来回购他们自己的股票来支撑他们的股价。对,这些钱最后都是进了股市了,所以现在的股市是非常的虚胖。



王孟源 36:35 

那这个是我上次说过,是会有一次像那个中型的危机,就是像 2000 的那种危机,那原本看来是明明年初会发生,但是现在看来有可能会被推到 20012021 年初。就是因为消费者还愿意花钱,因为消费者在过去这十年他们的那个借贷没有太过分,所以你如果要让他们继续借贷,还是有空间的,然后那个年初会又已经放了 5000 亿进去。



王孟源 37:05 

那我昨天才看到一则新闻,说过去这 10 年是第一个 10 年,第一个decay,美国没有任何经济消退,就是连续 10 年的成长。OK,你认为这是好事吗?其实很简单,你全部现在这个联储会的量化宽松的总账户,账面上总全部是多少?4万亿出头。那你平均不是每年 4000 亿吗? 4000 亿是你 GDP 的多少? 2\% 点多,你这么大的等于是无中生有,往经济里面每年加 2 点多的那个成长,那这样子一来,你当然不会有衰退了。衰退是说你降到 0 以下嘛,对不对?那你这个原本经济它的那个原动力只有在 0 左右,现在也变成表面上变成2,对不对?就是基本上是他每年无中生有的冒出 4000 多亿的钞票,这是他为什么美国过去这十年经济这么好,也为什么联储会这么紧张?因为这种事情你干得过头,就会让美元失去它的国际储备货币的地位,一旦失去这个货币的地位,你再这样印钞票就会就是有后果的,不会有外国人来为你买单。



王孟源 38:24 

这也是为什么两三年前他们开始 unwind ,开始回收那些资现金,但是结果他一发现说这个美国经济已经吃药上瘾了,这个没有办法。所以他今年又马上一看风头不对,马上又把这个又放了 5000 亿,而且好像是还要继续地放。那继续地放的话,那这刚好正中 Trump 的下怀了,因为 Trump 的意思就是说我不管你怎么搞哈,反正在大选之前经济不能跨。哈哈哈,那我过去这一年一直说我觉得Trump的大选有危险,这个危险其实不是你现在所谓的这个 impeachment(弹劾) 什么的这样的东西,嗯,而是你如果明年有经济衰退的话,那很明显的他就不可能选得上。但是所以明年的大选就取决于这个经济衰退到底是 2020 年发生还是 2021 年发生。那我在我的博客上已经说过了,这就好像雪崩,你知道它一定会雪崩,但是你不晓得它是什么时候。



史东 39:33 

所以这个我们在这个节骨眼上也顺便谈一谈有关于美元的这个霸权。你刚谈了美元霸权的事情,你觉得在 2021 年美元霸权松动的可能性有多大?因为现在有些这个周围的,这 Repo  的,就周围的这些迹象慢慢显示它是在不断的在往前进行之中。我就是说会不会什么时候会从从一个量变到值的改变?



王孟源 40:06 

这同样也是像雪崩,对不对?嗯,一直下雪,慢慢就是。



史东 40:10 

也就是一般人讲的这个最后一根稻草的理论。



王孟源 40:14 

对,最后的那一片雪花。



史东 40:16 

对。



王孟源 40:17 

对对对。但是你不晓得是哪一片雪花。这个美联储他知道你如果这些量化宽松的这几万亿不回收的话,这个你就会增加美元被取代的几率。那但是他,我刚刚说过了,他们现在是骑虎难下,没办法稍微回收了一点点,不到1万亿,这个金融界就锁住了,所以他们必须要反过来继续做量化宽松。



王孟源 40:53 

当然有很多国家会主动地想要把美元搞下来,因为美元是美国霸权的基础之一。对,那中国跟俄国,尤其是这里面最积极的。其实在 2008 年的时候,Putin是一个很有战略跟战术修养都很强的人,他那个时候就建议中国对美国落井下石。其实我觉得那时候中国没有支持,反而会反过来全力支持美国复苏。我觉得是不对的,就是你不需要跟着Putin一样那样做明显地打击美元的事情。



王孟源 41:37 

就是那当时Putin说你应该把美国的国债统统卖掉,那即使你卖掉也没有用,因为你那时候就是1万亿的国债,美联储要印4万亿了,但他再多印1万亿,也一样把这个国债买下来,就是你不需要去做这种很明显有敌意的事情。但是 2008 年的时候,中国投了几万亿在自己国内经济过度刺激,所以一直造成现在的这个房地产虚胖,到现在还债。这是一个错误,这是一个战略错误。那现在来看,我想习近平的政权是很了解这个形势,而且美国现在也更弱而更孤立。所以我觉得很有意思的是,一个多月前金砖国家的会议在南非举行,当时由俄国人提议金砖组织合作发一个数据货币,就是虚拟货币。



王孟源 42:39 

嗯,这个为什么呢?因为这个其实最合理的就是在 1944 年,他们那个 Bretton wood,就是当时二战期间,他们准备要好开会争吵,说为此在战后要怎么重建金融体系的时候,那个时候卡因斯就已经说过了,最好的设计是应该有一个世界货币就是不由一单独国家控制的。结果那时候因为美国的拳头最大,他就说我不管,你必须要用美元来做国际储备货币。我想现在第一个人民币还没有那个本钱,可以说取代美元,事实上连欧元都还没有办法取代美元。嗯,那你退而求其次,最好的办法就是建立一个国际性的货币。那刚好有现在有这个虚拟货币,现在的什么 bitcoin 那个都是骗人的。



王孟源 43:37 

我上次一年多前我跟你聊过,那个是没有主权背书的,然后根本就没有货币的任何特性,纯粹就是一种金融炒作工具,但是有主权背书的虚拟货币就是很严肃的东西。那在这个金砖组织的会议上面,他们讨论是不是要合作发一个这个货币,我现在还在等后续的消息。就是说我想俄国人提议以后中国人,中国代表一句话都没讲,但是实际上你可以确信他一定是中国跟俄国在事先就商量好了,但是公开来看,中国人这一句话都没讲。那南非的总统给了一个演讲,说他支持这个。那印度,印度目前是自顾不暇,因为他们刚好现在经济开始有问题了。那但是他们也被 Trump 打击了两下,所以我想他们应该不会有什么啊反对的意见。



王孟源 44:40 

这个真正最大问题是巴西去年选的那个总统是号称是拉丁美洲的Trump,对,所以他,哈哈,他的,尤其是他的儿子被任命为一个部长,他的那个儿子是绝对的亲美派,真的是把美国当做宗教一样来崇拜。但是很巧的是他在那个会议里面不作声,之后不到一个月 Trump 就对巴西开了关税,所以我觉得我乐观其成,我认为这个搞成的机会很大。不过那个会议之后一个多月我们都没有听到任何消息,我想是因为……



史东 45:18 

我想台面下一定。



王孟源 45:20 

台面有很多动作作。对,我可能还要再等半年才会看未来几个月了,说不定还要半年。



史东 45:26 

因为大家都意识到,这个不需要再谈一些,大家都意识到美元这些年来的所谓的倒行逆施。



王孟源 45:34 

这些不只是现在……已经不只是它这个量化宽松这样子掠夺了。搞那个就是还搞制裁。这个我们讲过华为跟。



史东 45:49 

法国。



王孟源 45:50 

那个公司,现在 Erickson 也被罚了,一罚罚了 10 亿美元。对,真的是太离谱了。你一个欧洲公司在亚洲行贿只因为用的是美元,所以你就被罚了 10 亿美元的?我想这下来说没有人会反对,除了美国人以外没有人会反对说创立一个国际性的新储备货币。



史东 46:16 

所以这次也是时间上的问题,来看美国能够档到什么时候了,能够威胁利诱到什么什么地步。



王孟源 46:25 

我想现在就是还在下大雪了,我们等着那个雪崩吧。对,我想这个金砖的虚拟货币应该是明年就会出现,那出现之后会比以前所有中国在金融方面的的进攻措施回响都要大,你这样看中国做了什么措施?其实很多,比如说他试着要跟人民币国际化,嗯,这个要跟人民要欧元直接转换,后来没有搞成是因为不方便。OK,就是因为习惯的关系。那另外一件事,比如说他搞那个石油的期货,搞黄金的期货,这个就比较有成,有一点成功。但是原来也没有说真正取代世界领先的那些既有的系统。但是这个金砖货币我是很看好的。



史东 47:28 

因为我在一直找一张资料,找不到了,不知道放哪去了,就去关于伊朗和欧洲有关于石油的交易,有他有一个。



王孟源 47:42 

他们建了那个结算。



史东 47:43 

对,结算体制。



王孟源 47:46 

他们建了结算体制,但是你光是一个结算体制,光是用来跟伊朗交易是没有用的,必须要……



史东 47:53 

对,其实这个因为这个结算体制也在迅速的扩大之中,现在也好像也有 14 个国家在里面了。



王孟源 48:00 

对,但是。



史东 48:03 

还是不够看,对不对?



王孟源 48:04 

不够看。对。那,哈哈哈。 Erickson 被制裁了。 OK 啊,看来是不了了之,欧洲大概又要哑巴吃黄连。但是德国为了北溪计划这个是昨天才发生的,所以我不晓得他们的反应会有多么强烈。这很大了,因为这已经不是一个公司被罚 10 亿美元的事情,而是整个北溪计划未来的天然气供应都会有问题。



史东 48:32 

在美国想搞这个计划已经搞了很久了。对,一直在搞。



王孟源 48:37 

对,德国其实已经确定就是他在美受美国的压力,不是在 Trump 之下才开始受压力。对,在Obama 的时期就受压力。对,那你 Obama 跟 Merkel 交情那么好, Merkel 到最后还是决定必须要做。



史东 48:53 

因为,对,我想他们对美国也有相当程度的认识,把自己的生命线交给美国这么一个国家,那怎么可怎么得了?啊哈,这对不对?这个也不需要太多的知识就可以知道的事情了。



王孟源 49:08 

我要生命线交给美国的国家就是乌克兰了。那你看美国是怎么对付对待乌克兰?



史东 49:15 

这又是一个,你要被骗几次才知道你被骗,哈哈哈,这句话,哈哈哈,对不对?谈到这里我想谈到谈一下这个有关于中国中美这个贸易协定,首轮可能要签协定的这个事情。你对这件事情有什么样的看法?



王孟源 49:38 

我还是一样,就是四五个月前我想上,我也在你的节目上提过,我认为最大的可能就是回到6月的那个停火协定,结果你看一看,现在细节还没有完全出来。嗯,事实上也就是回到6月的那个停火协定。



史东 49:54 

现在 Trump 好像又跟习近平打了一个电话, Trump 说这个好像他还蛮乐观的。



王孟源 49:59 

可能会在1月初签。那我所知道的现在目前是泄露出来的一些信消息,大致这样,他就是回到6月的那个协定,但是中国必须要再买,有说 400 亿,有说 500 亿,美国的农产品。基本上,所以你这个,我四五个月前我在你的节目上说,我认为这一次谈判谈出来就是回到6月的那个停火协议。那现在回头看,的确就是回到6月,那停火协议加了一点点,就是中国要买这些农产品,然后我跟你解释一下为什么中国愿意再加上这些农产品。



王孟源 50:42 

第一个是中国现在的那个猪肉很贵,因为他们有那个非洲猪瘟,所以他们的确是可以进口一些。而且美国现在的那个农产品因为现在正在经过过去三四年最糟糕的农产品不景气,所以有很多农场主正在破产关门。那美国的农产品硬是要比巴西他们阿根廷那些还要便宜,所以中国事实上如果要自由出去买的话,本来就应该选择美国。那这种购买其实并不是中国退让给Trump,而是互利双赢的,所以他们愿意,这是第一点。第二点是他帮助 Trump 连任,这当然是长远的战略考虑,因为你这个战略机遇从 4 年变成 8 年,当然是要好很多,对不对?最后一个好处是 Trump 最近已经对德国还有欧洲磨刀霍霍了,你如果现在中国跟美国达到这个贸易协议,你基本上就鼓励 Trump 去跟德国动手,那对中国来说又是一个战略利益。所以就是因为有这三个三点,第一点是中国自身有需要,所以并不是一个退让,那第二个是帮助Trump连任,第三个是让Trump去跟德国打架,这三个考虑,所以结果导出来就是回到当回到半年前6月的那个停火协定,再加上中国去买一些农产品。



史东 52:26 

那么看起来这个中国政府还是希望Trump连任的是吧?



王孟源 52:30 

哦,当然了!当然中国不是Putin,它不会像Putin一样搞那些,但是这种合法的手段,就你就是谈国际协定嘛,这你不能够说是什么见不得人的东西。那他还是希望能够帮的Trump,然后让他安抚他的农民的支持者。我刚刚已经讲过,现在正有一大批那个农场主破产,上一次有这么严重是在 1980 年代了,就是每一次美国搞产业空心化以后,它刚好也是创造那个环境,就是以大吃小,就是方便那个财团资本来吞并小的农民。上一次是在雷根的任期这样子,那这次又是被 Trump 搞成这个样子的。因为你的那个资本资金很便宜,资金太便宜了,然后那个你那个农产品市场价格上稍微有一点波动,这些小农场就活不下去了。所以美国在过去这四五十年完全被资本控制,资本财阀控制,有的是土豪,有的是国际财阀,他们已经完全掌控了不只是政权,而且是话语权,就是他们的这个学术界也完全被收买了。其实要买几个教授来写有利你的思想的论文,其实是最便宜的,也是他们一开始的第一步就是 1970年代,我刚刚说智库还有学术界的那一步,所以现在美国已经是完全入误入歧途了,你想想看它的那个高科技工业,最后除了芯片之外,(芯片)就是 Intel ——Intel现在它的制程已经赶不上台积电了,他制程已经落后给台积电一代——然后那个波音是另外一个,就是它基本上就只剩下Intel 跟波音,那波音现在的情况我写了好几篇文章,完全是被他们这个,他们现在这个商学院过去这 50 年搞的这些完全金融化的那些手段,把他的那个技术的文化给挖空了,变成短视近利,完全是追求股票的价钱才会搞出现在这个毛病。但是这是文化的问题,就是整个企业文化已经烂掉了,这不是换一个 CEO 就可以挽救的。就好像美国也是一样,整个文化都已经烂掉了,整个体制都已经烂掉了,不是换一个总统就能够挽救的。



史东 55:07 

你刚刚说的那个有关于美国现在被财团或者国际财团或国内产生控制,然后是不但是立法控制,然后话语权的控制,然后花一点钱请一些名人教授写一些文章,这些事情,东西。这句话是彻彻底底的点出了美国现在的状况。



王孟源 55:31 

那些都是1970年代就开始的,最可怜的是英国完全相信了这一套,就是英国的保守党已经完全是的照抄美国的这一套,那所以英国现在也这么惨,那另外一个完全相信这一套的当然就是乌克兰,那惨到那个程度我想大家也都知道的。



史东 55:50 

OK。在这个节骨眼上,我把话题带到今天最后的一个段落,那就是有关于 Trump 弹劾的事情。



王孟源 55:59 

我觉得这完全就是作秀,那个民主党之所以会弹劾,纯粹是因为他们终于抓到实证了。而且这件事情你要讲理的话,Trump是绝对讲不过的,但是问题是美国的这个政治现状是两端完全极化, 百分之四十几的选民绝对会支持民主党,不管你外面风吹雨打,另外 百分之四十几也绝对会支持Trump的,不管天崩地裂。那你中间剩下的大概 10\% 出头的中间选民,你去争取才是。你这么多的动作,其实就是为了争取那么一点点的中间选民,他们搞的这个 impeachment, 这个弹劾案其实完全就是为了要羞辱Trump,结果是什么呢?结果是把中间选民对创的支持率从百分之四十几变成 百分之三十几,所以你中间选名是10\%,然后你搞了这个弹劾案,又得了10\%,所以你基本上是得到全部选民的1\%,这就是它众议院搞的事情。



史东 57:10 

你从这个角度看起来,其实你就很清楚了。



王孟源 57:15 

哈哈哈,搞了半天就是这个,然后Trump也知道,也心知肚明。所以虽然美国的那个参议院的共和党,包括 McConnell,就是那个参议院的领袖,说是要把它很快地解决。Trump他自己是说,我要你们好好地搞,慢慢地搞,仔细地搞,OK,把它变成一个长篇的大论。为什么?他觉得我可以对着你干,你在众议院里面搞这些听证来羞辱我,我可以在参议院搞这些听证来羞辱民主党,羞辱你。但是他们这样争来争去,其实就是那 1\% 顶多 2\% 的选民。那事实上决定这一次大选的不是这 1\% 2\%,而是明年的那个经济衰退到底是发生在 2020 年还是 2021 年,因为你光那个经济衰退就会影响将近 10\% 的选票,所以我觉得他们都是在浪费时间的,这就是现代英美民主制的情况,这些政客不务正业,百分之百的时间都花在为自己选举,为党派的利益。



王孟源 58:30 

你看看那个我刚刚提到英国的那个 LibDem,它卖身给那些国际财阀跟实业派,然后以为这是一个大好的良机能够窜起,结果搞到最后,机关算尽,那个党魁自己落选,哈哈哈。



史东 58:50 

孟源,关于 Trump 这件事情,它被开除的可能性是0,这是一个是这个毫无悬念的事情,对不对?



王孟源 59:00 

这民主党他们自己也知道这个是零,他纯粹就是为了要那个听证会上面羞辱Trump,作秀。



史东 59:07 

对,我在离开这个议题之前,因为这个问题很多人都在讨论,包括我看到很多的美国的媒体都要觉得就是为什么民主党要这么做,完全是他们是基于一种报复的心理吗?还是他们对于 Trump 当选是极度的愤怒?



王孟源 59:24 

那这是共和党,那是共和党的说辞,民主党还没有那么笨。其实那个民主党一旦掌控众议院,就是他掌控众议院已经一年了,对不对?他一掌控。就有一些比较年轻气盛的人说要弹劾Trump,但是那个Pelosi一直说我不行,因为你那个没有好的借口的话,你这样子太难看了。那这次是因为罪证确凿,事实上你照理说也是可以弹劾他才会真正去做,然后但是他实际上去做的原因, Pelosi会去做的原因,她也是搞了几十年政坛老狐狸,她也知道你这个必须要有实利,这个实利不是为国家,是为民主党,也就是我刚刚讲的。



史东 01:00:12 

为自己的利益。哈哈哈,不是为国家的利益。



王孟源 01:00:16 

不是国家的利益,对。



史东 01:00:19 

最后一个问题,我们今天谈了那么多,从香港谈到了 Trump 被弹劾,有关于这个 2019 年的过去,以及看到 2020 年的来到,你有些什么样子的感想,我会有一些什么样子的这个观念可以和大家沟通沟通的。



王孟源 01:00:42 

我在这里顺便跟大家说一声,新年快乐。嗯,Happy Holiday。我觉得过去这 5 年多,我在我博客上面强调的一些比较理论性的、政治性的、战略性的发展,在 2019 年都已经开始实现,就是它的后果已经开始呈现出来。我想你如果是真的睁开眼睛,用理性去考虑这个世界的现况,你现在应该是非常容易地接受我所指出的这些现象跟事实。我认为中华民族自己要有自信,英美这个体系现在其实已经是在夕阳西下已经走到头了,你可以看出也是一种末代朝廷的感。当然他家大业大,跟台湾不能比,但是你事实上可以看出他到处都在衰退。我想,你如果看罗马史的话,他们到了第三世纪之后,你连那个他们的护甲都开始偷工减料。



王孟源 01:02:04 

OK,那所有的好的东西都是两三百年前做下留下来的,现在美国已经开始有这种感觉了。你看现在他们要送人登月,花了那么多钱,结果连……前天才送了一个火箭要去跟那个国际空间站对接,结果居然又出了毛病,没有办法上,他们要以前那个土星 5 号, Saturn V 火箭,世界上最大推力的火箭,他们为了重建这个搞了已经 16 年了,什么东西都没有,到现在还没有一个影子。



王孟源 01:02:48 

你说这像是当年全盛时期的吗?现在的那个电脑这么先进,你说这在工业设计上方便的太多了,然后又有太多的科技的进步,你反而倒退回去了。你要建一个大火箭,反而是一副力不从心的样子,这是你在这方面这一点上就可以看出美国已经不是 50 年前的那个美国,那它之所以会衰退,其实是一个从政治上、经济上、文化上、学术上总体的衰退。



王孟源 01:03:25 

那中华民族我刚刚讲过,要有自信,要遵从理性逻辑来观察事实,来仔细地、客观地想,什么才是最佳的策略?不能够盲目地照抄。台湾过去这 30 年的困境,其实就是,唉……蒋经国过世前后那段时间对美国的崇拜达到顶点,然后照抄了他们的制度跟思想,那时候就其实就已经注定其后 30 年腐化的过程。我希望其他的华人社会不要重蹈这个覆辙。



史东 01:04:11 

是,这是大家都看到的,也就是在过去这 30 年, 40 年是你说的 50 年,慢慢看到这个事情的演变。而且你刚刚在节目中也提到了美国对其他国家,世界上其他国家像,就像乌克兰那些其他国家,像日本,像苏联这些国家,就是手段以及他们的做法,也提供了很好的一个给中国见识、长见识、受教训的机会,不是吗?对不对?



王孟源 01:04:44 

对,就是我想中国过去这 40 年的发展太快了,那所以文化上、体制上的自信,思想上可能还转不过来。你现在 50 年, 50 多岁的那一代很多还是崇美的,就是宗教信仰已经成为一种宗教信仰。那事实上你在心理学上是一个已知的事实,就是到了 35 岁之后,很难改变一个人的人生观跟世界观。那这些人因为他们成长的过程中,中国跟美国的差距太大,所以他们完全的自卑感,我觉得这是很不好的。所以我一直希望中国能够教育中国年轻的这一代,然后让他们能够有自信。这个自信不是对中国的什么东西都盲目的自信,不是自大,对理性思考逻辑的自信就是,我相信,我们的逻辑思考能力不会逊于美国全盛时期的那些人。所以他们当初搞出来的崛起,所依据的那些思想跟那个技术,我们也可以搞出来。



史东 01:06:03 

非常好。孟源这个有耽误你的大概一个多钟头的时间,听了你这个非常精辟的见解,最后说一声, Happy New year, happy New Year,好棒,谢谢。

\twocolumn[\begin{@twocolumnfalse}
\section{新冠肺炎}
\subsection{20200224}
\end{@twocolumnfalse}]Credit: anonymous



史东 00:14 

观众朋友你好,我是史东,从今天开始连续几次节目,我们想为您做一个系列,这个系列是和 2020 年的新冠肺炎有关,我们这个系列定的名称叫做:观察 2020 年新冠肺炎。这件事情,因为我们都知道这个事情发生到现在,我们也意识到了,这个世界以后会不一样了,这个事情以后对我们有什么样的各种方面的影响?从国际关系,从医药的治疗以及从卫生的防疫,或者就是叫公共卫生体系上的防疫,还有从生活习惯,从人际关系,从社会的互动,以及各种经济的行为和商业的这种模式,国际之间的贸易。更有人谈到军事上的这种影响,譬如说生化武器这些,还有很多我都没想到的事情,不过有一件事情我觉得你也会同意,我们都知道今后我们在这个地球上的生活以后会不一样了,我们这个系列的这个做法是请一些我们非常尊敬的这个所谓的 opinion leader 就是意见领导者在节目中和我们分享他们对这件事情各种角度、各方各面的观察,我们可以从他的观察之中能够学习到一件事情。今天我们在节目中为您请到的是您非常欢迎、非常喜爱的一位评论员,就是王孟源王先生来跟我们谈这个事情。那么节目中我们会谈的事情,我在节目前和王先生稍微沟通了一下,他说他希望能够从这个医药治疗,还有疫情的防治,还有这个对经济上的影响这三个大方向来着手和我们谈谈这件事情,首先我们把王先生请入我们的画面之中,谢谢,欢迎。



王孟源 02:11 

史东,很荣幸再上你的节目。



史东 02:15 

有关于这个事情的这个观察点您就开始和我们谈的,因为我不想把这个话题设定得太小或者太严谨,这样的话反而失去了你的这个观察的尺度。



王孟源 02:29 

几天前其实你刚联络我的时候,我也跟你说过,因为我自己是学科学出身的,所以我想我就从科学的角度开始谈起。一般人对传染病学跟病毒的特性都不是很熟悉,那其实人类的传染病几乎都是从动物跨越物种跳过来的,那尤其是这种所谓的瘟疫或者是大流行,他们基本上有三类,第一类是靠接触传染的,这个就像是 AIDS 跟Ebola,是你体液像流血或者是唾液这样接触传染的。



王孟源 03:12 

第二种是靠害虫来传染的,比如说即使最近它有鼠疫,将过去的所谓的黑死病都是靠跳蚤或者疟疾是靠蚊子来传染,那最后一种是用呼吸道传染,那这个呼吸道传染的也有很大的威胁,也可以成为所谓的大流行。在英文里面有两个字,一个叫做epidemic。 Epidemic 指的是在一个国家或者一个地区已经开始大规模的人传人,那最可怕的一个观念叫做Pandemic, Pandemic 是说在世界上很多个地区跟国家都在很快速地人传人,大规模地在社区里面这样传,那但是因为 WHO 世界卫生组织原本有正式地定义Pandemic,那我们现在这个新冠肺炎其实已经很接近了。



王孟源 04:13 

因为过去这一周你看到在日本跟韩国的染病人数都接近1000,然后在中东、伊朗甚至意大利都有接近都有 100 左右的病例。那在这个在这种情形下,如果是按照 WHO以往的定义的话,就已经是可以考虑要叫做Pandemic。但是我插几句话,他们刚刚改了规则,就是把这个 Pandemic 这个字拿掉了,不再有正式的意义。



王孟源 04:46 

为什么呢?因为 2009 年的时候美国有一个猪流感,那个猪流感后来传染得很快,到现在还是外面传来传去,那他们就宣称pandemic,但是他没有想到那个他的传染力比原先预期的低,所以就没有像真正的我们常见的这种流感,就是后来他的那个感染力是上去的,但是当年的感染力其实并没有很高,远远的没有到这次的新冠肺炎这么严重。所以后来 who 被美国人骂说,没有那么严重,你还没有pandemic。



史东 05:27 

哈哈哈,一定是影响了某些人的生意,是吗?



王孟源 05:31 

哈哈,是美国人不喜欢有人指着他的批评,所以 WHO 后来就把这个规则改了,他们不再用 pandemic 这个字了。不过不管怎么样。



史东 05:46 

不过现在已经有很多人,很多学者专家开始呼吁要这个 WHO 要慎重考虑,要用 pandemic 这个状况来形容这个事情。



王孟源 05:59 

其实 WHO 一直对中国很有信心,就跟我一样,我对中国人这一次这个新冠肺炎是自从 1918 年的 Spanish flu 以来,世界上最危险的新传染病,就是以传染力为来讲是最高的,所以百年一见的瘟疫,那老实说这种程度的这种传染力的瘟疫,世界历史上还没有任何一个国家能够真正靠隔离检疫把它压下来。所以这一次中国到目前为止,其实你如果不考虑这些外国的像日本、韩国、伊朗这些国家的话,中国现在已经控制住了这个疫情了。如果你不考虑这些其他国家的话,我想再过两个礼拜他们可能会可以宣布胜利。现在的问题就是已经传到这些治理能力比较弱的一般的国家,所以中国所做的努力可能会前功尽弃就是,当然也不是说完全没有意义,就是他至少拖延了一个月,那像这种呼吸道传染的疾病就像流感一样,它有季节的,就是在冬天从 11 月到2月之间是最流行。这是因为冬天的它的空气比较干燥,比较寒冷,这时候我们的呼吸就到的粘膜保护作用会下降,所以就特别容易传染到。



王孟源 07:30 

那所以这是为什么流感会有季节,它这个从北半球到南半球,然后再从南半球回来,那因为我们现在已经到2月底了,下个月就是春风,那对我自己还是抱着谨慎的乐观态度,就是希望虽然有像伊朗这样的国家,他们的治理能力是有点可疑的,但是毕竟也是在还是 100 多个病例,这个还不算太困难。



王孟源 08:03 

这个中国是有8万个病例,事实上一定是少算,不是故意少算,而是因为这个新冠肺炎很麻烦,我待会解释它的特点,所以,并不是说一定会就变成Pandemic,但是有这个危险了。其实我在一个月前开始就开始评论说如果不是中国,如果是发生在任何一个其他国家的话,就像就会像 2009 年美国那个猪流感一样,就是两手一摊,让它自然发展。但是因为,它的传染性不但很高,而且致死率是大概 2\% 左右,那这个是比流感高一个数量级还多,流感的致死率是千分之零点五,不过流感的致死率会低,像是致死率这种东西,它不是光看病毒本能生的能力,也看我们医疗方面的准备。



王孟源 09:05 

对,如果,对你,如果每个人都有打疫苗的话,那当然感染力就下降,然后致死率也会下降,那这是因为一个全新的病毒,所以当然没有这些东西。事实上刚开始的时候我在我的博客上也提过一定有战争迷雾,就是刚开始没我们,现在事后诸葛亮可以看得很清楚这些有什么特性,我可以在这边开怀的谈,但是一个月前我就没有这些消息,对不对?这大部分这些信息都是过去这个月出来的。我也注意到有一个很特点,就是即使是英美这种平常找到中国的尾巴就要狠狠咬一口的时候,这个他们上层社会的媒体这次都很有节制,因为他们知道这是一个自然界的反击,中国其实是在替人类挡第一线,所以你看像是像是经济学人,像是外交政策这种都是半专业性的期刊,很程度很高的期刊,他们就不会说什么中国出力不当,什么这个所谓中国处理不当。这些都是在网络上或者是那种所谓的八卦新闻,他们才会这样讲。



王孟源 10:30 

那我看到这些新闻我就觉得很好笑,因为我自己在读世界历史的时候,看到那个拿破仑的战争史,这就是所谓的事后诸葛亮,你说拿破仑不是军事天才吗?他是几百年一出的军事天才,但是事实上你没有那个消息,你没有那个资讯,你就没有办法,不可能做出正确的决策。所以我觉得这一次中国政府被很多事后诸葛亮在批评的是很不公平的事情,他们真正没有做好的是预防。就是上一次 SARS 已经释放了一次。那释放了什么呢?释放了就是这种新的流行病。新的 Pandemic 或 Epidemic 是怎么来的?英美的这些比较高层的媒体,他会客观地讲,也是因为他们。事实上这个危险已经讨论了几十年了,不是新的。



王孟源 11:27 

最近这一波讨论是大概 40 年前, 1980 年初,HIV AIDS传到北美的时候,那个时候大家说这个是哪里来的对不对?后来经过 20 年,他们才终于做了很多的侦探工作,最后发现连用基因去算它,这个 Hive 其实是从黑星星的一种 SIV, Simian Immunodeficiency 跳过来的。那这个他们算一算,结果刚好是大概 100 年前 1920 年的时候跳。



王孟源 12:06 

那当然你已经不可考当初是怎么跳的,不过他是接触传染,所以很可能是西非的部落,部落的那个民众猎人去抓了黑猩猩,把它宰来吃了,在宰杀的过程中见到血液或者是体液,然后然后病毒刚好突变。病毒这个突变跨越物种是一个小几率的事情,你通常是要有大量的病毒一次传染,然后多次的尝试才会发生这种换句话。



史东 12:40 

说不是常见的事情,是少见。



王孟源 12:44 

1/ 1000000, 就是你这个非洲的部族,去抓黑猩猩来吃,这个不是一天两天的事了,这几千年来都是这个样子,但是几千年来都没有得到HIV,但是刚好就在 100 年前得了HIV,然后得了 HIV 以后也没有马上就传出来,他在那个部落乡下部落这样传来传去也就是几百个人,然后一直到 1960、70年代才传到西非的大城市。



王孟源 13:15 

然后到了大城市以后酝酿了一下,然后这下有了几万个人,几万城市里面有几万个人,然后的最后才因为有飞机的联系,就是现代的交通方便,所以才会,所以基本上瘟疫这种东西自古以来就一直有传染病,这些传染病都大部分都是从动物跳到我们人体身上,这些病毒或者病源在各种野生动物上其实是以佰甚至以千计的。然后我们每天跟这些动物接触可能也都是一百以千计,但是因为它跨越到人体的这个几率是小于 1/ 1000000 的,所以你看起来就好像好久才来一次啊?我现在提一下这个一个例子,具体的例子这个新冠肺炎,这个是一种冠状病毒,我们知道冠状病毒会影响人类的只有 7 种,包括有一些是我们普通感冒,也有些也是冠状病毒。嗯,那另外就是,但是最新的三种就比较恶名昭彰,就是SARS, MERS,SARS是 2003 年, MERS 是 2012 年,然后这次是 Ncov就是新冠。



王孟源 14:39 

但是 3 年前因为MERS的关系,有人到Borneo,就是东南亚那个岛屿,那个到 Borneo 去找蝙蝠洞,去研究那个蝙蝠有什么病毒,结果他们不测也还罢了,一测吓了一跳。他们光一次采样就在蝙蝠的体液里面找到 400 多种冠状病毒, 400 多种!这蝙蝠真的就是冠状病毒特别喜欢的天然寄主,然后他们就到附近的村落去,看看那附近的村落的居民有没有这些抗原,有好几种,还不只是一种。只不过这些冠病毒,跨越物种之后,传染力还是很低,它的致死率也很低,所以就没有人注意到。而且这是穷乡僻壤,它传染率又低,就没有办法传到城市去,因为这些乡下人它很少到城市里面去,所以我们现在会有这些,这些传染病其实是会加速的。为什么会加速?就是你时间往前走,人口越多,我们越侵占这些野生动物的生活空间,所以我们接触我们的人也多了,跟这些野生动物接触的机会也多了。然后最糟糕的是中国人喜欢吃野味,所以这次这个野味市场,你基本上是把病原带到城市的中心,它一旦跨越物种到了人以后就根本挡不住了,HIV当年等了 50 年才传到城市去,对不对?你还有好多像我刚刚讲的那个在 Borneo的那些冠状病毒,也不晓得是几十年几百年了,但是还是没有传到城市去。所以你政策上应该做的是就是避免去惹这些野生动物,尊重野生动物的居住区,然后绝对不能够把野味活体带到城市的中心。



王孟源 16:54 

事实上美国的法律就不容许这样子,你如果到 YouTube 上去看,他们会教你,教你到那个山林里或者是乡下去打猎,打野猪的时候,那个他们即使是用陷阱抓住野猪也必须当场一枪把它毙命,不能够用活体送上卡车的,因为他们的法律就是禁止这样。为什么野猪的带的疾病?这倒不是因为他们曾经因为猪流感而学乖了,而是因为野猪在的疾病绝对会传给家猪。不能够,哈哈哈,你绝对不能够把野猪带回。



史东 17:32 

有一个问题,如果是活牲口的话就会传,难道死牲口就不会传吗?



王孟源 17:39 

死牲口也会传,我刚才讲不是那个 Hib 就是因为你抓了一只黑猩猩,然后在那边宰杀。但是中国的话你,看看中国最近搞的一些什么猪流感,禽流感 ,SARS 跟这个 Ncov 都是呼吸道疾病,为什么中国不会去感染到其他内疾病,而是老是搞呼吸道疾病?因为呼吸道疾病有两个天然寄主,冠状病毒是蝙蝠,然后另外的那个流感病毒就是禽流感,是鸭子,这两种生物都是跟病毒已经因为几百万年下来已经和平共处,身上都有几百种病毒,但是就是它没有什么症状,所以它们就是所谓的 Reservoir, 病毒的Reservoir,。



史东 18:32 

那有储存库,储存细分库。



王孟源 18:35 

对,然后那中国很不幸的就是你蝙蝠会飞,所以的,哈哈哈,它只要到处拉屎了,你这个其他的哺乳类动物也一样会感染到,那你这时候再把它们当野味弄到野味市场,而且还是活的,他们继续在那边咳嗽、打喷嚏、呼吸,那旁边有几千个人,你有几百只野味在那边,这等于是你一下就提高了几百倍。



王孟源 19:08 

因为有几百只在那边,然后有几千个人,这样来来去去几万个,又再乘以千,再乘以万,这下子你这个被传染,它那个跨越物种的几率就提高了,然后又已经在城市中心。所以一旦跨越物种,那就不得了,就没有办法被自然隔离掉。所以这是为什么?一连两次,一连三次冠状病毒跨越物种,其中有两次发生在中国的。原因就在这里



史东 19:37 

孟源,有很多人在讨论,或者甚至在质疑,有关于这个冠状病毒是不是人类合成的,你对这个事情有什么看法?



王孟源 19:53 

你如果是做这方面专业的话,就会知道这个假想是很可笑的,因为第一个我们人类还没有这个技术,我们现在人类有的技术。是啊,你把这个其实昨天才有一个消息,就是好像是瑞士的一个实验室,他把病毒的基因复制了,大家说那,那你不是可以复制病毒,他这个所谓的病毒基因复制不是说他们真的复制了这个基因,不是设计了一个基因,而是照猫画虎。就是你已经知道了分析了这个Ncov的基因,然后你可以把这个基因放到酵母树里面去,所以它理论上可以产生产同样的蛋白,但是它不是独立的,它不是那个病毒,病毒是除了一个 RN 外包,外围还要有正确的蛋白质把它包起来的。你没办法做那个现在的技术,没有办法到做那个,他们所谓做到是就是能够把你因为现在的那个基因分析,可以把把基因很快的解释出来,就是一个个体,不是说人类的基因的图是怎么样的,而是我的基因是怎么样子,你的基因是怎么样?这个现在都是一两天就可以做出来,花费了是几千块美金,这是很快的进步。



王孟源 21:16 

那另外一个进步是 2012 年的时候,他们有了一个新的工具,叫做Crispr,就是以前的时候。以前你可以分析基因,但是你没办法基因组出来,你没办法,编辑的时候很困难,事实上你要用一些病毒去编辑基因,然后这个错误率很高、很贵、很慢。但是在 8 年之前,这个是 MIT跟 Stanford,我想或者是Berkeley 。



史东 21:45 

是不是好像跟一个诺贝尔奖有关系?好像一个。



王孟源 21:50 

他们好像还没拿到诺贝奖。



史东 21:52 

没有也是……因为这是一个划时代的,一个划时代的进步,划时代的一个进步。



王孟源 21:57 

就是他忽然把这个基因修改或基因编辑,也变成是随便哪一个大学实验室都能做。



史东 22:03 

基本上就是一种 cut and pace,就是剪贴,对不对?



王孟源 22:07 

对,就是剪贴,但是你能够剪贴,但是并不代表你懂这个基因会怎么作用,对不对?你要设计一个病毒,如果是有一个天然的基因组合,你要把它像我刚刚讲的照猫画虎是可以,就是你已经分析出Ncov是什么样子,新冠是怎么样的,你去把照样的编出一个基因序列,然后把它放在酵母树里面都可以。



王孟源 22:38 

但是你要凭空创造一个病毒,光是有能力活下去,有传染力,就已经是为远远超出我们目前的人类的那个科技能力,因为我们不懂这些,这种这些基因跟蛋白质的交互作用会怎么样?太复杂了,我们现在的知识根本不到。那另外一个问题是病毒,尤其是这种新冠状病毒,它是一种 RNA 病毒,就是连 DNA 都没有, DNA 至少还是所谓的 double helix,就是 2 串。 2 串的意思就是说它其实是有备份的,就是一个钥匙,还有另外一个备份。 RNA 是单串翻串的话就代表它是在复制的时候就会一直出错,所以它突变非常非常快,你想想看流感,HIV。



王孟源 23:30 

所以你原先设计得好好的,你如果要拿来用来当生化武器的话,第一步你自己要先有疫苗,对不对?但是你把这个生化武器做出来以后,你有了疫苗放出去以后不到几个礼拜,它已经又突变了,你的疫苗已经又没有用了。那所以你要说拿 RNA 病毒来当生化武器,这个是外行人的想法,内行人绝对不敢,对不对?而且我刚刚讲过,你光是要实验到说一个新的基因组合能够是一个有用的病毒,目前根本没有那个知识,你必须要试错。要怎么试错?找真的人来试错?你能够找几千个人吗?我上一次人类能够这样做是 731 部队,现在你现在有这种组织吗?你想想看,中国的实验室能这样做吗?美国的实验室能这样做吗?对不对?他们有这个技术能力,但是没有那个政治意识。如果现在是在打第三次世界大战,或许可以,但是没有。



史东 24:39 

不过那个孟源你刚刚讲的一件事情,我就我印象很深刻,给我的印象很深刻就是这个新冠肺炎这些这些事情,如果这个假定是正确的,如果可以用在军事上面的话,它的前提就是你一定要有解药,对不对?你一定要有解,你不然的话你连自己也被杀伤到了。然后你刚刚提到的一点我觉得很有意思的,我觉得很重要的一个观念就是说它的突非常强,它的解药,不,你即使有了解药你也不一定有用。



王孟源 25:15 

流感已经……我们在做流感的疫苗已经做了 100 年了,每年还是都常常会猜错,以后就十几万人,光美国就十几万人,你想我们要是有办法能够做出可靠的疫苗的话,不会先做流感的疫苗吗?对不对?所以这个都是外行人的乱猜的,你如果有这个专业知识的话都知道这是不可,而且事实上就是我刚刚讲的你,我们现在人类对这个基因的了解跟那些蛋白质的交互作用了,根本就还没有到能够重新设计的程度,就是我们能够照抄,但是我们没办法重新设计。



王孟源 25:54 

所以那有人就说那说不定这个新冠病毒是照抄SARS,然后修改一下,但是它最关键的是人传人的时候,他们SARS刚好跟这个新冠都是用我们支气管,就是很靠近肺泡支气管的那个细胞上,那个细胞表面有一个蛋白质,有一个抗原叫做 ACE2,他都是靠套上这个  ACE2,它们就是病毒有它的那个蛋白质,这叫做 spike glycoprotein。它也是一种抗原,就是套在外面,它的它是冠状病毒,就是因为它有这些蛋白质,弄出来像个皇冠一样。



王孟源 26:44 

OK,那它这个这些它的这些 spike glycoprotein 就套上这个 ACE2,那接触以后就连续反映这个 ACE2,就说好,我认识你,你可以进来。 ACE2就像是那个看门的人专门的一样。那照理说 SARS 跟那个新冠都是用同样的一个看门人来渗透到同样的支气管细胞,所以他们症状有点像。



王孟源 27:11 

但是那个后来这个是大概一个多礼拜前,在 10 天前,他们把这个两个的基因组合仔细一看,发现他们所用的那个spike glycoprotein 是不一样的,他们这个是所谓的趋同演化,就是说原本的是原本的这个,他们的这个蛋白质都是针对蝙蝠或者是其他的动物来设立的。然后他们都必须要有突变,然后变了形状,才能够跟人类的ACE2接触,才能够这时候才获得人传人的能力。这就是所谓的我刚刚讲了半天的跨越物种。



王孟源 27:56 

但是 SARS 的那个人传人的ACE2跟新冠的人传人ACE2是不一样的,他们发现的不一样的,那这个就远远超过人类实验室所能做的。因为我说过我们只能够照猫画虎这种设计一个新的方式,新的蛋白质能够做同样的功能,而且事实上是更好。你可以,事实上这个新冠这次会这么危险,就是因为它的传染能力,人传染能力比SARS要高很多。事实上我刚刚讲过是 100 年来是从 Spanish Flu 来最强的,百年一见的,所以它这个演化是全新的演化,是所谓的趋同演化。那你这样子自然就是不像是人工合成的,人工没有那个能力设计。



史东 28:43 

一个人工只有一样画葫芦的能力,没有创造的能力。对。



王孟源 28:49 

所以,而且我其实我个人不太喜欢这些阴谋论。定义阴谋论的话,你必须要提出正面证据,不能够光是拆一个可能性。你没有正面证据的话,这种所谓的可能性是有无限多的。那你有无限多的可能性,你光挑一两个出来讨论是浪费大家的时间。



王孟源 29:12 

在逻辑医学上,这是一个很常见的谬误,也是被常常批评的谬误。 Bertron Russell 就是 100 年前 20 世纪的一个很有名的哲学家,逻辑学家,他还特别举了一个例子叫做 Russos Teapot,我觉得我希望能够把它传播到给中文的知识界,所以我这边也特别提一下他举个例子说——他这个时候其实是在针对宗教了,因为这个对最典型的这种没有正面证据的信仰就是宗教——所以他就举了一个指说如果我性质来了,说在火星轨道上附近漂流的一个小小的精致的英国茶壶,你有没有责任去找?反正送一艘太空船去看,有没有这个责任?不是人家反对的人说不可能有,然后你不能说,诶,你要你没有证据说他没有,你要送个太空船去看,你不能够这样讲;是提出这个可能性的人的责任,要提出正面的证据。你没有正面的证据,像这种凭空拿出来的可能性有无限多个嘛,对不对?你无限多个选一个。



王孟源 30:27 

我们大家都知道 1 除以无限多是等于0,就是它的实际意义等于0,那你这个逻辑上的意义是 0 的就没有讨论的价值。所以我这一次看到这个阴谋论问第一个反应点,就是哪里有正面证据?什么都没有正面证据,而且有很多的反面证据。那这那你这时候再讨论的话,而且他有实际的恶果,就是我刚刚提到正确的改革,就是预防的话,你就是必须要禁止运送野味,对野生动物不能够在活体运输,不能够在活体集中,不能够在活体贩卖,尤其不能够在人口多的地方对密集的地方做。那你要做这种改革的话,当然可以从上到下就说我法令就是这样,那这东西我个人觉得徒法不能以自行,你必须要教育民众,让人了解这个法律是有它的道理的,大家要心悦诚服,这个法律执行起来才容易,才能够长治久安。对,一大堆人在那边偷偷摸摸的那个走私……



史东 31:40 

那我想经过这次的经验,应该是一个很好的教育。



王孟源 31:44 

对,所以我觉得这些阴谋论是很糟糕的,因为你不承认这一次这个疾病的真正来源的话,你就没办法让整个社会了解这些真正的危险所在,然后有正确的反应。你光是立法这一次,我想那个中共和政权是已经学乖了他们,现在我从目前他们的宣布来看,我可以确定他们一定会严格立法全尽。但是如果你不能够说服全民都了解的话,你也知道中国人的习性。



史东 32:18 

这个叫上有政策下有……



王孟源 32:20 

对策,我正是要说这个,哈哈哈,所以我必须要在这里强调,这真的就是因为蝙蝠所以病毒原本那些spike glycoprotein 是针对蝙蝠的细胞来做的,来传染的,然后它先传给其他的哺虏动物可能是猪,那然后下一步再接更接近就可以跳到人。每次突变都是 1/ 1000000 的机会,但是你如果让它有几百只动物放在那边,有几千几万个人天天经过的话,你等于是在帮助那个命运女神要给你厄运。



王孟源 33:01 

好,中国本身我刚刚已经讲过,就是蝙蝠本身是一个很大的问题,你这次在中东那个MERS,事实上 Ebola 目前也怀疑是从蝙蝠来的,从非洲的蝙蝠来的,所以蝙蝠是一个很大问题,你要解决蝙蝠的问题就要禁止野生动物的市场。第二个,另外中国还有一些弱点,就是我刚刚讲过另一个所谓的病毒的Reservoir 蓄水库就是鸭子。那中国当然是世界上最大的养鸭国家,而且中国的养鸭不是像美国这种工业化的,这样不太好,因为你会打很多抗生素什么的,这用很多药;但是这种小农式的放任就代表这些鸭子会跟野鸟接触,这些野鸟是飞来飞去的,所以你即使原本这一池塘的鸭子没有什么流感,它也会因为跟野鸟的接触而得到流感。所以中国也是特别容易得到禽流感的,因为你那些养鸭人家也必须要跟鸭子有密切接触。



王孟源 34:15 

那另外一个常常会跨越物种的来源是猪,因为猪的那个生理其实跟人很接近,他这个跟人的血缘是很接近的。除了灵长类动物之外,猪算是最接近人类的一种哺乳动物了。那所以很多这种虽然猪不是原始寄主,比如说原始寄主是蝙蝠,它可能就经过猪而跳到人的身上,或者甚至是从鸭跳到猪,先跳到猪,然后从猪再跳到人,所以我们有所谓的猪流感。那当然中国也是世界上最大的养猪国家。



史东 34:51 

我们讲这里,我们讲到这里,我想一想把话题带进一个,疫情防治的这个角度,哼,这也是你想谈的,你觉得因为照你刚刚这样的形容几乎是无解嘛?你不能说不吃鸭子不吃猪嘛?你不能,你的养殖方式对不对?如果你要自然怎么讲这个 Free range 要它自然养殖的话,你也没办法控制。你刚刚讲了它要跟其他的牲禽之间的接触等等等等的,在这个前提之下,我想不出有任何一解决的方式。



王孟源 35:30 

其实我刚刚讲在 HIV从 1980 年代传到欧洲跟北美之后,美国的、欧美的有识之士都早就已经预料到这种情况。我刚刚讲的这些情形就是就现代工业化的国家越多,人口继续增长,这种跨越物种的病原体就会越来越多,我们这一次只不过是撞到一个传染力特别强的新冠,没有什么稀奇的,像这种跨越物种的病毒每年都有,甚至每个月都有。



王孟源 36:08 

我们没有注意到是因为他们的传染力很低,或者致死率很低,或者是被局限在穷乡僻壤。但是现在越来越没有穷乡僻壤了,因为你光看那个航空业的那个载人的人数,就是指数成长,过去 50 年基本上是指数成长,然后乡下也不断地被开发,那个亚马逊森林每年都被烧掉几万亩、几百万。其实对像这种侵占野生动物的生活空间,这些因素加起来就代表我们看到这种新的传染病,新的严重的传染病会越来越快。



王孟源 36:52 

那这种事情我刚刚讲过,就是三四十年前欧美的有识之士都已经预见了,所以你看美国的 CDC 现在这个名誉很好,为什么?因为他们是 70 年代、 80 年代就已经开始准备这些事情,所以我那时候人家问我说这个中国的政策应该怎么做?我说第一个,你就先禁掉野生市场,因为这是中国特有的很危险的东西。



王孟源 37:19 

但是除了这个以外,你必须要模仿美国的CDC,建立一个很高效、很专业的组织,而且权力很大。事实上权力还不够,因为他们那次那个钻石公主号在日本被日本人胡搞了一通,结果凭空闹出来五六百个感染,其中有 40 个是美国人,他们要他们送那个专机,运输机要专门运回去的时候, CDC 是不想让他们上。



王孟源 37:50 

CDC 说这些已经确认就是检测出来已经有病毒的被传染病,应该就留在日本,然后可以好好地隔离,有专门的医院可以隔离,但是那个是国务院 American first,所以必须要把他们统统带回来,所以他硬是带回去了。硬是带回去以后,那你就代表除了那 40 个人之外,其他同级的人都必须要再观察 14 天了。不过至少 CDC 在那个这这种隔离他们有设立隔离区的效率很高,所以我相信这些人会被妥善地处理。隔离 14 天,他们只是不想隔离两百两三百个人,我想。结果国务院硬是把他们压下去。



王孟源 38:35 

所以我刚刚要讲的就是你必须要建立一个专业的管道,这个管道是干什么呢?就是第一个是有可疑的情形的时候,要能够很迅速地上报,这个上报必须是独立的专业的通道,而不是通过一般的行政管道。因为行政首长他们要关心的是主要是经济对不对,你要是莫名其妙风吹草动就把整个城封起来,这是不可能的。尤其是现在中国的经济原本就有点衰弱,刚刚从美国的中美贸易战有点喘息,你一下封城。武汉也是一个很重要的经济中心,比如说它有有一个DRAM,就是储存的芯片的主要制造中心,就在武汉那,然后也有汽车生产在那里。那你除非确定说这是一个新的传染病,还要确定他可以人传人,还要确定他人传染的能力很强,你过光这三点就至少要三四个礼拜,你看看他这一次武汉的处理,的确就是等了三四个礼拜。他等的是什么?等的就是第一个要确认有新病毒,第二个这个新病毒是造成这些肺炎的原因。



王孟源 39:54 

第三个这些肺炎是会人传人的,第四个这个人传人的传染力比 SARS 要只高不低,否则的话你不必太担心嘛,对不对?你可以把它好好地隔离起来,你光要等确定这 4 点就是三四个礼拜,刚好就是实际上三四个礼拜,那所以你要有好好的准备。第一个就是要避免它一开始就传到大城市,你最好是能够先在乡下跟野生动物接触,或者是农村里面先开始,农村是最容易隔离的,对不对?你一个,他们自己本来种田的,有那个,你把那个路封了就封了,反正很多地方也是过去 10 年才有公路。然后你要好好地上报的时候,就是医院里面如果看到不对劲,能够有专门的通道上报上去以后行政组织跟这些专业医疗组织都必须要有预案,就是连地方上都有专门的组织。



王孟源 40:52 

然后说如果是呼吸性的传染病你要怎么处理?如果是接触性的传染病要怎么处理?如果是鼠疫之类跳蚤传染的怎么处理?蚊子传染的要怎么处理?OK,你如果是所谓的 norovirus,这是一种有点像霍乱之类的东西,那你这个要怎么处理?这些要有预案,而且要定期演习,然后最后是 CDC 在中心要有一系列处置的权利,就是 CDC 本身必须要有权利宣布隔离,不是武汉市长来宣布说我要隔离,这权利应该是在 CDC 手里。



王孟源 41:34 

这个其实亡羊补牢我相信中共会做,但是美国人他们是因为 80 年代经过了HIV,所以他们已经做过了。那在 80年代之前,在 70 年代 Ebola 出来的时候,他们也学了一过一次,然后前几年在 Ebola 又重新在非洲、西非泛滥了一次,他们这个又是重新演练了一次。



王孟源 41:56 

我必须要跟大家讲一下这个目前对病毒来说有一些抑制药,但这事实上除了疫苗之外没有很好的办法,没有所谓的特效药,我们感觉上说OK,找到一种特效药,然后药到病除,这个是对细菌才有这样的,就是抗生素才有这样的。病毒本身是非常非常难缠的,它的能够真正能够对抗病毒的只有一个,有可靠有效对抗病的就是人体的免疫系统。



王孟源 42:33 

所以你说现在这个新冠病毒开始传染肺炎,你说送到急诊室他们在搞什么?他们都只是让你续命,只是让你不要死得太快,然后等着让你的免疫系统来好的工作,做工作。对,治好你自己。就是因为大部分这些新冠肺炎致死,是所谓的ARDS,就是Acute respiratory distress syndrome,这种就是呼吸器性肺炎,急性肺炎。



史东 43:07 

呼吸器官衰竭。



王孟源 43:07 

对,就是那个肺泡整个都发炎,肺泡发炎以后他就没办法整个都泡在那个体液里面,就没办法做空气交换,你就没办法没有呼吸你就死掉了,对不对?那所以现在处理这些肺炎其实都是流感,也一样,流感也是搞到 ARDS 以后就死亡率。



史东 43:29 

就是基本上也是你刚刚讲的,因为它突变得这个速度非常快,所以说没办法。



王孟源 43:35 

突变速度很快,所以每一年都有新的变种。然后我们流感其实一开始大概就是像这一次新冠一样,大概几千年前或几万年前,就是像这情况下,但是因为跟人类和平共处了几千年以后的演化的趋势是让这些病毒它的传染力越来越强,因为传染力强的自然就存活下来,对不对?我们说演化的原则是适者生存,所以一代一代他们的传染力会越强,但是他的致死率会越来越低,为什么?他如果把你把寄主宰了以后,那寄主身上的那一种,那个病毒的变种也跟得都死光了?所以事实上我们看到现在所谓的感冒流感,他们原本都是几千几万年前跨越物种。



史东 44:27 

这是不是代表了人的抗疫力也增加了。



王孟源 44:31 

我们的免疫,我们对这几个的免疫力也增加了,是一种共同的演化,寄主跟病毒,对的,共同演化。所以现在我们在讲说新冠病毒,如果我们没办法把它根除,就是如果新冠病毒愿意遵守海关,只留在中国,我相信中国是有能力把它根除的,大概再过一个月可能就把它根除掉了,但问题是现在已经传到日本、韩国跟中东了,所以有可能就已经是马放南山,已经没办法的。It’s too late to close the Barn door。



王孟源 45:07 

那如果是这样子的话,很可能就是它演化几年之后就变成另外一种流感,就是不是流感病毒。但是,从传染病学来看就类似,流感就是它的传染力,现在已经很接近流感了的,可能就达到流感的程度,但是它的致死率会往下降,降到流感的程度。流感的程度是什么呢?这是 1/ 1000 或者是 1/5000 这个程度。现在这个新冠肺炎是大概 2\% 1\% 左右。



史东 45:37 

那换句话说就是它会达到一个和人类和平共处的地步。



王孟源 45:44 

恐怖平衡,恐怖平衡。



史东 45:46 

所以这个话我说的不是很好,就恐怖平衡。



王孟源 45:48 

恐怖平衡。对,嗯,所以流感就是已经跟人类达成恐怖平衡的一种大流行传染病。那新冠病毒如果没有被灭绝,没有像 SARS 那样被灭绝的话,就会变成另一种流感,所以这是一个可能,那如果是走这条路的话,经济的影响就非常的大了,我待会再谈,不过我想再谈一些比较在针对我刚刚讲的一些东西,在讲一些技术细节。为什么我会认为说这次中国的处理已经是达到人力可能的极限?我刚刚讲到这个新冠病毒是百年来第一次有一个呼吸道病毒,一跨越物种就达到流感级别的传染力,能传染的传染力,OK,这个是非常百年一见,但是还好它的致死率不是太高,不过这个传染力跟致死率都是会因为人的免疫能力跟那个医疗系统的准备程度而变的,就是我们现在看流感的传染程度还可以接受,其实也是因为我们每年都打疫苗,然后那个医院都已经准备好了,季节到了就有病传控出来。然后流感的致死率会低于 1/ 1000,有很大原因也是因为疫苗。还有那个ICU,就是 intensive care unit,急诊病房已经知道看到流感,然后开始有肺炎了,马上就送到精准病房,然后把呼吸器放上去。



王孟源 47:23 

这个你看这次的这个统计资料,就是目前有两个月的统计,只要一个很奇怪的现象,就是湖北,尤其是武汉,他它的死亡率特别高,你如果只看湖北武汉的话,它的死亡率其实是 4\% 5\%,你是全国平均以后才到2\%,那你如果是看全世界的话甚至是可能会低于 2\% 的。



王孟源 47:50 

那为什么会这样?我想是因为在武汉集中爆发这个病毒有三四个礼拜是没有被隔离,没有被正确地隔离认知,所以等到全面动员要处理的时候,案例其实已经上万了。那这个时候,因为,因为根本就没有那么多急诊室,你根本就没有那么多呼吸器,所以就没有办法挽救那些肺炎。照理说这些肺炎你如果送到急诊室,马上放上呼吸器,有 2/ 3 会存活,但是你如果没有放上呼吸器的话,这几乎百分之百都会,。



史东 48:29 

这跟它的潜伏期很长有没有关系?



王孟源 48:30 

我待会讲,哈哈哈。然后另外一个,另外一个因素是因为正因为他已经散布出去了,然后又要等几个礼拜才有可靠的检测,检测器,检测器材,所以武汉的因为感染太多,你不可能每一个都去确诊。然后又因为这个新冠还有一个很讨厌的很危险的,就是它比流感还要糟糕的一个特性,就是它的这个潜伏期是很不定的,目前是已经确定从 2 天到 14 天都有,而且是还没有症状的时候就有传染的传染力,这个是非常糟糕,比流感还要糟糕。然后现在他们甚至还在讨论几个特别的案例,就是有一个,比如说有河北省还是哪一个省有一个女孩子,她感染之后到现在还没有症状,就是她是所谓的无症状患者,但是他全家人都因为他而受感人得了新冠肺炎。OK,那你看这有多危险?他这个你如果一直都没有症状,就是一个月、两个月之后都没有症状态,一直都有传染力。那你这个,你事实上这个潜伏期就是超过 14 天,你就是潜伏期基本上就是无限,一直到你对你自己免疫为止。对,那所以你说你刚刚我一直说这个新冠病毒其实传染力很接近流感,所以它是百年一见,事实上它比流感还有更糟糕的就是它的潜伏期是非常的不确定,而且它没有症状的时候也能够也有很高的传染力,这个比流感还要糟糕。所以你说这一次因为一开始三四个礼拜没有很快的反应,导致感染了几万个人,然后一直,然后又花了一个月才控制下来。



王孟源 50:36 

老实说主要的原因是因为这个病毒实在是百年一见非常难缠的一个对手,你如果不是遇到中国的话,其他的国家老早就束手就缚了。你看 2009 年我刚刚提过,那个时候那个猪流感,美国根本就没有试图把它完全压下来,它事后就变成在全世界流传的另外一个流感。所以我觉得这些很多事后诸葛亮,尤其是中国自己的网民在这边批评说他们的处理效率不够,就好像很你像说你现在在读那个拿破仑的战争史,然后在被批评拿破仑的战争智商不够。



王孟源 51:18 

所以我说中国有错,它错在什么呢?事先没有预防,尤其是你有 SARS 的经验,结果没有。把野生市场禁掉,这个是非常不对的。然后你经过 SARS 的经验,没有建立一个强力的CDC,这也是错的,对不对?你美国人经过一个 HIV 他们就学乖了, AIDS 他一次就学乖了,中国要两次才学乖,这个是你可以说他有错,但是我个人认为说你这次遇到一个这么难缠的对手,事后的处理能做到这样子已经是独步天下了,你有什么再苛求的,你可以说他预防没有做好,但是事后的处理你没有什么好说的。



史东 51:59 

你刚刚提到中国CDC,你也提到了说这个中国的CDC不但要另存一个体系,这个通报体系,而且这个CDC必须有非常大的权力来执行他要做的事情,所以这两个重要的元素把它放在一起的话,以中国这么大的幅员,这是一个相当庞大以及相当有锂的一个机构不是吗?



王孟源 52:24 

不是,就是这个机构。如果你出现了这种疫情的话,有,或许是有国家主席或者结合适当的领导人授权,然后他就可以开始调动军队,调动省级的一切机关。这个我觉得这是是合适的。



史东 52:48 

基本上你的要求如果我没听说你的要求是希望能够把它的机动性极大化。



王孟源 52:54 

对,就是在美国的机制是这样的,它上报以后跟总统,总统报告,然后总统宣布紧急状态的话,这个时候 CDC 可以调动任何东西了,它基本上你旁边有什么国家警卫队,它要调动国家警卫队就调动警卫队了,然后可以封路了。



史东 53:12 

应该是这样子才可以。



王孟源 53:13 

对,这个不必等到这个省委书记也理解。



史东 53:19 

层层上报或者层层下迁,对不对?



王孟源 53:22 

不必等到中央派一个新的省委书记来,然后给他授权了。就是CDC,你直接说,有紧急状态,由 CDC 处理,然后由专业人员来做,当场做决定,这是因为时间很重要。



史东 53:36 

我想这是一个经过这么一次,我可以说伟大的经验,我觉得中国官方应该是很容易下这个决定来做到这一步的。



王孟源 53:51 

这个有关野生动物市场,我相信是绝对要禁的。但是我觉得舆论上你也必须要这个。



史东 53:58 

要经过一番一些教育的过程。



王孟源 54:00 

对,教育的过程,要不然大家阳奉阴违没有什么意思。对,然后这个体系你光是给他权力不够,他必须要常常演习,然后有完整的预案才行,对不对?你每年必须要找一个省来做完整的演习,因为事实上我们现在看到是不到十年就有一次,SARS是 2003 年,MERS是 2012 年,现在这个是2019/2020,不到十年就有一次。而且你这个全世界中国已经是一个现代化的国家了,那个每天飞进飞出的那个,对,几十万人、几百万人你没办法说,保证说能够把它堵在国门外。那如果是从国内养猪养鸭弄出来的,那你更是必须要当机立断地好好地隔离处理。我觉得禽流感跟猪流感你如果有好的体系的话,是有可能能够挡在乡下的。



王孟源 54:58 

事实上中国现在有经济衰退的问题,然后又有整个2月都是停工,这个对经济的影响很大,它一个很好的弥补方法就是你投资在医疗体系上面,因为事实上过去的 20 年来他们基建做了很多,但是主要是在交通,还有那个交通还有能源上面他没有做到。医疗上面真的投资过低,正常的医疗花费在一个现代化国家,应该是大概 9\% 10\% 左右,美国是18\%,那是太浪费了,因为他们那个保险系统太可笑了,叠床架屋。但是正常是应该9\%,我想中国好像是 4\% 5\% 左右,就是他可以可以加倍投资,那加倍投资以后你有了高素质,然后就是第一个,你就是先给医生跟护士加薪,给他们一个比较好的薪水,因为他们事实上是工作非常辛苦而且又危险了。然后你扩大规模,那增加像急诊室的呼吸器这种东西,你如果数目多一点,就不会有一开始这种手忙脚乱、忙不过来的情形。然后最好是能够普及。



王孟源 56:21 

现在我对中国的内地的情况不是很熟,但是感觉上好像你有大病还是得要到省会的医院才比较可靠,这样是不太好的。你这最好是每个县每个镇都有一个可靠的医院,能够处理流行病。那这样另外一个好处就是这些医院能够在当地就看到新的传染病的苗头,可以更早上报的时候他还局促,在乡下就就很容易隔离,因为这也是跟时间赛跑的事情。对,要对,你要跟把病毒隔离起来,是越早越好。



史东 57:05 

谈到这个事情,以我们可以换一个话题吗?还是你讲经济?讲经济,我眼前有两个新闻,一个是黄奇帆先生,他是我讲大家都知道,是清华产业转型顾问委员会的主席,他也是中国国际经济交流中心的副理事长。他说了一句话,他说这次疫情已经造成了几百万亿的经济损失。他说中国公共卫生系统要花将金旅两三千亿来补这个短块,就补这个缺口,oh,是几百万,说几万亿,不是几百万亿,他说几万亿的经济损失。



史东 57:54 

另外这个我们今天做节目,是我们这边美西事件是2月 24 号,就是李克豹新闻网,他说全球股市重挫, Dow Jones 狂跌 101031 点,德国它的股市也跌了将近 4 个percent。它继续讲,就是它说 Dojon 是狂蟹 1031 点,标准普尔指数狂落 111. 86 点, Nasdaq 狂落 355. 31,然后伦敦的FTSE,它的指数大跌247。 Frankfurt 的 DAX,它的指数重挫 554. 09。巴黎的CAC,它的指数跌了 237. 85 点。全世界没有一个地方几乎是可以逃过这种现象。



王孟源 58:53 

全世界的金融跟经济都已经连在一起了,这个都是全球性的。我先说一句话,就是黄奇帆先生是我非常佩服的一个人,他特别的地方在于他不是学术界的人,他有他的经验跟知识跟看法,都是实战出来的,这非常务实。那我个人也是讲究务实的,所以非常佩服他这个能够在了解这实践上的许多细节,然后同时能够达到一个很高的高度来看大局的,这是很了不起的。



王孟源 59:34 

那我完全同意,像我刚刚讲的中国的那个,我让中国的 GDP 一年大概是 100 万亿人民币那个数量级,那我刚刚说他这个需要再多花 4\% 5\% 在医疗系统,那么你算一算,这就是大概四五万亿嘛,对不对?嗯,就是你这一次因为停工的损失也差不多也是那个数量级,对不对?你还不如把它花在医疗系统上。你本身也是创造GDP,因为这个东西是有回报的,然后人民生命(史东:有保障了)素质缓缓往上涨,所以这亡羊补牢了,应该是投资在这个医疗系统。



王孟源 01:00:23 

就是我觉得中国应该至少花费大概 9\% 10\% 的 GDP 在医疗体系上面,这是不但是你的生活水准,人民生活水准到某一个程度有足够的中产阶级以后应该关注的项目,而且你还必须要考虑到中国会常常有这种传染病疫情,新的病原体跨越物种,那你这时候如果有一个完整体系一直到乡下去的话的这个差别就非常的大,这基本上是能争取时间。最重要的因素就是你把专业的医疗体系一直升到每一个村庄、每一个镇上,这乡镇上。



王孟源 01:01:12 

至于这个股市狂跌,这其实是因为伊朗跟意大利的新闻出来,所以原本金融跟经济市场觉得大家都很佩服中国的行政能力,认为中国能够把这个病坑压下。嗯,他挡得住,他这下一砍,中国挡住没有用,他已经传出去了,我觉得他们还也会观望,因为他大家也知道他们也会去找专业的人士来评论。



王孟源 01:01:44 

那专业人士也会跟你讲流感季节已经差不多快完了,流感季节就是 11 月到2月。那你那个新冠肺炎传染的方式跟流感很相似,所以理论上它应该也是遵循流感季节,就是天气暖了,湿湿暖了之后,我们抵抗它传染的能力,人类抵抗它能传染能力就会上来,这时候它的那个人传染的能力就下去了,或许会帮助这些其他的国家挡住这个病毒的扩散。但是当然我们 we hope for the best, but prepare for the worst。对对,所以还是有那个可能说他会就会出传染出去。那我这个疫苗大概还需要半年左右,最快大概还要半年左右才能够出来。那下一个冬天不是什么太大的问题,但是这一个春天,然后接下来南半球可能要稍微吃点亏了。



王孟源 01:02:42 

对,也不算是失控,但是可能是一个中度的 pandemic 没有像,当然没有像西班牙流感那么厉害。西班流牙流感可怕的地方在于它的致死率是 10\% 20\%,那里跟 SaaS 同一个级别的,然后它的传染力比 SaaS 高一个级别不止,所以他才会目前的估计是 5000 万到1亿人死亡,那当时世界地球的人口只有十几亿,就是基本上死了 5\% 到 10\% 左右。



史东 01:03:22 

对他,所以听说他死亡的人数超过第一次和第二次世界大战死亡人数的总和。



王孟源 01:03:31 

对,它其实,目前的猜测认为它是一个猪流感,也不是现在才有人才有兴趣。过去几年,过去十几年其实已经是一个热门的学问,就是自从 SARS 出来以后,又提醒大家说这种流行性传染病,新的流行性传染病很可能会随时出现。然后MERS, 2012 年MERS,所以这个已经有很多人去挖坟,就是他们到那个冻土加拿大的冻土去挖那些当初死亡以后的那个 mass grave 大坟。然后希望能够找出那个病毒,他们已经,基因已经都解出来了。嗯,然后所以他们知道这个可能是一个猪流感,可能是从美国中西部的养猪场传出来的,因为那时候美国养猪还是跟现在中国的小农养猪一样,不是工厂式养猪。然后因为当时美国还在刚刚加入第一次的大战,所以他先传到几个兵营,马上就把这个兵营的那个阿兵哥……



史东 01:04:35 

歼灭。



王孟源 01:04:37 

那个大批歼灭,但是那时候军令如山,你那个大战打得如火如荼的,所以那些人还是被硬塞到那个运输船上,送到欧洲去。然后到了欧洲以后,就很快的就英国人也开始大批死亡,大国人也大批死亡,德国人也大批死亡,但是大家都不知道为什么,然后都新闻封锁,因为那时候西方世界一打仗就新闻封锁,所以一直传到西班牙。那时候是中立国家,西班牙是第一个公开讨论,说我们有一个很奇怪致死率很高的流行病,所以后来就叫做西班牙流感,其实应该叫美国流感才对,现在大概不会到那么严重的地步,但是我们现在必须要讨论的是它的经济后果。第一个中国的经济后果就是中国2月基本上停工 SARS 的时候,他们有一个月,这个有一个季,它的 GDP 成长率掉了2\%,就是从 10\% 掉到8\%。我想这一次因为感染传染力比较高,感染人更多,而且它必须要封锁的地区更多,所以至少那是一个底线,至少会掉2\%。那现在这个中国 GDP 成长率大概是年成长率 6\% 点多这个速度,所以 GDP 可能会掉到 4\% 左右,或者甚至更低,这都有可能。



王孟源 01:06:14 

但是这不是问题,这中国可以承受这个,那世界也可以承受这个,就是中国的进口跟出口对世界都会有影响。那因为现在我们已经是全球化这么多年了,二十,三十年了,那个全球化程度很高,所以但是一个月并不是太离谱,一个月是可以恢复的,现在的问题是如果你在欧洲或者美洲也造成恐慌的话,那这个这个经济的对经济的打击就非常的危险了。所以原本全球的股市认为说如果是局限在中国,只停工了一个月还可以承受,但是如果是传到欧洲或者是欧洲、非洲或者是美洲这些国家,包括美国在内,大概都没有能力把它全部挡下来,那么这时候对全球的经济影响就会非常非常的大。



史东 01:07:21 

会造成一个什么样子的情况?大萧条吗?



王孟源 01:07:26 

我刚刚讲说个时机是很糟糕的,因为事实上去年的后半年就应该是全球的一个萧条的开始阶段,因为中美打了贸易战,然后中国原本就已经经济周期到了,然后欧洲也是,欧洲,法国因为社会动乱一直一直有街头示威,已经一年多了。德国它的那个汽车出口一直下降的,主要是因为电动车兴起,而它德国的车系在电动车上面是落后的,所以德国是全世界对出口依赖最高的,远远最高的的大型工业国家,它的这个顺差达到 GDP 的 6\% 7\% 左右,中国是 2\% 点多,根本不算什么,所以他们这个出口最大宗的就是机械跟汽车,尤其是汽车,所以它汽车一旦是停下来以后,它的经济就有问题,所以德国其实在去年后半年已经是在衰退的边缘,就是成长 0. 1\% 0. 2\%,那英国在脱欧,我们已经讨论过了,这个英国会死得很惨。日本去年的最后一个季度,年率化就是 annualize 的速度,是降低了 6. 3\%,衰退了 6. 3\%,它原本就很衰弱,然后它又把那个消费税 8\% 提到10\%。



王孟源 01:08:59 

全世界有三大三小,六个经济板块,三大是中美欧,三小是后进国家、资源国家跟日本,那日本已经完了。资源国家我刚刚讲过,现在因为欧洲跟中国的经济都趋缓,所以现在价钱正在崩盘,崩得最快的是天然气。我想你如果注意商业新闻的话,你会注意到很多中型的天然气厂商在美国正在破产的过程中。



王孟源 01:09:33 

我刚刚讲过中国跟欧洲的经济都很衰弱,其实美国也很衰弱。我去年跟你上节目也说过,我预期说很快会有经济萧条,结果是美联储出来放水了 5000 亿美元。 5000 亿美元是什么意思?是他们年度 GDP 的 2. 5\%,那他是一个季就放出来,第四季放出来,那他占一个季产出,总产出,全国总产出了10\%,那你说这种程度的放水,如果经济不马上复苏的话反而是奇怪了。



王孟源 01:10:11 

如果凭空忽然多出了 10\% 的财富,对不对?问题是现在全世界都在萧条,只剩下美国,你连那个后进国家都不是一致的。就是你看印度,因为我讲过它的金融系统出了问题,过去两年出了问题,所以印度也经济也不行。现在目前经济还可以的是把在中美贸易战捡了便宜的。



王孟源 01:10:36 

像越南跟印尼,他们因为啊订单被转到那边去了,但是他们的经济规模太小了,对世界基本上没有什么什么影响,所以现在基本上就是美国一个撑着,那这个撑着美国原本就是虚胖,原本就是在继续把泡沫往上吹。基本上美国现在的大银行跟大金主都已经是作壁上观了,他们觉得太危险了。我半年前跟你讨论那个 Repo 回购市场的问题的时候,我跟你说这个他之所以会当然有很多原因,但是实际上的导火线是因为这些大银行都认为很危险的,他们不愿意在再把资金拿出去冒险。还是我们上一次聊天的时候我举的那个例子,你知道它一定会雪崩,但是你不晓得是哪一片雪花把它引起了。现在世界经济基本上就是美国一家撑着,而美国撑着的原因是因为它有美元霸权,所以可以美联储可以放水继续买时间,但是能买到什么时候很难说。



史东 01:11:46 

今天旁边的三个角度,从医学治疗,从疫情的防治,还有从这个经济方面,疫影响能不能够给我们做个结论。对,今天你谈的话,我们今天所谓的 take away 应该是什么?



王孟源 01:12:02 

take away is,这是一个教训, it could have been a lot worse。当然我们说这次比 SARS 还要糟糕,还要严重,就是这个病毒比 SARS 还要厉害很多。



史东 01:12:15 

难缠,更难缠



王孟源 01:12:17 

对,传染力高,而且它的 incubation period 很不确定,然后在没有症状的时候也能够传染。这个都不是 SARS能够比的,然后但是它的致死率不是很高,然后,中国又为世界争取了一个多月的时间,对不对?所以现在已经到2月底了,流感季节已经快完了应该。第一点你必须要了解到这是一个自然界对人类的一个pushback,就是我们人类对地球过度开发的一个自然后果,我们的……



史东 01:12:58 

一种反扑,是吧?



王孟源 01:13:00 

一种反扑,对,是,所以我们必须要了解到这种新的传染病是现代社会的一个必须要认知,而且准备好的一个随时会可能发生的一个危机。然后我刚刚已经提过有好几个政策方向,这个主要是预防了,就是建立好的医疗体系后,避免那个野生动物到城市中心来。然后所以这是第一点。



王孟源 01:13:30 

第二点是大家不要恐慌,其实治疗病毒除了有疫苗之外,最好的方法就是靠你自己的免疫力,那你增加免疫力不是去吃什么中药,这些中药都是骗人的。因为你必须要做完双盲实验才能够确定它有效,那根本没有时间做双盲实验。我不管你是中医、西医,那你既然还没有做出双盲实验,就是说,我不知道我没有证据说它没有效,但是我有证据说你不可能有证据说它有效,那你如果你如果没有证据说他有效,你还在这边说他有效的话,那你就是在撒谎嘛。那撒谎的人你根本就不应该听他的。



王孟源 01:14:13 

OK,那所以有病毒感染,有流感是这样子,新冠也是这样子,你就是最好的办法就是睡觉。因为一个人能够增强自己免疫力,最有效的方法不是去吃什么中药、西药,而是多睡觉,确定你能够睡足 8 个小时? 9 个小时睡眠充足是对免疫力补充的最有效的方法。所以事实上你说送到 ICU 除了那个呼吸器以外,他们对这个重症病人最大的贡献就是镇静剂,他们也重症病人好好地休息,对不对?所以我希望大家不要恐慌,不要浪费钱,让那些骗子赚你的钱,然后反而去危害自然界的生物。另外就是这次有一些实验,这个还没有确定,不过我认为是很可能是对的。我刚刚提到这个 SARS跟新冠都是靠支气管层细胞上面的一个蛋白质,叫做 ACE2。那上个月有一个研究说你抽烟越久,你的支气管细胞上面的 ACE2 就越多。



史东 01:15:32 

就是说你接受这个的新冠的可能性就越大,是这意思吗?



王孟源 01:15:39 

你这个新冠感染到支气管的几率就大。感染到支气管是什么意思?就是变成肺炎了,不是只有轻的症状,而是重的症状。所以你现在如果看中国的话,它感染的人数男人女人几乎是一半,但是死亡人数是 2: 1。



史东 01:16:00 

男人抽烟抽得多是这个意思。



王孟源 01:16:02 

因为对中国的我查了一下,中国的成年男性抽烟的比率是大概一半,那个女性是低于5\%,这就是它的差别所在。倒不是说这是病毒有性别歧视,而是因为中国的男性他的抽烟的比率要高很多,所以他就因此而重症的比例比较多。就是你同样感染,但是因为你是老烟腔,所以你就比较容易感染肺炎,所以我相信这为了大家自己的健康,为了公共卫生,能戒烟就戒烟。



史东 01:16:40 

对,因为戒烟还有其他的好处。当然我们自己不抽烟,所以说我们讲起来比较容易了,如果抽烟,哈哈哈。



王孟源 01:16:48 

我觉得抽烟主要还是个人自己在选择,因为后果是你自己承受,所以我觉得……



史东 01:16:56 

这样,理智上这样讲,我是不是马克吐温讲的一句笑话,他说戒烟有什么了不起?我这一辈子把他戒了好多次烟了已经。



王孟源 01:17:07 

我每天都戒一次,哈哈哈。但是那个吃野味,你危害的是公共环境还有公共卫生,这时候这个就不是你自己的自由,你又没办法说这是我个人自由,这是一个公共卫生议题,就必须要由公共的角度来看,来决定



史东 01:17:29 

非常好,孟源一个多钟头的时间,我们的获益匪浅。



王孟源 01:17:36 

哪里哪里……我不能够确定这个经济上的危险程度有大多大,主要是因为我不能确定中东还有意大利它这个疫情会怎样。



史东 01:17:48 

这疫情,我们几乎可以保证一定是不会像中国控制的这么好,这是最保守的。



王孟源 01:17:56 

但这问题是春天快来了,所以他们说不定会有老天助攻,也不一定。



史东 01:18:01 

对,希望是这样的,而且这个对的确希望这样子么?在这也祝你这个一切顺利好不好?祝你也是,祝你的身体更加健康,好不好?



王孟源 01:18:15 

希望常常看到你的节目,很多我的读者在问你,你的节目怎么样?因为他很多急着在看。对对对。



史东 01:18:22 

这个我,最近我,我会有一个小的这个 video 会贡献给大家,不过我现在稍微讲一下,就是我有几个原因没有,最近这一段时间停了没有做节目。一方面是这个新冠肺炎的事情,这个事情大到让我无法拿捏,无法琢磨到一个就,而且他的信息来得如此之快,走的也如此之快,而且你没有办法去真正的去知道这个信息它的可信度到底有多高,所以说真的假的都混在一起。这个造成了我在做节目上的困扰,这是第一点,第二点就是我的一直是在透过 YouTube 来向我的观众群发布我这个节目的信息,基本上是一种广告了,那么我的 YouTube 的这个频道在一个月以前开始被骇客霸占了,所以霸占就我现在以自己进不去我的频道去做任何的修改,或者做任何的添加,是不是是另外一个人霸占的人在里面?所以说在一个月以来这个频道上所出来的东西都不是我放的。



王孟源 01:19:50 

我想大家都理解,但是大家都奇怪为什么 YouTube 不帮你把它拿回来?



史东 01:19:55 

现在问题就是 YouTube 这种,Google 这种大公司,大公司,老实讲像你这种小虾米它不会在乎的。



王孟源 01:20:04 

哦,OK,基本上是他们不帮忙,不愿意帮忙。



史东 01:20:09 

我,我不知道他们是不愿意帮忙还是他们是根本,就因为基本上我是觉得他们是根本不在乎。你的生死,对他们来讲我没有什么太大的这种意义。但是我从我的角度来看的话,第一个我已经知道我已经想到 YouTube 会有这种态度,所以说我并没有把我的节很多的观众叫我问你又不把节目全部放在 YouTube 这个平台上,我说我千万不能这么做,你想想如果节目全部放在你平台的话,今天会什么样的一个结果?其实 YouTube 对我来讲只是一个,一个广告的一个策略,那么今天我做最坏的打算,如果 YouTube 这个频道不管是任何原因我拿不回来的话,我就另外再辟两个和辟三个。听到我今天才知道所谓狡兔三窟是什么意思。



王孟源 01:21:08 

好,观察者网还是很欢迎你,能送你的视频过去,他的也是可以接触大陆观读者的一个好的管道。



史东 01:21:18 

对对对,而且我自己的这个订阅群也我在告诉我自己,如果即使是这个 YouTube 这个频道不能拿回来,我另外创一个新的频道,这是另外一件事,但是基本上我不能辜负我的订阅群,这是最重要的。



王孟源 01:21:39 

我了解。不过老实说,我一直觉得我过去这 6 年写博客对台湾一点影响都没有。说真的,唯一的贡献只能说是现在大陆的读者,不再说台湾人都是白痴,哈哈哈,都只说台湾人,大部分是白痴,我想你也有同样的贡献,哈哈哈。他们也喜欢。



史东 01:22:06 

对我们来讲,我这,我觉得这我们是我们,我不知道这么说是不是恰当,我对台湾来讲我们是个异类,我们不是一个常态。



王孟源 01:22:21 

至少异类存在。



史东 01:22:23 

总比不存在好。好。对,所以说我会继续努力,我会把这个最新的消息,会用各种我所知道的方式去告诉我的观众,或者我听众,或者我的我的订户,来一步一步来,这个对我来讲也不是一件坏事。就像这个我们今天谈的新冠肺炎的事情,这是一个教训嘛,也是个教训嘛,对不对?以后就知道了,以后就聪明了。



王孟源 01:22:54 

就像,我在想这个一个官僚体系,其实公司里面也是这样子,他们而且不管你是西方还是东方都是这样子,就是事后处理,你是英雄,你如果事先预防的话,根本没有人知道。对对对,嗯,要功劳。对,想想看像老布希,他在打完了第一次海湾战争以后,他事先看到你如果把萨达姆 侯赛因搞掉以后,伊拉克就变成乱局,然后会变成一个泥沼,所以他没有把萨达姆 侯赛因打掉,这是无限的智慧。对,但是他被美国人臭骂了 10 年,结果他的儿子上来第一件最重要的事就是完成这件事情,结果反而是反面地证明他老爸的智慧。



史东 01:23:52 

没错。



王孟源 01:23:53 

对,我觉得我刚刚讲了一大堆预防必须要怎么做,事实上什么样的官僚体系他们都会有这个趋势,就是都是事后反应,而不是是想怎么样事先预防。没办法,人性天生就是这样子。



史东:

对,好,谢谢孟源。



王孟源:

哪里哪里,我很荣幸。好,拜拜。



\twocolumn[\begin{@twocolumnfalse}
\section{波音}
\subsection{20200313}
\end{@twocolumnfalse}]史东 00:28 

各位朋友,你好,我是史东,在今天节目中我们再度地为您请到的王孟源,王先生,今天跟您谈的是波音,在谈波音之前,我们也会谈谈冠状肺炎的一些最新事情的发展。那么首先谈波音,我把我手边几个,我刚从网上Download下来的,一些有关于波音的最近的消息,跟您报告一下。第一个就是,波音它要一个很大的贷款,它的英文是 13. 825 个billion;第二个标题是 The cost of Boeing 的 737 Max ,对 Boeing 所造成的损失是 18. 7 个billion。然然后他说 and still counting 就是到目前为止,所以这将来的数字还不止这么多;第三个标题他说 Tesla目前已经超过了波音,成为美国现在最有价值的公司;第四个标题他说 Is Boeing Done For,说波音是不是已经完了?这个新闻标题是说在它的股价跌了 64 个 percent 之后,他问这个问题,也是波音已经完了。这也是今天的节目中我们要为您谈的讨论的一些问题。那么现在我们首先把我们的好朋友请入我们的画面之中。孟源,你好。



王孟源 01:47 

很荣幸再上你的节目。



史东 01:49 

哎,是的,我刚刚很快地把这 4 个标题交代了一下。然后我希望你从最新的有关于这个冠状肺炎的形式这些发展来帮我们带入到这个主题,好不好?



王孟源 02:03 

在过去两个月,我其实花了很多时间在钻研像分子生物学,还有这个传染病学,因为我的专业其实并不是生物或者医学方面的。那我写稿的话或者做评论的话,如果不是确信百分之百的是正确的话,我不敢讲,所以就往往写一篇文章要先阅读几百篇文章来做研究。所以最近的确是很忙,结果产出反而不是很多,因为这不是我原本的专业。



不过我想我在过去 2 个月所做的评论基本上都是正确的。就是在一两个月前我就说这是一个百年,一件非常厉害的传染性病毒。我们现在可以确定它的确是从 1918 年的西班牙流感之后最厉害的新型变病原体。然后因为西方国家太过傲慢,他没有利用……我想我上一次在你上你的节目我已经谈过了,这其实是中国在为世界党挡第一线,这是一个人与大自然斗争的战争,那中国在第一线基本上迟滞了这个病毒的攻击,给予世界人类大概 2 个月的额外时间来做准备。但是你可以看出基本上除了 2003 年 SARS 曾经影响过的几个地区,像台湾、香港跟新加坡之外,其他的世界的国家都不把它当一回事,所以有足足两个月的时间,他们这些病毒,就在他们的社区这样子隐蔽的传染。那那这其中表现最差的就是美国了,那其次是英国跟日本,你不要看德国,它现在好像一副很散漫不管事的样子。其实他们的检测你如果看他们检测的比率,就是前两天我看到一篇文章,就是观察者网上有一篇文章讨论这个死为什么死亡率,为什么新冠的死亡率会有高到6\%,也有少到0. 5\%,0. 6\% 的,那其实这个新冠的死亡率是很稳定的,就是它只受两个事情影响,一个是你的那个ICU,就是 Intensive Care Unit 急诊病房,能不能把快要死亡的病人放上呼吸器来续命?这个会影响你的死亡率。那当一个国家的那个急诊病房还没有被过多的病人淹没的时候,这个因素基本上是一样的,就是除非你到像伊朗或者是非洲这种国家,否则的话你在欧洲跟东亚的这些先进国家基本上都能够满足这种要求。那武汉是有受影响,因为刚开始的时候他们的确是被洪水淹没,所以他们的那死亡率是 百分之三点多。那其实我认为如果您每个重症病人都能够给上,就是给上呼吸器的话,它的死亡率应该是 1\% 到 2\% 之间,所以你对在目前这个态势,你看欧美的这些国家,你要看他们事实上确诊有多少人,反而是用死亡率来倒推最简单,因为第二个不确定的因素就是你有没有去检测。



史东 05:42 

就是它的分母这个数字就不对。



王孟源 05:45 

它的分母这个数字,其实是最大的变数。那与其想要把这个分母搞对,然后来算这个死亡率,其实应该用这个死亡率,因为死亡率是比较天然的,比较是有规律的。你假设它就是 1\% 到2\%,然后反过来用这个往回去算的人数,这样算回去就是乘以百分之,乘以 50 或者100,你就可以估计现在这个真正得了病的人有多少。对,但是你得病之后需要 2- 7 天才会有症状,然后症状出来以后大概会 7- 10 天才会死亡。所以事实上你这样用这个死亡率倒归,倒推。比如说现在意大利死了 800 多人, 900 人,你如果乘以乘以 100 的话,就是9万人感染。但是这个9万人感染不是现在感染,而是 14 天前的感染。因为你要倒推,因为它你先得病,然后出了症状以后还要再过一段时间才会死亡。



史东 06:51 

所以还要再乘上一个基数。



王孟源 06:54 

还要这你过去这两礼拜还又有指数成长,所以这个你这样乘以 50 就是,除以百分之二或除以百分之一得出来数目其实是 14 天前的,对不对?所以现在欧美的这个情况其实是非常非常的严重。你看德国的话还好,因为德国是在这一次疫情一开始的时候,它虽然防治的工作都没有做,但是它检测器是从1月就开始大量生产,所以它的检测就做得比较彻底像美国跟英国,尤其是美国就非常的糟糕,那日本也是。所以我认为,我自己已经是一再地提醒我儿子要尽量避免出门呐,出门的话一定要戴 N95 的口罩。



王孟源 07:46 

不管怎么样,我觉得现在会有非常强烈的经济的影响。就是你算一算的话,如果在意大利有十几万人,实际上有十几万人被感染,德国、法国跟西班牙至少也会达到十几万人。他基本上已经没有办法做出中国所做的这种防治,因为它没有办法做出像中国那么严厉的隔离,你即使是像意大利这样全国隔离的话,它没有办法真正禁止你出门。他们的所谓隔离只是说你不能够从这个城市到另一个城市,那你事实上也不可能叫每家每户待在房子里面,因为他们没有办法像中共那样组织把你的那个杂货送到家。美国是更糟糕了,美国那连提都不用提,那英国是今天刚刚宣布他们直接跳过所谓containment,就是隔离防治而直接到 delay 的地步。这个所谓的 delay 的意思就是什么?他们知道这个会指数成长,所以现在他们……



史东 08:57 

现在弃守第一线。



王孟源 09:00 

弃守第一线,就是现在比如说你的那个检测剂只用来对重症病人来用,你如果是轻症的话,他根本不管你了,就是完全放弃追查轻症,我觉得这是故意的。就是你可以说 Trump 在美国这个是人谋不彰,但是我想它只是不在乎人民的死亡, Boris Johnson 却可能是有意的要让穷人死亡。你知道,Boris Johnson是在 2007 年的时候被采访,那时候他还没有当上国会议员,结果他们聊天聊到那个 Jaws ,就是大白鲨那部电影。大白鲨那部电影的一开始是因为是有一个大白鲨到一个度假胜地的小镇,然后开始吃人,然后它里面的那个坏蛋角色就是那个市长。那个市长不管怎么样都说为了我们的经济,为了我们的日常生活。



史东 10:03 

不能够把消息传出去。



王孟源 10:05 

对,那在 2007 年的时候 Boris Johnson 被访问,他就说我认为那个市长才是这整个事件的英雄,他抓对重点了,那这个人原本就是这样子。不顾人家的死活,比那个 Trump 还要careless。那你可以想想,这一次他这样子,到现在还没有学校停课。你知道吗?英国是所有先进国家里面做得动作最慢、最不在乎的,我觉得他可能是有意的,你如果考虑到他 2007年说的话。不过我觉得他这样是很短视的,因为我们上一次聊天的时候谈到对中国的打击,可能是在一个季度里面会降低你的 GDP 生产率, 2\% 3\% 或者更多。在中国你还有6\% 的成年成长率,还可以损失一些。在像英国或者美国这种原本就只有 1\% 2\% 的成长率,你再掉下去,然后又有人大批死亡,然后这不但是所谓的 demand side,就是经济上的需求不见了,你连那个工厂还有商店都没有办法营业,这个 supply side 也不见了。这个是非常严重的事情,有可能会造成恐慌,有些地地区可能会造成社会秩序的崩溃,我觉得这个不确定性很大。



史东 11:38 

我想到一个事情,在所有的你刚刚描述的在欧洲的这些问题之外,土耳其在放难民。



王孟源 11:47 

现在的问题是欧洲跟美国原本经济跟社会秩序就有一大堆的问题,像你所提的这个难民也是一个问题。所以这个新冠病毒是二战以来人类社会所面临最大的全球性灾难。所以我在上一次上节目的时候提到,我们只能够指望到了夏天它自然消退,否则的话这个真的是。



史东 12:19 

我再打断一下你所谓的灾难,我想你讲的不只是生命的损失,有很多是财产经济上的损失。



王孟源 12:30 

对,我们今天讨论波音其实只是这里面很小的一个例子,但是我怕的是在一旦在经济跟生计

上的打击太大之后,会整个掀起社会的秩序的崩溃。



史东 12:51 

那就是大损失了。



王孟源 12:53 

对,就是我刚刚所说的这个公共秩序可能会崩溃,那我们希望不要到那个地步。那我上一次上节目的时候说因为流感每年到了春末夏初就会消失,所以也希望这次的新冠也是这样的。不过事实上不一定,因为比如说流感跟新冠有一个很大的差别,流感他攻击的是支气管的上部,就是大概这个地方,那新冠是直接攻击支气管的末端,就是在肺泡那边,所以他是直接攻击肺泡。所以你说流感的死亡率是 1/ 1000,它这个大部分的死亡也是死于肺炎,但是它这个肺炎通常不是流感病毒直接引出的,而是流感把那个支气管那上面的这个表层细胞摧毁之后,支气管不再有抵抗,然后由其他平常无害的那个细菌跟病毒,从上部的呼吸道传到下面,然后最后才弄到肺泡。所以这是为什么流感的死亡率比较低的原因。那新冠不一样,新冠是直接攻击肺泡那个地方,就是直捣黄龙,所以它是不是会跟流感一样,因为夏天的空气温暖潮湿而就停止传播?其实我们现在还不能确定,不敢说。那我原本看的印度没有什么病例,还算是比较乐观。不过你看现在在欧洲传染的最严重的两个国家是意大利跟西班牙,都是南欧的国家,就是说他们的气温其实是比较高的,都是至少有十几度摄氏,所以很现在很难说。但是正因为这件事这么重要,所以你不管讨论其他的什么政治、经济,什么其他的议题,都要看这个疫情的发展是什么样子。



王孟源 14:59 

那我觉得现在只能够假设说在未来的这两个月会继续呈指数传播,然后到5月看看能不能因为天候的关系和稍微放缓点,因为过去两个月,其实最重要的,我刚刚也提到,就是你至少要像德国那样子,大量地批量生产检测剂,所以你这样才有可能做到containment,所谓的 containment 就是把每一个病例都标列出来,然后强制隔离。那这个前前提就是你每一个可疑的病例都要被检测,那美国在过去这两个月基本是完全浪费时间,那个 CDC,它到现在还没有检测到1万个例子。你想想看中国每天就是一两万的检测,南韩也是做到每天几千个甚至上万的检测,美国到现在两个多月总加起来还没有1万个检测。这真的是很可笑。那你现在再宣布什么国家紧急状态都没有用的,因为美国的这个体制,是所有欧美国家里面私有化也最碎片化的一个,那他对这种公共健康事件,是完全没有办法来防止的,因为所谓的私有化就是以利润为先,那利润对为先的第一步就是你把费用削减到 5 头上,就是所有不需要的费用通通削减掉。那我们待会提到波音也是这个样子,那你削减费用就是:平常99。 9\% 的时间都没有这个新冠肺炎,那你为了他储备的那些产能就全部都不见了,全部都被砍掉了,因为那些都是浪费钱嘛,你 100 年来都没有用上,那为什么要投资这样上面?你那在这种情况下美国基本上是完全没有准备。当然Trump本身人谋不彰,因为他反智反科学,把美国联邦政府里面的一些专业官僚全部都统统掏空了,这也是我们现在看到这个情形的一个一个助力,不过美国的体制原本就不适合应对这种大灾难。



上一次我上节目,我说在 2009 年 H1N1 流感之后,因为 WHO 很快就宣布是Pandemic,美国后来就跟他抗议他们后, WHO后来就回去改了 Pandemic 定义,所以重新定义Pandemic,所以后来这个 Pandemic 定义就严了很多。所以这一次,一直拖到我看那几天前,三四天前他们才宣布是Pandemic。其实他们的意思就是说:如果每一个国家都能够像中国或者新加坡这样子来做,那么我们还可以做到containment,那事实上这个假设基本不可能成立的,对不对?看欧洲还有美国就知道,他们这个是完全不会有办法。其实英国这个做法还算是比较诚实的,我刚才已经解释过 Boris Johnson 本身就是存心要把美英国的穷人弄掉几十万。我上次上你节目的时候提过,英国在过去 9 年 10 年由保守党执政,就至少已经有7万人。那这一次我想这个 boys Johnson 比他的前任要狠心得多,而这是肯存心就是要把英国人口最穷的那几百万通通所谓的 Cull, the Culling, C U L L。他其实是比较诚实的。那你看看他们这种做法,基本上就是完全私有化之后的,由资本来佐证之后的那个的逻辑,行政逻辑,就是他们现在说所谓的 HERD immunity。所谓什么是 HERD immunity,就是靠着大部分的人口都已经得病,来停止这个流行。



史东 19:25 

很可怕的一种思维。



王孟源 19:27 

很可怕,你算一算,那个死亡率是 1\% 2\%,然后你要得到  HERD immunity,就是大,基本上要有 60\% 到 70\% 的人口。在欧得利。对。



史东 19:38 

在这个前提之前,我们可以想象它所有的这个保护的措施,它的种种的这种要花钱的保护的措施,以及它的医疗措施会 concentrate 在这些高收入的阶层以及富豪。



王孟源 19:55 

精英,对。



史东 19:55 

富豪精英资财阀,这又是一个制度上的可怕的不公平。



王孟源 20:01 

其实你看现在欧美,他们诊断出来确诊的大部分都是政商精英 ,Tom Hanks,对不对?为什么 Tom Hanks一下就被确诊了?因为他有钱,他可以负担得起,所以他去……



史东 20:16 

OK,了解。



王孟源 20:21 

我想有了这个前提,所以我们现在可以开始讨论这个经济的影响,那我再说一次,我认为目前看来有两个可能,一个就是说现在一定会继续指数成长到5月,然后到5月的话,天候看能不能有帮助,如果天候有帮助的话,这个全世界至少还是会有几千万人感染。如果没有帮助的话,那最后就会有几亿甚至几十亿的人感染,。



史东 20:49 

好可怕的数字。



王孟源 20:50 

然后其中有 1\% 到 2\% 会死亡。我先谈谈会死得最惨的两个的工业,第一个是我从去年这个时候就已经一再说已经供过于求,所以我就预言在 2000 2020 年会很有很多企业破产。那您想想看,在新冠病毒出现之前就已经快要破产的这个就是那个页岩油业,就是石油工业。我已经说了一年他们会有一大堆那个中小型企业破产,这是因为美国的页岩油业基本上是中小型企业。不是那些传统的大油公司。那他们的钱,他们的经营模式基本上就是去借钱,而且不是跟大银行借钱,而是跟中小银行借钱。我其实我在我的博客上也讨论了,我认为是Putin有意要对美国人落井下石。



史东 21:55 

同意,完全同意。



王孟源 21:57 

嗯,这个 2007 年, 2008 年美国金融危机的时候,普京就跟中国讲,说你要不要跟我联手把他们的国债卖掉落井下。当时中国人说不行,我们是共荣的,合作共荣的结果反而浪费了一大堆钱来撑起全球的经济。事后美国人感谢吗?没有,对不对?一次普京就没有找中国人来商议,他直接去跟沙乌地阿拉伯,沙特来商议,因为沙特他依靠美国的驻军,他不能够冠冕堂皇地来给美国人戳刀子,所以他们还唱白脸。说来说去就是你原本就供过于求,你不管怎么样减产,这个石油的价格还是每桶会降到 30 块钱一桶它,那你还不如大家都一起增产,把它降到 20 块钱一桶来确定这些美国的页岩油商全部死光对不对?因为你看起来差50\%,实际上就是十块钱,对他们来说,反正就是再多再一多借一点债这样子。但是你一旦把美国的页岩油商全部打倒之后,两三年之后石油复苏之后,他们就可以又回去毒独霸市场了。所以这件事情其实是Putin搞出来的一个国际政略的手段,我觉得蛮高明的。那这次中国我想也已经学乖了,不会再出手去救美国。



不过这次我从去年就开始说美国这次的经济衰退会跟 2008 年不太一样。最大的不同就是 2008 年是所有的大银行也参与那个狂欢,而且它的那个狂欢的重点是所谓的Subprime。 Subprime 就是次贷。那次贷那时候有多少?是 13000 亿美元的次贷,这大部分都是由大银行经手,所以一死的话,那个消费者就没有办法再继续借贷,然后大银行也就崩溃。那这些大银行都是 too big to fail,所以当时美联储就必须要出手去救他们。那这一次是不一样,这一次我在讨论 Repo 的时候,半年前跟你讨论 Repo 时候我已经说过, Repo 之所以会出问题,就是因为大银行已经作壁上观,他们看出苗头不对。



史东 24:26 

至少不进场。



王孟源 24:27 

现场他们已经躲到地下室去了,要准备躲掉这个龙卷风了。这并不代表金融界会没有破产,就是你想想看几家这个页岩油的商,他们除了在企业债券举债以外,就是 corporate debt。我上次也说过 corporate debt,上一次是subprime,这次是 corporate debt,上一次 subprime 是 13000 亿美元。这一次的 corporate date,美国的 corporate date 已经高到 18000 亿,比上一次的次贷还要多了。那他们除了发这个企业债券之外,他们还会去跟当地的中小银行借钱,比如说像 bank of Oklahoma,因为页岩有有一个油田是在Oklahoma,所以他们就大就地在跟那 bank of Oklahoma,这是一个中小型的银行,这些中小银行全部光是借贷给这些页岩油商就有3000 亿,就是你除了企业债里面所有的企业债是 18000 亿,然后你还要再加上至少 3000 亿是页岩油商跟这些中小银行。其实到目前为止……



史东 25:41 

页岩油的倒闭会拖垮一连串的金融事业,包括银行的倒闭。



王孟源 25:52 

中小型银行会。但是这个,因为他们是中小型银行,所以可以让他们倒。我想美国联邦政府会让他们倒,就是因为他们都有保险的,就是他们的那个零售的存款人都已经有 FDIC 的保险了,他们这个所以要倒就倒不关联邦政府的事。但是我之所以提这个是是说目前美国这个衰退的核心,在金融上面。这个基本上它的债务问题比 2008 年还要大,只是不同类的。它是加起来刚刚我提的1.8 trillion,再加上 0.3 trillion,就已经 2. 1 trillion 了,对不对? 21000 亿,这个已经比 2008 年的次贷 13000 亿还要多了,对不对?然后但是差别在于上一次这些次贷会直接打击大银行,这一次不会打击大银行,所以它的性质是比较像 2000 年的那一次, 2000 年那一次也是股市,还有那个企业借贷,为什么会这样?为什么说?我说股市跟借债会连在一起?很多大公司你当然有页岩油这样子,是借了一大堆钱,几千亿来是来挖石油井的,但是呢,一般的公司你不需要这种 capital investment,他们借了这几万亿来是干什么呢?是来 Stock Buyback,就是为什么这个美国的股票会一直往上升。其中一个原因就是那些总裁跟经理,借了一大堆钱之后,把这些钱来回购自己的股票,低利息的,他反正你这个现在的利息这么低。这也是因为那个量化宽松之后,有 4 万多5万亿的美金,反正在到处流,那你不借白不借,你说 1\% 2\% 年利率的利息,借了以后你就回购自己的股票,那股票每两年翻一翻,这有什么好处?董事会非常的高兴,华尔街日报把你吹捧成经营之神,然后你自己的股票期权也大赚侧赚,因为你这些总裁他们年薪就是 2000 万、 3000 万,这其实是小钱,你这些期权一赚就是好几亿了,对不对?所以过去这几十年玩的这种花样,这变本加厉,所以这一次这个其实是像 2000 年那一次的衰退一样,只不过规模会比 2000 年要大上四五倍。而美国的国力跟那个经济的承受力都已经不如 20 年前了,所以这次的话\%



史东 28:42 

而且中国也不会帮你了。



王孟源 28:44 

而且不像 2008 年那样子,中国会出手帮忙。所以他的这个性质是像,论性质是像 2000 年,但是论规模是像 2008 年,可能会甚至更多更大,我原本去年我还没有认为规模会这么大,但是因为新冠疫情是没办法预测的,这个新冠疫情至少让这个规模变成原先我预测的 3 倍。那我之所以要谈这些事情,因为这些特点都呈现在波音上,你说这个矿业是所谓的第一产业,但是波音是原本是,就像你刚刚提的,你在节目一开始提的波音是美国第二产业,原本的头号公司,他在过去这二十几年被搞垮,就是因为他的文化被渗透了,被颠覆了。从一个工程生、工程师主导的以品质为先的公司,变成一个 MBA 主导的一切以股价至上的公司。而这个始作俑者其实不是波音,而是GE。就是我在三四个月前写了一篇文章,叫做波音衰败之源。在我的博客上你可以去看看这个始作俑者,即使是 40 年前的 Jack Welsh,他刚好在几天前过世了。Jack Welsh他把 GE 拿来当实验,就是一切以利润为导向,追求利润来作为最高指标,那对你追求利润最简单的办法就是削减你的投资,那些所谓的开发新技术这种东西都是五年十年,甚至你像 GE 的那些涡扇发动机,那个是要 20 年 30 年才会有回报的。你要你要赚钱,最简单的办法就是把这些投资砍下去,对不对?那所以他在 1980 年代就把 GE掏空了,但是 GE那个时候股价直线上升,真的是指数上升,对不起,不是直线上升,而是指数上升。到了 2001 年  Jack Welsh 退休的时候,他已经被称为,居然有美国的媒体把他称为世纪的经营之神,就是放到发明流水线的 Henry Ford 的上面去。那么离谱,就是他被吹捧到那个地步,而且他训练出来的一大堆 GE 的一大堆干部,后来都遍布到美国的所有的那个实业界去。



波音那个时候就是在 1997 年的时候上了贼船,他买下麦道。当时麦道的那个总裁 Stonecipher,Harry Stonecipher就是 GE 出身的,所以完全就是搞 Jack Welsh 的这一套东西。



史东 31:52 

我插一句话,你知不知道当初 Boeing 为什么要买麦道?它的目的是市场占有吗?还是技术还是什么?



王孟源 32:03 

我在我的文章里面有分析,其实是就是波音的整个董事会跟当时的总裁都被Jack Welsh  忽悠了,他们想要搞 GE的那一套,但是因为波音有4万名高级工程师,他们的文化传统就是工程师至上,所以董事会拗不过那些工程师,他们就想出一个曲线救国的办法,就是你先去把麦道买下来。那麦道已经 MBA 治国,所以你合并之后,那当时合并的时候就有很多人说这真是一件奇怪的事啊,好像是卖到用波音的钱把波音买下。为什么这么说?哈哈哈,因为你合并之后的公司。所有的经理都是卖到的人,就只有总裁一个,还是原本的波音的总裁,然后他很快地退休,退休之后就是由那个 Stonecipher 接任。那Stonecipher 接任之后,在Stonecipher接任之前,他们就已经把总部从西雅图搬到芝加哥。所以你以前在芝加哥有什么?什么都没有,所以 20 年前他们把总部搬到芝加哥,就是一个光棍的总部,我所知道的正确的企业治理哲学都是你的经理应该跟第一线的工人越靠近越好。结果波音反其道而行,他们把总部搬到 1500 英里外的地方,他的目的就是要完全砍掉那些工程师的影响,因为你这样一来工程师就没有办法去跟你啰嗦,你就可以用遥控的,方法来为所欲为。



史东 33:51 

就开始改变整个公司的文化。



王孟源 33:55 

对 Stonecipher后来当了一年CEO,以后,他后来在一次那个开会的时候,还很高兴地夸耀说,我很骄傲地可以告诉你们,我把我们的 capital expenditure。什么是 capitalexpenditure?就是科研,还有买机器这种东西,为了生产或者为了研发这种投资,砍了60\%,我非常的骄傲。你看,你这样一下来,然后接下来的过去这20 年,他们把以往视为保障的那些资深工程师,当做是liability,就是包袱,拼命地想办法找借口把他们开除掉,然后去削减他们的pension,就是退休金,就是你把你的那些科研人员当成包袱,当成累赘,而不是当成宝贝。这就是 Jack Welsh 的那个的经营哲学。那这样的结果就是波音在Stonecipher之后,下一个 CEO 也是直接从 GE 那边来的,后来这个人退休以后,换上一个波音本身出来,就是Muilenburg,但是他之所以会被从波音内部拔擢起来,就是因为他完全相信这一套,他是身体力行,然后所以才会当上CEO。所以后来……



史东 35:33 

所以说他自己虽然是波音科班出身,但是他所信奉的教条是 MBA 的教条。



王孟源 35:42 

事实上他做得如此之离谱,连David Calhoun,就是在两个月前他被开除之后接手当新 CEO 的。 David Calhoun,也是GE出生的,但是David Calhoun 在一个礼拜之前就是3月 5 号在纽约时报上被采访,他居然说 Muilenburg注重股价了。你想想看,我刚刚说这一套 MBA 完全只追求利润来操弄股价的,这一套就是 GE 出来的。那这个 GE出身的人都觉得 Muilenburg 做得太过头了。



史东 36:18 

你觉得,他这么说,因为你在节目之前你也把这个访问 forward 给我,我看了一下,你觉得这个 Calhoun 他这么说是真的是他这么感觉,还是说他现在的位置,他必须要用这种说法来攻击一次他的前任的这个CEO?



王孟源 36:39 

其实他(Calhoun )不能够攻击他( Muilenburg)的,因为他(Calhoun )从 2009 年就是董事会的成员。



史东 36:45 

对,我也想这个问题,你是董事会的成员。



王孟源 36:47 

所以事实上这个访问一出来之后,两天之后他就公开道歉了,他说他不应该说那句话,所以我认为他是真心的,他真的是认为这个真的那个Muilenburg太过分了。我从去年4月开始写波音,然后想要揭穿他们这些文化上的……



史东 37:09 

对不起,我插一句话,孟源,你是从回忆一下你是从什么时候开始注意到波音这个问题。



王孟源 37:18 

我原本写波音最早的一篇文章在我的博客上是去年4月,那时候纯粹只是为了要解释这个波音 737 Max 会出问题的技术原因,因为当时他们还想要在公关上面欺骗数据。对,事实上是因为他们换了更大、气流更多的新的引擎之后,这个飞机变成不稳定,所以必须要用这种电传操控来稳定它。但是问题是它们没有足够的时间来把它做对。因为 Airbus 他们先搞出那个 A320 的新引擎的版本,所以波音基本上只有三个月就要完成设计,那你三个月之内做重要的改动都没办法,除了就是把引擎装上去以外,你其他的都不能做。他们就是引擎装上去以后发现会造成不稳定性,但是你没办法改了,没办法改以后你就要偷偷摸摸地,用软件来把这个硬件的问题压下来,那这个过程中就必须要欺瞒FAA,那所以现在在过去这一年透露出来的一系列的内部通讯都显示,就是他们自己也知道他们是在骗人,不是只有这么一个,有几十个这样的那个内部邮件被透露出来。



王孟源 38:55 

那大家就都说这是因为 FAA 被他收买了。噢,的确, FAA 在过去这几年,他们的那个安全部的部长是完全是波音的人,所以的确是被收买了。但是我们不要忘记 FAA 从来就都是偏袒美国制造商,这一件事情不是过去这 10 年的新鲜事情,也不是过去 20 年的新鲜事。你看看FAA成立到现在六七十年,他有停飞过几次?只停飞过三次,三种飞机,第一次是 DC10,这是麦道的飞机。那时候一连有三个坠机,这是三十几年前, 40 年前,一连有三个坠机它才停飞。然后第二次停飞是 2012 年 2013 年787刚刚出来,波音 787 刚刚出来的时候,因为它的锂电池会自燃,所以停飞。然后这次737Max是被因为中国民航局先率先停飞,然后全世界都跟上去了, FAA 才不得不这样的。很多人以为 FAA 是在最近这几年被渗透的。没有错, FA 在最近几年被波音渗透得很厉害,但是这并不代表在它被渗透之前,它就是一个公平有效的管理机构。过去这几年的确是越来越不入流,但是原本 FAA 在过去几十年,它对波音跟麦道原本就是极为呵护,只是说你们不能够来告诉我怎么做,我自然会去照顾你,我自然有、这个方式。



王孟源 40:46 

NTSB 也是这个样子,这个 NTSB 就是美国的那个交通事故调查委员会。我举一个例子,在 2009 年的时候, Turkish airline 就是土耳其航空公司 ,Flight 1951,从土耳其飞到荷兰的阿姆斯特丹的时候,它要降落的时候摔在跑道的前端,就是它距离那个跑道的边缘还有大概半公里,他就摔到地面上去了,死了大概 10 个人左右。 9 个人还是10 个人。后来那架飞机是737NG,就是 73G Max 的之前的那个型号,它是怎么出事的?是因为 737NG上面有两个小雷达,是所谓的地面高度侦测器,他那个负责侦测地面的高度,因为其中一个失灵了,所以飞机以为它的高度是负值,那飞机就认为它已经在跑道上,所以它主动地把自动导航关掉了。但是关掉的时候,没有跟。飞行员讲,飞行员以为他们还在自动导航,所以飞行员就没有去注意到他们的那个速度跟高度都过低。等到他们注意了到的时候已经来不及了,因为正在降落的时候,就差了那几百英尺,他们就撞到地面去,这跟后来这个 737MAX 的Mcas一模一样。也是一样,他们的飞机上只有两个传感器,传感器里面两个传感器里面只取一个的值,而不是两个的值都看。这个飞机就会自动地做出错误的决定,然后会让飞行员没有时间来反应,没有办法来反应,跟后来这个 MCAS一模一样。结果事后是由那个 Dutch safety board,荷兰的飞航安全局,跟那 NTSB 联手来调查的,结果全部都怪到飞行员头上。然后两个月前纽约时报也是因为在挖那个波音的丑闻报道出这件事情。他们原来当初荷兰的非航安全局去雇了一个大学教授,荷兰的大学教授来帮他们做这个调查的报告,原本是把重点放在高度测量计失灵的事情上,结果被 NTSB 把它否决掉了,就是把它整个涂改掉了。那这出来以后就变成一个丑闻,荷兰飞行安全局就要求重开这个案件,这个要求被波音跟呢 NTSB 否决了,这个都是一个多月前发生的事情,这就是我刚刚讲的,并不一定是波音渗透了NTSB,而是 NTSB 还有这个 FAA原本就会照顾自己国家的生产商。



史东 44:07 

这个这么大的公司所谓的财大气粗,一个财那么大的公司你可以想象它在公关上、在旋转门上、在各个方面的影响力的投入,对不对?



王孟源 44:20 

不过我认为那是难免的,真正的问题出在美国这个体制上。对,这个美国就是鼓励这种体制,然后他们基本上就是合法贪腐,只要你不是直接立即的利益交换,而是间接的,那你就可以就一切合法。过去这两个月其实有一大堆波音的新闻,我原本要写一篇文章,后来你问我说是不是在上你的节目,我就想干脆把收集的资料拿出来,在这边讲一讲收集到什么样的资料。光在二月,一个月就有4 个新闻,是 FAA 主动翻出波音 737 Max 的问题。在2月 14 号, FAA 宣布737 Max 的网线不符合最新的标准;到了2月 22 号, FAA 说737 Max的油箱里面 ,Oil tanks 里面的发现了有多起垃圾,就是有莫名其妙的东西在里面;到2月 25 号, FAA 宣布 737 的 engine panel 就是那个引擎的Covering,引擎的罩不够强,所以引擎出事故的时候会有碎片打伤乘客;到2月 26 号, FA 宣布737 Max 机翼上面有一层铜箔,这个铜箔是负责防止雷击的,因为它的那个机翼现在都是用复合材料做的,就是碳纤维做的,碳纤维不导电的话,你被雷击的话就会有问题,那个电荷都会累积在这上面,所以他就在碳纤维的表层底下还要再藏一埋一层铜箔。他说这一层铜片设计得不好,有可能会把控制引擎的电子线路割断,所以这个也要去改。你看它一连两个礼拜之内丢出事件,要737 Max 去改的东西,我的解读不是说FAA 要刁难波音,反过来是 FAA 准备要放水,放手,准备要让 737 Max重新上市。所以他必须要把所有的隐患都讲清楚,否则的话就这些事情如果在重新上市之后被揭出来的话,大家脸上都不好看。



王孟源 48:02 

为什么波音急着要上市呢?其实去年我在9月的时候也写过一篇文章,主要的原因是Stonecipher在接手波音之后,一个很重要的增加利润的手段就是outsourcing,他把他连原本是波音的分布统统割出去变成独立的公司。比如说。在 Wichita, Kansas的那个 spirit Aerospace (编注,应为Spirit AeroSystems),就是专门负责替波音造机身的,整个机身的生产,原本是波音的机身生产部门,整个割出去变成一个独立的承包商。



史东 48:39 

对不起,我打断一下他们这么做的原因,我猜从一个外行人来看这个,他们做这个,是不是希望能够把这些原来的 cost repartment 变成一个 profit center?



王孟源 48:52 

就是这有两个原因,第一个是波音的福利好,所以人员的薪资费用高,你这分出去变成小公司以后,你就可以大量地去砍他们的福利跟薪资。第二个是这样子,连那个譬如说做机身的都独立出去,那有关机身的那些研发还有器材,就是我刚刚提到的capital expenditure。就变成这些公司要出钱,对不对?这样一来,你的这个波音的投资报酬率,就还有那个利润就是可以上去了,因为你这个花大钱的不是波音出钱了,而是这些公司出钱,那你可以去挤压供应商,所以他们的那个地位、政治地位比你弱,你可以去挤压,让他们的利润率低。波音的核心,就变成利润率就可以弄上去了。这是MBA 要把一个公司的利润率搞上去的一个手段之一,Outsourcing的好处是在这里,不一定说是效率高,但是……就像可口可乐装瓶的那家公司,不是可口可乐本身,可口可乐只卖糖浆,那个糖浆卖出去以后,装瓶真正送到超级市场的是另外一家公司。为什么?因为装瓶那个东西需要机器,需要雇一大堆工人,这种真正物理上的要投资一大堆钱的,你的利润率先天就不会高。所以你他们把那个所谓的智慧财产集中在一个小小的总部里面,就只有几个律师,还有几个 MBA 在那边管。然后那个买糖浆的话……



史东 50:30 

你这个解释非常清楚。



王孟源 50:34 

要这就是他们美国商学院在过去这四五十年来的创新,就是搞这些,怎么样把利润搞上去?但是他们这样搞的结果,就是把整个产业都搞得空心化了。美国的这个现在经济,这个产业空心化,不是中国人弄的,而是他们自己放弃的。所以我为什么觉得 FAA 是在准备要让737MAX重新上市呢?因为波音在整个 2019 年,在Muilenburg被开除之前,一直都还是在高速生产 737MAX,它有 100 多架接近 200 架的 737 在西雅图的停车场上等着。他不是傻子,他明明知道这个复飞是遥遥无期,他为什么会这样?是因为你这些供应商,他有这么几百个供应商,这些供应商以后又有第二级的供应商,这些供应商都是原本就是中小型的企业,原本利润就很微薄。我刚刚解释为什么要outsource,就是要把他们的利润,低利润的部分通通割出去,他们原本利润都很微薄的话,你如果没有生意,如果他们一下子停产,他们撑不下去,他们撑不了多久,绝对撑不了一年。所以过去这一年波音才会像你在节目一开始讲的,他们花了100 多亿美元,来继续生产 737MAX,为什么?就是为了要让这些供应商活下去?这些供应商要是活不下去,关门了以后他们那些生产器材还有熟练的工人就都不见了,你一年两年之后要复工也找不到人。而事实上Muilenburg被开除,在两个多月前被开除之后,Caulhoun第一件事情上台,第一件事情就是停产737,为什么?他看到的是,哇,你这样子,每个月几十亿美元我们撑不下去。我们的全部的……你刚刚提过他的全部的那个 head line 也就是 130 亿美元,对不对?你,你这样子顶多也就是撑一年,我们不能够再撑一年了,你这个如果复飞撑到年底的话,那咋那还得了?但是呢,到了2月,Caulhoun刚刚进入状况,他就知道为什么 Milenberg 会这样子。在2月的时候波音反过来去雇了 700 多个机械工,他们在2月反过来去雇人,就是你现在已经没有工作可以做,他还跑去雇人。为什么?因为你必须要复飞,你不复飞的话这些供应商很快都要死掉了。所以他这一次不是去跟 FAA 讲说我要你替我们作假,他们只是跟 FAA 讲说我们要活不下去了。所以到那个 2 月 23 日波音的民航部门叫做 commercial division,他的那个总裁叫做 Stan Deal,他宣布将特别发出 40 亿美元的现金给所有的供应商,来帮助他们渡过难关。所以我的预期是在2月的时候,就是一个月前FAA 的计划原本是要在4月左右就让他复飞。



另外一个线索是,因为这次 MCAS 已经变成众矢之地了,所以他们不可能再说不要求那个飞行员做训练,所以飞行员必须要先受训练,以后才能够飞737 Max。问题是因为他们原本没有要求这一件事情。全世界737 Max 的训练模拟器你知道有多少吗?只有 36 台,全世界只有 36 台,也是一个月前传出消息,那个供应商叫做CAE,开始量产,提升那个产量,开始生产这些。所以从这些蛛丝马迹我也可以看出,大约是4月准备要让他复飞。那复飞之后他就可以开始交货,然后供应商就可以活得下去。那现在的问题是,一个月前他们还没有预期到新冠疫情,对不对?他们那时候还认为新冠是中国的问题,对不对?哈哈哈,他们认为我们损失的中国是顶多损失中国市场。事实上,中国在对波音的出的问题是认识地,是很清楚。过去这一年除了中国民航局是第一个禁飞 737 Max 的管理单位之外,他还把基本上几乎所有的 787 的订单都砍掉了,所以波音必须要减产787,就是因为从中国这订单都不见了。事实上 787 有两个工厂在生产,一个是在西雅图,一个在南卡的,那个工厂是由麦道的人建起来的,所以管理特别的差。那他那边生产的787 是全世界都知道是所谓的lemon,有很多公司,包括中东的航空公司,在签约买 787 的时候都特别指定,我的 787 必须是西雅图那个工厂生产的,我不要南卡。



史东 56:07 

这好可怜呢。



王孟源 56:08 

中国是在3月四月禁飞,去年3月四月禁飞 737 之后好像就有做检讨了。然后你知道中国他们这个行政的方法是他们只做不说的,他们就闷声不响地把 787 的那个订单全都取消了。就是他们大概也是认知,到波音的问题,不是只有一个机型,而是整个企业的文化都已经烂掉了,所以你连 787 也不能够信任。那当时很多人都说 737 再怎么烂七八糟,再怎么烂你还是要买,为什么呢?这个世界的民航是一个Duopoly,空中巴士占一半的市场,波音占一半的市场,那现在原本这个民航机是供不应求的,看一下他去年 2019 年空中巴士交了 550 架飞机,是拿到 800 架的订单,所以民航机市场在一年之前还是供不应求的。 OK 交机是 550 架,订单是 800 家。所以,当时一般的观察员都说,你原本两家都供不应求,你如果大家都不买波音要去买Airbus, Airbus 再怎么增产也是没有办法满足这个需求,所以波音无论如何还是会活下来。那现在最大的变数就是新冠,我们刚刚已经讲过这个新冠是二战之后最大的全球性灾难,它打击的除了我刚刚讲的石油业之外,其实更直接的是旅游业。现在不但是美国跟欧洲很多国家这里禁飞,那里禁飞,事实上也没有几个乘客愿意冒险到飞机场或者是飞机上面,因为飞机上面那个空气这样一直循环,很容易传播,非常容易传播这个病毒,所以大家都很理智地要去避免飞行。所以现在已经有一大堆的cancellation,就是波音在2月已经面临了很多的 cancellation 波音交机了,2月交机了 18 架,cancellation反而有 67 架。一个月cancellation 反而比交机要多出 3 倍。还有连Airbus 在两个礼拜前都宣布他准备要砍他们的产量,就是Airbus 现在已经要砍他的产量了。那这是什么意思?就是航空公司不需要去排队,你根本不需要等737了,原本大家是供不应求,现在你的那个需求一下子被腰斩,还不止,那反而是供过于求了,就只有 Airbus 一家也已经供过于求了,你需要这种单通道飞机的就直接去跟 Airbus 买 A320 就行了,对不对?所以这么一来,波音即使是有FAA的护航,,能够在4月5月复飞,然后准备要交机。





所以我认为,除了刚刚提到在节目一开始我提到的一些那个页岩油的厂商之外,另一个很可能会破产的知名企业,事实上是一个超大型的企业,就是波音在年底之前破产,至少有一半的机会。那当然美国政府不可能眼看着它消失,所以一定会像 2008 年去救通用那样子,由联邦政府接手。



王孟源 59:43 

所以波音的股票,我其实是两三年前跟你说我要谈波音的,结果昨天那个波音的股票跌了40\%,其实它一定会跌到 0 了,我应该说有大几率会跌到0。因为新冠疫情的影响,原本支持波音生意的最后一个因素,就是整个世界是一个duopoly,所以供不应求,大家没有选择,他的声誉再怎么烂,大家至少在未来两三年还是必须要跟他买啊。现在没有人要买,原本就没有人要买了。那你还有什么生意可以做?我顺便再提一下,中国的那个 C919 就是跟 737 竞争的那架机型,因为造大飞机实在是一个很复杂的工程,所以他一延再延,现在看来至少要延到 2022 年,就是两年之后才能开始交机,而且刚开始交机的第一年产量一定很低。那我呢,在去年讨论波音的时候,我就说一个解决的办法,可能是你用比它小一级的飞机来提供来做替代,部分的替代,然后用比它大一级的飞机来做部分替代来想办法弥补这个空洞。今天我刚好看到有新闻出来说中国商飞它目前只有一型飞机是在量产,就是ARJ21,它就是比这个 737Max 要小一号的飞机。他已经决定要建立第二个生产线,就是要把产量加倍,这是另外一个证据,就是说中国已经准备要想办法替代 737 Max,即使 737 Max 复飞,中国也不可能在大批的订购。而这些这种决定不是在新冠疫情出来以后才决定的,我们现在听到这个宣布都是几个月前就已经决定了,所以他们在新冠疫情出来之前就已经知道要替代波音的 737 ,而且已经停止大批购买波音的787 ,所以波音基本上生意原本就已经有 1/ 3 不见了,现在在新冠疫情出来连剩下的 2/ 3 也没有。



史东 01:01:52 

就一个基本的从一个 consumer 的角度来看这个事情。波音到今天,它的 737 Max 已经刚刚讲了种种的问题,到今天他的声名已经臭得非常非常厉害了,对不对?这个臭的声名再加上我相信航空业,全世界的航空业并不是一个很大的一群人,而是很 close net 的一个一群人。他的信息的往来大家都知道是怎么回事,对,对不对?所以说在这个前提之下,波音怎么可能在能够恢复到从前的光荣吗?不,我这个是明知故问呐,对不对?



王孟源 01:02:35 

是不可能的。所以它原本的所有的因素,你包括乘客,他们现在准备把 Boeing 737 Max 改名叫做 boring 737-800 什么的,OK,改名字,哈哈哈。但是这基本上是欺骗乘客,那航空公司也是无可选择才会买波音,但是现在他们可以不买。事实上2月的时候已经cancellation,现在新冠疫情在这个月一下子指数上升。那我相信你在这个月,跟下个月,到5月,那个取消订单的不会是只有五六十假,而是几百架的。这下波音基本上就是完蛋了。



史东 01:03:19 

那就雪崩了。



王孟源 01:03:22 

对。我过去这一年一直跟你讲,美国的经济像是一个山上雪积满了,随时都可能雪崩,我那时候说不晓得哪一片雪花会把它引起来,结果没想到是来了新冠疫情,这好像有一枚小型的原子弹轰炸。事实上你说的这些波音的名声臭了。我再跟你讲一件小的消息,就是在去年6月的时候,FAA还特别成立了一个专家小组来研究波音 737 Max 的事情。他们到了 10 月的时候,我把报告搞出来,这个叫做 JATR报告,你如果有兴趣的话, JATR报告,是由一个退休的NTSB 的commissioner 总管,来挂名叫做Hart ,所以叫做 Hart commission,来做这个所谓的JATR。 报告出来,你如果仔细看的话,都是说我们从此不能够让波音为所欲为,我们必须要加强监管,就基本上就是一一堆空话。这个空话出来以后,到了2月 25 号,有三个民主党的参议员,我要提醒大家,参议院是共和党的,这个所以你少数的党派没有什么用,但是他们还是出面说你这个 JATR 报告里面的建议太空洞了。光是这样子,我们不相信,必须要成立一个独立的监审机构,就是在 FAA 之外再成立一个独立的机构,专门来搞 aircraft certification,就是认证的事。就是他要把飞机认证的事情从 FAA 拔出来,交给一个新设立的独立机构,然后接下来众议院到3月 4 号有消息传出说他们正在讨论,说要有一个法案也是沿着这个思考路线。哇,你或许会问,这不是自己找自己的大公司的麻烦吗?我可以跟你讲,这同一篇报告里面就提出,这一篇报告就提出。 sponsor 是谁?是美国航空业的服务员工会。



史东 01:05:42 

你的意思就是说这是他们自己为自己的安全来做的事情?



王孟源 01:05:46 

对!所以你这个,你刚刚我为什么会想到提这个?因为你刚刚提到说波音的声名已经臭了,航空公司还是无可选择,因为他们必须要买飞机。但是你说这些飞行员,这些空乘员,他们是拿自己的命来开玩笑的,他们才不愿意这样。我并不是认为这些民主党的法案会真的会通过,因为事实上一旦波音说我的口袋都是空的,我一穷二白,我要破产了。他们不可能真的再去为难波音。到时候,我刚刚讲过,我预期联邦政府都要去 rescue 波音的,所以我不认为这个方案会通过,但是至少会闹一阵的原因就是因为,这些公会要求对波音施加压力,这基本上都是谈判的筹码,他们希望波音这次能够表现出诚意出来,然后让他们有点信心说这个飞机不会每两年掉一架。事实上在过去这一年爆出来的几十个坏消息里面,就有一个是说在 2018 年 10 月第一次印尼的失航的计算机 Max 坠机之后, FAA 内部的评估就说 MCAS 的每隔一年半到两年就会掉一家,他们没有想到是 5 个月掉一架。结果他们已经知道每一年半到两年会掉一架,他们还是两手一摊。



史东 01:07:17 

他们说这一年半到两年掉一架的原因是因为平时故障的问题还是其他问题?



王孟源 01:07:22 

他说那个因为那个软件本身设计就有很大的问题,有瑕疵,对我来讲,因为那个软件我刚刚举过那个详细的例子,就是 2009 年的那一次crash,还有这个 MCAS 的两次的坠机呢,它们都是同样的,就是它波音的飞机 737 上面都只有两个传感器的话,你就不能够两个都用。为什么呢?你如果有一个坏掉的话,电脑不晓得哪一个是对的,对不对?你如果要投票多数绝的话,你至少要有 3 个传感机。Airbus 的飞机都是有 3 个或甚至 4 个传感器,因为它是原本就是都是电传操控。 Airbus 的飞机从 30 多年前就全部都是电传操控,都是用电脑来帮飞行员来做决定的,所以它设计的时候传感器都至少有 3 个。波音没有这样子,波音没有这个电传操控的传统,所以它的传感器都是为了省钱,只有两个。那只有两个的时候我刚刚讲过,你就不能够让电脑都两个都看,它,一次只能看一个,所以你只要一个,那个你在看的那个如果出了毛病,电脑就会发疯了。 2009 年土耳其航空公司的那次坠机是这样子。



最近这两年的两次 737 Max 的坠机也是同样的原因,就是它的这是 737Max,是那个攻角传感器出了错,然后那个电脑认为他的攻角太大,有可能会失速,他就赶快把那个机头压下去。事实上攻角没有问题,你是平飞,但是他把机头压下去以后,你那个飞机就直接栽到地上去了,你说要不要再加一个攻角传感器?你要不要再加另外一个传感器?要加可以,但是你需要的时间是两三年,然后再加上重新认证又是两三年。因为飞机这种东西,刚刚我提过中国商飞的那个大飞机也是一直延,就是因为认证很难,因为这种东西,它是很复杂的机器,但你偏偏又要能够做到它的坠毁率,出事率非常非常的低,是 1/ 1000000,1/ 10000000 的出事几率。那这样一来你就这个认证还有品质管制就要非常非常的严。所以 Jack Welsh 搞的那一套, MBA 的那一套,你搞消费性的产品没有关系,你搞玩具,那个玩具,玩5 分钟就破掉,谁在乎对不对?你钱已经骗到了对不对?小孩子不高兴也就不高兴,下一次他们直接忘掉。飞机这种东西,你不能够搞那种利润至上,你利润至上,这样之后一定会牺牲品质管制。



史东 01:10:01 

所以你搞利润至上,今天的结果就是今天的这个样子。



王孟源 01:10:04 

对,你把消费性的,他们把美国商学院跟 MBA 搞这些利润至上的东西,先用在消费性产品上面,发现没有问题,诶,我们的利润真的是 10 倍了,但是他们没有想到说你不能够把这个用到重工业上面,尤其不能够用到飞机上面。火车汽车出了问题,你还只不过是抛锚弄到路边去,你这个飞机出了问题可不是没有办法随便乱到处降落的。我自己也是学过飞机,我自己是一个所谓的 general Aviation 通用飞行的飞行员,我有执照,我 140 小时的飞行经验,这个你上去之前要照理说要花半个小时,把你这个飞机检查一遍。我有一次飞机飞上去以后,发现右手边突然噪音大增,看出去以后结果发现那个有一片蒙皮掉了,就是机翼下面有一片蒙皮掉了,啵啵啵这样子,在风中这样子飘来飘去,我赶快紧急降落,后来发现是他们那个维修的时候把那个蒙皮的螺丝拿下来。





史东 01:11:11 

没有再装上去。



王孟源 01:11:13 

没有再装回去,所以我吹上去以后被风一吹它那个蒙皮就被吹掉了。这种事情不是我的错,因为我的飞行前检查没有包括那一块蒙皮。但是你说出这种事如果是出在汽车上面,有没有危险呢?没有什么危险,就弄点噪音,你停到路边去,然后看一看就好了。但是我那个时候吓得心脏都快要跳出来了。



史东 01:11:36 

有关于波音这个事情,我想因为就像任何一个,我前几个节目和播谈这个国民党,我,我在讲,我在说,我在说国民党。老实讲我也不是国民党的党员,我对国民党老实讲是一个老的品牌,我跟他从小在一起长大的,对不对?多少有点感情,你这个品牌慢慢的没有之后,我总是觉得有点惋惜。我想波音也是同样一个状况,这么久的公司,我刚刚在节目前看了一下, 103 年了,一个公司要维持 103 年真的是不简单呐。



王孟源 01:12:11 

对不对?美国最后的一个机械工业的明珠就这样子被搞毁了,是他们自己搞毁的,真的不是,对,不是竞争者把他们弄掉了。



史东 01:12:23 

我是他们,我这么样来做一个结尾。好,从波音的经验,孟源,我们应该学了什么样子的教训?任何一个,不管你是做生意也好,做人也好,或者是你从事任何一个事情。



王孟源 01:12:45 

我认为资本主义被美国推到这个程度太过头了,你即使是产业本身在这种利润至上的治理之下。

都会自我崩溃。那你更不要提,像是处理新冠疫情这种社会性的问题,我们人类追求全民利益的最大化,这个即使局限在经济上面,你都可以用波音做例子说这种利润至上的经理哲学是 self defeating,自我损坏损伤的。那更不要提超过一般经济的范畴,有一些危机、政治,还有与大自然的斗争,这样的事情你绝对不能够拿利用。像那个我这几年,其实我对中共的印象好转,有一个很大的原因,是因为他还有公有制,但是在过去这 20 年他们其实也开始搞私有制,我觉得已经是,就是被像 Jack Welsh 这样的人忽悠的。你看看像联想,中国的那个最大的 PC 公司联想,据说他们的那个创始人柳传志就是 Jack Wells 的崇拜者之一,对不对?那联想搞到现在,我不知道他们的利润怎么样?你追求利润的话,也许真的可以得到利润,但是他在产业的能力科技能力上,有没有为国家供做出什么贡献?有没有什么新的发明?对人类有什么?人类社会有什么新的发明创造?什么都没有,就是它完全没有那些附带的价值。世界上国家竞争,我们事实上还存在一个很不合理的世界格局,就是分成 100 多个国家,然后国家之间有很强的竞争。这原本就不是一个不合理,不是优化的一个局面。在这个局面下,你如果接受这是一个客观事实,然后想要对它来做政策的优化的话,你必须要追求工业化,你必须要追求科技的制高点。那这些事情,这些都是不能够靠追求利润来得到的,因为你追求利润最简单的方法就是Outsource,然后金融化。Jack Welsh 到掌管 GE 二十年,头十年是靠裁员,后十年是搞金融,把 GE capita搞出来,2008 年的金融危机爆炸的最厉害的公司之一,金融公司之一就是GE capital。你如果去跟着美国人搞这一套,就是饮鸩止渴,就是我刚刚说的self defeating。这个,我希望世界上其他的国家不要再受这些人的忽悠,他们这些理论提出来以后受到资本控制,媒体,所以那些像华尔街日报那些媒体都大肆吹捧,他们那些经理也一个比一个要热烈的。我刚刚提到像 Miulenberg 去追求股票价格,追求到连Calhoun 这种 GE出生的人都认为太过分。他们为什么要这样子?我刚刚讲过,因为你不管年薪是几千万,都比不上你那个股票期权几亿的赚要快。那基本上这些哲学,这些美国的经理哲学,是最适合政商富豪掠夺国家经济来自肥的,所以一个国家如果要照顾全民的福利的话,绝对不能够采纳这种政策。



史东 01:16:52 

问题在前提就是说这国家并没有把全民的福利放在第一优先。



王孟源 01:16:59 

就是你如果让,富豪,股票所有人,跟经理人的利益最大化的话,这些利益一定是来自全民,来自其他国民的利益,包括你的客户,包括你的员工,还有包括整个国家的国力,都是要为之而损耗的,所以波音其实是一面镜子,这已经是美国机械工业的最后一个明珠了。波音倒下去以后,美国还有什么?就是Intel,还有一些软件公司,这些其实就是纯粹消费性的,你像 Google 或者是Amazon,这是纯粹消费性的公司,这都是无根之草;真正的根基只剩下 Intel 一家,还有一些做半导体器材的公司。这样的公司,这样微弱的根基,能不能够撑得起一个20 万亿一年的GDP?绝对不够,你光是光是波音今年就影响 GDP 成长率1\%,所以 Trump 原本就很急,你要上个月,他原本还说要停止对商飞 C919 的引擎供应,就是我想就是因为拍了脑袋,人家跟他讲说波音这个 737 Max 的问题会影响这 GDP 1\%,所以我们要赶快打击他的竞争对手。其实不是他的竞争对手的问题,是波音自己上了岔路,上了邪路,你自己走火入魔,怎么能怪别人呢,对不对?当时他的官员就瞠目结舌,马上几个小时之后就泄露了,给那个媒体上就报道出来,然后当然 GE就是那个引擎的供应商马上去游说,然后游说的最重要一点就是商飞的那个飞机还要两年才会出来,也就是在大选之后才会出来,那你进来去拦它干什么?所以我觉得中美贸易战,还有这个波音的这件事情,出现的时机都很好。都是因为现在习近平还在掌权,他是有改革的决断的,如果能够提醒他的幕僚说,这都是错误示范,就是教导你不能够做什么,那么他们或许能够悬崖勒马,把一些私有制、私有化的那个进程扭转过来,因为真的有些工业你不能够走私有化,OK?中国最成功的高科技公司是华为,没有上市,华为要是上市的话,怎么能够做到今天这个地步?对不对?



我在最近的博客上有一个评论,我觉得要跟大家讲一讲,就是你最起码有三个产业不能够走资本主义自由市场,第一个是医院、医疗,第二个是法律,第三个是教育,这三个都应该是国有化。为什么呢?因为生医是有关人命,法律是有关公理,和教育是有关年轻的下一代的人的心智。这些东西都不应该在自由市场上让人家买卖,你不能够卖给的 highest bidder,对不对?就是你。



史东 01:20:40 

换句话说,不能变成一个 profit center,不能赚钱,一个赚钱的机构。



王孟源 01:20:45 

不能够让别人为了利润的考虑来做决定。对因为你人命,跟公理,跟下一代的心智这些太重要了,



史东 01:20:56 

这个比金钱更重要。



王孟源 01:20:58 

比利润重要太多。所以除了这些以外,其他像是高科技发展,你看中国要发展高科技,凡是让他们那个自由竞争的就都不成功。比如说私家轿车,凡是用国有企业去搞的,比如说高铁非常的成功, 10 年之内就超过原本的老师,对不对?你可以看出这并不代表说你坐了高铁以后有利润,他们并没有赚钱,但是他们对国家社会的贡献极大的,就是社会利益,社会收益是非常大的。这里的真正的矛盾,就在于股东的利益跟社会的利益往往是对立的。



史东 01:21:42 

这句话很好,这句话非常好。对。



王孟源 01:21:45 

考虑私有制跟公有制的时候,要考虑你是要对股东的利润最大化,还是要对社会的利益最大化,比如说你医院,美国的医院现在都已经私有化了,在过去这 5 年,我的私人医生退休了。为什么他做得不爽?他以前是个体户,现在他服务的那个医院变成私有的,他变成一个雇员,他不想干了。对,因为一切都是利润至上。对,公司发一篇公文说你必须要多开这种药,因为这个药的利润是百分之几千,他必须要做,因为他是雇员,他不愿意干这种事,对不对?在一个医院里面,医院的股东的利润最大化真的会是对病人的利益的最大化,一样吗?我认为不是。



史东 01:22:42 

对,这个不是大哉问,而是大哉思考。这个论点是非常非常重要。这我觉得在今天我们步入 21 世纪,我觉得人类要重新思考这些事情,因为从前我们被灌输了很多观念,现在慢慢发觉是不正确的。



王孟源 01:23:01 

英国跟美国,在过去 200 年是先后任的霸主,然后他们就拿这个霸主的地位,说我最成功,所以我的制度最好。我要提这有两个问题,就是你事实上是只有两个统计点,两个数据点,在统计上两个点是没有意义的,就是说你这个可能是一个巧合,可能是只是因为你的天赋特别好。像美国这些天赋不好吗?北美这个地方矿产丰富对不对?气候温和,地大人稀,然后有两洋的战略优势,你说他们的天赋不好?



王孟源 01:23:44 

所以这是第一点,第二点是他们得到霸权的过程中的体制其实跟现在是不一样的。英国在拿破仑战争时期就已经确立是世界的霸权,那是 1812 年,1815年就两次打败拿破仑。你知道那个时候英国的公民有投票权的占百分之多少?占3\%。那算是民主吗?美国的那个初选是由党员投票,这是什么时候出现的? 1960 年呐,美国是最晚,到二战就已经是霸权确立了,对不对?



王孟源 01:24:34 

他们得到霸权的时候的制度跟后来现在采用的制度完全是不一样的。尤其是美国现在的这个企业的哲学,这种经营哲学我说过了,是 70 年代、 80 年代 Jack Welsh 出来扭转的,在那之前不是这样的。在那之前 60 年代的时候,他们的的那个经营哲学是说一个企业是有社会责任的。对,他们那个讲的时候不是Stock holder,而是Stakeholder。Stakeholder就是除了员工、顾客之外,连整个社区的居民他们都有责任的。你现在看到的这个制度不是他们兴起的时代的制度,所以他们拿他们那个对外宣传的时候忽悠人家一个很简单的办法,就是说我们是世界霸主,所以我们的制度一定是最好的,你们都要来抄我的,这完全是骗人的。因为你们抄了我的制度之后就不可能赢过我,因为同样的制度,我是霸主,我是 ,你们不可能竞争得过我,对不对?你就不可能有更高的效率。然后你看像中国或者伊朗出了事情,这次出了疫情,他们一开始的时候处理,因为中国是因为对病毒的性质了解不清,不晓得它的传染力这么厉害,伊朗是完全没有那个科技能力来做基因分析,来做检测器。所以你根本你说不像美国是人谋不臧,不去检测,伊朗是根本没有那个能力检测。但是你这样子,1月的时候中国疫情扩散,2月的时候伊朗疫情扩散,你看欧美的那个媒体讲些什么?他讲说是他们的体制的问题,他们是独裁的,跟独裁有什么关系?伊朗的问题是他们穷,中国这个问题是因为病毒是从那边开始的,你刚开始的时候搞不清楚状况对不对?他们就是占了便宜,就说是他们自己的体制好,别人出了问题就说是别人体制坏,其实根本就不是那么回事。



史东 01:26:59 

其实在他们的观念里面,他们还真的相信他们自己所说的这些话。



王孟源 01:27:04 

对,这次欧美的他们的疫情处理会这么糟糕,一个很大的问题。就是他们把自己的民众忽悠了。所以那个像德国,他们没有去试图做隔离,就是因为他们认为我们是自由民主,我们不能够跟中国一样来做。那个 WHO拼命地要跟他们讲,他们真的是已经很委婉很客气,但是一再反复地讲,请你拜托,拜托你们去照抄中国的。他就在今天, who 还在讲说为时未晚,如果就所有的国家都能够采用采纳中国的政策,我们还是能够把它防治下来,但是他们不愿意,对不对?因为他们已经被自我忽悠了。



史东 01:27:54 

对,这个 2020 年我们都知道会是一个动荡的年代。但是在这个动荡里面的这种,我也希望,还是希望我们能够不断的成长,不断的健康。



王孟源 01:28:07 

我认为,当然这还不是大几率事件,但是就是有一个小几率,但是一个有限的不是无限小的几率,就是欧美会有一两个国家会因为社会秩序的崩溃而做出深刻的改革。当然这要试这个疫情到底弄到多大,如果他夏天就平缓过来的话,也许就每一个欧美国家就这样子的蒙混过关了,死了几十万人以后,然后就假装是没有发生过。



史东 01:28:43 

反正都是穷人,对不对?



王孟源 01:28:45 

都是穷人,对,反正对他们来说这些穷人只是领福利金的人,这是一个负担。对,就好像不把国民当作一种资产,当做一个你的主人,他们这个所谓的民主都是放在口头上骗人的。所谓的权利跟自由,一个最重要的人权就是生命权。你不把人家把老百姓的生命放在第一位的话,你讲还其他的都是空话。



史东 01:29:21 

对,现在我想意识到这些事情的全世界的民众越来越多了,我想经过这种 coronavirus 这种这样的事情,也是一点一点的或者一点一滴的教训,慢慢的老百姓可能看得更清楚了。我。



王孟源 01:29:44 

我过去在 6 年写博客,我希望,对世界的贡献,其实能够遏制那个大对撞机,那个虽然是大概千亿美元级别的浪费,但是如果能够挽救世界上,譬如说中国的人民或者其他第三世界人民,能够理解到,英美的这些宣传其实是忽悠,让大家比烂,你必须要用理智、理性的逻辑态度来分析什么是正确的政策,然后以全民的利益来作为第一考虑的、方向。能够有这样的理解,能够把这样的资讯传播出去,我想会是比大对撞机更大的贡献。所以我会不断地在我的博客跟任何一个机会,比如说在这里讨论这些事情。因为这真的是很重要的事情。



史东 01:31:00 

对,我一直觉得对一个人,在这个世界上的一个人最好的一句赞美词,是我在一本书上以后无意间看到的,但是给我印象非常的深刻,就是这个世界因为你的存在而变得更好。



王孟源 01:31:19 

我想我能做的贡献里面,这是最重要的,所以我会一直往这个方向努力。



史东 01:31:27 

对,是的这也是我个人努力的方向,所以说这个我们有志一同,我们互相鼓励。



\twocolumn[\begin{@twocolumnfalse}
\section{美国大选}
\subsection{20201022}
\end{@twocolumnfalse}]Credit: anonymous



史东 00:13 

各位朋友你好,我是史东在今天节目中为您请到的这位来宾。我想不用我多介绍,我想支持你期盼已久的一位来宾。那是谁?那就是王孟源王先生啊,王先生有好一段时间没和您见面了,也好一段时间没跟我见面了。那么当然了,透过您的很多询问呢,我也把您的询问都转给王先生,我说很多人想要看看王先生最近情况怎么样,你上个节目吧,王先生说我最近想谈一些美国大选的事情,我说好好好,谈美国大选,我陪你谈美国大选。现在你赶快把孟源兄请入我们的画面之中,孟源说一声谢谢,说声欢迎。



王孟源 00:53 

非常高兴跟大家见面。



史东 00:55 

这个我想大家看到你第一件事情我带他们说了。哇,你瘦了。



王孟源 01:01 

一直是去年夏天我回台湾的时候被妈妈骂了,那个时候真的是一辈子最胖的时候,到了 180 磅,妈妈骂了以后,回来以后就刚好那个时候我写了一篇博文,研究为什么最近糖尿病会这么流行,然后发现增胖的主要原因是糖而不是脂肪,所以我就 take my own medicine,遵从我自己的理论,然后在饮食中完全戒糖。然后每天运动 5- 10 分钟,我很快地就掉了 25 磅,就是一年下来就掉了 25 磅。最近受伤了,所以比较没办法再继续持续这个趋势。不过还好我最近,你也可以看到我这个是手脚都受伤了。这是因为手脚都受伤了。



史东 01:52 

你脚也受伤了。



王孟源 01:53 

哈哈,真的是,四肢倒有有 3 个受伤。



史东 01:56 

够你忙的。



王孟源 01:57 

这是因为我准备要搬家,所以要把房子整理好的话,美国这边要自己修理房子,很方便。你到 Home Depot 买一些建材都可以自己做。那像是我最近把那个整个房子外面的那个 deck 翻修了一遍。但是你身材搞好了以后,并不代表这些好看的肌肉就像年轻的时候一样强壮。 55 岁还是 55 岁,你不敢,肌肉中看不中用,做了一点体力活就开始拉伤。那我因为年轻的时候没有这些问题,所以一开始还不以为意,没有好好的保健,没有复建,所以后来就越来越严重。一直到最近两个月,实在是连很简单的事情都没办法做了,那我才开始照顾自己,开始让这些肌肉慢慢地恢复。



王孟源 02:59 

这是一个原因,所以就不太能够打字,因为刚好伤的地方就是手腕,然后我爸爸 2 个月前过世了。那我因为目前这个新冠疫情的关系,国际旅行不太方便,我就没有回去参加丧礼,目前就是在家里继续当寓公,过去 5 年离婚后的责任就是父兼母职,哈哈哈,因为我儿子上了大一了,那因为也同样也是新冠的关系,所以没办法到校园去上课。他是上网课,那上网课的话,其他的课他当然是自己能够应付,但是化学101,普通化学里面有量子力学的东西,他有需要我帮他辅导一下,所以我每天跟他辅导个一两个小时那样。



王孟源 04:01 

嗯,另外就是继续做阅读跟思考,我最近主要思考的问题是,我一直说人类 21 世纪最大的三个问题,是贫富不均、全球暖化跟霸权交替。那你如果看我的博客的话,你会知道,在这一次新冠疫情一出来,在2月或3月我就已经说霸权交替没有问题了,因为欧洲不会再跟着美国的屁股后跟前跟后的,像个哈巴狗一样。如果中国能够获得更大的发言权,美国即使像 Trump 这样的继续地阻挠解决全球暖化的议题的话,也不会有决定性的影响。光是中国跟欧洲加起来,就足够地推动全球来做出一些适当的反应。



王孟源 05:00 

那最后的一个问题就是贫富不均。过去 6 年来,每次谈到这个问题,我都说这个问题非常的大,没办法,没办法解决。但是我最近这几个月又重新思考了一遍,觉得还是有可能解决的,只不过是在未来两代人之内,不可能;要到 21 世纪的后半。如果我们这次今天讨论有时间的话,我可以谈谈这个问题,不过他就有点离题了,因为我们今天主要谈的是美国的大选,以及大选后结果会带来 2021 年跟 2022 年的影响。最后一个原因让我没有一直特别关注我的博客,是因为说真的,没有什么真正的新闻,几乎每件事我都讨论过了,就是过去两三年都讨论过了。然后你如果回去看两三年前我写的东西,完全都还是正确的,就是当时的预测。过去这几个月基本上只不过是印证我当年的讨论,所以就没有必要去重复,因为我一向不喜欢重复自己或重复别人的论调,这一次我会忽然想跟你谈美国大选这件事情,就是因为忽然终于有一个新的重大新闻,这个重大新闻是我过去这一年一直在等着发生的事情,也就是Hunter Biden 的email,就是上个礼拜换了 New York post 一个专题讨论说他们拿到了一些Hunter Biden的电脑里面的档案,然后里面有关于乌克兰的一些电子邮件。



王孟源 06:53 

为什么这个对我来说会是这么重要?会是从年初新冠开始到现在最重要的新闻。这样我们去年9月的时候我有上,我上你的节目,曾经讨论过全球的金融跟经济的议题,那时候我就谈,我就提到Trump特大选要连任,它的前提是必须要经济搞好,就是 2020 年下半的经济态势必须是好,但是一年前我们已经可以看出他的这个他对美国的经济过度刺激。



王孟源 07:30 

2017 年的减税,然后到其后的一直,他一直逼迫那个美联储去消减利率,这个都是打兴奋剂,这种兴奋剂的短期效果通常就是 6- 9 个月,那他做得太早,所以到了 2019 年我已经我想我用的比喻是一个雪山上堆满了积雪,随时都可以雪崩,对不对?既然Trump的连任必须要依赖这个经济的态势,而我们在一年多前就可以看出美国在 2020 年下半的经济不可能是顺利的,那么是不是就可以预言Trump不可能连任?那其实我已经在过去这一年多一直讲 Trump 连任的希望不大,就是因为我刚刚分析了这个道理,但是我一直没有说他绝对选不上,为什么呢?因为你必须要考虑 10 月惊奇的问题,就是搞选举这种东西搞到现在基本上各种污招滥招太多了,然后在选举前的两个礼拜,尤其是会有莫名其妙的丑闻跳出来,然后大选一过,大家就忘光了。



王孟源 08:58 

那像这种 10 月惊奇呢?我们不可能事先越靠逻辑分析和预测,虽然美国的民众已经被玩了这么玩,弄了这么多次,你可以你可以说他们可能有点免疫力,但是事实上你不能确定,对不对?所以必须要等这个 10 月惊奇真正发生之后,你才能够确定这个大选的方向,然后从大选的结果你才能够预测未来政策的走向。那从政策的走向你能预测国际事务的真正发展?所以 Hunter Biden 这个 email 就是我等了一年的这个 10 月惊奇。



王孟源 09:38 

哈哈哈,那他一出来之后我就有话题可以谈了,新的话题可以谈了。因为过去这一年每次谈到这个大选的话题,我都说我不看好Trump,但是还不能确定。那现在可以确定了,现在可以确定就可以,就可以来跟你谈。那首先我们先谈谈这个 Hunter Biden 的这个 email 到底是怎么回事。我自己去看了一下,就是Hunter Biden把一个 MacBook 送到电脑修理店去修,然后他事后就忘记了,因为像他这种纨绔子弟,钱多得很呐,换一个 apple 电脑一两千块根本不算是吧。



王孟源 10:21 

第二天那个 Cocaine 效应一过,他根本就忘记,照理说这个这种 10 月惊奇出现的时机太巧。OK,就是话题太巧,时机也太巧。你第一个就是必须要打个问号说怎么会这么巧呢?OK,你像我都已经等了一年多,要看这个10 月惊奇到底是怎么回事。那你可以知道你不能够他一登出来,这个报道一登出来,你就照字面全部接受,但是你如果去看那个细节,样样都是合理的。而且对我来说最有说服力的一点就是这个证据太弱了,根本就不够看。你看看它里面真正公开的就是两封信,一封信。是啊,有个乌克兰的商人跟Hunter Biden说谢谢你昨天介绍我跟你父亲见面,OK,然后另外一封信是同一个商人隔了一段时间说,我现在正在受到乌克兰其他政派的迫害,他们要跟我要钱,要权。所以请你设法,或许通过你爸爸来减轻这些压力。OK?像这样的证据,即使它是百分之百是真的。在美国违法吗?不违法,太弱了。如果这个是假造出来的话的那个假造的人太笨,所以正是因为他太弱了,所以我觉得他很可能是真的。



王孟源 12:03 

大概 5 年前, 2015 年底写过一个博文讨论,当时是伊利诺州州长Blagojevich贪污被判刑,然后当时他上诉到州的最高法庭,然后当时的就减免了好几项罪名,就是贪污的罪名,当时是他们州的伊利诺州的最高法庭说,你政客收钱没有关系,只要检察官不能够证明说你这个收钱不是 conditional on 某个特定的事情就不算行贿。你说这离不离谱?你那个检察官能够抓到政客私底下收钱已经很了不起了,但是这样还不够定罪,那个你必须要能够证明他收这个钱是要办某个特定的。是,而这个事情是不应该收钱的。



史东 13:07 

就是有没有银货两讫了。 Quid pro quo。对。



王孟源 13:12 

这是他们的Quid pro quo,对,就是除了你收钱之外还要交货。哈哈,对,不是,你知道他们用的理由是什么吗?理由是说政客跟商人收钱是民主运作的正常步骤。你说这可不可笑,我那篇文章发表之后不到半年又出了一个很类似的案例,但我后来没有写。是,这是在 2016 年中, Virginia 的周长被 FBI 也是同样抓了正着,这一次是联邦的,这不是周的,这一次是被联邦抓了个正着,所以是在被联邦政府起诉。联邦法庭起诉有一个卖维他命跟药品的老板给他送了 13 万块钱,包括劳力士手表等等的。



王孟源 14:05 

这在世界上任何一个国家,你一个政客莫名其妙地收了一个劳力士手表,还有一些现金,还有房地产都是非法的。但是在美国,它上诉到美国最高法院,真的就上述到美国最高法院,这个是联邦最高法院,不是州法院了。所以这个代表着全国都必须要照着这个判例来判决,他同样的完全无罪,因为检察官没有办法证明他拿了这个钱是要办什么事情,他们只能证明他的拿了钱,他没办法证明他拿了钱是要必须要办什么事情,就这样子就是无罪了?那你既然这样子真是基本不可能,不可能有人定罪了,对不对?那你想想看,它的标准这么高。



王孟源 14:58 

Hunter Biden 的这两封 email 哪里算是什么 smoking Gun呢?这基本上就是像他们这个美国联邦最高法院也是同样说的,民主政治的正常运作。所以我的逻辑就正是这个,这个不像是无中生有,如果要无中生有,不会生出……不会花了那么大的麻烦,还生出这么弱的指控。



史东 15:29 

那现在照你的思维讲讲,如果是不是无中生有,而是有中生有的话,嗯,这个有中生有的情况对于这个 Joe Biden 的选情会有多大影响?



王孟源 15:44 

Good question. Almost zero 事实上这种丑闻原本的用意就是要把它闹出来,然后闹个两礼拜,到大选用后就丢了。这个实际上没有什么能够起诉的东西。大家都知道,美国这个法律都是为了有钱有势的人而定的,所以根本不可能真正地定罪。那你看 2016 年其实也有一个同样的十月惊喜,就是当时 FBI 局长他并不是共和党那边的,但是他因为没有想清楚这个政治的implication,就是他不小心在选举前的一个月,忽然又把 Hillary 的 email 给就是重新的主动去侦查。他以为这个就是侦查,侦查不代表定罪对不对?但是从政治方面来说,你刚好在选举前一个月被右翼的媒体有心地去吹嘘了一番,对 2016 年的选举是有很大的影响的。事后你看, Trump 已经讲了 lock her up,讲了 5 年了,有没有任何prosecution?没有。不可能。因为事实上 Hillary 搞这个email,这当然是违规的。但是有权有势的人违规,在美国绝对可以大事化小,小事化无,不需要以怎怎么隐藏。它唯一的后果是政治上的,就是在选举上的有后果。



王孟源 17:26 

当初 Hillary 会想要去违规,把那个它的 email server 放到自己的伺服器上面。自己家里的伺服器上面原本就是怕人家无中生有,或者是断章取义,无限上纲。她如果把那个 email 留在国务院的释伏器上面,就有可能被泄露,被共和党人泄露出去,然后拿来用来做政治炒作,这媒体炒作,没想到她原本的用意是怕被对方炒作,而这个动作反而被炒作成是在掩饰罪行。所以这个是有点 irony 就是反,有点反讽。民主党哪一方面本身就是搞这种无中生有或者是欲加之罪,何患无辞,断章取义,无限上纲这一方面的专家,你看过去这 4 年这个 Russian gate 就是通俄门搞了半天,其实很简单,就是绝大多数都是无中生有,所以那个国安顾问被定罪是因为他在一开始被调查的时候撒谎,所以你看这个跟 Hillary 的 email gate 完全都是一样,就是你的这个动作本身没有错没有问题,至少不是犯法的,你去遮掩它的时候反而是犯了法。



王孟源 18:55 

那为什么会变成这种扭曲的现象?就是因为媒体的势力太大,媒体能够把小势变化大,或者是大势化小,看他们是不是你这一派的。而美国大部分的主流媒体的确是站在民主党那边,美国的选民过去几十年来他最受不了的有两点,第一点是你这些金融精英跟政治精英的搞全球化,搞了以后你这些蓝领阶级的工作都不见了,都外包去了。但是你看那个主流的媒体,他始终都在吹嘘这个全球化有多么多么的好,这个经济的走向有多么多么好,他们觉得不对劲,他们或许不知道是怎么被screw,但是他们知道他们被screw,那另外一个就是政客的作为,偷鸡摸狗利益输送。但是他们看到至少在主流媒体里面,共和党员往往是被定在十字架上,而民主党人就比较容易蒙混过关,尤其是民主党里面真正有实权的。就是因为这样子,所以Trump才会经济,才会有基本盘,他的那 40\% 的基本盘就是这些受害的民众。他们不懂问题真正出在哪里,他们只知道出了问题了,他们不信任政经精英,他们也不信任主流媒体。在这样的背景下, Trump 才会在 2016 年出人意外地一枝独秀,这样的站出来。



王孟源 20:35 

为什么呢?因为Trump本身他同样也不了解这个问题的根源跟脉络,他只是感觉到他的观众愿意听这些话,所以他就去讲这些话,做这些事情,让他的听众感觉到他好像是在解决问题,实际上他不懂这个问题。他的认知完全是间接地从他的选民那边选民的不满而来的。他的认知,他认知到的是选民的不满,而选民的认知是系统的不公平。但是他们不并不了解不公平在哪里,问题在哪里,那更不可能了解这个问题的解决,正确的解决方案在哪里。



王孟源 21:19 

但是话说回来,这个 Hunter Biden 的 email 这么弱,如果主流媒体都跟在后面吹的话,还是会有效的,因为现在这个Trump基本盘还是非常的坚硬。它的问题是在过去这一年先是有了黑命贵的那个activity,我有一篇专文讨论这个,为什么这个这一次的这个黑命贵示威运动会比 4 年前, 6 年前,其实光是过去这 10 年就有几十次类似的事件,就是黑人在街上当场被警察无端杀死的案件,事后也都有示威抗议。



王孟源 22:06 

但是今年这次是第一次上升到国家会想要做一些事情,为什么呢?因为这是选举年,这原本就是民主党设下的局,民主党设下了几十个可以用来攻击 Trump 的口实,那这个呢?刚好这么一发生就成为一个很方便的武器,那也真正的发生了作用。所以  Trump 的支持率一直都是在 40\% 几,就是它有 40\% 的基本盘,然后再加上 5\% 左右的中间派,来自中间派的支持,所以它有45\%,跟 Biden差不了多少。



王孟源 22:52 

Biden就是一直都是 40 多,大概48~ 49,一直到4月都还是这个样,到了4月因为黑命贵这个东西,这个才有了变化,那Trump的民调才会一直掉到 40\% 左右,然后 Biden 升到 50\% 以上。到了6月,新冠疫情一下子爆发,而且爆发是在那个美国内地的农业地带,所以Trump这次这个选举的态势全部面临了三个大的逆风。第一个是经济,当然是不行了,这个因为也不但他们,我原本一年前就可以看出他这个美国的经济已经强是强弩之末,再加上这个新冠疫情,那真是更是雪上加霜,不可能可能为现任总统加成。



王孟源 23:45 

第二个就是黑命贵这件事情,所以 Trump 一直在讨好那个 Neo Nazis,极端右派,这个真的是一直被中间选民睁一只眼闭一只眼,因为他们毕竟是中间派,对不对?他们不太在乎这些议题。但是经过多年的炒作,他们终于成为一个道德议题。民主党在从 2016 年开始注意到Trump的这个弱点,他逐步地慢慢地洗脑,说这真的是一个道德的问题,如果不支持我们就是不道德,那这个真的是有效的,真的是有效的。然后最后一个就是这个新冠疫情,因为到了6月的时候,欧洲已经控制下来了,结果美国突然爆发,而且爆发的一开始4月爆发,3月,4月爆发的是纽约这样的国际州就是民主党周。但那当时那个红脖子们还觉得兴高采烈,但是一旦到了6月真正爆发,那疫情传道这些红州就是内陆的农业州。



王孟源 24:58 

Trump 就不再有任何借口,尤其是跟欧洲一对比,你一开始说那个中国或者是南韩或者台湾,你还可以说他们这些东亚国家,他们有所谓威权局势,可以搞绝对隔离,但是你看德国就控制得很好,那这下一来美国人就没有借口了。所以就是这三重的宣传打击下, Trump 目前大概就是 low forties 41\%,对 Biden 的52\%,就是它到9月底 10 月初开始它的落后超过了10\%,那目前最新的民调是落后11\%。



史东 25:45 

左右。讲到这我插一句话,孟源,嗯,你刚刚讲到这个新冠的事情,嗯,我想了解一下你对这个 Trump 得新冠这件事情,你的看法是这样。



王孟源 25:56 

又是很反讽的,他活该。但是对中间派选民来说,对他的基本盘10\%,就是他也就是在医院里面待了 3 天,然后就跳出来一样,活蹦乱跳,到处乱爬等等。对他们来说,OK,他说这个新冠没什么大不了的。的确对他是没什么大不了的。当然,但是他是总统,什么什么药什么最新的医疗手段(都能用上)。



史东 26:20 

我想问这个问题,特别是因为你也是新冠,曾经是新冠的患者。



王孟源 26:27 

我自己没有经历过什么严重的那个病症,所以我其实不算,因为他至少是要住院了,他不但发烧,而且他的那个血氧量降低到93\%。对,那是蛮严重的,他又是 74 岁,过度肥胖症,我觉得不能比了。不,这个这件事情他得病真正的真正问题是在中间选民看来是很尴尬。我其实不是特别喜欢讨论民主选举的问题,因为这大部分的选民都是这样的笨,你讨论的那个话题都是如此,哈哈,低级,哈哈哈。真正影响他们的是这种……



史东 27:19 

我们改一天找一个时间,真正……我真的是,我现在真的是很想谈一谈民主制度这个事情,它的存在价值是什么?我现在越来越怀疑它的存在价值。我们改天再谈这个事情,因为这是个大题目好不好?我,我想就你谈到现在为止,我想我们听你讲你已经几乎可以作为一个结论,就是说王孟源她认为这一次大选拜登胜的胜选的机遇率是非常非常大的。



王孟源 27:52 

这个目前的那个权威的分析是 80\% 多,就是Trump还有 1/ 8 胜选的机会。



史东 28:00 

我现在想谈一下,就是说你认为这个 Biden 当选之后,这个世界、美国这个世界还有美国和中国之间的关系,嗯?会是一个什么样的状况?



王孟源 28:16 

OK,我之所以会等到 Hunter Biden 的这个 10 月惊奇出来以后才肯谈这个问题。因为虽然Trump连任的机会一直不大,但是一直到两三个月前,民主党拿下参议院的机会还是少于50\%,一直到两个月前,他的这个机会才达到一半。随着从9月到 10 月放的民调又掉了2\%,这个民主党拿下参议院的机会现在就是根据这些权威机构已经提升到 3/ 4 了,我个人认为是更高了大概 80\% 多。



王孟源 29:00 

这代表什么意思?就是 Biden 执政的头两年会是全面执政,就是民主党会控制总统,参院跟众院,三个地方全部控制,所以基本上他们可以为所欲为到什么地步?目前炒得很厉害的就是 Trump 一连任命了三个最高法院的大法官,那这下一来就会把他们原本偏红跟偏蓝的比率从 5: 4 调整到 6: 3,那一旦有了 6: 3 的,那他们就不必担心之中出一个叛徒,那这样一来他们就有可能去推翻 Roe vs Wade,但是更立即的一定会想办法推翻的是 Obama care。



王孟源 29:48 

那美国这个事情很奇怪,就是表面上看来大家都讲理,实际上说到底都是政治斗争,而他们还认为这样,这代表着文明就是一切都是虚伪的。像这样子的国家,你怎么能够当到世界超强呢?两件事情,一个是他真的运气很好,欧洲一连打了几次大战把自己打垮,那另外一个就是 20 世纪初,尤其是小罗斯福的改革,真的是曾经建立过一个廉洁高效的行政体制,所以就是这两件事让他成为世界霸主,成为前所未有的成功国家。但是从70年代开始他就开始腐化。



王孟源 30:36 

其实这个跟我刚才提过的有关那个贫富不均的讨论也有关系,不过我扯得太远了,我们还是现在还在谈这个大选的事情,我们还是先把这件事情讲清楚。目前一个很重要的议题就是即使 Trump 落选了之后,美国大法院的法官还是 6: 3,然后接下去他们有能力推翻Obamacare,就是不管你,即使你在立法继续推进或者在确认Obamacare,最后的最终还是取决于大法院,说这是不是危险?偏偏美国的体制就是这个所谓的宪法的解释,拥有最终的决定权。



王孟源 31:22 

你想看总统签署两院通过的法案,你这个大法官会议高兴否决就否决,那这算是什么?民主不当前?这个大选的问题就是除了Trump这个总统的选举之外,还有参议院的选举,然后除了参议院选举以外,还有大法官现在在任命的这个大法官。那问题是这个大法官共和党已经铁了心了,就是要把它 INS 通过,而且应该是在下个月。很可能,如果可能的话,我想他们是希望在大选前通过了。不过即使是在大选后,反正交接还有 2 个多月,他们也是有足够的时间把它塞进去。那这样一下来怎么办?美国大法官是终身制的诶,你如果控制了参阅美国的宪法,没有规定大法官的人数。其实上个世纪罗斯普要搞改革的时候,那时候共和党的的那个大法官也是站出来反对,罗斯普就威胁着要增派大法官。宪法里面讲得很清楚,新任大法官是由总统提名的,所以我如果从 9 个人增加到 15 个人,那是另外 6 个都是我提名的。所以这个现在他们在讨论的这个问题,这件事情其实并不是新的发明,而是当年在 1930 年代已经已经是一个议题。



史东 32:59 

这不是基本上是一个变相的修宪嘛。



王孟源 33:05 

问题是宪法没有规定大法官有几个人。



史东 33:06 

这个我知道,但是就是这个就有……当然我这个是狗吠火车了。这是不是有一个道德层面上的问题呢?



王孟源 33:16 

Well,你经过共和党人 4 年的这样的胡搞,我想美国再也没有什么 norm 就是常规。你事实上你看英国跟美国都是所谓的民主体制,什么宪法,这个其实都只是形式上的,他事实上很多细节都是靠 norm  常规就是不成文的法,常规来规制。那过去从 2016 年到现在, Trump 最大的贡献就是把美国所有政治常规基本上都全都打烂。同样的英国的那个 Boris Johnson 比Trump的效率还要高,他只当做首相一年也是一样全部都打烂。



史东 34:01 

他,现在不是抱怨他,都是他的薪水不够。对,不能过日子。



王孟源 34:09 

像你看这种海洋法系的国家,它原本就不是像大陆法系那样子,法条非常的详细,什么事情?然后法律没有规定非法的理论上就是合法,那是大陆法系、海洋法系,是你有什么事情找法院去问,那法院就必须要遵照前例,你要违反前例,可以,但是人家一定会上诉,法庭可以决定是不是可以推翻前例。



王孟源 34:41 

那如果上诉法庭也想要推翻前例,那么人家就可以再上诉到最高法庭,看最高法庭怎么办?他们的这个逻辑是这个样子:我现在讲一个故事,这方面的一个故事。在我们刚刚提到那个黑命贵的事件的时候,我写了一篇博文,详细解释为什么美国的警察会腐烂到这个样子。我提到在里面提到一般非美国的人不了解警察,还有一个特权叫做 qualified immunity。什么是 qualified immunity?就是违反人权,对警察来说是可以的,免责的。可以这么选择的。对,只要法律没有实际的前例,定为非法,只要没有前例就可以免责。那你看看,这是很有趣的,你怎么创造前例?你必须要定罪,但你怎么定罪?必须要有前例。这就是基本上你无法再用违反人权这件事情来起诉警察。这个在美国叫做 catch twenty two,因为有一本很有名的书。



王孟源 35:56 

一个实际的案例是这样的,这个有一个,他们警察要去逮捕一个通缉犯,跑到他前任女友的家里去,而前任女友说,我已经好几个月没有见过他了,他绝对不在我家。那警察说,但是你同不同意让我们进来搜索?他说我同意,结果警察怎么办?他的火力全开,手榴弹什么的丢进那个小房子里面去,把那个房子炸得稀巴烂。然后事后那个女人说,我只同意让你们进去搜索,我没有同意你把这个房子炸烂,你至少要赔我的房子。不赔,为什么?不违反财产权,不违反自由权,因为没有前例。为什么没有前例?因为 qualified immunity。



史东 36:51 

这是在哪一州在哪里发生的事情?



王孟源 36:55 

那好像就在加州嘛。你们加州?



史东 36:57 

我真的。



王孟源 37:01 

反正任何人有兴趣的话去 Google 一下 qualify immunity,这不是我发明的。这种事情你随便Google 一下,有几百个案例可以看,事实真相本身就有很多幽默,你不需要再去发明。真的是。



史东 37:17 

对,我同意。所以你讲到美国国会之间的这个态势。



王孟源 37:23 

所以我现在讲的是这个美国是海洋法系,所以它这基本上是谁有钱说了算。它跟这个,跟东亚这种儒家社会的差别在于它不是统一的,不是一个理性的金字塔型的,而是在金字塔底顶端是一个Oligach,是一个多头的,所以这是唯一的差别。事实上因人而治完全是一样的。你跟中国古代的那种封建时代的那个社会完全一样,就是人治,而不是法治,只是表皮。史东:(对,这个我同意)所以我刚刚已经提过有关这个大法官的事情,那我个人认为医疗、教育跟法律之前的都是应该人人平等。对,因为我们必需要有市场经济,你必须要容许贫富差距才能够有动力让经济发展下去。那但是你至少要给予公平的机会,Equal access 平等的。



史东 38:46 

对这个是我在小学,我相信你在小学,不是小学,中学的时候也读到有关于这个孙中山的三民主义的,里面的一种所谓的平等的思维,就是孙中山他所说的平等,是立足点的平等,而不是齐头式的平等。



王孟源 39:05 

对,就是基本上就是同一个哲学。对,那实践上我认为有三大方面特别重要。第一个是法律,你不能够发展成美国的这个法律界,这样子是有钱人能够基本上为所为。对,反正。



史东 39:22 

你看 Biden 的这个,Hunter Biden 的这个,这个披露的就是为所欲为。



王孟源 39:29 

他就是为所欲为,因为事实上他们都是这样。而到最后法官也干脆就说,最高法院也干脆就说,这就是我们的常态。所以你不能够定罪,基本上因为法条没有明定,所以基本都是看哪一个人能够请更好的律师,然后有更好的关系,来争辩,来找前例。因为这些前例本身太多了,你那个你如果不是一个一个客观的法条在那边,而是几千万个前例让你随便找,你一定可以找到里面的矛盾。



史东 40:11 

对,那我打一个岔,我打一个岔,我想到这个AI,很多人提到 AI 这个事情会让律师这个行业一做很大的改变,因为你刚刚讲了这个案例的研究,嗯,是一个很花精神的对一种工作。对, AI 出来之后,嗯,就完全是经过 AI 来搞的话,这个就很能,这改变就会很大了。



王孟源 40:37 

因为你有几千万个前例,绝对是里面有前例。



史东 40:41 

有漏洞。



王孟源 40:42 

对,有漏洞,反正你就挑出那些矛盾来。但是只有有钱人才雇得起律师,花上几百个小时。



史东 40:51 

几千这个才是重点,这个才是重点,小老百姓根本就是知,即使知道也没办法做得到。



王孟源 40:58 

对,所以这种所谓的海洋法系根本就是为了有钱人而设计的,就是为了有权有势的寡头和设计的。中国至少没有搞成这个样子,台湾有点这个趋势,哈哈哈。那至于教育跟医疗,你说人命是不是这次这个Trump得了新冠,他所受的医疗的待遇跟一般人一样吗?完全不是同一个待遇



史东 41:25 

这是一个教训。



王孟源 41:27 

对,所以你不能够这样,那个医疗私有化之后,私有化市场经济的第一优先一定是利润,利润一定是靠剥削,可以剥削的对象,那可以剥削的对象是什么?弱者,对,那同样的教育基本上是唯一保证社会垂直流动的一个管道。你如果私有化以后,如果这些高价的私有学校不能够提供更更好地教育,社会上的上升的管道,也就是升学的管道根本没人会去。那你为什么要私有化?如果他能够提供更好的升学机会,那你就是在创造贫富不均来固化阶级,那你为什么要这样做?对不对?那基本上是聪明的贫家子弟也没办法竞争得过平庸的富家子弟,就像美国这个制度一样,对不对?美国的制度在小罗斯福之后,它的改革不只是政府,而是整个社会都开始追求公平。



王孟源 42:46 

从 1930 年代一直到 1970 年代, 40 年之间,美国公平化就在那段时间全,那是全面的,他的最高教育,即使像哈佛、耶鲁这些,他们也都设法改变以往专收富家子弟的办法,到最后多到有 1/ 3 是平家子弟。但是从 8 零年代开始,他们就有一个反动。现在这个哈佛跟耶鲁的这个收学生的标准越来越像, 19 世纪就基本上都是富家子弟进去。



史东 43:22 

这边你还可以加一个就是对于亚裔学子的态度。



王孟源 43:25 

亚裔,你即使是有钱也是弱势的。因为种族的关系,当初犹太人在 20 世纪初期也一样,因为他们是第一波重视这个学术研究的族群,他们在 20 世纪初期经过了我们现在亚裔学子同样的歧视,后来的改变除了他们逐渐地收买了美国的政经精英,而且参与了美国的政经精英之外,还有一点就是经过小罗斯福的改革之后,美国的社会的确有反思,有 40 年的反思开始减轻这种不合理的现象,开始要把这个教育机会拓广到给社会的全面各个阶级。



王孟源 44:16 

我之所以会想要顺便提一下这个,就是中国在过去这 20 年,学术界被美国式的思维渗透得太厉害,以至于连他们的中央政府,就是国务院的教育部,还有负责医疗的也开始搞私有化,这真的是很大的倒退,也很大的反复。你绝对没有资格把自己叫做共产党。如果你连在医疗跟教育上也开始搞私有化,也基本上就是独立富人,这是非常恶劣。



史东 44:56 

对,其实我看这个事情难道不是就是美国的用意吗?他花了这么大的精力来教育你这些来自从中国来的这些人,把你再送回中国,他的目的不就是在这吗?



王孟源 45:10 

对,这为什么要这样子?因为列宁曾经很相信,说全世界的社会主义者或者无产阶级会 unite 会联合起来,事实上那是非常困难的。因为 identity politics 是更强的instinct,就是国籍,对国籍族群的认同跟种族的认同,绝对是对那种下层阶级的普罗大众更有吸引力。真正很容易联合起来的是资本。国际上你不管是哪一国的巨富,他们的利益基本上都是共通的,(史东:对,一拍即合)。而且我们一谈这个他们的利害关系都是几十亿、几百亿的事情,在这种背景之下,他们根本不在乎什么种族国家,对不对?其他都可以看过,最重要的是保护我的几百亿。所以中国的中央政府被美国美式思想渗透到这个地步是非常不幸的。



王孟源 46:17 

我自己在过去这几年也是多次批评这个方向,其实这一次这个新冠造成全对全球经济都非常大的影响,中国算是承受力最好的,就是恢复得也最快。但是我觉得这个新冠如果要说它最重要的教训是什么?它最重要的教训是必须要尊重科学,而且有健全、全面、公平等的医疗体系。



王孟源 46:50 

那中国在后最后这一项就没有做得很好,那你要利用公共花费来刺激经济的话,其实投资到公共医疗体系是效率最高的,尤其是在星光还没有消退的的背景之下。我不懂他们为什么没有再进一步地去做。当然在大方向上面,他们没有错到像美国或者欧洲那样的地步,但是在旁边看着他们没有最优化,总是让我有点失望。



史东 47:25 

你是你认为他们应该如何的?像你建议是这样子,在公共卫生方面加大一些他们的挹注。



王孟源 47:37 

全世界凡是市场经济的国家都是医生,都是高收入所得,所以医学院都是大家抢破头的。但是中国是例外,中国的医生收入非常的低,但是他工作没有比美国的医生轻松,美国的医生随便一下子就是熬两三天夜值班的,这个是非常非常的累的。那你所以你第一个就是提高他们的社会地位,并且是提高他们的收入,获得这个收入必须是要给第一线的工作人员,而不是给行政人员。你如果大家都抢破头去当医院的院长或者去行政人员,那你这个纯粹就是官僚嘛。你这个一个他们这样子比较中央集权的体制,最怕的就是官僚体制过于庞大低下。



王孟源 48:45 

你要刺激经济,不是刚好就应该提高他们的薪水嘛?那提高薪水最好的对象是什么?就是这些医生,因为中国真的是对医生的付出不够,中国社会对医疗整个的投资不够。而且我说的不只是construction,就是建设,而且而是operation,就是日常的消费,你这些医生跟护士他们的工作那么辛苦,那么重要,你如果不给他合适一点的薪水的话,你怎么样吸引第一流的人?不可能,现在大家,中国的社会,大家都抢着去玩政治嘛?



史东 49:33 

现在还有一个现象,我不知道中国现在当下的情况怎么样,因为我前一阵听说有很多的这个,因为可能是你刚刚谈到的这个收入的问题,有很多这个医商勾结的进这个状况。



王孟源 49:49 

对,这也是从美国弄过去的,你如果对认识美国的医生的话,你可以去有朋友是在美国医界的话,你可以问问美国发明的这一套。OK,就是医学界用收买政客的手段去收买医生。对,来让他们去开药。



史东 50:14 

这是在美国很平常的,这个在美国不,如果说打医生不这么做的话,都已经是很古怪的人了。



王孟源 50:22 

因为医生虽然在美国是高收入的,但是 20 万 30 万一年的收入,工作很累。你如果跟那些商界政界的精英相比,那个钱来得实在是的太辛苦太痛苦了。对,太辛苦,哈哈哈,你要多赚钱的话,最简单的就是收一些这些红包。



史东 50:44 

对,因为钱是永远不够,也可以再多一点的,对不对?



王孟源 50:49 

所以我觉得中国行政一个很大的问题就是过去这 20 年被美国的思想腐化的很严重。从内部腐化,我刚刚提过中国的国际态势其实是很乐观的,就是霸权交替。因为欧洲在Trump的之下学乖了,他知道不能够再继续当美国的看门狗。那所以这就留给中国崛起的空间,但是所以中国真正的问题在于内部的组织,那中国内部组织还有很多不合理的地方,那这些不合理的地方你如果不趁早解决,就越来越难解决。很多事情你在经济开发程度还没有很高的时候,解决的代价都是还可以接受的。



史东 51:51 

对,我想任何改革也是这样子。其实套你的思路来讲,一旦你尾大不掉,然后成了一个庞大的既得利益集团,那就很难弄了。



王孟源 52:01 

连你原本想要照顾的那些民众一旦都成为中产阶级以后,他也成为一个利益集团。对,反而会分成一个一个的山头来照顾自己的利益。比如说你在美国要盖一个高铁怎么这么困难?因为你没办法征地嘛,对不对?你为什么 70 年前要盖高速公路那么容易?因为当时美国刚刚从二战跟那个大萧条复苏起来,他们还有那个条件做这些改革还可以,简单的就征地。政府说我要征地就征地,OK,民众不会那个当钉子户。中国到现在连钉子户的这个解决方案都还没有统一立法,真的是很奇怪的。嗯,经济的发展程度避免人家落后 1 百年,但是人家 1950 年代就已经简单解决的事情,你现在还没有解决。



王孟源 53:02 

无论如何,言归正传了,我们还是谈一谈美国大选。okay。因为 Hunter Biden 的这个 email 出来以后,第一个是它本身就是很弱,第二个是美国的主流媒体一方面尽力封杀,如果非要谈不可能,就请一些 blogger 或者是第三方的评论员来抹黑他们。他抹黑?比如说是什么呢?我看到有一个民主党派的那个民主党的评论员,他说这些 email 很明显是假造的,因为他那个 email address 里面用的是.ukraine,Ukraine,而 Ukraine 的 domain 应该是 .ua。



王孟源 53:57 

你如果不是真的回去看那个原始证据的话,诶就觉得,诶很有道理,这很明显是假造的对不对?因为你这个国家的这个密码,中国不是China,而是 .cn 对不对?变成.ukraine 很明显的是假造的,根本不可能是真的email。但我回去一看,那个.ukraine不是他的那个 domain name,而是那个是乌克兰那个商人的名字,就是。



史东 54:27 

这  Ukraine 在前面,不是在。



王孟源 54:29 

Ukraine 在那个 domain 的前面,是它的名字的一部分,很明显的它那个名字是斯拉夫与里面常见的名字,所以为了跟别人分辨他的那个domain 其实是Gmail。所以你可以看到民主党的这些主流媒体其实也是继续玩这些抹黑的花样,但是它很有效,因为这次是连那个 Facebook 跟 Twitter 都跟着一起封锁。对,所以我认为不会有什么效果。而且另外一个你要注意的,过去这一年民主党我刚刚提过对Trump的攻击有好几个有效的着力点,这些着力点有没有依赖Biden本身是出污泥而不染的,这件事情没有,他们都是针对Trump,而且都是讲 Trump 已经恶劣到不论如何一定要下台的程度。



史东 55:26 

我想今天我们谈话谈到这,如果 Biden 确定当选的话,我们可以说确定,但我想到选择。



王孟源 55:32 

今天对我来说真正确定的,真正消除的不确定性是参与。嗯,就是因为这个 10 月惊奇太弱了,所以我认为参议院也会变成民主党。那所以我今我们这个礼拜发生的新的大事,世界上新的发大事是我们可以确定未来这两年民主党会全面执政。



史东 56:00 

如果是这样子的话,你刚刚谈了一点对美国的意义,我们可不可以把时间分出来谈一点对于中国的意义,还有甚至对于台湾的意义。



----------------------------------------------------------------------------------------



王孟源 56:56 

如果未来两年美国通过这个白左教自由民主人权这一套来抹黑中国,他是有可能不花一分钱,无非一兵一卒就主动让德国跟中国斗起来。



----------------------------------------------------------------------------------------



史东 57:29 

如果是这样子的话,你刚谈了一点对美国的意义,我们可不可以把时间分出来,谈一点对于中国的意义,还有甚至对于台湾的意义?OK。



王孟源 57:43 

我想我们先谈一下有关Covid,因为这个很简单,这个你看现在欧洲进入第二波,美国进入第三波。其实我一直觉得很奇怪,这些人贵为一国的首脑,他们有这么多的智库幕僚,怎么会这么简单的问题都没办法解决?因为很简单,你只要一人发一瓶维他命D。我从一开始 1 月2月开始报道新冠,我就跟你讲,事实上在疫苗出现之前,唯一能够帮助你阻止这个新冠大规模传播的,除了戴口罩还有这个隔离之外,最有效的就是吃维他命D。因为现代工业社会里面,人类普遍在室内生活,而我们演化的过程中基本上都是在室外追逐猎物或者是收集果实的,所以我们的日晒严重不足,那么我们的生理没办法适应。其中最严重的后果就是维生素D 不够,必须要用饮食来弥补我们自己皮肤生产的,偏偏维生素 D对于免疫是一个关键,你如果维生素D 不足,这个免疫力就会大幅下降。



王孟源 59:10 

这是为什么每年流感是冬天流行的原因,这也是为什么新冠这一次在美国流行是黑人最严重的原因之一。为什么?因为黑人的皮肤,特别是为热带设计的,所以他们在日晒下生产维他命低的那个效率比白人还要低,所以黑人欠缺维生素D的比率比白人要高得多,白人大约是有一半那个维生素 D严重不足,会造成免疫力特别弱,黑人你至 3/ 4。所以你这样一看下来就知道新冠疫情绝对会影响黑人更严重。



王孟源 59:57 

其实你去看看,像是英国的检测,现在 10 月的每天的检测量跟半年前4月的检测量第一波的时候的检测量相比,提升了 13 倍,不是 13\% ,是13 倍。你如果看欧美这些国家,跟半年前3月、4月的时候相比,他们的检测量一般都是提升 10 倍到 20 倍之间。



王孟源 01:00:23 

当然德国是例外,德国大概3月4月的时候,它的那个检测量已经是比较高的,所以它大概是提升了四五倍,因为你现在看到这个确诊看起来好像提升是第三波或者是第二波,欧洲看起来是第二波,实际上你如果把它除以 10 或除以 20 下来了,这基本上就是一个小波澜了。第一波要比现在严重得多,而且你现在看看这个死亡人数也是也可以看得出这基本上就是这个比例了。现在的这个 10 月的这一波比起4月的那一波还不到 1/ 5 的强度。



史东 01:01:06 

现在的死亡率是多少?



王孟源 01:01:09 

现在的每天死亡大概也跟4月的时候每天死亡也大概也是 1/ 5。就是你如果只看欧美的话,现在的这死亡率已经降到 1\% 点多了,



史东 01:01:24 

我还以为低于 1\% 不是已。。



王孟源 01:01:30 

在欧美可能已经低于 1\% 了。嗯,就是你可以从死亡率还有这个检测数目都可以看得出,这一波事实上比4月就是——我只看,只讲欧美,我不讲南美或者亚非。OK——事实上都比4月的那一波要小至少 5 倍。那你这个,我从 1 月的时候就讲这个,虽然一般疫苗的开发要 2 年到 3 年了,但是因为这件事关重大,全世界都是不计代价的投入,所以很有可能在年底前就会出来。



王孟源 01:02:08 

嗯,现在基本上可以确定至少会有两三个疫苗在 12 月就出现,而且因为大家政府都重视这件事情,所以生产上应该不会有瓶颈。那很可能一开始就是几百万甚至几千万,每周几百万、几千万的这样生产出来,所以没有什么太大的问题。这个 Covid这件事情到等到 Biden 上台是明年1月的时候,已经不再是热门新闻了。



史东 01:02:39 

提到这里的对不起,我再插一句,因为你提到这个疫苗的事情,今天有很多人担心疫苗的可靠性,特别是刚刚出来的时候,你对这个事情有什么看法没有?



王孟源 01:02:51 

我觉得你设法加速是可以,但是该做的双方实验还是要做齐,哈哈哈。当然你可以因为增加投入,增加人力财力物力来加速,你不能够偷工减料来加速,就像波音那样子,那个是不可接受的。



史东 01:03:13 

我们再把话题带过来,再把话题带回来。



王孟源 01:03:15 

所以对 Covid 来说,我之所以先谈它,是因为这是最简单的话题,到 2021 年基本上就变成一个次要的一节。



史东 01:03:28 

那你觉得对不起,我再插一句话,哈哈哈。你觉得 Covid 对我们的生活的情况,我们的生活习惯会不会有一个永久性的改变?还是说Covid 过的时候,我们生活的习惯又回到了Covid 之前的这种状况?



王孟源 01:03:44 

我觉得最大的改变是那个逼迫大家上网,原本没有,对不对?原本你必须要去办公室的,现在可以在习惯在家办公。然后你一旦练习了,强制所有的人口都练习了半年,一年以后大家就会习以为常了。对,所以等于是加速了这个上网的经济,我们这个网路的科技。



我想下一个话题,我们谈一谈在内政好了,好好内政,我们先谈一下这个警察的事情,然后就刚刚会花时间谈这个 qualify immunity,是因为 qualify immunity 就是民主党左派,也就是Biden 不算左派,它是民主党里面的建制派,也就是民主党里面像 AOC 或者是像 Sanders 是比较偏注于经济方面的,以及……但这个民主党左派也有比较关注社会议题的。那他对他们来讲最大的社会议题就是这个种族歧视,尤其是警察执行上的问题,也就是这次黑命贵的问题。



王孟源 01:05:02 

那这次黑命贵的议题刚好也是民主党用来敲打 Trump 的大棒之一,而且很成功,所以照理说他们也应该是有所动作,要不然 4 年之后人家也说你喊了半天选举都让你赢了,你什么都不干,对不对?那我觉得我在我的那个博文里面有分析过,这事实上要真正解决美国的黑人的地位的问题,是不可能根治的,因为它这个美国这个种族歧视的问题是很强的历史遗留,而且是实际上是一个经济阶级的问题,他是先把黑人排除到中产阶级之外,然后才加剧了这个种族歧视的问题,所以只能够治标。那治标的话,除了找中国人开刀把我们的升学就业的机会继续送给黑人之外,另一个比较简单的就是改革这些太离谱的法律保护,比如说像 qualify immunity。



王孟源 01:06:20 

所以到了 2021 年初,你会看到美国的新闻界一阵风潮,我预期会有一阵风潮来讨论说是不是要怎么样解决消除这个很可能是立法来改变。不过这件事情可能要等到他们先把大法院对Obamacare 的威胁解决之后,他们才会真正有所动作,所以可能会闹上几个月。就是在媒体上,你会每个礼拜都看得到,但是也不是头条,但是他们就是在后面慢慢地升温,等到他们有空了,他们转过头来处理这件事情。这是我预期在 2020 年中之前可能会发生的一件事情。另外的内政,比如说像 student loan 这些问题,这都是继续花钱的。那在我继续讲之前,我想我们必须要谈谈 Biden 的财政。



王孟源 01:07:27 

OK,一年前我在上这个节目,谈那个,谈那个美联储的财政跟金融管理的问题,我那时候,就那时候还没有新冠了,就已经那时候我说他们应该到 2030 年会破产,会接近实质破产。那现在这个新冠基本上是至少加速了 5 年。我跟你一下这个数据在放在哪里?OK,这个 federal reserve 就是美联储有一个研究员在上个月刚刚发表了一篇论文。



王孟源 01:08:09 

OK,这很可能是反映他们内部的共识。这是一个高级研究员,叫做 Michael Kiley,他建议为了解决新冠对经济的影响,他建议整个 monetary stimulus 就是货币的刺激,分发的资金应该是相当于 GDP 的30\%,那你如果用美国的 GDP 去算一下,这是大概 65000 亿。



史东 01:08:40 

OK,他的说,他说的对不起,我猜就他说的这个刺激和 QE 有什么分别?



王孟源 01:08:47 

就是QE , QE 是他们的部分。对对对,但是他们现在已经开始改名了,不叫QE ,因为 QE  已经是声名狼藉了,被人家臭骂过了。



史东 01:08:58 

对,OK我就想了解这一点才想起,哈哈哈。



王孟源 01:09:05 

OK。他们预期是应该放 65000 亿,他们现在已经花了多少?那个最新的统计数字是我上个礼拜看到的,他说在过去的这七个月,就是从3月中他们开始有疫情了,开始到现在他们所花的是一个 annualized 44000 亿的deficit,就是这是联邦政府的赤字了。那你这样算一下,那是annualized ,就是 12 个月,那 7 个月就是 2. 6个Trillion,就是 26000 亿,就是他们过去的这七个月的那个赤字是 26000 亿,那事实上他们刚好那个 2020 财政年度是到9月底结束,那您会去看这个是从去年十月到今年9月 12 个月的财政资质,总共是 31000 亿,所以你可以算一下,这大约 4000 亿是前五个月,就是去年 10 月到今年2月,然后从3月到九月是另外的 26000 亿,那他基本上是每个月 4000 亿的。当然现在看起来他们的这个他们讨论了 5 个月的第二波刺激法案,就是民主长是要求 22000 亿,共和党是 Trump 是给 18000 亿。结果现在看来可能大选之前可能过不了了,如果大选之前过不了的话,这可能要拖到拖一段时间。



我觉得Trump的真的是蠢到让人没话说,这种慷公家之慨来让选民开心的事情,他居然也还在那边讨价还价,他为的是什么呢?他们卡在什么地方?卡在这个unemployment,因为第一波刺激的时候就是4月的时候,4月的时候的第一波刺激它的 employment 是一个每周 600 块,后来民主党说我们的第二波也是每周 600 块。诶,共和党说不行,我顶多只出 300 块。为什么会这个样子?因为他们根本不把老百姓当人看,他怕如果你这个 unemployment payment 太多的话,他们就不想工作,也就是他把你当成牲口,你这些牲口如果能吃饱,你就会不想,不会想要拉车。他必须要确定你是半饱,他不愿意让你多拿,你如果多拿的话,他就怕你明天不上班,把老百姓当成这样子。这就是共和党的逻辑。真的是没有办法,他们为了这个,蠢到从5月谈到现在, 5 个多月,没有办法,没有办法通过这个2万亿的刺激方案,你不发这个钱人家怎么会选?最后的一定是怪到现任政府上的,一定是影响现任政府的那个选情,他们居然会连这么简单的逻辑都想不通,你可以想象他们这个对照顾底层民众的反感,这就说不但没有那个意愿,而且是有很强的畏惧,这个畏惧强到已经是如此的非理性,很明显的是他对他们的大选有利的,他们都不敢不愿意去做这个,真的是。你居然中国还有人相信这是山坡上的灯塔的?是,就是这是那个雷根的自我吹嘘 Shining light on the hill,这个真的是可笑到极点,他们就是生活水准高了,这个生活水准高是全球掠夺历史的累积而来的,很明显的你会为了 300 块跟 600 块争执了 5 个月,你想想看这个逻辑是什么?唯一的逻辑就是因为他们有极强的非理性恐惧,认为这些蓝领工人,这些底层的工作人员是像牲口一样喂饱一天,明天就不工作了。你如果了解美国世界的这社美国社会的真相,而还不是一个社会主义者,你绝对是一丝良心都没有的。



史东 01:13:30 

是这是王孟源的名言,我把它记起来。



王孟源 01:13:33 

哈哈哈。你如果不是社会主义者,在美国你如果不是社会主义者,而美国 99\% 都不是社会主义者。只有两个,第一个是你无知,你不懂这个真相。第二个是你根本就没有良心,你是一个 socialpath,我想Trump的是两者兼有。



史东 01:13:50 

那句话怎么说的?好像是什么三十岁以前,如果你不是民主党,就没有,你就没有良心。 30 岁之后,如果你不是共和党,或者你是保守党。



王孟源 01:14:03 

就没有头脑,没头脑,就是说你没赚钱,等你赚了钱以后,你就准备加入资本家集团,来欺压老百姓,这算是哪门子的有脑子。



史东 01:14:13 

对?我觉得一个人的追求,我顺便趁但趁这个机会讲一个出题的话。我觉得一个人追求财人财富并不是一个不好的事情,我觉得更重要的事情就是人类希望能够有一个更高的追求,就是你有钱之后你要做些什么。



王孟源 01:14:33 

Well,这是道德的问题,你不能够指望全社会的人都有道德。



史东 01:14:37 

就所以说这是一个我,所以说这是一个追求嘛。



王孟源 01:14:41 

但是你一个有效的经济体系个人,你必须要求他们追求自己的财富,但是政府也就是社会的集体来看,必须有时要遏制这些个人扩大私利的冲动。所以……



史东 01:15:01 

再度我们再回老讲的这个新冠,其实就脱了很多制度的裤子,对不对?



王孟源 01:15:10 

其实对我们来说,这些美国制度的缺陷都是很明显的,因为我们住在这里几十年了,我们知道他们说一套重要,中国大陆那些学术界的人名义上叫做学者、专家或者大师,可是他们全都吃这一套,实在是很可笑。我一直到今年才明白,原来整个大陆就只有一个叫翟东升的教授,他是在 2010 年之后就看出美国会想要跟中国脱钩,想要整中国,这个对我们来说是很明显的,对不对?对我来说是非常明显的。 2009 年之后我就可以看出他们的宣传不对劲了,但是他在大陆过去这 10 年是被当成疯子看的。



史东 01:16:01 

该请翟东升先生上节目谈一谈嘛,哈哈,和你空早和你做个对谈。



王孟源 01:16:07 

我对他的说法不是完全同意,他认为这个新的趋势是去全球化,也就是像一战之后到二战之间那 20 年那样子。不过我认为那个时候是十几个强权自己搞自己的山头。那当然大英帝国的圈子是最大的。现在不一样,现在只是中美两方分裂,那还有很多其他的区域强权,还有新兴工业国,还有潜在的的开发中国家,他们可以自由选边。所以我不把它叫做去全球化,我把它叫做中美脱钩。



王孟源 01:16:44 

为什么这个值得提?因为你 Biden 上来以后,他因为这个财政的问题太大,内部的问题太多,他可能在中美脱钩这个上面他绝对会减缓,它不一定会反转,但是它绝对会减缓,但是中国已经不能够再信任美国了,对中国本身也有动力要去做脱钩。所以这个中美脱钩这件事情已经是定了。即使我们知道Trump会落选,即使我们知道 Biden 没有Trump的那个那种疯狂或者邪恶的冲动,我们中美脱钩还是未来 2 年,其实是未来 5 年到 10 年的一个国际大局势。



王孟源 01:17:32 

OK,所以我们现在顺便把这个话讲清楚,我刚刚在讲的是美国财政上的问题,你看这个过去这七个月,在新冠疫情下联邦的财政资质是 26000 亿,然后这个美联储说应该是 65000 亿,那每年除已经花了多少钱?他已经花了 35000 亿来干什么?买联邦债券?联邦债券不够他买,他去买什么?就去买那个公司的债券。



王孟源 01:18:04 

这个其实我去年已经都解释过了,它真的就是直接下场去买私人的,私家的那个公司债,像波音的债券,但是他已经买到那些还没有濒临破产的公司,都已经说,可以,够了。Powell,就是那个美联储的主席在上个月站出来,他说什么你知道吗?他说这个我们这个 Monetary Stimulus已经到极限,你们国会必须要赶快把那个刺激方案——我们刚刚讨论了半天,拖延了 5 个月的那个刺激方案——通过,为什么?这个背景就是他想要花 65000 亿,他已经花了 35000 亿,他还有3万亿要花,但是他没有东西可买。这就是美国财政当前的可笑局面,他们印钞票是疯狂,疯狂到这个地步,就是他想要印,但是没有东西可以买了。然后上个礼拜我看到 Larry summers,我不晓得你记不记得他曾经在担任过哈佛大学校长,还有那个 Clinton 的财政部长,他很可能也会出任 Biden 的官员。我想最可能的是他的首席经纪顾问,他出来说 dollar is fine or worry about it。这是什么意思?就是继续印钞票,继续搞赤字,不要管这个,美元的霸权没有问题,不要担心。



王孟源 01:19:36 

实际上的意义就是说我们可以预期 Biden政权上台之后,美国的赤字不会明显的下降, Trump 我在去年上的节目的时候曾经解释过,他一下子把 Obama 任其内的赤字从 4000 亿美元一下子跳到1万亿,每年1万亿就是他减税,然后又乱花钱。但是因为美国现在的这个经济的态势, Biden上台不但不能够消减赤字,而且反而是必须要继续的硬钞票下去。这个就像是奥巴马 2009 年上台的时候,并不是他想要乱印钞票。



史东 01:20:22 

而是在这个态势上他无上没有选择的预定。



王孟源 01:20:25 

没有选择,留下来的摊子太烂。我觉得这个 Larry summers 出来这么一讲,我基本上百分之百确定明年的赤字一样还是在Trump这个水平,就是3万亿左右。我这是假设那个,我是把这个下一波的 stimulus 这个 2 万多亿也算进去了,就算到,他这个大选前没办法通过,那么就必须要等到Biden上台再通过,当然也有可能大选,那个民主党全面执政以后,共和党说算了,反正我们要下台了,随便你这种细节要看实际的结果,不过从财政跟经济上来看就是这样子,那他这个钱会花在哪里?共和党要乱花钱,主要是减税,把钱还给有钱人,然后就是做construction,就是做那个公共建设。



王孟源 01:21:18 

为什么呢?因为这图利那个承包商。民主党花钱的方向不太一样,它这个美国跟中国一样,都是中央政府比较有钱,地方政府没钱,因为这地方政府那收的税率一般都是很低的,所以民主党会把一很大一部分的钱拨给地方政府作为补助,这也是需要的。光是这笔钱可能就是一两万亿啊。这个我们现在他们现在在炒的那个 2 万多亿的那个方案里面,就不到1万亿是给地方的。我觉得如果民主党全面执政的话,会把它提升到 1 万多亿。



王孟源 01:22:05 

然后另外一个就是 operating cost,譬如说在美国你如果要开一条新的公路,刚建的时候联邦政府会,但是一旦建成以后,联邦政府一般是不管你的。对对对,要修补这条路或者是维修之类的,都是地方政府自己搞。像这种东西民主党会比较愿意花钱,因为它事实上你那个东西建好了以后有没有用还是必须要靠 operation 来决定,尤其是比如说你有那个捷运系统,如果公共汽车如果是每个小时才来一般的话,那你只有真正养不起车的人才会愿意做,对不对?你如果是每个 15 分钟来一班的话,那有车的人也会考虑要可以可以做这个,做捷运那然后再加上一大堆环保的议题花费,这个都是都是几百亿几千亿这样算下去的。所以我认为 Biden 第一年的赤字到3万亿不是太离谱的估算。



王孟源 01:23:17 

第一点是它对美元实际上会有继续的压力,这就是我另一个一直在建议中国着手的重点,然后一直都看没有看到他们出手,而且有点Frustrating,就是美国霸权现在最后的根基就是美元,因为他印了这个钱,美元的购买力不管你怎么应它还是维持在一定的水平,能够维持这样子唯一的理由就是它是全球的通用的储备货币。所有的国际贸易有一半以上都是用美元来定价。那我一直说中国的人民币还没有到能够取代美元的地步,那你必须要,如果不是用欧元,就是要利用一个区域货币,比如说金砖货币来取代。



王孟源 01:24:10 

我最近看到大陆的讨论,讨论到这个方向,他们都会说人民币准备要数字化。这个是外行人的说法了,这个数字化不数字化,只影响零售跟运用。你数字化的人民币还是人民币,你还是会有汇率的问题,你还是会有这个结算,最终跟那个央行结算的问题,不会因为数字化而就能够让第三国家愿意用,你要用的话,这个必须要是外交上的非常高阶的一个动作。我一直没有看到有这方面的着力,所以觉得是因为你如果把美元的那个霸权消灭了或者取代了,就是釜底抽薪,他就没有办法再支持这样的制止,他本身的内部的购买力一下子会崩溃。以后。



史东 01:25:15 

应该如何做?



王孟源 01:25:17 

上策是跟欧元,也就是德国跟法国两个国家来做一些密谋了。



史东 01:25:25 

对,你提过这个主意。



王孟源 01:25:28 

想办法让欧元变成一个国际的交易货币,这是事实上在未来三年这是唯一一个可行的方案。中策是金砖货币,但是这个金砖货币有印度在里面,实在是让人不放心。南非现在的这个总统也是拍好了,但是巴西就很有问,不过巴西这个总统大概也做不了多久了,下策则是等到人民币能够取代,而这个至少 10 年了,不切实际。



王孟源 01:26:00 

美国这样子倒行逆施,而且拼命印钞票,事实上是利用这个印钞权来搜刮全球,这是不合理的。然后它这些收刮来的财富用在什么地方?用在军工上面,然后用这个军工这么多武器再去欺压其他国家,你说这合理吗?拿从你压榨来的的钱做了武器,然后再来欺负你。嗯,要打破这个循环,我觉得最弱的一个环节就是美元。那这件事情我已经讨论了五六年了,但是没有看到中国有了很什么积极的动作,所以很奇怪。



史东 01:26:41 

我想意识到这个问题和如何的执行,这是一个非常不同的概念,对不对?



王孟源 01:26:49 

我想的问题在于我讲的每一个专业,什么金融了、货币了,或者是战略了,他们都有顶尖的人才能够达到同一个层次的答论。但问题在于他没有一个人才能够兼顾两个或者三个或者所有的方面,然后把它放在一起做大局观。这个是,这是一个问题,这是因为他们的那个智库的组织,照理说智库的用意就是这样子的,把不同方面的专家集合起来,然后众志成城,然后做brainstorm,然后可以分享各自的专业。但是中国的智库除了生产官僚体系下的文书之外,好像没有什么用处。你想看连翟东升的这种战略观点,他好,他居然都是全 14亿里面的唯一的一个,那你可以想象,那么整体的做智库的水准是什么样子?我看到一些,听到一些故事,说两三年前他去开会,国内开会,然后提出他的看法给演讲,结果被人群起而攻之。这是这……但是他居然……当时没中国的主流,不只是主流,几乎每一个人都相信美国不会跟中国决裂,美国会不顾他这个霸权所带来的额外利益,而坐视中国慢慢和平崛起。



王孟源 01:28:17 

你看看 Hunter Biden 搞的这是什么?我以前写过一篇博文是讲Bob Dole,它为中亚的,是吉尔吉斯坦的一家银行,他去当那个董事,凭空赚了几百万几千万。Hunter Biden也是一样,这个纨绔子弟一辈子什么事都什么,正事都没干,但是他就是因为美国有霸权,能够左右乌克兰的内政,所以乌克兰的财团才会想要让他去争当董事,付他几十万几百万的钱,要等他干什么事?就是等他去介绍这些财团去给他老爸。



史东 01:28:56 

其实他做的这些事情在美国的政经高阶是常见的事情。



王孟源 01:29:00 

不只是常见,这是他们的,他们必要的生财之道。所以你说你指望这些人会坐视中国崛起,然后断他们的财路,让他们没办法再继续……



史东 01:29:16 

这也是你说在真正说到点子上的一件事情。



王孟源 01:29:21 

Hunter Biden我们讲了半天,这是他们的常态。不只是常态,连他们的大法律、最高法院大法官出来明文地说这是我们民主的常态,你还认为他们不会为了维持他们的民主常态而跟你拼了,生死搏斗,那你真的是太天真了。



所以我现在谈了他的内政,谈了他的经济财政,谈了他,我们现在谈谈他的国际好了。他的国际第一件事情是因为他的内部如此虚弱,他也拿不出钱来,所以他能做的就是比如说重新加入TPP,这个 TPP 现在是那个日本帮他暖着,他随时来就可以坐下来,这是很简单的,这一定的;但是跟欧洲的 TTIP 就不一定了。欧盟在过去这一年多跟 Boris Johnson 折腾了这一下子,我想他们对这些海洋法系的国家,就是Anglo-Saxon族的国家会有些警惕。



王孟源 01:30:26 

在这个 TTIP 上面,我不太看好 Biden 不太可能在 2 年之内有什么突破,因为我现在谈的是 2021 年跟 2022 年的前瞻,到 2022 年又中期选举,但可能如果继续维持参院跟众院的掌控的话,那他可以继续考虑进一步的动作。但是如果他失去其中一院的话,那么美国又要回到那个党政空转的。最重要的一个,最重要的最后的一个话题,就是我对美中关系的期望。



王孟源 01:31:03 

我已经讲过了,这个基本上会是Trump政策的延续,就是贸易战,虽然不会继续升级,但是也不要指望作业一下就解决了。事实上他会开另一个新的战线,只不过是不是贸易战场或者是企业上的战线,而是这个,这刚好是这一题的最新一期的economist。我一看到追齐的封面,我就说,好玩了,我正等着你。为什么呢?他这个他追齐里面是什么呢?有三篇文章来骂中国。第一篇文章的要点,自由民主正在全面退却,全世界正在陷入暴政的威胁之中,而其中最大、最危险、最优先的就是中国在新疆的暴政。



第二篇文章详细地列举了三个资料来源,做出了二三十个严厉的指控。这三个资料来源是什么?第一个是一个在美国申请政治庇护的维族人士,你申请庇政治庇护的时候还会在乎讲实话?第二个,这个第二只是引用了两句,就是这是一个德国的白左教授。第三个是澳洲那个很有名的 CIA 资助的智库,它就是基本上用这三个来源,然后用典型的那种主流媒体的,把可以看到的模棱两可的证据,跟完全没有证据无法佐证的严厉指控,交错运用,给你一个印象是:不但又严重,而且又有证据,又罪证确凿,非常的有趣。然后最后一篇文章讲的是中国在新冠上面,他讲的是虽然中国最后控制了新冠,但是在1月的时候,是因为中国没有诚实处理,所以才让新冠扩张到全球。我从 1 月开始就一再地写博文在讲,你那个时候连这个病毒是什么都不知道,这病毒是不是新的都不知道,你怎么能够指望任何一个国家、任何一个政府下决心做出正确的举措?这都是用事后诸葛亮的角度来找茬的。你自己看看美国在 2009 年那个猪流感的刚起来的时候的前 6 个月是怎么做的,根本什么都没做。我为什么会特别重视这个?因为economist是算是一个很高级的刊物。



史东 01:33:45 

我就想问你这个问题,你看了这三篇文章,你对这个杂志的看法有没有什么改变?



王孟源 01:33:52 

没有没有,我一向都知道他们有两套编辑团队,就是负责前半的政治团队跟负责后半的经济专业团队。你如果看,所以有的时候在同一个同一本经济学人里面,在前半把中国骂得狗血淋头,在后半这边……



史东 01:34:15 

所以说我从一个从事媒体人的角度来看这个事情,这个杂志他们的做法就是用他们在经济方面的 credibility 来带动他们在政治方面的就是一个宣传。



王孟源 01:34:30 

即使是在那个政治抹黑的文章里面也是用模棱两可的证据,就是这个证据事实上不足以达成任何结论。



史东 01:34:42 

这个就是西方媒体最厉害的地方。对。



王孟源 01:34:46 

然后掺杂着很严厉很绝对的言论,但是跟那个证据没有什么逻辑关系。哈哈,这样子交替的用,那给你的印象就是真实而且又严厉。



史东 01:35:00 

OK?所以说你是觉得这个维吾尔,这个议题。



王孟源 01:35:05 

我一直怀疑 economist 的那个,那一套政治编辑团队是美国外宣体系的一个欧洲跟就是 European council for foreign relations 的一个附属组织,所以我个人很怀疑这个是他们未雨绸缪,因为搞宣传这种东西,重要的是你要事先做,提早做,反复地做, 3 人成虎对不对?而且越是提早地讲越有说服力。你如果是事后需临时需要,而且临时抱佛脚,一般都是来不及的。这种洗脑的事情必须要时间的。我认为这是民主党的建制派已经准备要为拜登的对中政策埋下伏笔。



王孟源 01:36:00 

你可以预期 Biden 的最终的斗争不会是直接来自美国国务卿或者是像Trump 的团队,现在这样搞的都是他们自己人出来指手画脚大声骂,而是要欧洲的白左团队这些间接的部署团队来抹黑中国,而他们最重要的抹黑手段——就是因为现在香港已经平息了——他们已经决定就是要用新疆来做着手点,做主攻方向。所以如果我是中国的话,我会未雨绸缪。



王孟源 01:36:37 

我很简单的,比如说你那些新疆的那些在职教育被召集拒在职教育的为主,你可以给他发个津贴什么的,因为集中运营是不会跟你发津贴。然后我想顺便谈一谈这个白左这件事情。白左其实追根究底是一个新的宗教,它是 70 年代创立来取代已经日益微缩的基督教势力。西方对外扩张的时候,维持内部稳定跟消除外部主力的一个很重大的成分就是它的宗教。这在历史上一直都是以基督教为主,你看他到外面殖民的殖民政府的时候,都是传教士跟着炮舰,然后炮舰一旦打开了这个通路以后,就是商人来建立剥削性的商业,拒绝,然后由传教士来吸收信徒来减弱土著或者当地人对外来剥削者的抵抗。到了 20 世纪,尤其是经过二战之后, 50年代, 60年代,尤其经过六零年代的那个 liberal movement 自由化思潮以后,基督教基本上已经没有办法满足对内安定,对外征服洗脑的那个作用,必须要有一个新的宗教,而且这个新的宗教必须是对新生的这一些自由主义派人员有很大的自然吸引力的。这是为什么白左兴起的原因。宗教一直是这些殖民国家的软实力的,所谓软实力为他们说软实力其实就是指的他们的宗教,但是他们在 1970 年代之后,这个宗教不再是基督教,而是白左的这套思潮,也就是所谓的……



史东 01:38:48 

我们可不可以把它看成是一个比较更世俗化的一种宗教。



王孟源 01:38:54 

就是,它其实还是一样,就是……



史东 01:38:56 

但是就是我是在外表上它不是像从前披着道冠的那种样子的形态的宗教,而是一个更趋近于社会、更趋近于似乎世俗化化的一种形态。



王孟源 01:39:09 

本质上一样是非理性的绝对信念。



史东 01:39:13 

我不是说本质,我是从外形,因为它的外面的包装的。



王孟源 01:39:17 

但是外表的包装是刚好相反,以往是那种原始的超自然的,这个现在变成——就是基本上基督教之所以会失去它的影响力,是因为从 16 世纪之后,然后到 17 世纪、 18 世纪启蒙时代,科学化跟理性化的思潮,逐步地把它那个超自然的成分的根挖掉了——对,所以你必须要建立一个新的宗教,那没有超自然的成分,而且在表面上好像是符合科学跟理性,实际上是完全违反理性。那这个他们经过了 60 年代、 70 年代的酝酿,到 80 年代才真正定义,刚好是因为他们正在冷战的途中。而对苏联的最有效的就是这套自由民主,所以就,顺势因势利导而发展出这套东西,它其实是一个宗教。



王孟源 01:40:21 

我刚刚已经讲过,他们在自由、民主、法治这三件事情里面根本就完全是口号。如果必须要违反自由,他们就讲民主。如果必须要违反民主,他们就讲法治。如果要违反法治,他们就讲自由,都是看他们方便。然后这个东西用来颠覆非常有效。因为你在东欧成功了一大批,后来在东亚也成功了一大批,最先成功的就是台湾。那接下来是香港,现在是泰国。



王孟源 01:40:55 

我觉得你要注意一下这个新的宗教,它跟旧的宗教当然不会共通;这个特点,反科学或者非科学的特点就在于他们不但本身逻辑不自洽,他们互相之间也是不相容的。所以你看美国的这个态势,分成左派跟右派,左派就是白左,一个新的宗教;右派坚持的是什么?基督教嘛,是旧教。



王孟源 01:41:25 

所以你现在看到美国这个分裂其实就是一个宗教战争,是一个新教跟旧教之间的战争,我到现在没有看到任何一个政治学家来强调这一点,可是我觉得很这很明显,这是一西方从过去 2000 年所有的战争,几乎都有宗教成分在内,最严重的都是宗教战争。他现在看到的,你现在看到美国的分裂本身就是一个宗教战争,白左教跟基督教的战争。



王孟源 01:41:59 

你看现在欧洲白左最严重的那几个国家是哪几个国家?第一是瑞典,第二是捷克,第三是德国,他们这些都是以往的新教国家,现在的宗教性已经几乎不存在了,几乎没有人每个礼拜上教堂。欧洲欧盟里面对基督教最执着的是哪些国家?是南欧的国家,像意大利这些国家为什么不相信白左?因为他们已经有一个宗教,他们还是旧教,所以中国必须要了解这个东西。是明年1月以后, Biden 政权不会直接出面,但是他会叫他们的欧洲利用他们的宗教,因为白左的教廷,就是核心了,还是在美国,对不对?因为白左这个宗教。其实我们说美宣是什么?美宣就是第一个宗教的传教手段,那它的教廷就在于好莱坞或者是美国的这些宣传大亨所在地,纽约这些地方。美国要跟中国斗争,用政治跟经济来直接上是下策,用军队下下策,用宗教是上策,为什么?不但这个是兵不血刃,而且中国内部已经有很多带路党,已经有很多信徒。最重要的是欧洲的迷信程度更高,直接叫瑞典捷克跟德国上就可以了。



王孟源 01:43:46 

美国连出面都不用出面,要解决这种事情,必须要拆穿这个宗教的假面具,而且要未雨绸缪,知道美国将明年后年的打手会是瑞典跟捷克这些国家,所以你有了机会就必须要揭穿他们的假面。他们的假面是什么?今年5月的时候,瑞典搞群体免疫搞到什么地步?他甚至禁止给老人院里面的患者做治疗。你如果去看英文报道,根本就没有,完全没有这个报道。瑞典内部有些医生看不下去了,写了一篇稿子投到英国的专业期刊。医学界的专业期刊。



王孟源 01:44:37 

我看到了,我说天赐良机,你这不就揭穿了他们所谓人权的那一套嘛,对不对?我在我的博客就赶快把它登出来,以后没有下文了。中国 14 亿人呐,他这个智库怎么搞到连这种事情都没有想清楚?你如果不事先把那个瑞典的人权制高点先拆下来,你怎么样来抵抗他们?未来两年, 2021 年、 2022 年一定会拿新疆跟香港来给你找茬的这个借口。他们要是能够说服德国的话,那你连跟欧盟的一些交涉都不用谈了。我刚刚已经提过好几次,欧盟是中国最重要的一个战场,必须要在至少避免他跟美国一鼻孔出气,那这里面决定性的就是德国跟法国两国,那这里面法国反而我不是太担心,法国人是比较现实的。你跟法国的问题在于必须要进一步去讨论欧元取代美元的问题。



王孟源 01:45:56 

跟德国的危险是在于德国的企业界固然经济固然跟你关系密切,但是他们深深地受到白左教的影响,你千万不能小看这些宗教在白种人之间的作用力。如果未来两年美国通过这个白左教自由、民主、人权这一套来抹黑中国,他是有可能不花一分钱,无非一兵一卒就主动让德国跟中国斗起来。



王孟源 01:46:35 

你要解决这些这种事情,最好的方法就是事先就把这套花样拆穿,让他们没有声量,没有这个市场,没有这个舆论市场,你如果能够每次他们提起人权的时候就反过来追着瑞典打,OK?那瑞典就没有话讲,就讲不出话来,他讲出来也没有人会听理他,中国政府还不了解,这是生死搏斗,美国人绝对不会放过任何一个机会,任何一个手段,你必须不能够说不要去惹事,这个事情会惹到你头上来。怎么做人家都会惹上来,你不事先准备的话就是等着挨打。



王孟源 01:47:30 

唉,我真的是,哈哈哈,最近这一年我给出的切实建议特别多,所以看到他们自若罔闻,我特别的frustrate,所以好,我想这基本上就是我对白等政权未来一两年的预警。我想这个对外行人来说,这包括中国的那些外交智库,这个他们被瑞典跟德国人骂打的时候,他们可能还觉得莫名其妙,可是我可以告诉你,这种事情都是很简单就可以两三年前遇见了,你看到这个就应该知道已经是确定了明年开始,这就是他们的跟中国斗争的组合。



史东 01:48:21 

有关于这个海峡两岸的问题,你觉得拜登会怎么处理?



王孟源 01:48:25 

我也是一直在讲这个,中国处理是高度优先,但是没有急迫性,所以有些大陆学者认为 2021 年 2022 年就会解决。我觉得如果是我不会那么急,这个应该是等到 2024 年或者 2025 年来着手是最合理的,因为最合理的是等到美国美元崩盘,这就是为什么美元崩盘这么重要。



王孟源 01:49:00 

你美元一旦崩盘以后,现在美国所面临的问题根本就是小巫见大巫了,美元崩盘之后它什么东西都负担不起了,如果去除了这个威胁,以后这个解决就方便的多,容易得多。对我来说这些事情都是有它的逻辑体系,有它的优先后次序,可以一步一步来做。但是呢,你看,像是两年半前我开始讲说中美贸易战,你如果不先把美国打痛的话,绝对会步步升级,没有什么能够退让妥协的。结果头一年一样,我事后才知道原来中国的学术界自个护界只有翟东升一个人有这个认识,然后他被当成疯子一样的笑,那难怪那个时候的政策就是只想着要减低战略风险,避免全面升级。但是这个想法是对的,但是风险不是在于对等反击会引发风险,而是在于你不对等反击,反而会鼓励Trump地去升级;你如果反击把他打痛了,反而不会升级,反而没有风险,而且是保持战略灵活度,就是你的战略选择。你随时都可以说我现在妥协,你一开始不妥协跟他对等反击,你随时都可以说我换成我妥协。你一旦选择了妥协,就没有再对等反击的余地。



王孟源 01:50:46 

人家会说,诶, 6 个月前已经这样,你现在等 6 个月才叫才要反击,这算哪门子事对不对?那一定会保证会升级,因为人家不会把你当做对等反击,而是会当做是进一步升级。所以你从他们这个大战略的观点是对的,但是执行起来因为对敌方或对手的认识完全错误,所以这个选择执行的策略的时候也就完全错误。我这是对。



史东 01:51:21 

就是知其不知彼。



王孟源 01:51:25 

对,知己不知敌。



史东 01:51:26 

对。



王孟源 01:51:28 

真的是,就是那句话,知己不知敌。对,你不要说我骂得很高兴。其实论语里面说,邦有道,危言危行;邦无道,危行言孙,这个意思是什么?就是我在美国我知道他邦无道,所以我说的一切话都不用英文发表的,必须要言孙。我愿意这样子公开的骂。中国的这个政策不够优化,不是最优化的结果选择,这是因为我认为他们还是有道的国家,所以可以危言,而不是言孙。那所以算是抬举他们的,我希望他们了解我不是在侮辱他们。



\twocolumn[\begin{@twocolumnfalse}
\section{孟源漫谈}
\subsection{20201112}
\end{@twocolumnfalse}]Credit: anonymous



史东 00:21 

各位朋友你好,我是史东,在今天节目中我为您请到了是您喜欢,您欣赏,您甚至我可以说崇拜的一位人物,那就是王孟源王先生。王先生跟我联系,他说最近有一些事情想和大家交流交谈,那么我就说,好,现在我们就把王先生带进我们的画面之中,王先生还是照样地说一声谢谢。欢迎。



王孟源 00:50 

很荣幸再上你的节目。



王孟源 00:53 

其实主要的原因是我最近那个手伤已经好几个月了,上次三个礼拜前上你的节目之后,很快的又旧伤复发,因为我自己还要做家事,所以我的博客一直没有去关照。



史东 01:12 

我插一句话,你的旧伤是肌肉性的还是骨折?骨头方面还是怎么样?



王孟源 01:19 

Tendinitis, 是拉伤的,tender 拉伤。



史东 01:21 

就是肌肉性的。



王孟源 01:22 

我现在基本上就是带着这个Bracer,一天 24 小时,那我因为连拉个床单都会再把这个手拉伤,所以我就强迫自己不去看。



史东 01:37 

你现在能做的只有保养而已了,就是保养对。



王孟源 01:40 

就能尽量保养,那我怕让读者失望,所以我觉得这个你的节目是一个很好的通道,所以虽然是今天所谈的,可能是比较啰啰嗦嗦繁杂一点。但是,因为你很特别,所有的节目主持人里面,你是最信任这个来宾的,就是能够容许我自由发挥。所以真的是一个荣幸,真的很感谢你给我这个机会。



史东 02:14 

是我,我应该说感谢,我也要带我的观众跟你说声感谢你。你在希望能够找一个管道的时候,你想到了我们,想到我们的观众,在这我说一声谢谢。



王孟源 02:28 

The honor is mine。我想先从 3 个礼拜前谈的那些事有一些小错误。首先后来有很多读者跟我联系,说我说是翟(Di)东升教授,那是因为那个字一般做姓的时候读Zhai,但是很多读者来跟我讲他自己发音是Di,就是这个字也可以通北狄的那个狄字,所以可能是他们家有特别的情况了。那我当时就是……



史东 03:08 

我的了解,对不起,我插一句,我的了解这个字也念Zhai也念Di,但我不知道是不是台湾跟大陆发音的不同。



王孟源 03:19 

后来我去查了好几个词典都他们都说做姓的时候应该做 “Zhai”,这也是我一向来的印象。



史东 03:26 

对,因为我有一个好朋友就是姓翟,所以我知道这个事。但是翟教授,翟教授,我们叫翟教授,他自我我以为是在他在一个录像上他自己解释,他说我们家叫狄,我们家自称这个姓念翟。



王孟源 03:44 

那我们当然是尊重他自己的选择。当然我上一次提到翟教授的时候,我还说他是唯一一个能早先就看出中美的关系必须决裂,而且不是中国的选择,而是美国必然会选择的。这也是我过去十几年一直在强调的事情。但是因为我对中国的学术界没有追踪得很熟悉,所以我一直到最近才听到、才看到一些翟教授的文章。我也没有看过他的视频,所以我不晓得怎么发音。然后上次的视频发表之后,也有读者跟我联络说还有其他的教授,比如说温铁军教授。这只能怪我自己孤陋寡闻,不是他们的错。不过我上次要谈的那个主轴是:他们的看法在两年之前在大陆还是绝对少数的,这个是没有错。那是一个人,两个人还是三个人,这样说其实无伤大雅。就是基本上他们是少数。一直到过去两年,因为受Trump所赐,我们大众才能够看清真相,拨乱反正,认清了美国的真面目。我其实从开始写稿以来,我的博客的座右铭一直是“事实与逻辑”,这也是我处事的原则。我做的一切分析都是根据事实跟逻辑来讨论的。那上一次跟你谈的时候有特别谈到,比如说 Covid-19。



王孟源 05:36 

就是新冠目前我们一般人所能够做的防治的手段,除了社会隔离跟戴口罩之外,其实就是吃维他命-D。后来有些,我看到有评论说,这听起来好像是个民科在吹嘘的偏方。你如果是完全不懂,只看比较的话,好像是如此。但是这个差别在于,我这样说是根据大概一两百篇论文,几十个双盲实验所做出来的,这个数目比所有市面上每年卖几百亿的那些中医药方所经过的双盲实验加起来还要多。事实上中医自从青蒿素之后,好像没有任何一个双盲实验真正被好好地做出来。



王孟源 06:40 

这是因为我讨论的,我追求的目标是要达到目前人类学术界主流所能达到的最高阶段。所以比如说我在讨论国际关系的时候,这个层次是什么层次?就是翟教授的层次,或者是温铁军教授的层次。如果有一些项目,我甚至是反过来批评这些主流的话,这不是因为我比那些主流的人物更聪明。一个很好的例子是我曾经批评过大对撞机,这并不是说我比 Weinberg(曾经是我的导师)要聪明,或者我跟丘成桐打过论战,这并不代表我比丘成桐聪明。而是他们自甘堕落,尤其是丘成桐走上歧途,拿着他们的明显来骗钱。是他们堕落了、腐化了,而不是说我的才智超过了学术界的主流。除了高能物理界之外,还有的例子比如说像美国的金融界。美国的金融界基本上像对冲基金跟银行所做的那一套,你在学术界的论文根本看不到。学术界讲的都是他们故意容许的,他们用来公开发表,来骗外人或者骗监管单位的。



史东 08:16 

都是台面化。



王孟源 08:17 

对,台面化的。至少十几年前我在主流的大银行工作的时候,大银行里面都会有两种 Researcher,一种是对内部的Trading 交易员负责的,他们研究的是实打实的东西。另外一个体系是对外发表的,他们发表的研究报告是送给客户的,这个就完全是学术性的。而且同样是做研究的,根本两边完全不会有任何交流。我记得曾经有一个不识相的新的研究主管,是在对客户方面的那方面的主管,他刚刚从学术界转过来。有一天他跑到我们的交易厅里面跟一个交易主管说:“你们怎么不用我做的研究?我做的研究都发给我们每一个客户,为什么你们交易厅连看都不看?”那个交易主管说:“你等等,我一个小时后回你。”一个小时后,他的研究员把这个对外研究方面所做的股票分析,在过去两年的股票分析上做了一个模拟。就是你如果根据他们的推荐来买卖股票的话。第一个推荐赔了,第二个推荐还是赔,第三个推荐还是赔。那个交易主管说,这就是为什么我们连看都不看。



王孟源 09:57 

所以金融界,美国的金融学术界一样也是不入流的。就是说他们的所谓的主流,根本对外所公开讲的那一套可以直接扔进垃圾桶。那另外一个行业也是这样,也就是他们的自由主义经济学派。这个我在我的博客还有在你的节目上已经反复讨论过了。这最早是 Rockefeller 资助芝加哥大学建立。整个芝加大学就是 Rockefeller 投资的,而且他事后说,这是我一辈子最好的投资。他建立芝加哥大学的用意就是要洗脑,编造出一套自由主义的经济学来为他们财阀独霸社会资源证明。那在他们搞出了大萧条之后到了 1932 年,罗斯福当选,然后推行新政,那其实是一个向社会主义走的新政,所以他们这些自由主义者曾经销声匿迹过一段时间,到了 1950-60 年代经济稳定了,和平时代的红利滚滚地流入。这时候他们还有一大堆社会契约制约着这些企业的追求利益的方法,所以这时候他们就收买了一个年轻的教授,叫做 Milton Friedman,他在那个时候收买了——真的就是收买了。那 Milton Friedman 就拿着 3 倍的薪水去招兵买马到全美最好最聪明的经济学的博士,到芝加哥编造出一套理论,后来成为Reagan经济学的理论基础。



史东 12:02 

一直我还曾经蛮欣赏Milton Friedman,结果现在发觉被骗了。



王孟源 12:12 

他是这个世纪对人类危害最大的骗子之一。



史东 12:19 

还好我不是个好学生。



王孟源 12:22 

所以经济学是另外一个。那还有其他的比如说像政治学,就是白左民主自由这一套。这其实是ch出于英国从一战到二战政治战争斗争的需要,必须要抹黑德国。因为当时英国的政府是比较弱势的,所以他们利用这个差别来强调德国人的邪恶,就德国是专制的。等到进入冷战以后,因为对象是苏联共产党,所以更合适使用这套说辞。所以英国跟美国人就把它发扬光大。其实我想很多人忘记了,历史上英国人就是欧洲最贼的民族,这个在 17-18 世纪他们跟法国人斗争的时候,法国人给英国人取了一个名字叫 Perfidious Albion,是 English 的另外一个名字。 Perfidious Albion:非常贼的英国人。就是他们对抹黑还有编造光明正大的扭曲逻辑方面是专家。所以后来是在冷战的期间被美国人继承了。你现在也看到他们试图用这套来用在中国上头上。这个国际关系跟政治学是另外一个学术领域,那是主流。所谓的主流其实是个根本没有资格把自己称为学术界。



王孟源 14:09 

那还有另外一个是……这个就比较冷门一点,是声乐界。我曾经在一篇文章里面提过,基本上到 70-80 年代以后,歌剧界的那些女高音、男高音,那些演唱家他们的水平就一落千丈。这原因倒不是因为有什么财阀收买他们,要让他们腐化,而是在 70年代开始有了Hifi麦克风,所以职业表演不再是必须像以往在一个广阔的歌剧院或者甚至露天剧院这样唱出去,要让 100 公尺外的听众都能够听得清清楚楚,所以你可以对着一个 10 公分前面的麦克风这样子来唱歌。这时候就有很多取巧的方法,连发音的那个方式都不一样了。现在你如果在美国,尤其腐坏得最快的是美国。所以我……因为我对艺术是一窍不通,我对艺术方面唯一喜欢的就是听歌曲,曾经在 Metropolitan Opera 大都会歌剧院买过季票,后来买了两三季以后受不了了。因为他们已经完全被这个美国的声乐学派给搞烂搞砸了。他们已经习惯于就是对着麦克风唱。这个时候音量不需要,就没有中气的,真正的差别就在没有中气,没有中气的话就传不远,所以你在现场听的时候跟录音完全是两回事。你在录音的时候因为这么近,所以你可以模拟出很漂亮的音色,但是这个音色没有中气,他们的唱法基本上就是(学过声乐的应该知道),他们现在流行的是所谓Sing into the mask:假想你有一个面具,然后你把声波唱进那个面具里面。这个是八零年代以后才兴起,而且是在美国特别流行。现在真的要听歌剧还是必须要到欧洲去听,在美国已经听不到好的歌剧了。



史东 16:40 

最欣赏的歌曲歌手是谁?



王孟源 16:47 

我现在听歌剧越听越老,卡拉斯基本上是最后一个全能的女高音了。在 18 世纪、 19 世纪的初期,那时候他们就没有分值。我特别谈一下,因为这个其实是一个很普遍的现象,在不是光是声乐,其实在几乎每个学科都是这样。就是在 20 世纪中期之后,它那个分科分得很细,你现在去看那个女高音的话,它至少就光凭音色跟音高可以分成七类。然后这次还是在女高音里面Soprano 里面。那比 Soprano 音色稍微低一点,音高音频稍微低一点,然后音色比较厚重一点的还有 Mezzo 就是次女高音,那又是另外,对,但是其实在 18 世纪莫扎特的时候,或者是在 19 世纪初Rossini 、Bellini 这些歌剧大师的时代他们的那个没有严格的 Soprano 跟Mezzo 的分别。基本上你如果是唱女高音的话,你就能够从最高的那个花腔女高音一路唱到次女高音。但是历史上到了 20 世纪就越来越精尖,然后大家所能涵盖的范围就越来越小。最后一个能够从花腔女高音一直唱 7 个阶,一直唱到然后唱到第 8 个阶就是Mezzo  次高音的就是卡拉斯。所以我现在听的都是最晚就是卡拉斯。比如说再早一点就会去听像杜兰朵的话我就会听听 Eva Turner(它可能是 20 世纪最伟大的杜兰朵了)。我不是故意要谈这些冷门的事,不过我必须要拿这个例子来解释一下当前学术业的现象。



史东 19:17 

顺便再插一句话,我想你一定有一套非常好的音响。



王孟源 19:26 

我想一般 30 多岁的男人都会经过一段阶段(哈哈!),花钱买很贵的音响。我已经早已过了,所以到了 40 岁的时候是在听现场的。现在回去听那些很旧很旧的录音,反而就是 YouTube, 反而不在乎这个。不过这个现象其实在很多学术界都是这样,比如说物理界,在 19 世纪末 20 世纪初基本上很多做理论的也可以做实验,很多做实验的也懂理论;到 20世纪中期之后,那个理论跟实验就完全分开,不再有任何的交集。



王孟源 20:13 

我的个人的追求,是说我并不要求自己走到最尖端,因为是目前所谓走到最尖端,就是你不断地还在创新,能在突破人类知识的极限。这个是因为我们现在的所有的学术都是已经很成熟了,都是有一两百年的历史,即使是很聪明的人也需要全神的投入,所以才会有刚刚我谈的这些非常精尖的现象。我要求的是:我自己说的每一件事情都是当时目前学术主流已经研究出来的结果,公认是最好的解释。我讨论的,我在博客里面讨论的,还有在这边跟你聊的其实是有三类,但是我想读者一般没有考虑到,但是我自己知道有三类。第一类就是科普这些知识;第二类就是我自己有创新。那我刚刚已经讲过了,现在这些学术界要有创新是非常非常困难的,我自己脚跨将近有10 个专业,那怎么可能?我不求在一个专业里面做创新,我把这些专业合起来。有很多现实的讨论,要求你对好几个专业同时有知识、同时有深刻的了解。这种需要多个专业交集的东西就可以有我自己原创的贡献。



比如说今年年初新冠刚开始的时候,有人讨论这有没有可能是生物战武器,那生医这个我懂一些,我只要花个几天的时间去读论文,我就可以把自己的知识提升到我刚刚说的,能够跟上主流的那个地步。然后我这几年对战略也很有兴趣,对军事研究现况也很有兴趣,所以我能够把它合起来讨论。然后下很快的下一个结论说这是不可能的,因为它是一个 RNA 病毒,它的突变速度太快,非常不适合做生物战的武器。我想这个今年年初,我上你节目的时候曾经提到过,做这些研究的好处是,比如说像生医界,我虽然不是我自己原本的本行,那我刚开始看,因为新冠开始之后,我开始去看这方面的论文,刚开始的时候当然是比较辛苦一点,但是我有那个能力,因为我毕竟受过博士的教育,所以我现在有关新冠的论文大概看了两三百篇。这大概比医学界从业人员就是医生大概 99\% 的医生都还要多,所以我很简单的一句说,“维他命 D 是目前一般人在疫苗出来之前,一般人能够服用最有效的防治新冠” 的事情,背后是两三百篇多所依据出来的那个结论。



王孟源 23:52 

我在3月的时候上你的节目好像就谈过了。然后, Anthony Fauci 到了夏天,6月到7月的时候,人家问他说你推荐美国人怎么样防治新冠,他当然就说戴口罩,然后隔离。然后他也是完全一样的,他开始在六七月的时候才开始讲Vitamin D,他推荐所服用的东西就是一个Vitamin D。



王孟源 24:18 

刚好上个月一个月前有一个最新的研究看出来这个是在西班牙做的。这篇论文是西班牙人写的,他们到西班牙因为西班牙疫情因此很严重,在4月、5月的时候非常严重,他们去看看那些统计所有的重症病人,看看他们是不是有维他命D欠缺症,结果发现 80\% 的重症病人有维他命D(欠缺)。换句话说,如果这些人有听劝告,早先就吃了维他命D,他大几率就不会进重症病房。这个 80\% 是什么意思?这个刚好上个礼拜美国的一个疫苗通过了第三期的实验。它的有效率是多少?90\%。然后两天前俄国也说他们有一个疫苗已经可以公开发售,他的那个号称的有效率是多少?是92\%。那你想想看,大家花了九牛二虎之力,几十万个专家花了一年的时间,那么多国家花了几十亿美元投入,大家强迫了头,发新疫苗,它的有效率是90\%。然后你有一个维他命D,就坐在那边,一瓶 10 块美元,大家从 1 月开始就可以简单地买来吃,它的有效率是80\%。那我是不是应该,特别把它提出来强调?



王孟源 26:07 

这就是为什么你可以看到。我 99\% 我讨论的话题都是政策性的建议,和我几不几乎不讨论这些私人的建议。我讨论过的个人的建议只有,第一个,你不要吃糖,因为吃糖是糖尿病跟肥胖症之源;第二个是你必须要,你如果担心新冠的话,你应该吃维他命D。他们在政策性的,他们不但是政策性的建议,就是我去谈他们其实是政策性的观点,但是我去特别把它拿出来跟读者说,你们应该要考虑参考这个,来做你的生活中决定的参照,是因为他们真的是对人命关天。我把它当做一个例外,我一般不是那种以前报纸里面那种写生活问答的那种人,我讨论的事情都是政策性的东西,学术性的东西。但是会谈到这种私人生活上的东西,我特别把它做了一个例外。这例外的原因就是——这个少吃糖跟多吃维他命D这两件事情——那这个当做例外是原因是,第一,他们原本就是政策性的建议。第二,人命关天。你如果有1万个人听我建议少吃糖,可能就有 5000 人少糖尿病;如果有1万人听了我的话,去吃维生素D,他们如果感染了新冠病毒,那么他们重症的几率就减少了80\%,对我来说是一个很大的功德,所以我才会特别破例。



其实为什么现在的这个读者对这一类的建议、生活建议会这样的排斥反感?就是因为现代社会骗子太多,对不对?没错,你不觉得吗?我跟你提过好几次那个Piketty的 21 世纪资本论里面,谈到自由市场经济有一个先天的趋势,会造成贫富不均,财富集中,贫富不均。其实有一个类似的现象,就是自由市场经济也会加强信息量的不平均。在英文里面叫做 information asymmetry,为什么呢?因为information asymmetry;是超额利润的根本。



王孟源 28:57 

所有的企业要赚额外的利润只有两个简单的方法,一个是 monopoly 独霸、独占性,另外一个就是信息不对称。我刚刚加入金融的时候, 20 多年前,快 30 年前我加入金融的时候,我的老板,当时的老板,第一个老板跟我讲,说去进了金融。我看你好像一个老实人,什么事情都是讲究脚踏实地。但是我要跟你讲,我的一个信条是以前的 Citi Bank 花旗银行的董事长讲的,他说信息就是金钱。



王孟源 29:39 

所以我刚刚讲了半天,说我的知识是到学术界的主流,那照理说这个主流的知识应该是全社会都知道的,其实不是,往往是这些主流的正确知识是隐瞒在一大堆假讯息之下。这是因为商人他们为了自己的利益,不但会遮掩这些真实的讯息,而且会不断地创造假讯息来欺骗群众。我刚刚谈的那个戒糖是减肥的的捷径,而且是对健康有益无损。美国的减肥工业,减肥业Diet Industry,你知道他们一年的收入多少吗? 72 个billion, 720 亿美元。如果我们大家就是公布说,少吃糖你一年就可以掉 25 磅,这样能够赚多少钱?你去看什么 Jenny Creig 或者什么的Zoom,美国现在最流行的就是他们帮你计划,你每天吃什么,OK?那个所谓的有效,其实不是他们有什么神奇的秘方,他们的神奇的秘方就在于他们的食谱里面没有糖,但是他不跟你讲。有 720 亿,大家分的赚不是很好?



史东 31:10 

所以一个既得利益集团的形成,形成之后是很不容易被打破的。



王孟源 31:17 

所以我们要了解,自由市场固然是促进经济发展的一个必须要的体制,它本身内涵的很多的缺陷,除了财富集中这个现象,就是Piketty 一再讨论的也是为什么人类贫富不均会是 21 世纪一个一大问题之外,这个信息的不对称也是一个很自然的结果。那我的使命就是因为我说我强调事实,跟着事实与逻辑,所以我出来讲这些主流的消息,像我刚刚讲那个糖或者是那个维他命D 这些事情,这是主流的意见,不是我的意见。我王孟源 一点一分钱都没有办法赚,而且我自己也没有做过什么实验,我完全就是看论文,我追寻的只是主流。至于这些主流的知识,为什么在社会没有人谈,这个背后的动力跟机制是自由市场哪些商人逐利的结果。因为卖维他命D,赚的钱远远比不上卖其他什么莲花清瘟。但是我觉得很糟糕的是,为了私利撒谎的现象在最近这几年也传到了学术界。



王孟源 33:22 

而且我现在先不谈中国,先谈美国。美国照理说是学术界要比中国先进成熟,而且廉洁很多。在 40 年前,如果有人被抓到论文造假,他会马上被开除,而且一辈子整个行业的人不会跟他谈话;

30 年前,如果有人被检举论文造假,而且证据确凿,他很可能会被停职调查很长这段时间,然后没办法晋升,有可能被开除;20 年前如果他被检举的话,他所属的大学会出来调查,但是这个最后的结果会怎么样就很难说了……他可能会拖个一年两年,看看公众是否淡忘了,是否有必要再真正采取真的措施,但是他绝对不会敢继续搞这种东西;10 年前,如果有人被抓个正着,大学还会装模作样地说,我们要调查,但是不一定会真的调查。他们只会开个记者会说我们会调查,我们听到了,我们会调查;现在 2020 年,如果你在美国出了这种事情,那个大学根本就假装没听到。



比如说我举一个例子,这个是上个月被揭穿的,去年的诺贝尔奖得主 John Hopkins 的名教授,一个他自己研究所的所长,名字叫什么?叫Gregg Semanza;被人刨出来几十篇论文,过去 20 年的几十篇论文,那个资料都有造假的,而且是很明显的造假,你只要仔细去看就可以看出他的那些照片,都是被 PS 过的,就是 photo shop 过的。结果闹了半年到现在, John Hopkins还是假装没有听到,没有任何动作。他真正开始大批量造假是他已经功成名就。到了2007年,

一下子批量了十几篇那个造假的论文出来。你知道在生医届,他们的那个发论文的惯例是指导教授是最后一个,就是所谓的 senior author,真正做动手去做的那个是 first author,排名第一个。OK。我注意到这些 2007 年的那些造假论文的 first author 是谁?第一个叫做张华凤的中国女研究员,而且基本上每一篇她所做的论文跟 Semanza 做的论文都有问题。她现在是科大的教授,合肥科大的教授。中国有反应吗?没有,一点反应都没有。这个 Samana 这件事情闹出来以后,张华凤一点问题都没有,连这种那种学术界的正义之士,好像美国这样子学术界的正义之士在网站上口诛笔伐都没有。我觉得这是很严重的问题,因为美国是一个衰颓中的帝国,它现在的这个腐化的,我们看到的是它已经严重腐化的后果。中国是一个新兴的工业国家,它应该一切都是朝气蓬勃的,一切都是更加理性、更加合理,更加高效的。但是你看他在学术诚实、诚信上的表现,比一个衰退中的老帝国还不如。你说这是不是一个很严重的问题?我在我的博客上口诛笔伐了好几次。



我最近在中国的网站上,媒体网站上看到有一大批中国的读者,就是很明显都是大学程度的以上的,他们都相信一个西方古文明造假论,这是什么意思?就是说他们相信全世界的古文明只有中国一个,埃及跟两河流域都是造假出来。你说这离谱不离谱?他们相信整个西方青铜器时代所留下来青铜器只有70公斤不到。这么离谱的论述,我一开始还不晓得说怎么会无中生有道摇造出这种东西。同样的,我再强调一次,这个任何一个考古系的教授都可以很简单地站出来说,这是胡说八道,但是在中国这种扭曲的说法传了几十年,没有一个考古系的教授愿意出来更正。为什么?因为这个说法是有一个叫何新的教授在几十年前出了一本书,他基本上就是妄想出来的东西。然后他出来以后,学术界主流不去更正,他就成了社会共识。这是中国特有的问题,因为我在欧洲跟美国都——美国除了触及宗教的东西之外——这种学术界上学术性的东西,至少学术界的主流会出来更正。就是除了宗教跟商业界的原的话题之外,就我刚刚讨论的那个因为有商业利益的关系,中国的造假不但比欧美普遍,而且他们的动机也比欧美轻浮很多,就是自己高兴,因为我可以骗人,哈哈哈,他们为了骗人而骗人。



我刚好在一个礼拜前又看到另外一篇文章,这还是在观察者网的那种食品,就是他们请,一般是请学者专家写专门的评论的那一栏,就是他们最重要的那一栏有一个评论,他评论说民主政治只有在全民化之后,全民推广以后才会腐化。我说这很奇怪,你的事实证据在哪里?看了半天,他有 3 个事实证据,文章写得洋洋洒洒,我相信对那些历史跟政治现况事实不懂的人很有说服力。但是他的证据说穿了只有 3 个。第一个是罗马帝国,在第三世纪就是 212 年的时候有一个个Antonio constitution,就是它有一个改革,把公民权扩大了。然后这位先生就说,你看罗马在第三世纪有一个有名的衰落阶段,这个就是因为 212 年的时候扩张公民权,然后所以扩张公民权以后,这个民主的问题就都出来了。你看像雅典也是,他们民主问题一大多把,所以把那个苏格拉底也杀了。你知道苏格拉底是被暴民定罪,民主的投票以后定罪,然后又输给斯巴达,在那个很快地就衰落掉。然后他就说你看像美国,他这个邮寄投票,这就是一种扩大公民权的一个小手段,就光这样子就逐一毁掉美国的民主。而这个扩大邮寄投票是 1970 年代以后才开始适用的,而且是为了不可抗力那大规模推广,是奥巴马期间才推广。(我觉得)非常的奇怪了,我在美国住了三十几年,你这你讲的这一套跟我所知的美国历史完全没有关系。



王孟源 41:26 

这个人谈的美国历史他谈得头头是道,可是事实上完全是编造出来的。而且光从他真正批评的重点是美国的这个邮寄投票就会看出它是Trump的支持者,也就是一个美籍华人。那共和党那方面的,他支持 Trump 那方面。所以他引用的很可能是美国政治界的编造出来的一些证据。你知道,共和党那边特别喜欢编造证据,这个民主党这边的是白左,他们是用扭曲逻辑,但是共和党那边就比较粗暴一点,直接编造事实。他刚刚讲的这些论据,其实要反驳起来很简单,我写了以后,大概半页的反驳就可以了。第一,罗马帝国在安东尼跟屋大维的内战结束,屋大维称帝以后就不是真的民主了,那个时候从参议院,在有了皇帝之后,你参议院还有多少分量?那你的选票,你在那说这个,讨论罗马帝国的衰败是民主的关系,这莫名其妙。他们连那个参议院的重要性都已经在 200 多年前就已经没有了。第二个是,罗马帝国的衰败,大家公认只要是对罗马史稍微有点了解的都知道它是始于第二世纪末,公元 180 年, Commodos 继承了 Marcus Aurelius。你这些名字你应该蛮熟的,因为这是电影里面 Gladiator 那个里面讲的那件事情就是 Commodos 继承了Marcus Aurelius之后,罗马帝国就中衰了。这比。那个公民权的事情早了 32 年。第三个字,他利用雅典来讨论,说民主政治公民权扩散以后会出现什么问题。但是雅典并没有公民权扩散,雅典的公民权一直都是限制在上层阶级的,所谓的 Citizen 里面,你根本仔细稍微想一想就知道,他这完全扯不上他实际的论点。至于美国的话,美国的邮寄投票是南北战争的时候,林肯特别为那北军的将士发明的,然后到了 1920 年代就开放给Citizen来用,就是平民来用,但是的确是一直要有借口,有理由,有正当理由。就是说,譬如说你现在有出政府公务,要离家,不能够在镇上投票。到了 1970 年的改革,其实是加州取消了这个要求,所以邮寄投票成为美国人投票的一个主要选项,其实是 1978 年加州开始的,跟他所讲的不一样,比他所讲的要早很多。然后我又查了一下那个,Obama任期 8 年,这个邮寄投票所占的比率的增加还不如小布西最后两年邮寄投票所占比率的增加的份。所以那个也完全就是编造出来的。



王孟源 44:59 

我认为整个舆论它的意义是让公众了解社会的需要,然后能够讨论一个合理有效的政策来为全民的利益做最大化。你如果连事实都是编造出来的,你怎么可能得到正确的结论?没有正确的结论,你怎么能够做出正确的处方?所以中共一直到目前为止解决这个问题的方法,就是他们内部有自己的一套民主,是不对外开放的,你社会上可以胡说八道,不影响不直接影响他们内部的决策。但是我觉得长久下来这不是办法,尤其是学术界不能够容忍这样的。你这个中国的文人像核心或者是这位旅美的作者,先定下结论,然后去编造事实出来。他们的结果就是像我这样很简单的科普一个主流学术界的主流的认知知识,结果都会被人家当作异端邪说来看待。这个,你说这样子是一个正常健康的社会吗?



史东 46:23 

我觉得这个您现在谈到这个问题非常非常大的问题,也是一个非常普遍的问题。生活在今天这个世界上, 21 世纪每个人都面对过,接受到过,但是我觉得我很高兴今天把它提出来,因为我直觉的感觉,我觉得大部分人不知道这件事情的严重性。嗯,就是今天我们生活在一个信息过量的时代,但是在这个过量的信息中,假信息永远大于真信息大很多。



王孟源 47:02 

99.9\%。不但数量多,而且他们特别能够刺激大家的好奇心跟恐惧感。



史东 47:11 

他们是有目的,这些目的或者是商业目的,或者是政治目的就是他们,他们是有他们……我可不可以说有备而来?基本上他们是有目的的。



王孟源 47:24 

所以说有商业利益、宗教性狂热、政治利益的人斗争已经是够累了,我们这些在乎事实的人。你还有一大堆文人为了撒谎而撒谎,就是自己高兴造谣,你说何新除了卖书以外,他造这些谣有什么好处?你有什么好处吗?



史东 47:45 

卖书就是一个不错的动机了,哈哈哈,不是吗?至少对我来讲对不对?还有一点我顺便提出来,我对这个事情,就你谈出这个学术界的事情,我是在好几年以前,我曾经做了一期节目,我就谈中国的这个假文化。嗯,我觉得这是一个很有趣的现象,因为当然我的观察的国家不多,文化不多,但是我基本上是美国、西方和中国文化,我觉得中国文化是唯一一个我看到的能够把假文化融入在自己的生活之中的,就是他接受了假文化,他去应变假文化很具体一个现象就是中国的古董文化。



史东 48:35 

古董文化这是一个很具体的,它明明白白的接受了这个东西是假的,你自己要负责,你被骗了,不是骗你的人要负责,是你自己要负责,你要心甘情愿地被骗。这是一个很有趣的一种,我可不可以说是一种心态上的扭曲还是什么?还是我们这个国家历史太久了,我们就太事故了,我们就不得不接受这种所谓的假文化存在,我们的四周,存在在我们的生活之中。



王孟源 49:07 

我刚好有考虑过这个问题,我个人的看法是因为中国工业化的历史太短,你在工业化之前,农业的社会里面有这些骗子,他们能骗的是谁?就是那些有钱人,那些有钱的财主、地主不在乎这些事情,但是在一个工业社会,你如果还是这样的骗子充斥,就会让社会的诚信度降低,最近连美国他们也自己开始反省,那么为什么他们的社会上的信任度在过去这 20 年降低得很快?其实是一个是国家腐化的很严重的现象,而它真正的意义,直接的意义也就在于一个工业社会里面,如果大家能够有较高程度的诚信的话,那么效率就会提高。因为你要防止人家骗你,你跟人家做生意,或者是即使是零售跟顾客交往,你要担心顾客占你的便宜,或者那个顾客担心卖家占你的便宜,对不对?卖家总是——尤其是卖旧车的——总是有信息不对称,对不对?你知道这个车有什么毛病,但你如果有……你说我们至少 30 多年前,我来美国的时候,那时候向往的欧美先进社会是什么东西?我来这边的时候像什么电梯、高楼,我在台湾都看过了,差别在于什么?在于美国人讲诚信,至少 80 年代的美国人底层,他的诚信程度比台湾要高很多。比如说你到商店里面去买一个电器,然后你用了两天,你不高兴了,你可以拿回去还,他让你换,我让你还钱,就他信任你。你是真正当初买的时候的确是有考虑要把它留下来用。而不是这个周末有一个重要的节目我要看完了以后就完。这是诚信。这个要能行得通,必须要这个社会绝大多数的人都有,都愿意遵从这个不成文的互相信任。那你如果没有这个诚信,也可以搞工业化,对不对?你中国也工业化起来了,但是在进一步提升的时候,这都是一个无形的 friction, 阻力,因为我们要防贼是必须要有投入的,要有人力物力财力的投入了,对不对?但是这些或许可以用高科技来解决,就是你如果一切都是在网路上做,但是在学术上没有办法,你 AI 不管做到什么地步,总是只要人类还是有参与。在这个经济运作里面,你最尖端最难取代的就是学术的尖端研究。那你在学术界如果大家都是忙着发假论文,然后搞升迁,那,那你这个学术界基本上不可信任,不可信任以后整个学术界都没有用。



史东 52:36 

你可以想见,去看一个医生,错了,发觉这医生的这个文凭是假的。



王孟源 52:41 

对,如果百分五十的医生文凭是假的,那你敢去看医生吗?你敢去赌那百分之 五十 吗?那剩下那些百分之 五十诚诚恳恳做事的医生,再怎么努力也没有用。中共目前就是非常的放任,非常地放着这个学术界,让他自生自灭去腐烂,因为他们真正重要的途径是由他们内部的、内幕的幕僚做出来的。。



史东 53:10 

我跟你讲孟源,我稍微打岔,我跟你讲一个笑话。好,这也证明了中国人的幽默感,说有一个人他实在不想活了,在中国他去买了一份毒药,吃的毒药等死,如果他没死,因为那毒药是假的。结果这事情过来之后,他很庆幸,他很高兴自己没死,然后他就去喝酒去庆祝了一下,结果死了。哈哈哈哈哈哈,这个酒是假的,哈哈哈。



王孟源 53:45 

这种事在美国也开始慢慢发生了,我不知道你注意到没有,过去这 10 年整个这个程序,社会程序跟商业诚信的程度已经降低到一个程度,是很明显的不如30 多年前我刚来。



史东 54:00 

这个我绝对是跟经济的衰退有直接的关系。



王孟源 54:06 

整个是,我觉得是整个社会的腐化,就是超越贫富不均或者是商业文化本身。因为你撒谎除了原本的宗教性狂热、商业性利益跟政治需要之外,互联网也使得有了为了骗人而骗人的那种变态快感——我还是觉得核心或者是那位,那位拿罗马的 Antonio constitution 来说事的那位作者是变态。你拿的你编造事实来骗人的,对你有什么好处?那他真的是社会……损失这么大,你自己又没有多大的利益干嘛?真的是一种病态。刚好上个月我也看到,我每天大概看 200 篇文章,有论文,也有新闻报道。不过我要提醒大家,我不是单打独斗,我的读者之中有很多世界一流的专家,像几年前我想要讨论激光物理的时候,我拿自己所读的心得写下来,结果我的读者就跟我讲你自己的错了,他曾经是世界激光记录的保持人,哈哈哈。我讲我说我要求自己科普的时候达到世界学术主流的一流水准,这个是基本上是没有问题的,因为我如果我个人搞错了,我自然会有读者来跟我要发私信。



史东 55:53 

对,藏龙卧虎,哈哈哈。



王孟源 55:57 

大概一个月前我看到一个新闻,他提的是——其实住我家这里很近,就是 university of Connecticut 康州大学的一个教授——他原本是苏联的,后来移民到美国,他在 10 年前 2010 年的时候提出一个新的理论,他的名字叫做 Peter Turchin,他提出一个理论。他说历史上的很多社会不稳定,是过度教育而造成的, over education。我先解释一下他的理论。我并不同意他的理论,至少不完全同意,但是我先解释一下,简单介绍一下他的理论。



王孟源 56:43 

他说,你高等教育过度普及之后,你制造了一大堆自认可以参与社会上层运作的人士,但是社会上层的金字塔的尖端没有那么多空间可以容纳这些人,对,自然有些人会不满,这些人不满就会造成社会不稳定的。这个来源他特别强调——比如说像马克思,他讨论的是阶级之间的差别;那目前的像这是个民粹,他们讨论的是宗教跟国族之间的争辩——我觉得这个人所谈的这个理论很有意思,因为他讨论的是不是阶级之间的矛盾,而是上层阶级之间得志者跟不得志者。之间的摩擦。他认为比如说法国大革命,我们大家以为是那个巴黎的暴民起来,其实他们的领导是一大堆新闻记者,这些新闻记者当然是受过高等教育,他们是被那些新闻记者,还有文人给挑拨起来的。同样的太平天国的领导是洪秀全,是一个落地秀才,对不对?落地秀才当然是受过教育,然后你看 1917 年的俄国大革命,列宁也是一样,一个受过教育,高等教育的,他也是不满,然后流亡到瑞士去。大家看一下,很有道理。然后他就说在 2010 年他就预言说美国的社会因为太多大学教育——他这是有道理的, 30 年前, 1990 年的时候,美国有大学教育的人占总人口大概是 1/ 6,到 2010 年代已经提升到1/ 3 了,就是每三个人,这还包括老头子跟小孩,哈哈哈,每 3 个人里面有一个是在受过大学教育的。那你当然不可能说这些大学教育的人都去从事需要大学教育的工作——他就说这会造成社会的不稳定。



王孟源 59:14 

然后来到 2016 年, Trump 当选之后,他就获得一些媒体的注意,然后就有人说他这个预言到我们现在这个经过的Trump 的这四年族群分裂的问题,那我想指出来他的说法有几个问题:第一个是美国目前真正的问题,其实还是阶级之间的差别,就是……



史东 59:48 

你说因为阶级而造成的贫富差别,还是阶级差别?



王孟源 59:52 

对,就是阶级的贫富差别。你说 Trump 所造成的不满,那些不满的人并不是受到大学教育的支持,Trump的人刚好就是那些没有大学教育的底层阶级,他们是全球化的受害者,所以他们才会起来挑战主流建制派。所以你说你仔细看这个细节,你就觉得他说的那一套道理或许在历史上有些可以应用的地方,但是对当前的美国还没有直接重要的贡献。这是我的第一个评论。第二个评论是说,他的这个所谓的 over education 太过的广泛,就是没有不够精确。我认为理工的人才再多,教育的再多都不是问题。为什么呢?因为尤其是像中国,它是工业建国。那你要以工业建国的话,真正重要的不是你生产线上的所花的费用,因为你那个可以自动化;真正让中国的工业能够跟先进国家竞争,就是先进国家当然有先来者的advantage,First Mover Advantage,你要跟他们竞争,你必须要以更高的效率、更低的费用来跟他们竞争。这个过程中你需要两件事,一个是高度的组织,第二个是廉价的劳工,而这里指的廉价劳工是工程师,所以你如果要有廉价的工程师的话,你就必须要过度生产,过度教育这些理工科的人。所以你再看看这个 Peter Terching 他所谈的例子,那些他所说的 over education 的那些负面的例子都不是理工科的,哈哈,而是文人。所以他所说的这个位现象,应该所谓的 over education 应该只限于文科。



王孟源 01:02:05 

然后在这里讲到理工科跟文科的差别,这个讨论我其实在我的博客讨论过了,我们做学术、做研究,当然有商业性,为金钱跟权力的追求,但是在比较理想化的的架构下,追求的不是金钱或权力的话,那就只有 三件事,就是真善美。很简单三件事,真善美,那求真的就是广义的科学或者是工程,求善的是所谓的 ethic 伦理学或者是宗教神学,求美的就是广义的艺术。当然也有比较应用性的,就是像工业设计一站这种东西,这三个大类之间的分鸿沟是非常大的,因为你第一类求真的话,你这个绝对不能够有编造事实,绝对不能够有逻辑的谬误,否则你就会得到错误的结论。那这个结论如果错误的话,其他都无从谈及,这些是像什么?你要像工程对不对?你建一座桥,不管你那个桥设计得多么漂亮,如果它隔了两个月以后就会倒塌的话,这一点意义都没有。但是你如果是一个写小说的,谁管你写的那个东西是不是真的对不对?你只要写出来能够感动人心,你用的名字是假的,地名是假的,环境是假的,时间是假的,历史是假的,都没有关系。这个我们叫做 artistic license。艺术家的撒谎执照,撒谎的执照。



王孟源 01:04:12 

所以基本的原因就在于他们追求的是美,而不是真,对不对?他们追求的是。要感性的触动读者的。情感,而不是理性的追求事实真相。



史东 01:04:28 

其实这里面有一点我觉得很有趣,同一个读者或者观察者、观阅者,或者你如果是一看一幅画,或者你读一本书,读者并不要求这本书里面的每一个形态都每一个状况都是真的,所以这是一个相互之间的一种。怎么样?是不是相互之间的一种理解?有点像我刚刚提出来的中国人对古董的这种对这种想法。



王孟源 01:04:57 

现在的问题是说有关政治,跟历史的跟社会学的分析,是属于真的那个范畴还是美的那个范畴?我认为OK,我百分之百的确定它是属于真的范畴,因为你政策选择错误,社会风气不利经济发展,或者历史的解说扭曲了人心,这些是有实际人命后果的。你必须要坚持事实真相。不管你有多大的欲望,说想要把它美化,你必须要考虑这个。他的代价是不是小于美化之后对社会的贡献,而这个贡献不能是一个人主观的。



史东 01:05:47 

对,另外还有一个角度会提出来探讨一下,你谈到美这个事情,我个人觉得真正的美和真是站在一起的虚伪的美是不会真正感动人的,所以这个就是另外一个层面的事情了。就是说你真正的,你说毕加索的这个这些图画他是忠于他自己的,就是从他自己的角度来看,这是真的,对不对?当然他画出来的和外面这个形象完全是两回事,但是那个你可以说那不是真,但是对于他自己来这一个创造者来讲,他必须要忠于他自己的感觉,那就不是就是真的一部分。



王孟源 01:06:29 

但是问题在于,你可以争议这一点,但是艺术的价值不会因为这个争议而被抹杀掉。但是一个政策的讨论,如果它基于虚假的事实。



史东 01:06:46 

对正确这个就非常容易讨论了。这个如果这个在不是不容任何虚假,如果有任何虚假的话,就像你刚刚那个桥梁的那个比喻,如果有任何虚假,这个政策不能执行,如果执行的话也会出问题。



王孟源 01:07:03 

我们已经看到这种社会上不诚实的讨论会造成什么结果,你看美国是这样子、英国也是这样子,对不对?我希望中国作为一个新兴国家,照理是说一切都是更加理性、更加高效的,但是我没有看到这方面的优越性,所以我觉得很失望。



王孟源 01:07:27 

我之所以要讲那么多,是因为我不想让大家认为我在歧视文科生。中国历史上。主要是文人,这些文人有很多成功和了不起的政治家。你像董仲舒,对不对?这是因为当时的教育是全科教育,你不但要学文学,而且你还要学经世济国的学问。这就像我这也是为什么我刚刚半小时前特别把歌剧提出来讲,因为那是同样的例子,你在 18 世纪的时候,那个女高音是全科的,必须从最高的花腔一直唱到女中音那里去,现在你才会有这些分差。那中国古代的文人是全科的,所以他们必须要有逻辑能力,必须要能够,了解现实,而且并出做出正确的政策处方。你即使到现在,比如说有一位中国的教授叫张文木教授,我相信他也没有受过,我对他不熟,但是感觉上他也没有受过科班的科学教育,但是他的逻辑就很强,这是天赋。有些人就是天生的就这样。对,但是光是因为有少数人有天赋是不够的,你必须要有系统地、后天地加强这种对事实真相的坚持跟对逻辑思辨分析能力的建立。至于天赋,没有这些能力,没有逻辑能力,没有对事实的尊重,后天教育又没有受到任何的改进,那这些人如果硬是要来搞政治的事情,就是如果不成功,那这很可笑;如果成功的话就是全民的灾难



王孟源 01:09:31 

我举两个例子,这个新冠刚开始1月的时候我就站出来说,这件事情会是让世界对中国刮目相看,因为中国有最强的组织力,能够能够最好地应对新冠的疫情。但是在那个同时,整个中国的舆论界其实是人心惶惶的。那最大的问题就是有一个文人叫做方方,写了一系列日记,这是煽情的夸大事实,她只顾描述那个疫情的惨重,她没有考虑到这个病毒的本身的特性,你,死伤遍地是因为敌人强大还是你自己无能?你这个差别是很大,对不对?她完全不管这个,那像这样的人,他对生医一点概念都没有,你说她有没有发言权?没有。但是她对当时国家社会的影响很大。



台湾也有,龙应台就是这样子。龙应台还是到现在还是著名的政治智库成员。她有什么资格?他有什么知识?她有什么逻辑能力?她认为她还鼓吹全民都去开发感性的文学能力。其实以方方跟龙应台来说,即使在文学界来说,她们有什么成就?如果把他们放到唐朝的话,她们能算是李白?



史东 01:11:18 

其实孟源我要说的也就是这个,其即使是在你是一个文学,你是一个人,任何艺术创造者你都要真,别人才会看你的东西,才会欣赏你的,才会崇拜你的东西。如果你一不争了,你的破绽就出来了。今天大家欣赏孟源,是因为孟源真,就这个字嘛,对不对?但是我同意你,不同意你是另外一回事,这因为他知道你是从心里发出来讲的话,以真,所以说大家愿意花这个时间去看你,听你讲话,同意不同意我们另外再谈,至少真,对不对?



王孟源 01:12:01 

长相是父母给的,但是我要不要坚持事实真相是我自己决定的,对不对?没有人能够拉住我说你一定要撒谎,但是李白跟杜甫这么伟大的文学家,他们有没有说他们要干涉政治?他们敢不敢说出来品评政治?你说方方跟龙应台放到唐朝,她们的作品能不能放到唐诗三百首里面?



史东 01:12:33 

我觉得孟源,我觉得你难为这两位了,她们不是在一个层次上的。



王孟源 01:12:40 

所以那龙应台她们就相当于古代的那些打油诗人。那龙应台建议的就是全台湾教育,应该让全台湾的学生都做学写打油诗。你说这对社会经济,公共事务有什么助力?有什么好处?一点好处都没有!她们根本就没有资格来讨论这种社会上的公共控诉。你要讨论政治,是众人之事,讨论公共事务的第一个先决条件,就是诚实,必须要坚持对事实。



史东 01:13:21 

对我完全同意,不能够扭曲事实。



王孟源 01:13:24 

不能够忽略事实,不能够假造事实。



史东 01:13:27 

其实这是不是也是我们生活在 21 世纪之中的一个问,面临的问题?每个人都有一个手机,每个人前面都一个麦克风,每个人前面都一个摄像头,对不对? 

王孟源 01:13:40 

20 多年前刚互联网刚刚开始的时候,那美国人还在说互联网会推广民主,让我们的这个现代自由民主社会提升到上一个层次。我说你在胡说八道,哈哈哈,它推广的是愚蠢跟谎话,我从一开始写博客开始,我就强调事实与逻辑,然后跟那相反的就是谎言跟愚昧对不对?事实的相反是谎言,逻辑的相反是愚昧。我遇到这些谎言跟愚昧真的是,这样公开奋战了 6 年,实在是有点累。最近到那个观察者网上遇到这些公然撒谎的事……



——————————————



王孟源 01:15:00 

在两个月之后,就是以月的时候,英国经济会受到什么样的打击?你现在已经可以预见,就是基本上是全方位的垮掉。就是他的那个服务业,因为那个是金融,是必须被监管的,你如果不接受对方的法律,你的金融业就不能够互通;然后你的制造业,因为这个关税,还有那个所谓的certification,就是规格标准,这个你也没办法流通,连那个零售业都没办法流通。那至于那个农渔业那更不用提的,那个先天就是层层保护的。对,所以英国会死得很惨,1月真的会死得很惨。



王孟源 01:16:02 

接下来我们来看看Brexit。Brexit 跟欧盟。因为我上一次上你的节目谈这件事情已经有一年了,那虽然我所做的预测是百分之百的精确,但是有一些新的消息值得讨论一下。就是去年我说过他们原本在 2015 年年底大选的时候支持脱欧的还只有33\%。然后 2016 年1月欧盟通过了那个反避税指令,马上到了2月他们就说要公投。然后到6月夏天的时候,公投的时候就有 52\% 的人支持,就是说有 19\% 的人在不到半年 4 个月之内就被忽悠成支持脱欧。到 2019 年重新大选的时候,支持脱欧的党派只拿了 46\% 的选票,但是却占了绝对多数。以为他们在4 年多前公投的,之前是承诺绝对不会无协议脱欧。但是我在去年做分析的时候已经说过,他们实际上最主要的背后幕后的力量,所以就是种族主义、排外主义的红脖子。



但是后来那些英国的土财主控制媒体的那些财阀,在短短几个月就贡献了 19\% 的选票,所以他们后来也就控制了保守党,就是 Boris Johnson 的政府,他一开始还没有控制那个 Theresa May 的政府,到了后来把 Theresa may 换下去以后,就可以真正为所欲为。Boris Johnson 既然是他们扶植起来的,当然必须要首先优先尊重他们的利益,就是绝对不能够再尊重欧洲的法律,他是名义上是不能够再尊重他们的法律来换取主权,实际上是不能够让欧洲的反避税指令在英国执行。



那么这么一来就有一个很大的问题,因为在这个谈判过程中,欧洲、欧盟始终强调一句话, level playing ground 就是公平的竞争,公平的竞争就是你的那个监管的法条必须是一样。那这下一来问题就大了,这是第一个矛盾,第二个矛盾是北爱尔兰,这个我去年也是曾经详细解释过,就是因为北爱兰有很血腥,英国在北爱尔兰殖民的后果。到目前天主教徒跟英格兰教会的那个信徒基本上是五五开,所以曾经过多年的内战,后来他们有了一个和平条约,这个和平条约其实也是美国仲裁的。那我那个时候就解释说英国要脱欧的时候,欧盟的态度是说你要脱欧就脱欧,我们唯一的要求就是北爱尔兰必须继续遵从当初的和平协定,那个和平协定就是说北爱尔兰跟爱尔兰之间没有边界。



王孟源 01:19:29 

那这个在当时可以,因为英国是欧盟的一部分,所以本来就本来就可以顺水推舟,然后很容易做到。但是现在英国要脱欧以后,这个北爱尔兰的身份就很尴尬,那 Theresa May,没有办法跟欧洲达成协议的问题也就在此。因为您要欧盟坚持北爱尔兰跟爱尔兰之间不能够有边界Theresa May,任何有精神正常的英国人也知道英国本土跟北爱尔兰之间不能有边界,否则你还算是一个国家吗?对不对?所以Theresa May为了解决这个问题就就说那这样好了,你如果他们那边不能够有边界,我们这边也不能够有边界。没有边界就代表必须要有equal playing level playing ground同样的法条。



王孟源 01:20:31 

那么北爱尔兰就必须遵从爱尔兰的法条,那爱尔兰的法条就是欧盟的法条,那我们要跟北爱尔兰之间没有边界,所以我们也必须遵从北爱尔兰的法条,也就是欧盟的法条。但绕了一大圈下来,她就有了所谓的权宜之策,但是这些权宜之策就威胁到那个避税。逃税的问题就是要返避税指令,就是没办法摆脱返避税指令。所以 Theresa May 争执了 3 年,她始终没有办法让自己党内的人通过它的提议,最后把她换下来以后。所以去年9月我就跟你讲, Boris Johnson 既然是这些财阀的人,他绝对是绝对是不能够像 Theresa 内那样认真地去解决这些问题。



王孟源 01:21:21 

那你我现在讲了两个问题,第一个是不能够有真正的自由贸易,第二个是不能够保持英国跟北爱兰之间的自由流动,人员流动跟货物流动。事实上他在去年签的跟 EU 签的那个脱欧协约,说好去年年底正式的脱欧,但是有一年的缓冲期,在这一年之中,也就是今年必须谈出那些细节来,然后在今年年底换成它真正英国脱离欧盟以后,然后由这个新弹出来的条文细节来来处理这个双方的关系。



王孟源 01:22:02 

那在去年所签的那个大纲协议里面,他是说我们尊重,绝对尊重北爱尔兰跟爱尔兰之间没有边境自由流通的这个前提,但是我已经解释过很多次,就是这个在逻辑上是一个无解的东西,就是你在那之间,北爱尔兰跟爱尔兰之间必须流通,北爱尔兰跟英国之间也必须要自由流通,但是英国跟爱尔兰之间又不能够执行同样的反避税指令。这之间是有一个无解的矛盾的。Johnson是以为说,他可以假装他愿意在海峡上建一个边界,假装说他愿意分裂国土来骗欧盟,他以为他可以事后反悔,来撕毁这个协议,结果两个月前,他说好,已经快要到年底了,我们把这个撕毁了。他以为反正你又不能够拿我怎么样,我已经要无协议脱欧,我也不在乎你的自贸协定,你顶多就是我踢出去,本来我本来就是被雇来做无协议脱欧的对不对?跟你撕毁协定就撕毁协定。



原理是这样的,在美国最有势力的族群当然是犹太人,他们支持的是以色列,但是一个 close second 就是紧接在后的第二名爱尔兰裔的天主教。他们是一个半世纪前爱尔兰的那个potato famine马铃薯饥荒的时候,移民到美国去的,而且之后因为他们同样跟犹太人一样,因为宗教的关系,保存了他们种族宗族之间的那个紧密联络性跟本土的那个向心性。很巧的是, Bidon 本身就是一个爱尔兰裔人,他们在美国的政治能量是仅次于犹太人的。



王孟源 01:23:54 

当初爱尔兰跟北爱尔兰签订和平协议,美国是重要的推手,现在 Johnson 要撕毁这个协议,这是什么意思?就是北爱尔兰跟英国之间还是自由流通。那这意思就是北爱尔兰跟爱尔兰之间必须要建立边界,而且他还很不诚实,等会我会详考虑讨论,详细讨论他怎么样的不诚实来搞这件事情。但是这种事情因为爱尔兰裔美国人的政治势力是贯彻两党,不是只有民主党。当然对 Biden 是更切身更强,但是即使Trump也不敢说我就让你这样做。所 Biden 在当选之前就说,我坚决反对任何违反爱尔兰和平协议的政治安排。那这下一来英国不就玩完了?



王孟源 01:24:47 

因为他们原本指望的是无协议脱欧以后要拿到的替代性自贸协议就是跟美国。现在 Biden 就真的让这个 Boris Johnson 很难处理这件事情,我在觉得他已经无法全身而退,因为没有美国的支持,英国不敢撕毁这个和平协议,不撕毁这个和平协议就只好建立这个边界。以后那这个 Johnson 的手下的那33\% 的红脖子绝对不不会容忍这种国土被分裂。所以你为了 19\% ,只给你 19\% 选票的那些财主。而把他局面搞成这个样子,最后你得罪的是那 33\% 的铁杆支持者。这是我认为 Boris Johnson 很可能在明年前半就会下台的原因。



史东 01:25:45 

就是反正他觉得那个薪水不够。



王孟源 01:25:50 

对,他还以为人家看不出来。那很简单嘛,你很简单的道理就是骗人的花样。



史东 01:25:58 

其实他我照你这么讲的话,我看他也知道这条路走不通,就死马当活马医,反正就是这样走。



王孟源 01:26:07 

对他这个他现在的他那个心理我认为是这样的,他第一个他必须要尊重他背后的财主,因为他将来退休以后要怎么赚钱,还要靠这些财主帮他安排的,对不对?所以那是绝对第一位。第二位是如果在这个前提之下满足财主的欲望,也就是不让欧盟的反避税指令进入英国的话,他必须要让那些红脖子高兴,让他们高兴跟照顾英国的经济利益是两回事,他并不在乎英国怎么损失,那个英国国土分裂,这都没有关系;他并不在乎英国国土分裂或者分崩离析,大英帝国整个垮掉他都不在乎。他在乎的是选票,他只希望能够多做几年,你要多做几年的话纯粹只是哄这些愚民。选票的话,那你可以用fetch,就是英文里面就是蒙混过关。我给你一个具体的例子。



王孟源 01:27:08 

英国脱欧搞了这么久,他说那你可以跟欧盟之外其他的国家谈自贸协定,他谈了几个自贸协定,谈了一个就是跟日本的自贸协定。今年年初他们还在交涉的时候,有人有细节被日本人泄露出来,说他们花了很大的功夫在谈一项英国特产的Cheese,就是只有英国才产的,欧盟不产的Cheese。日本原本对欧盟进口的的那个起势的关税有一个固定的水准,所以那个英国如果对出口日本的那个Cheese也是遵守这个关税水准,那你如果是退出欧盟之后没有跟日本的自贸协定,那这个关税水准会提高。



王孟源 01:27:58 

那日本就说好,没有关系,何苦浪费时间,这样的吵来吵去,我们就干脆以前你作为欧盟的一部分,所获得的优惠待遇我照样给你。然后我所获得欧盟给的关税待遇你照样给我,就是保持原状,大家皆大欢喜。那结果英国的交涉代表说什么都要改。改的就是只有一点,就是英国有一个土产的那个Cheese,是欧盟、法国他们不产的。那他说要日本专门对这个Cheese低关税,这很明显目的是什么呢?目的是要有一项能够拿来做广告的项目,就是完全为了欺骗选民,他为了这件事跟日本吵了六七个月,到最后日本还是说不干,最后他们签了一个自贸协定,到我们现在为止细节还不敢公布,因为欧盟的经济体量是英国的 6 倍多,任何一个已经跟欧盟有自贸协定的国家怎么会为了跟英国的交易而冒跟欧盟重新谈判的危险?因为这些标准的自贸协定都有标准的条款,就是说如果你给其他的国家更优惠的条款的话我有权利可以重新要求重新谈判。这个条件对。没有精神正常的国家会愿意。



王孟源 01:29:32 

为了英国而牺牲跟欧盟的利益。我觉得基本上Johnson是四面楚歌,他在到处乱钻,结果到处撞壁。



史东 01:29:45 

我记得上一次我们谈这个事情的时候,你曾经说过一句话,说英国会变成美国的属国,现在还是朝这个路上走吗?



王孟源 01:29:55 

我很高兴你记得这句话。这件事其实也有一些新的发展,就是原本Trump的要求是说,你英国如果要跟我美国签自贸协定,一个前提是你不能够跟欧盟有任何协定,就是你一定要无协议,完全无协议脱欧。所以很明显的 Johnson 的在未来这一个多月的选择就是真正的,绝对的诚实的无协议脱欧,或者签一个很弱很弱的自贸协定。我想他是一直到 11 月初都还没办法决定要怎么做,因为如果Trump连任的话,他就会直接无协议脱欧,如果是 Biden当选, Biden 反而会鼓励他跟欧盟签一个自贸协定,他他就会选择第二条路,签一个很简单的、很基本的、没有那个反避税条例危险的、一个很简单的自贸协定。那现在的问题是你如果要签这样的自贸协定,任何这样的条约都在欧盟,都必须要全部 27 国的国会通过,那这个手续一般认为是至少要 5 个礼拜,所以你现在我们已经进入 11 月下半的话,那现在是你的时间真的不多了。所以未来这几个礼拜会有在英国脱欧这件事上,会有几个头条新闻,大家值得继续看戏。这个美国跟英国在过去这 4 年,四五年真的是很热闹很精彩的政治剧。



史东 01:31:37 

只能叹一口气,看听这种故事只能叹一口气,他们叫,吃瓜群众,我们就看戏。



王孟源 01:31:45 

在两个月之后,就是1月的时候,英国经济会受到什么样的打击,你现在已经可以预见,就是基本上是全方位的垮掉,就是他的那个服务业。因为那个是金融,是必须被监管的,你如果不接受对方的法律,你的金融业就不能够互通。对。然后你的制造业因为这个关税,还有那个所谓的certification就是规格标准。这个你也没办法,没办法流通,连那个零售业都没办法流通。那至于那个浓渔业。那更不用提的,那个先天就是对层层保护的。所以英国会死得很惨,1月真的会死得很惨。



——————————



(编注,此处逻辑上应该承接1:15:00的讨论,即视频的第2段)



王孟源 01:33:12 

那些被质疑的不是我个人——我说我独创的那种,有多个, multiple discipline,很多个领域综合起来所做的判断——而是既有的学术主流的意见,我只是转述而已。这个都要出来讲出来被人家质疑,我就是这莫名其妙,这个社会有病到这个样子。这种现象出现在美国和英国我还可以理解,因为毕竟是衰败的老帝国。你一个新兴国家怎么会是这个样子?这很难理解。



史东 01:33:47 

有老习惯的新兴国家。



王孟源 01:33:53 

好了,我这发牢骚发了很久了,我可以谈美国大选。



史东 01:33:57 

没有很好。其实今天这个机会我觉得非常难得。我觉得今天你的这个谈话非常赞,而且你今天这种谈话,老实讲用文字表现不出来的,对不对?这是我从一个制作人的角度来看这件事情。



王孟源 01:34:15 

我们谈谈美国大选。好吧,个三个礼拜前我上一个节目的时候我说。 Biden 会选上,因为它那个乌克兰门,强度不够,结果当时的民调是 Biden领先11\%。我们从 2016 年的经验来看,当时 Hillary 赢了2\%,但是还是输了大选,所以你可以估计大概普选票必须民主党必须要领先 2. 5\% 才能够打平。那我们现在看到 Bidon 基本上是胜了,他目前领先,今天是领先 3.4,我想最后可能是领先到 3.5 或者更多一点,但是 3.5 或者是甚至 3.9 好了,跟 11\% 还是有很大的差别的,所以这件事是值得讨论一下。就是说我的预测是对的,我认为 11\% 是太宽了,即使那个乌克兰门的的 impact ,影响完全实现之后,你还是不足以把 11\% 砍到 3\% 以下。那如果不会掉到 3\% 以下,那拜登就会胜选。那是我 3 个礼拜前的判断,但是因为我当时认为说误差应该只有在一个标准差之内,一个标准差就是3\%,所以我也认为民主党会拿下参议院,然后会全面执政,当时我认为他们拿下参议院的可能性是大约百分之三八十。



那我先说一下,这个是我在博客讨论的第三种,也就是拿这个,拿这些知识来对未来做预测。那这个是我所做的一个预测,在做预测的时候,当然世界的事物发生是有随机性的,我当时认为有 80\% 的可能,但是基于当时已知的民意调查等等,我们来看看后来有多少多大的偏差。现在已知的是参议院目前是 48 对50,所以并不是说民主党一定拿不下来,但是剩下的那两席是都是从乔治亚洲 Georgia 来的。美国的这种大选是各州各定他们的规矩,那 Georgia 有特别的规矩,就是看这种联邦的参议员选举,胜权者必须得到 50\% 以上的选票。



王孟源 01:36:55 

如果没有拿到 50\% 以上的选票,那么由大选的里面投票最高的两个人来做第二轮的,一对一的、独特的、这个是在1月初会做第二轮的、投票。那因为刚好 Georgia 有一个参议员退休了,所以他今年两个参议员的员额都空出来,所以有一个正常选举,一个补选。两个选举出来以后,第一个正常选举是现任的共和党参议员得了49.7\%,对民主党挑战得了 47. 0\%,所以都没有到50\%,他们必须要进入第二轮。他这个规矩就是头一轮选第一轮选举的头两名进入第二轮。可以是同一个党,如果头两名都是共和党的话,那你第二轮就是共和党人跟共和党人来拼,那蛮有意思的。另外一个比较特别的州是Nebraska,他们有他们自己的规矩。我待会提到,因为Georgia 到最后被Biden拿下来了。以我想看它是以0.3\%差距险胜那。所以这个1月初的参议员二轮选举,我们现在还不能够断定说共和党会赢,所以民主党还是有机会拿下来。不过你如果看看这些参议院的选举,它的误差跟民意调查比起来,比总统选举还要大。



王孟源 01:38:32 

我上一次上节目时候有提过,其实 2016 年的民意调查一点问题都没有。如果是看投票前一周到两周的时候,那时候他们认为 Hillary领先 6\% 到7\%,但是当时有一个十月惊奇就是那个她的电邮门,就是她的那个电子邮件的伺服器是放在自己家里的。那个被搞出来以后,希拉里的那个支持率一路下降,所以其实到大选前那一天她的领先率已经降到3\%,实际大选的结果是 2\% 这种差距 1\% ,所以你不能够说它有什么偏差的了。



王孟源 01:39:18 

这个基本上是很精确了,但是今年才是真正出了问题。今年我刚刚讲过,在大选前两周是11\%,然后出了一个乌克兰门,我那时候断定这个乌克兰门的影响不会不会到达8\%,就是不会把它压到

3\%,然后如果那个 Biden的领先小于3\%,这个大选就就会有问题。但是你如果看继续追踪这个民意调查,它是有影响的。而且除了这个乌克兰门以外,那个 Trump 的竞选团队还找了一个很好的着力点,就是有关煤炭还有石油。因为 Biden 的确是站出来说我要搞新能源,这个新能源当然产生的新的工作会比那个石油业损失的老工作要多,但是问题是那些老工作的工人亏损是他们的工作,而新能源的那个产生的新的职位,大家还不知道谁会拿得到,所以这也是民主直选制的一个内建的矛盾。所以这个攻击在比如说Pennsylvania就很有用。对,因为宾州是美国的页岩油的生产地,在德州也很有用。这两个都是很重要的battleground state。所以你把这个算进去以后,其实民意调查到了大选的前一天已经降到低于 8\% 了,我们算是 7. 5\%。好了,这个7.5\%。



王孟源 01:41:01 

跟后来现在我们现在统计出来实际上3.5\%,还有 4\% 的差别,这个我如果说是误差太大,因为标准差是3\%,而且你如果分开来看各州的话,它是很普遍的,太过普遍。如果是误差的话,应该是随机的,有的州会误差比较多,有的州误差比较小,但这不是比较普遍,很有一致性,所以比较可能的是偏差。我这里用的是科学的定义,所以我特别解释一下,不一定每一个读者都知道,在这个做预测的时候,所谓的误差是先天的不确定性,就是你做了这么多统计,做了多这么多研究。它有可能,但是这个误差的重点是它是两边都一样的,可能你可能说得多,也可能说得少。



王孟源 01:41:57 

那你如果是为了某种缘故而往一个方向有了错误,这个叫做偏差。Bias,英文叫Bias,误差是error,我认为这 4\% 里面有一半多是偏差的成分,就是Bias。当然有很多共和党方面的人在事先说Trump会赢,因为这个名义的调查里面的偏差太大了,事实上偏差并没有大到会扭转这个选大选结果的程度。这些偏差是来自什么呢?这个民营调查的方式是基于 1960 年代, 1970 年代的社会形态。那就是当时家家户户基本上都是中产阶级,中产阶级就是你自己有一个住房或一个公寓,那里面有一个电话,有线电话,这个民意调查只要打电话去问就行了。现在不是,现在大家都有手机,你可能有两个手机, 三 个手机,然后那个电话联系其实上是你手机的许多应用程式中比较不重要的一个,很多人根本就不回答陌生人打来的电话,这样一来你的取样它的统计意义就会远低于你事先预期的。而且Trump这是一些过去四年一直讲 fake news,然后 mainstream media,所以很多这些民意调查统计是这些主流媒体做的。所以共和党的选民先天对他们就反感,先天就不愿意接他们电话。那你可以想象,共和党的选民支持率有 47\% 点多,这里面只要有不到 1/ 10 不想接电话,就会造成那 4\% 的偏差。 2016 年选举之后,他们有人开始批评说现在的民意调查已经完全过时,那时候我还觉得 1\% 的差,差别不足以下这个论断,但是现在有将近 4\% 的差别,我认为是可以下这个论断,因为这已经超过一个标准差了。



王孟源 01:44:04 

尤其更重要的是,你如果看那个参议院的选举,在参议院的民意调查比他的那个误差跟偏差比总统选举还要大。我把那 12 个最重要的战场州列出来,这 12 个战场州分成三类。第一类是只有一个州,就是佛罗里达。这个佛罗里达它的妙处在什么地方?就是Trump的得票率四年前还要高出2.2\%。你会想说:四年前是Trump胜选,那今年是Trump败选,那它应该是支持率会降低。但是在这12个战场州里面,佛罗里达是唯一一个他的支持率上升,而且上升了 2\% 点多,这是在全国的支持率降低的那个背景之下,是很特别。然后另外有 5 个州是Trump的,那个支持率只降低不到 1. 5 的,这些州是像Iowa,Nevada,Ohio, Pennsylvania 跟Wisconsin。其他的六个州它的支持率都降低超过2\%,那就跟全国的那个统计差不多了,甚至更多。所以我们现在要讨要解释的问题是,为什么民意调查会出错?



王孟源 01:45:32 

这个我刚刚已经讨论过了。为什么 Trump 会只以3. 5\% 来败选,而不是我们事先预计的 8\% 或 7. 5\%。这个我想现在讨论一下。主要的原因是大家低估了共和党,就是那个蓝领阶级白人对共和党的支持率。第二个是我刚刚讲的有关那个能源的问题,那是另外一个对 Biden 选前对Biden打击的着力点,我相信它贡献了大概 1\% 左右的选票。第三个是年终的时候有那个 black lives matter 的侍卫,当时我还写了专文讨论,我说这会对 Biden 是利多。的确是这样的,但是后来 they overplayed the hand ,就是玩过头了,玩过头就是他们是去占领市区,然后打砸商店。这些对那些中间选民是非会引发很大反感的。而在我们看到选举的最后一个月,也的确共和党拿这件事来大做文章,这又可以贡献大于 1\% 的选票。然后最后一个出人意外的地方就是大家事先没有预期到 Hispanic 就是西班牙裔的选民会对共和党如此情有独钟。



王孟源 01:46:59 

大家认为因为Trump在过去 4 年,他是用过很多侮辱性的词,说他们是强奸犯。这种很政治不正确的那个字眼,大家认为说头脑正常的西班牙裔选民不会仅选它,但是你看事后大家去统计,发现民主党获得选票。最出意料、最失望的族群就是Hispanic,而且不(只)是Hispanic。再说得更精确一点,不是 Hispanic 的woman,不是女人,而是男人。这有两个原因了。



王孟源 01:47:41 

你刚刚我提到那个 Florida 是所有战场州中。那个Trump唯一一个 Trump 的得票率比 4 年前还高的,在佛洛里达州的Hispanic,他们是来自古巴跟委内瑞拉的,是当初社会主义或者共产主义革命逃来的难民,所以他们非常吃共和党的那一套。不管民主党的白左怎样争取,他们都不接受。所以共和党跟Trump在佛洛里达的得票特别高,是因为受他们的所赐。另外一个原因是全国性的,即使是那些墨西哥来的Hispanic,只要是男性,我们必须要了解,他们一般都是天主教社会,而且是大家族性的观念。



王孟源 01:48:35 

但是他们这种中南美洲的这种家庭教育的方式跟中国人不太一样,他们对女孩子管得很严,但是对男孩子很放任。结果就是 Hispanic 里面在学校成绩好,能够上大学,能够得到优待的机会,而上大学的大部分是女孩子。那些男孩子跑去做什么呢?做蓝领工人,就是我刚刚讲的像石油工人或者是做警察,也就是说跟蓝领白人同一做同行,他们认同他们的同事,认同他们的阶级利益。所以我想先对这一次大选做一个反思,所以这是对这个选举细节。我其实对选举细节没有什么太大的兴趣,因为我不是这方面的专家。如果从比较宏伟的观点来看,嗯,美国一直在南北战争之后,那个南北之间的分裂一直没有弥合,尤其是在 Lincoln 死以后,到了 1876 年他们的reconstruction。



王孟源 01:49:49 

这是那个重建,其实就是占领方占领南方所谓的美其名为重建,然后强制他们做种族平等等的政策。到了 1876 年那次大选以后就被取消了。南方基本上是实质独立,就是在政治跟法律上实质独立,在经济上仍然是美国的一部分。那个大选也很有意思,有兴趣对美国近代史有兴趣的读者,我很鼓励你去研究一下 1876 年的大选,因为那一次是美国大选里面做票最严重的一次,当时那个民主党跟共和党的那个根底还是——跟现在刚好相反,它是在那个 Nixon 的任期之内才交换的——就是南方白人原本是民主党的支持者,到 Nixon 之后才变角色互换。对,所以那个时候民主党代表的南方他们为了强迫北方政府联邦政府取消这个这个reconstruction的政策,大量地做票,就是想办法让那些黑人的票、还有北方来的所谓的copybeggar 的票不让他算。弄到什么程度?南卡开出来的票比总人口还高1\%,他们做票做到那个程度,结果是民主党的总统候选人眼看着就可以拿下来了,那个共和党人就说你这不行啊,你这很明显是作弊。那你后来民主党,就说我们在乎的是你要取消这个reconstruction,只要你同意取消这个 reconstruction。 我可以让你选上。他这个是密室里面搓圆仔汤搓出来。想要了解美国近代史,很值得去读一读。那一篇,那一个章节。



王孟源 01:51:50 

后来南方就继续搞他们的自己的文化,他们有自己的邦联旗。但是我想过去这 40 年有一个很大的转变,就是北方退休的人往南方走,还有一些新兴的工业在南方建立,对,南方的城市化程度比北方要高得多。程度当然是不够,但是它的城市化的速度,比如说像乔治亚洲的亚特兰大,现在是一个很兴盛的新的都会区,那这些很多移民是来自加州或者是新英格兰的,那他们就把他们的政治效思想跟他们的所受的白左的那种社会意识带过去了。现在美国,这也是为什么 Georgia 最后是 Biden 拿下来的原因。



王孟源 01:52:46 

他在北卡也很接近。而我刚刚提的 12 个战场周中并没有包括Virginia。你不要忘了, Virginia 原本是南方南军的首都所在,邦联的首都所在。所以我们现在可以安全地说,美国已经不再是一个南北的分裂,而是一个城乡的分裂,就是城市里面受过大学高等教育那 1/ 3 人口来对抗乡下跟小镇里面的那 2/ 3人口。那些没有受过大学教育的 2/ 3人格。



王孟源 01:53:23 

然后我想再提一下,美国大选里面大家都忽略的真正重要的一个结果。就是我上一次在来这里的时候也没有提,就是美国大也是每两年是选国会,总统是四年一选,但是它还有两件事是每 10 年,就是在那个上一次是 2010 年,今年是 2020 年,这是两件事是什么事?第一个是 Census 人口统计,第二个是选州议会。你说州议会也是每两年选一次,怎么会说 10 年?我说的是州议会选出来以后,每 10 年重定选区。就是我刚刚有提到过他们这个国会选举或者是总统大选,这个规则不是联邦统一定立的,而是各州自己定的。那各州由谁定?大多数的州是由州议会决定,包括那个选区的划定。那这就触动了一个我们很多人都知道的名词,叫做Gerrymandering。



王孟源 01:54:27 

美国的总统大选,共和党会有2.5\% 的固定优势,就是你民主党必须要赢到超过 2. 5\% 以上,你的总统才会有打平的机会,是选举人团的结构所造成的,这是一种联邦式的宪法上的的问题,但是在众议院选举上共和党有更大的优势,而这个优势不是来自于这个选举人团,因为那个众议院的选举是另外一套办法,它的这个优势来自于Gerrymandering,就是各州的选区划定。那这个选句划定我也以前也写过博文讨论过,但 Gerrymandering现在搞到现在,在10 年前有一次质的提升。为什么这么说?因为那个时候电脑化到了一个,就是你可以全自动地做 Gerrymandering电脑帮你优化到,因为电脑里的有了资料库,大到知道每一个选民住在哪一个地方,他的投票趋向是什么。那你这个  Gerrymandering可以有选区可以怎么划,可以把共和党的席次最大化。你这个州有多少个选区,是由我刚刚提到的 Census人口普查决定的。然后你在这个州内部这个众议员的有这么多众议员的名额,你这个选区怎么划?是由这个州议院来写。



史东 01:56:07 

我对这个事情我听过最,最好的一个比喻,让人家一听就懂。可能不是完全懂,但是这个利害关系可能马上就清楚。就是一般人家讲选举,是选民选候选人,Gerrymandering之后变成候选人选选民。是可以这么说,对,就是他把那个区域按照选民的投票倾向,把他按到他的投票倾向这幅画,所以说他的区域里面就是胜选的可能性是非常高的。



王孟源 01:56:47 

所以我给你几个数据,你应该就知道这个事态的严重性。 2008 年的比率是共和党14,民主党27,到了 2010 年变成共和党25。民主党16,然后到 2018 年共和党 20 民主党15,到 2020年今年出来结果我刚刚出是共和党23,民族党16。过去这十年民族党吃了大亏,一直到 2018 年才拿回众议院。首先人口普查因为新冠的关系,必须草草也草草了事,上个月刚刚草草了事,所以数字很可能是会低估新移民,这些新移民大部分都是Hispanic。那我刚刚也讲过Hispanic,虽然他的男性还是有出人意料的共和党力量,但是整体来说还是有 60\% 多,就是女男性女性一起算,还是有 60\% 多的选票是民主党的,而且越年轻的选票越偏信民主党。所以你那些新的人口增长其实是对民主党有利。那你这些人口普查草草了事,低估了这些新增的人口对民主党就不利。



王孟源 01:58:13 

这还是州与州之间分配的问题,你在这个州内部分配,我刚刚谈的是所有 50 个州来算。其实这个影响对深红跟深蓝的周影响不大,真正的影响还是那我刚刚提过那 12 个战场州。那我去看了一下, 12 个战场州里面只有两个州是民主党控制了州议院,另外 10 个州是共和党控制的,所以这 10 个州共和党还是一样可以为所欲为地做Gerrymandering。所以民主党过去这 10 年在众议院的选举战争中所吃的亏还是要再继续下去。



王孟源 01:58:54 

另外就是这次选举光是公开报告的投入的资金就是140亿, 14 个billion。你这个钱拿来做基建或者发给发那个 unemployment 多好。这还是直接投入的钱,不是大家间接浪费的工作时间。anyway,在外交上我上一次讨论的就是Trump是直接出手谩骂,然后用各种贸易限制来打击中国。Biden 会改为用于用人权来为借口来联合欧洲。我觉得这个判断没有修订的必要。但是在内政上,上次我说那个他们有可能会扩大联邦大法院的规模,从 9 个人到 15 个人来避免那个压倒性的多数,这个可能就困难很多了,尤其是参议院要看到……我们要等到1月初的 Georgia 的那两个席位的第二轮选举,才能够确定是不是可行。不过也可能因为取消堕胎和 obamacare 兹事体大,这些大法官说不定会良知发现,不真正动手。比起我三个礼拜前所说的 Biden 会全面执政,我觉得这其实是一个对中国来说是一个更加好的结果,就是他们双方的内耗会继续下去。就是 Biden 所能寄望最好的结果,就是在回归奥巴马的第二个任期,就基本上是仍然是被参议院掐着脖子。然后没办法任命他的自己的官员跟法官,然后对外的话他可以得到日本跟澳洲的合作,但是欧洲我觉得很就很困难。要判断欧洲……Biden当然会试出橄榄枝,希望重新联合欧洲来对中国做斗争。但是我觉得这不是一件很直接确定的结果。你从 EU 的观点来看,美国目前国内的分裂跟英国很像,就是我上次去年来这边聊 Brexit 的时候,英国脱欧的时候。我说。英国基本上是五五分,就是有一半的人想要脱欧,有一半想要留在欧洲。



王孟源 02:01:43 

欧洲在这一次的大选,虽然 Biden 盛选,他也清楚地看到有超过 47\% 的选民坚持地支持Trump,而且支持共和党的国会议员的更多达到百分之四八十九。那一个这么彻底分裂的国家。他的政策随时有可能会 180 度转弯,就是两年后或者是四年后,随时你说要让他在长期的战略上。就因为 Biden 这次胜选而全新的投入他的怀抱,我觉得是非常不智的。我认为他们这个美国跟欧洲联合对付中国,真正的危险还是在我上一次所提的在人权上面,左手如果那个白左的势力被忽悠了,就可能有些危险。否则的话,光从实际战略的利益来考虑,欧洲人,欧洲的那些所谓的 Realist 现实主义者会知道。不能够跟美国走得太近。



王孟源 02:02:55 

然后。大家都很好奇说这个 Biden 上台以后会不会追杀Trump。这个我也可以很放心地做出一个预测,就是不会。不会,是因为什么呢?因为Biden 已经讲得很清楚了,他说 he wants to be the president for all American。就是所有美国人的总统。那这个意思就是说那 47\% 的美国选民他也必须要讨好?其实Trump在闹这些,看到这些邮寄选票或者什么作弊之类的这样下来以后,他至少有那些选民的坚定支持, 47\% 的民意在他的背后。那这样子你 Biden 政权就很难对它做追杀。因为政治后果太严重,会造成很大的动乱。但是Biden只能管联邦层次的。



王孟源 02:03:55 

我相信在州级别有一大堆民主党人不吃这套,他们会继续要清算Trump,尤其是他的那个逃税的事情。这个就是纽约州的逃税的事情,所以我认为这个会很有意思。好了,这个美国大选就讲到这里。



\twocolumn[\begin{@twocolumnfalse}
\section{美国权利交接与后川普时代}
\subsection{20210125}
\end{@twocolumnfalse}]王孟源 00:00 

我是真的希望中国能够好好地发展起来,取代美国的,为什么?这跟我是哪一国人没有关系,重点是,美国不是一个好人的国家,它的整个建国之初,就是以 genocide 种族屠杀来建立的,然后从此,他又学了(英国)……他这个霸权是继承英国,那英国的这些殖民统治基本上都是,每一个英国人的财富跟利益都是建立在几百个土著的生命跟福利上面。



史东 00:50 

各朋友你好,我是史东。在今天节目之中我们聆听到的是王孟源王先生,又是有一段时间和他没有见面了,我们想跟他聊一聊他最近的情况怎么样,以及他最近对一些我们都所观察到以及关心的一些事情,他的看法是如何,以及他的感想是如何?今天我们今天晚上谈的就是所谓的观后感,讲到这儿就和孟源打声招呼。孟源,非常谢谢,非常欢迎。



王孟源 01:18 

大家好,非常荣幸再上你的节目。



史东 01:20 

是恭喜你的这个新的麦克风。



王孟源 01:22 

哈哈哈,跟上个月跟,观网做视频节目的时候,我想是因为回声的问题,跟麦克风没有什么关系,不过读者还是普遍要求我升级麦克风,所以我还是买了。



史东 01:37 

这种事情多多益善,这个器材的事情永远不会错。我想大家这个可能也从你的这个这个文章上,网上的文章上看到了有关于您的近况的一点点描述,但是描述得很短,能不能再问我们谈一谈您最近的情况怎么样?



王孟源 01:58 

好的,第一个是我一个月前去耶鲁的骨科看了我的手,那后来换了不同的Bracer,这个 Bracer 比较有效,所以,而且做家事很方便,可以一天 24 小时都带着,所以生活品质一下子好了很多。好了以后我就开始准备要重新写文章。那刚好这个时候大陆有一个智库跟我联络,问我要不要加入他们。那原本我是有兴趣的,因为我每隔几个月就会有新的话题,那刚好是很很实用的政策建议。我想想,你想想看,当年两年前我们谈华为的时候,那是一个例子,然后最近刚好有另一个政策建议,就是有关科技部队学术的管理,我也做了文章。



王孟源 03:01 

这一些东西如果能够直接内参上去的话,是我自己心里比较安稳,因为我博客写了 7 年,一开始两年,主要是知其以不可为而为之,想要唤醒台湾的大众。其实我也知道台湾的现状是无可挽回的,这个非理性的趋势只有越来越糟糕。但是到了 2016 年,我想你也记得,就是有关大对撞机整个计划,那是无心插柳成荫,结果对大陆实际的政策讨论有了可以看到的影响,那对我来说真的是一个惊喜。



王孟源 03:46 

因为对台湾,我是知道你不管讲了多少实话都是没有用的,在美国讲实话是有生命危险的,哈哈哈,所以我甚至不用英文来发表意见。但是在大陆,你讲一些理性的谏言,居然会有实际的效果,对我来说也是出乎意料的惊喜。那后来我的文章转的那个视角越来越偏中国的实际政策的选择,那至少有几十个建议,后来都是隔了一年、几个月或者一年或者两年之后就实践了。这举个例子,比如说我批评他们的教育私有化或者医疗私有化这些事情,但是这些事情我不能够确定说我有间接的影响,就是我只能够影响舆论,我不能够影响政策。那不能够确定说这个一传十、十传百是不是真的有任何的贡献。我真正有可以确定有贡献的一个是大对撞机的那一次讨论了,另外一次是有关那个美国陷阱那本书自我介绍给华语听众,那我可以百分之百确定是我写的文章。然后后来观察者网也很重视,然后他们又进一步的去推,就全在全中国都有传开,所以这几年下来我是希望能够有内参的管道。那我也不在乎名利。



王孟源 05:30 

这个在过去我说过,从 2016 年开始知道说我自己能够对现实社会有真正贡献之后,其实我对自己设想的一个地位就是一个在媒体上能够拨乱反正。当然在大众媒体,现在的资本主导的大众媒体之下,你不可能对普罗大众做任何实际的教育,但是我相信有一小部分理性的知识分子,他们总是需要有人来引导这些话题来给他们做反馈,那我相信这还是会有贡献的。所以到两年前我才明白我应该追求的榜样是谁,就是Walter Lippmann,我不晓得你熟不熟。



王孟源 06:22 

20 世纪前半叶,一个很重要的专栏作家,他甚至没有硕士或博士学位,一辈子都没有进过学术界或智库,他连专,他连所谓的专栏作家都不是。像纽约时报这样的,他是一个 syndicate writer,是一个自由的专栏作业,然后写的文章,那些报纸高兴用就用,不高兴用就不用,他没有怎么牵连。但是他写的文章罗斯福一定要看,到后来毛泽东也一定要看。你说全天下有哪一个人敢说我的文章是罗斯福跟毛泽东同时都佩服得不了的,哈哈哈,我希望我自己也能够遵从他的榜样,能够达到那个高度。然后对真正有权利的人,就是全人类社会,不管是哪一个有权利的人,他们都能够听从这些从全人类角度来看最优化的政策建议,所以这个智库对我来说就是有一个可以上传的管道,后来跟他们聊了一下,我决定不加入他们,就是不挂名不收钱。那有几个原因,第一个是我现在还很忙,因为我小孩子还是在上网课,所以我必须要诱导他。那另外一个是因为刚好两个礼拜前发生 MIT 那个陈刚教授的事情,我不知道你熟不熟这件事。



史东 08:11 

陈刚不熟,这件事情知道。



王孟源 08:15 

他获罪其实没有什么我们传统上认为是间谍或者是违法的事情,他真正的原因就是他在中国的智库门大学任职,也没有传递什么真正的机密消息,他们也说不出是什么机密的消息。就是基本上这是一个美国对在美华人学者的另一个升级的 Prosecution 的另一个升级,从他再到我,如果我也去中国的智库任职的话,基本上只有一个差别,就是他是 MIT 的教授,我是一个退休的闲人,OK。但是我从两年前讲华为那件事就是美国陷阱那件事情,我已经一再的强调这个美国司法部去找华为中心的麻烦,跟 Trump 本身没有直接的联系。



王孟源 09:11 

Trump 是完全独立的,就是美国的司法系统,本身是就是检察官的系统,其实权力非常的大,他们的美国的法令细如牛毛,而且往往互相矛盾,然后法律又定的这个惩罚又定得很严,基本上他是给检察官近乎无限的自由裁量权。那你这当然不代表检察官是没有制约的,制约是什么呢?制约是在他的政客还有媒体上。那在目前中美争夺霸权的背景之下,基本上这种由政客跟媒体来制约检察官的这个机制,对华人完全不适用。就是反正他们就认定你是坏人,所以检察官要是要起诉你,那是你活该。那这就是为什么华为当初我就说这个,你像中兴那样子,寻他们的合法管道去申诉什么的,抗辩那个都是没有用的。因为大环境下已经把中国妖魔化了,所以那些检察官如果为了私利或者甚至就是为了业绩找你麻烦,你都是没有办法自我挽救的。所以我在美国也就是在半年多的时间,那没有理由冒这个危险,因为我们现在的这个全世界的通讯,其实 NSA 都全部全世界最大的超级电脑,就是不算计算能力,而算储存能力的话是 NSA 的一个计算中心。



史东 11:06 

我想是不是在山洞里面那个?有人报道在山洞里面那个很大的。



王孟源 11:10 

不是在山洞,因为它太大了,所以我见过。它是一个仓库型的建筑,比 Google 的计算中心还要大,是完全公费建的那个,所以全世界的每一个通讯它都有录下来。那如果,如果 他们为了业绩的原因,你随便一个中级或低级的官僚随时都可以跑自动搜索程式,那凡是,在中国的智库或者是大学任职的人都有潜在的危险,就是目前已经到了只差一级就会惹火上身,到这些人的程度,所以没有必要冒险,而且我也不在乎那个钱,也不在乎那个名,所以这个内参的管道就变成他们,变成另外一个我高兴发文给他们,就发文给他们(的管道),哈哈哈,就是我不受他们,完全不受他们的节制,就是他们完全是一个 optional 的outlet。不过,他们如果觉得要用的话,我必须要给他们 10- 14 天的 exclusive period。那这就是为什么我最近写的那篇文章。写原本要写的第一篇文章就是新年的回顾跟展望,交给他们之后必须要压着的原因。所以今天跟你谈的其实有些就是已经写下来的,两个礼拜前就已经写下来了,然后还没有发表。那我想先用口语聊一聊,也不算违规啊。哈哈,反正我想听众们大概也会有好奇的事情,而且现在国际的情势,真的,我从开始写博客就一直说我们生活在很有趣的时代,那 2020 年更是一个非常有趣的年份,我们现在……



史东 13:14 

这是毫无疑问的事儿。



王孟源 13:17 

对,到了 2021 年1月,我们还是有一连串的新闻可以讨论,所以今天就是想跟大家聊一聊一些事情。



史东 13:25 

你刚刚讲的这个华为孟晚舟的事儿,你刚刚提到。



王孟源 13:32 

已经 2 年了。



史东 13:33 

对哇,我真的日子过糊涂了,真的没想到已经2年。



王孟源 13:37 

哦,还好孟晚舟不是关在牢里。



史东 13:39 

那个,前两天我看了一个消息,说这个新一任的美国总统和这个加拿大总统有一次电话的联系,好像是和孟晚舟的案子有关系。我不知道这是一个,这是一个小道消息还是怎么样,但是我希望这是有一个比较正面的这个发展的可能性。我是希望能够这样子看到。



王孟源 14:02 

你刚刚提到你说的是那个 Biden 跟Trudeau的电话,是不是?对对,那是 Biden 跟国外领导人打的第一通电话。嗯,应该是只是问候性质的,因为实际上这种事情,还不是他们两个人之间交涉的最关心的事情。他们之间比如说最严重的事情是那条输油管就是 Keystone xl。那个因为 Biden是继承 Obama 的政策,就是民主党倒也不是说特别反对输油管,但是因为 Obama 已经做过决定,后来被 Trump 扭转了。那 Biden 为了不打他前任的脸,因为他前任毕竟有帮他助选,对不对?上台的话就必须要再扭转那这种事情。当然那个油是来自加拿大的,所以加拿大不是特别高兴,那这件事情才是他们两个人关心最重要的事情。



王孟源 15:13 

至于孟晚舟这个主要是司法部级别的人在搞,就是中级官僚在搞,那司法部的终极官僚其实大部分是民主党的就是,但是这个丑中的大环境,就是把中国妖魔化的大环境,是两党皆有的共识。所以其实 Biden 上台对它没有什么影响,就是他们要追孟晚舟还是会继续的追,如果不是更激烈的话。



王孟源 15:43 

现在的问题是,嗯,加拿大愿不愿意走那个程序。就是,当然在英美的这种体制之下的它其实是分权,就是分权以后你表面上有独立,然后最后才有一些制约,政治上跟资本在幕后的制约都是比较隐性,然后比较间接的。那这个法官本身就加拿大主案的那个法官,他本身当然也是生活在这个社会里面,受他们的主流媒体的洗脑,所以他的观点当然也是认定中国人就是妖魔鬼怪。那只要是我们的英勇的警察已经去确认这个是有罪的话,那么孟晚舟的情势就非常不妙。那这个律师,当然他们华为请了很好的律师,然后跟他们据理力争,据“法”力争,其实因为美国的法庭是不讲理的,那但是这个他能不能听得进去?其实都是还是自由裁量。我刚刚说过了,美国的这个英美这种法系其实就是专为资本设计的,就是它英美的政治社会法律体系,美一层都是方便权贵,就是权贵自动的免责,然后中产阶级如果真正在乎的话,有可以去争辩的自由,那你如果是底层没权又没有钱的话,根本想都不用想。因为你看像这种他们的这个法律系统基本上都是谁的律师好,谁赢嘛,对不对?如果你设计成这样,那当然就是……



史东 17:40 

我到美国来很多年了,我到美国来最早学到的经验并不是我自己牵涉到里面而得到经验,而是我的观察得到的经验是在美国是一个有钱就可以无罪的一个地方。



王孟源 18:00 

偶尔有例外,然后他们就会大吹特吹在纽约时报或者是好莱坞做电影,然后再是加以渲染修改,然后给你的印象好像就是他们都是无私的,然后大家公平的。其实刚好相反。



史东 18:18 

你刚刚讲到了这个陈刚,你有没有说什么什么想法可以跟我们分享的?



王孟源 18:25 

没有,因为这种事情,他们也没有公布细节。不过我看了一下,他跟以前的案子不太一样。以前的案子是说他们至少怀疑,说他们偷窃了或者转移了什么机密的消息,就是至少是商业机密。但是这一次他们讲不出所以然来,就是真正用上的罪名是逃税。多少钱?1万美金的收入。



史东 18:53 

就是欲加之罪。



王孟源 18:57 

这种收入美国人的国税局还真的懒得去追。



史东 19:02 

哈哈哈,我今天看了一个消息,今天还是昨天,我看的消息是哈佛的这个我不能说有关当局,哈佛的这个主管事情我忘了他的头衔是他说那个陈刚,他是代表哈佛。



王孟源 19:16 

陈刚是MIT。



史东 19:18 

对对对, MIT。MIT 他的这个主管事情说他跟中国大陆的这一个科技大学的合作,那是学校跟学校之间的合作,跟陈刚,不是陈刚个人的这个合作。换句话说,这个 MIT 是基本上是公开了为陈刚背书的。



王孟源 19:40 

在美国这边他要找你麻烦的时候,他所用的那个罪名其实往往都是莫须有的。OK?就是,而其是喜欢用那种你无从辩解,逃税是很常用的。更狠的就是像是你那个什么 Pedophile(编注:恋童癖),你看像那个QAnon,要说民主党说 Hillary 怎么样,就是说她在那个 DC 的一个披萨店里面有一个 Pedophile的Dungeon,可笑到极点了。



王孟源 20:14 

但是问题是怎么说?他们的那个主流媒体搞白左搞得,反智得太久,到最后这些红脖子他们隐隐也感觉到就是不太对劲,因为你说的这些事情在逻辑上自我矛盾对不对?而且你们讨厌的人用的是一套标准,比如说就是完全没有证据,只不过是随便找一个人空口白话,你们一样把它当做事实来报的,但是你们喜欢的人证据确凿了,你们也不报,也在那边,勉强要报也是说某人号称怎样这样。然后重点是在那个举报的人的问题。



史东 20:58 

对,这个情况已经是众所周知的一件事情了,我曾经我不知道在节目里面有没有提过,我曾经,提出一个问题问人家,我说如果水门事件发生在今天,你觉得美国媒体会报道这件事情吗?



王孟源 21:19 

这是美国政治水准的整体降低。



史东 21:24 

对,这是一个然后具体的,这是一个很具体的一个现象。



王孟源 21:29 

在学术界也是这样子,在那个 60 年代, 70 年代其实他们对学术不端的行为惩罚非常严厉,只要是有一点风声跳出来,这个人学术生涯就完蛋了。



史东 21:45 

就你讲到了,就是你讲过的这个所谓的这个,真和假之间的关系,对不对



王孟源 21:53 

对,他们那时候是很在乎真的。但是我也讲过了,这是因为 70 年代就刚好是水门的那个时候,刚好是美国的那个富豪开始成立一大堆智库,然后开始经过学术界,主要经过芝加哥学派。



史东 22:11 

他们开始发现他们的短处在哪里了。所以……



王孟源 22:15 

不是,他们是整个全方位的洗脑攻势,就是要推自由主义经济的那一套。对,然后它的那个有三个分支,主要分支就是以学术、智库跟媒体。你这个实际的目标是整个经济,还有政府,尤其是经济政策监管的那些事情。但是你要拆他们的台,必须把他们的根都挖出来,要挖根的话就必须要打破对事实真相的坚持跟尊重。



王孟源 22:52 

所以在 1989 年的时候,那个雷根已经经过了一根 8 年的任期,美国的政府已经开始腐化了,然后这时候学术界也开始腐化,一个分水岭的事件,我不晓得大家有没有注意,就是当时的所谓的 cold fusion 那个冷聚变,在那个 1989 年的事件之前,学术界有什么成果是不会发新闻搞公关稿的。



王孟源 23:26 

这种事情是企业界搞的,学术界不屑做这种事情,因为你这个做出来的东西对不对?要等到期刊,然后同行,行内人检讨之后,往往需要好几年甚至几十几年才知道是不是对的。那你怎么能够自己单方面的对普罗大众宣布说我已经成功了,然后往往吹得很离谱。但是他的第一个案例就是 1989 年的那个冷聚变,他还没有发表论文就先开了记者会,他当时也是受到批评,而且事后证明他们的确是错的。OK,但是这没有逆转,因为整个美国这个帝国的衰退是全方面的,而且已经是无可逆的。所以光是学术界几个比较传统的的大佬出来批评了没有用,就是很快的一个接着一个学校就沦陷了,大家然后大学也开始雇公关主管,对,这个专门就发这些……



史东 24:34 

一种坚持和一种腐败之间的选择,腐败是很容易的选择,坚持是很不容易。



王孟源 24:43 

对,所以我觉得中国去学这个非常的糟糕,因为中国现在是上升期,那美国在相当的时期就是 100 年前,其实是在严格的改革自己的学术界跟思想界,然后设定越来越高的标准,那中国刚好相反,就是他们的学术界。现在其实刚刚有一个案例,就是现在炒得很厉害,那个有一个名教授举报另外一个名教授的论文都是作假的,结果虽然证据非常的明确,科技部居然看来又来宣布说查无实据,证据就摆在你面前,你故意不看,那我也没有办法对不对?哈哈哈,那上个月我还啊,我受观察者往访问的,其实谈的就是假大空的问题。



(编注:此处谈论的是饶毅举报事件)



王孟源 25:40 

那这个其实是一个很深刻的问题,我认为是目前我刚刚说我要希望能够做Walter Lippmann,中国的Walter Lippmann,那我认为我可以改进的最重要的,最严重的当前的一个问题就是中国学术界的腐败问题,你一个正在上升期的一个新兴工业国,结果你的学术界比一个衰老的帝国还要败,而且是在 21 世纪,大家都是科技帝国,科技比拼国家。



王孟源 26:19 

你如果没有一个好的学术界的话,那你基本上永远都必须要引进外来的尖端科技,那就是那中国在过去的 40 年都是在跟,不过,那是没关系,你原本就落后几个世代,所以你去引进模仿都没有关系,但是现在你准备要超越人家了,结果你还是一个完全烂透的学术界,一个完全烂透的科技部,这是等于是准备要自杀的那个态势,非常的糟。



史东 26:51 

我觉得从一个中华民族的,大的角度来看,很需要一个像你这样子的一种,我不能说机敏不语而丰余?,这种人。就是你,你不受,这是很重要的一点,我必须在这指出来,我们两个互联。就是叫真,绝对要真,绝对要对得起自己,绝对不能受任何利益的牵扯。



王孟源 27:21 

其实这个在这,跟这跟时代、地方、国家制度都没有关系,你随便找一群。人类的天性就是这样子,有两个极端,它是一个钟形分布,钟形中型分布。那有好人就是在乎这些道德水准,他不愿意损人利己的,然后有坏人,他不在乎损人利己。而且很世上有些坏人是很喜欢损人利己,他们很有快感,哈哈哈,中间的一大部分是一般人,他们介于其中。



王孟源 27:59 

那我在我的博看博客上也反复解释过了,就是你的那个所谓的社会风气,学术界有学术界的风气,整体社会有社会的风气,企业界有企业界的风气,这些所谓的风气就是。中间的这些大部分的人选择要学哪一个极端?这是所谓的风气,但是你决定这个风气是往哪个方向走,是公权力、好人的奖励跟对坏人的惩罚来。



史东 28:34 

决定的。对我个人的这个有一直有个想法,作为一个领导者或作为一个领导人用大的,我们说国家类领导人,小的可以说一个公司的这个主事者,或者是你更小最重要的一个工作,作为一个领导人就是要制创造一个大环境,这个大环境就是你说的这个大环境,是一个量臂驱除劣币的大环境。



王孟源 29:00 

所以很多人说什么,那个董事长不懂得用电脑,或者是那个什么总经理不懂的,这个是行业的专业,不是重点。对,在他们那个层级的,本来他们的责任就不是这个,他们的责任是管人,OK,管人,所以他必须要能够把能人、好人提升出来重用,然后设立一个好的风气。但是科技部现在他对作假的反而包容,我觉得这是很好笑的,就是他在奖励坏人,然后他反而批评了那个举报的人,所以他是在惩罚好人。



王孟源 29:42 

那你想想看这个风气会是什么样子?那等于是把中间的这大多数的人都说,你们要跟的坏人学,你们都要作假,那这样子中国的学术界还有什么前途?你不要说是支持一个新的世界第一的强权,你连在后面跟,都会很累。我上个月在官网的那个所谈的,其实我认为打假这件事情根本是太基本了,连谈都不用谈,你就是一律从严就行了。你这个只要是证据出来了,这个就这个人就应该立刻被开除所有的职位。然后我觉得这种事情管起来,我觉得。



史东 30:26 

这种坚持。上次我们也谈到这个打假的事情,对假大控的事情,对不对?你的感触非常多,我也谈到中国的文化似乎有一种可能是太世故的关系,可能这种与假共存的这种心态。



王孟源 30:40 

就是和稀泥的心态。但你看看这种文化、这种风气,你放任了公权力放弃了之后,在雍正放弃了之雍正之后,乾隆放弃了,放弃了以后就是什么?你摸我的背,我摸谁?哈哈,背就算了。对,那这样子这个国家社会还怎么运行呢?你中国改革开放才 40 年,现在基本上还是中等收入国家。你已经这个样子了,你的学术界已经是这样子了,那对你还谈什么未来?你还谈什么什么发展?我是真的希望中国能够好好的发展起来,取代美国的。为什么?这跟我是哪一国人没有关系,重点是美国不是一个好人的国家,它的那个整个建国之初就是以 Genocide 种族屠杀来建立的,然后从此,他又学了(英国),他这个霸权是继承英国,那英国的这些殖民统治基本上都是一个人,每一个英国人的财富跟利益都是建立在几百个土著的生命跟福利上面,所以他们当时是奴隶制,一直到了 19 世纪解除奴隶以后,这个他们所利用的对象变成间接的,就是他们所谓的自由贸易,其实就是让他们的大公司去打击对第三世界那些脆弱的没有基础的小公司,然后来利用他们的劳工、廉价劳工跟原料。



王孟源 32:25 

这所以,我觉得中美的霸权交替对我来说意义最大的是,把这么一个邪恶的帝国替换下去,这才是真正重要的意义。那但是你如果中……因为目前我所看到中国,你看像一带一路的,他们这个都是比英美在道德上不晓得要高出多少层级的,这种事情他真正愿意,他帮这些亚洲、非洲的落后国家在那边做投资的时候,钱不是放到那些没有用的,纯粹是转一圈又回到英美大公司、大财团口袋里面的东西,而是真正去帮他们做基建,真正去帮他们发展新工业或者是现代农业。



王孟源 33:23 

像这种东西就是你如想象如果整个世界有一个开明的政府的话,他们应该要做的。就是美国,他故意不做,但是中国愿意做,所以我很乐意看到中美的霸权交替,但是这个前提是中国自己本身要能够拼得过竞争。这个我跟我的儿子解释过了这个种族歧视这件事情,还有英美的这个霸权的哲学,它的基本其实是一个很深刻的自卑,就是他不信任他们自己能够在公平竞争赢得过黄种人,所以必须要用其他的手段来确保自己的优越性。



史东 34:12 

他们好像也吃过几次从黄种人来的亏。对。



王孟源 34:19 

所以你看中国如果要能够继续推行这种王道的政策、国际政策,你自己必须要有很足够的科技发展能力,你自己的学术界必须要是世界一流的,那现在美国的学术界已经在过去四十几年衰退到这个地步,结果你比美国学术界还差一大截,甚至是两大截。这,这还得了?你这个作假由行内的专家、知名专家公开举报,你居然还敢就这样一手遮天?这个,这真的是黑压压没有天理的事情。你就是,我真不晓得共产党对那些高官如果明白他们的科技部在搞什么东西,那他们会有什么脸把自己叫做共产党人?你这个我觉得基本上,基本上现在就是学术界有了一大堆土豪,对不对?



史东 35:27 

这个事情我觉得至少需要有人把它指出来,这是第一步,是第一步。



王孟源 35:34 

我觉得,我,我觉得这些他们现在连造假这种事情都这样子公开的想要一手。



史东 35:43 

那就表示说他们的势力已经不小了。



王孟源 35:46 

他们何止是不小。就是科技部还有每一个行业基本上都是已经被收编了,不是已经收编就是被收编了,就是他们那个科技部的那个结论其实也是由行业内部多数投票选的,就是多数的教授已经赞成,就是因为靠,因为人情的关系他愿意。



史东 36:17 

这慢慢就跟美国很相似了。



王孟源 36:19 

美国还没有到这个地步。我跟你讲美国的学术界还没有得到。



史东 36:23 

美国的学术界跟美国美,我说美国的产业界跟美国的,这个。



王孟源 36:29 

美国的政治界跟媒体界或许是已到这个地步,但美国学术界他们也都没有这地步,美国学术界如果遇到这样的情形,他们顶多就是闭口不谈,假装没听到,但是他不敢这样子来公开地做一个整个行业的讨论。然后这个整个行业居然还说,诶,我们都没看到,直接这个,这一只鹿就是一只马。所以哈,这个是真的很可怕,我上个月还没有想到会是这么可怕。所以我还在谈说怎么处理啊,夸大骗经费的事情,那你如果连作假都不处理的话,那个夸大根本不算一回事,对不对?



史东 37:53 

我们换一个角度,我想和你谈谈这次解释,大家都非常希望能够透过您的说法来谈谈你对最近发生的一连串在美国的事情,大选,大选之后的交接,你的看法跟你的判断,我这几个问题。嗯,第一个就是这对于这次的大选和大选之后的交接,1月 6 号那次以及1月 20 号的交接,最令你难忘的几件事是什么?



王孟源 38:26 

1月 6 号的那件事情我有点意外,因为我没有想到Trump 会那么笨,OK,当然我知道他不是原本就知道他不是很聪明的人,而不是很理性的人。但是,1月 6 号先鼓动那个群众去冲击国会,然后他又故意不支援那些警察,所以就是不设防,让他们能够冲进去。然后但是他随即就躲回自己白宫的卧房去了,他说我要带你们冲,然后演讲完以后他就躲回去了。躲回去以后他真的以为这个冲击会有对他有正面的结果,真的是匪夷所思,蠢到不得了。



王孟源 39:11 

美国的政体其实基本上是大地主,当初在 200 多年前就是为大地主之间协商的一个平台,对不对?那所以你这个总统做完以后,虽然没有法律规定,事实上还是有很多不成文的的惯例,就是你他们不会去找你麻烦。就是当初大家对水门案义愤填膺的时候, Nixon 觉得他不能够信任这个不成文的惯例,所以他就安排让福特给他特赦了。



王孟源 39:55 

Trump 可以自己给自己特赦对不对?但是他这么一搞,这个给了民主党一个借口,就是你这次叛国,他们说用的字是insurrection, insurrection 就是背叛,反叛。一旦把他定位成叛国的话,那不秋后算账反而说不过去了。那你即使在联邦阶层大概没有什么兴趣,这个民主党的白左派也会要求州政府继续的追诉,尤其是纽约州的那个检察官,原本就对这种党政的事情很有兴趣。纽约的州警察官是一个黑人,他原本就是搞政治出身的,所以创的基本上是他这次这样的恶搞。是啊,对他自己也是相当的不利的。那基本上是把原本这些不成文惯例,君子之争的那种表象给他的保护他自己把它撕掉了。



王孟源 41:14 

所以我觉得,这是没办法事前预料的,因为你没办法事前预料一个人会有多么笨,你只能够事前预料。说最佳最优解是什么,那然后你希望大家都能够达到同样的结论,就是最优解。从大局来看,由于有这一次的这个动乱, Trump 基本上不可能在当选总统,他自己不可能再当选总统,那他的理想当然是能够继续驱动他的红脖的支持者,间接地来控制共和党。你说共和党的这些参议员中议员真正相信这些 Trump 式的trumpism,他们说Trumpism,特朗普主义, Trump 主义,这个特朗普主义,这个 Trumpism的这个民粹思想的有,但是主要是在少数几个众议员、参议院的那些出来跳得很高的那几个。其实都是怎么说, very cynical 犬儒主义的,就是他们知道这是不对,他们也不同意。但是他们看到他们的选民,选民,共和党内的选民对这个反应非常的激烈。



王孟源 42:46 

就是你看当 Trump 宣告说你大选是作弊的时候,一点证据都没有,搞了两三个礼拜,一大堆共和党人在州的级别去想要挖出证据,结果一个正当的证据都挖不出来,结果还是有 80\% 几的共和党人相信这是真的。那你可以想象如果 80\% 几的共和党人都相信Trump的是无辜落选的话,那你是作为一个共和党的一员,你能不能得罪这些人?这个跟蠢人打交道最糟糕的一个手段,就是跟他明白的讲,你所相信的话是蠢话,你所相信的是那个故事是假的。你这些参议员如果不跟 Trump 站在一起的话,就等于是自绝自己的政治生涯,那现你看这不是因为这些人蠢,也不是这些人坏,他只是照顾自己的利益对不对?那是这个体制强迫他们这样子,那你如果说了解到是体制的必然的话,那你怎么能够说这是一个好的体制?人家还说这民主是不坏的体制,他不坏是因为他们具有世界霸权,能够利用殖民的系统去对全世界吸血,所以看起来还不坏,暂时还不坏,但是其实是非常非常坏的一个。



史东 44:20 

你觉得 Trump 会不会另组政党?



王孟源 44:24 

这个问题我考虑了,就是刚刚我已经提过 Trump 是一个非常非理性的人,他做这些决定往往都是冲动,为了自己的虚荣而做的冲动。所以那个他如果要组政党的话,有可能。最可能的那个 scenario 就是,我刚刚讲的这些参议员,假如哪一个成为这些红脖子的新宠的话,那 Trump 就会因为虚荣跟嫉妒的心态会走更极端,那这其中就包括这个建立新党的选项。不过在目前为止,他还是他以这些红脖子选民来牵制整个共和党,因为共和党还有很多,除了选民之外,他这些选民只是他们的资源之一,就是他们养的一些牛群。这个实际上共和党的资源还包括地方党部的党组织,以及几千个几万个亿万富翁的捐款,还有他们附属的智库组织,那这还有企业。



王孟源 45:44 

所以 Trump 如果叛出共和党的话,那就是强迫这些人做一个选择,那这些人不一定会跟着那些红脖子选民一起走。所以在从理性的来看,他是没有叛出的理由,就是借着控制选民,可以四两拨千斤,同时获得这些共和党传统的资源。但是你如果是有其他的人威胁到他的权威地位的话,就很难说了,就是因为他是一个非理性的人。



史东 46:17 

我有一个问题,如果今天你有机会见到Trump,或者是有机会跟 Trump 说话,你会建议他往后的路应该怎么走?



王孟源 46:34 

well,我会建议他,就是做Kingmaker,对不对?就是他现在很自然的就可以成为一个所谓的kingmaker,典型的 king maker 是谁?波兰现在的那个那个真正的执政的人,他不是总理也不是总统。



(编注,这里应该指的是Aleksander Kaczyński,雅罗斯瓦夫·卡钦斯基,曾经出任总理,现在是波兰第一大党和执政党法律与公正党党魁,但是目前自己只是议员)



史东 46:57 

哈哈哈,很多人说那个英国的那个 Murdock 是一个 Kingmaker。



王孟源 47:03 

对,那是更间接的。对。



史东 47:08 

Kingmaker,你觉得他做得成吗?



王孟源 47:12 

以他的个性而言,就是他个性其实是成事不足败事有余了,所以。哈哈哈,你说要我给一两个建议,就是希望他能够……



史东 47:26 

上上策是 Kingmaker。



王孟源 47:28 

对,但是你可以看出Trump他个人本身,又从另一个观点证明,这个美国的这个资本主义体制跟直选系统体制是非常的恶劣的。一个这么蠢这么坏的人,又蠢又坏的人能够做到大富豪,然后从大富豪又变成总统,像这样的你容许这样的事情的一个体制,你还好意思说自己是全世界的模范?哈哈哈,是不要脸。



史东 48:03 

这是一个一针见血的评论。我再把话题稍微换一下,很多人都在看这个 Biden 上台之后,它跟中国的关系以后 4 年怎么走?中国往后跟美国要 deal with 美国的话,中国应该怎么办?我这个问题,这样子问好了,你觉得未来四年之中中国最应该注意的是哪些事情?



王孟源 48:33 

好,我在过去 5 年一直都说中国对抗美国的打击关键在欧盟,因为从一开始他们的那个建制派,就是尤其是民主党,就是还有他们的专业官僚,就是国务院的那些人,他们的这个计划都是要依靠制定新规则。这个制定新规则的妙处就在于,你如果制定一个新规则,把中国的体制的每一个细节都定义为非法的话,那中国就会进退失据,动则其咎,就很自然的就会自我崩溃瓦解。然后美国不但不费一纸之力,而且可以在过程之中捞很多的油水,就像 30 年前苏东崩溃那样子,这个是效率极高的方法,基本上是因为苏联的瓦解,是美国的宣传洗脑成功了。



王孟源 49:41 

就是我刚刚讲的,说“我们的体制非常的优秀,你们要学我”。其实是因为苏联没有那些殖民传统,而且它本身的那个领导人也出了一届布里兹涅夫(编注:即勃列日涅夫),那是非常糟糕的一个领导人。但是在过去 5 年我就一直强调,因为美国对中国打击最高效的手段就是制定新的国际规则。那当时在 5 年前,你其实已经看到,就是第一个他要跟欧国,欧洲跟日本,那时候已经定了时间了,就是在 2017 年的2月,好像我记得是 2017 年的2月,他们要开一个联席会议,决定要怎么样跟中国发最后通牒,就是要联合欧洲、日本、WTO、国际贸易组织什么的统统联合起来跟中国发最后通牒。然后与此同时跟日本、澳洲还有一些小国定TPP,就是那个 trans pacific 的自由贸易协定。那自由贸易协定这个是一个很特别的自由贸易协定,它不是像中国的去签了那些RCEP,或者是两国之间的一对一的自由贸易协,它里面的重点其实是一个仲裁庭,就是有一个超越国家的,超国家的法律机构。



王孟源 51:07 

那因为这个 TPP 里面,就把中国的这些国营企业统统定义为非法,然后再加上这个仲裁庭容许企业去告国家,就是把企业提升到企业法人,提升到国家的层次,那这样一来中国真的是死定了,就是你这个 TPP基本上就是样板,这个样板以后接下来再传播到欧洲,传播到WTO,那这样就可以把中国自然的绞死。



王孟源 51:43 

这个 Trump 上台马上就把 TPP砍了,然后跟欧洲日本的那个会谈也被取消,这对中国真是一个天赐的机遇。上一次有这么好的机遇还是那个911,就是我以前说过,就是在 911 之前。其实Rumsfield有迹象显示他是准备在 2003 年, 2004 年鼓动宣布独立,然后借个机会打一场胜利的,然后帮小布希获取连任,结果因为911发生了,结果他们的注意力转到阿富汗跟伊拉克去了,那是中国逃过一劫,然后 2017 年中国又逃过一劫,这次是Trump的关系。然后与此同时,在 2016 年英国也脱欧了,通过脱欧公投,那这个影响就是比较间接了。



王孟源 52:43 

2015 年的时候我还写过文章称赞当时的 Chancellor of the Exchequer,就是他们的副首相了,财政部讲相当于他们的副首相,Osborne, 就是Osborne,我说他是一个很硬的、传统的、睿智的政客,就是很厉害的,眼光玩手段都不错,尤其是有战略能力。其实是我对他的评价远高于我对丘吉尔的评价,因为丘吉尔他们把他吹得一塌糊涂,实际上是把是葬送英国帝国的人,你看一看那个历史上英国的帝国到什么时候结束,就是到丘吉尔结束,他把英国的那个的霸权给葬送了。但是 2015 年的时候我还写文章夸赞Osborne,到 2016 年他们就有那个脱欧公投,然后换上来的一代不如一代,他们从 Osbone 的亲中政策,他 Osbone 原本是欧洲最亲中的领导人,结果换上来,变成到 Johnson ,变成是最亲美的领导人。还好因为英国脱欧了。那英国在过去的三四百年一直是执行一个大战略,就是所谓的 Continental balance of top power,中文一般都是翻成离岸平衡手。这是什么意思?就是反正我就在大陆外面,那大陆上我绝对不容许有一个统一的强权,只要法国兴起了,我就联合德国去打法国。西班牙兴起了,我就联合法国去打西班牙。哈哈哈,德国兴起了,我就联合法国跟俄国去打德国,那这已经执行了几百年了,那他当初加入欧盟,其实他的用意是从内部去拆台,所以欧盟一直在过去二十几年,就是他们成立欧元区的时候,英国也拒绝参加。



王孟源 54:59 

这个就是把欧盟本身分成欧元区跟非欧元区了,这是内部的一种分裂,对不对?但是你一旦英国脱欧以后,德法原本就走得很近,就可以凝聚成一个真正的核心,那这件事决定就只需要德法两国同意就行,这之后就可以做决定了,对不?那然后德国跟法国之间,是德国是比较强势一点的,因为它的经济比较大,比较好,他的工业也维持得比较,维持得比较有远见一点。



王孟源 55:37 

刚好Merkel,你如果记得 Merkel 在 2006 年上台的时候,她也是很天真的,就是相信白左的那一套,但是她得罪了中国以后,他们的工商界去跟他讲,你不能够这样子,我们在中国的利益太大了。你不能够这样子没事去扯一扯人家的尾巴。后来她就学乖了,他就转换为对中国是以比较务实的态度,但是她没有真正的,那时候,还没有真正的智慧,她什么时候开始在这种大战略上面有智慧?第一个是 2003 年(编注,应为口误,实际上是2013年)的时候,她的手机被美国的NSA监听。



史东 56:23 

那个真是哈哈哈笑破人肚脐。



王孟源 56:28 

你知道在那之前她以为他跟 Obama 是好姐弟,哈哈哈,这下才知道美国人这个下三滥的态度到什么地步。然后到了 2015 年的时候,她最后一次听白左的建议就是——德国也是白左在欧洲的重点基地之一——就是你看现在德国的第二大党是什么?是绿党,绿党就是白左党。所以我一直觉得欧洲有三个白左国家,就是瑞典第一,然后捷克第二,德国第三。那所以他必须要顾虑这些白果的白主的势力。



史东 57:15 

那讲到这儿,我想还有一句话,你对于这个中欧木投资协定你的看法怎么样?



王孟源 57:25 

对,这个中欧投资协定就是 Merkel经过这十几年的学习经验之后结晶的结晶,最后的结晶。因为她一直到 Trump 上台,心中已经有一些怀疑了。对,就是就美国的那一套有一些怀疑了,但是还没有下定决心。是 Trump 这四年的不懈努力,对,你可以看得出来,每一次他跟 Trump 见面那个他的那个态度,又回退了一点



史东 57:53 

又往后退了一步



王孟源 57:56 

对,又往后退了一步。这退到最后 4 年,就是到去年的时候,塔已经知道这个欧洲如果在 patch on the back of the American wagon 就是继续在搭美国的大篷车的话,基本上没戏,基本上只是会是美国刀俎上的鱼肉,所以她是这一次中欧贸易协定的主导。



王孟源 58:29 

那这个我其实在去年中我就说过,因为当时有很多迹象就是欧盟的高官都出来说我们要正视中国的崛起,OK,等等等,那你要又刚好德国成为 2020 年后办的欧盟的主任国,OK,这是每六个月轮流联值一次,然后现在的欧盟的那个理事长是德国人吗?那么所以 Merkel这么一讲,你只要 Macron 不反对就可以过得去了。那所以我从去年中我就一直说赶快把握这个时期,哈哈哈。



王孟源 59:16 

快去去游说Macron,因为Macron这个人他对中国没有好感,就是法国人是非常自大的,这就是谁都看不对,看不上眼,但这个好处就是他连美国人也看不上眼,哈哈。所以你只要去跟他谈这些大战略上的事情,很容易说服他们。所以你只要……那个跟德国你反而不需要交涉,因为这种事情是必须要 Merkel 他自己学乖的。那 Merkel 已经学乖了,你不需要再去锦上添花。你去做外交工作反而太高调的话,反而会引起那些媒体跟白左政客的反弹,OK,但你真正突破的重点在法国,你只要那个Macron说好,点头说好,那么这个中欧的这个投资,双方投资协定就可以过关以后,这就代表 5 年前欧洲准备要跟美国、日本一起讨论的给中国的重最后通牒,中国给了欧洲了,就是你要的东西我都给你了,OK,但是我只给你欧洲,你日本跟美国别想。那为什么呢?就是我刚刚讲的,从大战略的观点来看,美中国所面临最大的危险是美国去制定新的国际规则,就是把中国的体制都弄定义成非法。那这个新的国际规则如果没有欧洲跟中国参与的话,美国跟日本、澳洲再加印度根本体量是不够,没有资格。对,所以你只要中国只需要把欧洲稳住就行了。



王孟源 01:01:00 

那既然欧洲 5 年前已经说我要这些东西,那你我把这些东西给你了,但是就只给你,哈哈哈,这个,这个是我在过去半年多一直讲说赶快赶快,要趁这个机会。这个很明显的,这个 Merkel也准也是愿意了,你只要把把控安稳下来就行。但结果事后好像是,我觉得中国人还是有点抠门,不等到那个大选结果出来,他们没办法下定决心。因为为什么呢?因为你这个大选如果是Trump连任的话,你就不需要,因为Trump会帮你做这些工作,你连这些对欧洲让利都不需要。哈哈哈。



史东 01:01:43 

算盘打得太精了。你是?



王孟源 01:01:45 

算盘打得非常的精啊,要等到这个,所以 11 月初他们大选之后,我为什么这么说?是因为到 11 月底好像是Reuter,Reuter有那个新闻报道说这个他们采访欧盟的官员,当然是匿名的采访,没有说是哪一个官员。他说中国是在 11 月中突然做了重大的退让,所以原本欧盟已经不做希望了,没有希望的这个谈判。诶,一下子很有兴趣,就是基本上欧盟所要求的东西,中国都说都点头说好,那这当然有点可惜了。我的,我当时去年五六月的时候,我是说你7月德国当了轮值主席的时候,你就应该认真跟他谈判。为什么?因为你这样可以跟他讨价还价,但是中方是觉得你如果Trump当选的话,连讨价还价都不用你,哈哈哈,连让利都不用。



史东 01:02:47 

就是他指望着 Trump 的破坏力。



王孟源 01:02:52 

对不对?指望Trump 的破坏力。那你知道我从我一直都相信Trump 是不会连任的,所以这也是我为什么可以提早建议的一个原因,因为美国有很多那个竞选的规则,从这个大局来看,基本上 Trump 是不可能连任的,就是唯一不确定的就是国会。那你如果回头看,我过去两年所写的文章都是这样预测的,主要是从经济层面就是Trump 他再怎么鼓动,这个在政治上玩弄民粹,他没办法超越经济上的规则。是选举,不论如何中间派还有温和派看的还是经济。那尤其一旦到那个新冠疫情发展出来,基本上完全可以确定Trump 会落选。



王孟源 01:03:40 

不过我当时的看法是,你即使Trump 当选,你也不需要去冒险对不对?你总是先把这个稳定下来,你这个大局就安定了,你过未来几年就在国际的形势上就安稳的多了,这是你的生命线,你应该是寻求安稳的选项,而不是去贪最后的一点那个经济利益。但是我想中方大概是觉得自己的灵活度够,所以可以等到 11 月中。



史东 01:04:16 

再做有关于这个中欧投资协定这件事情,你觉得对 Merkel 来讲是不是它的一个legacy?会不会变成 Merkel 的一个legacy?



王孟源 01:04:25 

这是我想这是 Trump 认为他自己留给整个欧盟最后的一个礼物。嗯,实际上我也同意,就是她刚上台的时候我对他的评价并不是太高,我觉得他就是刚好她的个性适合德国人喜爱,但是她本身我一开始没有看出什么智慧。她有一些像那个税务上的决定,一些内政外交上的决定,我觉得都是很因循的。老生常谈,事实上是错误的,但是她肯学习。她年纪很大了,但是还是愿意一点一点地学习。你看她做总理就是一年一年地在进步嘛,对不对?很可惜是他今年已经确定要退休了。



史东 01:05:15 

那所以说这个有关于中欧投资协定这件事情,我们讲了一些欧洲方面的事情。这件事情对中国的意义有多大?



王孟源 01:05:25 

我认为中国已经尽了人事了,但是还是有点反复的,可能因为 Merkel  要退休了,你接下来连Macron明年选举,也是……



史东 01:05:37 

看样子他的胜算不大,不会很大



王孟源 01:05:39 

噢哟,他现在民调的数字非常的糟糕,所以哇,当然还有一年我们不能够确定对不对,但是如果连Macron也下台的话,那这个 Biden就有机会见缝插针。对,我们再看这个欧盟,至少目前主导欧盟的那些官员都是明白事理的,就是已经跟  Merkel 一样学过同样的教训。但是最终欧盟还是由德法两国来主导,最重要的还是德国总理跟法国总统两个人对不对?刚好都是今明两年要换人,所以是有一点危险,但是这超过中国本身的影响力,这没有什么,你只能够尽人事,尽了人事听天命。



史东 01:07:18 

还有一个大家都很关心的话题,我想在这请教你一下你的看法跟你的意见。这个 vaccine 不同的 vaccine 现在都推出来了,不同的这个药厂,不同的国家,普遍的来说我想接受 vaccine 的人是不少,但是也有很大一部分人不敢接受vaccine。你个人对于这件事情的看法怎么样?



王孟源 01:07:46 

你觉得这是西方的反智思潮,在过去这四五十年就是我刚刚讲的,因为美国的富豪需要反扑,英国的富豪需要反扑,然后你就有这个财团媒体,像是 Murdock 这样子出来做他们的打手,所以这虽然是多方面的进行,但是其中最大的一个阻力就是反智。那你想想看这个多数决,还有自由主义、个人主义,还有这个选票至上论,这些都是其实是蛮新的,就是基本上是冷战的时候有需要发明的,然后但是被提升到这个神圣的地步,然后因为这个冷战美国胜利了,所以刚好可以拿来做宣传。其实是 70 年代、 60 年代、 70 年代之后富豪反复的有需要。那有需要的话,那你要对事实跟逻辑,对理性跟智慧做反扑,你需要什么?第一个需要先需要多数的普罗大众变成愚民,这在台湾也是这样的吗?李登辉的那个的那些政策是同一套。对,你先要把整个社会愚蠢化,然后愚蠢化之后你还必须要把整个政治哲学变成选民的偏好,就是超越一切的,超越事实、超越理性、超越他们自己的利益,就是他们他们的要求,他们可以要求来损害自己的利益。就是我后来才是说这些Turkey Voting For Christmas



史东 01:09:26 

就是这些偏好是不是被制造出来,被 manufacturing 出来,。



王孟源 01:09:29 

对,就是,你除了……你要能够让制造这些非理性,让他们自我损失利益的话,你除了要让他们蠢之外,你还要制定新的政治规则,政治跟社会规则。就是,就是如果选民的偏好跟他们自己的利益有冲突的时候,那一个优先就是他们的偏好。你看过去这 40 年,基本上他们的这些政治哲学、经济哲学都是往这条路走,就是选民的要求就是至上了。



史东 01:10:01 

选民要求什么?



王孟源 01:10:04 

选民要求至上。



史东 01:10:06 

选民要求。



王孟源 01:10:06 

对,至高无上。哎,至高无上,你即使是要危害他们自己的利益,即使是火鸡投票要圣诞节,你还是一样要把这些火鸡选民拔毛。



史东 01:10:16 

对,而这些就是你,我们也谈过这些舆论你就是制造出来的嘛?



王孟源 01:10:22 

对嘛?但是这个的用意就是因为资本控制媒体,控制舆论,舆论控制民意,对不对?



史东 01:10:29 

所以这在这个 vaccine 这方面,你觉得老百姓被愚弄了还是怎么样?



王孟源 01:10:37 

对,因为那个资本,我说过资本同时资助这个右翼民粹跟白左,OK。右翼民粹是附庸在基督教,白左本身就是一个宗教,就是,把科学的表象跟它背后求真的那个本质脱离,然后把科学的一些表象拿来做一个新宗教,就成为白左教。为什么呢?这个白左教对富豪来说也是非常重要的,因为这个所谓的左派原本是讲究平等的,这个平等的,所以左派原本是对富豪利益独占,在社会上利益独占一个最大的威胁,那你必须要让转移他们注意力,那转移注意力有好处,就是第一个这个白左不会去要求,不会把他们的攻击重点放在这些无良的富豪身上。



王孟源 01:11:33 

第二个是因为他们去要求,转而去要求的是一些傻事,对不对?鸡毛蒜皮的小傻事。比如说两年前有一个白坐的运动跑到那个中国城去,纽约的中国城去把那个卖活鸡的,从篮子里面放出来了,他们解放那些活鸡,你应该去解放贫民才对,不是解放活鸡。结果你做这种傻事的话,你就是这样,别人都看不起,不要说这个右翼民粹的、那些红脖的,就是中间派也看你不起。这样子你就不可能成事,等到你在做搞那种占领华尔街的时候就是。孤掌无鸣,没有人会去加入你,因为你干的这些示威的太多了,而且都是为鸡毛蒜皮的小事。那你再为真正重要的事情示威的时候就没人理你。



史东 01:12:31 

所以说你,你不需要打疫苗,你已经有抗体了。



王孟源 01:12:35 

我没有抗体,你没有。我,我虽然有症状,但是我后来到 10 月的时候去检查抗体。那这有两个可能,一个是当初的那个症状不是新冠,另外一个可能是当时的症状是新冠,但是后来抗体消失了。那我也不急着去发现,反正我还是很小心的,就是戴口罩,然后避免。



史东 01:13:01 

那你个人会不会打呢?还是你会不会鼓励你的亲爱的四周的家人打呢?



王孟源 01:13:08 

有机会当然是打,但是那个即使在这方面,你看看美国的那个在搞疫苗也跟中国不一样,就是美国人的疫苗基本上是看大企业方便。那大企业照理说,你如果是从国家的观点,国家民族百姓的观点来看,你是应该发展可靠、迅速并且廉价的疫苗,对不对?但是从大企业的观点来看,可靠不可靠他不在乎,廉价的话他反而少赚钱,对不对?快速的话也不是特别重要,反正你公关上吹答应说我要提供几千万、几亿,到时候再延,你又能怎么样,对不对?像欧盟现在拿不到疫苗,你又能够怎么把他们怎么样?那我说他们这样子做是什么意思?就是在实践上,他们实际上去觉得这是一个很好的机会,可以拿公家的资源,就是基本上政府,全世界的政府都没办法讨价还价,那所以很适合去由他们去做高风险的新科技,对不对?这种东西原本就很昂贵,然后风险又很大,所以对他们来说最好的就是把传统的疫苗弃之不顾,然后去开发这些高科技的



史东 01:14:35 

高科技,对他们来讲不是一个赔本生意。对。



王孟源 01:14:39 

成功了赚钱,反正到时候如果做不出来或者有什么问题,倒霉的又不是倒他们的。



史东 01:14:44 

输了,是国家的;赚了钱是他们。这是高招



王孟源 01:14:48 

对对对,你可以注意到那个中国的疫苗,开发出来就跟美国的疫苗不一样,美国的疫苗是所谓的 MRNA 疫苗,这是一个全新的,在这一次之前根本就没有前例。那中国的疫苗就是传统的疫苗,就是减毒或者是灭活的疫苗,这个这是我们人类做了快要 200 年的疫苗,都是这样做,对不对?那你说新冠这么严重的事情,你为什么不用这种廉价可靠、迅速的手段来做?光凭这一点就可以看出这个美国的这个体制不是最优的,远远不是最优的。你看连这么紧要的关头,他们那些关键的机构还是在想着自己的私利。



史东 01:15:34 

所以回到一个基本的问题,这个疫苗是打还是不打?还是你又打的好理由?不打也有不打的好理由。



王孟源 01:15:44 

最早一定是让那些权贵打,然后其次才是那些医疗人员打嘛,那我们这种一般的老百姓根本也不用急了,你要急也没有用。



史东 01:15:55 

他也没,也轮不上你,对不对?这个有道理,这是一个客观现实事情,这个决定。



王孟源 01:16:04 

如果你说事情我根本就没有再想。那你说真的,你说,好,我们现在也有疫苗,你可以参与,但是他那些医生跟护士都还没有打全,你说我愿意去打吗?我个人是不愿意的,因为我觉得他们才应该先优先来打,对不对?真是好,我就是老好人嘛。我想这种事情总是先以世纪社会公益。



史东 01:16:34 

结论:我现在想想问题就是说对于这个 2021 年的展望,你觉得有哪些是最值得我们注意的?



王孟源 01:16:44 

我觉得新冠疫情已经peak,就是大约 3 个礼拜前是最高点了,那因为疫苗已经出来了,那所以它这个最高点其实是因为年底有一大堆假期、圣诞节跟过年,所以很多人,尤其在欧美,他们根本不在乎这些规律,他们还是冒险,之后自然就会传播。那到了1月初、1月中是最高点,然后现在因为死亡率要比感染力要晚 3- 4  个礼拜,所以我们现在还在看着死亡率是现在正在进入最高点,但是从这之后会慢慢的消退下去,因为疫苗这些事情,生产、供应、运输这些问题都会慢慢地解决。所以新冠疫情本身会是一个慢慢消退的问题,但是它的间接影响,今年一整年我觉得都解决不完,就是一直到明年为止都很难说。这个问题在于什么呢?就是它扩大了贫富不均。这个,对对,因为你新冠它主要影响的就是两种工业,一种是零售的服务业,那另外一个是劳力密集的工业,比如说像那个肉品产去处理猪肉、鸡肉的这些工厂,这些都影响的都是低收入的人。那然后为了维持经济,那个美联储在去年跟今年都各要投入3万亿美元左右,那这些钱都必须要经过银行业进入才能进入经济。



王孟源 01:18:47 

换句话说就是经过金融业,经过,那你一旦经进入金融业的话,那第一个炒上去的是那个金融资产,所以你现在连Bitcoin都是创新高,这是很自然的,这是完全自然的事情,你所有的资产都会,你一下子有6万亿。而且这还只是美国,你其他的欧洲,然后全世界基本上都在印钞票,除了中国没有怎么印之外,其他的主要经地体都在拼命地印钞票,那一个加起来十几万亿美元的那个新货币去追逐有限的金融资产,那所以这个这有钱人一下子越来有钱。但是刚好这个时候没有资产的人,就是那些凭劳力来赚钱的底层民众,就非常的凄惨。



王孟源 01:19:43 

那在英美的霸权体制之下,过去这一两百年,即使已经有很严重的贫富不均。就是尤其是他们一旦开始把经济外包,把他们的生产业外包之后,从全球的观点来看,是平衡了贫富不均,因为这时候你容许像中国或者是越南这样的国家接手一些生产业,然后来致富。但是从他们内部的观点,也就是过去这几年的民粹,就是他们的中产阶级的生活水平,其实相对地来说反而是下降了,这就是为什么他们会有民粹兴起的基本原因。你你原本就有很严重的贫富不均问题,然后又有新冠这样的大幅冲击,那所以你会有像今年1月 6 号我们刚刚谈的那些政治动乱很自然的事情,那而且这绝对不是唯一的一件事情,全世界都会有越是越是不合理的体制,越是相信美国的这一套的国家,问题就越大。所以我觉得 2021 年它还有很多很多新闻,我们可以期待。嗯,就是基本上是善后的问题。



史东 01:21:14 

我个人对 2021 年的我现在最关注的就是它的经济方面,就特别对老百姓民生。你刚讲到了这个,对一般老百姓这个inflation,这么大笔的钞票往这个社会里面丢,我们现在在四周已经感觉到了这种inflation,我不知道,你可能也说你去买这个日用品的时候,这个价钱已经是涨得很多了。



王孟源 01:21:42 

因为你刚好新冠的疫情之中有一些日用品会稀缺,所以它对它涨价你也没有办法,也没有人会抱怨。



史东 01:21:54 

那然后还有一点,我把它丢出来,您可能一定会有个好一点的解读,就是关于这个新冠之后,人们不用去真正地去,去到点上班了,在家里上班。换句话说,他人在哪里已经不重要。换句话说,人现在已经很多是离开了这个很昂贵的这个大城市。



王孟源 01:22:19 

对,所以这是 depopulation of the cities,就是,。



史东 01:22:24 

对对对对对,然后这个造成的房价的波动。



王孟源 01:22:28 

对,就是市中心的房价下降,而市郊区的房价上升。这是我觉得这些都是次要的,就是我,因为我不是,我不喜欢讲投资,因为刚好有人问,我在博客问我说为什么这个投资要怎么?我说我其实不喜欢谈投资,为什么呢?因为这些资产市场,他们的那个价格有很高的非理性成分。



史东 01:22:53 

所以有很多高的被人控制性。可以这样看吗?



王孟源 01:22:59 

对,一方面是有内幕消息,所以他们一定要占你的便宜,然后他又有兽群作用。 Herd mentality,这个不是 herd immunity 的,哈哈,  Herd mentality,嗯,这个作用,所以你可以计算它的那个真实价格。但是问题是你这个名义上的价格可能偏差偏离真实价格几十年,那你这时候真实价格还有意义吗?没有意义对不对?那你那个所有的计算跟知识都没有用,对不对?所以你还不如去跟,为什么?华尔街跟经济学里面早就知道投资的一个很可靠的获利手段,就是跟,就是牛市的时候你就跟着炒,然后熊市的时候你就跟着卖,为什么?因为,他们所做的计算就是假设你是在华尔街的一个交易员,那华尔街的交易员当然他的资讯不是最早的,就是他是你有内幕消息的人,是最早的,但是他在这个金字塔就是老鼠会里面也不是最后一代,最后一代是那个真正零售的投资人,所以他经济学这些研究的这个资料都是基本上隐性的。假设就是你是一个没有内幕消息的交易员,那你这个因为你是在金字塔的中层,老鼠会的中段,所以你就只要跟着这个炒作,涨跌的炒作你就可以赚钱。



史东 01:24:42 

就是别人出你也出,别人进你也进。



王孟源 01:24:45 

对,所以这是已经 100 多年的老知识,你如果去看华尔街 100 年前写的股票投资指南,里面最重要的要就是这个,哈哈。



史东 01:24:59 

问题就是希望你不是第一个进,也不是希望你第一个出对不对?



王孟源 01:25:06 

因为你真的不追求第一个进,因为说不定没有人跟,对不对?但是你绝对不想是最后一个,那这时候你就必须要靠那些零售的,中国人说是韭菜,对不对?



史东 01:25:18 

对对对对,这个结论。 2021 一句话,你对我们观众有些什么样子的建议?



王孟源 01:25:32 

发誓会是一个很有趣的年份,大家继续努力。我觉得从国中国国家的观点来看,它最大的问题学术界的腐烂。习近平的反腐是很成功的,也是他过去这七八年执政成功的的基础。就是你如果没有这些纪律的话,你其他的行政策都没办法贯彻,对不对?但是问题是从邓小平之后,学术界基本上就是一个独立的山头,一个封建王国,不受其他的政治官僚体系的节制。



史东 01:26:14 

可不可以说是一个利益集团。



王孟源 01:26:17 

就变成一个非常大的利益集团。这一次他们刚好上个礼拜,这几天前,它公开地藐视事实,保护自己的,腐败的研究主管,这是一个非常恶劣的例子。你如果是做政治的模拟,就好像哪一个县长被抓县刑贪了几亿,然后,然后其他的县长开个会说不算贪污,哈哈哈,你说这样子能行吗?这样的政治体制能行吗?哈哈哈,但是这就是中国学术界的现况,你说这样的学术界能够支持他的高科技发展吗?真的吗?你看中国现在比较像样的就是军事体系,但是因为它的那个军事研发、军工研发体系有独立的,不靠他的那个学术界。如果是靠学术界的话,我早就现在根本就没办法挺直腰杆来面对美国。



史东 01:27:30 

现在,所以说如果我们今天把 2021 年的最重要的事情对于中国来讲就是打假。



王孟源 01:27:40 

最重要是打假。我,我真的希望这个,我在未来这 5 年,我认为中国最大内政最大的问题就是他在学术界的腐败。我上个月还不明白问题有烂到什么地步,所以我还在谈那些夸大骗经费的事情,像大对撞机,或者是可控核聚变发电,现在看来你连作假都可以这样子一手遮天的话,那这你都不需要谈这些事情了。



史东 01:28:09 

谢谢王老师。



王孟源 01:28:42 

这个体制可怕,不只是让你愚蠢,而是让你不断地愚蠢,不是光是到了愚蠢就停止了,而是它还会不断的强迫你越来越愚蠢。这 实在是真的很可怕。



史东 01:29:10 

我们现在就谈这个英国的事。



王孟源 01:29:14 

英国的事很有趣,哈哈哈。因为英国是美国的前任,美国的现在的这些邪恶的制度,资本在背后的操弄,不管是本质还是他的手段都是从英国学的。你看那个美国媒体在过去 40 年的腐败,其实是学习英国Murdoch那一套。 Murdoch是 60 年代就发迹的,从澳洲先发迹,然后很快就转移到英国去。对,然后到了 80年代才到美国深耕的。那,他们英国当然是几百年的,首先是欧洲的霸权,然后来变成全球的霸权,他的根深蒂固,根底相当的厚。在二战之后,到 1956 年第二次苏苏一次危机,被美国人教训了一顿以后甘心做老二。



王孟源 01:30:18 

这个衰退也是非常缓慢的过程,就是慢慢的一步一步的退后,也就是有序的衰退。但是你在过去五年就是从脱欧之后这五年,这是一个断崖式的衰退。那我在两年前上映节目的时候已经详细解释过为什么他们会脱欧,基本上就是 2016 年1月欧盟通过了反避税指令这个。而好笑的是这个反避税指令原本箭头指的是苹果还有 Google 这些美国企业,但是因为,英国的这些大亨们,就是尤其是内部贸易的大亨们,他们不愿意让欧盟来查税,所以他们就与那个红脖子做了一个联盟。



王孟源 01:31:16 

那我也跟你解释过,就是红脖子贡献了大概 33\% 的选票,然后这些大亨利用他们控制的媒体,就是英国有 12 个主要媒体,其中 10 个都是支持脱欧的,那这 10 个主要媒体就贡献了另外 19 percent 的选票,所以最后是以 52\% 通过了。



王孟源 01:31:38 

这些人在过去这五年一再地否定脱欧的经济损失。事实上任何有一点点经济知识的人都知道这会是很大很大的损失。那为什么呢?其实是因为这些人,他们嘴上说他们不相信会有这个损失,实际上他们的心里是不在乎这些损失。为什么这么说?因为他们有他们的私下的优先顺序,但是又不好意思讲。跟台湾也是一样,台湾那边说什么走南进策略,或者是跟大陆切割,不签服贸不会有经济后果,不是因为他们真的相信不会有后果,而是他们自己私下真的觉得有其他的优先,但是又不好意思讲。因为你讲了以后你必须要说经济下跌,底层的民众会生活困难,甚至会死亡,你怎么能够好意思强迫让他们生活困难?他们不想解释这些事情,他们实际上就是我不爽,以前我们对大陆在 80 年代、 90 年代有优越性,怎么现在变成他们跟我平起平坐,然后现在要超越了,然后变成主导似的。我看不惯,我宁愿关起门来在家里面做老大。这种心理你能不能公开的讨论?不能讨论,所以你必须要另外找借口。



王孟源 01:33:13 

英国的红脖子也是这样子,就是我记得帝国的荣光。以前欧洲二战是我们解放的,以前我们当了几百年的全球霸主,那结果现在是什么?波兰那些东欧国家的那个打工人每年来个几十万,我不愿意了。OK,我要把这些人都能赶回去,那这才是他们真正的第一优先。那些红脖子三百分之三十三的红脖子为什么会支持脱欧?这是他们真正的目标。其他他们不在乎你,你那个什么经济,怎么贸易,会有多少损失他们不在乎。



王孟源 01:33:58 

it's not their problem。



王孟源 01:33:59 

对,就算是,如果是他的problem,因为一般人的心理是非理性的,如果他们不愿意相信一个事实,他们可以拒绝相信就是即使这些人他本身就是做的跟欧盟做贸易,比如说过去一个月有一个卖鳗鱼的,还是卖什么鱼的,他的市场全部都是欧盟。但是过去几年他是那些脱欧派的人,每次要做广告,就把他拉出来做广告。他说我跟欧盟做生意,但是我支持脱欧,因为脱欧之后我的那个鱼就可以卖到全世界各国。这是可笑到极点了,欧盟从来没有说你不能够把鱼往别的地方去卖。



史东 01:34:43 

就是说你本来就可以卖,是吗?



王孟源 01:34:45 

对,你本来就可以卖,而且事实上欧盟跟很多国家有自由贸易协定的,你反而是更因为在欧盟内反而比较容易卖出去,但是他就是睁着眼睛说瞎话。为什么呢?因为他自己私底下有他的小九九,所以他不愿意面对现实。他骗人不是只骗别人,他连自己就先骗了,但是在过去一个月他就出来说我的生意垮了,都是你们自己选错了,我被骗了。明明是你入伙跟他们一起骗人的,你怎么能够说你被骗?



史东 01:35:19 

中国不是有一句话吗?中国有一句话叫自欺欺人呐。



王孟源 01:35:24 

对啊,就是自欺欺人了。对对,那所以你说这个脱欧的过程是什么?就有两个优先,第一个优先就是这些红脖子要求你这些东欧的人不能够再来英国找工作,因为你现在欧盟的规矩。是啊,他们不是要先有工作才能够移民,而是你随时可以进去,然后再慢慢找工作。那他们这个一定要杜绝,这是 33\% 的要求,另外 19\% 他们是来自那些媒体大亨。



王孟源 01:35:54 

这些媒体大亨要求什么?要求的就是你不能够执行那个反避税制定,所以你必须要有独立的税务跟金融法规。那这下就很糟糕了,对不对?你那个没有人员的自由流动,欧盟就不接受货物的自由流动,你不接受同样的金融监管法规,欧盟就不接受你银行业来做服务,来对欧盟的公司或者个人做金融服务。英国最大的产业是什么?金融业。那金融业就一定完蛋,然后剩下的所有的货品贸易也都要完蛋。那事实上货品贸易还不是最严重的问题。 Johnson 他本身是被这些媒体大亨捧出来的,他捧他最早第一个工作就是记者,所以他的那些人记忆的圈子都是这些媒体大亨,所以他出来当首相,最后出来当首相不是偶然的精挑细选出来的这些大红,精挑细选出来他就是最适合的打手,最可信任会不要脸,可以讲几万个谎言的。



史东 01:37:10 

也最听话。



王孟源 01:37:10 

和最听话的最不要脸的。所以他就很忠实地执行了硬脱欧,就是不接受欧盟的金融法规,也不接受欧盟的人的自由流动,就是这两件事情他一定要遵从,其他的他都不在乎,就是你这个经济损伤再怎么厉害他都不关他的事,因为反正这些大亨找他来并不是要让他执政 5 年 6 年,而是让他把这个硬脱欧做完,那只需要一两年。一两年做完以后硬脱欧的那个后果出来以后要有人背锅,这他这个背锅也是他的。



史东 01:37:55 

哈哈哈,也是他的责任之一,是吗?



王孟源 01:37:59 

所以他是不可能做到 2024 年大选的。事实上待会我会解释,他其实在今年年底,就今年后半年就会有很大的问题,所以我……



史东 01:38:12 

其实他阶段性的任务已经达成了。



王孟源 01:38:15 

对就是他阶段性的任务已经差不多要达成了的,对不对?那从这些媒体大亨来说,就由他背锅,然后反正到 2024 年大选的时候,他们自然会炒作出其他的烟幕议题来分散选民的注意力,对不对?不一定会的落选,因为你即使是最温和的、最温和、最理性的保守党人也是比工和工党要强,从这些大亨的眼中来说,说对不对?那宁可是一个最温和的保守党人,也不会让工党获胜。所以他们还是要希望能够下一次大选能够获胜。但是获胜不一定要由Johnson来领导,对,他又不是有任何能力,你如果照能力排名的话,那说不定是全英国最后一个人员。我。



史东 01:39:04 

可不可以问请教你这么一个问题,我们谈谈2021。你觉得 2021 年对对英国的走势以及他们将会面临的几个问题。大问题是什么?如你可能你先前都讲到过,但是能不能再我们重复一遍?



王孟源 01:39:21 

所以我刚刚跟你解释的就是 Johnson 跟保守党或者是他幕后的大亨都知道,事先都知道是会有什么惨状,所以基本上我们今年就是等着看一个月比一个月更惨。那目前已经到1月底了,你这个正式脱欧其实在 12 月的时候,他就 Johnson用他的那个拖字诀一直拖到 12 月,据说他是到 10 月的时候还有一个谈判,就是虽然他们已经谈了好几年了,但是 Johnson一直都不交出说我建议我们这个自贸协议是什么?一直到 10 月的时候才交出一个草案。那欧盟当然是几年前就有一个草案做,我建议我们照这样做。那你说你到了 10 月才拿出草案,这算是有哪门子诚意啊?你就像我刚刚讲的,中国跟欧盟谈那个中欧投资协定,谈了好几年了,那中国做最后的让步是最后的 6 个礼拜,那已经是很紧急了,但是英欧的这个自贸协定, Johnson 根本连一个版本都没有拿出来,一直到 10 月才拿出来。所以他们的谈判事实上是 10 月开始,然后到了 12 月最后一个礼拜,他才终于让步。



王孟源 01:40:51 

让步是说让到什么地步?基本上就是全面弃守,就是接受欧盟的那个草案,那所以你总结来说,它这个自贸协定就是无条件投降了。那你说这样是不是最糟糕的决定?是连个自贸协定都没有?为什么?因为你连自贸协定都没有的话,自然就会有关税。欧盟的规则是很详细的,就是你如果没有自贸协定的话,我们就依照 WTO 的规则来做,那什么东西,有多少的关税,这个都是事先已经给了决定好的,那英国跟欧盟这个自贸协定解决的是什么?基本上没有关税就是好,你说那很好,没有关税,那为什么欧盟会给他这么优越的条件?这其中的辛秘让我讲给你听。



王孟源 01:41:46 

第一点你必须要知道欧盟对英国有很大的贸易顺差,所以这个就是在货物上,各货物贸易上有很大的顺差,所以这个没有关税。你说对哪一方比较有利?当然是对欧盟比较有利,对不对?这样子你还能够维持继续的出口对英国的出口,但是欧盟对英国有很大的服务逆差。



史东 01:42:11 

服务逆差。



王孟源 01:42:14 

服务英国有顺差。对,就是尤其是金融,对不对?有一些会计、建筑这些设计这些东西。那这次的自贸协定就对服务业一字都不提。对,那个金融我刚刚解释过,银行业的那个执照,通行执照是明文的取消,你说这样子是不是单方面的优惠欧盟?你就可以理解,为什么这是欧盟的草案,对不对?货物贸易的时候就是你虽然没有关税,但是还是有手续的,就是进出口手续的,以前是没有海关的,现在是你必须要海关,为什么?海关你必须要有进出口检查,然后有什么生产地证明,还有那个工业标准合格书这些的一大堆的东西对欧盟来说完全不成问题,因为他们的法规已经实行了几十年了,为什么欧盟周边有一大堆欧洲国家没有加入欧盟自贸区,最重要的就是挪威跟瑞士,所以这些现机关现成的人员现成的制度,现成的细节,现成的电脑城市现成的。



王孟源 01:43:30 

哈哈哈,连那个企业基本上也都知道,他们企业的那个电脑网路里面也都有现成的软件,也有现成的人员懂这些会计,懂这些事情,所以这对他们来说根本不是一回事。但是英国这些全部都没有,那你以 Johnson的这种行政效率当然是不可能有的。所以现在的这个贸易现状是什么样子?从欧盟到英国的货车是通行无阻,英国根本就是不设防,那就是因为你这个检查进口是你的权利,但这不是义务,进英国他可以不管,但是从英国出口的东西就必须要层层检查,而且是英国先做一次检查,然后再欧盟来确认说你做了检查。



王孟源 01:44:25 

所以现在在英国的那个口岸大排长龙的东西,它是英国出口的,那边单方面的大排长龙,据说是一个货车,典型的是要三四天,这还不是最严重,哈哈哈。最严重的是这个的间接影响就是英国凡是小企业根本就可以放弃对欧盟的生产出口,中大型的企业就必须要转移一部分的生产,后勤转移到欧盟去。他们的首选是荷兰,因为荷兰的法规跟英国最接近,都是重商主义的国家,那这样一来小企业就会倒闭,中大企业必须裁员,这个这都是很自然的结果。你说这对经济有什么好处?绝对没有什么好处,而且只会越来越糟糕。为什么呢?因为欧盟虽然容许你现在是 0 关税,它保留了单方面随时调整关税的权利,就是欧盟什么时候高兴调整关税,可以不召回英国,直接就调整关税。那你这样子。



史东 01:45:45 

这样子做生意的风险太大了。



王孟源 01:45:48 

太大了,你说我如果要车子卖到欧盟的话,我敢不敢把汽车厂开在英国?绝对不敢,对不对?你即使是现在现在这样子 0 关税,但是有一堆那个手续的时候,你或许可以说这种大工厂可以有足够的人手处理这些手续,但是你随时有关税被提升的危险,那这时候你就不会像这种长期的投资,你不可能再冒那个危险嘛,对不对?



史东 01:46:26 

有一点就是我们也都知道英国一直希望能够和美国有一个FTA。



王孟源 01:46:34 

我上去年上节目的时候已经说过了,这个英美的 FTA 一样是极端的不平等,所以你绝对还是让美国占尽便宜的,对不对?



史东 01:46:48 

对对,因为这是现实,这是现实,谁求谁。



王孟源 01:46:52 

英国的体量是欧盟的 1/ 6,是美国的 1/ 8,还多一点,大概将近



史东 01:46:58 

换句话说,像这么说的话,即使有FTA,英国也从美国里面掏不到什么好处。



王孟源 01:47:07 

掏不到好处,因为你这个 FTA 顶多是把关税降一降,手续还是要有的。以前跟欧盟是完全同一个贸易区,这不是自贸协定,是同完全同一个贸易区,对同一个区就是连手续都没有。嗯,欧盟跟英国的体量比是大概 6: 1。你想一想,如果中国对全世界原本可以自由贸易,你说中国会不会选择一下子脱离全世界,然后现在忽然要进出口关税,不要说关税了,说零关税好了,但是有进出口手续。



王孟源 01:47:46 

你知道中国占全世界的那个经济体量是多少?22\%,就是,世界所谓的外国跟中国的经济体量比就是稍微超过 3: 1,不到 4 比。那这样子,你只要设身处地想一想,即使是那样的比例,你愿不愿意损失完全没有阻力的自由贸易,那是无可想象的损失。英国只有欧盟体量的 1/ 6,你说自绝于那个贸易区,这是一个理性的选择吗?绝对不是。



史东 01:48:32 

我只能说四个字,匪夷所思。



王孟源 01:48:35 

匪夷所思。对,就是为了两个自私非理性的小集团。而那个红脖子是比那个媒体大亨还要非理性的,因为它媒体大亨的理由是损人利己,而这个红包子却是损人不利己,这是真正愚蠢。



史东 01:48:56 

其实我们今天讨论这个事情,孟源,这又让我觉得是一个很强有力的佐证,民主制度的荒谬。



王孟源 01:49:05 

对因为他没有想到,市场经济自然会演化出大企业、大资本,然后这些大企业、大资本自然会去扭曲这个规则。你如果一切都是任凭自由竞争的话,那就没有节制他们的力量,对不对?唯一可能节制大资本力量的是什么?大政府力量。你还有什么其他力量能够跟他们的对比?



史东 01:49:29 

还有你的信消息来源被控制了,你被洗脑,对不对?



王孟源 01:49:34 

对这个整个体系就是包括媒体对不对?宣传是很重要的,对,资本绝对会控制媒体的。对。



史东 01:49:45 

我,我讲一个笑笑话,这个你可能听过,他们说这是什么时候讲出来?几年以前讲出来,他说欧洲基本上有两部分的国家,第一部分是小国,第二部分是不知道自己是小国的小国。我想英国可能就是一个不知道自己是小国的小国吗?不是吗?



王孟源 01:50:10 

我德国跟法国也是,他们是理性的面对这个新的世纪,就是有这些。第三世界国家都工业化了,他们明白自己的体量不足以单独在世界上有所作为,所以他们集合起来,这是一个非常理性的东西。但是他们一直有英国在里面,做一个中国人说的搅屎棍,就是故意不让他凝结。



史东 01:50:38 

我让他团结。挑一个问题,觉得英国是美国的搅屎棍还是美国用的搅屎棍?还是你的回答是什么?



王孟源 01:50:48 

从二战丘吉尔所设定的路线,就是英国其实是为美国服务的,在欧盟之内的内奸,这并不代表说美国没有其他的控制手段,比如说德国的媒体跟智库也是被渗透得一塌糊涂。但是从国家的层面来说,英国是美国的代言人,在欧盟里面的代言人,所以他这个脱欧对欧盟是一大利好,对中国也是间接的一大利好。所以……但是报应来得也很快。你这样子,国家其实是一个利益共同体,你如果损害国家的话,最终损害会触及每一个人,那即使这些资本觉得这是值得的,就是因为这些损害是全国全体的公民一起承担,而他的利益是只有他们自己独占,所以他们觉得是有利可图。那你这个国家的体制当初设计的时候,就应该避免这种损人利己的的机制,对不对?但是他们的这个民主直选制,还有这个自由主义经济体系,就是完全不考虑这个这件事情,就是完全忽略人性的自然趋势。你说然后他们好意思说自己是最优秀的意思?我觉得很搞笑,他们这个报应会什么时候来?其实来得很快,就是今年5月的时候是他们的地区选举,他们其实现在已经是高度自治了,像是威尔斯或者是苏格兰或者北爱都有他们自己的区域议会,专门负责内政。那比如说教育还有卫生这些问题都是当地的区域议会处理。那这个目前苏格兰,他们的苏格兰国家党,也就是他们自持独立的那一档已经是占多数,但是还没有绝对多数。目前的民调是预言他会大胜,就是从现在的基础上还要再大胜一步。



王孟源 01:53:06 

北爱也是一样,因为北爱的话是一半一半支持英国新教,一半支持爱尔兰的天主教。但是因为,这一次脱欧的过程,北爱一直是坚持支持脱欧,他们以为说这样子能够摆脱跟爱尔兰共和国的任何联系,结果刚好相反。现在欧盟拿到的是什么?就是海关真的是建立在爱尔兰海峡上,然后从英国到北岸的那个货物贸易反而是需要进出口检查,那这个检查是由欧盟的办公室来监督的,欧盟在北爱新设的办公室监督。你说这样子,哈哈哈,这样子,那些北爱的支持英国的亲英派会高兴吗?他们是绝对被背叛了对不对?所以我看这一次选举他们的势头也不怎么好。那在5月之后,如果这样一闹,接下去就是下一步就是要求独立公投,尤其是苏格兰,因为苏格兰国家党已经讲得很明白他们的这是他们的党纲。接下去就看 Johnson 处理这件事情,处理这个苏格兰独立,如果只要稍有差错,你就可以准备让看包社党换首相,换党可以换首相。所以这就是为什么我估计今年后半江森很有可能下台。至于他们会不会真正搞出独立公投,我们还要在观察看他们的折充和结果。



史东 01:54:51 

不管怎么样,还是那句话,好戏还在后面。对。



王孟源 01:54:56 

你弄到这么惨,真的是,就是因为少数几个媒体大亨的私心,把国家搞成这样,几百年的霸权。对,现在不但是。



史东 01:55:07 

不可思议。



王孟源 01:55:08 

真是不可思议,不但把整个国家的经济搞成这样的,而且国家四分五裂,这都是事先就可以预言了,至少我在几年前就一直在讲。



史东 01:55:19 

其实我觉得他英国的这个这些事情也给世界上其他国家,譬如说中国也在内一个很好的一个负面教材。



王孟源 01:55:31 

英国、美国这样作死,对愿意理性思考的人都是一个很好的教材。很可惜是台湾政界的高层,好像已经没有任何愿意做理性思考的人了,因为这其实都是很好的类比。



史东 01:55:47 

对,我觉得台湾现在可怜是他们已经没有做理性思考的空间跟选择权了。对,他们只能这样子混下去了。



王孟源 01:55:59 

你这个体质可怕,不是让,不只是让你愚蠢,而是让你不断地愚蠢。嗯,不是光是到了愚蠢就停止了,而是他还会不断地强迫你越来越愚蠢,这实在是真的很可怕。我们这种理性思考的声音其实先天就是少数,但是一个正常的社会应该是理性的,知识分子的声量最大,所以即使他们永远都是社会的少数,他们确实会是社会主流的声音。对,这不但是一个健全的社会,而且事实上历史上也是主流。历史上。你想想看孔子在论语里面讲的那些话,民可使由之不可使知之。那你事实上那些世代夫政治不管怎么改革,有没有人说我们这个农业政策要由农民来投票公投的,没有对不对?你都是由士大夫理性的为他们来考虑,为他们来辩论,来做论证。



史东 01:57:10 

特别是你的信息被控制,对,这个是更要命的事情。现在。



王孟源 01:57:19 

这个把多数决上升到神圣的地位,其实是冷战的需要,先是冷战的需要,他们才在 6 0年代做那些改革,然后接下来就是美国富豪,美英美的富豪要夺权,所以就进一步地顺水推舟才把这个多数决,然后把愚民的声量无限的提升。你一旦是比人头的话,那当然是愚蠢的人占多数对不对?



史东 01:57:51 

套用你刚刚说过的一个概念,世界或者这个社会永远是有所不为,无所不为两股人之间的,怎么讲?竞争或者斗争。



王孟源 01:58:07 

我们所谓的好人就是在乎社会公益,也就是不愿意损人利己,不太愿意损人利己。你如果是小损人大利己,或许可以考虑一下,但是,所谓的坏人就是即使是大损人小利己,他也在所不惜,对不对?是对,那好人必然是少数,但是坏人也不一定是多数,我觉得多数基本上是一般人,你必须要建立好的风气,这是所谓的民族性,对不对?你所谓的民族性优越就是国家兴亡匹夫有责,你见到损害公益的事情,也就是坏人在行为,你愿意跟他对立,那这样子就是好的民主性,就是好的风气。那这个就是一个健康的社会,一个健康的国家才最好,长久上……



史东 01:59:03 

非常非常好,孟源,我祝你成功,尤其是你的这个你的愿望,我希望你能够达到,而且我觉得你没有可能不能达到。尤其是今天那个时候,你用syndication,你写专栏,那是文字,那是印刷,现在是电子,对不对?你的这个周围的条件更适合你做这件事情。



王孟源 01:59:29 

就是我还不能够确定我对政策有多大的影响,就是实际的影响就是到……



史东 01:59:33 

我个人的感觉,可能是我是一个比较Naive的一个人,就是我觉得好歹你就先做下去,你不要考虑你会有多大影响,只要你路走了,证这个思想不受人家的这个利诱。所谓的利诱,我觉得你,因为你现在存,你现在的存在已经受到很多人的肯定了。只是说可能这个肯定可能没有达到你的预先的要求,或者对自己的自许,但是这个会来到,因为我还是那句话,可能过于难,因为,但是我是深深的觉得只要你一步一步往前走。



王孟源 02:00:18 

希望如此,这需要大家的努力,我一个人是绝对做不到的。



史东 02:00:21 

对,但是。



王孟源 02:00:22 

但是我注意到,嗯, 7 年下来说长不长,说短不短,我慢慢的可以看得到媒体上的舆论慢慢地向我这边倾斜。嗯,也许是偶然,但是有的时候他们所用的那个智慧,还有那个frame,就是那个框架是我所用的,是套用我所用的。其实我是很高兴你引用我的论点。



史东 02:00:48 

这个代表是你在影响别人,就是。



王孟源 02:00:51 

大多数人不知道这些舆论是从哪里起源的,但是我很高兴,就是我至少有几十个例子,几十个不同的论点,我这 7 年来有 300 多篇文章, 8000 多个留言讨论,我至少谈了将近 1000 个话题。就是你如果细分的话,有大概 100 个主话题, 1000 个小话题。



王孟源 02:01:16 

我想,真正给一些真正有好奇心、有求知欲的人,能够有系统地对世界建立一个正确的观感。因为 21 世纪的社会是非常复杂的,对不对?你如果只相信媒体,不管是白左还是更糟糕的右翼民粹,都是先天就是错误的,然后再加上有意扭曲虚假,那我希望能够建立一个我说的避风港,就是给理性、诚实的人的一个避风港,然后大家互相进修。



王孟源 02:02:01 

但是这个要有作用,就是必须要大家出去把这些真实的消息、正确的观点传播出去,然后看到有坏人撒谎骗人的时候,你要那个,要有那个决心,要有那个勇气站出来跟他们做斗争,因为这是一句老话,坏人要得志,只需要好人不做事对不对?我们这一次科技部包容作假这件事情还没有完,我会尽我的力在舆论上批评他。那我相信中国本身也有良心之士,愿意的举报的那个就是一个,本身也是一个名教授,对不对?他也是损害自己的政治利益来维护国家的公益。



史东 02:03:02 

所以这种人需要勇气,这种人需要保护。



王孟源 02:03:05 

需要良心跟勇气,这是真正的勇气,能够跟能够跟自己周边的群体作对,说反话,然后坚持事实、坚持公益的,这个叫非常大的勇气。



史东 02:03:23 

择善固执,对不对?这是一个非常难得的一种人格上的特性,这是我觉得任何一个公民都应该跟,都应该赏识,都应该呵护的一种特性。我觉得还是我们彼此这个互相勉励,要做得正,要做得真。



王孟源 02:03:48 

很不幸的,资本媒体不容许你这样子。对,所以像你这样子一个独立的媒体,我们虽然是永远都是少数,那我们触角没办法传播到每一个家庭,但是我们做自己的本分,我们尽力,尽我们的努力,希望其他有志之士能够加入我们,然后为国家跟人类的前途来奋斗。



\twocolumn[\begin{@twocolumnfalse}
\section{乌克兰、北约、美国}
\subsection{20220127}
\end{@twocolumnfalse}]Credit: anonymous



史东 00:00 

你好,我是史东,昨天晚上我和很久没有联系的王孟源先生有一段相当长时间的谈话,我们谈话的内容包括了乌克兰,包括了北约,包括了美国现在的财政、货币和金融的情况。另外我们也有一些相当分量的一些即兴式的、推心置腹的谈话,我个人觉得非常精彩,值得为您介绍,我们整理之后马上就会为您退出。



————————————————————

王孟源 00:33 

英国是还想要把这个事情闹起来,美国现在基本上就是,我去看他们的这个外宣,还有他们的外交的宣布,基本上都是只是要找下台阶。



史东 00:45 

北溪 2 号的使用是迟早的事情。



王孟源 00:49 

是迟早的事情,就是我上一次已经说过这其实Putin也不急,就是。嗯,德国人也不急,因为它是备用的东西,嗯,对不对?只要是在那边备用,你现在用不用都不要紧,但是乌克兰要是被打烂的话,你绝对是有正当的理由马上就把它打开,对不对?所以这个,理解它的意义在于存在,而不在于被使用。



史东 01:18 

哈哈哈,这句话说的很有意思,哈哈哈,存在的意义大于使用的意义。



——————————————————————

王孟源 01:39 

先跟大家拜个早年,新年快到了,大家新年快乐。有关乌克兰这件事情,因为过去这几天我又接到很多的电子信件,还有博客上面的读者也留言来问,所以我在这里很明确地讲,就是乌克兰的军事冲突危机在去年的3月,4月的时候是真的,就是那个时候乌克兰的确是把部队往东方边调,然后俄国也就增援了一些部队在边境,那是真的。然后来他们把重装备留下来以后就回去了,回去以后一直到现在都没有在征兵。



王孟源 02:27 

OK?你们现在看到的,这个过去这三个月其实是 10 月底开始美国的,美国的好像是那个 foreign policy 杂志,我不太记得了。 foreign policy 还是New York Times 最早从那个时候到现在大概还不到 3 个月。这个炒作完全是无根之草,就是无中生有。你们现在这个被洪水般的消息……我真的是,有点不胜其烦。因为我我自己收到的那个电子信件里面一直问说,诶,是不是要开打了?然后我说根本没这回事,都是谣言。然后隔了两天同一个人又送一个新的,中国时报的文章说,你看这非打不可吧,然后这样周而复始的五六遍、七八遍以后我真的是不胜其烦。所以我在这里讲,斩钉截铁地跟大家讲,完全是无中生有的。OK,我跟你讲, 10 月你如果回头去看这个,整个这一波炒作,最早开始就是将近 3 个月之前, 10 月底开始最早的那篇文章,它唯一的证据就是两张照片,两张号称是那个卫星照片,就是俄军调到乌克兰这边。后来。当然现在网络这么发达,就有人自己到那个民用的卫星地图里面去找。



王孟源 04:06 

我这两张地图是哪里来的?它真的是在乌克兰边境,一个距离乌克兰边境大概 50 公里,一个距离乌克兰边境大概 250 公里,是这两个都是永久性的营地,那个美方媒体刊出照片的时候是故意把它裁下来,所以你只看到他们的停车场,你没有看到他们的军营。你看不出那个军营是永久性的建筑,因为它把它裁掉了,那然后人家从这个地方就可以去顺藤摸瓜,然后就找出来这两张照片,一个是 144 机械化师,另外一个是第三摩托化师的 237 坦克团。 144 师跟第三摩托化师,刚好就是俄国常驻在西部军区面对乌克兰的两个师。换句话说,你所拍的那些照片是他们的永久性营地的照片,没有什么征兵,在边境征兵的东西。我这样讲,但不知道是不是清楚,就是他们是制造假新闻,所以拿了这个卫星照片拍了边防部队的营地,然后把它剪裁出来,只放出那个停车场,你一看一两百辆装甲车在那里。



王孟源 05:39 

事实上这两个师是俄军在过去十几年经过好几次军改, 2009 年的时候先师改旅,所以他以前很多历史悠久的功勋部队统统不见了,那个师都变成旅去了。然后到了 2014 年以后,他又反悔了,又反过来旅改师,所以到 2016 年完成这个旅改师的步骤。他面对是乌克兰方向,就是这两个师,一个是第三师,第一个是第144师。这两个师都是 2016 年成立的,他的那个营地都是从2016 年就建到那边,那个停车场,那两三百辆车是从 2016 年就待在那里,你说他到了 2021 年 11 月才开始炒作,这是什么样的居心?



史东 06:37 

这个让我想到这个,我让我想到这种行为很像新疆的那个劳改营,感觉其中应该是。



王孟源 06:46 

一模一样,今天好玩的是一开始炒作是美国人在炒作,那你或许会问,为什么美国人会挑那个时机来炒作?我个人的猜测,当然这次我没有办法逼着布林肯跟我说实话,对不对?这个,我个人的猜测是那个时候刚好是乌克兰的那个石油天然气的储藏用完了,你知道过去这一年Putin开始玩天然气外交玩得很强硬,然后基本上就是你有签长期契约,我照其供;你要买现货的话,对不起,哈哈。没有,哈哈哈。



王孟源 07:31 

所以这么一来,因为乌克兰对天然气是从苏联时代留下来的那个病就是非常的依赖,那原本在那个北溪 2 号建成之前,俄国的天然气往中欧就是像德国那样提供的话,主要是两条管道,一条管道是走Belarus,然后经波兰到德国,另外一个就是走乌克兰。那么乌克兰的天然气它就是在这个过境的时候,不但要收过境费,而且还顺便抽成,就是自己要用的,就把它抽下来。



王孟源 08:16 

问题是刚好到 10 月底,大约就是去年三个多月前,就是乌克兰有天然气问题,然后后来就爆出真相,就是在其后的几个礼拜里面,那个管道其实不是从东往西输,而是反过来从西往东输了,就是德国把他自己的储备拿出来送给乌克兰,因为乌克兰真的是见底了。OK,然后那个俄国的供应又因为长期合约已经满足了他,就把它那个开关给关掉,以后乌克兰没办法抽成,那只好去求德国,德国就说好,那我把这个管道逆向的从西往东送。所以我个人认为当时是因为 Zelenski 就是乌克兰那个总统,它的经济还有能源供应有问题,所以他准备要闹事,先跟美国照会了一下。那美国人是这个——中国人是兵马未动,粮草先行——那个美国人是兵马未动,宣传假新闻先行,所以就先造了那个假新闻。那这个假新闻从 10 月底开始炒作,然后炒到 11 月,但是是到了 12 月中以后,他一下升级搞到这个我们现在看到的这个洪水般的消息,什么俄国即将入侵了,怎么样?然后这个其实是 12 月下半才开始的,而且鼓吹的不是美国就是美国,一开始还只是说可能有这个危险,然后大家要小心,愿意捐款给乌克兰的,请大家赶快捐款要。



王孟源 10:04 

当然捐款美国人自己是很吝啬的,捐款是从他是欧盟跟德国要,哈哈哈,慷他人之慨,你说要美国人拿出十亿 20 亿美元,他不愿意当那个凯子,那你造一个假新闻让德国人去当凯子多么方便,对不对?但是,但是这个到了 12 月下半,忽然这个肇事这个假新闻就升温了,而且更加的极端,那个说的有鼻子有眼睛的,这个是哪里来的?不是美国人了?现在改成英国人了。



王孟源 10:42 

而且你到了1月初以后,美国人基本闭嘴了,不再讲为什么,因为那个时候那个 Blinken 已经 Biden 的团队跟 Blinken 已经内部去问过了,他们问过他们的智库,他们的幕僚,问过国防部要拟定这个反制的措施,结果发现没有一个反制的措施,实际制裁。



史东 11:07 

对不起。什么反制?什么反制?俄国。



王孟源 11:08 

制裁俄国,OK?他不是说那个时候考虑了要把俄国踢出swift,结果人家结果智库拿上来说你把它剔除 swift反而是鼓励欧盟的所有的国家都改用各国的那个支付系统。那这下你不是自己自己捅自己的刀?所以然后能不能送部队过去?不能,你打不赢,能不能送武器?那个也只是纯粹象征性的,你能不能对它禁运,农业禁运,工业禁运什么的,统统没有用,统统都没有用。所以一旦那个报告出来以后,美国人的那个假新闻就开始消停了,但是英国的假新闻在过去这个月里面是这个一波接着一波,那个声浪越来越高,所以我们现在看到的这些假新闻都是,尤其是中文,基本上都是台湾的媒体也要照翻什么 Reuter 什么的 BBC 这些东西。这个为什么英国人会忽然这热衷?这其实有长期跟短期的两个原因。



王孟源 12:19 

第一个长期是,MI6 原本就比 CIA 还要热衷于在国际,到处制造动乱,然后这个打起来,对英国当然不痛不痒,你东欧打起来,英国是在欧洲的最西端,但是真正最重要的原因就是我认为是 Boris Johnson,他在 12 月中的时候出了丑闻,不是吗?他那个因为在 2020 年封城期间,他一年开了好几次宴会,party,酒会,而且被爆出来是一次比一次的恶劣。



王孟源 13:07 

那最糟糕的是 Boris 这样子,这个人的人品非常的低下,他这从上任以前,从他刚出道当新闻记者,就是著名的不要脸的撒谎,然后开始从政之后他的贪腐新闻也是永远都不断的。但是因为他很会说笑话,就是我说他是不世出的小丑,所以很受英国保守派选民的欢迎,所以他才是他在 2019 年临危受命出来把硬脱欧给推销过去。是啊,就是因为他的这个小丑的欢迎度。这个你想想看,这个跟 Trump 一模一样。 Trump他是啊,做 reality TV ,就是Trump也是一个现场秀场,现场秀,然后这个Boris Johnson的虽然没有他自己的秀,但实际上就是一个小丑,讲笑话的。



王孟源 14:13 

那他这过去当这首相当了两年,才刚两年,这个丑闻,每个月都有新的丑闻,但是为什么到这一次才忽然爆起来?就是因为虽然这个是一件小事,封城之间的时候开 party 是一件小事,但是他们在这个 party 的过程中,他的发言人主动地去讥笑他们的,直接地讥笑他的选民,笑他们笨,他们笑他们好骗,就是走上那个发言人的那个台阶,然后假装回答记者的问题,然后找各种的借口。就基本上就是嘲笑他们太笨了,太好骗了。



王孟源 15:05 

你别的事情都可以,你怎么贪腐怎么胡搞,把国家搞砸搞分裂都没有关系,但是你笑这些笨蛋,说他们的笨蛋,说实话反而是最毒性的那个。丑闻一出来,这个他的民意支持率就直接掉到底了。这个现在他下台已经是固定了,问题只是在哪一个月。我在我的博客上有讨论,我认为这可能是5月底,但是目前还有好几个势力在折冲之中,因为他树敌也很多,他这个人真的没有一个朋友,没有一个真正的朋友,全都是互相利用出来的。但是你可以想象在 12 月这个丑闻爆发之后,这个要度过丑闻最典型的手段是什么?就是创造一个国际危机。



史东 15:59 

转移焦点是吧?



王孟源 16:01 

转移焦点,而且保守党控制了英国 12 个主要媒体中的 10 个。然后他们也知道 Johnson 在 2022 年初就下台,是太早了一点,就是我在我的博客博文里面有详细的讨论,他们原本的计划是让他在 2022年底下台的。那你现在这样提早一年的话,对他们下一次大选的布局很不利,因为他们下一任的新首相新党魁呢,也是 11 月12 月才搞定的。就是基本上那个保守派的那 10 个媒体里面最大的一个是Rupert Murdoch后,那Rupert Murdoch是在 11 月左右才选定那个外务大臣 Liz Truss作为他的Lap Dog。对,哈巴狗,那这个都很简单,你只要去读他们的社论都可以看得出他们是在捧谁。然后另外的比较有可能出出现的是财务大臣 Rishi Sunak,他有着其他几个比较小的媒体财阀的支持,所以基本上就是这两个二选一了,那这两个人基本都没有准备好,其中尤其是 Liz Truss绝对没有准备好,因为她刚刚才被钦点,所以你需要几个月来造势,这至少是要 5 个月, 6 个月,你想上一次脱欧,他们也是花了 5 个月来造势。



王孟源 17:50 

所以你如果 Johnson 在1月或2月就下台的话,对 Liz Truss是非常的过早的,那所以我说大概5月底,这也是原因之。那你一下考虑到这一点,那 Johnson 他出来要创造这些国际危机的话,或者制造至少制造国际危机的假象就是让选民——他其实不在乎这个乌克兰的死活,不在乎俄国的死活,不在乎美国的死活——他在乎的就是——要让保守党选民。所以你看这个民主政治就是这样的,为了英国保守党的那 1000 多万选民,他们冒着全世界打一起世界大战的危险, OG AH。空地创造出这个俄军在威胁乌克兰的假象。然后你看过去这一周的新闻是什么?这周的新闻是英国首先把他的那个大使馆撤离了,然后接着美国跟着撤。为什么?因为英国最需要制造恐慌,制造恐慌以后他们的选民才会团结在首相的身边,因为现在他们的 Johnson 还在幻想着要一直做下去,做到 2024 年,但是至少他的那个保守党的媒体大亨的支持者是希望至少能够撑到年中的5月或6月。



王孟源 19:18 

所以那你现在制造为你看看这些人多么的自私,就是他们那个胜选的机会提高5\%,他们就愿意让全世界去打战争,这个是那么世界邪恶可恶 的两个高级强权。



史东 19:36 

你刚刚这句话是画龙点睛啊,画龙点睛。



王孟源 19:39 

啊,真的是邪恶到极点了。不论如何,你看的话,这基本上现在过去这个月在造谣的都是英国的外交官,人家有人质问说,那你已经说俄国要打乌克兰,从去年3月打到现在,怎么到现在还没打?他就说在北京的一个英国外交官居然回答说,诶,是因为要冬奥,你们的这个一石两鸟啊



史东 20:08 

要等冬奥结束。



王孟源 20:10 

这好像你说。然后我接着他们就去问那个中国外交部的发言人赵立坚,说是不是你们是不是因为习近平叫普京等到2月底再开的打。



史东 20:23 

我这么,我这么请教孟源,这件事情我想你已经说得很明白了,这个你是你认为这个事情是无中生有的事情,也不会打起来,你觉觉得这个事情将来,我不是说将来,我说最近的将来怎么做好了会有一个怎么样的结局?



王孟源 20:45 

美国的所希望的结局是德国人乖乖地当他的炮灰,经济炮灰就是像 2014 年那样子,去做全面的制裁,因为在所有的外国,在俄国的投资最多的就是德国,比中国还要多很多。 2014 年的时候那个德国的企业真是哀嚎遍野。那其次还有像我上次也提过,就是意大利跟希腊,还有甚至西班牙都有投资,这些国家都不是特别高兴。



王孟源 21:31 

所以,那你也知道Merkel刚刚退休,OK,如果Merkel还没有退休的话,就会有人出来跟乌克兰讲,叫你不要造势,你造势的话大家都没有好处,那结果是Merkel退休了。是,这也是部分的原因。就是为什么 10 月底的时候美国人觉得闹一闹可以去挟持德国,因为那时候德国连新政府都还没有出来,对不对?真的,那你这个趁他们的乱的时候要挟一下,然后多要点钱,但是美国人并当时并没有想,我觉得美国人当时并没有真的想要打。美国人只是要帮乌克兰去跟欧盟勒索一些钱包,后来真的鼓动大家打挺来的是英国人,就是从 12 月一个月前。



王孟源 22:31 

那真的是很恶劣,你看现在带头的不是带头起哄的,不是美国人,是英国人。但是英国人起哄是行,就是现在英国在国际上的影响力基本上就是外宣,因为他还是控制了这些英文媒体,对不对?你看台湾的那些名嘴,哪一个不是拿了英美的国际新闻然后就照念的?但是真正要发展下去的话,有潜力有实力,会受影响的基本上是俄国在一边,美国在另外一边,这个乌克兰在其中是 Zelensky 想要榨取一些个人的政治利益,其实他的作为跟陈水扁非常的像,所以大家台湾的观众应该很容易解读他的这些把戏。



王孟源 23:31 

然后最关键的其实是德国,还好,德国这个政府最后是在 12 月终于搞出来了,搞出来以后它这个马上就面临这个危机,一开始是他那个外交部长,就是绿党的 Baerbock,还在胡说八道说这个 北溪2 号绝对不能开。其实,嗯,其实Merkel在他任期的最后 7 年忍辱负重就是对着美国的压力,基本上是尽可能地敷衍。但是他始终坚持一件事情,就是要把北溪 2 号做出来。为什么呢?因为只要北溪 2 号做出来,你这个乌克兰打得再烂都不关他的事。你这个德国要置身事外的前提就是要有独立于乌克兰的天然气通道。所以她这个我说,我再说一次,我觉得她真的是忍辱负重7年才,把北溪 2 号搞出来。



王孟源 24:43 

那现在的这个总理Scholz,他在 Merkel的内阁里面也是当过高官。当然你知道那个德国的情报系统其实是很庞大,而且被美国渗透得很厉害,所以我觉得 Merkel 是不敢把她的真实意图在那个会议或者是跟幕僚开会的时候直接讲的,因为连他的手机都被窃听了。但是跟Scholz,我想大概Scholz也是个很有经验的有智慧的,所以我想他们私下可能是用一些间接比较委婉的方式沟通过了。



王孟源 25:31 

就是这个欧盟要怎么样寻求独立,要避免再像 2014 年一样被美国出卖,跟俄国搞得两败俱伤,这是他们的头号的 concern,就是第一优先的战略准备,所以 2014 年的时候就是因为乌克兰事件搞得德国,在跟俄国搞经济制裁,搞得两败俱伤。他们最大的希望就是要北溪 2 号建成以后,他就是等于已经有恃无恐。



王孟源 26:13 

所以你看过去的这个月美国的第一个反应就是,我先看我们能够怎样制裁国,这个制裁恶国手段,其实你看点像那个 Swift 或者是什么科技,什么制裁,经济制裁、贸易制裁,这些全都是让德国去当炮灰,德国跟欧盟去把当炮灰,那德国就站出来说不,那这个法国说我不。



王孟源 26:43 

因为这个法国虽然没有那么大的经济利益在俄国,但是Macron一直是一个泛欧大联盟主义者,所以有一个独立的,独立于盎萨集团的欧盟一直是他的理想,这个是他上任以来就一直讲得很清楚的,所以有这个机会它也是顺水推舟。所以你看过去的这几个礼拜,美国的态度越来越软,原因就是德国跟法国直接就说,NO,NO,NO, NO NO。你,你建议什么?我都说NO。那你们现在看到那个英国,跟美国是还想要把这个事情闹起来?美国现在基本上就是我去看他们的这个外宣,还有他们的外交的宣布,基本上都是只是要找下台阶,美国人已经知道他们玩不下去,这个必须要找个借口来保护自己的面子。



王孟源 27:53 

那举个例子,他们两天前说要动员8000,快速 8000 个部队,要快速反应部队,这是在美国的部队,要准备要派到东欧,他说东欧是乌克兰吗?不是,是保加利亚;这个动员是像上个月俄军两天内就把一个空降旅送到 Kazakhstan 那样的吗?不是的,他们这个不是快反部队,是预备队,他的这个预备的时间是10天,上个月俄军到 Kazakhstan那个快反部队是 24 小时的,快反部队就是 36 个小时就已经全部到达,从下令到到达 36 个小时就全部到达。这个美国的这 8000 个部队是原本是 10 天之内可以开运,说这个不太一样,开运不是到启程,哈哈哈,启程?对, 10 天内可以启程的,他这个所谓的准备要送过去,其实只是把 10 天提升一级,变成 5 天,就是开始做一些早期、前期的准备,所以 5 天内之内可以一起走,你说这不是哄骗外行人的东西?对,基本上就是找个下台阶,假装自己有在做事情。



史东 29:24 

所以我现在为你的刚刚讲的话做一个结论,你开始正确还是不准确?美国是现在是希望找台阶下,因为美国知道这条路走下去自己是玩不下去的,英国是唯恐天下不乱,因为反正如果乱的话,英国不倒霉,是法国和德国倒霉,特别是德国倒霉对不对?乌克兰的话是希望从这些中间,乌克兰的现在来的总统希望能够获得查一点自己的政治力。



王孟源 29:57 

他刚刚拿到 12 亿的欧盟的那个援助。



史东 30:03 

OK,我跟你讲。顺便我插一句,哦,我今天看了一个评论,他是他在开玩笑,他说美国政府并没有给乌克兰政府多大的援助,但是老实说这个评论老实说如果给了很大的援助的话,这经过了乌克兰的政客大哥也剩不下不了多少,哈哈哈。哇。



王孟源 30:26 

美国人吝啬得很,要花钱的东西最好是叫盟友。对,你刚刚讲的这都对,我补充一下,就是现在这个法国其实Macron已经预定要跟Putin做直接的谈判了。就是,你这个知道美国人现在只是要早下台阶,那这个既然美国人不能、不愿意软服,那就让法国人来软服好了。



史东 30:59 

照照这样子的形式发展下去,我看北溪二号的使用是迟早的事情。



王孟源 31:06 

就是我上一次已经说过这其实 puting 也不急就是。嗯,德国人也不急,因为它是备用的东西。嗯,对不对?只要是在那边备用,你现在用不用都不要紧,但是乌克兰要是被打烂的话,你绝对是有正当的理由马上就把它对打开,对不对?



王孟源 31:31 

所以这个它的意义在于存在,而不在于被使用。



史东 31:36 

这句话说的很有意思,哈哈哈,存在的意义大于使用的意义。







——————————————————



王孟源 31:59 

这个美元还是国际储备货币,他在国际上占的额份,国际货币上占的额份是大约60\%,那实际上美国在国际贸易上占的比率只有 40\% 左右。其实他是有一个 6: 1 的杠杆,他每印 6 块钱,只有一块钱是真正在他的国内流通的,其他 5块钱是被国际其他国家吸收。换句话说,他每产生 6 块钱的通货膨胀压力,只有 1 块钱是他自己承受。



——————————————————





王孟源 32:50 

对,所以你看现在,你刚刚讲的这个美方,我就像我讲的是准备要找下台阶。那英方的话,他反正美方还有顾忌。美方的顾忌是什么呢?就是如果真的闹大了,然后美方又什么都做不出来,他这个盟主的脸摆不下去,所以他必须要找下台阶。英国,连这个都没有,都不用在乎,所以他还没有这个顾虑,所以他要怎么闹都可以随便。所以这现在你过去这个月这个媒体的这些胡扯都是英国人在搞的,这个外交官,还有那个媒体一起搞。



史东 33:30 

这个事情。讲到这,我们再看看俄国方面,俄国将会达到它需要达到的目的吗?他达到目的,他的目的也不高,他就是第一个乌克兰不能入北约,第二个你美国或者是北约不能再继续地往东扩建。



王孟源 33:46 

那是 2022 年的条件(编注:结合上下文应为2012年),现在……而且是从 2014 年开始,他就做了一连串的提升。你记不记得 2014 年, 2015 年他被经济制裁的时候那有多惨?而且是忍着、咬着牙,忍着那个经济制裁就是要把那个Crimea吃下来,为什么要保存黑海舰队呢?嗯,对,你这个没有黑海舰队以后,你这个整个他的南方的腹地就完蛋了,跟那个美军的军舰来了,你什么都没办法敢做。然后那个时候欧盟对他做经济制裁的时候,他还主动地加码做反制裁,就是你要跟我切割脱钩,我就做得彻底。为什么?因为他就是打算六七年之后要来跟你摊牌,你要摊牌的话,你自己的经济要你先适应完全脱钩,然后想办法。



史东 34:45 

他先把自己的体格练好。



王孟源 34:47 

要先把自己的体格练好,反正下一次对方还是会要,会威胁着要进一步脱钩,那我就先把能脱钩的全部都脱掉,那你就没有筹码。现在美国面临的这个没有筹码的窘境就是这样来的,不是运气好,是Putin花了 7 年咬着牙忍着极大的痛苦创造出来的局面。他这个我说他过去这三个月没有征兵,这并不代表说他没有升级这个态势的意愿。3月跟4月的时候,他的确是把西伯利亚的部队都调到西部军区去,我刚刚讲过他们的确是在三四月的时候,的确是大兵压境。在那之后Putin觉得这是一个很好的时机。既然乌克兰要搞鬼,我就跟你摊牌。但是他摊牌并不是一次性地要达到他的长远的战略目的。他长远的战略目的是什么?他要重建缓冲区、安全区、缓冲区,或者是你可以,也可以说势力范围。



史东 36:00 

所以说停止西进并不是它的最终目的,要停止西进。



王孟源 36:05 

只是 2022 年的阶段性目的,因为他觉得他的实力消长,就是他跟中国结盟之后,就是非正式结盟之后,这个实力消长,时间还是在他这边,到了 2025 年之后他可以获得更好的条件。但是你如果是一个蠢蛋的话,也许就直接说:那既然那时候比较好,那现在就忍辱负重,那美国要怎么欺负我随便让他欺负。不是这样子。你这个很多这种驻军或者是边界或者是惯例,这些事情是一旦创立以后就很难改。所以你越是战略态势,是长期战略态势,越有利你越要立即止损,不是ultimatum,不是最后通牒,而是要止损画下红线。这个Putin我上次已经讲过,是非常有智慧跟聪明才智的人,。



史东 37:21 

他这个就是,至少你不要让事情更加的恶化。。



王孟源 37:26 

对,就是你这个乌克兰又来闹事,每次闹事的话就是改变现状的一个机会。嗯,那他怕的是这个美国又利用这个机会,美国跟北约又利用这个机会往前推,所以他现在这个要求是画下红线,止损,立刻止损,你这个乌克兰跟 Georgia不能够再加入北约,不能够再跟西方有军事合作,但是这不是长期的目的。然后长期目的他也讲了,但是我相信他并不要求那个北约立刻满足,他的长期目的是要北约把设在东欧的——就是原本华约国家——的那些战略武器,尤其是在波兰跟罗马尼亚有两套反导系统,这个必须要撤走。嗯,我相信这是他最终的战略目的。当然一个次要的战略目的是引发美欧之间的隔阂,而且他也的确很成功。



王孟源 38:44 

对,嗯,就是他知道为什么可以这样做,因为这又回到那个北溪,因为北溪建成,北溪建成以后德国可以有本钱,可以跟美国说不了对不对?所以虽然 Merkel退休,我想不仅也知道Scholz 跟 Merkel 有默契,所以他知道如果他跟美国现在做这个有限的摊牌的话,就可以引发美国跟德国的决裂。就是现在即使已经实质决裂,只是表面上都还在粉饰太平。



史东 39:22 

还有一个 related question,我们把这个问题解决之后,我们就换一个议题谈。嗯,你刚刚也谈到了这个台湾这个事情跟台湾的有相当程度的联想性,你觉得这个事情以你的观察对于有些什么样子的启发?



王孟源 39:51 

我认为美国的套路永远都是这样,但是从过去 200 年套路一直都是这样子,就是自己不出头,叫盟友去当炮灰,然后而且这个盟友是分第一线跟第二线。你看以这个欧洲的这次事件,乌克兰是第一线的军事外交炮灰,第二线的经济贸易炮灰是德国,所以 2014 年的时候它大这个策略又大获成功,对不对?就是这个乌克兰损失了领土,然后损失了军队,然后损失了稳定的经济发展的条件,然后德国则损失了几百亿的德国企业在俄国的经济利益,还有贸易的经济利益,美国出了什么?什么都没有。他在两边左右逢源,对不对?这个就是德国跟俄国……俄国提供能源,不行!美国也是能源工业国了,对不对?那个德国的企业吃亏了,那美国就可以趁机要去抄底。对,你去找底就是都是间接的利益。



王孟源 41:12 

我在我的博客上也讲过,这是一本一个无本万利的生意,所以你也难怪美国人这么热衷,过去 200 年玩来玩去就是这一套。但是问题是这个要玩这个无本万利的生意,有一个前提,就是你的外宣要彻底,外宣要洗脑,洗到彻底才会有心甘情愿的炮灰让你利用,对不对?因为最倒霉的就是这些炮灰,这些炮灰损失比你的对手还要大,德国跟乌克兰的损失在 2014 年下来比各国还要大,那所以你不只要有第一线,还有第二线的炮灰也都要愿意。



王孟源 41:58 

那你说在东亚这个第一线的炮灰是什么?就是台湾,还有现在找出来的Lithuania, Slovenia 这些都是第一线的炮灰对不对?然后第二线的炮灰是什么?对着台湾背后的第二线炮灰就是日本。这 Lithuania 跟 Slovena 背后的炮灰是德国。



王孟源 42:23 

我不晓得……日本是没有什么智慧的,日本没有战略智慧,从来都没有。我觉得中方对Lithuania 的制裁不够,太客气了,直接公开讲说你这个违反国安,直接断交就可以了。德国必须要找借口,见死不救;但是他们的问题你不必要,不要担心他们会找什么借口,民主国家的政客找借口是他们的专业。



史东 42:58 

你不用担,不用替他们担心。哈哈,对,OK,谈一谈美国,谈谈美国的财政,谈谈美国的货币金融。



王孟源 43:09 

对,其实这个乌克兰这件事情我想讲到这里已经很明白了,对不对?就是 95\%- 99\% 的几率是无疾而终,就是乌克兰再要到十几亿、 20 亿的援助,然后那个俄国跟德国、法国的和解,然后就慢慢地沉积下去,那背后吵闹的还是只有英国,对不对?美国人也没有什么兴趣……



史东 43:40 

我刚想到一个,我觉得很重要,你觉得北约的前途如何?



王孟源 43:45 

Well,我觉得这么一闹,更加的帮助Macron,因为 Macron 一直说要建立欧盟自己的军队。你这个要先建立欧盟自己这欧盟的,你要减弱北约,先决条件是要先欧盟有自己的军队,要不然欧盟的国家自己一个都打不起来。这个就好像你德国要置身事外,必须要先有北溪就好,对不对?这看起来好像是很间接没有联系,可是实际上你真正仔细去想,这个它的战略因果关系的话就很明显这是必要条件。



王孟源 44:28 

那本来 Macron从 5 年前上台到现在,他上台第一个外交的 initiative 就是要创建欧盟的军队,那个 Merkel就根本假装没有听到,为什么?因为 Merkel连自己国内比如说像Minsk agreement,或者是像那个对俄制裁,他都不敢公然地为违背美国的意思,就是北溪 2 号那个必须偷偷摸摸做,你不能够明说,我是要摆脱你的那个桎梏,这个必须要很间接的,很间接、很委婉、隐性地来。所以 Macron这个高高兴兴地说我要建立欧盟的军队。Merkel一直就笑一笑,然后不当一回事,因为这不是priority,不是她的priority。但是这个形势在去年改了。这改的是什么呢?就是 AUKUS那件事情。我不晓得,你有没有注意到我在博客上谈到的。



史东 45:39 

就是你关于这个原子能潜艇的事吗?



王孟源 45:42 

对,那个AUKUS 是那件事情跟现在这个乌克兰的形势如出一辙。也是 Boris Johnson 在在那个为了自己的政治私立,把 Biden 摆了一道。你看现在 Biden 这个窘态,不是被英国拱了吗?美国人已经准备要找下台阶了,你英国反而在里面造势,这并不是真的帮忙,可是他又不能够说是这样的,你这个英国人先撤了大使馆,美国人也要跟着撤啊,对不对?要不然你脸上不好看说不过去。当初那个 AUKUS也是一样的,那个事实上是我在我的博客上有详细的论证,就是这个一定是英国先跟澳洲先搞出一个协定,因为如果这个英国……美国跟英国都有核潜艇技术,澳洲到底要核潜艇,这路从哪里来?如果他是要跟美国买的话,根本就没有英国的事情,所以,之所以会搞成三国的协约。一定是原本跟英国买,因为英国的核潜艇技术里面核心技术,就是那个核的技术和反应炉的技术,那还有一些子系统都是要跟美国买的,你只有一定是澳洲原本是先跟英国交涉,然后才去找美国人首肯,所以美国人原本是一个第三者。



王孟源 47:32 

另外的就是这个  AUKUS是搞成一个烂摊子,一个很大的原因是它对法国保密,这个Macron是到新闻爆出来了才从电视上看到。哈哈哈,你说这个够不够气人?为什么要保密?美国没有理由保密,澳洲没有理由保密,要对法国保密,唯一有理由的是Boris Johnson,英国也没有理由保密,但是 Boris Johnson 有理由对法国保密。为什么?这个他是靠硬脱欧搞出来的,这个硬脱欧的基础之一就是排外的思想,就是他的那个保守党的选民都是排外的。那这个为了解释,为了 distract 他的这个政策的一塌糊涂,准备得非常地……说不充分都还是很客气的。他一直都鼓动跟法国人闹,尤其是那个在渔权的事情上面,两国差一点就派军舰打起来。这个我不知道你记不记得,这是半年多前的事情。嗯,所以刚好这个 AUKUS就发生在那段时间。所以它这个建立一个比北约还要小小的一个小圈子。北约已其实已经是一个很强力的这个军事同盟,你建立在这个同盟里面,又建立一个更核心的小圈子,而且还对法国保密,然后还把他的生意给抢了,羞辱法国人,这绝对是有益的。



王孟源 49:14 

OK,而这个有益的最后的推动者,唯一有动力有理由会受益的就是 Boris Johnson 个人——不是英国不受益——Boris Johnson 受益。然后后来Macron就得理不饶人对不对?那个这 Boris 去捅出这个娄子来吃愧的是谁?吃亏的是Biden,为什么?因为要道歉的是Biden。



史东 49:45 

吃了哑巴亏。



王孟源 49:46 

他吃了哑巴亏。后来这个事情闹出来以后, Biden 必须要打电话去跟Macron道歉。那个电话后来没有公开,但是我当时在博客就讲,我可以保证Macron同意原谅 Biden的条件,一定是让他跟他说我要建立欧盟的军队,你不能反对。



史东 50:10 

所以说未尝不是个好事。



王孟源 50:14 

对,这其实未尝不是个好事。然后同一个礼拜, Boris Johnson 到华盛顿去拜访Biden, Biden 全程板着脸,他刚刚理解到是怎么被 Johnson 给玩弄的那个,他的心情如何可想而知。所以那个 Johnson 那一次访问他的外交目的是要推动自贸协定,就是英美的自贸协,这个原本美国从 Trump 到 Biden一直都是模棱两可,就是说看条件怎么样,但是Boris的会见结束的时候, Biden字直接说no,谈都不用谈,背后的这个气氛之尴尬,这可以想象。嗯,我觉得这一次再经过这个乌克兰,这一次这么引导欧盟的军队应该是可以搞。



史东 51:19 

因为我个人一直在看这个事情。北约自从……老实讲,从一个小老百姓的角度来看这个事情,自从苏联解体之后,北约已经没有它存在继续存在的价值了。



王孟源 51:33 

其实 CIA 跟 MI6 也没有存在的价值。,对,那你跟那些人说,那有人就是权力很大。



史东 51:39 

是啊是啊,我们今天讲的是北约对不对?



王孟源 51:41 

然后原本天性就很坏很恶劣的人,不讲什么道理。



史东 51:47 

就是一个尾大不掉,而且不断地为自己的存在制造机会的一个庞然大物,一一个既得利益者,对不对?



王孟源 51:57 

就是对,我不是说北约马上就会解体了,我的意思是说光欧盟建立军队可能就需要另外 5 年,而且还要看Macron是不是赢得这次大选,我们还要再等到四五月对不对?他的继任者,如果他没有获得连任的话,他的继任者不见得有这个兴趣,因为现在挑战他的那三个,排名前四名的除了他自己之外都是右派的,右派的都是对欧盟的整合比较不乐意的。



史东 52:32 

哦,这样子。因为听来这个欧盟建军这个事情是一个吃力不讨好的事情。对不对?



王孟源 52:37 

应该做的事情。



史东 52:39 

确,如果做的事情应该是。



王孟源 52:40 

因为,对,应该做,为什么呢?因为他是北约解体的前提,必要条件就是我不是说北约在未来 10 年会解体,我是说你这个,你如果要看北约慢慢的越来越不重要……



史东 52:58 

我这么了解好了,如果北约不解体,如果希望北约不解体的背后的力量是什么?除了美国之外。



王孟源 53:09 

所有的那个欧洲也有很多这种强硬的,比如说就是那种逢俄必反,逢中必反的人,他们会希望北约不要解体。



史东 53:21 

那么那么如果欧盟去建军的话。



王孟源 53:26 

欧盟建军是必要条件。对,但是不是充分条件。



史东 53:30 

对,欧盟建议欧盟如果建军不建军的话,北约解体连谈都不用谈。



王孟源 53:37 

连谈都不用谈,不可能,没有人敢谈。



史东 53:42 

这就了解了。好,台湾问题,你还没讲完呢是不是?



王孟源 53:47 

哈哈哈,台湾问题,我说不定要回台湾了,但你现在台湾已经立法了,说不定我就莫名其妙地关起来了。



史东 53:56 

哦,这样子。OK。



王孟源 53:56 

我倒是不在乎被关了这个。说实话被关,为了说实话而被关,其实是一种光荣的事。



史东 54:05 

对,哇塞,你也了不起。



王孟源 54:07 

但是我回台湾的目的是照顾我的年迈的妈妈,如果在她过去过世之前我就先被关起来,反而影响她的心情。这个就违反……



史东 54:20 

这个事情我们就我们心照不宣了,我们不能说不谈了,心照不宣了好不好?谈谈这个货币跟金融,美国的货币跟金融。



王孟源 54:28 

其实我觉得跟乌克兰相对起来,这个乌克兰是茶壶里的风暴,就是没有什么实质,但是表面上炒得非常厉害。美国的这个通货膨胀是刚好相反,就是你媒体没有怎么深入地谈,没办法谈得很确定,但是它是却是非常深刻、影响非常重大的一件事情。我从开始写博客, 2014 年开始写博客,我就一直强调在冷战后的这个美国霸权基本是建筑在美元之上。你说在冷战期间,你还可以说他的军事、他的科技、他的企业,他的宣传文化等等的都是他的霸权的支柱。但是我想到现在应该是很明显的,就是他的这些统统腐化了,你的那个……他的军事,你看看他们的现在造一个新的航空母舰,他们福特号航空母舰,基本上就是企业号换的几个子系统,但是拖了四五年,到现在还没有真正复议,你说这个,是前所未有的。他现在美国的经济烂到经济跟科技烂到什么地步?你看看那个加州要建个高铁,细节我也不用讲了,这个钱花了比中国多了 10 倍,但是到现在越裁越短,而且那个建成的时间越推越后,基本上就是一个笑话。



王孟源 56:13 

美国现在是什么事情?就是全面的腐朽,就是从学术界开始,学术界、智库界、思想界,然后到企业界,到到军事力量,什么以全面的复兴,现在还一副太平盛世的样子,全都是靠印钱。就是你看他的那个财政资质现在已经是固定固化了,就是每年 1 万多亿的财政赤字,然后他的贸易资质也是每年一换。对,你像这样子,他还能够每年的经济成长比欧洲要高,比德国要高, GDP 成长还是固定地每年都比德国要高,然后股市什么的欣欣向荣,然后一大堆什么新的投资,你看看这些投资的是什么东西?我已经解释了,反反复复把所有的那个技术角度都已经解释得很清楚,就是核聚变这种东西——我这只是举一个例子——核聚变,这个核聚变发电这东西是别说是 10 年、 50 年,对, 100 年、 1000 年都做不出来。它不可能有经济性,因为它的那个污染性太高,你技术上要跨越的困难太多。你同样发一度电会比核裂变发电贵至少 100 倍,那这种东西,在科幻小说里面没有人会跟你算账,但是在实际经济里面,你只要贵超过10\%,就在经济上就不可能竞争得过。



王孟源 58:07 

你说比人家贵 100 倍,你怎么可能做出来?那你何必浪费几千亿、几万亿去开发呢?这种事情都是可以事先预测的,就像永动机一样,你为什么要浪费钱,浪费人才去开发永动机?但是美国现在,在过去两年印了 5 万多亿。印钱,这个货币这个东西……标准的巨观经济理论就会跟你讲,这个印钱你多印1万亿,不是你的经济产出多一万亿,而是会多出好几倍。



史东 58:51 

因为他的流通的关系。是吗?



王孟源 58:53 

流通的关系,对,就是你这个钱应出来交给银行,借给地产商,建了1万亿的房子,卖出去以后,地产商又拿回这1万亿,他拿去豪车买游艇,那这些卖豪车卖游艇拿到这1万亿又转过头来去买地产,OK?又或者是去买bitcoin 这个东西,这叫做 velocity of money。



王孟源 59:28 

我不是开玩笑,真的是钱的速度。这个钱的速度就是说你这 5 万多亿照理说应该对经济有大概 3 倍左右的刺激,一年之内你说美国的经济这样欣欣向荣,跟德国比起来是多产出了多少?没有多少,绝对没有 15 万亿,没,绝对没有超多产出。 15 万亿就是……德国的,德国刚刚在 2021 年的第四季度基本上恢复到新冠疫情开始之前的经济产出能力。美国是早了 2 个还是 3 个季度?这早这两三个季度是怎么来的?是靠着印了 5 万多亿。 5 万多亿是多少?是它的 GDP 的 1/ 4,它的 GDP 是 20 万亿,你多印了 GDP 的25\%,结果你的GDP 成长只比人家要高 2\% 3\%,你说这个经济的底子好吗?可以告诉你,实际上是摧毁了很多的价值,所谓的 value destroying。



王孟源 01:00:44 

不懂经济或者是金融的人常常会说这个国债,其实国债根本就不是问题所在,美国根本就不可能还不出国债,因为他们的国债是用美元定价的,反正你到时候还不出来就再印。这所谓的财政赤字跟贸易赤字,他的问题是,你现在印了 5 万多亿,为什么好像都是只有好处没有坏处?因为美国这个美元还是国际储备货币,他在国际上占的额份,国际货币上占的额份是大约60\%,那实际上美国在国际贸易上占的比率只有 10\% 左右。就是他是有一个 6: 1 的杠杆,他每印 6 块钱,只有一块钱是真正在他的国内流通的,其他 5 块钱是被国际其他国家吸收。换句话说,他每产生 6 块钱的通货膨胀压力,只有 1 块钱是他自己承受,其他 5 块钱是中国、俄国、土耳其、南非、巴西、印度来承受。





——————————————————————



王孟源 01:02:18 

美国现在已经没有一个理性的决策者,就是整个决策团队里面没有一个人有理性。



史东 01:02:25 

都被踢掉了,也不敢有理性了。



王孟源 01:02:29 

就是他这个,就是我刚刚讲的,他这个体制腐化得太厉害了,逆淘汰太厉害了,但是他有一个东西一直没有碰,他比这个,这比金融改革,比银行改革,比房地产改革、比税制改革,我觉得长期来看都还要更严重。就是他的学术界。



——————————————————————

王孟源 01:02:52



对,就是中国现代制度。这个学术之所以会这样的腐败,是因为他不但对专业权威,就是不是正确的专业认知,而是学阀有绝对的崇拜,有一宗教性的崇拜,而且更胜一步赋予这些学阀政治权力。那你这样一来连从政治上、体制上都不可能去挑战这些学阀,就是即使他们只是靠造假夸大做出来的东西,你都没办法在科学专业上去挑战他们。而中国的这个学术管理体系、体制如果不改革的话,这就是必然的结果。因为别的有关人类的事情或许是随机而变的,但是人性永远都是自私,会腐化的就是没有怀疑的。你只要这个体制保护这些腐化的既得利益者,那么这些腐化既得利益者就只会越来越做大。



——————————————————————



王孟源 01:04:14 

他这里最大的问题是,他必须要假定他自己所占的那 60\% 的额份是永远固定的,就是不降,不下降。如果这个额份开始下降的话,这个额份怎么下降?就是人家把这个美元不用了送回去还给你美国,我跟美国你买东西,或者跟其他愿意用美元的国家买东西,然后呢?花完之后我改用其他的货币,对不对?你这个额份开始下降以后,美元就越来越不值钱。



史东 01:04:47 

就是它的稀释能力越来越低。



王孟源 01:04:50 

稀释越低。对,所以就是你原来的那个正反馈,现在反过来变成负反馈。美国的维持美元始终是因为它有一个完整的——就是在 70年代,他最早先打破Brettonwood system,刚好开始采用浮动汇率的时候,就是到现在大概 50 年了。一开始的时候是因为冷战,所以欧洲、日本、英国这些先进国家——那个时候先进工业还是集中在很少数的国家里面,就是你真正的所谓的工业经济,其实就是那几个国家——那些国家拿他没办法,因为都是军事上的附庸,你在都是在北约,要靠他的,那他说怎么样就怎么样,那到了 1985 年,他连这种间接的占便宜都嫌太慢,直接就召开广场协议,强迫他们——就是原本是十几年前是他主自己说浮动,浮动以后他就拼命地大印钞票,然后占了便宜之后,当时其他的这些先进工业国也就跟着印钞票。就是你当时有两个选择,就是你改用其他的货币,那你如果不敢改用其他的货币,因为你是他的军事附庸,那你就只好跟着印钞票。70 年代之所以会有全球性的通胀或者滞胀,就是那这样来的。那之后到了八零年代,美国人说这样不行,因为在八零年代早期,他把这个他忍痛,那当时有一个美联储的主席,他把利率升到了18\%,终于把这个通胀给制服了。但是这么一来,他连续两年那个经济的严重衰退,然后其后一直恢复不了元气,就是他把美国率先把通胀给制服了之后,就跟其他的国家说,那现在因为我把通胀制服了,所以我的货币开始坚停了。坚停了以后虽然解决了通胀问题,但是在国际贸易上开始失衡,就是我的进出口开始有逆差。不行不行,这不是我的问题,是你的问题。OK,这不是我编的,当时的,呵呵,当时的美国人……



史东 01:07:31 

就是这句名言了。



王孟源 01:07:32 

有一句名言, dollar is our currency but your problem 哈哈,这真的不是我编的,是美国的财政部长讲的,哈哈哈。所以他后来到了 1985 年又反过来去强逼那些国家也回收货币,OK,而且要很快,他们强迫你,一年之内要让你的货币升值百分之百。你说这真的是要命,他真的就把日本的命给要了,那英国的去工业化也是在那个时候完成的,那时候唯一逃脱的主要工业国是德国嘛。那后来这个欧盟跟欧元都是经由 1970 年代跟 1985 年的两次经验,(史东:教训)就好像过去的 7 年, Merkel是为了避免 2014 年的重演,当时的德国的领导人也是为了避免 1972 年跟 1985 年的重演,所以才赶快加入欧盟建立欧元。现在美国人他没有想清楚,当初德国跟日本可是有苏联在威胁的,没有办法,他们不能够公开反叛,你现在根本就没有这些辅助的威胁的工具。



史东 01:08:59 

客观情况,不一样。



王孟源 01:09:01 

不一样,你现在还在这边答应美钞,然后这基本上是重新玩50 年前, 1970 年在通货膨胀的那些花样,就是我印钞票,但是因为是国际货币,所以大家都必须收,就是印出6块钱,有 5 块钱是你们收的,他自取灭亡。只要其他国家,我相信德国跟法国连在乌克兰这件事情都不愿意再让他敲诈,在美元滥发美元这件事情上,他也不会再愿意配合美国,让他榨干自己的血汗。所以中方跟俄方——其实俄国人早就提议了,俄国人在十几年前就提议了。对,但是当时中国还没有准备好,现在中国如果愿意的话,很容易,因为中国的实经济的实体规模已经大于美国了,它在国际贸易的占比已经达到13\%,就是比美国还要高,跟俄国。



史东 01:10:08 

这是什么意思?它 13\% 是怎么一个概念?是全世界贸易的13\%,是和中国贸易,是这个意思吗?



王孟源 01:10:17 

对对,全世界 13\% 的。



史东 01:10:21 

贸易金额。



王孟源 01:10:22 

是中国的出口



史东 01:10:24 

和中国的交易。



王孟源 01:10:26 

其实是 15\% 了, 15\% 的出口,大概 13\% 的进口。然后你看看,再看到另外一个很大的危险,就是从奥巴马开始,他就一直要想要从中东抽身,然后专心转向亚洲转向,这其实是非常不智的。因为你从那个美元的观点来看, 1970 年代除了他,他对北约跟日本的军事掌控之外,另外一个大家不敢反抗美元霸权的原因,是跟沙乌地阿拉伯的持有美元的绑定。那现在你看他过去这几年阿拉伯联合大公国、阿联酋——大陆说是阿联酋,那台湾说是阿拉伯联合大公国——就是UAE,然后还有沙乌地阿拉伯,他们对美国都离心离德。你光看沙乌地,现在正在,我们现在讲话的这个同一天,沙乌地正在跟伊朗谈判要恢复外交关系。沙乌地当初敢跟伊朗断交,就是靠着美国这个靠山,他现在跟着伊朗重新谈,就是觉得美国靠不住了。



史东 01:11:45 

靠山靠不住了





王孟源 01:11:47 

要不然他不会去跟你讲谈。对,所以在过去这十年、十几年、 20 年,基本上就是 2003 年第二次海湾战争之后,沙乌地一直把伊朗当做头号敌人,他现在必须要改过来,不是他愿意,而是不得已的。你再看那个UAE, UAE 在上个月才刚刚宣布决定不买 F35,这个交易可是在Trump的任期内, UAE 跟以色列建交换来的,你看这个建交换来的,他忽然觉得说不干了,为什么?他说他这不干的理由是什么?是美国换成Biden上台以后又要求他不用华为,然后不让中国去建他的港口,他说不行,我要跟中国维持关系。然后像这样的是不是……人家都已经准备好了,就是等着你去跟人家谈判交易,中国很简单就可以跟他们讲,我不用美元跟你卖石油了。俄国已经不用美元卖石油了,委内瑞拉不用美元卖石油了,你们也不要用美元卖石油。反正美元现在在通货膨胀正在贬值嘛,对不对?你们手上的那几千亿几万亿的美元,你们自己也是头痛得不得了,换成人民币多好。



王孟源 01:13:08 

哈哈哈,这两个墙角一挖,我跟你保证这是美国霸权终结最可能也是最理想的一个脚本,这个机会就出现在今年,所以 2022 年是一个很重要的年份。我为什么说今年?因为其实美国的这个量化宽松还在进行之中,它是所谓的Taper,并不是说我开始回收这些钱了,而是我印钱的速度降低。但是 12 月的时候,就是上个月的时候宣布要加速taper,然后现在预定要到3月结束;3月结束之后很可能要加息。目前今天早上是 JP Morgan,现在是 JP Morgan 是全世界最大的 investment bank,那个 Dimon 是他的CEO,他说他预期今年加息 7 次。



史东 01:14:10 

七次是几乎是两个月一次。



王孟源 01:14:16 

就是不到两个月就要加一次,不到两个月加一次。这个这样紧收银根的话,第一个反应是土耳其就完蛋了,因为这个土耳其现在已经通货膨胀一塌糊涂对不对?就是越弱的死得越惨,第二个,另外一个是南非,另外一个是阿根廷,巴西也是很弱,如果这个继续扩大的话,印度也会倒下去。但问题是它一面加息,但是这个通胀不见得就下来,我认为通胀不会很快下来。你这个加息……不过一方面加息会造成造成全球性的银根紧缩,但是在美国却没办法。这有两个原因。第一个是你这个目前的通胀是供给链有问题,不是、不完全是货币的问题,很大的原因是因为受疫情影响,所以供给链供不应求。那远洋运输也不畅,不顺畅,美国国内的运输也不顺畅。第二个原因是:我们两年多前我跟你做了一个节目,讨论 2019 年,那时候是9月底还是 10 月初,我们讨论当时的这个回购利率暴涨。对,一个利率暴涨的意思就是大家都没有现金,所以大家都抢着要买回购。那现在情景刚好相反了,现在这个银行的现金太富,哈哈,所以他们也有一个东西叫做反回购。



王孟源 01:16:04 

那问题是美国提升利率以后,这些美元从世界各国——我觉得这些比较弱的经济体,像土耳其——他们回归美国以后,反而是美国本土的钱更多。他本土这个通胀这么厉害,本身就已经是钱太多,你银行的钱太多,投资人的钱太多,然后企业的钱太多,然后你再从海外把这些美元吸收回来,然后大家都拼命地去买联邦债券什么的,那光是股票往下稍微修正一点是没有用的,你整个这个通货膨胀的局势还是在那里。所以我觉得,他们现在这个财政跟货币的情形是非常的尴尬。就是照理说你牺牲了,你这个回收了这些美元的时候,牺牲的是外面的这些替死鬼,然后你的这个通胀的情形就会改善,但是我觉得很可能是该死的还是死了,替死鬼还是死了一堆,但是通胀还是在那里。一方面股市往下跌,然后一方面现金还是过多,然后供给链还是供应不上来,所以这个是说得不客气一点,落井下石的天赐良机。



史东 01:17:38 

你觉得……我插一句话,美元,你觉得这个你刚刚谈到通货膨胀的原因,提到这供应链的问题,这是一个很大的问题,你,你觉得这个美国对中国课这个进口税是多大的一个factor?在整个的这个通货膨胀的这个氛围。



王孟源 01:17:59 

不是决定性的因素,因为它那个……不是决定你性能因素,但是,当然是火上加油,有一点点,那任何一个理性的决策者都会早就,刚看跟中国互降关税,然后来解决这个问题。但是,美国现在已经没有一个理性的决策者,就是整个决策团队里面没有一个人有理性。



史东 01:18:27 

理性都被踢掉了也不敢有理性了。



王孟源 01:18:31 

就是他这个就是我刚刚讲的他这个体制腐化得太厉害了,逆淘汰太厉害了,所以真正有理性有头脑的人都被踢到一边去了。你连那个诺贝尔奖得主,像是Stiglitz这些人,根本就没有人听,他以前还有一些什么职位,你现在根本什么都没有。嗯,你连诺贝尔奖得主都被踢到一边,你这个年轻出来的年轻学子,只要是有头脑的一定会被排斥出去。嗯,我当初也是被这样被踢出高能的。



史东 01:19:05 

对对对,这个就是劣币驱逐良币,就是我们常见。



王孟源 01:19:08 

驱逐良币。对,因为那时候不是我一个人,高能物理那时候还愿意说实话的,全都转完了。因为不但您有原则有品德,而且您有能力重新创业,对不对?那些反而是那些没有本领的人不敢走。



史东 01:19:33 

很多事情都是类似的。对,比如说在一个公司里或者邦无道,或者公司无道的话,你先走的都是最有能力的人先走。



王孟源 01:19:43 

对对,但是那个公司不见得马上的倒,就是,他就是那边慢慢地浮现,浮。



史东 01:19:50 

转过头来,你观察一个公司的盛衰,或者一个国家的,你就看他的人才往哪边走。对,人才走的方向已经看出来了。



王孟源 01:20:01 

我一直……我现在顺便讲一下,过去这两年习近平所进行的改革,我是很佩服的。就是四五年前Trump在打贸易战的时候,我还很生气地或者很着急地说你不能够退让——就是他们那个时候还是很官僚,很本能地就退让——但是你不管怎么说,共产党的学习能力是很好,在过去,到了 2019 年之后,基本上他们已经看清楚大势。就是而且不但在外交战略上采取的都是正确的大策略,在国内的改革上他们居然也有那个胆量同时在面临外部的压力的时候,同时进行深刻的内部改革。



王孟源 01:20:58 

你像教育、医疗这些改革是我多年来一直呼倡的,对,会做呼吁,但是他连娱乐也去改革,娱乐业也去改,那这个我就很佩服。因为如果是我的话,我会觉得这个是比较次要的,可以先搁一个,但他连这个都愿意去马上去做,就是剑及履及地去做,这个我就很佩服。但是它有一个东西一直没有碰,它比这个,这比金融改革、比银行改革、比房地产改革、比税制改革,我觉得长期来看都还要更严重。就是他的学术界,中国整体来说是一个很兴旺的新兴国家,就是每一样改革起来, Can Do,他能够把事情办成,那唯一的例外就是学术界,它在学术界,比美国这种我刚刚讲已经腐烂了,在过去三四十年不断地腐烂的这个情形还要糟糕,就是贪腐还有造假夸大的情形比美国还要糟糕。



王孟源 01:22:20 

这个根源是什么呢?根源其实是文革之后,邓小平掌权的时候矫枉过正,因为你知道在的时候,很多科学家被斗倒、斗臭、斗死,那邓小平当时觉得要以科学治国,所以他也尊重专业科学家,这个原则没有错,但是他没有想清楚这个人。科学家也是人,你把权利给他以后,绝对的权利造成绝对的腐化,他们不但有专业权威,而且还可以当中央委员,可以想想看这有多糟糕。科学这个东西必须要摆出来,公开的挑战,公开的辩论,OK,这才是正确的、科学权威的根据,而不是你的头衔。但是你如果对不起这个你可以,你要裁定。



史东 01:23:22 

没关系,我在……我因为我想到一件事情,你再讲我就不愿意马上插进来,但是这样的话我就是,我就顺便提一下:这个好像科学的知识和科学的权利是不容混淆的,不能混淆在一起的,不容过对不对?其实就像神跟权,就政治权力跟神的那个力量是不能,不能够。



王孟源 01:23:52 

不能搞在一起,不能够搞在一起。宗教跟政治必须要分开,专业权威也跟政治权力必须要分开。



史东 01:23:59 

这是我刚刚脑筋里跳进来的一个思维。



王孟源 01:24:03 

对,这是中国现在制度,这个学术之所以会这样的腐败是因为他不但对专业权威——就是不是正确的专业认知,而是权威——学阀有绝对的崇拜,有宗教性的崇拜,而且更胜一步赋予这些学法政治权力。那你这样一来,连从政治上、体制上都不可能去挑战这些学阀。就是即使他们只是造靠造假夸大做出来的东西,你都没办法在科学专业上去挑战他们。他如果是省部级的大官,OK,你怎么能够去挑战他所说的那些造假出来的科学论文?不可能的,你连登都不登不出来。应该也是,所以这个中共现在学术界的腐烂是自己创造出来的。



王孟源 01:25:06 

嗯,是必然的结果。体制必然结果。这个东西如果不改的话,你的这个科学研发的速率会美国当霸主之下的世界还要慢。我一直是支持中国的兴起,我一直是希望能够有霸权的兴起,但是如果中国的中国的崛起代表着全世界人类科技研发的速度要减慢的话,那这反而是一件坏事。而中国的这个学术管理体系体制如果不改革的话,这就是长期的结果,必然的结果。因为别的东西,别的有关人类的事情或许是随机而变的,但是人性永远都是自私,会腐化的,这是绝对的,就是没有怀疑的。你只要这个体制保护这些腐化的既得利益者,那么这些腐化既得利益者就只会越来越做大。这是好,我认为中国当前的头号危机。



史东 01:26:25 

我希望重要的人物和人士能够听到你的呼吁。



——————————————————————————



王孟源 01:26:47 

我觉得,我自己有那个知识,能够真正影响世界人类的福祉,那人生也就是不到百年嘛。你光是汲汲营营地想着自己多赚钱有什么意思?你如果愿意的话就可以跟我一起奋斗。我觉得我或许看得比较远一点,但是我绝对不是危言耸听。这个中国学术界腐化的问题,是非常非常严重的隐性娼妓,把国家的科研经费全部浪费在上面,你说这可恶不可恶?而且我说的不是说平常的这个分配,这个多一点,那个多一点,我是说你到十四五计划,这是全国最重要的五年计划,里面排名前几名的全都是假科技,你说气不气人?



就是我小时候我家出生,其实没有什么钱,就是贫贱不能移,我已经试过了。富贵不能淫,我也试过了,就是我在华尔街干了快 20 年,但是威武不能屈我还没有试过,所以我想试试看。



——————————————————————————



史东 01:28:13 

我再把话题带回到美国这个地方,我在听你讲那个事情,我觉得,美国现在所处在的处境和它的前景让人觉得可悲的。替美国可悲的就是说似乎这件事情从美国的角度来讲已经无解了。



王孟源 01:28:38 

除非他革命,但是事实上这个体制也不容许革命了。他有一个非常高效的警察系统。



史东 01:28:43 

说你不管怎么样看这个事情,你已经是到了无解的地步,你只能看着它一步一步地往下滑,你没有什么可以作为的机会跟能力。



王孟源 01:28:57 

中国的历史上有十几个主要的王朝,那就是做到 200 年以上的,没有一个能够摆脱这种宿命,就是你这个既得利益者阶级固化之后就无法改革了,就是他们的既得利益集团太多,然后权力太大,你不可能做出任何改革。就是你即使知道该怎么改,也不可能真正做下去,而且越是改动越是危及整个国家体制。那美国现在就是这个局势,这问题是美国家底太厚,所以我不是说像明朝的崇祯片,这现在美国还像是明朝的万历,大概还有好几十年甚至上百年的时间了。



史东 01:30:03 

讲到这个节骨眼上面,你可以下一个结论,其实虽然你已经下了不少结论了。对对对,可以我们做一个总结吗?



王孟源 01:30:14 

乌克兰唯一打起来的 scenario ,脚本,是 Zelensky,主动去打东乌,即使在那个脚本,如果俄军可以靠炮兵轰击而把他们击退的话,也不会真的有俄军入侵的事情,所以你可以说 99\% 是不会有俄军入侵的事,OK, 99\% 以上。所以这个是,这基本上只是一个,现在剩下的就是外交事件,就是美国怎么样找下台阶下去。然后,当然现在这个铺天盖面的英国假新闻我看得很烦了,请大家不要到我的博客转述谣言,哈哈哈,辟谣…………



史东 01:31:06 

辟谣。可以想见这个谣言的威力有多大,他们的传播力量有多大。



王孟源 01:31:12 

对,我辟谣十几次已经很烦了,请大家不要,哈哈哈,说什么这个请问为什么/什么时候会打起来对俄军要怎样付钱怎样?这胡扯,俄军根本就是在他们的这个原宿舍里面,原本就是他们的驻地,这要花什么钱?唉,真是拿他们没有办法,这些消息其实都在那边,你只是要用心去查。



史东 01:31:44 

对东西这个消息,俄军驻军的这个消息已经被一些所谓的自由媒体人在访问的时候已经 debunk 掉了。



王孟源 01:31:54 

说对都这些都是回事。对,你只要愿意去找都可以找得到。这个比如说现在这英国一旦撤了大使馆以后,Zelensky一看不对劲了,因为现在反而是那个乌克兰的国民怕打仗开始到那个银行去把存款提出来,他吓了一跳。现在是 Zelensky 的政府,从总统、总理到部长一个一个接着出来,说实话,说俄国根本在过去几个月完全没有征兵。但是你在 Reuter上有没有看得到?你在 BBC上面能不能看得到?绝对看不到、但是这种东西你只要到 RT 上面去就可以看得到,这你自己不 ……RT 就在一个互联网上,你自己不会去看……你这么懒的话就没有资格在公共论坛上……



史东 01:32:46 

上。你我都是有心人,我们会主动地去找,大部分的人每天为了柴米油盐酱醋茶在忙着,这也是他们的最主要的听众和观众,。



王孟源 01:33:01 

而这也就是Anglo-Saxon这一套宣传系统有效的理由嘛,对不对?这其实也是合理的,那你如果没有那个时间去追究真相的话,那就,也可以,你就知道自己没有下功夫去追求真相,那你就不要在公共论坛上散布谣言,你没有努力就没有资格传谣,这是一个最基本的修养。我觉得你说不懂这些事实真相,这不是问题,不懂事实真相还在网络上胡扯,这个才是真正的问题。



史东 01:33:40 

对,就是不懂事实真相,还以为自己懂事实真相,这他们大部分人都是这样子。



王孟源 01:33:45 

对不对?乌克兰这件事情真的是,唉,没有什么好谈的。嗯,这是这一定是就等着看美国人怎么样找下台阶啊。然后真正的问题是看这个通胀的问题,如果中国不好好出手的话,说不定又被美联储蒙混过关了。那这个蒙混过关不是一件好事,你看看这美国,他在过去这几十年他的这个两个赤字坑是怎么补的,都是靠印美元去搜刮全世界的那个本土资产。中国至少是愿意做互利双赢,而且做实体经济,做真实的建设,而且愿意帮助后进国家建设一些基本的基建,还有基本的工业,美国人是继承英国人的那一套,就是口蜜腹剑的那一套。对我认为整体来说霸权交际在中短期来说是对世界一件好事。



但是从长期来看,如果中国不整顿他的学术界的话,说不定中国,我……这个中国的科技部,我真的是这谈起来就要摇头,你如果再继续纵容学阀到这个地步的话,那中国成为新霸主会是人类的一件不幸之事,只是或者养肥的只是少数的学阀。倒霉的是全世界,不只是中国,全世界的人民。



王孟源 01:35:24 

我常常在再譬如说在那个观察者网留言的时候,就会被人骂,而且是骂得很痛恨,因为我是很高兴看到这些骂我的,就是很明显的是我触到了他们痛处,而且得罪人。那我得罪的到目前为止真正得罪的只有 5 个集团, 5 个都是卖国集团。我不晓得那个,那些人是哪一个卖国集团来的,但是他们绝对是卖国集团的分子,所以他骂我我很高兴,显然是我讲的话对国家有好处。这些卖国者是谁?这是,第一个最大的是台独,但是在观察者网上,大部分那些人不是台湾来的,所以不是台独。大陆的卖国者是哪些人?就是高能所、等离子所,那是做核聚变的,做氢经济的,那也是我常常在讲是骗人的假未来科技,还有就是做那个量子通讯跟量子计算,这 4 个方面,这 4 个卖国集团对我是恨之入骨,所以我看到他们在大陆的论坛对我爆粗口是非常地欣慰。因为我如果对社舆论、对社会没有影响的话,他们就不会来理我。他们愿意对我爆粗口是代表我有影响。



史东 01:36:52 

那我有个问题,你是刚刚提到的这几个category,这几个项目你反对是因为里面的人不对,做事的方法不对,还是你反对就是说。



王孟源 01:37:07 

这个项目就不应该存在,因为他们就是基本上是 20 世、 21 世纪的永动机,他们有技术、物理或者是经济上的理由,是永远不可能有实际效益的。然后他们霸占了——中国的那个科研经费,其实是很紧的——然后他们霸占的绝大多数。



史东 01:37:30 

我知道他们的存在是骗资源,骗政府的资源。



王孟源 01:37:34 

绝对是骗资源的,而且不只是浪费资源,而且是负面的。嗯,这继续,长期来说也是腐化整个整体的学术风气。嗯,因为你这个吃香喝辣的人在中国学术界、科研学术界并不多,结果你吃香喝辣的都是骗子出身,对不对?你记不记得十几年前他们有一个做芯片的叫汉芯,那个 家伙他就买了一些 Texas instrument 的芯片,然后把它上面的标签抹掉,然后重新印上自己的标签,然后假装是自己创新的芯片。那个账户被抓了以后被曝光之后什么事都没有,他骗到的那个公费还是他的,他还是开着豪车上班,还是当所长。中国对学术骗子就是这样的鼓励。那你如果鼓励奖励学术骗子,结果你的学术人头能变成骗子不是理所当然的吗?



史东 01:38:37 

这个真的是一件大事,非常重要的事情。



王孟源 01:38:40 

对,所以现在中国最吃香喝辣,十四五计划里面排名最前面的投资最大的重点都是骗子计划。那你说这不是问题,这是非常大的问题,只不过是他不是一个紧迫的问题。嗯,你这个长期的影响非常非常的恶劣,就是我死之后,全世界的几百亿人可能都还会承受这些恶劣影响。所以我现在给人家怎么骂怎么恨,我也是要讲这一件事情,只要我还活着,我就会每天的,每天有机会就要讲。那你要不要把它登出来?是你的自由。我知道,我刚刚已经讲过了,他们的有些人是官至中央委员,你这个如果做出来的话,对你在大陆的那个的传播是很不利的,我是已经准备要让他们禁了,所以。



史东 01:39:38 

这种事情就是每个人都是,每个人有一个价值观念的取向。就是你不到那个节骨眼,你不知道自己的价值观念取向是什么,往往是这样。





王孟源 01:39:52 

就是去年9月的时候,我左右看一看,觉得习近平把该改革的东西全部都做了,我没有什么好再进言的,唯一剩下的就是这个最重要、最隐性、最难解释的学术问题。所以我在那之前我不愿意被禁,因为我如果被禁的话,我的文章没办法在大陆被转。到了去年9月,我觉得这可以,这个牺牲可以做,其实我在过去 7 年,很多大陆的那个媒体的要来跟我合作,说什么赚钱,因为美国现在大陆流行就是这样,他们习惯付费消费这个所谓的 content 内涵。



王孟源 01:40:43 

我一直都是坚持要做免费,为什么?因为我要坚持有说实话的自由。我知道这些实话我其实放在心里很多年的,但是我要等到合适的时机才能够说出来,因为我知道说出来以后我就 Burnt the bridge, i cannot turn back 就是我的在大陆的正规管道基本上就统统被终结了,都完了、完结了我的没办法再当专栏作者什么的,正因为这样子,绝对不能当网红。当了网红以后你就会受名利所困,不能够说实话,中国 14 亿人有足够的精力能够看清世界局势的人已经不多,那其中又大部分又受名利所困,不能够畅所欲言,像我这样子台裔美籍,不受大陆羁绊,他们的官做得再大也不能够直接拿我怎么样。像我这样的真的是独一无二。如果我在为了在网络上赚那几分钱,把自己的人格给卖出去的话,那中国就没有希望。所以我一直是坚持,就是自愿免费提供消息,就是为了这个自由,为了说实话的自由。很好。



史东 01:42:16 

很好。其实我一面在听你讲,我觉得我们两个的这个诉求有很多很相似的地方。嗯,说先讲这个经济的操作,我做媒体做了大概现在看起来有 40 多年,快 50 年了。嗯,不同的电视、电台什么几乎……除了平面没有做过之外。我一直非常非常的反对,不喜欢,我可以用反对两个字,这个广告模式,媒体之中的广告模式,因为基本上我先年轻的时候,我一直觉得什么地方不对劲,但是我说不出所以然来,等我慢慢长大了之后,我知道为什么广告会如此的有这么强大的影响力和杀伤力。因为你就是你是为付你钱的人说话的,对不对?所以,这句话说,你整个,就不管付你钱的是厂商、还是说是政党、或者是财团、或者是什么东西,付你钱就控制你的嘴,如果你不要希望你的嘴被人家控制,你就不能接受的,所以说我才采用所谓的会员制的制度,每个人……



王孟源 01:43:37 

就是那种广告,其实还有网红这种推销,这大陆很流行。这其实我认为是一种间接隐性的娼妓模式,就是卖的不是肉。



史东 01:43:51 

这个说的是有点露骨了,有点一针见血。



王孟源 01:43:55 

你卖的不是肉,而是你的意见,而是你的……



史东 01:43:58 

其实老实讲,你接受广告的媒体何尝不是这样子?媒体为什么要公司、财团,政党为什么要在你这边放钱?他的目的非常就是很相似,就是跟你讲的很相似,所以说我一直找到一个我觉得很适合我的个性跟我的诉求的一个 business model,就是我小额,但是我就希望能够聚沙成塔,但是我不被任何一个强势,不管这个强势是政治的,是经济的,是什么样的来主导。因为我觉得我这个节目是做给每一个观众或者听众听的,我在我的自己的所谓的 mission statement 下,我关注的是中华民族的前途,我不是拿一个政党或者什么东西前途,我觉得你是完全一样,你不是吗?



王孟源 01:44:50 

我在这边我花了这么多时间,我在这里投入了这是第 8 年的努力,完全就是为了自己的一个理想,因为我觉得,我自己有那个知识,能够真正影响世界人类的福祉,那人生也就是不到百年,你光是汲汲营营地想着自己多赚钱有什么意思?最终这个意义是还是要让人类社会的福祉最大化。



王孟源 01:45:37 

我如果有那个能力,我刚好有那个能力,有那个机遇,有我父母所赐给的聪明智慧来做一些贡献的话,我不去做,我觉得于心有愧。对,我想我们是臭味相投,所以你也是我唯一固定在做的视频管道,那我,尤其是去年我决定要公开的批评中国学术界的那些腐化现象的时候,基本上就是不可能再有正式媒体敢登我的东西,那。你如果愿意的话就可以跟我一起奋斗。我觉得我或许看得比较远一点,但是我绝对不是危言耸听。嗯,这个中国学术界腐化的问题是非常非常严重的隐性娼妓。



史东 01:46:39 

问题,我觉得这个是非常值得所谓的 address 的一件事情,对不对?而且你讲,嗯。



王孟源 01:46:46 

你论证这些东西是永远不可能有有实际效益的东西,这些论证我在博客上都已经详细解释了,而且是多方面解释。比如说核聚变这个东西,它问题不只是一方面的,它有至少三四个方面,随便哪一个都可以证明它永远不可能是有实用,我想是因为一方面中国的这个学术管理体制是完全错误的,赋予学阀太大的权利。另外一个原因是因为他们还是崇洋媚外,就是美国的那个资本市场,美国资本上运作的这个科技投资基本上都是私家投资的,那 私家投资怎么要到钱?你必须要夸大,要欺骗,。



史东 01:47:37 

你要炒作,就是炒作嘛



王孟源 01:47:37 

要炒作。对,这 Elizabeth Holmes 刚刚被判刑,但是你如果去看的话,其实我在硅谷就是 Silicon Valley也曾经打过交道,他们每一个人都是这样子,每个人都是这样子,几十万个那个投资人或者是什么创业家都是这样,他们完全都不知道什么叫做品德,什么叫做实话。这个比如说你现在,现在有一个在做 Solid State Battery 的那个 Quantam Escape,现在已经估价估到几百亿美元了。怎么几十亿?几百亿?我也不晓得,那个印度人开的,我进去看他的那个科技介绍,哈哈,一直看一直摇头。那些硒谷的,那些 second value 的,那些投资人一个比一个笨,他们就是有钱,但有钱是为什么?是美国累积的,美国国力累积的,并不是他们聪明。



王孟源 01:48:40 

所以中国如果崇洋媚外,尤其是科技部那些管理员崇洋媚外,他们这个核聚变为什么这么容易就上去?他如果有严谨的评估论证的话,绝对上不去。这要能够上去只有两个通道,一个是他们迷信科幻小说,OK,在中国最现最近这几年他们的确是很流行科幻小说,另外一个就是迷信美国的那个资本炒作,我一个人都可以简单的逻辑论证出他们这些是永动机级别的假科技。那您一个科技部连这种级别的论证都不做,直接拿着照抄那个科幻小说,跟了美国的资本炒作,然后接受它,然后把国家的科研经费全部浪费在上面,然后这可恶不可恶?而且我说的不是说平常的这个分配,这个多一点,那个多一点,我是说你到十四五计划,这是全国最重要的五年计划里面排名前几名的全都是假科技,你说气不气人?



史东 01:49:53 

其实我们的出发点是一致的,是一样的,我觉得就是一个爱,或者说大爱,这是所谓的大爱。然后我们追求的目标是一个更好的世界,更好的社会,这个我们在此前几次的讨论中,我已经我们已经互相交心过了。对这件事情,但是我觉得还是有一条路可以被你走出来,一条很漂亮的路,所以说不要灰心,有时候真的要不会灰心。对对对。



王孟源 01:50:31 

我跟你说过,他们越骂我越高兴,骂了以后才代表有影响,对不对?对,没影响的话谁来骂?



史东 01:50:40 

对,所以这个事情总是会有一个老实讲,从历史上的种种的故事,我们都可以学到一些经验,做好事的人往往刚开头都是被人家猜忌的,都是被人家骂,但是我这个说起来好像有点像传教了。对,但是我觉得你是一个对自己有非常高的自许跟自信的一个人,所以说我相信在这一方面,这个路一心可以走下去,而且会走出一条路来。



王孟源 01:51:17 

我已经快要没有牵挂了,我的爸爸已经过世了,对,我妈妈由我弟弟照顾,我的儿子已经进了大学了,我有什么牵挂?你就算是把我抓到牢里面关着,我也不怕。



史东 01:51:30 

我,我想不会到内部了。但是我觉得你的勇气是大家都看到的,而且你敢说话的勇气不和人家和稀泥的这种勇气,这需要很大的勇气。不和人家和稀泥你知道吗?这个是可能对你来讲可能是天生的一种状况或一种心态,或者一种能力。但是对一个其他人来讲,或者一般的社会上人士来以不同。这个是怎么讲?合流,不同人家合流是需要很大勇气的,就像有人讲过说实话是需要勇气的。



王孟源 01:52:11 

其实勇气我不知道,我自己还在挑战自己。就是我小时候我家出生,其实没有什么钱,就是贫贱不能移,我已经试过了。富贵不能淫,我也试过了,就是我在华尔街干了快 20 年,但是 威武不能屈,我还没有试过,所以我想试试看。



史东 01:52:35 

非常非常好。我们今天就在威武不能屈的这个词句之下结束今天的访问。我希望以后我也带我们的观众来跟您请求,希望您三不五时地上节目和我们聊聊,和我们谈谈。



王孟源 01:52:53 

那是很大的荣幸。我很高兴。



\twocolumn[\begin{@twocolumnfalse}
\section{俄乌战争}
\subsection{20220322}
\end{@twocolumnfalse}]Credit: anonymous



史东 00:25 

各位朋友,你好,我是史东。今天的节目中继续跟您谈的是,当然您关注的问题,那就是关于这个乌克兰和这个俄国之间的冲突的问题。我们现在这一个两个礼拜多以来,事情的发展可以说是各种方面,各种角度,我们都可以看得到,也都可以谈得到。在今天节目中我们非常高兴为您请到的是您的好朋友王孟源,王先生来和我们谈谈这个事情,我们是想从他的这个观察之中知道一下他对这个事情关心的角度是什么。讲到这我们就把孟源请入我们的画面之中,孟源,谢谢,欢迎的。这个事情大家都关心,然后当然我第一句话问你,就是说你关心的角度是什么?你关心的哪是哪些事情?



王孟源 01:16 

我从写博客开始一个初衷,就是因为当前的世界媒体或者说国际媒体的话语权是掌握在英美的媒体财团手里,那他们为了维护自己的霸权利益,在过去这一个多世纪里面是有意识地,有组织地去扭曲事实真相来麻醉第三世界被剥削的人民,那我一生最不喜欢的就是谎话,你如果看我的博客的话,基本上就是到处去揭露那谎话。



王孟源 02:01 

所以当前世界最大的谎言的来源,最大最常见最严重的谎言的来源就是盎撒的这一套宣传洗脑体系。所以,在过去的 8 年写博客一直都是要跟他们搏斗,那我们过去一个月前,我们面临的是这个世界,也是这个英美霸权的一个转捩点,就是他们第一次在冷战后,冷战结束后 30 多年,面临一个悬崖式的崩溃的可能,那给向往自由和真相的世界人民一个机会,能够打破他们的桎梏。



王孟源 02:53 

所以我今天也是继续跟大家讨论这背后的事实真相,我也不是孤单的,这个当前的国际媒体上有很多自愿的,非正式的,就是不在主流媒体里面从业的媒体人,他们都想尽办法传播真相。比如说我在博客上提到的有一个智利的自由记者,就是 freelance reporter,叫做Gonzalo Lira,他现在甚至人就在那个Kharkov,在乌克兰的Kharkov,所以他是亲身报道,然后他真正的去访问他自己认识的乌克兰人,跟侨居在乌克兰的其他人,所以这种第一手的资料是非常的珍贵的。然后你如果到网络上的话,你会找到很多现在那种自愿地去挖掘真相的人,你只要用心,可以看得出这些人是不是有偏见。比如说有一个新加坡的人博客作者,他叫做什么?叫做 defense politics of Asia,他就主动去看俄国跟乌克兰看出来的各种Twitter,跟其他的立即的消息,然后他把它整理出来,就是他尽可能的做客观的去整理。像这种讯息就可以让你即使我自己不完全不懂俄文,我也可以得到很快的几个小时后马上就得到立即的消息。所以像这种事情,只要是关心时事的人,没有借口说相信英美主流媒体的那一套。



今天是3月 21 日,那今天早上的时候,早上美东时间的早上,俄国用一个重磅炸弹,我估计大概是一吨级的炸弹,基辅城内大的购物中心,一个 Mall 给炸平了,当然第一个反应就是这是残杀平民,你为什么要这样做?然后接下来 BBC 就把它放在头条新闻,然后整天都是头条新闻,然后其他的那个英美媒体也跟进。可是事实上在他炸了之后。俄国国防部马上就公布了视频,这个视频是他们的无人机拍的,大概有 3 分钟长,你只要愿意去俄国国防部看,就可以找到这个视频。这个视频是显示什么呢?它显示一个连的,乌克兰的火箭炮兵从基辅向俄军阵地发射了火箭弹,这种火箭弹连射是,大概一个车大概有 30- 40 枚火箭弹,那他有 4- 6 辆车。噢,那我那时候我自己可看到有 3 辆车了。那但是你想它是一个连,所以不止 3 辆。它发射完了之后,它那个阵地是在公寓的一个公寓旁边的一个空小空地,光从这一点你就知道是乌克兰有意的把他的部队部署在住宅区来避免被俄军反击,然后发射完以后立刻就转移阵地,从那个阵地开上公路,绕了几分钟以后,他就到了那个 mall 的那个购物中心的停车场,然后从后面的那个的货物进出口那边进去了。整个连就进去了,那几辆火箭炮车就这样进去了,很明显的是他们去那里重新装载,也就是这个购物中心是被乌克兰军队主动转化成一个补给中心,那你像这样的话,你什么样的公约都是都会承认这是 legitimate target,就是可以合理打击的军事目标。



王孟源 07:21 

所以你在根据俄军的那个国防部的视频,你可以看到他们进去以后几分钟后那个炸弹就下来了,那这个炸弹的那个爆炸威力,显然是超过一般无人机可以携带的那个的重导弹的重量,这一定是用那个攻击机,像是苏 34 这样的攻击机才能够带得起一吨重的炸弹。那他然后用那个由无人机来指引用,可能是用激光(或者我们台湾的什么镭射)来指引,然后就这样一旦就把它炸平了,炸平之后,你说这种信息是马上就有的,但是 BBC 他们就假装没看见,一整天他们的头条新闻就是俄军炸平了基辅的一个购物中心,像这样的,这其实是我一直说这个英美的国际霸权是它的套路,最简单来说就是,嗯,一方面是心狠手辣,对殖民地打打杀杀,另一方面是厚颜无耻,他们这个宣传洗脑是完全就是说谎不脸红。那其实我们你如果去注意看英美政这种政治体系,而且他把这个体系,这种政治体系推广到欧盟,而且日本跟台湾这种所谓的西方民选制的体系里面,他们选出来的领袖都是怎样的人?在美国一般都是职业政客,从议员这样做起来,或者是律师转行的。在英国的话是往往是先做过记者,比如说前任的大臣 Osborn 现在是做编辑,那现任的总理首相Johnson,他是从记者做起的。然后你想想看哈最近比较有名的,Zelensky,他本人是个演员, Trump, Reagan 也是一个演员。 Trump 是一个Reality Show 的演员



史东 09:54 

可以说是演员。



王孟源 09:55 

也是一个娱乐人员。对,那其他的还有什么呢?他们的幕僚都是一些做公关的,或者也是职业幕僚。然后还有一大堆那个宗教性的团队,比如说美国国会就有好几个宗教性的团队。然后他们这个所谓的 Liberal democracy,其实我把它叫做白左教,它本身就是一个宗教性质的东西。



史东 10:26 

也可以算是一种,这个叫什么基本教义派。



王孟源 10:30 

对,所以你把这些人统统加起来,职业政客、律师、记者、公关演员,或者是主持人,或者是宗教性的人物,这些全步一个共通点。这个共通点是什么?他们都是说故事,你从这一点就可以看出它这个整个体系就是建筑在“骗”这个字上面。所以我,你问我说我这一次跟踪这个俄乌战争着眼点是什么?最重要的着眼点就是这里面有多少骗人的谎话,那我可以跟你讲俄国人他这一次的宣传,他是啊,很谨慎,就是基本上跟中共一样,自己确信确认有可靠的消息才会讲,但是不保证说所有的消息都会讲。就是他如果觉得不合适讲的他不讲,但是他讲的你基本上觉得是有事实基础的,而不是凭空捏造的。那但是英美那边就是完全一意地捏造。



史东 11:40 

其实我的想法,孟源,你讲的这个现象我完全同意,而且我有一个解释,就是象现在为什么这样?因为这个媒体的强势跟弱势之间差别太大了,所以他们处于弱势的话,他们一定要步步为营才行。处于强势的话就可以漫天撒谎,因为他们知道,他们撒谎之后,他们所得到的这种后坐力是minimized,是最小的一种后坐力。



王孟源 12:18 

那我想这里最重要的一点是,这不是偶然的,这是他们几百年来的套路。对,你如果去看美国的话,他在他称霸之前,从 1812 年之后他对外发动了上百场的战争,但是没有一场战争是他主动去跟一个同级的强权做军事斗争。比如说你看一战、二战都是欧洲已经打起来了,然后他最后等了几年以后才下去。渔翁斗渔翁得利,在 20 世纪初的时候打美西战争,把, Porto Rigo 跟 Philippine 拿下来,还有一大堆太平洋的岛屿,这个时候也是因为西班牙很弱,所以是所有殖民帝国里面最弱的,所以他才挑软的柿子来吃。而且如果对历史有点记忆的人就知道,他们那一次也是利用一个所谓的 false flag,就是找假借口栽赃才搭起来的。



王孟源 13:31 

栽赃是怎样的?他们那个美国有一艘战斗舰叫做缅因号,他到加勒比海的西班牙所属的港口去访问,结果在港口里面莫名其妙的就爆炸了。这个你去访问的时候出了事,第一件事应该看看是不是意外,那美国人不管,因为刚好这是他觉得是一个很好的机会去把西班牙的海外殖民地吃下来,他马上就咬定是西班牙人破坏的,这是美西战争的开始。所以你后来到2003 年说伊拉克有 WMD,那个根本就是 200 年来的老计俩,你必须要对历史完全健忘才会觉得奇怪。现在,他们……事实上我对普鼎这次出手进入乌克兰本身不是很赞成,我认为他太急了一点,他在外交上应该还可以再花两个月,至少两个月来做折充,因为我看不出有什么非要在2月底就出手的理由。就是你所谓的那个生物武器什么的,这个都没有及到说。



史东 14:50

但是生物武器这个等一下,如果有时间我们可以谈了,我个人不觉得生物武器是它发动战争的。怎么讲?最后一根稻草。我觉得我跟你这私下提过,因为我从不同的方向,而且这些讲话的人对我来讲是值得相信的。有 RT的评论员,还有一个美国的,这叫做MacGregor,是不是那个退伍军人?(王:Doug MacGregor)?对对对,还有一个什么什么?对,还一个什么人?我一下想不起来。就是说他们讲这个事情是因为本来普京并没有要动手,而是在动手之前他接获一个情报,就是说乌克兰的军队准备集结在东部两个省,要开始清洗,要开始,他们用“清洗”,要开始清洗这个东乌克兰的这些人,所以在这个情况下他才决定要出手,因为时不我与,再出手晚的话……这个说法我跟你提过,你并不这十分的相信。



王孟源 16:08 

因为这很明显如果我是 Putin的话我就再多等两天让他们动手,动手以后再狠打。



史东 16:15 

因为是啊,我觉得在理性,像是这样子,但是我觉得你不要忘了普丁他一直认为乌克兰是他自己人,你看这次出手的那种碍手碍脚的那种情况,他就是因为乌克兰是自己人。



王孟源 16:30 

但是,2008 年的时候Georgia也是这样子,也是,但是他也Georgia先动手,对不对?他跑到北京去参与奥运会,因为那样子才名正言顺。你看到他现在,你现在他打仗,为什么?你刚刚讲他的缚手缚脚的就是因为政治性考虑,就是为了要收拢人心,要有正当性,那如果连战术上都愿意自己牵制自己的部队的手脚,让他们以更多的死伤的代价来避免伤害平民。如果是这样子的话,那你当初让他们多开打一两天,你的付出的代价更低。当时的确是有一个起因,就是乌克兰的部队在东乌开始炮击了,就是炮击开始升级了,但是在 2014 年这个冲突开始之后,这个炮击一直是没有间断的,就是有的时候比较强一点,有时候……时强时弱



王孟源 17:46 

所以他那个时候我现在回头去看就是一个月前他为什么会忽然出手?只有两个合理的解释。第一个是2月 15 号,德国的新总理 Scholz 到莫斯科去跟Putin开会,开完会开了两个小时会,会议出来以后,在记者会上面两个人是当面的吵架,就是至少普京是当面的斥责Scholz ,这很明显的是两个人谈的不欢而散。那过去这 8 年不仅之所以能够沉得出气,一个很大的原因是Merkel 跟他有默契。那这个默契体现在两个协定上,一个是北溪 2 号,另外一个是Minsk。诡异。那当然Minsk会议,Minsk 协议的那个条文德国一直没有逼乌克兰去实践,但是我想Putin去问 Merkel 说为什么您让他们签了这个条约,然后又容许他们违约?Merkel 可以跟他说签建好北溪再说,因为美国不会让我把北溪建好,我们要先把北溪建好以后呢,我们德国才有置身事外的那个余裕,就是本钱。对,就是你那个乌克兰,不管打得多么烂,我的天然气还是有管道,可以直接输到德国那去,那我想Putin愿意等 8 年。就是因为 Merkel 跟他有这个默契,而且她的解释也是合理,这样他才愿意等下去。但是一旦 Merkel  退休,事实上这个整个冲突从去年 10 月开始,就是因为 Merkel 要退休了,那个美国人才会叫Zelensky 去升级这个冲突的。然后不仅一开始陈兵边境,我想也是没有准备要打,只是觉得有可能这个是为了保险的一个手段。



王孟源 20:03 

就是,但是我从一开始就说这个态度完全取决于德国,整个事件的结局完全取决于德国。结果就是这个 Scholz 在 2 月 15 号去墨斯哥,他们两个不欢而散的时候,一开始我还不相信,一个当到总理的人会笨成那个样子?所以我想说,也许是有一些小事上没有谈好,但是从事后来看,他事后……当时还在办冬奥对不对?那冬奥的时候大家休会,等到冬奥结束以后,其实也没有什么重要的事,就是只是炮战开始稍微升温点,然后Putin就打了。这是很明显的是,现在回头去看,很明显的是那一次的会议非常的失败,就是他对Scholz完全失望,就是Scholz之鲁钝,他们连谈都谈不到一块去。所以他知道很明显的是,他一旦动手打乌克兰,那北溪就没有意义了,Minsk协议也没有意义了。对,就是他过去这七八年努力的结果,统统要放泡汤。你说要能够下这种决心,你必须要对德国的新总理完全失望才有可能。如果不是完全失望的话,你总是可以希望他回心转意,因为你七八年的努力就这样子放弃了,他的那个北溪的欧盟公司现在全部裁员掉了。



史东 21:38 

对,今天我已经宣布破产了,我看到消息。



王孟源 21:42 

对,就是反正打乌克兰打下来以后有现成的管道,你还需要北溪干什么,对不对?所以他之所以去打,基本上就是因为他知道Scholz完全没有 Merkel 的智慧,就是英美的一个傀儡,知道这样子再谈判下去的话,只会给美方额外的时间来压迫他。那这样子勉强可以说可以动手,但是他动手之后他又犯了,我不能说他错误了,他又做了一个我不太同意的选择,就是他没有选择速战速决。理由当然是很多,比如说他只动员了……乌克兰境内作战的俄军的陆军,只占他们全部作战部队的15\%。这个结果是什么呢?它的总兵力就是陆军的总兵力。正规军的话跟乌克兰全部的作战兵力相比,基本上是 1: 1 的,一般进攻的话,如果要有正常的速度的话必须要 3: 1。当然俄国还派了一些非正规的部队,就是乌东的那两省的一些民兵。嗯,车臣还有那些?对,然后再加上车臣的民兵。所以事实上他们那个现在打的最硬的那些城市战,就巷战,就是叫这些民兵跟车臣的民兵去打。



王孟源 23:24 

正规军不打。哈哈哈,正规军打,正规军搞运动战,只去抓交通要道。。



史东 23:30 

不是他,这不是他们最强的部队。



王孟源 23:32 

这些不是他们最强部队,有一些空降部队,有一些特种部队,但这只是像芝麻一样撒在那个面。嗯,来,然后那个面本身就是普通的二线部队,就是机械化或者摩托化的部队,然后有一个装甲师在北线。当然他就像你刚早先讲的,他打的时候非常的顾忌民众的伤,人民的伤亡,那这是很合理的。你如果看他现在打了 24 天,他的占领的土地是多少,他事实上只攻击了你,如果从北边的基辅到南边的,Odessa 画一条直线的话,OK,刚好可以把乌克兰分成两半,东西两半,他只打东边这一半,西边完全还没有陆军进去。打东边的这一半呢。



史东 24:29 

一半,那条河叫什么船?叫聂伯,什么河?第涅泊河?对。



王孟源 24:35 

第聂伯河。对,但是第聂伯河不是一条直线,它是这样弯的但是他从这个,因为这样子二方从基本上是对东半边的乌克兰三面包围,所以他从东边、北边、南边都推进。那这个它的南北宽大概是 500 公里,然后一个椭圆形的国土,那它基本上是吃了,沿着这个边境占领了 100 公里的纵深,所以我去估计一下,大概是整个乌克兰东半的一半,这个一半国。



王孟源 25:18 

你先把乌克兰切成两半,西半它没有进去,东半的话,它沿着这个边界有一条把它吃下来了,那这吃下来的大约占东乌克兰的一半,那这个速度快不快?相当的快了,尤其你考虑到它的这个兵力基本上就是 1: 1。。



史东 25:40 

对对,我有人分,我有人帮我分析,你看看你同不同意这个分析。他说你不能讲俄军在这次打乌克兰的选择,它的速度太慢,他说你绝对不能这样,他说你拿什么去做比较?你拿伊拉克,美军当初打伊拉克做比较,而且美军的战法跟俄军的战法是完全不同的,俄军的战法就是你刚才讲,我,就我刚刚讲的碍手碍脚,就是你要照顾一些这个老百姓的事情,美军是地毯式的轰炸,然后再进去,所以这是完全两个不同的这个方式。他说即使在美军那个形式形象,就是美军还花了多少 40 多天还是几天才把这个事情摆。



王孟源 26:24 

速度跟美军攻击伊拉克是一样的,但是差别在于美军的陆军进去之前已经先轰炸。



史东 26:32 

已经摆平了,这基本上炸平了吗?。



王孟源 26:35 

已经把所有的城市都炸平了。



史东 26:36 

对,他不在乎老百姓的死活,这件事这是美军不在考虑之内的。对,这就是这个差,这两点差别就我觉得就很大了。那另外我想这样子,我们谈到这儿把这个速度稍微加快一点。你对俄国这次能够达到他们的战略目标的这个机会或者可能性,你是抱着什么样子的乐观或者不乐观的看法?百分之百,那下一个问题就是大概多久?



王孟源 27:11 

这问题就是他不急。嗯,他如果要速战速决的话, 3 个礼拜就可以打完了。所以我在这个战士刚开始的时候,我说大概就是两三个礼拜,看他的运气怎么样,结果后来一看他根本就不急。为什么不急?除了我们刚刚讲的,他非常在乎保护民众,尤其是我刚刚讲他占领的就是东乌克兰那一半,北、东、南三个边缘。但是你想想看,刚好这个东跟南那两条边缘地带是俄籍人占多数,占到70\%,对,所以他在这个那些地方打起来特别的小心。那北边就是基辅跟Kharkov那个地方,说俄语的人也占到40\%,几到50\%,就是他在那边也没办法放开手脚,真正说乌克兰语,而且是信天主教的是都在西乌克兰那边,还没有被打下去,对不对?所以他这个这样小心谨慎的,好像进了瓷器界店一样的,其实是很有道理的,因为是他自己人。



史东 28:25 

其实我跟你讲,孟源,我们在讲这个,我们似乎了解或者尝试着了解普丁对于这个乌克兰的这百姓的这种心情,但是我相信我们绝对没有办法真正的体会到底普丁以及那些俄国人对乌克兰他们的感情是什么。



王孟源 28:48 

我完全同意,因为尤其是沿着他的那个黑海边境的南部的那些,对,都是当初凯萨琳大帝亲手建立的城市。



史东 29:00 

而这种心态,这种心情,我觉得在媒体的报道上是很重要的,可是没有提这个事情,根本不会提。



王孟源 29:08 

对,当然因为西方媒体现在一开始他们先讲俄军作战如何的差劲,死伤如何的惨重,然后现在看着这个纸包不住火了,他们开始要转向了。这个为什么他们一开始的时候要讲那个俄军作战不利?一方面是要鼓励乌克兰的那些民兵拼命死战,你如果跟他们讲实话说你们根本没有希望的话,而且跟他们讲那他老早就投降了。其实。



王孟源 29:43 

这个乌军的士气能够撑到现在一个很大的原因就是俄国没有占领宣传,他连那个通讯都没有打掉。所以这些被包围的乌克兰的军队跟民兵看到的还是都是西方的那些媒体,他们一样还是去用 Facebook 还是用Twitter,他们看到的都是乌军大胜,眼看着要又要歼灭一个斯的那个俄军,他们乌克兰报道的俄国的阵亡数字之离谱的,是 3 万人阵亡, 28000 人阵亡。



王孟源 30:23 

你说这个,如果俄军真的有 28000 人阵亡,那是快要亡国了。



史东 30:27 

跟你讲那个我在之前我访问另外一位上海的学者,他说这个数字,有网友说你有这个数字,你就已经打到莫斯科去了,哈哈哈。



————————————————————————————————————



王孟源 31:12 

这是西方媒体在过去这三四个礼拜拼命地说俄国俄军作战不利的原因之一。第二个就是德国虽然听美国的摆布,对俄国,跟俄国撕破脸而且做制裁,但是有一点他不愿意做,就是他仍然继续要买俄国的能源,因为美国这个的战……美国搞这一个事件的战略考虑就是要把欧盟绑死在他的马车上,而且不是绑在马车上后面当你的护卫,而是绑在你的马车的前面,就是当肉垫了。当肉垫的话呢,就必须要逼他把俄国的能源进口也断掉。那这样子一来,欧盟跟俄国两败俱伤,这是美国的战略要求,而且一旦这样都做过了,等美国要制裁中国的时候,对中国做制裁,欧洲欧盟所损失,所会面临的经济损失是小于这个前例的。那这样子他就可以名正言顺地要求欧盟去做,所以美国才会自己主动地禁止俄国的是原油进口。因为你如果去看的话,美国原本就是空口白话,他自己没有任何的损失。嗯,对,但是经过两三个礼拜这之后看清楚了,这个Scholz虽然笨啊,但是他国内的那些工业巨富跟他讲明白了,你不能够断我们的天然气,你断了天然气供应的话,我们的整个工业会崩溃。结果呢,自己国内的财阀的警告,他还是听得进去的,所以他就没有断了那个俄国的供应,所以美国在过去这 7 天到 10 天,它的这个战略的重点、焦点就是要让德国再进一步。那进一步这有两个条件,第一个你美国要先做示范,所以美国就做了示范,因为美国对那个俄国的原油是有一点进口的,但是这个依赖性比起欧盟连 1/ 10 都不到,所以是一个可以做的牺牲。然后第二个就是所谓的 false flag,就是你说接下来俄国人就应该用上化学武器了,用了化学武器你就可以逼德国人,那什么手段都必须要做出来。那这个是这个伎俩,不是我编的。这个他在叙利亚已经干过了,而在叙利亚的时候,那个时候因为奥巴马不想派兵公然介入,那个时候是 MI6,英国的 MI6安排来的这个假的叙利亚化学武器。



史东 34:09 

白头盔吗?不是这白头盔吗?



王孟源 34:11 

对,安排的假的化学攻击,结果后来这个国际化武组织,就是禁止、管制化武的那个组织,他的那个主席是土耳其人,就是一个北约派去的。那他就篡改他们的那个调查员的报告,调查员说查无实据,他把他改成证据确凿发表了,然后那个调查员在那之后的六七年拼命地想要揭发真相,后来是那个美国的一个网站叫Greyzone把它公开,给那个英语的听众,所以有兴趣的人都可以去看。那个调查员自己出来说我查无实据,我不晓得这个报告是怎么一下子就变成相反的。



史东 35:04 

对,这个就回到你刚刚节目一开始讲的有关于媒体这个事情,我的感叹就是现在我们处于一个时代,在网络的时代有一个现象,大媒体都不可信,小媒体大部分不能信。就从一个读者的角度来讲,这是我们所面对的现象。可是我从一个我这方面的从业者,以及我你刚才讲的Greyzone这些都是自媒体,都是小媒体,因为大媒体他们不会生存的下去,也不准他们生存下去的。这种所谓的我叫他有良心的人,有良知的人。



王孟源 35:44 

因为他们是已经逆淘汰了。



史东 35:45 

对,他就是逆淘汰,讲的就是,所以说我觉得就是你我,自己拍拍自己的肩膀,拍拍你的肩膀。我觉得我们做的这个事情在这个大环境之下是特别有意义的。



王孟源 36:01 

是这样子没有错,对,你这个论述是只有自媒体上才找得到真话,但是这个跟,自媒体都是真话,不一样。



史东 36:10 

这是两回事。



王孟源 36:12 

哈哈哈,自媒体念谎话也还是在播出,所以你去看的话,俄方像是那个RT 它就拼命的辟谣,就是说好 ,BBC今天又有什么谣言? CNN 今天又有什么谣言?MSNBC,今天又有什么谣言? new york time 今天又有什么谣言?结果你去看那个CNN,他也反击。好,我们也来做一个 fact check,但你去看他们fact check的那些俄国的,所谓的俄国的宣传洗脑,都是俄方的自媒体,就是他没办法去反驳俄国国防部,或者……



史东 36:51 

是,就是还是他们自己人呐,就是他们自己人做的。。



王孟源 36:55 

但是问题是西方就是欧美的大部分民众就是这么笨。这个差别,这之中的这个奥秘即使很明显嘛,对不对?嗯,你去拆穿人家的谎言,结果这个谎言都是网络上的谣言,网络上本来就不可靠,你去拆穿网络谣言有什么用?人家国防部讲的话你又不让人家发表,你把整个 RT 都给禁了。



史东 37:20 

说到RT,你现在还可以看到 RT 吗?



王孟源 37:25 

我现在还可以去看 RT,只能够到它网站就是它的那个……



史东 37:28 

对对对,我也是只能到RT网站。



王孟源 37:30 

RT  America 已经没了。



史东 37:32 

对对对,就是它,我那个原来 RTMA是被redirect,不是我自己。是,就是他们大概有些什么运作?就是你到他 RT America 他会 redirect 到 RT.Com 还是到 RT 什么?就是莫斯科的 r t 。



王孟源 37:47 

就是俄国本部还有,对,而且你去的话,因为这个他们接受所谓 Denial of service Attack,所以他你进去的时候还要先等几秒钟,他先确定你是一个真人,才会让你进去。



史东 37:59 

我觉得讲到这儿我想把话题再往前推一步,我想跟您谈谈这个有关于这一次的这个制裁,这个事情,你所观察到的现在制裁这件事情,以及更进一步的这个制裁。过了之后,你刚刚想说希望马上这个战略解释。这个世界我相信是有很多是 permanently damaged,就是永久性的。。



王孟源 38:28 

就是我一开始的时候,一开始的时候提到说,我们这个时代刚刚面临了一个 30 年一次的转变,我指的就是这些制裁,而不是这个战争。因为你看了过去这 30 年里面的战争很多对不对?连二国都已经打过格鲁吉亚了对不对?也在Crimea都打过一次了。



王孟源 38:50 

不过我在我深入讨论这件事之前,我想再补充几点,就是我们前一个话题,就是它这个Putin的这个战术运作,一个很让人吃惊的就是他没有试图做大包围,就是他慢吞吞的就是抢占了一些战略要点以后,他到目前为止吃下来的两个城市都是二线城市,都是只有 40~50 万人。第一个是叫做Kherson,这个科省是距离那个Crimea很近的,所以当时是因为趁虚而入,乌军根本就还没有布好防御,所以他就把它简单的吃下来,那其他的那个城市就是一线跟二线的城市,前十名的城市没有一个真正的打。现在你看看最大的城市是Kiev,第二个城市是Kharkov,然后你还可以说它还有一些中型的城市,像Sumy这些城市,它都是全部都是三面或者四面包围之后就坐下来等。。



史东 39:58 

你说他留一个口吗?



王孟源 40:01 

对,留一个口,他这个,为什么呢?我想一个很大的原因是你如果去攻坚的话,就把这个城市全毁了,而且这个平民的死伤很严重。



史东 40:10 

你说的这个事情跟他所留下的所谓的人道走廊是一回事吗?



王孟源 40:15 

他人道走廊也是在叙利亚已经行之有粘的一个手段,这个是有前例可循的,就是他真正是这样的,让你只要放下武器,即使你是那个战士也可以,也可以坐巴士离开。这原因是什么呢?原因是他的重点,俄国军事的那个思想,一直都是歼灭敌人有生战力,这个战力是指战斗力,还有战斗力是你拿着枪才算有战斗力,那个放下枪放下就没有战斗力。



王孟源 40:53 

哈哈,刚刚有提到,我个人是觉得应该速战速决,速战速决有一个好处,就是在战略运作、作业跟那个战术层次,你都让你的敌人措手不及,来不及反应,就是他们会很被动。那,但是我想是因为Putin跟他的将军们在事先讨论过,为了要保护那些历史古老的俄国城市,还有城市里面的俄语居民,他不得不慢慢地打,所以他真正打到现在唯一一个团团围住然后硬攻的城市就是Mariupol。Mariupol是在亚速海岸,一个港口城市,为什么它会特别对这个城市这样硬包起来打呢?因为它刚好是亚速营的驻地。



史东 41:47 

亚速营这个议题我们现在刚刚才点到,其实在整个的战争的这个所谓的因素里面,是一个很重要的一块因素。



王孟源 41:55 

他对所有其他的城市都是围而不打,战略的重点是要把乌克兰的这个野战军,乌克兰的野战军大概有8万人,全部都在那个乌东前线,就是原本他们是在乌东前线打那两个州就是俄国人。俄军也没有说对这些部队硬打,他真正硬打的只有两个目标,一个是 Donbas Battalion,这个他刚刚宣布已经全部歼灭,另外一个是就是这个亚速营,亚速营比 Donbas Battalion还要大,而且是还要极端,那他刚好驻地就在那个Mariupol,目前原本的估计是大概1万到1万5000 人在那里,结果四面包围起来以后俄国的正规军也没有去打。他是等到那个民兵来了以后,就是后来就是车臣的民兵从东边打,然后那个东乌两省人民兵从北边跟西边打,这样三边的打。打的时候还非常的小心,是先从住宅区打起。为什么要先从住宅区打起?你要让那个这些死硬分子有地方可以退,要等他退到工业区以后你才能够狂轰滥炸。你要是在住宅区就这样子硬攻的话……



史东 43:22 

应该,而且有很多报道,就是这些军队,就是乌克兰的这些军队,我们讲到亚速营这些人,他们都是用老百姓做肉盾的。嗯,这个这也是。



王孟源 43:38 

今天他报道说Mariupol 的那个的剧院,他们号称里面有 4000 人,然后被俄军炸死,这很奇怪,我们不是在谈地震,我们不是谈洪水,我们不是谈台风。你民众躲到一个地面建筑去干什么?打仗的时候,有炮火的时候,你会几百几千人躲进一个剧院吗?对,头脑正常的人都不会嘛。所以要么就是他完全编造的,就是里面根本没有人。OK?里面根本没有平民,都是亚速营自己的人。要么就是平民被亚速营用枪逼着到那里面去当肉盾嘛。你想想看,这个仗打起来的第一天,第二天,基辅的民众自己是躲到哪里去?躲到地下铁去。你躲炮火的躲炮火躲炸弹,是到地下室去躲,不是到国家剧院去躲,你躲到国家剧院干什么?给人家一炮就端了几百个人,这个普通老百姓也没有那么笨,但是欧美的老百姓就是有这么笨,他们讲出来以后全部都相信。



史东 44:54 

这个,这个我个人也对欧美的老百姓,我是对他们抱着一种可怜的心态来看他们,因为实在他们没有一个相对可以听到的另外一种声音,他就他们所听到的声音完全是铺天盖地的这样子过来,他没有任何其他的选择。



王孟源 45:16 

你看现在整个东南亚,整个非洲,整个中东、整个拉丁美洲,没有人相信他们这一套。



史东 45:26 

我知道,这是一个改变,而且我觉得就是迟早就是在欧美这个这些地带,被这个欧美的媒体Bombardment 的这种可怜的这种老百姓没有。我想他们迟早有一天会觉醒,因为只要他们愿意看,而不是说他们看不到。如果就是你我都可以看到的东西,他们也应该可以看得到。



王孟源 45:54 

这种你根本不需要有反面的事实。你光从逻辑就可以看得出它不合理。



史东 46:01 

我一般人,不,我一般人需要反面的事实比较好。



王孟源 46:06 

你要反面的事实。像Gonzalo Lira,他认识一些刚刚从 Mariupol 逃出来的人。他说那些他们的确是逃出生天,因为他们已经饿了好几个礼拜了。那他们之所以饿了好几个礼拜是因为亚速营,只要你想要逃走,逃离城市,他们就开枪把你打死。



史东 46:31 

看到这个消息,另外我还想到一点,我必须在这提出来,就是也是对于这个英美的普通老百姓不利的。就是像你刚刚说的这些我们敬佩的这些自媒体,这些人或者这些频道,他们都随时或者有的人已经被 YouTube 这些平台已经被切掉了,有很多正在被切正切的边缘。



王孟源 47:00 

对,Gonzalo Lira在第一个礼拜开始报道,那时候他刚好人在基辅,他在基辅的那个酒店里面,他要报道的时候,他连线到那个一个主流媒体去,他们开始,然后那个主流媒体就开始介绍了,然后他就举手说不对啊我现在人就在基辅,你讲的这些都不是事实。马上就把他踢出去了,就说你这个人是俄国间谍,哈哈哈,事实上是这样子,这个你只要稍微有点脑筋想想这个动机,俄军为什么要炸那个购物中心?为什么要炸那个剧院?他如果其他的几万个大型建筑都没有炸,为什么要偏偏去炸那两个。



王孟源 47:50 

那然后他们又说俄军把那个想要从通道跑出来的老百姓,无辜百姓这样射杀,他为什么要这样子?你要杀老百姓简单得很,何必要这么麻烦对不对?真正有需要有动机去杀这些想要逃难的老百姓的,是想要把它留下来当肉盾的防御者才会这样做。这种事情你只要稍微设身处地想动机,就会知道是哪一边在撒谎。但是问题是欧美那几亿的人,美国有3亿人口,欧洲有5亿人口,他们就是愿意相信这些很明显的谎话,所以真的是让人摇头。



史东 48:35 

其实这个这一今天我们谈的这些有关于媒体这些事情,对一个从事新闻、从事媒体,从事宣传,从事文化宣传,是都是一些很棒很棒的教材,不管是正面的或者负面的教材是很棒很棒的,就是老百姓会反应到这种地步,或者老百姓无知到这种地步。你知道吗?就是什么样子的情况,他在控制一些什么样子的手段,他周围……



王孟源 49:02 

过去 200 年都是这样子的,他们有 200 年的孕育……



史东 49:05 

这么长时间以来,你想想看这些,不用200年,一个人从小长到大他所知道的就是这些,以现在给他一个另类的一种现实,那还要需要一点时间才能够,我不是说在替他们讲话,这是一个我从事一个媒体人。我知道我所要面对的也只是我要面对的现实,我做出来的东西不一定每个人都相信,都不一定每个人会同意,但是我能够做的,我相信你也是这种态度,我能够做得尽我能力。我曾经在节目里讲已经讲了一句话,我说我不敢说我的节目犯错,但是我敢保证如果我犯错的话,那是一个无心的错误,是一个 innocent mistake。



王孟源 49:52 

而且是把你这个论据所根据的事实跟逻辑讲清楚的。所以你可以回去看我们是在哪里犯错,不是凭空摘下一个结论,然后凡是反对这个结论的人都是间谍,或者是……



史东 50:08 

哈哈哈,甚至不是有特定目的地在说一些话,故意的把这个,现在他们叫什么?叫带风向是吧。在我们那一代,在我那一代,你比我年可能年轻,在我们就叫造谣,就是故意的造谣,对不对?而且有目的的造谣。你再讲,你再讲。



王孟源 50:31 

我再很快的讨论一下他现在的一些战术的特点,好,然后可以谈这个制裁,还有这些战略上的,外交上的折冲。我一开始的时候以为俄军会想办法把他在乌克兰,在东乌的那个重兵集团,我说过有8万人,包括亚速营区全部都切断。那你这个切断的话就是从南北这样子四五百公里把它打下来就行。结果不是,它是这样子慢慢地齐头并进,并没有很快地要把它切断。后来我想一想,觉得因为这不是二战了,现在已经有无人机了,还有卫星的等等的东西,所以在 2014 年的时候,这个乌克兰军队就已经尝试过要打运动战,结果一整个机械化旅连敌人都还没有见到面就被长程火力歼灭在路上。我想大家可以回想一下,在那个 1991 年海湾战争的时候,不是有所谓的死亡公路吗?有一条那个伊拉克的战车要从嗯,科威特撤退的时候,被美军用空中火力给炸死了几万人,就是连敌人都没看到,你就炸死在公路上。所以我后来想一想是这样子了,乌克兰军队学了乖,所以他们这一次都没有试图长途机动,就原地旧地防转为防御。所以如果是这样的话,你就没有必要没有必要真正把它包围起来,因为事实上他也没有准备要跑。



史东 52:16 

但是反过来,所以有一个问题我想提起来跟你请教。反过来俄国的战车队倒反而被这个乌克兰……



王孟源 52:24 

第一个它不能够做火力准备,(史:火力准备?),火力准备就是先轰炸,因为政治原因不能够做火力准备,对,第二个原因是因为他动员的兵力不够,所以都是我看到他们被伏击都是一个排,一个排或顶多一个连,你这样基本上是他们在做穿插的时候,就沿着那个公路高速要抢到一个什么交通路口的时候,在路上被人家伏击了,对不对?那这个这也是没有办法的事。你这在二战的时候,你被伏击的时候,你装甲车被伏击的时候遇到的就是一个反坦克炮,那这个反坦克炮的那个准确度跟威力都有限,但是现在你遇到的是反战车飞弹,哈哈哈,这个飞弹是一打一个准。对所以……



史东 53:18 

而且打了就跑,就你不用等着了。



王孟源 53:20 

就了就跑的时候,你也没办法拿他怎么样,但是他这个战果是被西方媒体无限放大,事实上俄军事实上想穿插的地方还是穿插成功,就是路上当然有些损失。而且是因为不能够做火力准备所以有些损失,但是基本上并不影响他这个胜局。就是他还是要占领的地方,还是占领,要割断你的那个补给,还是割断补给。至于那个大包围嘛,不需要大包围嘛,反正你也不敢动,你动的话我的无人机还有那个空中的轰炸的火力比你还要强很多倍,但是就是它不但没有打断它的水电供应,就是比如说Gonzalo Lira



史东 54:05 

这一点,我觉得很有意义,我不是说很有意思。(王:非常人性化)就是说你在攻击一个国家,你不摧残它的infrastructure,这一点我……



王孟源 54:14 

不但连水电都没有断,就是连那个手机还是照样进行。



王孟源 54:26 

做得很漂亮。比如说他今天才刚刚跟亚速营讲说你们已经被团团包围,被我们围歼了两个礼拜了,那现在的估计是他们已经只剩下一半,就是他们那个城市只剩下一半。然后他们在今天跟他们讲说今天 10 点到 12 点的时候让你们自由离开,结果被他们一口回绝了。一口回绝为什么?原因之一就是因为他们的通讯还是很畅通,所以他们觉得这个俄军显然是没有全面掌控局面。然后他看到的那些宣传都是乌克兰大胜,现在乌克兰自己国防部宣布是战损比是大概什么? 4: 1,就是一个乌克兰人的伤亡对应着 4 个俄国人的伤亡,其实我认为应该是反过来才对,就是实际上俄国的伤亡大概应该是 7000 人左右,就是死亡跟受伤,那 7000 人里面大概有 1/ 5 是死亡、阵亡的,那这里面 1000 多个阵亡的人里面又有很多是民兵。



王孟源 55:37 

所以他的正规军其实死伤很轻。对,整个的战局事先就决定的。那现在我们看到似乎没有很快,实际上是已经很快了,尤其考虑到他没有动用那么多的兵力。然后又没有用火力准备,使在Mariupol这种,他实际上是真正要置亚速营于死地的。对,我相信他。这个是如果你问Putin诚实回答的话,他会说我宁可他们一个都没跑掉,而且在乌克兰这个很方便,因为他们的 Neo Nazi都是有刺青的。对,所以要看你是不是……



史东 56:18 

就是很高调了,他们都是很高调的人。



王孟源 56:20 

对,很高调的对,这个等于是事先已经帮这个俄国人准备好了。要怎么样清除啊?该清除的对象,即使在Mariupol,我觉得他也是很小心,就是尽量减少平民伤亡,然后攻下一栋公寓之后优先把平民疏散。这些我跟您讲这个根据Gonzalo Lira的报道说,这些平民都已经饿了几个礼拜了,就是因为他们连出去买菜都没有办法,你到了街上就会被亚速营射杀,这样一来当然会很慢。你就算即使在Mariupol这种事存心要把对方全部歼灭的,照理说你急着打的话, 3 个礼拜就可以打完,现在都已经快 4 个礼拜了,所以说真的可以。



史东 57:12 

一个打,如果打上了,你存心想不杀人呐或者不伤人呐,这个仗很难打。



王孟源 57:20 

不伤平民。对,花时间。对,即使是你愿意要杀死敌人,但是你不想伤害平民的话也是一样很花时间。所以我想即使在Mariupol大概也还要两三个礼拜才能够打完。打完以后现在俄国有,我想大概有五六个旅在那边把它包围起来,那这五六个旅拿出来以后,就可以从南线再往整个东乌克兰那一半的腹地打,那打下去以后就可以真正把它切成两半,切成两半以后就可以对剩下的那六七万部队做包围。那他至于基辅跟Kharkov这些大城市,他们真的是不会强力攻坚的,因为现这些城市事实上也没有什么战略意义,就是现在那边有很多部队在防守,但都是民兵或者是武装警察这种三线的部队,你歼灭他们没有什么意思,所以我预期他们还是继续的围而不打,就是在未来的两三个礼拜还围而不打。那这样一来你当然会说,他可以一个月内结束的战局,为什么要拖到两三个月?其实从战略上还有一个考虑,就是他如果签合约的话,有两个考虑,第一个是乌克兰由谁来继承,谁来跟他,谁代表乌克兰跟他签合约;第二个是欧盟要不要有国家参与这个合约来背书?我先谈这个欧盟的问题,欧盟的话,很显然的Scholz是个白痴,我相信Putin已经认定Scholz是个白痴,他这个能够交涉的对象就只有Macron,就是法国总统。那Macron现在正在大选,要到4月底才选完,那你这个4月底选完之前,马控显然是不能够违逆那个大众媒体报道,对不对?所以你要跟Macron交涉,只有从4月底开始才能够交涉,所以现在打完,没有什么好急的,对不对?从欧盟那边来看是这样子。在乌克兰的话,其实我在博客上反复地讨论说,我认为Putin这一次出手用军事解决这个问题,这很可能的因素是他过去 8 年实在受够了,事实上不止 8 年,从他上台开始北约一直东扩。他对这个,你可以看得出他对这个英美的这套好话说尽,坏事做绝的这一套真的是受够了。然后他是真的希望能够拯救说俄语的同胞,在乌克兰的同胞,所以他是有点情绪化那但是我相信他在这个用兵的这个如何收尾,他还是会理性地预先地考虑。这个收尾最困难的一点就是谁能够代表乌克兰来跟他谈判。我们先看看他现在他跟乌克兰要求的,事实上就是要求他在宪法里面定 6 条原则,第一个是绝对不加入北约,第二条是不迫害俄国人跟说俄语,第三个是要禁止 Neo Nazi,纳粹右翼,极端右翼。第四个是要demilitarized ,就是要有限度的非均军事化,第五个是要承认Crimea回归俄国,第六个是承认东乌的那两个州独立,两个共和国独立。







——————————————————————————————————————

王孟源 01:01:41 

你到时候如果吗,比如说有台海战事,说不定台北就会有一个小的核子弹爆炸了,然后说是中共投的,那到时候你欧洲人绝对没有、绝对不敢说不。对不对?你即使他们有点矜持的话,你说美国人会不会卑鄙无耻下流到在台北放一个小的核子弹?



王孟源 01:02:15

Hunt Biden 所谈的那些事情远远比他们几年前用来弹劾要罢免Trump的那些罪名还要严重得多,因为这是真正的贪腐,利用外交对来做国际贪腐。然后现在民主党的选情非常的恶劣,基本上大家可以确定他其中选举会损失国会。至少损失众议院,那如果损失了众议院跟参议院两个都损失的话,那么共和党可以很简单地提起罢免案,那民主党又刚好内部的金融系离心离德,对那个建制派的非常不满的话,那这个罢免案就有通过的可能。哈哈哈,美国历史上还没有成功罢免的总统。这个所以我觉得很有意思。



————————————————



(上接1:01:25)



王孟源 01:03:08 

你不论谁签这个都会有很大的政治压力,因为你毕竟是丧权辱国,对不对?是不是?国土。最理想的是就是Zelensky他本人,因为他是正统,他这个他代表是正统,当然Zelensky本人是一个演员,他事实上是一个财阀的傀儡。对,是媒体财法的傀儡。对,所以他没有什么品格可言。对于争取Zelensky 这件事情不是很enthusiastic。但是事实上你也很难找到一个正统性合法性这么强的替代人物。当然他的第二选择,Putin绝对会有备选。他的备选是谁?是他的那个反对党的领袖,叫做Viktor Medvedchuk 。其实乌克兰根本不是什么民主自由的,他们这个Neo Nazi在过去这 8 年强奸绑架折磨,然后杀害几千个俄系的……



史东 01:04:09 

他们用的数字从 13000 到14000,有人说15000 那么多人



王孟源 01:04:12 

对,就是你如果算上他们在那个东乌那种战场环境下所干的事情要上万,他连乌克兰派去跟俄国谈判,就是 3 个礼拜前第一次谈判,谈判完回来,他们的乌克兰自己的代表团的一个成员就被他们抓起来枪毙了。间谍罪名下面,为什么?一定是因为谈判的时候他替俄国人说好话嘛?你如果有任何证据说他是间谍的话,怎么会派他去当谈判团的成员?



史东 01:04:47 

所以这一点我觉得很很有意思。



王孟源 01:04:53 

然后Zelensky也是已经把反对派的人都打的打,关的关,驱逐的都驱逐了,连前总统都曾经到波罗帝海三国里面的一个去躲了一阵子,最近才回去。那这个Viktor 他本身是亲俄派,就是代表那些俄语人民,所以他早就被软禁在家里面。那目前最新的消息是前两天俄军在包围基辅的这个部队里趁机把他营救出来了,所以他大概是一个备用的,如果没办法争取到 Zelensky 的话,可能就由他。



史东 01:05:32 

对,但是我看这个情况,如果能够由 Zelensky 来代表乌克兰的,还是一个最好的选择了。。



王孟源 01:05:41 

但是Zelensky ,事实上你不管你怎么小心, Zelensky  还是可以很简单的到美国去当寓公,对不对?



史东 01:05:48 

这个是后来的事情了,这个事情我说从一个,从这个普丁的对手而言,因为Zelensky……





王孟源 01:05:56 

因为Zelensky是正统的总统,愿意签的话,最理想的情况。



史东 01:05:57 

对对对,我说的就是这个,你再少一个,退而求其次的话,你就显得是好像你特别挑选的一个傀儡来跟你唱唱双簧嘛,对不对?



王孟源 01:06:10 

对对,但如果是Zelensky,你刚好过去这一个月,英美媒体把他捧成那个样子的超级英雄,这都反过来打他们的脸对不对?所以这事实上是最理想的,就是,但是问题是你再怎么理想。也需要Zelensky配合才行,而Zelensky不一定配合,那你要他配合有一个必要条件,就是先把这些 Neo Nazi全部杀光, 否则他留下来连性命都保不住,他怎么可能愿意留下来?



史东 01:06:41 

对,我在想这个亚速营这个事情我可能需要做专门做一集,因为在西方媒体对这个亚速营这次在里面所扮演的角色,以及它历史上所扮演的特意的淡化的。作为你刚刚讲的这个西方的这个老百姓根本很少人知道他过去的恶行,我也是后来才知道,亚速营这个东西,当初在二次世界大战的纳粹的大本营就是在亚速营,建在这个乌克兰,这个地方就是西边。



王孟源 01:07:14 

他就是复活二战时期跟纳粹德国合作的那个。



史东 01:07:19 

队他们做了一些坏事,真的罄竹难书。从你要从历史,从二战一直到今天。而且他现在是乌克兰军队之中的主要的一些成分,组成分。



王孟源 01:07:32 

开始只是一个营,所以才叫做亚速营,现在老早不是一个营,现在老早就是……



史东 01:07:37 

最这些我要点出来,这样的话我们观众会有一个,有概念,就是说不要像西方媒体把亚速营整个的事情把它淡化掉。



王孟源 01:07:47 

而且亚速营的很多的干部在过去 8 年逐步的被调派到正规军里面。对,作为他们的骨干,如果我是Zelensky——这句话的意思就是我是乌克兰现在的总统,而且我还是我,就是我还是关心老百姓的死活的话——其实Putin的这些要求都是很合理的,你唯一可以跟他讨价还价的就是东乌两个共和国的自治,就是你可以要求说不要独立出去。



史东 01:08:24 

不要独立出去,还是我的自治区。



王孟源 01:08:26 

不要独立出去。对啊,那这有两个解决方案,一个是像现在Georgia的那个解方案,就是一中各表嘛,俄国人承认他们独立,但是乌克兰不承认他们独立,这叫做一中各表嘛。另外一个是俄国退让一步,让这两个回到当初 Minsk 协议,就是让他们高度自治。那我觉得这一点是刚刚我谈过。,Putin 6 个条件里面唯一一个俄方有可能退让的,其他都是 non negotiable。这个可以……你现在的在讨价还价都是浪费时间,就是双方都有更多的死伤,而且大部分都是乌克兰人。



王孟源 01:09:12 

我刚刚说过这个俄乌的战损比大概是1: 4,就是乌克兰要大概 4 倍,这还没有包括平民,然后还有经济上的损失。所以你从人道的人道主义的立场来看,如果我是Zelensky 现在就赶快跟他谈判,说你前面 5 条我全部接受,但是第6条我要他们回到当初 Minsk协议的高度自治方案,那这个是有可能让Putin同意的,那同意的话,你如果因为你又送给他这个正统性就是我是Zelensky ,我是正统总统,我们和解,那这样我觉得是有足够的补偿,那Putin是可愿意接受。但是当然Zelensky 只想着自己的前途,那他的前途是必须要讨好他背后的那些真正的势力,那这背后势力就是乌克兰的财主。现在,以前是乌克兰财主,现在是英美的那个建制派,就是 MI6 跟CIA 的那一帮。所以现在的问题是我们根本不能确定怎么发展。我是,我相信是连 Putin他自己也没办法确定事情会怎么发展。他当然是有一套计划,然后最理想的计划如果失败了,他会有备用的手段。就是我刚刚提到那个备用人选都已经准备好了,但是你说他有没有把握成功?当然是没有。那我想这样,这就是我在战术方面可以讨论的,其他的就是国际上战略。



王孟源 01:10:53 

你刚刚讲到制裁的问题,我们现在可以谈。



史东 01:10:56 

可以谈?好,我们谈一下制裁的问题。



王孟源 01:11:01 

有关制裁的问题,我觉得这次整个事件是英美挑起的战略,目的是什么呢?就是 Merkel在过去这十年是很有独立性的,然后尤其是等到Macron当了总统之后,当了法国总统之后,他们两个都是想要主张让欧盟获得一些外交独立权。所以这样一来,这个英美盎萨集团在打击中国跟俄国的时候,就没有欧盟的配合,没有碰欧盟的配合,它就势单力孤。就是即使连像是在贸易争端这种问题上面,都没办法真正的施加足够的压力。你比如说要到 WTO 去订立新的贸易规则的话,如果没有欧盟的背书的话,美国跟盎萨还有日本样子是玩不响的,所以他们经过去这十年就是美国,其实是到 2009 年,就是上一次金融危机之后才注意到他们自己开始衰败,而中国从后面迎头赶上了。他们从 2009 年之后才想到要全力打压中国。



王孟源 01:12:16 

但从那之后一个最讨厌的的碍手碍脚的人就是 Merkel。在过去 13 年来, Merkel其实就是以一人之力挡住了欧盟为美国站队,而且是随着时间离美国的立场越来越远。但事实上你也看到她 2013 年的时候有 Edward Snowden 的那个丑闻,对不对?她被监听,然后到了 2015 年的时候,叙利亚的难民变成难民潮,她也吃了一次大亏,所以在这个过程中越到后来,她对美国——然后等到 Trump 上台,当然他更加受不了——所以过去这 1213 年,其实是Merkel 当了中国的盾牌,所以中国有这十二三年喘息的机会,没有被美国全力打击。但是这一次,这个这一次的乌克兰危机基本上就是 Merkel定了退休之后,定了退休时间之后,我就说这东欧会有危险嘛。这个结果果不其然,就是事情被搞起来以后,等到Putin确认Merkel的继任者者Scholz是个美国的傀儡的时候,他这个就直接掀桌子的话,当然第一个结果就是美国如愿以偿,那个把欧盟的彻底收编,就是你虽然Macron心里,像Macron或者是意大利的Draghi这些人,他虽然明白背后的真相,但是你这个宣传铺天盖地的,整个民意沸腾。就是他们不可能公开违逆这样的民意,所以他们也只好应声附和。那这样一来美国就可以一面强力打压,让欧盟跟俄国同归于尽。



王孟源 01:14:30 

而且就像我刚刚才提过的,就是建立一个前例,就是下一次他找到合适的借口要制裁中国的时候,就可以要求欧盟做出绝对性的配合。因为欧盟跟俄国之间的这个联系要比跟中国的还要强,就是它这个在能源上的依赖是绝对性。对,所以你如果能够对俄国作出全面制裁的话,那你绝对没有借口说不对中国做全面的制裁。而且你说对俄国,它顶多就是指控俄国人用化学攻击你,到时候如果,比如说有台海战事,说不定台北就会有一个小的核子弹爆炸了,然后说是中共投的,那到时候你欧洲人绝对没有、绝对不敢说不。对不对?你即使他们有点矜持的话,你说美国人会不会卑鄙无耻下流到在台北放一个小的核子弹?我觉得,你现在……



史东 01:15:37 

类似的事情以前不是没有做过嘛,所以说这不是很出乎意料的事情。



王孟源 01:15:44 

台湾其实已经有很多人说会把那个核电厂sabotage,可是你 sabotage 核电厂的话,当常林的员工是看得到的,就是有很多人会知道的。但是你如果只是偷运一个小的核子弹进去的话对美国人来说小菜一碟。



王孟源 01:16:06 

这个是他的战略原因。所以当Scholz,当仗打起来,然后Scholz同意对俄国全面制裁的时候,这个美国人真的是乐昏了,他们这个过去 13 年梦寐以求的战略目的达到,但是这就是为什么我还是对他们这些人的战略能力评价很低的原因,就是固然在过去这 13 年争取欧盟是好像是第一要事,但这是其实建立在一些逻辑前提上,这个前提就是你不能够损失更重要的霸权基础。美国更重要的霸权基础是什么?是美元。你不能够为了得到欧洲的欧盟的支持而把美元的霸权给送出去,但你事实上也可以两者兼得,就是你不要做得太过分,你这个Scholz同意为你站队之后。



王孟源 01:17:07 

你就适可而止,给俄国人放一条活路。但是他 Jake Sullivan,就是他们的国安会顾问,就想要赶尽杀绝,就是要第一个就先把那个SWIFT,把俄国从SWIFT那个网路踢出去,SWIFT就是那个国际银行的转账管道,转账网路,这个踢出去的用意就是他不能够再用欧元做交易,也就是没办法再跟欧盟做贸易了。问题是俄国在过去这八年做了长久而深刻的深入的准备,他储备了 6000 多亿的外汇储备,那你即使把它的贸易全部切断,他光是靠这些外汇储备也可以活上好几年,基本上就是你的进出口,尤其是进口的时候需要动用外汇,那他这个 6000 多亿真的是可以用上好几年。那你这样子,这个像这种宣传洗脑的手段就必须要打火趁热,因为这个你要骗的是这些愚民,愚民的热度是很短暂的,就是他的这个注意力像小孩子一样。 attention span 很短,你不能够说让他这样撑几年,所以这是为什么?Jake Sullivan 就说我们把他们这个外汇储备也全部扣下来以后,这个他马上就没办法还债,没办法为进口的那些货物付钱,那这样一来对外贸易完全中断,然后内部的这个金融机构要宣布违约破产,这样一来就有可能可以造成全面经济崩溃,然后这个全面经济崩溃才能够达到他最终的目的,也就是政权更替,要把普京搞下台。



王孟源 01:19:20 

那问题在于啊,你这种扣押主权资产,因为这个外汇储备是一种主权资产,这种事情是完全违背现有国际金融体系的默契的一件事情,因为美国在过去这30年的那个国力消耗得很厉害,基本上都是靠金融手段来赚钱,而且有很多他们手上的资产其实是国外所有的资产,就是日本、中国其他国家所拥有的国际资产,然后把它交给美国的金融机构来经营,他们从这之间赚取暴利。但是你如果这样随便去扣押人家的资产,连主权资产都可以扣押,那你这个私有的资产当然更不用说了。扣押,人家放在你这里就是因为这是所谓的 risk free,没有风险。



史东 01:20:09 

这就是杀鸡取卵。



王孟源 01:20:11 

对,你现在杀鸡取卵之后,这证明他其实是高风险的,你这些资产就会全部跑掉了,跑掉以后刚好现在美元又正在超发了好几年,原本就已经通胀的这个压力。.你还记不记得两年半前我们在 2019 年的秋天我谈那个美联储会继续的量化宽松,会超发,然后最终会像是雪崩一样的这个通胀,现在就是这个雪崩的时候了,雪崩已经快要发生了,他们居然还去把真正的国外的资金驱赶出去。这件事情他们刚刚把这个方案提出来的时候,美联储就强力反对。Jake Sullivan 就是因小失大,因小失大就是不愿意放弃,这个目光太转浅了,这样一来。



史东 01:21:04 

是不是英文叫做 Penny wise pound foolish,是不是这个样?



王孟源 01:21:08 

对,你不能够因为争取欧盟而放弃美元霸权,但事实上你也可以争取欧盟而不放弃美元霸权,但是你必须要放俄国一条生路,他们太贪心了,不愿意。



史东 01:21:24 

这基本上就是再度地显示。是,我们也谈讨论过,就是这些美国现在这些当权的人还有权势的人都是一些三流货色,对不对?这是现在他们显现出来的。



王孟源 01:21:39 

我小孩现在大学里面,他也是说他们这些同学都是很懒,又很笨,他们唯一花时间去钻研的就是怎么包装自己,他们的那个,他们的那些 resume 履历表都是光鲜得不得了,创立什么什么,得到了什么什么东西。但是你实际上认识他们之后就发现他们根本都是炮制出来的,然后他们真正互相学习的就是说空话,嗯,然后买衣服,然后怎么打扮啊?然后怎么样制造图表,嗯,那个他们的那个唯一真正下苦功去学的技巧是怎么做Powerpoint。



史东 01:22:30 

这就是一个——这个话稍微讲偏离一点,但是我觉得这个时候是应该可以讲一下——这就是一个制度上的结果,这个制度的结果就是一个劣币驱逐良币的结果。我觉得这是一个一定会发生的事情,因为这是制度使然,而不是说暂时的一个人或者两个人造成的这种现象。



王孟源 01:22:54 

因为他们的经济,他们原本在二十世纪中期曾经有过很强的国内经济,但是那其实是因为他们打赢了两个,打赢了两次世界大战,然后再加上运气好,有罗斯福这样的一个务实的总统,做了四任,所以留下了很好的基础。但是到了七零年代,八零年代那些财阀在反扑的时候,他们反扑的目的就是要方便掠夺,那这个掠夺的对象是国内国外都有。但是一旦把整个经济,政经体系转化以掠夺为目的的时候,您就不再需要下苦功了吗?掠夺的时候谁胜出是看谁最不要脸,谁最心狠手辣。而不是看你谁最聪明,最能干,最懂事,最尊重事实。



史东 01:23:51 

就像考试一样,如果考试我可以作弊的话,我何必去读书呢?哈哈哈,对不对?,民主制度也是这样子,你如果可以做作弊。



王孟源 01:24:01 

Jake Sullivan 已经是他们所谓出类拔萃,但是他唯一懂的就是这种权势斗争。嗯,他就是对经济跟这个战略没有真的懂。



史东 01:24:12 

所以说我觉得杨洁篪是看不起他。也不是一定没有道理。也是有道理的。我想他杨洁篪是对他的认识可能比我们对他的认识来的更深切一点。



王孟源 01:24:23 

我至少还没有跟他在同一个房间里面打过交道,我相信跟他打过交道的人大概感觉会跟我儿子,对,那个想要当个纽约市政府官员的风险感觉。我刚刚在我的博客上做了一个评论,就是过去这几天有一点两个小新闻,嗯,我觉得是很有蹊跷,就是很值得回味。嗯,第一个是 5 天之前,这个 New York times 纽约时报突然报道,两年前在那个总统大选期间,大约刚好Hunter Biden就是那个Biden的儿子,Biden的那个花花公子儿子,他把他的MacBook,他的那个笔记型电脑丢在一个修理店忘了拿回来了,然后那个修理店老板后来等了一段时间以后把那个内容拿出来看,发现里面有很多色情的东西,但倒也罢了,但是它里面还包括了很多那个电子信件是跟乌克兰的财阀谈怎么样贪污、怎么样利益输送,怎么样安排跟拜登跟老爸见面这件事情。那当时是这个偏共和党系的 New York post,报道出来以后,马上就是民主党系的主流媒体——民主当现在主流媒体占多数,占绝对多数——那他们就马上出来说这个是Putin栽的赃,这个俄国创造的假新闻。这个领头的是谁?就是 Washington post。在过去这两年,基本上所谓的主流媒体,大部分的美国知识界,就是,尤其是偏民主党那一派。谈到这个议题,马上就说这个是俄国人假造的。而且 5 天前……



史东 01:26:22 

而且这个策略很成功。



王孟源 01:26:25 

对,很成功,他们全部都接受,就是这是假的。但是 5 天前这个 New York Times 刊出一个文章证明这是货真价实的,跟俄国人一点关系都没有,就好像当初 2016 年他们也是说那个大选中是 Putin操弄那个选举,事后也是证明其实就是Jake Sullivan ——当时是竞选团队的国安总管,国安首席顾问,他安排的——他而且层层转包最后制造假新闻,这个最后最厉害的当然还是英国人,所以最后是包给英国人去做,后来都查出来了,根本就是跟俄国人什么事都没有,就是完全是民主党跟那个英国人合作搞出来的假新闻。这一次这个也是一样,就是他们自己的丑闻被揭穿以后马上就说是Putin制造的,但现在 5 天前这个 New York Times 把他揭露,这个就一个很大的问题,因为New York Times 本身就是民主党系的一个重镇,而且 Biden本身不只是民主党,而且是民主党建制派出来的一个傀儡总统。你怎么会去拆这个自己建制派的人的台呢?



王孟源 01:27:47 

这个我想了一想只有两个事例,对 New York time 来说要比民主党建制派还要强大,一个是以色列,但是以色列跟这件事情没有什么利害关系,OK?另外一个是华尔街,那华尔街有什么理由?很刚好就是两个礼拜前他们把各国的这个外汇储备给扣了,扣了以后谁最吃亏?就是金融界。我一开始的时候还觉得这只是间接的证据,这个你只能够说很奇怪,纽约时报去揭发这件事情非常奇怪,其中必然有些猫腻。但是是不是华尔街,你可以说逻辑上指出有很大的可能,但是你不能够确定。但是今天我看到华盛顿邮报,就是Washington post 那个当初为 Biden 开脱的那个报纸,他刊出一个社论。这个社论是谁写的?是 Council on Foreign Relations。我以前上你的节目,几年前上节目之后我就说过 Council on Foreign Relations是美国对外宣传洗脑的一个核心机构,一个协调性的机构,他们的这个外宣的那些总编辑很多都是从这里派出去的。这个机构名义上是一个NGO,它内部也有一个,也有一些所谓的经济学者,他的首席经济学家写了这篇社论,题目是什么呢?是说你们担心美元的霸权会被动摇,是杞人忧天,证据是什么呢?根本就还没有动摇嘛,已经发生两个礼拜了,没有什么动摇,这本身就很可笑。你这个,你只要学过大一经济学都知道这种事情至少要一年两年,你制裁俄国才两个礼拜,还没有出现恶果,所以证明永远都不会有后果。这本身这个论述就很明显不是专业性的论述,而是纯粹政治性的洗地。如果是一个随便拿一个人来写这样的文章的话,你还可以把它忽略,但是这是  Council on Foreign Relations,他们的外宣的主管机构,也就是Sullivan所属的那个建制派的外宣核心派一个人来通过 Washington post 来反驳金融财阀的担忧,那你觉得这是什么意思?这是他们在背后真的有这个争议。就是很显然的是,他们这些建制派就是政治跟外交跟国际关系出身的,这一派把美元的霸权威胁了,把这件事情搞砸了,被从金融出身的那些人质疑,私下批评,所以才会派一个人出来这样子洗地。所以我认为的确……



史东 01:30:58 

你觉得这会不会有效?



王孟源 01:31:01 

我认为这个,Wow,这么可笑的否认。



史东 01:31:07 

就是这个,我的意思就是这个,我的意思就是这个。



王孟源 01:31:10 

就是那么低级的否认,华尔街看了只会觉得越来越生气而已。



史东 01:31:16 

所以我的感觉就是这篇,写这篇文章的人也是受命而为,他受命而为对不对?



王孟源 01:31:21 

他受命而为,他就是老板说……



史东 01:31:23 

对,你叫我写我就写吧,哈哈哈。



王孟源 01:31:26 

你也必须写一篇这种文章就写出来,以后不管内容多么差劲,那个华盛顿邮报还是必须要照印嘛,对不对?这件事情很……虽然我不能够确定,就是事实上你也不永远都不可能确定,因为他们永远都不可能对公开真相。但是它是有很严重的后果。这个严重的后果是什么呢?就是 Hunt Biden 所谈的那些事情。远远比他们几年前用来弹劾要罢免Trump的那些罪名还要严重得多,因为这是真正的贪腐,利用外交对来做国际贪腐。然后现在民主党的选情非常的恶劣,基本上大家可以确定他其中选举会损失国会。至少损失众议院那如果损失了众议院跟参议院两个都损失的话,那么共和党可以很简单地提起罢免案,那民主党又刚好内部的金融系离心离得,对那个建制派的非常不满的话,那这个罢免案就有通过的可能。美国历史上还没有成功罢免的总统,所以我觉得很有意思。



————————————————————————————————————————



史东 01:33:00 

有一个问题,谈到这,孟源,我趁这,可能今天我们可以用这个弹劾案,做一个今天我们谈这件事情的一个结束,一个结尾。你觉得到5月,就像你想到5月这个事情结束之后,这个世界会是一个什么样的世界?



王孟源 01:33:17 

人类社会永远都需要无私的为公益奋斗的人,他们也许是少数,你不能够期望每个人都是这样子,而且事实上他们永远都会是少数。但是一个制度的优劣其实就是取决于是否能够容许这些有理想以公益为上的人能够发挥作用嘛。我觉得很奇怪的就是台湾并不欠缺这些在乎公益的人。我从台湾社会长大,至少我的那一代,有理想愿意牺牲的人很多。OK,50年代, 60 年代, 70 年代就是靠这些人建立,一直到 80 年代, 90 年代,靠这样的人争取张忠谋回来建立台积电,因为引进了台积电,台湾最聪明的人才没有去当律师,没有去搞金融,没有去做新闻搞宣传没有。甚至没有去做商,他们都去研究半导体去了,这是真正有用的实体应用。OK?光是这一点,台湾的经济就比美国健康的多。



————————————————————————————————————————



史东 01:34:45 

Joe Biden如果照这个思路走下去,孟源,如果罢免成功的话代表什么意义?



王孟源 01:34:51 

上一次几乎要罢免成功的就是 Nixon 的水门案。那他眼看着就是罢免案要过了。



史东 01:35:00 

所以他先辞职嘛。对对。



王孟源 01:35:02 

那如果是 Biden 真的走到那里,我是说这个提出罢免案的几率,我们说 70\% 好了,嗯,罢免案会通过的几率又是70\%,那这样子七七四十九就是大概有一半的几率会看样子要过。那这时候 Biden 就只好辞职,辞职以后就是 Harris 上台,天,哈哈,我们这个美国的政坛会非常的精彩,我原本预期这个英国的政坛到今年年终的时候Johnson可能要下台,然后会有很精彩,因Johnson下台以后他们就会有一阵子乱。然后如果女王再去世的话,现在他的那个健康,那个苏格兰独立就真的有谱了。



王孟源 01:35:51 

你如果美国也是这样子乱成一团的话,先是罢免案,然后 Harris 当了第一任有色女性总统,又不学无术,哈哈哈,他比George W Bush 还有 Trump 还要更明显的不学无术,这真的是天大的笑话。你看现在这个沙特就是 Saudi 还有UAE,他们都已经不理美国了,这个美国叫他们生产石油,他们那个连电话都不想接了。



史东 01:36:22 

而且他们说这次油价跟我们没有关系,这是油价上涨,跟我们没有关系。对。



王孟源 01:36:31 

就是基本上就是我讲的Sullivan这一次以为同时绑定了欧洲,而且打倒了俄国,但实际上它损失的是整个第三世界,再加上美元霸权,欧盟的分量比不上美元。打倒俄国事实上也没有成功。你如果现在去看那个卢布的汇率的话,就会发现俄国已经成功稳定下来。就是你即使成功打倒俄国,也没办法弥补你损失所有第三世界,所以他们这一次真的是偷鸡不着蚀把米,损失大了。这,但是当然这个前提是中国必须要顺势而为,改变他们过去这个韬光养晦的战略,把彻底把握这个形势。就是人家说危机,危险是一个机会,那事实上我们一直说霸权交替到要到 2030 年甚至 2040 年才能够底定,那现在真的是加速了,这个就在 2025 年之前就会有水落石出,这是怎么一回事。奉劝蔡英文一句话就是靠,如果他还想做到总统任其届满,不是今年底就到纽约当寓公的话,哦,必须要改变美国人她怎么搞怎么闹,她就跟着起哄的那个态度了,因为美国现在已经没有刹车了,就是美国的那一边已经没有刹车了,唯一能够刹车的就是蔡英文自己。这时候你不能够再拼命地踩油门。



史东 01:38:17 

好可怕的一个图样。



王孟源 01:38:20 

当初陈水扁拼命踩油门的时候,是 George Bush 帮他踩刹车,现在美国那边已经没有刹车皮了。



史东 01:38:26 

对,这个我想你,我知道蔡英文不是菲律宾的杜特地,是完全两种不同的人。



王孟源 01:38:34 

要说叫她改变说辞,她这个要在内政,我刚刚说过,这个,现在搞英美这个体制的全部都是骗老百姓的对不对?他们的全职就是骗老百姓,对,他的政治的整个核心任务就是骗老百姓,我没有说他不能够骗老百姓,他要继续骗就继续骗,他的内政要怎么搞都可以;但是,美国人叫他搞事的时候他要先想清楚后果,那个不能够说,美国人说搞,他们就真的冲到最前面,因为泽兰斯基就是这样的,美国那边现在没有刹车了,欧洲欧盟那边也没有刹车了。嗯, Merkel 已经退休了,现在唯一能够刹车的就是,第一线的炮灰能够说,看来好像不太对劲。



史东 01:39:28 

哈哈哈,对,但是第一线她能够被人家用作炮灰也不是没有原因的,就是因为她听话,或者她有什么把柄或者小辫子抓在你在手上吗?威胁利诱是吧?也不过这两种对不对?所以我对蔡英文如果会做任何的改变,我会,我完全没有这个感觉,没有这个希望,但是我还是希望我是错的,你是对的,你知道吗?我希望参与们能够做到。那对我来讲,我觉得对中华民族来讲是一个很大的一个……



王孟源 01:40:06 

找借口搪塞不作为是他们的核心技能,哈哈哈,这个你跟美国人搪塞一下,就是你做做姿态应和一下,但是……



史东 01:40:20 

我要这样的到这个节骨眼上,因为过去我们也看到美国是如何对付第三世界的一些领袖或者一些首领的,就是我刚才讲的威胁理由一定都是小辫子抓在手上,不知道什么小辫子抓在手上。然后在现在我从蔡英文的这个行为趋势看起来,我绝对相信她有小辫子抓被人家抓在手上,而且不只是他一个人,她周围的人都有这个美国的做情报也不是那么也不是什么,那句话怎么说不是不是吃素的嘛。他们也不是一天一夜到达这种,控制第三世界首领的手腕他们多得很呐,对不对?



王孟源 01:41:06 

民进党在台湾的民意支持之强,你随便出个什么丑闻没有什么用,他不怕出丑。



史东 01:41:18 

我觉得民进党他在台湾民意之强有很多也是被媒体操作出来,就是另外一种形状的一种媒体上的操作了,然后也是一方面的信息排山倒海的过来,然后让老百姓你不知所措,或者不知道该怎么办,或者是故意的,他们故意不愿意去,或者视而不见或者闻而不听。我不知道是怎么一回事了,但这基本上他的结果就是像蔡英文和她周围的一圈人,人就是完全他们的出发点是从美国的利益来行动,而不是从台湾的利益来行动。基本上你讲的就这个事嘛。



王孟源 01:42:05 

哦,我指的不是叫他们发以台湾的利益为上,那是不可能的,我说至少以她自己的利益为上。因为你坐直升机撤离的时候有可能被打下来,你必须要先想清楚这些。



史东 01:42:19 

我觉得可能。



王孟源 01:42:20 

Zelensky现在活得很高兴,活蹦乱跳,在基辅这样跳来跳去,这是俄军故意不去打他。如果真正要跑的时候那就是另一回事。



史东 01:42:32 

以现在的科技,现在你要从这种场可以逃跑的人,那被人家就是瓮中捉鳖。以现在的科技来讲怎么跑得掉?



王孟源 01:42:41 

Zelensky周边的那些人里面一定有俄国的间谍,不一定是他的高层助理,但是那个低级的司机什么的一定有,一定会有那个俄国的间谍,所以他如果真的要跑的时候,这个最好是不要做直升机。



史东 01:43:00 

我觉得我的感觉是跑不掉,以现在的 available 的这个 technology 来讲跑不掉的。



王孟源 01:43:08 

而且对这个。



史东 01:43:14 

我相信他自己,我相信我想至少Zelensky 有这一点智慧,他知道他自己跑不掉。



王孟源 01:43:24 

我觉得很难说,就是如果Putin跟他保证说会要让他继续当总统,而且当完总统之后不会追究他。他固然可以信任Putin,但是问题是这些乌克兰内部的新纳粹分子。还有 CIA 都可能会要他的命,放不过他,所以这个是很危险的事情,因为事实上 CIA 还算是客气的,比 CIA 还要狠辣的是MI6。所以这件事情因为我看至少还要 2 个月才能够尘埃落定,因为我稍早的时候提过,就是要等法国大选先结束,然后俄国在战场上也是打的非常的温吞的,不急的。那所以要把 Mariupol拿下来,把亚速营全部歼灭就已经是4月的事情了。然后然后再把部队调动一下,调到去打 Odessa,Odessa其实没有什么战略意义。Dnipro,然后彻底包围基辅,那又是一两礼拜,完全包围之后,人家如果拒不投降,一定不投降了。又是要围上好几个礼拜。这个事情不到5月是不可能结束了。现在看这个样子。



史东 01:44:52 

有一个问题谈到这,孟源,我趁这可能今天我们可以用这个做一个今天我们谈这件事情的一个结束,一个结尾。你觉得到5月,就像你想到5月这个事情结束之后,这个世界会是一个什么样的世界?



王孟源 01:45:10 

这个事件本身。



史东 01:45:13 

这个世界,不是这个事件,这个是,这个是,这是world。



王孟源 01:45:16 

OK,首先Putin已经表明立场,而且我相信他事先就已经想清楚了才会动手,就是他已经要跟欧美做全面脱钩。



史东 01:45:29 

什么?全面脱钩的,嗯,这个我同意。



王孟源 01:45:34 

然后我刚刚提到其他第三世界的主要区域强权,包括土耳其、巴西,更不要提 Saudi 跟了UAE 或者是印尼、印度这些全部加起来没有一个愿意为欧美站队,因为俄国所做的事情是完全合理的对,他们自己都干过。你现在Saudi和AUI还有很多部队,现在利比亚跟在也门打,对不对?他们自己刚刚干过的事情,他们看到欧美对他们对俄国赶尽杀绝的这个态度,他们都颇为心寒,而且对他们的这个宣传洗脑的,这的威力和疯狂的程度也是非常的忌惮。所以我们可以确定的是俄国已经完全脱离欧美的轨道,然后其他所有的第三世界主国家统统想要脱离欧美的轨道,脱离轨道以后你不能一个孤家寡人,你必须进入一个新的轨道,那全世界唯一有足够的重力,能够作为这些国家轨道的核心的,我想大家都知道。我在本月初,3月 1 号,就是这个俄乌战争刚打起来第一个礼拜,我就写了一篇文章,因为我第一时间就看到这一点,这个是我给中方做了一些战略进,就是完全扭转他们过去 40 年来的战略。那所以我在这里又不多赘述,我只想强调这些建议不是我一个礼拜之内就想出来的,是我一辈子慢慢累积。其实都知道这是应该做的事,只是客观条件不允许,因为你中方如果想要跟欧美脱钩,你自己不能一孤家寡人一个脱钩。嗯,你必须要跟第三世界,但是问题是没有一个引子,第三世界没有那个魄力。现在你面临的是由于俄乌战争,然后接下来欧美做那个制裁,做得太过分,所以俄国已经没有选择,必须要另寻出路。



王孟源 01:48:06 

其他的国家也都是在观望,而且是只要有一个出路,他们就愿意走。所以这时候忽然中国另建国际体系这个战略变,不但变成一个可行的选项,而且是必须选走的选项。他这个执行的细节就是我在本月初写的那篇博文里面。那我在这边不再赘述,大家有兴趣可以到我的博客可以去看。



史东 01:48:37 

好基本上就是这个世界对于一种旧的价值已经唾弃了,这个世界需要一个新的价值,而谁能代表这个新的价值?这个他的价值观念一些如何的取得全世界的国家的公信力,以及他的道德能力。



王孟源 01:48:59 

欧美这种靠压榨全世界来吸血、来满足自己累积财富的这种欲望,过去 300 年就是基本上就是殖民主义的。你们说什么希特勒,我去看过希特勒的传记,还有很多深刻的研究评论。希特勒它的核心思想是什么?他想要模仿英国,他最崇拜的就是英国,因为英国在那之前是 200 年的世界霸主,全世界的到处殖民的希特勒完全他建立纳粹德国,他的梦想是什么?就是复制英国的霸权,而问题是英国的霸权是建立在海权之上,而德国以后来者的身份不可能去挑战这个海权,因为他们一战的时候已经试过了,所以他做了一点稍微的修改,他把这个殖民的对象转向东欧,就是苏联,那是他的核心目标。但是他这个策略基本上不是他独创,不是他新创的,他完全就是照抄英国的殖民政策。



王孟源 01:50:17 

他在希特勒在苏联想要做的事情,就是英国在印度做的事情。他在欧洲对犹太人跟吉普赛人所做的事情,就是英国人当年对美国印第安人跟对欧澳洲土著所做的种族灭绝。过去 300 年的欧洲历史可以说是盎萨,就是英国人开创了一条殖民主义的道路,后来又有资本主义跟帝国主义来作为补充,那这它的核心就是,稍早讲过的,好话说尽,坏事做绝。然后欧陆的强权才先后地想要拷贝,一开始其实是法国对不对?一开始是拿破仑想要拷贝的,然后接下来是希特勒想要拷贝的,他们都是后来者,都是学习者,而且都没有成功,成功的是美国人。那现在这个我们已经到了 21 世纪了,而且过去 30 年的全球化造成了先进工业的普及,就是世界上有资格自称为工业国的,现在比 30 年前多了。这个第三世界的势力才是让俄国能够有底气说我可以跟美欧美脱钩的一个历史背景,也是我为什么对中国做出那样的战略建议的一个历史背景。这些都是前所未有的,就是都是我们面临全新的状态,事实上美国的腐化也是在过去四五十年很快的加速的。然后我们又看到现在我刚刚提到,即使在短期的未来,这个英国也面临了分崩离析的可能,那美国会面临很严重的政治经济混乱,就是经济上会有通胀,现在你刚刚这个上个礼拜三他们那个第一次加息,还肯小心的只加了一码,就是 0. 25\%,原本大家认为应该要加 0. 5\%,其实你就算是加了 0. 5\%,对通胀也已经来不及了,因为你现在看看这个,嗯,美国的通胀有多么严重?2月的通胀是 7. 9\%,英国的通胀是多少?5.3\%,就是美国比英国高 2. 6\%。但是现在这个资本市场对英国年底、今年年底的通胀的预期是多少?是 8. 3,就是他预计英国的通胀会从 5. 3 涨到 8. 3,涨3\%。但你如果去问资本市场,美国今年年底的那个通胀会是多少?现在已经是 7. 9 了,它居然预期是 5. 4,你照同样增加 3 的话,它应该是 10. 9,但是它们预期 5. 4 就是只有实际应有的那个数值的一半。为什么?因为英国人没有印那么多钱。



王孟源 01:53:34 

因为美联储刚刚印了5万亿,在过去两年印了5万亿,这5万亿钱全都进了资本市场。进资本市场的意思是什么?就是他把这个像是债券这种预约都炒到非常不合理的高价位。那你如果了解债券的话,就知道它的价位越高就代表它的利率越低,OK,它的利率如果低的话,你就可以——你如果用Naive天真的方法去读的话——就会觉得利率低就代表着 implied 暗含的这个通胀。 implied inflation 是很低,所以事实上不是合理的预期是 5. 4,合理的预期应该是 10. 9,但是这个市场被扭曲了,正因为美国的市场扭曲的比英国还要严重得多,所以他连这个读数都读不出来,这在经济上是他们的这个问题。事实上 10. 9 我都觉得还算太客气了。嗯,如果是我来估的话,我会估12\%,就是今年 12 月的, 12 月的通胀会是多少。美联储现在已经摆明了就是他如果加息去弄通胀的话,就会把这个资本市场的泡沫给刺破,如果他加息温吞吞的话,通胀就会继续的严重化,他现在已经表明了就是他宁可让通胀继续恶化,也不会去刺破那个市场,这是在美国政治体制下的必然。其实因为他们必然是短视近利,因为你刺破这个泡沫的话,这是一个短期的灾难,而通胀是可以拖下去的,你让他们选择要马上面临后果,还是要 kick the can down the road。对,就是拖下去的话,他们绝对会选择 kick the can down the road。



史东 01:55:43 

这是然后这也是又回到制度,这是他们制度标定的一种方式。



王孟源 01:55:48 

他们从来就没有什么制度优势。这个完全就是因为他们的殖民主义压榨全世界的所有财富,帮助他们进入了工业革命以后领先全球,所以他们一直都是第三世界的财富,高了一个、两个三个数量级,你有了那么多的财富,你就可以高薪养廉,那才是他们的那个廉洁高效的政府的假象的来源。你如果那个经济发展程度是刚果那样子,你怎么可能高薪养廉呢?然后你这你经济已经局势是这么糟糕了,英国要分崩离析,然后期中选举之后可能又有一个新的罢免案,你说这么热闹,这其实中国根本就不需要想办法打击美国,所以事实上不是中国在打击美国,美国也不是怕中国的打击,美国是自己把自己搞死了,然后怕别人来占它的位置。



史东 01:56:52 

你相不相信命运?



王孟源 01:56:54 

我不相信命运,我相信理性,理性的预期,因为你不能够说他们作恶多端。他们作恶多端不是现在开始。



史东 01:57:06 

不是,我的意思就是说你在看你,今天我们谈这个大局势,这个走向,这些东西,我一直觉得冥冥中是就是他的这种身不由己,我只能用命运两个字来诠释,懂你意思吗?并不是说他希望这样,他也知道这样子做不对,他也知道这么后会有这种结果,但是人基本……



王孟源 01:57:33 

这是恶贯满盈。自作自受恶贯满盈



史东 01:57:34 

一而再再而三地就会有出现一些情况,一些态势,把他往那个方向推,包括他自己的智商也是在里面,也是这种态势的元因素之一,对不对?这个就是我所谓的命运。这就是说 out of your control,就是完全,你只能说。



王孟源 01:57:53 

你不要忽略,这里面中国从远远落后,不到 100 年前远远落后,一直到现在迎头赶上,这之间有多少人努力?而且很多这些努力的人是全凭理想的无私努力。这个我觉得英美现在这个现代体制跟他的表现思想体系最大的问题就是为了为资本掠夺正名,而把人性说得太坏了。他们认为自私,Greed is Good 这个是,这是 1980 年代他们思想腐化的一个核心。你事实上人类社会永远都需要无私的为公益奋斗的人,他们也许是少数,你不能够期望每个人都是这样子,而且事实上他们永远都会是少。但是一个制度的优劣其实就是取决于是否能够容许这些有理想、以公益为上的人能够发挥作用,因为他们是少数,你一个自由竞争的市场体制下,一定是有钱人胜出,然后有钱人再去操弄多数。



史东 01:59:21 

其实我这个话我们又讲回来,在我们节目开始讲到的这个所谓的媒体,现在的这些自媒体,这些媒体人,你刚刚就是就符合你刚才讲,就这就是有一些人为了公益、为了自己的良心去从事一些事情,就是我们所尊敬的那些人。。



王孟源 01:59:40 

我觉得很奇怪的就是台湾并不欠缺这些在乎公益的人,我从台湾社会长大,至少我的那一代有理想,愿意牺牲的人很多,OK?你在50年代、 60 年代、 70 年代就是靠这些人见面,一直到 80 年代、 90 年代靠这样的人,争取张忠谋回来建立台积点,因为引进了台积点,台湾最聪明的人才没有去当律师没有,去搞金融、没有去做新闻、搞宣传,没有,甚至没有去做商,他们都去研究半导体去了,这是真正有用的实体应用。OK?光是这一点台湾的经济就比美国健康得多,最聪明的人才全都进了商学院,法学院,还有那个政治学院。这个,哎……



史东 02:00:40 

这些也是一个大环境,这也是大环境。



王孟源 02:00:42 

而且他们的聪明才智不是应用在为世界创造福利。而是应用在怎么样互相欺骗,怎么样在这个掠夺体系里面站到一个高位,一般人有了这么好的基础,怎么会相信美国人的那套说辞,然后照单全收?



史东 02:01:08 

对,我还是那句话,讲到台湾,我不觉得台湾人是相信或者不相信,我觉得现在台湾人,特别是现在的一些当政的人或者掌权的人身不由己,被人家控制住了。这是这当然,这是我的……



王孟源 02:01:25 

我觉得不是诶,我自己的亲戚朋友,我回台湾的时候我吓了一跳;他们都认为这些美国人的这一套说辞是理所当然的,就是天理,他们认为就是那个,对他们民主自由 这几个字,只要拿出这招牌,只要拿出来……



史东 02:01:49 

我觉得你讲的层级跟我讲的层级不一样,你的层级比我的层级高。我在说的,譬如说我从我的视角看出去,我自己是一个国民党教育下长大的人,我当初对国民党所所交给我的事情,所传输给我的事情,我也是照码全收的。但是我有另外一边,最开始让我,我想你也一定想到过,最开始让我想到的是那时我很小的时候,如果你把共产党讲得如此的不堪,那你为什么会被人家赶到台湾来?从这边我就开始去找了,找答案了,慢慢就知道,但是不一定每个人都会有我这么一个契机的。我相信在台湾很多人都有这种想法,只是他们大环境的情况下,他们没有去表达这些事情。



王孟源 02:02:49 

这是你的信仰体系,至少要逻辑自洽了就是这样。



史东 02:02:55 

对对对,我刚刚讲的就是要我跟你讲你的毛病啊,孟源,你太理智了,你太逻辑了,你知道,哈哈哈,就我还跟你另外讲一个小故事,我那个时候刚刚开始在电台而不是电视台。电视台做的时候我做一个 live show, talk show,你打电话来,我那个时候的口号叫做不设防不设限,你谈什么我都跟你谈,什么议题我都跟你谈,然后引就引起了很多人的关注跟这个注意。因此我也在那个小圈圈,在那个地理环境中很受欢迎,很受注意。



史东 02:03:38 

我发觉有一个事情,你不管是从共产党的环境里长大,或者是国民党的环境里长大,你如果万一触及了他们一生,这一辈子——这些都是老先生的六七 八十岁的老先生——他一辈子的价值观念,你跟他反转过来之后,不管你的理由是多么正当,他是没有办法接受的。后来我理解这件事情,这个跟我们讲的可能有点脱了,后来我理解的原因,就是说如果你对一个——特别像老头子或者老太太——你把他的这个一生的价值观再转过来之后,他会发觉他这一生是白活了,他这是他没有办法面对的一件事情,这是我将心比心的听他们讲的说我不知道我讲这个和我们今天讨论有什么任何的关系,但是我只是在尝试着在说有些人的确是没有办法转过来的,或者他不愿意转过来,或者他根本没有办法转过来。



王孟源 02:04:41 

但是知识分子应该有逻辑能力,而英美跟台湾现在这个体制不只是有意去摧毁这些逻辑能力,而且是必须摧毁。对,否则他们本身体制的正当性就被威胁。



史东 02:05:02 

我完全同意,而且他们做得非常的成功,他们做得非常的好。



王孟源 02:05:10 

但是问题是这种虚假的宣传,因为它不是建筑于于事实跟逻辑之上,所以它没有普世性,它他们自己自称是普世的,普世价值的,其实没有普世性,因为就是一种宗教性的口号。对他的这个所谓的民主、自由这些东西事实上是自相矛盾的,很多自相矛盾的东西,所以你要推广的时候,第三世界一开始将信将疑的采纳了,到现在缅甸还有很多人在相信,但是慢慢的大家看事实越看越多,所以美国的这套说辞后来就演化变成专门针对欧美的这些受过大学教育的白左来做。那这样一旦开始专门化之后,他就失去了对第三世界洗脑的能力,也就是奠立了我们现在看到的这个全世界离心离德的这个现象的基础。 20 世纪一个很重要的社会结构转变,而且是全面的结构转变,就是大学教育的普及,你即使是先进国家欧美在 100 年前的时候是能够受大学教育的人口的不到5\%。现在各个只要是,即使新进工业国家,像台湾或者是中国都有 25\% 30\% 的人可以受大专级的教育,你这种成倍增加以后,这些人他们的信仰体系你不能够保证他们就跟你的那个原先的英美的体制保持一致,因为你只有 5\% 的时候,你可以确定这些知识分子他它本身不是财主,就是为财主服务的中产阶级。你如果一旦这个扩展到 25\%- 30\% 或者更高的时候,他们就会有很多人脱离这个上层社会,就是有这个知识能力,但是没有那个收入来配合,这时候你要在满足他们的意识形态跟信仰的需求就比较复杂,所以他们才会发展出新的这一套白左的东西用来麻痹这些人。



史东 02:07:45 

我曾经听有一个人讲过,你不要以为在西方世界上他们希望你有独立思考的能力,恰恰相反,他们最怕的就是你独立思考的能力,呵。呵呵,好,我们谈到这儿,暂时谈到这儿,好不好?



\twocolumn[\begin{@twocolumnfalse}
\section{乌克兰、巴基斯坦、亚元}
\subsection{20220411}
\end{@twocolumnfalse}]Credit: Anonymous



史东 00:09 

有句话,有关于这个 Bucha 的屠杀的事情你是怎么看的?



王孟源 00:13 

基本可以确定是乌方假造的,目前有一些传言说是英国帮助乌克兰做的,但是也有可能是乌克兰。这个我觉得MI6 参与的证据诶,还没有直接证据。但是证明是乌克兰方所假造的,是很明显的。



王孟源 00:36 

就是基础基本教育跟媒体宣传双管齐下。对,然后你养成一群人云亦云,宗教迷信式的相信这些这一整套说辞的人,然后只要在他们成年以后继续用媒体的报道,四面八方的轰炸,就能够满足他们的 cognitive resonance,就是认知的调和,就是自我感觉良好了。



————————————————————————————————————————



史东 01:26 

我们再次为您请到的是你的好朋友,王孟源,王先生来和我们谈一些他最近想谈的事情。我们废话不多说我们就把孟源请入我们的画面之中,孟源再次的说一声谢谢,欢迎。



王孟源 01:44 

很高兴能够再跟大家聊天。。



史东 01:47 

对因为事情是层出不穷,每一天都有每一天这个新发生的事情,这次你主要是想跟我们谈的是什么。



王孟源 01:58 

我想当前的大新闻还是乌克兰的战事,过去这三个礼拜的发展基本上是证实了我们原先的一些推测,澄清了一些不确定的事情,当然也有一个很重要的新发展。这个我都想谈一谈,不过我最今天之所以急着要上节目其实是因为看到一则新闻,就是有关巴基斯坦的新闻,我觉得从这里可以管窥中国政府内部在战略政策制定上的一些不足。我想拿出来翻出来谈一谈。我想我们在我在二月初的时候好像有上节目,一月底二月初的时候上节目,那时候我还信誓旦旦地说这个仗打不起来,当时其实逻辑很简单,就是啊,根据当时的国际战略跟战术形势呢,普京没有理由马上出手.就是虽然当时已经有传言说乌克兰准备在美国的支持下,对东乌的两个共和国做全面的进攻。而且过去八年来,NATO北约对乌克兰做了很多的军援,基本上把他们重建起来,而且战力事实上上比八年前还要强。然后这八年之间他们的那个亚速营还有其他的一些比较小的Nazi组织也都成长到旅的级别,从营到团然后到旅的级别,分派出去大概已经占到主力战斗部队的1/10,而且控制了国内主要的警察机构还有国安机构。所以他们的那个不能说是纪律,但是战斗意识很强,非常的残忍,就真的是像当年的纳粹的SS里面党卫军里面最糟糕的一些部队,当年那个党卫军甚至还有专门用囚犯就是强奸犯跟杀人犯组成的部队,那专门在占领区去干他们的专长。那么原本就没有受过军事训练,他没有军人的职业道德,更加是手段是用得极为狠毒,完全没有节制。



王孟源 04:57

不论如何我在两个半月前上节目的时候我的判断是战争不会马上打起来,原因是即使乌克兰急着出手,俄国也有一个比较好的应对政策,就是用长程炮兵来摧毁汤的桥梁铁路公路油库这些后勤措施,然后呢用火箭炮还有长程重炮来摧毁他们的野战攻击部队,至少这样子先打上个四五天啊先削弱他们的锋芒,然后呢你再反击,那么名正言顺,而且可以掌握先手,就是你因为对方已经摊开来那个阵势,是不适合他们防御的。所以你可以从三面包夹的局势,选择最有利的地点来来做攻击,我是根据这个逻辑来做的,而且从大战略来看,英美跟欧洲都即将面临也很严重的通货膨胀危机,普京只要再多等一两年,基本上这些西方国家自己就会乱成一团,所以实在是没有理由急着出手,但是呢普京还是出手。那我们是后来检讨他的确是犯了一个错误,第一个他太急;然后第二个他在出手的时候那个作战计划太过乐观,这一点我在上个月上节目的时候也谈到了,就是他没有遵从正确的军事原则,作出速战速决的部署。然后用派兵又不够,他前后只派了十几万部队进入乌克兰,乌克兰光当时的现役军人——就是二月底的时候还没有全民动员的时候——他的全部现役军人就有26万,在东乌前线的精锐部队有8万,你用19万的部队去进攻而且他的二线三线部队可以依据城市打巷战,那一线的野战部队在东乌前线已经建筑了八年的坚固工事,所以这些都不是很简单就可以攻坚下来的,不是打的运动战;结果俄军进去的时候还是运动战的原则,基本上是轻装突进没有没有做好先期的轰炸,那我们上一次谈的时候也讲过,这是因为普京想要攻心为上,一方面攻心为上,一方面呢也是要照顾这些俄裔讲俄语的人民不想让他们受苦,不过你要做出这样的决定,一个先决条件就是就是你认为可以不必打硬仗就可以赢得这场战争。



王孟源 07:02 

那么我们现在回头来看,有很多的消息已经透出来,可以确定两点:第一个就是在二月底的时候,普京的确是拿到情报——而且这个情报的确是正确的——就是现在已经证实了,当时的乌克兰军方的确是准备在三月底对东乌进行全面的攻击,普京的确是做所谓的preemptive strike。当然我说我刚刚解释过,就是你即使要做 preemptive strike也有很多不同的做法,那他这个做法是在国际外交上还有战术上都不是最优的选择;第二个是啊现在也已经有消息反复地证实,一开始他们这个作战计划拟定的时候呢,是假设许多乌克兰的地方政府会主动投诚,就是他们只要抢占交通要点然后对城市做出三面或四面包围,那么很多啊比较有亲俄的市长啊或者地方首长就会主动投诚,结果一个都没有。他们真正轻松这样进去的只有Kherson一个城市,而且Kherson连前十大城市都排不上,之所以能够拿下来是运气好,因为当时乌克兰根本没有部队在那里,就是一个空城,就被俄军军趁虚而入。



王孟源 09:54 

那为什么后来乌克兰没有像普京预想这样子有投诚的现象呢?很简单,第一个是长期的因素,就是在冷战结束后三十年来乌克兰的基础教育已经都改写了,把一九三零年代斯大林做农民公社的那一次的改革,其实跟大陆的那个大跃进的很像也是全苏联死了几千万人,其实乌克兰不是死最多的,死最多的,当时死最惨的是哈萨克斯坦,哈萨克斯坦也是一个重要的农业区,那俄罗斯本身也有很多地区跟乌克兰的那个灾害一样的严重,但是呢你在冷战之后因为乌克兰成为一个独立的国家,他们最怕的、威胁他们独立最大的就是当年的苏联和后来的俄国。所以他们就改写了他们的教科书,把那一次的灾难写成专门对乌克兰做种族灭绝的一个企图。那我想这几年熟悉台湾这个教程修改的大概会这个动作是似曾相识。其实啊,在乌克兰早就做了,比台湾还要早。所以你三十年下来很多那个天主教,就乌克兰本身跟俄国最大的那个冲突,就是他的西部,其实是以前波兰的属地,是天主教的,那些是反俄最厉害的地方;然后再有了这个历史上把一个苏联共同的灾难,说成是斯大林专门为了对乌克兰人的种族灭绝的那个残忍事迹,那所以就又种下了历史仇恨的那个心态。所以乌克兰其实对俄国整体来说仇恨的人比较多。就是当然还是有说俄语的那些人,集中在南部跟东部的几个省份。那为什么……



史东 12:15 

对不起,其实我觉得这个普丁,当然我是在一般在猜了,一般也是我的观察,他在乎的也就是东部这一部分。



王孟源 12:28 

对,在1月的时候大家还在议论纷纷会不会打,我那时候我在的我的博客还说这个,当时我当然就说我认为是不会打,因为它不是最优解,我认为是不会打因为他不是最优解;但是我说如果打起来的话呢,他不会占领,不会试图占领整个乌克兰,最极限是把整个东乌跟南乌的俄语区占领下来也基本上就是吃下他的海岸线,包括Odessa,Odessa基本上就是极限了。待会我再会谈这个当前过去这三个礼拜的局势发展,会对这些评估有什么新的改变。





王孟源 13:13 

我刚刚是说到他原本预期会——不能说是他准备老百姓箪食壶浆,但是呢他是没有预期到会有那么坚定的仇恨——然后即使是在俄语区他也没有遇到地方长官主动投诚的,这原因很简单就是我刚刚提到的那个亚速营已经散布出去八年的时间,已经散布出去掌握了所有的国安跟警察机构。所以战事一开始的第一个礼拜,他们立刻枪毙了两个市长——就是最公开亲俄的两个市长——如果是有亲俄历史的早就躲起来了,所以俄军也就没有办法得到任何的当地人的响应。这次为什么他这个整个部署很明显的可以看出来是一副计划落空的样子,就是他急着抢先,没有预料到乌克兰的抵抗意志会这么坚硬,没有预料到地方长官没有一个想要投诚,没有预料到他们乌克兰的步兵愿意做小分队来做伏击,这个是需要很强的仇恨才可以做得到的,因为你即使得手,一个车队里面有十几二十辆装甲车,你得手一辆、两辆、三辆,其他的装甲车对着你这些拿着飞弹的兵来,你也是等于是自杀性的攻击。就是要逃跑的机会并不多。这些这些小分队的兵,你跟俺就是一次十个人,二十个人,一两个班,他们基本上就是自杀性的,你战斗意识要非常的坚定,才能做到。所以我想俄军也是没有想到会遇到这样的情形,然后打着打着,很快地经过两礼拜他们知道苗头不对了,那当然他们的作战计划不是完全一厢情愿的,就是他们也会有最坏的计划,就是如果对方真的出乎意料地抵抗非常得坚韧的话要怎么办。结果,我觉得是他们的计划的The worst case scenario, 最糟糕的可能出现了,所以他们在两个多礼拜前三个礼拜前,做头一个月战事的回顾的时候说,我们基本上在实现第一期计划,后来上个礼拜吧,还是上个礼拜出来说,我们第一期计划完成了。完成了呢,这个攻击基辅的呢,这支部队其实只是佯攻,对你现在完成要对东乌的乌克兰野战兵团完成包围,那我们要专心来把它整个包围歼灭。他们没有说谎,事实上俄国的国防部到现在的一两百个公布的事项,我还没有看到一个他们是有意说谎的,相对的乌克兰……



史东 16:36 

我想插一句,有一个人叫做 Scott Ritter,我想你大概也注意到了。



王孟源 16:42 

他其实 20 年前就很有名,就是当年……



史东 16:44 

我第一次注意他,就是 20 年前的时候我注意到他,结果最近我看到他几次接受这个不同的节目的访问,谈到有关于这个,而是有关于这个战争的事情,就是俄国和乌克兰战争的事情。他提到的事情就是和你刚刚讲到的这个佯攻基辅的事情,后来是普丁的发言人,还是国防部的发言人,他也讲了这个事情,就是有关于他的目的,就是他们的目时候他们就是俄国,俄国的目的就是要把乌克兰的军牵制在这个乌克兰的首都附近,让他不再能增援乌克兰东边的这些部队。



王孟源 17:30 

这绝对是从一开始计划里面就有的,但不是计划的全部。是计划遇到最坏状况下,你这样子部署必须只能够继续,这样继续下去,他们最理想的状态当然是经过的那些城市就不战而降,然后把基辅包围起来,然后Zelensky也投降了,大家都很高兴地回家。我上一次上节目的时候提到他们的 6 点,主要要求其实不止 6 点,但是其中的重点就是我讲的那 6 点,然后基本上一切顺利的话,两三个礼拜他们就可以让乌克兰修宪,然后他们就可以鸣金收兵,大家都没有什么死伤。而事后发生的事就是因为乌克兰亚速营的那个骨干非常的仇视俄国,而且事实上是不择手段,已经到了不择手段的地步。当然这种硬仗也是他们原先计划里面就考虑过的,只是只不过是 worst case scenario,就像你刚刚讲的,那我刚刚讲到是俄国国防部公布了 100 多个宣言,我没有看到一次他们是有意撒谎的,然后我看了乌克兰的宣言,我看了几十个,没有一次看到他们是有意说实话的。我觉得这个对照……



史东 18:55 

这个对比很明显,哈哈哈。



王孟源 18:58 

不论如何了,这个他们现在必须要执行这个战略,就是开战的时候出乎意料必须打硬仗,那么以 19 万打 26 万,再加上后来预备部队动员,你当然是不可能像美军那样摧枯拉朽,尤其他们不愿意用火力,全部火力这样的盲目地攻击。连水电网路都没有断,一直到现在都还没有断。我从一开打的第二天,第三天我就说这很奇怪,我在我博客上讲过。



史东 19:38 

讲到这我就,对不起,我再插一句话,有关于这个 Butcha 的屠杀的事情你是怎么看的?



王孟源 19:44 

基本可以确定是乌方假造的,目前有一些传言说是英国帮助乌克兰做的,但是也有可能是乌克兰。这个我觉得MI6 参与的证据诶,还没有直接证据。但是证明是乌克兰方所假造的是很明显。因为他们完全提不出这个时程时间的流程。他们提了十几个不同的解释,就是,所谓他们就是指 盎萨的媒体像 New York Time,或者是Reuters,或者是 Washington post,或者是CNN,他们提了十几个不同的解释,但是每一个解释说的这些死者的身份,他们是什么死?身份是怎么死的,死在什么时候都不一样。你是指控方,你怎么能够有 11 种不同的指控?答辩方当然有可能说因为是栽赃的,所以我完全 have no idea what you are talking about。我不知道你在胡扯一些什么,所以答辩的时候有可能是十几种不同的答辩。这个要证明的责任是在你,指控方怎么可以东说一句一句。连那个,后来连德国情报局,都闹出来说有什么拦截的无线电信号的证明。后来我去听了一下,没什么嘛,就是他们说了一句气话,就是说这里有这个老百姓对我们不客气,然后里面有一个说操他们的,对不对?嗯,这没有,这完全不算是什么证据。而且时间也对不上,地点也对不上,就是你完全没有办法指明是哪一个俄军部队在什么时候,为什么在那个地点做了这件事情。那我们现在到现在还是众说纷纭,说这些死者到底是真的死者,这些死者是亲俄的民众,OK,还是在俄军占领期间——他们那个镇受到了乌克兰的重炮轰击了好久,已知是有很多民众被打死——所以也有可能这些民众是当初更早这个乌军的炮火所打死的人;也有可能是演员;我们不知道,但是我知道的是指控方的说法漏洞百出。最严重的是3月 30 号俄军就撤离了。3月 31 号当地的市长很高兴的在网络上……



史东 22:34 

那个录影对我来讲很重要。



王孟源 22:38 

对,很高兴地说,我们已经解放了,这一切安好,等待我们国军。



史东 22:45 

就我们胜利了,我们把他们赶跑了,对不对?



王孟源 22:48 

对。结果,这个什么屠杀什么的一句都没提。然后到了4月 2 号,乌军才姗姗来迟,因为他们很怕俄军做跟他们一样做的事——他们做的事就是他们在撤离的时候,因为战争的早期他们是一直往后撤的,都是俄军在前进,所以一直到3月底才有俄军后撤,让乌克兰进军的机会。那他们很怕俄军做跟他们一样的就是留下很多 booby Trap,还有地雷什么的——所以他们小心翼翼的。所以虽然那个 Bucha其实距离Kiev不远,它其实算是一个城郊的一个小镇。对,不是独立的一个地区。但是他们还是姗姗来迟,等了 3 天才来,然后当天还没有做什么,现在已经有证实的报道说一个没有目击到陈尸在街道上的那个地区,有一点距离的,一个人说4月 2 号乌军来了以后到处抢劫,就是他们忙着的事是每家每户地进去搜那个值钱的东西。



史东 24:03 

这个消息我也看到了。



王孟源 24:05 

然后世界第一次听到说有这个什么Bucha屠杀是4月 3 号,所以其实是俄军撤离之后4天了,消息才出来。那这之间,你 Bucha 有水有电有网,而且大家老百姓还是住在那边,然后到了第三天乌克兰部队才到。你怎么可能说怎么可能是说那俄军屠杀了,然后留下几百具尸体,然后这三四天都没有人管,完全不合常理。这你只有完全被西方媒体洗脑,因为西方的教育在过去这 40 年他们的整个社会的衰退,其中的一环就是公共教育的退化。公共教育跟高层高等教育的退化就是学生已经不再学。所谓的 critical thinking,就是逻辑思考,逻辑思辨。事实上发生的,事实上有证据的是乌克兰官方宣称俄国官方杀了乌克兰的平民。你连这种最基本的事实观察到底要怎么样精确描述都做不到的话,你不可能去辨识真假,考虑真假。



王孟源 25:30 

就是说他们的教育的衰退,刚好就是这些资本家在过去 40 多年掠夺国家权力跟财富的过程,那非常的切合他们的需要。所以他们很高兴地鼓励这件事。你想想看美国的这个快乐教育,也就是 40 多年前开始的。



史东 25:53 

对,我刚刚算了一下, 40岁的 1980 年代初开始。



王孟源 25:57 

就是Reagan 上台的时候。



史东 26:02 

对他们。而且我再把这个话题再 expand 一点,你刚刚所提到了就乌克兰的教育,你讲到台湾的教育,我还想到香港的教育,这都不是没有原因的,它的剧本是很相似的,都是有一套同样或者类似的剧本在后面推动这件事情。



王孟源 26:22 

对,就是基础基本教育跟媒体宣传双双管齐下。对,然后你养成一群人云亦云、宗教迷信式的相信这些这一整套说辞的,然后只要在他们成年以后继续用媒体的报道,四面八方的轰炸,就能够满足他们的cognitive resonance 就是认知的调和,就是自我感觉良好了。



——————————————————————————————————————————



王孟源 27:20 

那么这个很重要,因为我刚刚已经论证过了,中国现在重点要争取的就是,你已经有俄国了,俄国已经没有选择,必须要另寻同盟。你现在重点要争取的是中东的那些产油国,那这些中东产油国最担心的,他们最严重的考虑就是能不能,你能不能保护他们的安全?美国要过来跟你搞颜色革命的时候,你能不能保证他们能够活下去?你如果连巴基斯坦——巴基斯坦跟中国的关系是多么的亲密你实在想不到一个更亲密的国家了,而且巴基斯坦也是伊斯兰国家,跟这些国家有很多军事合作,他们都是很熟悉的——结果你连巴基斯坦都保不住你,你怎么好意思去拉人?我如果是外交部派出去的大使,我跟他说你跟我们合作,你用改用人民币或者是怎么用我都不好意思讲,你连巴基斯坦都没办法保护好,你凭什么叫中东产油国站到你这边来?



——————————————————————————————————————————



史东 28:44 

觉得我们再把话题往前跨一步,你觉得今天因为我们上次谈的时候,我们对于这个乌克兰和俄国之间的和谈还是抱乐观的态度,对不对?看起来还是有一点,有一点可能性



王孟源 28:59 

当时我其实就隐隐地有点担心。就是因为俄军当时还是有威胁基辅的态势,所以 ——Zelensky 他当然完全不懂军事,即使他除了演戏以外什么都不懂——所以虽然他身边的军事人员会跟他讲五万的部队根本就不够进攻基辅做巷战,但是他一定还是将信将疑。因为你没有那个专业能力,所以他当时愿意表面上做出局部的退让,其实是因为在威胁下,兵临城下的结果。现在俄军……



史东 29:42 

你觉得Zelensky 有多大的能力越来决定这个谈判的内容跟谈判的方向?而且它还是,它完全是我们所认为的一个完完全全的傀儡,而不是说部分的傀儡。



王孟源 29:59 

他一开始是一个部分的傀儡。我想我上次上节目的时候已经提到他们第一次谈判的代表有一个是银行的,开银行的,那个人后来开完会回来就被枪毙了,他一定是亲俄派的。嗯,所以你说会把那个人派出去,显然决定派谁去,还不是完全由亚速营来决定。那现在我想已经很明显的是由亚速营这些Neo Nazi完全掌控。就是事件,事情越拖越久。就是夜长梦多嘛。就是为什么我刚刚提到Putin的这个作战计划问题很大,就是你一旦出了问题以后就变成消耗战了。那这个反应不只是调兵军事上的反应,也有后勤,还有民心时期宣传,还有我们刚刚谈到的造假、栽赃,这些都是需要时间的。你如果当初 Putin是以雷霆万军之势,以 50 万或者是 60 万的军队,一个礼拜之内就对所有的主要城市完成包围,我想我们现在也许很可能他们已经签了合约,然后俄军已经准备要开始撤离西乌克兰。



史东 31:24 

好,我们话讲到这个点上,然后你帮我们往前看,你觉得这个事情会怎么发展?而且你觉得普丁会怎么样的在处理这个事情?



王孟源 31:37 

我刚刚也提到,暗示了一下,就是过去这三个礼拜最在这个乌克兰这个领土上发生最重要的事情,就是亚速营掌控了,对乌克兰整个政治军事体系的掌控,越加地严密。那他掌控严密之后又刚好遇到盎萨媒体对他们的绝对纵容,就是不管他们编出如何离谱的谎话,总是有那些,总是有西方媒体的绝对配合。比如说在Mariupol,他们有一个妇产科医院被炸了,结果在那个 Reuter 拍了照片,然后有头条新闻说这个俄军轰炸了Mariupol的妇产科医院,然后有一个产妇满面鲜血地跑出来,一个照片,那传遍了世界。我不晓得你记不记得。后来俄军当然是收复了Mariupol,然后找到这个产妇,问她到底这怎么回事,我去看了她的采访,那产妇说当地有三个医院,其中两个被乌军占了,她只好到第三医院去,到第三医院去呢乌军还过来抢他们的食物。然后事发的当天他们只听到两个爆炸声,但是这个爆炸不是很大的爆炸,是炮弹那种爆炸,不是炸弹那种爆炸。我想跟大家解释一下,就是重炮的话,它的那个火药也只有 6 公斤,如果是步兵携行的那种火箭炮的话,它这个就 1 公斤级的,一两公斤级的火药。但是那个重磅炸弹就是比如说当年 1999 年美军一直最喜欢用的,包括 1999 年他们把中国驻南斯拉夫大使馆炸垮的那个炸弹是一吨级的,就是 1000 公斤级的,那你 1000 公斤级的炸弹下来大概一两发就可以把普通的楼房给炸平,那么  Reuter 的报告是这是俄军的空军轰炸,用重磅炸弹轰炸,但是这个产妇说其实爆炸不是很严重就是当然他们可以听得到感觉到那爆炸,但是炮弹级的爆炸或者是更小的爆炸。然后爆炸以后他们到地下室去躲了10分钟,然后大家就说,好,没有爆炸,大家都撤离的时候, reuter 的记者已经在那边等着了,就是他们还没有出来,还在地下室的时候,Reuter 的记者已经进去找。换句话说有两个火箭弹或者是两个炮弹落在这个建筑上面,然后在 5 分钟之内, Reuter 的记者已经开始采访,你可以想想这是什么意思,很明显的就是专门为了那个 Reuter  设计的,所以他们很可能就是用普通的火箭弹,但是虽然是火箭弹的爆炸威力不大,最后还是有一个孕妇死亡,就是她不幸被那个玻璃割到所以流失血过多。



王孟源 35:02 

你想如果是连一吨的重磅炸弹,里面有一两百个医护人员跟孕妇还有病人的话,怎么可能只有一个人,而且是轻伤然后失血死亡的?根本就是你只要去追究一下那个细节就都凑不上。我刚刚要提这件事情,主要是要跟你示范一下,就是这像是那个什么Bucha 屠杀,还有这个 Mariupol的妇产医院。其实这个事件到光是我知道的,在过去这一个月我知道的就至少有二三事件都是很类似,就是无中生有的栽赃,然后他们这些西方媒体所需要的就是一个头条,头条之后他们就不管了,你要再提的话,你就是俄方的宣传人员。



王孟源 35:50 

那正确的态度是什么呢?比如说像 Bucha 屠杀的话,你必须要知道是谁被屠杀,这些人是有名有姓的,那屠杀的人是用什么手法?为了什么?在哪一天?怎么样?这些东西是必须要调查出来的。那我接下来讲,反过来说,正因为这一些 Neo Nazi 可以为所欲为,然后反过来由西方的媒体来替他们擦屁股来遮掩,然后倒打一耙,所以他们越来越没有忌惮,在过去这个月有几个视频露出来他们是如何虐待残杀战俘,这是他们真正自己发出去夸耀。你看我们,我们这个抓住了一些这些俄国禽兽,怎么样折磨他们,怎么样杀?有比如说那个用步枪把他们的膝盖打烂,然后或者甚至把下腹部打烂,然后在那边笑着看他死亡这样子。事后俄国不但识那些受害士兵的身份,甚至临拍摄里的视频里面为首的那个Neo Nazi 都能够提供身份的指证,这样才是叫做证据。因为你要指控一件事情,你必须要主动提供细节,你光是在那边说这件事情发生了,这种事情是只有骗子才会认为是足够的。



史东 37:25 

因为这个也是常理的。撒谎的人不必要提供证据。证明人家撒谎的人才需要提供证据。这个 burden 的 proof 基本上是反过来了,知道吗?



王孟源 37:39 

就傻子才会以为是这样的。但是问题是西方的民众 95\% 是傻子。你看到现在乌克兰,讨论乌克兰这件事情,有一个Doug Macgregor 上校,对,有你刚刚讲的Scott Ritter,然后有很多南美的,Gonzalo Lira 对,有南欧的人 Alexander Mercouris他的朋友也是希腊人。 Alex Christoforou ,然后有一个法国的女记者。



史东 38:13 

那个法国的记者我们的节目里面没有提过,其他人我们这节目的上一次都讲到了,都提到了,就是那个法国女记者我们漏掉了。那个法国女记者很棒。



王孟源 38:23 

对,她也是因为她已经在 Donbas 8 年了。对,她完全理解事实真相。就是乌军已经盲目地用炮火,还有各种各样的手段要残杀,然后迫害当地的亲饿的百姓,她很有个性的,大家有空去看看。



史东 38:50 

她的那个纪录片,我顺便在这里 plug 一下他的这个纪录片。她那个纪录片,我怎么形容它的?这边应该看,但是看起来会很难过,因为在记录里面你们记录的事实是非常让人难过的。但是这个纪录片的价值太高,而且这个纪录片我在网上找到有英文字幕的它,另外我也找到了中文的字幕的。翻这个翻译版就是我跟观众打声招呼,如果你有兴趣让我去找一下就是了。



王孟源 39:23 

要麻烦你把她的姓名列出来,忘了哎。



史东 39:26 

对对对对,我等一下找到列出来。她那个法国姓名我记不太清楚。



王孟源 39:35 

而且这些人很明显都是没有背后,没有财团在资助的。



史东 39:38 

其实我们上一次有良心的人,有良心,是有良心的人,对,这个世界上总是有一些这种少数让人家敬佩的这种有良心。



王孟源 39:49 

的人。这些是做视频的,那除了这些人之外,你写博客的更多,比如说 Moon of Alabama,这个也是写了 20 年的博客。



史东 40:00 

对那个是你介绍给我的。



王孟源 40:03 

Moon of Alabama功力很深厚,这个对真相世界政这个国际事件真相有兴趣的读者有必要去看看。过去这将近两个月,开始的时候,俄国的民意其实是很极端的,就是有极为好战,就是也有极为亲西方的。 Putin其实是比较中间派的,很有意思,不要以为他是一个强硬派,但是俄国人他们的那些学术精英只要还有一点脑子,都已经放弃了他们的所谓的西方的 liberal ideology。已经了解到这些,这么多年来,他们那些国家主义者所说的必须要站起来抵抗西方的侵略,说的都是实话。因为你西方不是讨厌你的恶行,而是就是讨厌你,因为你就不是西方的一员,就是基本上除了这些少数人以外,现在俄国的内部的民情空前的团结。



史东 41:22 

对,我也听到这个这种报道,就是很多原来比较亲美国的那些职业圈,或者是那些文化圈或者那些人物圈,现在都慢慢地回过来,而且很快地转向。这个支持普丁这个评论员说这转向的原因主要的是西方,美国跟西方他们做得太过分。



王孟源 41:47 

我觉得是因为俄国的精英,他们所受的逻辑教育还是要比中国强。你看中国的那些所谓的公知,他们在当前的这些英美的丑陋真相公诸于世之下,他们只是暂时的忍气吞声,并不是真正的悔改。但是在俄国就有全面的悔改,他们的他们相当于公知的那些人现在是真正知道自己当年错了。同时我们在东欧有两个选举结果,就是两个亲俄的候选人都大获全胜,一个是匈牙利的Orban,另外一个是 Serbia 的Vucic。同时美国也利用机会,因为印度跟巴基斯坦都拒绝站到欧美的那一边去,他们拿印度没有办法,那个Modi的印人党是一党大d的根本,而他们一党独大的原因我也在我的博客上深刻讨论过,就是 2 年前,因为他们控制了基础教育,这其实跟土耳其那个 Erdogan 是一模一样,就是当地的公立学校很差劲,所以穷人子弟很少有凭自己的学识能力做 social mobility 往从往上爬的机会。那所以都是宗教性的。在土耳其是伊斯兰,然后在印度是那个印度教极端分子,他们建立了一个私家学校的网路,专门以低价提供优质的教育,但是这个优质的教育是绑定着这个意识形态的,所以经过了二三十年以后,这些训练出来的新的穷人出身的知识分子就变成他们新生的力量。然后他们成为公务员,成为教师,成为警察这些中产阶级的时候也同样地将他们的这些极端意识形态渗透到整个社会里面去。经过二三十年大笔的投资,有一些 billionaire 投资,然后资助才建立出来。所以当年那个马英九,台湾的洗脑却是由公家公费来出,来对新一代洗脑。所以当初我也讨论过说这个这么明显的,让台独捡便宜的事情,马英九却不知道要扭转的,光是这一点,他就是历史罪人,因为你把一整个世代的台湾的新生代的脑子都给腐烂掉了,而且不是事先很难预见的。事前很多人都跟他讲这必须要做,但是他就是不做。在匈牙利跟 Serbia 选举的同一天,那个由美国鼓动反对党对抗他们的总理Khan提出不信任投票。



史东 45:12 

你说巴基斯坦对不对?



王孟源 45:13 

巴基斯坦,对,因为证据确凿,就是很明显的是美方的运作,作为他不支持美国在联合国提议的一个惩罚,所以Khan就由那个议院的副议长了宣布,这是违宪的。因为是作为外国的代理人, 他们这个不信任投票被就被裁定是违宪,然后Khan就去要求总统叫他立刻安排重新大选。重新大选的意思就是因为事实上巴基斯坦的民意对美国是并不友善,你如果现在用大选,然后又刚好有这一条新闻,他们基本上抗可以确定Khan会大获全身。所以一开始大家也以为说事情就到这里为止了。然后结果昨天晚上、今天早上消息传来说,反对党上诉到最高法庭然后,最高法庭裁定支持反对党,就是不能够重新大选,就是完全站到美国那一边。



史东 46:26 

不信任投票的行为通过,并不是说投票的结果已经达成,对不对?



王孟源 46:32 

反对派敢提,就是他们已经掌握足够的国会议员……



史东 46:36 

他们有这个信心。



王孟源 46:38 

对,就是他们已经收买了足够的国会员。那这个钱是哪里来的?大家都可以想象。我看到的时候很生气,怎么会笨成这个样子。为什么会有这个感觉?因为当前俄罗斯跟西方的斗争其实是有两条战线,一个是我们刚刚谈的很多的军事战线,另外一个是经济战线,这是美国典型的打击手段,就是用制裁他们的目标,短期是希望扰乱俄国的进出口,然后通过它的货币,让它的货币无限贬值,来造成内部的经济混乱,然后达到政权颠覆的目标。其实我从 8 年前开始写博客,就已经,不知道描述过多少次,这是他们典型的手段,因为这个手段的代价以小而利益极大,百试不爽。即使有代价,这个代价也不是美国人自己负担,而是有他们的所谓的盟友负担。这次是德国在这个经济战线上这样的出手。我在上次上节目之后也已经解释过这个是因小失大的傻事,因为刚好现在美元已经超发了 20 多年,正在面临极大的通货膨胀压力。你再去扰乱一个世界数一数二的能源供应国的进出口,你基本上是保证这个世界的能源供应进一步地促进通货膨胀的速率。其实上你的节目也已经谈过好几次了。



王孟源 48:19 

这个,在冷战后这 30 年,其实美国的霸权的基础就是美元,他们已经放弃了他们的制造业,他们的军事也开始有腐朽,被中国从后面后来追上的态势,然后至于文化、教育、社会等等商业也都是腐烂的已经快要认不出来的地步。现在靠的就是用无限印钞来解决任何的所有的问题。这样子如果美元被取代的话,就是真正的釜底抽薪,一次性地解决盎萨这个殖民主义帝国对人类过去这 300 年的桎梏。



王孟源 49:06 

人类工业化 300 年来的历史,其实就是盎萨压榨第三世界,同时打击挑战者的一段历史。这期间不知道有多少个挑战者被他们一脚踢下去,然后他们为什么会变成谎言帝国?因为他们坐在霸主的地位上,有那个资源,有那个需要来创造这些谎言,而且有 300 年的时间来把它精雕细琢,传播起,传播出去。虽然他们上一次打败挑战者是只有 30 年前,就是冷战结束那一段时间,他们同时打败了日本跟苏联。其实,这不是 30 年一次的转变,而是 300 年来人累历史的第一次,第一次盎萨面临这么严重的危机,而且他因为他们自己内部的腐朽,这一次是真正的危险。那既然他们的危险的根源是美元,我们这些坚持真相,关心世界人类整体福祉的人,当然是希望能够针对性地打击美国现在的弱点。



王孟源 50:22 

所以我在我的最新的一篇博文,这是 6 个礼拜前写的,特别去提到中国如如何在这个通货膨胀高涨的历史阶段,决定货币最重要是什么?就是大众货商品,尤其是能源,这是为什么?俄国这一次基本上有恃无恐,即使他犯了错误,也可以很简单恢复过来,现在他的经济已经恢复过来了,而且他军事虽然是一开始的部署不对,但是现在反正国内团结了,你继续打下去的话也只会越打越顺。这个长期打下来乌克兰是没有希望。既然如此,这是一个天赐良机,中国可以拉拢俄国,然后双方合作,由中国提供财富跟人员跟制造能力,俄国提供部分的军事,还有他的油气他的能源。这时候的关键是什么?关键就是中东产油国,我们说石油美元,石油美元,这已经是 50 年的事情了。你这时候我们现在正在面临又一次通货膨胀,上一次是美国抓准了通货膨胀的机会,把美元跟全世界最重要的大宗货品绑定。现在我们又面临了一次通货膨胀的危机,中国应该在绑定俄国之后也把中东绑定了。你要绑定中东的话,你必须要给他们特别的甜头,因为毕竟他们国内是有美军驻军的。



王孟源 52:05 

所以我设计了一套国际篮子货币用来吸引他们。但是你光是有胡萝卜不行,你还是要能够解决那个大棒的问题,就是他们当地有美国驻军,而且美国人最喜欢搞颜色革命,这个很重要,因为我们我刚刚已经论证过了,中国现在重点要争取的就是你已经有俄国了,俄国已经没有选择,必须要另寻同盟。你现在重点要争取的是中东的那些产油国,那这些中东产油国最担心的,他们最严重的考虑就是能不能,你能不能保护他们的安全?美国要过来跟你搞颜色革命的时候,你能不能保证他们能够活下去?你如果连巴基斯坦——巴基斯坦跟中国的关系是多么的亲密,你实在想不到一个更亲密的国家了。而且巴基斯坦也是伊斯兰国家,跟这些国家有很多军事合作,他们都是很熟悉的——结果你连巴基斯坦都保不住你,你怎么好意思去拉人?我,我如果是外交部拍出去的大使,我跟他说你跟我们合作,你用改用人民币或者是怎么样,我都不好意思讲,你连巴基斯坦都没办法保护好。



王孟源 53:35 

你凭什么叫中东产油国站到你这边来?我刚刚讲了这么多,说为什么我一看到这个消息,你说那些管国关的,做外交部、智库跟幕僚的人,他们应该是这一行的专家,这种事情应该是他们吃饭睡觉、做梦喝水都在想的事情。这个方法很简单,中国只要两通电话,一通电话打给巴基斯坦的参谋总长,一通电话打给他们的ISI,就是他们的联合情报局的局长。你如果要跟我中国站在一边的话,你不能够让这个总理看就这样子被人家搞下去。你不要说巴基斯坦,就是美国的那些司法官,大家已经刚刚看过孟晚舟的事情搞了3 年,还有人相信司法是独立的吗?搞到国家主权国家安全的事情,司法还可能独立吗?这种事情,巴基斯坦的最高法庭只要接到他们军方来的电话,绝对会乖乖地照他们的需求做出适当的判决。这所以我今天早上听到这个消息,就是很显然的中国没有做这个电话,打这个电话。



王孟源 54:55 

你以前韬光养晦,而且不干涉内政,时间变了,我们现在正在个历史的关键时刻,你还在找借口不作为。而且这不是干涉内政,这是防止美国干涉内政。看看Putin。如果不是去年对 Belarus颜色革命果断出手,今年年初的时候对 Kazakhstan 的颜色革命果断出手,现在有人会管他吗?你想想看,现在 Belarus会让他那个军队进出自如吗?美国已经撕下脸皮就是决心要治中国跟俄国于死地了,你还在坚持?人家美国在你的重要盟帮邻国搞颜色革命,然后你袖手不理。我真的不晓得这是什么样什么级别的乡愿。也许只有那个马英九级别的乡愿才可以相提并论。我今天之所以会想要上你的节目,就是因为胸中拗不下这口气。为什么?我刚刚说过,盎萨这个邪恶帝国已经掌控人类自从工业化以来 300 年的命运,这期间有多少亿人,几十亿人被他们剥血残杀,很多人是真正奉献了他们的生命。我自己不才。你想看我过去这 8 年在在写博客,在 2016 年以后我就有一点名气,而名气以后就有很多大陆的出版商或者是其他的传媒公司要说跟说要跟我合作,我都是一口回绝,为什么?因为我要说实话,而且说的是如何改进中国内政的实话,尤其是要批评那些缺乏诈骗集团的话,我就不能够有盈利的事业。更进一步说我现在坐在人在美国,我在批评英美的这些谎话有没有危险,所以我从来不用英文写在中国发表的刊物十几篇二十几篇,稿费我从来一文都不拿,我只我要说实话,就不能够给他们借口。你光看美国过去这几年关了多少中国人,这根本其实还算是。极小极小的牺牲。



王孟源 57:26 

你看看Gonzalo Lira,今天我去看他最新的报导,他详细讲述了过去这一个月,他已经很侥幸地逃脱了两次,乌克兰秘密警察要抓捕他的企图。就是他两次都刚好不在房间里面,这个是真的有生命危险的,他现在还是躲起来躲在朋友家里面。这些人都是有理想的人,我们,也许我们并没有什么真正的贡献,说让盎萨集团现在有危险,或者是中国或者俄国现在能够兴起来挑战他们,但是负责 14 亿人命运,然后又刚好可以控制又可以影响50、60亿第三世界人民福利福纸的官员,幕僚,智库教授院士尸位素餐。很明显的一些战略思维没有做出来,放任官僚的惯性,眼看着这个大好的机会就在面前,却不愿意动手,那你光是说这是百年未有的变局,有个什么屁用?



王孟源 59:30 

中国内部不作为、懒政或者是鬼混,他们混的那些钱不是要点。但是他如果妨碍国家在这个关键要点做出正确的政策的话,那就是真正人类历史上的罪人。我一个一届布衣平民,我都还拼命地想要解释,把这些事实真相,然后好的建议解释出来,这是他们的职业责任。这么多人没有一个能够做出正确的建议,真的是让我很生气。



史东 01:00:12 

我也不知道该怎么劝你,因为我觉得你不需要劝。我只是觉得很珍惜一个在今天这个世界中,有一个能够真的是那么推心置腹,把心掏出来说话,把心掏出来做事的人,我觉得是越越难得了,越来越少见了。这是我的感触嘛



王孟源 01:00:38 

我本来就完全不是求我个人的名利,im in this for the betterment of mankind,你不认为Gonzalo Lira是做那些视频是为了要讨那些赏钱?不是的。还有人会傻成这个样子,他如果愿意同流合污的话,早就找了一个那个国际传媒公司去当记者去了, 20 年记者的薪水有多少?他这个是真的门外面就有秘密警察到处要找他的。



史东 01:01:18 

这种人就是英雄了。



王孟源 01:01:20 

但是至少我敢直面批评像王贻芳,或者是像潘建伟,或者中科大那一票那一帮这些侵占国家公益的诈骗集团。中国过去这三四十年,因为相信英美的那一套。有很多人迷失了,很多精英迷失了,他们真的相信了美国那一套自私自利,以邻为壑的那套哲学。他们那个是,那是盎萨霸权殖民帝国的核心,你相信那个有什么用?人家不会让你去当那个世界霸主的幕后权力阶级的,人家只是想要把你骗来背叛你的自己的国家。俄国人都能够醒悟,我觉得这个,中国过去这三四十年教育的失败非常的严重,除了像台湾、香港这种系统化的腐朽之外,他这种是被非官方管道渗透,是非常非常严重,真的是要好好地检查。



史东 01:02:31 

这是一个相当厉害的手段。你不得不同意,这是一个非常非常厉害、非常有效的一个手段,而在历史上也证明它是一个成功的手段。



王孟源 01:02:43 

你光是习近平一个人不够,他不可能面面俱到。我自己当过十几二十年的主管,我知道我手下就是20 个人,我都没办法全面地去盯。而且正因为我当过主管,所以我其实对行政的时候要斟酌各方各面的考虑,那个困难,非常地理解。今天才有又有人在我的博客问我说,这个上海的防疫的措施是不是太过分了?是不是应该放松了?我跟他说这种事情呢,除非你有明确的证据说政府的决策是错了,你应该信任他们,因为他们才有那个视角,才有那个资料,才有那个全面的考虑。



王孟源 01:03:37 

但是呢,今天在巴基斯坦这件事情上,我可以证明他们渎职了,所以我才这么生气。我真的是我一般的哲学,喜欢在我博客看留言讨论的人应该知道,我一般的哲学就是我刚刚讲的,除非你有证据,你不能够指控政府的决策错误,但是现在这是证据缺阻。



史东 01:04:02 

你觉得以后应该要走什么样子的路,用什么样的方法?我指的这个问题并不指的是巴基斯坦这件事情,包括了巴基斯坦以及你所谓的整个的这个中国未来的外交格局,和中国这次俄国和乌克兰战争结束之后的外交的新格局,你觉得中国应该有什么样的盘算?



王孟源 01:04:28 

当务之急当然是很显然的这个全球化已经结束了,这个世界已经要分裂,这时候你在通货膨胀弥漫全球的时候,你的重点是争取最多的能源跟资源出产国。你要知道盎萨那个集团,他们凭借着 Genocide 种族灭绝占领了美国,加拿大,澳洲,New Zealand这些都是资源丰富,人口不多的国家,他们完全可以自给自主。所以这次这样乱搞,真正倒霉的是欧盟而已。盎萨你即使分裂也没有什么太大的问题。 他们除了…… 即使是石油,加拿大也是世界上主要的石油出口国之一。美国现在的石油已经自给自足,所以中国在当务当前的这个关头,紧要关头是抓紧所有还没有跟定盎萨的资源出口国。



王孟源 01:05:43 

要做这一点,你第一个要给他们一个货币上的替代,让他们能够独立,第二个要给他们安全的保障,都是很简单的事情。我真的不懂为什么还要轮到我一个人在康州的人来解释。看得再远一点,如果中国成功地逃过这一关,通过这个分水岭世界成为两个集团,这两个集团之间的竞争一定是靠科技发展作为主战场,这个今年所发生的俄乌冲突只是高度地加速了这个过程。我老早就可以预言这是会是在 2030 年代最重要的议题,在现在至少提早了五年,就是在 2025 年,在 2025 年之后,科技的发展速度跟效率就是会两个集团竞争最重要的一个决定因素。所以我,如果你记得的话。我在在1月的时候或者是 12 月的时候,有上你的节目批评那个量子的计算,然后我过去这几年从大对撞机开始,然后批评氢经济,批评核聚变,这些都是骗子计划。你要谈效率,第一个是要先消除骗子,因为效率最低的投资就是把资源被骗走,那是最低的是负的负,百分之百的的投资,而且事实上可能超过百分之百,因为连带的,大家有样学样上行下效。你如果最成功的、吃香喝辣的院士都是靠骗,那还有谁愿意做正经活?这些事情都是我在过去十几二十年都已经知道的事情,而且从博客一开始就已经开始做准备,一步一步地解释给读者听。我们现在只不过是因为事件急转直下,所以我列出更多的细节来讨论。很多事情都是预见。我在一个小时前我跟你讲到,我原本没有预期 Putin会马上出手搞军事行动,是因为它真的不是最优解,你看着事后 Putin也是后悔,因为即使他运气很好,这个整个第三世界还有中国、印度通通的在支持他。你说他如果能够重来的话,会是这样搞吗?我想至少在细节上会有更正。



王孟源 01:08:28 

我刚刚讲了半天,这些中国的这些幕僚,这些智库不像样,不像话,他们的思路应该是什么?应该是量化的几率预估。我举一个例子好了,现在有一个可信的研究跟你说,这个地方会发生大地震,在十年之内有一半的机会发生大地震。好,50\%,十年是500个礼拜,大约500个礼拜,50\% 除以百分 500 个礼拜,就是每个礼拜有 0. 1 percent 的几率会发生大地震。好,然后现在有大家恐慌了,有些人在媒体这个吵闹,说马上要大地震了,大家赶快都迁离。然后一个正确的分析,是说有99.9\% 的机会在一个礼拜之内不会发生,所以我们应该进一步地研究观察。与此同时,大家应该了解地震来的时候你要怎么办,先做好准备,但是先不要panic,先不要恐慌,那其实就是我在这一次战前所做的。我说开战几率大不大?但是要不要做准备?要做什么准备?做金融方面的准备,做宣传方面的准备,做外交方面的准备。



王孟源 01:09:55 

这些都是我其实不是,两三个月前讲的,是过去10年、8 年,在博客一遍一又一遍都在讲着做着该做的准备。但是我们现在这次就是刚好大家闹了半天说会打,结果真的打了它是一个小几率的事情,所以才叫做黑天鹅嘛。小几率的事情你可以不预期,但是你必须要做准备。我在过去 8 年一直慢慢地在培养自己的公信力。我觉得我还没有培养到足够的公信力。所以这一次你看这个巴基斯坦的事情之后,我除了生气之外,我自己也是懊恼,就是因为它发生得太快了,我 8 年的努力还没有让自己的公信力提升到我所做的预测能够让足够的人相信的地步。对这一次这件事情,我是愤怒而且懊恼,因为这些事情都是我已经事先预做,建议过要预做准备。然后事情发生之后我给出正确的回应细节,然后 Nothing happened



史东 01:11:12 

你指,你指对不起,你指的是这个俄国和乌克兰的事情,还是你指的是巴基斯坦的事情?



王孟源 01:11:22 

我指的是巴基斯坦的事情。这个我当然知道,我所写的都是用中文写的。Putin不可能参考。



史东 01:11:34 

那不一定。



——————————————————————————————————————————

王孟源 01:11:51 

美国Biden这一次为了要治俄国于死地,光是切断Swift——原本以为大家说是什么核弹级的制裁,其实不是——他真正最过分的是扣押了外汇储备。



所以你赶快给他们一个通路。还有什么新的货币是自己的货币更理想?我的货币是其中的10\%,但是 60\% 是人民币。人民币是是未来最坚挺的东西,然后偏偏它不能浮动。你说我要换 1000 亿美元的外汇储备,换成人民币方便吗?不方便,因为它不能浮动,不能够自由汇兑但是做成一个合成货币,一个篮子货币皆大欢喜。



我是希望在这个转折点之下,中国能够争取到越多的资源出口国越好,作为盟友,至少作为朋友。虽然我这一次上来抱怨的是那些,外交还有国安方面的,但是我觉得真正难搞,影响恶劣的是那些高科技的骗子,



——————————————————————————————————————————



王孟源 01:23:22



中国在这次是开始落后手了。我建议的那个亚元其实是为了当前中国的需要,贴身设计的,根本就不是什么区域货币,它只是挂着那个表象,是一个方便人民币跟的Ruble马上结合起来,然后可以进一步用来邀请中东产油国入伙的。



史东 01:13:48 

这个概念和当年,所谓当年,就前几年中国和日本和韩国谈的那个亚元的概念完全不一样。



王孟源 01:13:57 

完全不一样,你跟他们的合作,那就变成欧元式的合作。嗯,这个绝对不是,这完全就是,我的想法就是,中国占里面 60\% 股份,然后俄国占20\%,然后剩下的 20\%给那些产油国分一分,但是基本上中国是占绝对优势,然后你还是有汇率跟财政的绝对自由,那实际上的行政管理由,中国协同大家合作,完全做中国主导。这个其实我在我博客上都有解释过了,就是这其实是一个公关的套路,是用来收拢资源出口国,像俄国还有中东这些国家的一个手段。但是实际上中国还是牢牢地掌控了自己的主权。你如果要是搞像欧元那样的,这个不合适,尤其是欧洲的那些国家,他们其实指在宗教、社会、经济发展程度上都已经很相似了,结果还是水土不服。你搞了欧元之后 10 年就水土不服,中国跟韩元、日元根本没有合作的必要,而且他们在军事上也是,基本上是军事,外交上也是美国的附庸,没有争取他们的必要,也没有争取他们的可能。





史东 01:15:20 

有关于你谈的这个亚元,叫亚元是吧?亚元的概念。



王孟源 01:15:25 

我只是随便找个名字,就是你叫什么名字都可以。



史东 01:15:28 

你是用什么方法来吸引这些国家加入去。换句话说,这些东西对他们会有什么样的好处?



王孟源 01:15:36 

就是你给他们股份,这是他们自己的货币OK?



史东 01:15:39 

取代他们自己本国的货币吗?



王孟源 01:15:43 

不是,是一个篮子货币,你占20\%,就代表你贡献你自己的货币,然后这个货币的量相当于 20\% 的总量,然后你的这个货币成为它这个亚元银行的储备。亚元的价值就来自于这些其他货币的储备,对不对?那这很这样一来,很显然哪一家的 GDP 最大,哪一家就主宰,那很简单,先天就是中国会处于主宰地位。而且这时候人民币要继续搞,不对外浮动完全没有问题,你根本不必搞什么金融开放,把那些华尔街引狼入室根本没有必要。这个人民币之所以不合适开放,是因为美国在金融掠夺上面经验丰富,连英国算在一起有 100 多年的经验,将近 200 年的经验,你监管单位根本就摸不清楚他们的套路,等到人家都已经卷款走掉了,你都还不晓得是怎么发生的。我个人认为这个人民币不适合对外自由兑换。但你如果不要自由兑换,你怎么变成国际储备货币?唯一的路子就是搞一个篮子货币,然后你刚好就是其中最大的一名。



史东 01:17:11 

那么这个货币它存在的目的就是专门为了做国际之间的交流和贸易。



王孟源 01:17:16 

对,只是国际大宗贸易的定价。因为货币的话,你说货币的话它有好几个功能,第一个是交易,第二个是定价,第三个是储备。事实上储备才是最重要的,现在当务之急在金融上面,当务之急我只要讨论到这个货币的事,我只要去看一些细节,就知道这个作者懂不懂,是真懂还是假懂。比如说你说到美国美元的额份,我说60\%,有人说40\%,为什么他们说40\%,我说60\%?60\%是以储备。来算,外汇储备来算。美元有 60\%;40\% 是以交易来算。现在当前这个美元的霸权重点是什么?是储备,不是交易,交易只是次要的。所以那些去谈交易 40\% 的,你一看就知道这个作者不入流,只是读史书。真正现在美国通货膨胀这个弱点,你要真正打击它,就是要削弱它所占储备外汇储备的百分比,每一个百分比削弱,就是第三世界国家几千亿的钱不再用美元了,就是他们要卖出美元买新的,比如说是亚元。卖出美元的意思是什么?这些美元就回到美国去了,美国已经通货膨胀了,他们最不希望的就是这些美元这样子像洪水一样地冲回来,美国Biden这一次为了要治俄国于死地,光是切断Swift——原本以为大家说是什么核弹级的制裁,其实不是——他真正最过分的是扣押了外汇储备。不要说是外汇储备,算是所谓的 sovereign asset 主权资产,你就算是私有资产。过去 300 年的 盎撒的 propaganda 说的是什么?宣传说的是什么?说的是财产权至高无上。现在都拆穿了这个骗局,拆穿了像是奥迪阿拉伯几千亿的外汇储备,瑞士将近1万亿的外汇储备,中国 3 万多亿,只要美国不高兴,随时可以一身制裁马上就把你通通没收。



王孟源 01:19:43 

那些财主更加不用说了,各国的那些Oligarch,他们的那个游艇、生意、住宅说没收就没收,你这样子还有谁敢把你的资产放成美元或者欧元?这是天赐良机。我说天赐良机就是这个世界上所有的主权资产,还有富人资产都在急着想办法从美元跟欧元换出来,所以你赶快给他们一个通路。还有什么新的货币是比自己的货币更理想?我的货币是其中的10\%,但是 60\% 是人民币,人民币是未来最坚挺的东西。然后偏偏它不能浮动。你说我要换 1000 亿美元的外汇储备,换成人民币方便吗?不方便,因为它不能浮动,不能够自由汇兑。但是做成一个合成货币,一个篮子货币,皆大欢喜。这些事情我都是已经千丝万缕考虑过了,这绝对是远远的最优解,而且当前的需要是非常急迫的,真的是犹豫不来,所以我……你要做这些事情。你说我今天讨论的这个巴基斯坦这件事情,这是等于是间接地对中东释放中国或者俄国可以提供的安全保障。这实际上的目的是要求他们在金融跟贸易上站到你这边来,但是你如果不提供安全保障的话,人家怎么会愿意呢?他们已经有这个动机了,因为为什么美国主动去扣押大家的资产,私有跟公有的资产,所以没有人会安心把钱放在美国。现在去建立亚元这种 alternative currency,international currency是最佳的时机。但是因为时间紧迫,你必须要有计划、有理解地、好好地去做,高效地去做。然后我就是没有看到他们在高效地做。



王孟源 01:21:58 

所以是。而且我那一篇文章已经出来 6 个礼拜了,真的没有想到这个 14 亿人所谓的精英竟然没办法看出人家已经点明的正道,还是不愿意去做,真的是让人很失望。我这样子一来,所谓的学术改革,增进中国科技研发的效率,那更加是缘木求鱼,更加是不可能了,对不对?连这么简单这么明显这么基本的东西都做不到?因为我知道视频不是讨论这些专业一级的一个最好的通道,所以我一般都是只在我的博文上面才做解释。



王孟源 01:22:40



好吧,我再最后再讲讲我预期今年 接下来会怎么发展,发生什么。就是乌克兰战事,俄国现在已经把包围威胁基辅的那5万人撤回来了。然后撤回到 Kharkov 的东边,也就是他准备要包围那个东乌集团的北线,然后 Mariupol 现在也基本上快要告一段落了,就是只剩下一小部分的人员留下来收尾。他们原本有4万人在包围那个城市,现在至少有3万人可以抽调出来,那就可以从南线,那基本上他们原本是这样的,这样子包围着那个东乌集团,现在就可以那5万人这样下来,3万人这样上来。然后,嗯,这大概需要1 到 2 个月可以完成包围,然后接下来很不幸的是不一定马上结束,这一仗这样打下去。然后很快的法国马上要选举。然后变成 Le Pen 跟Macron竞选。这个也是一个一大变数。我原本建议说去年秋天,就是半年多前我就已经说Merkel已退休,如果是绿党。执政的话,这个德国就不可靠了,必须要找法国来做交涉了,那现在我们就只好静等大国交涉完毕。不过,过去这个月昂萨媒体的拱火,还有这个,Neo Nazi 很多离谱的行径,这个事件恶化得比任何人预期的都快。所以这个Macron是不是还有——即使马控当选——他是不是还有能力跟意愿来来调停已经很成问题,或许要我们也必须要等到德国的政府垮台,就是绿党垮台,他被踢出政府,我们才能够有真正的决定。那这个就很可能要拖到秋天,就是第一个你光俄军吃下这个东乌野战军团,如果他们是在5月合围的话,然后他们在死战,那等到打完大概都快要7月了,然后打完了以后还至少。



史东 01:25:07 

你的意思是……我这么看。我这么解读你的讲法,就是至少在明,在今年冬天以前这个事情会告一个段落。



王孟源 01:25:15 

我希望如此,因为你如果能够在7月把这个东乌军团解决的话,接下来要打Odessa,就是不仅至少要把Odessa 打下来,我觉得是因为——我在节目刚开始的时候我也提到说——这个 Neo Nazi 所做的残杀战俘这件事情是很有很严重的后果。另外一个长期的后果就是我想现在俄国人人喊杀,所以原本的那个,我上次上节目讨论那个Putin开的那 六个条件恐怕是不够了,恐怕是真的要把那个乌克兰南线的沿海地带全部吃下来。



史东 01:25:57 

你觉得他会不会把乌克兰全部吃下来?



王孟源 01:26:08 

可能不会吧,因为要全部打下来,可能要达到明年,就是因为它这个不是跟一个普通的国家军队打仗,不是你把它包围然后绝望了,眼看着就是歼灭了他们就投降,他们不投降,因为那些亚速营的人知道他们投降也一样是死,所以你光看Mariupol,我上次上你节目说大概还要两三个礼拜,现在已经快要 3 个礼拜了。对,真的就是打到现在他们还有 1000 人在那边死守,就准备,真的是准备要战死。那你如果每一个师都是这样子的话,那你这个这真的可以打好几年,就是你最快也要到7月才能够解决东乌野战兵团,然后在那之后才能去打Odessa,打完 Odessa以后你至少要威胁一下基辅才能够,要重新威胁一下基辅,才能够逼那个Zelensky 签合约。就是……



史东 01:27:12 

我觉得难的是什么?从俄国的角度,他的困难就是不管是乌克兰的国内还有外部的影响,都不希望他谈和,就是他没有……



王孟源 01:27:24 

中国想要他谈和,法国可能想要谈和。



史东 01:27:28 

但没有什么影响力。现在对不对?对,美国不想要谈和。



王孟源 01:27:34 

美国不想谈和。对,而且现在的局势就是欧洲人太笨,尤其是那些精英太笨了,美国人随英国跟美国随便造个谣,他们就热血沸腾跳到前面去自杀,是不是很可笑?



史东 01:27:52 

好,这个事情我们还是当然了,还是那句话,继续观察着,继续看着,继续观察下去。



王孟源 01:27:59 

我是。希望在这个转折点之下,中国能够争取到越多的资源出口国越好,作为盟友,至少作为朋友。再用几年好好地整顿自己内部的学术界。虽然我这一次上来抱怨的是那些,外交还有国安方面的,但是我觉得真正难搞、影响恶劣的是那些搞科技的骗子,是科技方面的最严重。因为你长期下来,接下来这个阶段,这个历史阶段我已经说过,欧洲是最衰弱的,世界分裂以后,盎萨集团没有太大的问题,就是他们的问题都只是他们自己制造的,没有什么克服不了的。他要资源有资源,要技术有技术,中国还是要必须要跟盎萨集团竞争很长的一段时间,那竞争期间这个科技发展就变成关键,所以,我才会在过去这几年一直很关心中国这个学术腐败的问题。

























史东()

我们再次为您请到的是你的好朋友,王孟源,王先生来和我们谈一些他最近想谈的事情。我们废话不多说我们就把孟源请入我们的画面之中



王孟源()

我想当前的大新闻还是乌克兰的战事,过去这三个礼拜的发展基本上是证实了我们原先的一些推测,澄清了一些不确定的事情,当然也有一个很重要的新发展。这个我都想谈一谈,不过我最今天之所以急着要上节目其实是因为看到一则新闻,就是有关巴基斯坦的新闻,我觉得从这里可以管窥中国政府内部在战略政策制定上的一些不足。我想拿出来翻出来谈一谈。我想我们在我在二月初的时候好像有上节目,一月底二月初的时候上节目,那时候我还信誓旦旦地说这个仗打不起来,当时其实逻辑很简单,就是啊,根据当时的国际战略跟战术形势呢,普京没有理由马上出手.然而当时已经有传言说乌克兰准备在美国的支持下,对东乌的两个共和国做全面的进攻。而且过去八年来啊,那种北约对乌克兰做了很多的军援,基本上把他们重建起来,而且战力事实上上比八年前还要强。然后这八年之间他们的那个亚速营还有其他的一些比较小的Nazi组织也都成长到旅的级别,从营到团然后到旅的级别,分派出去大概已经占到主力战斗部队的1/10,而且控制了国内主要的警察机构还有国安机构。所以他们的那个不能说是纪律,但是战斗意识很强,非常的残忍,就真的是像当年的纳粹的SS里面党卫军里面最糟糕的一些部队,当年那个党卫军甚至还有专门用囚犯就是强奸犯跟杀人犯组成的部队,那专门在占领区去干他们的专长。那么原本就没有受过军事训练,他没有军人的职业道德,更加是手段用地极为、很多完全没有节制。





不论如何我在两个半月前上节目的时候我的判断是战争不会马上打起来,原因是即使乌克兰急着出手,俄国也有一个比较好的应对政策,就是用长程炮兵来摧毁汤的桥梁铁路公路油库这些后勤措施,然后呢用火箭炮还有长程重炮来摧毁他们的野战攻击部队,至少这样子先打上个四五天啊先削弱他们的锋芒,然后呢你再反击,那么名正言顺,而且可以掌握先手,就是你因为对方已经摊开来那个阵势,是不适合他们防御的。所以你可以从三面包夹的局势,选择最有利的地点来来做攻击,我是根据这个逻辑来做的,而且从大战略来看,英美跟欧洲都即将面临也很严重的通货膨胀危机,普京只要再多等一两年,基本上这些西方国家自己就会乱成一团,所以实在是没有理由急着出手,但是呢普京还是出手。那我们是后来检讨他的确是犯了一个错误,第一个他太急;然后第二个他在出手的时候那个作战计划太过乐观,这一点我在上个月上节目的时候也谈到了,就是他没有遵从正确的军事原则,作出速战速决的部署。然后用派兵又不够,他前后只派了十几万部队进入乌克兰,乌克兰光当时的现役军人——就是二月底的时候还没有全民动员的时候——他的全部现役军人就有26万,在东乌前线的精锐部队有8万,你用19万的部队去进攻而且他的二线三线部队可以依据城市打巷战,那一线的野战部队在东乌前线已经建筑了八年的坚固工事,所以这些都不是很简单就可以攻坚下来的,不是打的运动战;结果俄军进去的时候还是运动战的原则,基本上是轻装突进没有没有做好先期的轰炸,那我们上一次谈的时候也讲过这是因为普京想要攻心为上,一方面攻心为上,一方面呢也是要照顾这些很多讲俄语的人民不想让他们受苦,不过你要做出这样的决定,一个先决条件就是就是你认为可以不必打硬仗就可以赢得这场战争。





那么我们现在回头来看,有很多的消息已经透出来,可以确定两点:第一个就是在二月底的时候,普京的确是拿到情报——而且这个情报的确是正确的——就是现在已经证实了,当时的乌克兰军方的确是准备在三月底对东乌进行全面的攻击,普京的确是做所谓的先发制人,但我说我刚刚解释过,就是你即使要做先发制人,也有很多不同的做法,那他这个做法是在国际外交上还有战术上都不是最优的选择;第二个是啊现在也已经有消息反复地证实,一开始他们这个作战计划拟定的时候呢,是假设许多乌克兰的地方政府会主动投诚,就是他们只要抢占交通要点然后对城市做出三面或四面包围,那么很多啊比较有亲俄的市长啊或者地方首长就会主动投诚,结果一个都没有。他们真正轻松这样进去的只有Kherson一个城市,而且Kherson连前十大城市都排不上,之所以能够拿下来是运气好,因为当时乌克兰根本没有部队在那里,就是一个空城,就被俄军军趁虚而入。



那为什么后来乌克兰没有像普京预想这样子有投诚的现象呢?很简单,第一个是长期的因素,就是在冷战结束后三十年来乌克兰的基础教育已经都改写了,把一九三零年代斯大林做农民公社的那一次的改革,其实跟大陆的那个大跃进的很像也是全苏联死了几千万人,其实乌克兰不是死最多的,死最多的当时是,最惨的是哈萨克斯坦重要的农业区,那俄罗斯本身也有很多地区跟乌克兰的那个灾害一样的严重,但是呢你在冷战之后因为乌克兰成为一个独立的国家,他们最怕的、威胁他们独立最大的就是当年的苏联和后来的俄国。所以他们就改写了他们的教科书,把那一次的灾难写成专门对乌克兰做种族灭绝的一个企图。那我想这几年熟悉台湾这个教程修改的很大觉得,这个动作是似曾相识。其实啊,在乌克兰早就做了,比台湾还要早。所以你三四年下来很多那个酱油就乌克兰本身跟俄国最大的那个冲突,就是他的西部,其实是以前波兰的属地,是天主教的那些事反而最厉害的地方;然后再有了这个历史上把一个苏联共同的灾难,说成是斯大林专门为了对乌克兰人的种族灭绝的那个残忍事迹,那所以就又种下了历史仇恨的那个心态。所以乌克兰其实对俄国整体来说仇恨的人比较多。就是当然还是有说俄语的那些人,集中在南部跟东部的几个省份,



史东

为什么对不起其实我觉得这个普京啊当然我是在一般在才来这边

也是我的观察他在乎的就是东部这一部分在议论纷纷会不会打



王孟源



我那时候我在博客还说,这个当时我当晚就说,我认为是不会打因为他不是最优解;但是我说如果打起来的话呢,他不会占领,不会试图占领整个乌克兰,最极限是把整个东乌跟南乌的俄语区占领下来也基本上就是吃下他的海岸线,Odessa基本上就是底线了。待会我再会谈这个当前过去这三个礼拜的局势发展,会对这些评估有什么新的改变。



我刚刚是说到他原本预期会啊不能说是他准备老百姓箪食壶浆,但是呢他是没有预期到会有那么坚定的仇恨,然后即使是在俄语区他也没有遇到地方长官主动投诚的,这原因很简单就是我刚刚提到的那个亚速营已经散布出去八年的时间,已经散布出去掌握了所有的国安跟警察机构。所以战事一开始的第一个礼拜,他们立刻枪毙了两个市长——就是最公开亲俄的两个市长——如果是有亲俄历史的早就躲起来了,所以俄军也就没有办法得到任何的当地人的响应。这次为什么他这个整个部署很明显的可以看出来是一副计划落空的样子,就是他急着抢先,没有预料到乌克兰的抵抗意志会这么坚硬,没有预料到地方长官没有一个想要投诚,没有预料到——他们乌克兰的步兵一座小分队来做伏击,这个是需要很强的仇恨才可以做得到的,因为你即使得手,一个车队里面有十几二十辆装甲车,你得手一辆、两辆、三辆,其他的装甲车对着你这些拿着飞弹的兵来您你也是等于是自杀性的攻击。就是要逃跑的机会并不多。这些这些小分队的兵,你跟俺就是一次十个人,二十个人,一两个班,他们基本上就是自杀性的,你战斗意识要非常的坚定,才能做到。所以我想俄军也是没有想到会遇到这样的情形,然后打着打着,很快地经过两礼拜他们知道苗头不对了,那当然他们的作战计划不是完全一厢情愿的,就是他们也会有最坏的计划,就是如果对方真的出乎意料地抵抗非常得坚韧的话要怎么办。结果,我觉得是他们的计划的内容最糟糕的可能出现了,所以他们在两个多礼拜前三个礼拜前,做头一个月战事的回顾的时候说,我们基本上在实现第一期计划,后来上个礼拜吧,还是上个礼拜出来说,我们第一期计划完成了。完成了呢,这个攻击基辅的呢,来自不对其实只是佯攻,对你现在完成要对东乌的乌克兰野战兵团完成包围,那我们要专心来把它整个包围歼灭。他们没有说谎,事实上俄国的国防部到现在的一两百个公布的事项,我还没有看到一个他们是有意说谎的,相对的乌克兰——



我想插一句哦,有一个人叫,我想大概也注意到了,他其实二十年前就很有名就是注意他就是二十年前的时候注意到他,结果在最近我看到他,其次接受我这个不同的节目的访问,谈到有关于这个但是有关于这个战争的事情就是俄国和和乌克兰战争的事情他提到事情就是和你刚刚讲到的这个佯攻基辅的事情,后来是普京的发言人还是国防部的发言人他还讲了这个事情就是有关于他的目的,他的目的就是他们的木球他们就是俄国俄国的目的就是要把乌克兰军牵制在这个乌克兰的首都附近,让他不再能生源乌克兰东边的这些不对



王孟源:

这绝对是从一开始计划里面就有的,但不是计划的全部,计划遇到最坏状况下,你这样子部署必须是,只能够继续这样继续下去。





他们最理想的状态当然是俄军经过,那些城市就不战而降,然后把基辅包围起来,然后怎办也投降了啊大家都很高兴的回家。我上一次上节目的时候提到的人六点,主要要求其实不止六点,但是其中的重点,一点就是我讲的那6.22然后基本上一切顺利的话

两三个礼拜烫就可以让乌克兰修宪然后他们就可以鸣金收兵啊

大家都没有什么时尚而是后发生的事就是因为乌克兰亚速营的那个骨干非常的呃仇视俄国却是让是不择手段

已经到了不择手段的地步当然这种硬仗也是他原先计划里面有没有考虑过的

只是只不过是你刚刚讲的那呃我刚刚讲到是俄国国防部公布了100多个轩去年我没有看到一次他们是有意沙皇

然后我看了乌克兰的尊严我看了几十个没有一次看到他们是有意义说实话的

我觉得这个对照这个对比很明显不论如何把这个呃他们现在必须要执行的战略

就是开战的时候出乎意料必须打印着那么一丝9万达26万再加上后来预备部队动员啊

你当然是呃不可能像美军那样摧枯拉朽尤其他们不愿意用火力全部火力这样的盲目的压制

对对您的水电网路都没有断一直到现在都还没有断我从一开打的第二件第三天我就说这很奇怪

-- 豆豆博客再讲讲对不起我再插一句换有关于这个Bucha的屠杀的事情你是怎么看的

基本可以确定是乌方驾照的目前的目前有一些传言说是啊英国帮助乌克兰做的但是也有可能是乌克兰

这个我觉得MI6参与的证据啊还没有没有直接证据

但是呢证明是乌克兰方啊说造假的是很明显因为他们完全提不出这个时辰时间的流程他提了十几个不同的解释

就是所谓他们就是指昂萨的媒体向应用tang m或者是或者是RE post

或者是他们提了十几个不同的解释但是呢每一个解释说的这这些一死者的身份

他们是什么身份是怎么死的是在什么时候都不一样

你是指控方啊不同的答辩方当然有可能说因为是呆账的所以我完全很多外地我就要掏钱吧我不知道你在胡扯些什么

所以答辩的时候有可能是十几种不同的哪边这个要证明的责任是在你指控方您指供方怎么可以东说一句说去奥丁那个后来连德国情报局都都闹出来所以有什么拦截的无线电信号是不是证明后来我去听了一下没什么吗

就是他们说一句气话就是说这里啊有这个老百姓对我们不客气然后里面有一个说操他们的对不对

转自什么证据而且时间也对不上地点也对不上

你完全没有办法指明是哪个俄军部队在什么时候为什么在那个地点做了这件事情

那我们现在到现在还是众说纷纭说这些死者到底是真的

这些死者是亲俄的民众ok还是在俄军占领期间他那个正受到了乌克兰的重炮轰击了好久

已知是有很多民众被打死对所以也有可能这些民众是当初呃更早这个乌军的炮火所打死的人

也有可能是演员我们不知道

但是呢我知道的是指控方的说法呢漏洞百出

最严重的是3月30号俄军就撤离了3月31号当地的市长很高兴的

在网络上那个录音对我来讲很重要很高兴地说我们已经解放了对一切安好等待国青队我们胜利了把他们赶跑了对不对

结果这个什么Bucha屠杀什么呢一句都没提然后到了4月2号乌军才姗姗来迟因为他们很怕呃俄军我跟他们一样真的是

他们做的事就是他在撤离的时候因为战争的找其他事一直往后扯的都是俄军代理前景

所以一直到到三月底才有俄军后撤让乌克兰进军的机会那他们很怕俄军做跟他一样的

就是留下很多卜匕出来还有地雷什么的所以他们小心翼翼的

所以虽然那个Buhca及时距离体验不远他其实算是一个成交的一个小灯对不是不是独立的一个地区

但是呢他们还是姗姗来迟等了三天才来然后当天还没有做什么现在已经有证实的报道说

一个没有募集到诚实在街道上的那个地区有有一点距离的一个人说4月2号乌军来了以后呢啊到处抢劫

就是按忙着的是是哪家哪户的进去去搜那个看到了

然后世界第一次听到说有啊这个什么Bucha屠杀是4月3号啊所以其实是俄军撤离之后四天了消息才出来

那这之间你补差有水有电有网而且大家老百姓还是住在那边然后到了

第三天乌克兰部队才到你怎么可能说怎么可能是说俄军屠杀了然后留下几百具尸体

然后这三四天都没有人管完全不合常理的

只有完全被西方媒体洗脑因为西方的教育在过去的四十年他们的整个社会的衰退其中的一环就是公共教育

你得退化公共教育更高层级高高等教育的退化就是学生已经不在学所谓的规矩就是逻辑思考逻辑思辨

事实上发生的事实上有证据的乌克兰官方宣称德国杀了乌克兰的平民

你连这种最基本的事实观察到底要怎么样精确描述都做不到的话

您不可能去啊辨识真假考虑增加就是说他们的教育的衰退

刚好就是这些把资本家在过去40多年掠夺国家权力跟财富的过程那非常的切合他们的需要

所以所以他们很高兴的鼓励这件事你想想看美国的这个快乐教育也就是40多年前开始对我刚刚算了一下

80年代初开始吗喂跟上台的时候对对他们而且我我才把这个话题在不在expand一点啊

你刚刚所提到了却乌克兰的教育你讲到台湾的教育我还想到香港的教育

这都不是没有原因的他的他的剧本是很相似的都是有有有一套同样或类似的剧本在后面推动这件事情

对对就是基础基本教育跟媒体宣传双双管齐下对然后您养成一群人云亦云啊宗教迷信是的相信这些啊这整套说辞的的人

然后呢啊只要在他们成年以后继续用媒体的报道四面八方的红枣就能够满足他们的呃调和就是自我感觉良好的

没有选择必须要另寻同盟你现在重点要争取的是中东那些产油国

但这些中东产油国最担心的他们最严重的考虑就是呃能不能你能不能保护她男权

美国要过来跟你搞颜色革命的这种您保证他们能够活下去

你如果连巴基斯坦巴基斯坦跟中国的关系是多么的亲密你实在想不到一个更亲密的国家了

而且巴基斯坦也是伊斯兰国家跟这些国家有很多军事合作他们都是很熟悉的

结果你连巴基斯坦都保不住你你怎么好意思去拉人我如果是外交部派出去大使跟他说你跟我们合作吧

你用改用人民币吧或者是怎么样我都不好意思讲你连巴基斯坦都没办法保护好你凭什么叫中东产油国站到你这边

买的隐隐的有点担心啊就是是因为俄军当时还是有威胁基辅的态势

所以泽连斯基的他当然完全不懂军事记得他除了演戏以外什么都不懂啊

所以虽然他身边的军事人员会跟他讲5万的不对根本就不够进攻基辅坐像站

但是呢他一定还是将信将疑因为没有那个专业能力呢

所以他当时愿意表面上做出局部的退让其实是因为在威胁下兵临城下的结果

史东()

现在你觉得有多大的权力来决定这个谈判的内容跟谈判的方向排球赛,还是他完全是我们所认为的一个完完全全的傀儡而不是说部分的傀儡

王孟源()

他一开始这个部分的傀儡我

我想我上次上节目的时候已经提到他们第一次谈判的代表有一个是银行的断开

那个人后来开完会回来就被枪毙了他一定是亲俄派的对

你说会把那个人派出去一次简单决定派谁去还不是完全有亚速营来给您

那现在呢我想啊已经很明显的是有亚速营这些你有哪几个完全掌控就是世界

事情越拖越久就是夜长梦多吗这就是为什么为什么我刚刚提到普京的这个作战计划问题很大

就是一旦出了问题以后就变成消耗战那这个反应不只是调兵军事上的反应也有啊后勤还有啊明星世纪宣传

还有我刚刚碰到的到家栽赃这些都是需要时间的

你如果当初不仅是以雷霆万钧之势以50万或者是60万的军队在一个礼拜之内就被所有的主要城市完成包围

我想我们现在也许可能他们已经签了合约然后俄军已经准备要开始撤离西乌克兰

史东()

话我们话讲的这个点上啊然后你帮我们往前看你觉得这个事情会怎么发展而且你觉得普丁会怎么样在处理这个事情

王孟源()

我刚刚也提到暗示了一下就是过去这三个礼拜最在这个乌克兰这个领土上发生最重要的事情

就是亚速营掌控了对乌克兰整个政治军事体系的掌控越加的人他掌控严密之后

又刚好遇到昂撒媒体对他们的绝对纵容

就是不管他们编出如何离谱的谎话总是有那些总是有西方媒体的绝对配合

比如说在马里乌波尔他们有一个产科医院被炸了

结果呢在那个路透社拍的照片然后有头条新闻说这个俄军轰炸了妇产科医院

然后有一个产妇啊满面鲜血的跑出来一个照片传给了司机我不想自己

再后来俄军当然是收复了马里乌波尔然后呢找到这个产妇问他去看了他的拜访的产妇

说当地有三个医院其中两个被乌军占了他只好到第三医院去到

乌军还过来抢他们的食物然后呢事发的当天

胖子听到两个爆炸声但是这个爆炸不是很大的爆炸炮弹那种爆炸不是炸弹爆炸

我我想跟大家解释一下就是重炮的话他的那个火药也只有6公斤

如果是步兵c型的那种火箭炮的话他这个这个1公斤级的12公斤级的火药

但是呢那个重磅炸弹就是比如说当年1999年啊美军一直最喜欢用的

包括1999年它们把中国驻南斯拉夫大使馆炸垮的那个炸弹是一顿记得就是1000公斤级的

那你1000公斤级的炸弹下来大概1两发就可以把普通的楼房给炸毁

那么容易走的报告是这是俄军的空军轰炸用重磅炸弹轰炸

但是这个产妇说其实爆炸不是很严重

就是当晚他们可以听得到感觉到爆炸但是爆炮弹的爆炸或者是更小的爆炸

然后爆炸以后他们到地下室去躲了十分钟然后大家就说好没有爆炸

大家都撤离撤离的时候容易走的记者已经在那边等着了

就是他们还没有出来还在地下室的时候多一点那个记得已经进去找

换句话说有两个火箭弹或者是两个炮弹落在这个建筑上面然后在五分钟之内入围者的记者已经开始采访

你可以想想这是什么很明显的就是专门为了那个设计的所以汤坑

可能就是用普通的火箭弹但是啊虽然是火箭弹的爆炸威力不大

最后还是有一个孕妇死亡就是他不幸被那个呃玻璃疙瘩所以失血过多

你想如果是你一顿的重磅炸弹里面有1两百个医护人员跟孕妇还有病人的话

怎么可能只有一个人而且是轻伤然后之前死亡的根本就是你只要去追究一下那个细节就凑不上

我刚刚要提这件事情主要是要给你示范一下就是这项是那个什么Bucha屠杀啊

还有这个网游普德妇产医院呢其实这个世界到

光是我知道的在过去的一个月我知道的就至少有二三十件都是很类似就是无中生有的栽赃

然后呢啊他们这些西方媒体所需要的就是一个头条头条之后他们就不管了

你要在停的话你就是呃方的宣传人员那正确的态度是什么呢

比如说像不巧屠杀的话你必须要知道是谁被屠杀

这些人是有名有姓的那屠杀的人适用什么锁手法味道什么在哪一天怎么样这些东西是必须要调查出来的

那我接下来讲反过来说正因为这些您有哪几可以为所欲为

然后反过来由西方的媒体来替他们擦屁股来遮掩然后倒打一耙

所以他们越来越没有忌惮在过去这个月有十几个视频录出来他们是如何虐待残杀战俘呢

这是他们真正自己发出去夸耀你看这个抓住了一些这些俄国禽兽怎么样折磨他们怎么怎么样招人

有比如说那个用步枪把他们的膝盖打烂然后或者甚至把下腹部打烂然后在明笑着看他死亡这样的

俄国二国不但时那些受害士兵的身份甚至零拍摄的视频里面为首的那个应用垃圾都能够提供身份的指正量才是叫做证据

因为要指控一件事情你必须要提供主动提供细节你光是单边说这件事情发生了这种事情是只有骗子才会认为是足够的

因为这个这个厂里来撒谎的人不必要对提供证据的证明人家撒谎的人才需要器官甚至这个班基本上是反过来了你知道吗

傻子才会想才会以为是这样的但是问题是西方的民众95\%现在在乌克兰讨论乌克兰这件事情有一个党校有

你刚刚讲的美的这样搞砸了利润对有难度哦的人哎咱的吗啊他的朋友也是希腊人

然后有一个法国的你记着这个法官他们继续我们的钱包里面没有剃过其他人

我们这节目的上次都都讲到了都提到就是你发文女士在我们漏掉了那个女记者很棒

对他也是因为他已经在Donbass八年呢对他完全理解逝世四十周年对无菌已经盲目的用炮火

还有还有各种各样的手段要残杀然后呃迫害当地的亲热的百姓

对他很有个性的这个大家有空就知道自己关机准备在这里Plug一下他的这个纪录片他那个纪录片

史东()

我怎么形容它了这边应该看但是看起来会很难过因为那进度里面你们记录的试试是非常让人难过的,但是这个纪录片的价值太高太高而且最纪录片我在网上找到有英文字幕的打另外我也找到了中文的

王孟源()

而是我的烦恼这个翻译版就是我我跟观众打招呼如果你有兴趣然后去找一下

就是了呀麻烦你把他的姓名也出来忘了他哎

史东()

对对对对对我我我我等一下找到列出来了

王孟源()

他那个那个法国姓名我记不太清楚而且这些人很明显都是没有背后没有财团

的自主的对有良心的人吗有良知有人对这个世界上总是有一些这种小树

让人家敬佩的这种有良心的人这些事做视频的那除了这些人之外写博客的

更多比如说Moon of Alabama这个也是写了今年的不是补课那个

史东()

那个是你请介绍给我的

王孟源()

 Moon of Alabama功力很深厚这个对真相世界正这个国际事件真相有兴趣的读者有必要去看看啊

过去的将近两个月一开始的时候俄国的名义其实是很极端的就是有极为好战就是啊

也有极为亲西方的普京其实是比较中间派很有意思啊不要以为他是一个强硬派

但是俄国人他们的那些学术精英只要还有一点脑子啊都已经放弃了他们的所谓的西方的了pro

已经了解到这些这么多年来啊他们那些国家主义者所说的必须要站起来抵抗西方的侵略说的都是实话

因为你西方不是讨厌的恶行而是就是讨厌你因为你就不是西方的一员

就是基本上除了这些少数人以外现在啊俄国的内部的民心空前的团结

史东()

对我有听到这个这种报道很多原来比较亲美国的那些职业群这,或许是那些文化圈或者那些人物圈现在都慢慢地回过来很快的转向这个支持普丁,这个评论员说是转向原因主要是西方美国人西方他们做的太过分

王孟源()

我觉得是因为俄国的精英他们所售的逻辑教育还是要比中国强

你看中国的那些啊所谓的公知他们在当前那这些精美的丑陋真相公诸于世之下呢

他们只是暂时的忍气吞声并不是真正的悔改但是在呃国就有全面的悔改他们的

他们相当于工资的那些人现在是真正知道自己当年错了

同时在东欧有两个选举结果啊就是两个亲俄的候选人都大获全胜

一个是匈牙利的Orban另外一个是武契奇

同时呢美国也利用机会

因为印度跟巴基斯坦都拒绝站到欧美那边去

他们拿印度没有办法那个Modi 的印人党是一党独大

而他们一党独大的原因我也在我的博客上深刻讨论国际的两年前因为他们控制了基础教育

这其实跟土耳其那个是一模一样

就是当地的公立学校很差劲所以呢穷人子弟很少有social mobility凭自己的学习能力往上爬的机会

所以都是宗教性的在土耳其是伊斯兰然后在印度是那个印度教极端分子

他们建立了一个私家学校的网路专门以低价提供优质的教育

但是这个优质的教育是绑定着这个意识形态的

所以经过了二三十年以后呢这些训练出来的新的穷人出身的知识分子就变成他们新生的力量

让他们成为公务员成为教师成为警察这些中产阶级的时候

也同样的将他们的这些极端意识形态渗透到整个社会里面

去经过二三十年大笔的投资有有一些亿万富翁投资然后资助才建议出来

所以当年那个呃马英九台湾的洗脑却是有功加工费来来对新一代洗脑

所以呃当初我也讨论过这个这么明显的让台独捡便宜的事情

马英九却不知道要扭转着光是这一点他就是历史罪人

因为你把一整个世代的台湾的新生代的脑子都给腐烂掉了而且不是事先很难预见

事先很多人都跟他讲就必须要做他就是不做

在匈牙利跟塞尔维亚选举的同一天的

美国鼓动反对党对抗他们的总理对Khan提出不信任投票

因为证据确凿就是很明显的是美方的运作

作为他不支持美国在联合国提议的那个惩罚

所以他们就有那个议员的副议长宣布这是违宪的

因为是作为外国的代理人他们这个不信任投票被就被裁定是违宪

然后Khan就去要求总统叫他立刻安排重新大选的意思就是

因为事实上巴基斯坦的名义对美国是并不友善

如果现在用大选然后又刚好有这一条新闻他们基本上Khan可以确定他会大获全胜

所以一开始大家也以为说自己就到这里为止然后结果昨天晚上今天早上消息传来说反对党上诉到最高法庭

然后最高法庭裁定支持反对党就是不能够重新大选就是完全转到美国那边

史东()

不信任投票行为我并不是说投票的结果已经达成对不对反对派敢提就是他们已经掌握了这个有这个信心

王孟源()

就是他们已经收买了足够国会会员

那这个钱是哪里来的大家都可以想象

对我看到的时候很生气啊怎么会笨成这个样子

我为什么会有这种感觉呢因为当前俄罗斯跟西方的斗争其实是有两条战线

一个是我们刚刚谈的很多的军事战线

另外一个是经济v展现这是美国典型的打击手段就是用制裁他们的目标

短期是希望扰乱俄国的进出口然后呃通过让他的货币无限贬值来造成内部的经济混乱然后达到政权颠覆的目标

其实我从八年前开始写博客就已经不知道描述过多少次这是他们典型的手段

因为这个手段的代价极小额利益当百试不爽

即使有代价这个代价也不是美国人自己负担的是有他们的所谓的盟友负担这次是德国

在这个经济战线上这样的出手我在上次上节目之后也已经解释过这个是因小失大的傻事

因为刚好现在美国美元已经超发了20多年在面临极大的通货膨胀压力

你再去扰乱一个世界数一数二的能源供应国的进出口

你基本上是保证这个世界的能源供应进一步的促进通货膨胀

其实上你的节目也已经谈过好几次了这个在冷战后这三十年

其实美国的霸权的基础就是美元他们已经放弃了他们的制造业

他们的军事也开始用腐朽被中国从后面后来追上的态势

然后至于文化教育社会等等商业也都是腐烂的已经快要不出来的地步

他们现在靠的就是用无限印钞来解决任何的所有的问题

这样子如果美元被取代的话就是真正的釜底抽薪一次性的解决昂萨这个殖民主义帝国对人类过去这三百年的桎梏

人类工业化三百年来的历史其实就是昂撒压榨第三世界同时打击挑战者的一段历史

这期间不知道有多少和挑战者被他一脚踢下去

然后呢他为什么会变成谎言帝国呢因为他们坐在霸主的地位上有那个资源

有那个需要要来创造这些谎言而且有三百年的时间来把它精雕细琢传播出去

虽然他们上一次打败挑战者是只有三十年前就这冷战结束那段时间

他们同时打败了日本跟苏联其实这不是三十年一次的转变而是三百年来人的第一次

第一次昂撒面临这么严重的问题而且他因为他们自己内部的腐朽

这一次是真正的危险既然他们的危险的根源是美元

我们这些坚持真相关心世界人类整体福祉的人当然是希望能够针对性的打击美国现在的弱点

所以我在我的我最新的一篇博文真是六个礼拜前写的

特别提到中国如何在这个通货膨胀高涨的历史阶段

决定货币最重要是什么就是大宗商品尤其是能源这是为什么俄国这次基本上有恃无恐

即使他犯了错误也可以很简单恢复过来现在他的经济已经恢复过来了

而且他军事虽然是一开始的部署啊不对但是呢现在反正国内团结了

继续打下下去的话也只会越打越顺这个长期打下来乌克兰是没有希望

那既然如此这是一个天赐良机中国可以拉拢俄国然后双方合作

由中国提供财富跟啊人员跟制造能力俄国提供部分的军事还有他的有其他的能源

这时候的关键是什么关键就是中东产油我们说石油美元石油美元最近是五十年的事情了

你这时候我们现在正在面临又一次通货膨胀而上一次是美国抓准的通货膨胀的机会

把美元跟全世界最重要的大宗货品绑定现在我们又面临了一次通货膨胀的危机

中国应该在绑定二国之后也把中东绑定

要绑定中东的话你必须要给他们特别的甜头因为毕竟他们国内是有美军驻军的

所以我设计了一套国际篮子货币用来用来吸引他们

但是呢你光是有胡萝卜不行你还是要能够解决那个大棒的问题

就是他们当地有美国驻军而且呢美国人最喜欢搞颜色革命这个很重要

因为我刚刚已经论证过了中国现在做重点要争取的就是你已经有俄国

俄国已经没有选择必须要另寻同盟你现在重点要争取的是中东的那些产油国

但这些中东产油国最担心的他们最严重的考虑就是能不能你能不能保护她安全

美国要过来跟你搞颜色革命的时候你怎么保证他们能够活下去

你如果连巴基斯坦跟中国的关系是多么的亲密你实在想不到一个更亲密的国家了

而且巴基斯坦也是伊斯兰国家跟这些国家有很多军事合作他们都是很熟悉的

结果你连巴基斯坦都保不住你你怎么好意思去拉人

我如果是外交部派出去大使我跟他说你跟我们合作吧你用改用人民币吧或者是怎么样我都不好意思讲

你连巴基斯坦都没办法保护好你凭什么叫中东产油国转到你这边来

我刚刚讲了这么多为什么我一看到这个消息你说的那些国关的做外交部智库跟幕僚的人

这种事情应该是他们吃饭睡觉做梦喝水都在想的事情

这个方法很简单吗我只要两通电话一通电话打给巴基斯坦的参谋总长

一通电话打给他们的联合情报局的局长

我要跟我中国站在一边的话你不能够让这个总理Khan就这样子被人家搞下去

不要说巴基斯坦就是美国的那些法官大家已经刚刚看过

孟晚舟的事情搞了三年还有人相信司法是独立的吗

搞到国家主权国家安全的事情司法还可能独立嘛

这种事情巴基斯坦的最高法庭只要接到他们军方来电话绝对会乖乖的到他们的需求做出适当的判决

所以我今天早上听到这个消息就是很显然的中国没有打这个电话

你以前韬光养晦而且不干涉内政时间变了我们现在正在那个历史的关键时刻

你还在找借口不作为而且这不是干涉内政啊这是防止美国干涉内政

普京如果不是去年对颜色革命果断出手今年年初的时候对哈萨克斯坦的颜色革命果断出手

现在有人会管他吗你想想看现在xx会让他那个军队进出自如吗

美国已经撕下脸皮就是决心置中国跟俄国于死地

你还在坚持人家美国在你的重要盟邦邻国搞颜色革命然后你袖手不理

我真的不晓得这是什么样什么级别的乡愿也许只有马英九那个级别的乡愿才可以相提并论

我今天之所以会想要上节目就是因为胸中吞不下这口气

为什么我刚刚说过昂撒邪恶帝国已经掌控人类自从工业化以来三百年

这期间有10亿人被他们剥削残杀

很多人是真正奉献了他们的生命

我自己不才你想想看我过去这八年再写博客

在九六年以后我就有一点名气

就有很多大陆的出版商啦或者是其他的传媒公司要说给我都是一口回绝

为什么因为我要说实话而且说的是如何改进中国内政的实话

尤其是要批评那些学阀诈骗集团的话我就不能够有盈利的字眼

更进一步说我现在坐在在美国我在批评这些谎话有没有危险有危险

所以我从来不用英文写在中国发表的刊物十几篇二十几篇稿费我从来都不拿

我要说实话就不能够给他们借口

你光看美国过去这几年关了多少中国人根本其实还算是极小极小的牺牲

你看看xx今天我去看他最近现在报他详细讲述了

过去这一个月他已经很侥幸的逃脱了两次乌克兰秘密警察要抓捕他的系统

就是他两次都刚好不在房间里面有生命危险呢他现在还是躲起来躲在朋友家里面

那些人都是有理想的人也许我们并没有什么真正的贡献

说让昂撒集团现在有危险或者是中国或者俄国现在能够兴起来挑战他们

但是负责14亿人命运然后又刚好可以控制又可以影响五六十亿第三世界人民福利福祉的官员智库教授院士尸位素餐

把很明显对一些战略思维没有做出来放任官僚的惯性眼看着这个大好的机会就在面前却不愿意动手

那你光是说这是百年未有的变局有个什么屁用的的中国内部的内部不作为懒政或者是鬼混

混的那些钱不是要点但是他如果妨碍国家在这个关键要点做出正确的政策的话

那就是真正人类历史上的罪人

我一个一介布衣平民我头还拼命地想要解释把这些事实真相然后好的建议解释出来

这是他们的职业责任这么多人没有一个能够做出正确的建议真的是让我很生气

史东()

我也不知道该怎么怎么劝你因为我觉得你不需要劝,我只是觉得很珍惜一个在今天这个世界中有一个能够之内是那么推心置腹把心掏出来说话,把心掏出来做事的人我觉得是越来越难得了越来越少见了

王孟源()

这是我的感觉我本来就完全不是求我个人的名利

你不认为xx做那些视频是为了要讨那些赏钱吗不是的

对为啥成这个样子我愿意同流合污的话早就找了一个那个国际传媒公司去当记者去了

二十年记者的薪水有多少他这个是真的门外面就有秘密警察的到处要找他了

史东()

这玩意就是英雄了

王孟源()

但是至少我敢直面批评项王贻芳潘建伟或中科大那一票来帮

这些侵占国家公益的诈骗集团中国

过去这三四十年因为昂撒英美的那一套有很多人迷失了

很多精英迷失了他们真的相信了美国那套自私自利以邻为壑得那套哲学

他们那个那是昂萨霸权殖民帝国的核心你相信那个有什么用人家不会让你去当那个世界霸主的的幕后权力阶级

人家只是想要把你骗来背叛你的自己的国家

外国人都能够醒悟

我觉得这个中国过去这三四十年教育的失败非常的严重

除了像台湾香港这种系统化的腐朽之外

他这种是被非官方管道渗透非常非常严重真的是要好好的检讨

史东()

这是一个相当厉害的手段呢你不得不同意这是非常非常厉害非常有效的手段而且历史上也证明呢是一个成功的手段

王孟源()

你光是习近平一个人不够他不可能面面俱到

我自己当过十几二十年的主管我知道我手下就是20个人我都没办法全职面丁

正因为我当过主管所以我其实对行政的时候要斟酌各方各面的考虑那个困难非常理解

今天才又有人在我的博客问我说这个上海的防疫措施是不是太过分了是不是应该放松了

我跟他说这种事情呢除非你明确的证据说政府的决策是错了你应该信任他

因为他们才有那个视角才有那个资料才有那个全面的考虑

今天在巴基斯坦这件事情上我我可以证明他们渎职了所以我才这么生气

喜欢在我博客看留言讨论的人应该知道我一般的哲学就是我刚刚讲的

除非你有证据你不能够指控政府的决策错误但是现在这是证据确凿

史东()

你觉得以后应该要走什么样子的路用什么样的方法,我指的这个问题并不止这是巴基斯坦这件事情,包括了巴基斯坦以及所谓的整个的这个中国未来的外交格局,俄国和乌克兰战争结束之后的外交的新格局中国应该有什么样的盘算

王孟源()

当务之急很显然的这个全球化已经结束了

这个世界已经要分裂这时候你在通货膨胀弥漫全球的时候你的重点是争取最多的能源跟资源出产国

你要知道昂撒那个集团他们凭借着真的种族灭绝占领了美国加拿大澳洲纽西兰有这些都是资源丰富人口不多的国家

他们完全可以自给自足

所以这一次这样乱搞真正倒霉的是欧盟而已

昂撒你即使分裂也没有什么太大的问题他们除了即使是石油加拿大也是世界上主要的石油出口国之一自给自足

中国在当前的这个紧要关头是抓紧所有还没有跟进昂撒的资源出口国要做这一点

你第一个要给他们一个货币上的替代让他们能够独立第二个要给他们安全的保障都是很简单的事情

我真的不懂为什么还要轮到我一个人在家康州的人来解释看得再远一点

如果中国成功的熬过这一关通过这个分水岭世界成为两个集团集团之间的竞争

一定是靠科技发展作为主战场这个年所发生的俄乌冲突只是高度的加速这个过程

我老早就可以预言这会是在2030年代最重要的一节但现在至少提早了五年

就是在2025年之后科技的发展速度跟效率就是两个集团竞争最重要的决定因素

所以我记得1月的时候或者是12月的时候上你的节目批评那个量子计算

然后我过去这几年从大对撞机开始然后批评氢经济批评核聚变

这些都是骗子计划要谈效率第一个是要先消除骗子因为效率最低的投资就是把把资源被骗走

那是最低最低的都是负的负百分之百的的投资而且事实上可能超过百分百因为连带的大家有样学样上行下效

你如果最成功的吃香喝辣的院士都是靠骗那还有谁愿意做正经话

这些事情都是我在过去十几二十年都已经知道的事情

而且从博客一开始就已经开始做准备一步一步的解释给读者听

我们现在只不过是因为世界急转直下所以我列出更多的细节讨论

不在一个小时前我跟你讲到我原本没有预计Putin会马上出手搞军事行动

是因为他真的不是最优解你看他事后不定也会也是后悔

因为即使他运气很好这个整个第三世界还有中国印度通通的在支持他

你说他如果能够重来的话会会是这样搞吗我想至少在细节上会有更正

我刚刚讲了半天也中国的这些幕僚这些智库不像样不像话他们的思路应该是什么应该是量化的几率预估

我举个例子好了现在有一个可信的研究跟你说这个地方会发生大地震在十年之内有一半的机会发生大地震

每个礼拜有0.1percent的几率会发生大地震

然后呢现在有大家恐慌了啊整个有些有些人在媒体这个吵闹说马上要大地震了大家赶快都呃迁移

然后一个正确的分析就是说有99.9\%的机会在一个礼拜之内不会发生所以我们应该进一步的研究观察

与此同时大家应该了解地震来的时候你要怎么办先做好准备但是先不要panic先不要恐慌

其实就是我在这次站前所作的我说开战几率到达不大

但是要不要做准备要做什么准备做金融方面的准备做宣传方面的准备做外交方面的准备

这些都是我其实不是两三个月前讲的过去十年的八年在博客一遍又一遍的再讲做做该做的准备

但是我们先再次就是当好大家闹了半天说会打结果真的打了他是一个小几率事情所以才叫做黑天鹅吗几率的事情

你可以不预期但是你必须要做准备哦我在过去八年一直慢慢地在培养自己的公信力

我觉得我还没有培养到足够的公信力所以这一次你看这个巴基斯坦的事情之后我除了生气之外

我自己也是懊恼就是因为它发生的太快了我八年的努力

还没有让自己的公信力提升到我所作的预测能够让足够的人相信的地步

对于这次这件事情我是愤怒而且懊恼

因为这些事情都是未经事先建议过要做准备

然后事情发生之后给出正确的回应细节然后呢你只对不起你这个俄国和乌克兰的事情还是一次就

我指的是巴基斯坦的事情我这个我当然知道我我所写的都是用中文写的普京不可能听到

史东()

那不一定的

王孟源()

美国拜登这次为了药治恶果于死地光是切断Swift原本以为大家他说是什么核弹核弹级的制裁其实不是

真正最过分的是扣押了外汇储备

所以你赶快给他一个通路还有什么新的货币是自己的货币更理想

我的货币是其中的10\%但是60\%是人民币

人民币未来最坚挺的东西然后呢跟骗他不能浮动

你说我要换一台千亿美元的外汇储备换成人民币方便吗不方便

因为他不能浮动不能个自由汇兑但是呢做成一个合成货币一个篮子货币皆大欢喜

我是希望在这个转折点之下呢中国能够争取到越多的资源出口国越好

作为盟友至少作为朋友

虽然我这一次上来抱怨的是那些外交国安方面的但是我觉得真正难搞影响恶劣的是那些搞科技的骗子

中国在这次是开始落后了

我建议的那个亚元其实是为了这当前中国的需要贴身设计

根本就不是什么区域货币

他只是挂着那个表象是一个方便人民币跟卢布马上结合起来

然后可以进一步用来邀请中东产油国入伙的

史东()

这个概念和就前几年中国和日本和韩国谈的那个亚元的概念完全不一样

王孟源()

你跟他们合作那就那欧元似的合作这个绝对不是这完全就是我的想法就是

中国在里面60\%股份然后俄国20\%然后剩下的20\%给那些产油国分一分

但是基本上中国是占绝对优势然后呢你还是有汇率跟财政的绝对自由

但实际上的行政管理由中国协同大家合作完全走中国主导这个其实

我在我博客上都解释过了就是这其实是一个一贯的套路是用来收拢资源出口国相册国还有中东那些国家的一个手段

但是实际上中国还是牢牢的掌控人自己的主权

你如果要是搞欧元那样子这个不合适

尤其是欧洲的那些国家他们其实在宗教社会经济发展程度上都已经很相似了结果还是水土不服

你搞了欧元之后十年就水土不服

中国跟韩元日元根本没有合作的必要而且他们在军事外交上也是美国的附庸

没有争取他们的必要也没有争取上的可能

史东()

关于你你谈的这个亚元

王孟源()

只是随便找个名字就你你叫什么名字都可以

史东()

适用您用什么方法来吸引这些国家加入换句话说这些东西对他们会有什么样的好处

王孟源()

就是你给他们股份

史东()

这是他们自己的货币取代他们自己本国货币吗

王孟源()

不是不是不是是一个篮子货币占20\%就代表你贡献你自己的货币然后这个货币的量相当于20\%的总量

然后你的这个货币成为他的他这个亚元银行的储备

亚元的架子就来自于来自于这些其他货币的储备

那这很这样一来很显然哪一家的gdp最大哪一家就主宰啊很简单先天就是中国会处于主宰定位而且

这时候人民币要继续搞不对外浮动完全没有问题你根本不必搞怎么金融开放把那些诶华尔街呀引狼入室

根本没有必要你这个金币人民币之所以不合适开放

因为美国在金融掠夺当面经验丰富连英国算在一起有100多年的经验将近两百年的经验

你监管单位根本就摸不清楚他们的套路等到人家都已经捐款走掉你都还不晓得是怎么发生的

我个人认为啊这个人民币不适合呃对外自由兑换但你如果不要自由兑换你怎么变成国际储备货币呢

唯一的路子就是搞一个篮子货币然后你刚好就是其中最大的一点

史东()

要么这个货币它存在的目的就是专门为了做国际之间的交流和贸易

王孟源()

对只是国际大宗贸易

货币的话他有好几个功能啊第一个是交易第二个是订单叫第三个是储备

储备才是最重要的

现在当务之急的金融上面

我只要讨论到这个货币的事我只要去看一些细节就知道这个作者懂不懂还是是真懂还是假懂

比如说你说到美国美元的额分我说60\%有人说40\%

60\%是外汇储备来算40\%是以交易来算

现在当前这个美元的霸权重点是什么

是储备不是交易交易只是次要的所以那些去谈交易40\%的你一看就知道这个作者不入流只是读死书

真正现在美国通货膨胀这个弱点你要真正打击他就是要消弱他所占储备外汇储备的百分比

每一个百分比消弱就是第三世界国家几千亿几千亿的钱不再用美元吗就是他们要卖出美元买新的的比如说是亚元

卖出美元的意思是什么这些美元就回到美国去了美国已经通货膨胀最不希望的就是这些美元这样子像洪水一样的冲回来

美国拜登这一次为了药治恶果此地光是切断所以原本以为大家说是什么核弹核弹级的制裁

其实不是真正最过分的是扣押了外汇储备

不要说是外汇储备算是所谓的主权资产

你就算是私有资产过去三百年的昂撒宣传说的是什么

宣传说的是什么多得是财产权至高无上现在都拆穿了

像沙特阿拉伯几千亿的外汇储备瑞士将近一万亿的外汇储备中国三万多亿

只要美国不高兴随时可以一声制裁马上就把你弄弄

没收那些财主更加不用说了外国的那些寡头给他们的那个游艇啊生意啊

咋样说没收就没收你这样子还有谁敢把你资产换成美元不走这是天赐良机我说天赐良机

就是这个世界上所有的主权资产还有富人资产都在急着想办法从美元跟欧元换出来

所以你赶快给他们一个通路还有什么新的货币是自己的货币更理想

我的货币是其中的10\%但是呢60\%人民币

人民币是未来最坚挺的东西然后呢跟他不能浮动

你说我要换1000亿美元的外汇储备换成人民币方便吗不方便因为他不能浮动不能个自由汇兑

但是呢做成一个合成货币一个篮子货币皆大欢喜

这些事情我都是已经千思万虑考虑过了这绝对是远远的最优解

而且当前真的需要是非常急迫的真的是犹豫不来

所以我您要做这些事情你说我今天讨论的这个巴基斯坦

这件事情这是等于是间接的对中东示范中国或者二国可以提供了安全保障

实际上的目的是要求他们在金融跟贸易商站到你这边来但是你如果不提供安全保障的话人家怎么会愿意

他们已经有这个动机了因为为什么美国主动去扣押大家的资产私有跟公有的资产对没有人会安心把钱放在你

现在去建立亚元这种我承认最佳最佳的时机

但是因为时间紧迫你必须要我计划有有理解的好好的去做高效的去做

然后我就是没有看到他们在搞笑的做

而且我那一篇文章已经出来六个礼拜了真的没有想到

这个14亿人所谓的精英竟然没办法看出人家已经点名的震荡还是不愿意去做真的是让人很失望啊

我这样子一来所谓的学术改革增进中国科技研发的效率那更加是缘木求鱼更加是不可能啊对不对

这么简单这么明显这么基本的东西都做不到

我知道视频不是讨论这些专业一级的最好的通道

所以我一般都是只在我的博文上面才做解释

最后再讲讲我我一起今年ok接下来会发生什么就是乌克兰战事

俄国现在已经把包围威胁基辅的那5万人撤回来了

然后撤回到xx的东边也就是他准备要包围那个东乌集团的北线

然后Mariupol基本上快要告一段落就是只剩下一小部分的人员留下来收尾

让原本有4万人在包围那个城市现在至少有3万人可以抽调出来

那就可以从南线那基本上他原本是这样的样子包围着那个东乌集团

现在就可以那5万人这样下来3万人这样上来

然后这大概需要一到两个月可以完成包围

然后接下后来很不幸的是不一定马上结束

这样打下去然后很快的法国马上要选举了

然后呃变成勒庞和马克龙竞选这个也是一个一大变数

我原本建议说去年去年秋天就是半年多前我就已经说默克尔一退休

如果绿党执政的话这个德国就不可靠了必须要找法国做交涉

那现在我们就只好等法国交涉完毕

不过过去这个月昂萨媒体的拱火还有

这个neo-nazi很多离谱的行径这个事件恶化的比任何人预期的要快

即使马克龙当选他是不是还有能力跟意愿来来调停已经很成问题

或许要我们也必须要等到德国的政府垮台就是绿党他被踢出政府

我们才能够有真正的决断这个就很可能要拖到秋天

就是第一个俄军吃下这个东乌也在军团如果他们是在五月合围的话啊

然后他在死战那等到打完大概都快要七月了然后打完了以后还要去找我

史东()

我这么看我这么解读你的讲法就是至少在明在今年冬天以前这个事情会告一段落

王孟源()

我希望如此吧因为你如果能够在七月把这个东乌军团解决的话

接下来要打xxx是不定至少要把xxx打下来

我觉得是因为我啊再见你们刚开始的时候我也提到neonazi所作的残杀战俘这件事情是很有很严重的后果

另外一个长期的后果就是我想现在俄国人人喊杀

所以原本的那个我上次上节目讨论那个不仅开的那六条条件恐怕是不够

恐怕是真的要把那个乌克兰南线的沿海地带全部吃下来

史东()

你觉得他会不会把乌克兰全部吃下来

王孟源()

可能不会吧不会吧因为就是可能要打到明年

因为他这个不是跟一个普通的国家军队打仗

不是你把他包围然后别忘了眼看着就是歼灭了他们就投降不投降

因为那些亚速营的人知道他们投降也会死

我上次上节目说大概还要两三礼拜现在已经快要三个礼拜了对真的就是打闹件在他们还有一千人在那一边死守

真的是准备要战死你如果每一个每一个师都是这样子的话那你这个这真的可以打好几年

最快也要到七月才能够解决东乌野战兵团

然后在那之后才能能去打xx打完我xx以后你至少要威胁一下基辅吧才能够逼那个只能给签合约吧

史东()

就是我觉得难的是什么从我的角度他的困难就是不管是乌克兰的国内还有外部的影响都不希望他弹劾。没有什么影响力嘛现在对不对对美国不想要弹劾

王孟源()

美国不小南河对现在的局势就是欧洲人太笨,其实那些精英太笨了,美国人随英国跟美国随便造个谣他们就热血沸腾跳到前面去自杀很可笑好

史东()

这个事情我们还是当然了还是那句话继续观察着必须看见继续观察下去

王孟源()

我是希望在这个转折点之下中国能够争取到越多的资源出口国越好作为盟友哦至少作为朋友

再用几年好好的整顿自己内部的学术界吧虽然我这次上来抱怨的是那些外交国安方面的

但是我觉得真正难搞影响恶劣的是那些搞科技的骗子是科技方面最严重因为你长期下来接下来这个阶段这个历史阶段

我已经说过欧洲是最衰弱的世界分裂以后昂撒集团没有没有太大的问题

就是他们的问题都只是他们自己制造的没有什么克服不了的

他要资源有资源要技术有技术对中国还是要必须要跟昂撒集团竞争很长的一段时间

竞争期间呢这个科技发展就变成啊关键对我才会在过去这几年一直很关心中国这个学术腐败的问题的问题

\twocolumn[\begin{@twocolumnfalse}
\section{莫斯科号、马里乌波尔、李显龙谈话、民主霸权、}
\subsection{20220415}
\end{@twocolumnfalse}]准备部分省略:





龙行天下。



唐湘龙:00:44 

好的,今天星期五的时间,欢迎来到龙行天下,我是唐湘龙,我在台湾,我在台北。你现在收看的观点平台,那星期五的九点半到十点半钟的时间,龙行天下的单元。呃,我从昨天的,不管是在我的其他的节目当中,我都已经预告了,今天的龙行天下,因为因为要安排,不太容易。好,那终于联络上了王孟源,然后王博士。孟源兄,有关于孟源兄的一些的生平简历,我们待会儿会附在后面,我就不浪费时间。



那我相信的喜欢王孟源很多人,大概背景跟我一样。不是你认识王孟源,不是因为你跟王孟源共事过,不是你认同王孟源的每件事情讲法,可是你会知道他在许多的议题当中呢,他有他自己的叙事的观点,以及他自己的背景,可以去支撑的那个可能是我们一般人不太容易思考到的面向。那这个星期的时间我的我的制作团队把……因为王孟源的大部分时间他不在台湾,所以呢,我们必须要透过越洋电话连线。所以你看到我除了麦克风之外,我带着耳机。好,接下去的一个小时的时间,跟王孟源,我会谨守我们的观众朋友们对我的提醒:就唐湘龙你讲话,今天的全部都给王孟源。好,没有问题。来,现在在我们线上的人在美国的王孟源,博士欢迎。



王孟源:02:09 

大家好,非常高兴你的节目。



唐湘龙:02:11 

孟源兄呢,他出生在台湾台南。年纪呢,其实我们年纪差不多。不过他大部分时间其实都在台湾以外,在海外,在美国念;在台湾图土生土长,在美国念书求学,虽然念的是理工科的背景,不过之后进到了国际的金融圈子里面。所以你在看事情的角度上面,跟传统的、纯金融的,或者像我念政治的,其实呢,有很多的视角跟你整体过去的观点呢,是不太一样的。



我有很多的问题要请教王孟源。第一个,我从最新的问题看起:这次的俄乌战争,在昨天呢,有一个戏剧性的转折,就是俄罗斯的黑海舰队的莫斯科号。莫斯科,因为他叫莫斯科,他又是那黑海舰队的旗舰,昨天俄罗斯的方面说呢,发生了爆炸,那船没有沉,在海上还在那漂浮的状态。但是呢,乌克兰的叙事观点说不,那个是我打的,那我发射了两枚的海王星导弹的命中了这个莫斯科号。他因此就是宣称的是自己的战果。我们不管这是怎么造成,总而言之,这艘号称黑海舰队的旗舰的莫斯科号,受到了重创,暂时失去战力,这个没有问题;但是如果真的是真的是乌克兰的导弹做的,难免大家就开始怀疑就是俄罗斯的传统,传统的武力当中的反导防空的系统,是不是有很大的问题?如果你连你自己黑海舰队的旗舰店都保护不了,那你还能干什么?你怎么看这件事。



王孟源:03:43 

哦,首先我对军事也很有兴趣,因为我当年在金门服役的时候就是一120师的师部的反情报官。我的责任就是,专门读当时从共军那边拿来的他们的内部的报告,然后我来写报告,然后上呈。所以这样写了一年多,从那开始我一直对军事都很有兴趣。



这次这个……首先,这个莫斯科号,它是光荣级,他是一个常规动力的巡洋舰,就是他并不是核动力的。然后他的主要任务是用来远程袭击美国的航空母舰舰队,所以它的主要武器是16枚超高音速的反舰飞弹。那很大啊,这个比现代的反舰飞弹要大得多,这好几吨那。所以虽然它是吨位很大,但是它只能够带16枚。而且他的速度很快,这个是三倍音速的超音速导弹。但是现在的话呢,他就没有太大用的。因为现在的,你看现在不管哪一个海军强国所推出的新舰艇,他都是要兼顾对地攻击,所以都会有那个巡弋飞弹或者大陆说是巡航导弹:就是对固定陆地目标攻击的;这个莫斯科号并没有。他这一次,还会保留下来参战,因为她已经40年的历史了然后在苏联解体之后,那一段时间保养的也非常差,这件事大家都知道的就是他们有十几二十年经济很糟糕,根本没办法。所以照理说40年的老舰连美国都是不要了。



唐湘龙:05:44 

都淘汰了,早就卖给台湾了。





王孟源: 05:45 

对,然后他们光荣级有三艘,现在分别是他们北海舰队,黑海舰队跟远东舰队……就是太平洋舰队的旗舰。这其中,就是莫斯科号没有更新,就是他完全没有现代化。所以它原本就是啊充充场面。不过这次因为打乌克兰之所以还把它弄上前线了,是因为它刚好装了那个s300的海军型飞弹。所以它是其实是为其他的海军舰队,采用电子做防空掩护用的。



唐湘龙:06:25 

那就更难理解了!我说那那这就更难理解了,他有他有防空能力啊?结果导弹还被打到。



王孟源: 06:34 

本身就是黑海舰队防空能力最强的舰只。你也知道s300的那个防空能力的是远远超过150公里的,所以他根本不需要冲在前面。所以他们是在那个Odessa的外海,是其他的舰只在前面,这个莫斯科号是在最后面。然后再加上乌克兰的国防部,我还没有看到他们战报是说实话的历史啊,就是从来没有看他们故意说实话。可能有碰巧刚好说到实话,但是没有看到他们故意的。所以这件事情我非常高度存疑。因为这个所谓的海王星导弹,其实也是就是50年前那一代的反舰导弹,非常容易拦截,对s300来说是小菜一碟。相对的莫斯科号本身它因为年代久远,而且保养的很差,所以他说在这样子勉强出勤的过程中,出了火灾或爆炸反而是比较情理之中。



那另外一个观点你必须要参考的就是,这次没乌克兰为什么在战报上敢这样子胡说八道,他们至少撒谎了几百次,一个很大的原因就是这个西方的媒体,尤其是英国跟美国的媒体,有意的帮他们传播。就是即使,他们撒谎撒得不够离谱。这些因为媒体会转过头来再加油填筑,例如昨天,在Mariupol的那个亚速营,他们发了一个消息,说是他们受到化学攻击了。这个化学攻击他们怎么讲的呢?它原始的资讯是说:我们闻到一个怪味,然后之后有用人觉得有点不舒服。哦然后这就是化学攻击:就是我闻到一点味道。他们已经被关到地窖里面去了,根本没有通风。



唐湘龙: 08:40 

他说他闻到味道。



王孟源:08:44 

没有电,没有通风,没有水,然后呃,在那边不晓得吃喝拉撒是怎么搞的?已经有已经关了一个多月了,你说他闻到一点怪味,有点不舒服,这样就算化学攻击。结果,马上,英国的媒体全部都加油添醋说这是证实是俄国的化学攻击,然后呢,英国的外相,如果说这是媒体的话,我还可以说这就是假新闻?英国的外相就出来指手画脚了,说俄国化学攻击了我们要加码。



哦。我认为在这出闹剧,现在Putin已经把它正式称为谎言帝国,我觉得是很好的。大家如果真的要看一点真相就是英美方的真相的话呢,最好是去看美国国防部。这个美国国防部的发言呢,都很有味道啊。这个比如两两个礼拜前他们有那个所谓的Bucha。



唐湘龙:09:51

布查事件的调查。,



王孟源:09:52 



Bucha惨案,然后那个英国的外交部跟美国的国务院都出来指手画脚的破口大骂,说他们是怎么种族灭绝,要求制裁的。但是你如果是去看美国国防部的发言的话,就很有味道,他说目前的统计数字是当地死亡了320人,包括军人和平民,未能确定如何分布。但根据过去一个月在当地的战况,这是完全正常的。完全正常,正常的交战结果。他的这个意思就是说,我们是职业军人,我们有道德,我们不能够撒谎;我们在国务院的那些同僚,他们说2+2不等于四。不过我们不能反驳,但是我可以告诉你,这个答案是比三大,但是比五小的那个数字。他这个说法就是这个意思,就是我们不能够反驳国务院,但是他们明显是在撒谎。



所以这次这个莫斯科号。被击中了以后,双方各执一词,你如果去看美国国防部的评论的话也是很有意思。他是说,我们认为这个火灾引起的几率比较大。就是,我想他也是专业人员,他能作为这个导弹没办法突破这个光荣级S300的防空圈。而且。这种老式的导弹根本就没有什么……你想想看70年代的那个电脑是什么程度?他不可能识别哪一个目标是真正要打的。就是到了目标地点,打开雷达寻标以后,看到哪里有舰只,就第一个舰只就先打过去。你看当年台湾那个从高雄出了意外,不是就打到一艘渔船吗?很小的船。既然莫斯科号是旗舰,而且是在后面前面有一大堆小的5000吨级护卫舰,怎么会可能打得到呢?前面那些小舰应该是先遭殃嘛,对不对?所以美国国防部也是这么说,我个人也认为是这样。



不过很不幸的是,因为现在黑海那边天气不好,所以他们试图把这个船拖回去的路上,他沉了。我们可能永远公说公有理婆说婆有理。不过这件事情一个比较简单的办法就是看它的伤亡数字,,因为它如果是火灾的话呢,伤亡应该是顶多几十个人。只如果是那种重型的,四五十年前的反舰导弹,当然现在看的很落伍,但是它有一个比较强的地方,就是它超大。他的那个弹头比现在的弹头要大,现在的反舰导弹的。那个弹头大一点的话也是四五百公斤,但是四五十年前的导弹都是几吨级的。所以如果是中了一颗那样的导弹的话,伤亡数就不可能小于100,所以光这一点就可以分析。不过,现在有这个数据的人是俄军,你去问俄军,不管俄军愿不愿意跟你讲,也没办法确实相信。不过相对乌克兰国防部胡说八道,我到目前为止看到俄国的国防部所发的公布的,我还没有看到一次,我认为是故意撒谎的。就是他们基本上不评论自己的战损,他们自己己损伤,伤亡伤亡多少损失,装甲车损失飞机什么的,他们都不评论。不评论的意思就是我不愿意撒谎。因为这种战果呢?自己是最清楚的。敌方来看都是有所谓的战争迷雾,那我没有必要为敌方来拨开这个战争迷雾。但是,我自己认定我对敌方有什么杀伤的话,他愿意讲。他讲的话都是很合理的数字。



唐湘龙: 14:29 

延续这个问题,你觉得这场的战争,因为现在在乌东地区顿巴斯地区,那俄罗斯跟乌克兰的军队现在都做大规模的集结。你觉得这会是一场决战吗?马里乌波尔的战事就已经接近收尾了,但是这次的莫斯科号,会让大家想/怀疑,就是说那接下去战争有没有可能往乌克兰的黑海区包括敖德萨,这些地区扩大。这场的战争短时间之内有结束的可能吗?



王孟源:14:59 

没有。事实上,在开战之前我一直说,我不预期Putin会开战。因为从战略跟战术来看,他都没有准备好,所以他的确是开战是很莽撞。所以当初2月24号,他动手的时候我自己也吃了一惊,就是他并没有准备好。所以我。这个用逻辑简单演绎就知道他必然会吃亏。但是这个吃亏并不代表他会打败。吃亏的意思就是说他这个胜利会不是那么轻松,而且需要更多的时间。他原本的用意,我现在回头来看呢,可以看出,他不是在军事上轻敌,而是在政治上轻敌。就是他以为因为攻击的地方主要是东部跟南部的俄语区,就是当地的民意,其实是支持俄国。所以他以为只要把机械化部队小部队冲到前面;把当地的政客或者是还有那个乌克兰的中央政府吓了一跳,他们就会投降或者是有地方政府会起义。



而事实上,因为乌克兰在过去这八年有很多新纳粹,所谓的Neo Nazi,他不但是有亚速营,另外还有好几个类似的组织。他们不但扩大成为旅级跟师级的组织,而且把他们的人员安插到所有的正规国防军里面,所以这些国防军都是必须要死在……你如果说要投降的话,他们这些人先拿先一枪把你毙了。然后在地方政府也是这样子,当时在南部的俄鱼区,有两个市长是很有名的亲俄派,在开战的头两天就被人家毙了。就是他们没有做什么事情,就因为他们有这个亲俄的背景,那这样子下来,还有谁敢说要投降,对不对?



所以,你可以看出他们的计划是要用很少的兵力,就是即使到现在,他们全部进入乌克兰的兵力也不超过19万,但是乌克兰在2月24号的时候已经有26万的现役军人,这就是还没有动员后备役的时候已经有26万。你进攻方呢?应照理说应该有3比1的优势。那他结果她连1比1都做不到,那他的部署是什么呢?就是派了大约是五万的精兵,这刚好就是他们的所谓的空降部队,还有他们的装甲师。你如果看到说俄军用到了,空降部队跟装甲师的,就是用在基辅前线。这五万人送到基辅前线,但基辅是一个300万人的城市,他有很大的市郊区。那然后当地又有很多的后备役,所以你这五万人,其实有军事常识的人都知道,不足以强攻击基辅这么大的一个城市。如果你威胁着要把他包围,然后这基辅下吓坏了主动投降,那是另外一回事啊。但是如果他们不投降,然后准备要跟你打巷战的话,五万人是绝对不够的。一个300万的城市,你要五万人进去大打巷战是不够的。但是呢,他们的战略就是我们先试图把他们吓坏,就是震慑他们,用速度来震慑他们,然后看他们会不会因为作战意志不够,而很快的投降。与此同时,当然也会要留了后手就是万一不成功的话,你也不能够说,然后就有一个大烂摊子。



但是你19万人。要攻打一个这么大的国家,那个国土……乌克兰其实是一个相当大的国家啊,只是因为跟俄国在一起,所以相形之下看起来像是小小的。其实它的国土比德国大,比法国大。你19万人去攻打这个,然后对方的正规军人已经比你多了,那然后你又想要投机,看看能不能够政治解决?那这时候唯一合理的部署,也就是它们真正做出的部署。他们有五万的军人投入基辅前线,那这个围攻的话呢,当然硬攻是不够的;但是围攻的话,至少就吸引了所有乌克兰中部跟西部的所有的后备役,跟二线跟三线的部队,牵制了十几万人。就是反过来乌克兰的变成3比1、4比一在基辅对峙。然后呢,还有四万人是从那个Crimea,就是南边的克里米亚半岛去进攻。进攻的目标是Odessa,那这个也是佯攻。然后这样子一来,他才在东乌拥有局部优势,很勉强的局部优势,也就是是还不到1.5比1的局部优势。那不过,东乌的这些乌克兰部队总共八万人,是乌克兰的最精锐的部队,也是最狠的部队。就是包括亚速营这些,还有他们的海军战队。比如说在马里乌波尔,昨天刚刚全部投降的36旅,那只海军陆战队。然后剩下的就是亚速营。



唐湘龙:21:34 

亚速营还在那个洞里面。



王孟源:21:37 

对,亚速营其实已经发展到快要算是师级了。



唐湘龙:21:40 

没错,它他不是营啊,他不是营级单位。



王孟源: 21:44 

八年前刚开始的时候还是一个营,现在已经是师级部队了。所以,仗打开来的时候呢,他们对那个东乌的这个野战兵团呢,他也是分成两半两部分,因为一部分在北边就是有六万人。这半部分也是保持接触,就是继续的攻击,但是跟它对峙的那个俄军呢,也大致只有六万人。所以大概1比1,他并没有再强攻,就是是做小规模的压迫,要让他们不能够脱离接触。真正的压力,就是过去这50天真正的压力,唯一全力在打,就是马里乌波尔。在马里乌波尔,他们的确形成了3比1的优势,就是用四万人去打一一万四千人。



很有意思的,大概打到上个礼拜一个礼拜前,他们大势已去,然后乌克兰派了一个直升机中队连续两天要去撤离。第一天,据说是一共六架,第一天六架去,一架被打下来;第二天,五架再去又被打掉两架;最后三架,但是你可以想象他们还是撤离了很多人!撤离的是什么呢?目前有很多谣言说是北约的军官,或者是你美国的军官。都是谣言啊,现在完全没有证据啊,大家不要传谣。事实上,我认为基本上就是亚速营的高级军官,亚速营他们不但控制了国防军,而且也控制了所有的警察跟公安单位。比如说基辅的警察总长就是亚述营的总参谋长。他兼基辅的那个警察总长。所以很有意思的是昨天,一开始是一个礼拜前,三十六旅有两百六十几个人投降,这是第一次有成百的投降,那其实是一个营打剩下来的。就是那个旅,其中的一个营打到剩下这样。然后昨天跟前天又有1000多人,然后今天又有100多人都是三十六旅剩下全部了,就是它原本五六千人,打到剩下1000多了,不到2000了!那全部投降。投降之后,俄军公布了这个消息,接下来乌克兰国防部就发布了一个视频。是马里乌波尔的亚述营的总指挥官,还有36旅的旅长,两个人坐在那里,指天画地,说他们还在抵抗。



唐湘龙:24:55 

没错,就是好像我们还在现场,没有人投降!投降都是骗人的,我们在继续的奋斗。



王孟源:25:02 

对,但是大家简单一看,就是这两个人不但有水有电,而且吃的很好,而且连那个睡觉都很够。就是完全没有选择是黑眼睛,烫的笔直,光亮的就是刚刚洗好的军服。我想反正有点脑袋的人应该都可以看得出来他们就是一个多礼拜前被直升机撤离的第一批嘛。因为这次俄军宣布那个36旅投降的时候,他们的最高的军官就是他们的副旅长。所以很显然这旅长跑掉了剩下副旅长了,然后副队长的投降了。



唐湘龙:25:44 

那您判断就是说这场的乌东的决战会开打。



王孟源:25:50 

哦,原本就是计划这样的,就是说她们他们的原本的计划。就是我们这样打:一开始就是很快的用小部队穿插去占领交通要道,如果对方愿意投降,我们双方都以极轻的伤亡就结束这个冲突,然后俄方的条件也是公开的谈,最基本的就是他们必须要立宪保证永远中立,然后承认东乌跟克里米亚一样,放弃的那些地方,然后修宪要法办那些纳粹。但是那个是当时的战略目标,然后后来,这个战略目标当然没有达成,因为这个乌克兰国防军的作战意志远超俄国的估计。所以就只好接着硬打下去。硬打下去的那个战略,你用19万去打20多万,就是佯攻对方的首都,然后包抄对方的野战部队,包抄对方野战部队你又没有3比1,你只好先吃掉最重要的那一小部分就是马里乌波尔。现在快要吃完了,大概再一两天亚速营那些人就……他们已经把消防车都调过去,要把那个河水……因为马里乌波尔有一条河过去,然后那个他们现在藏起来的那个钢铁厂是乌克兰最大的钢铁厂,它在河边。



唐湘龙:

准备用灌水。



王孟源:

对准备用灌水,他们都是躲到地下去的。



唐湘龙:27:39 

在我们台湾呢,我想孟源听得懂,叫做灌蟋蟀(音:Daogao),你用灌蟋蟀,蟋蟀就会从洞里面出来。就可以活捉这些人,这是最后救你的方式了,让你投降的名正言顺。



王孟源:27:55 

对,其实这些人不投降,我想俄军也不会太难过,因为36旅还是算是正规军,这些人是亚速营,收了起来反而麻烦。



唐湘龙:28:08 

没有错,本来就要歼灭它的。

王孟源:28:15 

然后他们的计划很明显的是,马里乌波尔这边吃下来以后,这四万人目前已经放出去大概三万人了,已经重新去整修部署了。然后从基辅,我们知道两个礼拜前撤下那五万人,也是整修,现在也差不多重新部署完了。而这九万的部队投进去,就足以对剩下的那六万乌东野战兵团构成接近接近3比1的包围。然后这个阶段,我预期他们北边有一个镇叫做Izyum。



唐湘龙:

嗯,伊久姆。



王孟源:

从Izyum往南打,然后从南边的那个黑海沿岸往北打,这个大概需要3到4个礼拜完成合围。完成合为以后就是,英文里面就是说Shooting Fish in the Barrel,就是在在桶子里面的鱼,你要怎么打都可以。瓮中捉鳖的意思。



唐湘龙:29:27 

好,现在就是看看,这个俄罗斯现在的部署看起来就就是一个口袋型的战术,最后要要收网的阶段了。那当这个会是一场的决战,除非乌克兰的会有投降的,会透过政治的方式去解决这场的大决战,否则这场决战……



王孟源:29:47 

哦,我跟你讲,这个已经不可能了。发生的事情是在过去这一个多月,发生了十几件……就是2014年的时候,俄军入侵的时候第一次入侵的时候,因为没有想到没有经验,所以他们的士兵都带着手机,然后就有一大堆士兵发视频,还有微博,就是Twitter那些东西。搞得很尴尬,很麻烦,一直泄露军情。所以他们这一次呢,他们自己的士兵是不能够带手机的,就是一般的士兵,不能带手机的,要有特别准许的军官才能带手机。所以,他们这一次从俄方就一直都没有这些小道消息流出来,反而是乌克兰,小道消息很多。这一方面让不用心的观众,会以为是乌克兰打的全面获胜,但是这这其实是观点嘛。你这个一场全面、强度很高的战争,你如果是选择性的,就算你的伤亡比是1比4,1比,4,但你真正真正要挑的时候,你还是可以挑得出你打胜的那些事情,几十个几百个事件拿出来挑。事实上就是这样子,但是在这个过程中,因为乌军没有强调这个纪律,所以你就看到很多他们虐俘的视频传出来。就是枪杀俘虏的,层出不穷,有十几个案件了,非常残忍,非常残忍。当然,英美的媒体是绝对不会爆的,但是在俄国,他们报的很厉害,在一报出来以后,现目前俄国真的是敌忾同仇,没有人愿意现在讲和。



我跟你讲,像两年前奥地利的总理,他叫做什么,Nehammer。对那个内哈默,这个为什么是奥地利总理呢?奥地利不是北约的成员,我想一般人不知道。相对中立,因为奥地利其实在二战后是苏联占领的。那拖了好几年之后呢,这个西方就是英美这方去跟苏联谈判,说你要这块地。太远了,对你没什么好处,你就让他,你就从那里撤军,我们也保证不让它加入北约怎么样?后来苏联同意了,同意了以后呢,所以奥地利一直是一个半中立的状态。然后他因为他是讲德语的,因为他基本上也是德国人,只不过是信天主教的德国人。所以这一次其实是德国的总理Scholz,请他去当说客。就是Scholz开始后悔了,Scholz他们的现在这个通货膨胀跟能源供应还有经济的前景已经快受不了了。他事先不知道,他以为,那些他内阁里面那些绿党的人怎么讲,他就怎么听。结果现在德国的工商的领袖开始警告他了,而且他本身是社会党的。社会党的话,他们很担心那个公会,他们公会是他们要支持者之一。那德国的工会跟美国的工会不太一样,他们跟企业家是分权,基本上是站在一致对外的。就是他们本身内部没有很强的矛盾。他们的企业结构事实上有两个董事会,有一个对执行董事会,还有一个监督董事会。这个监督董事会里面有一半就是公会的人。就是上层的那个董事会有一半是工会,所以工会其实是很支持工商业的发展的。



所以这次出了问题以后,经过几个礼拜问题呈现出来以后呢,那些工商界领袖可以很简单的跟工会领袖解释这件事情。然后工会领袖再去跟Scholz解释这件事情。这时候呢,Scholz就开始着急了,着急了以后呢,他就……然后他又已经制裁俄国制裁的太过火,这个Putin估计已经不跟他讲话了,所以他才找了Nehammer。



唐湘龙:34:37 

可是内哈默没去也没有,也没有谈出什么。



王孟源:34:41 

就是因为我刚刚讲的,因为现在俄国民气可用,而且又已经打出这些伤亡。你说乌克兰方当然夸大了,俄军伤亡的数字,但是世上俄军实际的伤亡数字,我认为大约是死亡了,阵亡2000人。那以3比1的算的话,那伤亡总共就是8000人。这还是一个很惊人、很高的数字,远比预期的要高。但是虽然伤亡这么多,他们认为是值得的,因为他们发现对方真的是比他们想象的还要恶劣。然后现在是全民喊打喊杀,就是真的是Putin即使现在要和他和谈,反而是会有政变。



唐湘龙:35:38 

OK就是好,所以所以我觉得孟源的这个这个叙事观点很重要,就是现在看起来已经和谈之窗呢,已经关起来的。除非乌克兰真的在在俄罗斯所要求之下,条件做出重大的让步,不过这个对乌克兰来讲可能也不具备这样的政治条件。美国,英国这些国家……



王孟源:36:02 

不只是这样的,现在乌克兰即使回去,说你当初开的那十几条条件,我全部照单全收。



唐湘龙:36:08 

俄罗斯也不会接受



王孟源:36:10 

你已经死伤了8000人了,现在Putin的看法是:既然如此的话,我不只要吃下东乌那两个省,我连整个南部的俄语区,就是整个沿海,我都要吃下来。



唐湘龙:36:26 

就是整个亚速海的沿岸这边的全部都要吃下来。



王孟源:36:28 

全部都要吃下来,但他又不好意思说我以前开的条件我现在收回来,总是有点不好意思。你要至少要先打下来以后再说嘛。这就是为什么Nehammer到俄罗斯以后他不但没有得到什么结果,而且俄方对他的态度很恶劣,就是故意要羞辱他;故意要不谈出什么结果来。这是很明显的,你如果看那个细节:就是他并没有……他们的会见并不是在Kremlin,而是在哦。而是在Putin的一个办公别墅里面会见,根本就不把他当一回事,就是故意去羞辱他的。故意不谈东西来。



唐湘龙:37:16 

好,我们因为这个部分的判断呢,是非常重要的,就是现在的俄乌之间的战争,看起来大家都把和谈的门,调子都拉的很高,和谈的门看起来是开着,但是基本上已经没有办法走过去了,所以他的势必就需要通过在战场上面来解决,再回头再回到谈判桌上。我来回头来关注一件事情,这事情其实今天我要跟跟王孟源访谈的重点,就是在三月底四月初的时候,东盟本来就是要跟美国开一场的峰会的。那拜登的说,“你们统统来,到美国来,我们来来办一场的10+1,美国跟东盟之间的峰会”。美国忽略东盟,忽略东南亚已经非常久了,虽然去年拜登上台了之后有比较密集的在经营东南亚,不过不看起来已经的时不我与。那这次当十个国家里面九个说我们没时间,时间凑不拢,但最后我们看到新加坡的总理李显龙去了,但他是用的工作会谈的名义去了,去了之后呢,待了几天的时间。他跟拜登的谈话大概都还好,所以看起来行李如懿,可是在李显龙在拜登之外,不管是接受华尔街日报的座谈的访问,或者回来了之后,他回到新加坡的时候,在他自己的李显龙的新加坡总理的官网上面的那个人发布的谈话,这个就让我觉得……李显龙的谈话呢有玄机?李显龙谈话两个重点,第一个就是说她在他在华尔街日报的那个座谈里面,对许多的这些的企业界所释放出来的信息认为,美国应该重新检讨1944年的Bretton Wood的之后所建立起来的全球金融秩序。这个金融秩序,因为全球的金融经济的环境改变太大了,像日本现在经济量体只有中国的1/3还不到。日本,凭什么主导整个亚洲开发银行?他的掌握最大的股权。同样的世界银行IMF都是美国在这掌控的,那中国在哪里?他说应该让中国呢在国这个金融体系里面有更大的影响力。第二个部分他提到,就是说希望不要有任何的把这场的俄乌战争的引导成民主跟非民主,民主跟独裁国家之间的战争,不要想要在这件事情上面去贴中国的标签,如果逼着中国去站队或者暗示中国站错队伍,这些接下去的风险都很高,这两件事情,第一我们看,你怎么看李显龙的这个动作?李显龙在在想什么?他所呼吁,尤其是国际的战后的金融的秩序,全球的金融体系,让中国有更多的参与,美国会放手吗?



王孟源:40:01 

是这样子,你跟Biden讲这些事情啊,一点用都没有,因为他既不懂也不管。他们是,Biden现在手下的那些官员,都是当初Hillary Clinton那一系的,比如说他的国安会主席Jake Sullivan,原本是Hillaryi的国安助理。后来,Hillary退休以后,从国务卿退休以后,刚好Biden当副总统的需要一个国安主管,所以呢,他才转任副总统的国安顾问。所以基本上它们的国安体系呢,都是Hillary那一派。那么Hillary这一派是什么呢?就是他们只懂玩外交权谋,他们的外交权谋呢就是传统的英美的那一套:用外宣来驱动炮灰,驱动你的所谓的盟友炮灰来跟你的敌人同归于尽;然后这样反复的做下去,一直到你的敌人有点实力虚弱的时候,你再继续的施压,然后骗他们,用骗或者是压迫他们来采纳自杀性的策略。这样子呢,美方毫无代价,而且可以事后呢,用金融的手段去搜刮当地的财富。



英国跟美国其实他们整个体制建立起来,就是要为巨富——不只是富人哦,而且是巨富就是我们现在说的Billionare——作为一个天堂。就是要一切都是环环绕着这些巨富的福利来设计的,他们的体制都是这样的。所以Hillary的这一套屡试不爽。你看冷战结束前的80年代,美国其实面临着两个挑战,在经济上,他面临的是日本的挑战,在军事上他面临了苏联的挑战。但是呢,在80年代末期,九零代初期他同时地解决了这两个对手。看起来手段是完全不一样,实际上是同一个套路,对不对?你看日本是怎么死的?愿意把他的日元兑美元在一年之内升值了120\%,然后放任它们的股市跟地产泡沫不断的放大,这不是他们愿意的,是美国逼他们这样做的。苏联同样的是,他们放任他们行政人员的纪律松弛,然后呢,去迷信市场化,把所有的国有资产都让廉价地开始所谓的自私有化自由化。这也是英美靠着宣传,把戈巴契夫洗了脑。都是让他们自己采用自杀性的特点。那你这样一来,美国不费一兵一卒,即使是要出点钱的,也可以跟前线的盟友要钱来补贴。



你看到后来打海湾战争的时候,要把几十万部队送到中东去打伊拉克,那是很花钱的,但是美国反倒赚了,因为他们是先跟日本啊,还有那个Saudi 阿拉伯先收了一圈钱,而且收得还很多。那这就是盎萨霸权的运行模式嘛,对不对?那所以这一次李显龙会这样做的,我首先在这个金融方面:这个在冷战结束后美国真的是他自己得意忘形,自废了武功。他把首先最重要的本身的工业就放弃掉了,就是Outsourcing。这样外流了30多年呢,他本身已经都变成虚拟化,整个经济就是金融化。那这样一来,然后社会上又有很多的腐朽,就是法律,还有行政制度等等等等,都是越贪腐的现象越来越多,越来越明显。这样一来,你还能够解决问题,最方便的办法就是多印钞票,比如说是这次新冠,出了新冠以后他们能防治吗?没有防治的能力,对不对?但是呢,它可以印钞票,印了多少?印了将近50000亿。那五万亿因为这个发行通货的时候,我第一个拿到这些钱的人,花了这些钱啊,收了这些钱呢,可以再花,嗯,这就有一个Multiplier。



唐湘龙:45:36 

乘数效应。



王孟源 45:37 

倍增效应。所以实际上对美国经济的刺激,可能是十万亿到十五万亿。



唐湘龙:

大概是三倍以上,通常3到4倍。



王孟源 

大约两三倍至少,那你看看它也是不断地发钱嘛,直接的向公民直接的发钱,呃,每个人可以拿总共拿到两三千三四千块。这个呢,就解决了民愤。对不对?他现在这个目前的经济,失业率很低了,但是还招不到人,为什么?因为,部分的原因是因为大家拿到这些福利,那他能够拿这些福利的原因就是基本上它的霸权现在只剩下美元作为支撑,然后凡是出了什么问题,反正就多印钞票来解决了。你这里面有什么种族的问题啦,然后什么宗教的问题啦,反正大家在那闹成一团的事后再多一点钞票,然后就能够把那蒙混过关压下去了。所以这一次,李显龙去讲了,其实他也是理解到这一点嗯。但是我其实听到他这样讲,我不是特别的高兴。



唐湘龙:

为什么?



王孟源:

因为他很显然的是为美国站队,他只想要劝Biden。就是他的立场,其实是在美国那一方,他只是想劝美国说你这个不要做得太过分。但事实上,所谓的过分是什么呢?就是这一次制裁俄国,因为Biden用Jake Sullivan那些人那个套路,去赶尽杀绝。那他们这个想法就是,你先用SWIFT 来制裁;然后制裁之后,干脆连那个鄂国的外汇储备也禁了。那这么一来,俄国拿不到欧元跟美元,他的这个外贸就完全……第一个外贸就完全切断,第二个卢布会断崖式的贬值,那这样一来就造成俄国的经济混乱,经济混乱以后就就可以推动了政权更替,这是他们的如意算盘。但是问题是他没有想到,这个俄国在2008年跟Georgia打一仗,已经吃了点苦头。2014年之后,又练习了一次。所以这一次,过去几年他们换了一个新的中央银行的行长,是一个很厉害的人物。



唐湘龙:

是。



王孟源:

对,你也知道是一位女士。



唐湘龙:48:36 

没错



王孟源:48:39 

他是一个很厉害的人物,所以Putin一直就是给她权力授权,叫她去做准备。所以这一次,她已经是预案都已经准备好了,所以我完全没有意外就是,她能够把这个卢布很快的又重回……



唐湘龙:

战前的水准。



王孟源:

对,基本上就是。他们到目前已经稳定了,基本上已经是立于不败之地,就是虽然有一点经济衰退,有一点点通货膨胀,但是事实上呢,他在一年到两年之内就可以恢复正成长。反而是欧洲跟美国,美国因为他本身地大物博,还不是太大的问题,其次还可以,跟澳洲加拿大进口那些东西;最惨的是欧洲,这一次,这个欧洲真的是被美国人坑惨了,它的那个经济崩溃已经是无法避免。现在的问题,我只想看到是只有德国的政府下台还是整个欧盟的结构会纷纷崩瓦解,这是外贸观察的未来的重点。



唐湘龙 49:59 

不管是欧洲,还是是德国的瓦解,或者是欧盟的瓦解,就战后我个人(认为),这场战争不管什么时候结束,结束之后可能有两个分裂。一个是乌克兰的分裂,乌克兰如果如果不割一块地,这场战争大概是没有办法收尾的;第二个就是那欧盟的分裂,欧盟的在战后大概不会是现在的欧盟,不会是过去的欧盟了,现在他也没有像样的领导人可以重振欧盟的士气。不过我们回到了刚刚的李显龙的这件事情,其实是当代的全球经济,全球金融秩序当中,它确实它跟现在的国际经济当中,中国所占有的份额不管是在金融,不管是在贸易,不管是在gdp所占的全球的经济体系,金融体系里面的份额,现在的国国际的金融的体系里面,不管世界银行IMF甚至我们在亚洲的亚洲开发银行,它显然都都是不成比例的,它不反应现状的。那李显龙龙提这件事情会让我(觉得)对呀,这个问题我们过去有关注过,终于有人提出来了。但是李显龙提的时候,最终就是美国根日本在掌控,美国把这些当做是它的核心的资源再分配给他的这些兄弟们,IMF就让欧洲的欧盟的这些官员去玩一下;那世界银行呢,就他自己抓;抓着亚洲开发银行的日本。我说了,日本现在的经济份额只有中国的1/3不到啊,可是亚洲开发银行,日本说了算,那李显龙即使这样子提了,是李小龙太天真,还是说美国真的有可能按照现在的全球的贸易,金融,经济,经济比例的比例重新能让这些呢这些战后所建立的金融体系变得比较合理,有这个可能吗?



王孟源:51:44 

完全没有可能。



唐湘龙:51:46 

美国不会放?



王孟源:51:47 

因为事实上你没收人家的外汇资产,就是3000多亿的外汇,最大的马上就颠覆了你这个美元跟欧元的既有霸权。因为这个世界的经济经过过去三四十年的全球化,有很多第三世界的国家累积了相当的财富,别的不说,至少那些产油国都是几万亿几万亿在算的。那他们的这些资产全部都是存在欧洲跟美国。那为什么呢?他们相信这个所谓的财产权的保护。就是你这个钱放在那边,美国几十年来百多年来,其实他们的经济系都教你所谓的,他美国的国债是所谓的Risk-Free Asset 就是无风险资产。但这下大家一看,你连本金都全部被没收了,这还算是无风险啊?开玩笑,这是最高风险啊,风险没有比这个更高了。那你真要想想看,为什么沙乌地阿伯会主动去跟中国讲,我要跟你换成用人民币。



唐湘龙:53:02 

去用人民币交易原油购买。



王孟源:53:05 

对,因为很简单嘛,因为这一次俄国还能够动用的那一半外汇就是人民币跟黄金。那黄金的问题是在于现在世界的经济体量太大,人类既有的黄金完全不够你拿来作为外汇储备,除非这个环境的价钱再增加十倍,但没有人愿意去赌这赌这件事情。所以呢沙乌地阿拉伯这个动作,其实只是一个很单纯的自保的反射动作,就是我需要一点人民币,要不然等我的所有的外汇都被没收了以后我什么都没有了。至少半个20\%的人民币,出了事情,我还有20\%的外汇。



唐湘龙:

没错,没错。



王孟源:

但是反过来看,既然每一个第三世界国家,你看这次大家看在眼里都怒在心里,所以其实中国有很大的出手的机会。我很喜欢读历史,我们还有五分钟,大概不能够谈这个了。



唐湘龙:54:21 

没关系你说,哈哈哈,下次再,你知道反正我在发你通告的时候,你还是要上就是了。因为我觉得我们今天根本才刚开始聊一点点嘛。你再说,没问题。



王孟源:54:33 

其实你看,这一次俄乌打仗,这个俄国人我说他一开始太轻忽了,历史上有一个完全一样的潜力,大家有空找一找,就是美国的南北战争。



唐湘龙:

为什么?



王孟源:

美国南北战争在一开打的时候,林肯也是以为你只要把那个联邦部队到边境那边秀一秀,南方人就会吓坏人,然后就投降。然后结果?打了四年,对不对?最后是把南方打烂了,才获胜。



唐湘龙:55:08 

没错没有错。



王孟源:55:09 

所以现在也是一样,Putin一开始以为,把他们吓唬一下就可以了,结果不行,现在眼看着要把乌克兰打烂。打完以后,乌克兰可能剩下不到以前一半的国土和人口。



那我们刚刚讲到这个美元的事情呢,其实这种霸权,他们这个霸权完全是继承希腊时代的霸权观点。希腊时代最有名,而且历时最久的霸权是哪一个?雅典。刚好就是他们所谓的民主制度。但是你不要以为这个所谓民主制度就是很现代的,什么很合理的。



唐湘龙:55:52 

那不是那不基本上是属于贵族的民主。



王孟源:55:56 

对,你只要去读苏格拉底的著作,你就知道他对它有多痛恨,而且事实上这些记录,都是在历史上有详细记录的,比如说他们当时雅典,就是靠他的海军到处去压榨几十个他们的那个小国小城邦,所谓的盟友呢,就是每年要贡献一大堆现金的。那如果你不愿意的话,他就过去把你打下来。打下来一个很有名的案例叫做Methylen。公元前420几年还是30年,Methylen就是在那个Lesbo岛上,就是Lesbo。就是现在我们说女同性恋叫Lesbian,就是取名自Lesbo。Methylen是Lesbo岛上最大的城邦,然后他觉得不愿意被雅典压榨,然后他起来试图反抗,结果雅典派了战舰把它打下来,打下来以后呢?所谓的民主嘛,当时一个雅典,一个城邦,大家开会开会,以后决定要把他们全部杀光,所谓的全部,就是所有的公民,公民的意思就是有有财产的男人。因为古希腊对女性的歧视是很严重的,他们那个公民是有4000多个。结果第二天雅典又后悔了,就是有人有人良心发现说我在开一次会,来复决,复决然后说不要了算了,这个只杀几百个。又送了一艘船,去说不要全部杀光杀几百个。结果传到的时候已经杀了1000,他们的公民有4000多个,杀了1/4。



然后十几年之后。有另外一个小岛叫Milos,这个Milos拒绝加入雅典的阵营,拒绝向他献金。他就派了舰队,这次连问都不问,就是事先跟他讲,你去那边就把公民全部杀光,然后女人跟小孩全部抓起来当奴隶卖掉。就把它灭了,就是一天之内把他灭了,这个就是所谓的民主霸权,也就是你现在看美国搞这个霸权。完全就是学当年雅典。



唐湘龙:58:29 

真是……当然,你刚刚讲的这个Milos其实是在爱琴海文明当中非常重要的一个岛屿,那个岛当然是观光圣地了。不过讲到了那个两三千年前历史的时候,总是令人有不胜嘘唏之感。但是,现在在我们线上的,我们现在的,在我的节目的两个视光灯当中,我是唐湘龙,我旁边的这位呢叫王孟源。好,今天的第一次呢,在在观点平台上面,在龙行天下的单元里面,你看到王孟源,我也是第一次访问他,我也是第一次看看到他。王孟源不知道信不信,因为我的节目里面啊,底下的留言板常常有人在敲碗中,王孟源,王孟源,你要找王孟源。好,今天终于找到我,王孟源,但是我们刚能这样子聊,一个小时时间都过得非常非常快。我开给王孟源的菜单,当然非常长了,后面还有很多没有谈到的,但因为时间的关系啊,时间到了,一定要在在美东时间的,现在已经是晚上的九点多的时间了。



第一个我要先感谢王孟源,然后我要再邀您再来上节目。但是呢,在节目结束之前来我把我们的一些的留言的,Donate的名单念一下。让王孟源的知道一下。





感谢Donate部分省略





王孟源:01:01:29 

我很很抱歉的,这个我在节目开始前,我也跟唐先生警告过了,就是我的坏习惯是话匣子一打开就关不住。



唐湘龙:01:01:41 

我跟你说,我根本就不是问题。



王孟源:01:01:45 

跟史东谈的时候,通常都是一个话题,然后要讲两个钟头;所以今天只有一个钟头,然后你事先给我六七个话题,我就我就知道绝对不可能。



唐湘龙:01:01:58 

没关系,我们因为其实每个话题,其实都可以。我觉得以王孟源的叙事,跟你对一些的材料,根据各种议题的准备,每个问题其实我们都给花更多的时间谈得很深很深入,下回如果若在您您在在愿意在上的龙行天下做,哪怕一一个小时,我们只谈一个话题都可以。不管怎么说今天因为在美东的时间,现在都晚上,感谢孟源,



王孟源:01:02:26 

哦,谢谢你给我这个机会跟大家聊天。



唐湘龙:01:02:28 

非常感谢,非常荣幸,好孟源谢谢谢谢,们下回了,今天的龙行天下,能就进行到这地方。\twocolumn[\begin{@twocolumnfalse}
\section{中国能否渡过全球经济危机?}
\subsection{20220520}
\end{@twocolumnfalse}]0621完成。



唐湘龙 00:20 

好,欢迎来到龙行天下,我是唐湘龙,我在台湾,我在台北。好,那今天星期五的时间我透过视讯连线,上个月是我第一次访问我的这位的来宾,当然在过去我很诚实说我们并不直接认识,那我纯粹作为一个在新闻工作上面,在媒体工作上面 30 年的人,在自己的专业水平之上的时候,我寻找比我觉得更棒的更优秀的来宾。



唐湘龙 01:05 

那能够除了充实节目的内容之外,我其实是有个私心的,就是我总希望访谈跟节目的内容不止对我的观众、听众是有帮助的,最重要是我自己也希望对我有帮助,所以各位可以理解,就是说在龙行天下,在我的节目里面我会邀的来宾,我一定希望这个来宾是比我更强很多的。好,那今天我透过视讯连线那专访王孟源博士,我注意他很久,上个月我第一次访问他,在公开访问之后,因为我对王孟源的理解就是纯粹是自己看节目,然后看很久,注意他的一些谈话。那我觉得他可以提供,因为我们使用华语,对华人观众来讲会有很多实事议题上面的启发也能够开启你另外的视野。



唐湘龙 02:04 

但是对它的背景我不了解,不过上个月在访问王孟源之后,有他的同学觉得非常的惊喜,就在我们的留言区里面留话,那个留话解了我很多的惑,就让我知道王孟源从小就是个天才儿童,他读书是这种的赶进度的,所以不只是资优保送了台湾的清华大学清华物理系,同时三年就已经毕业了。哈哈哈,我觉得人干嘛这么的聪明。



唐湘龙 02:33 

好,然后,也把这王孟源的这些这些涉猎兴趣就是大概补充了我很多对王孟源的认识,我相信许多的那些面向是大家不太知道的,那让王孟源的近乎到了一个神级的人物一样。好,那因此我当然就是很冒昧的就是说邀请王孟源,希望他能够成为龙行天下的固定来宾,我希望每个月有一块的时间是割让给王孟源的,让王孟源畅所欲言。好,那今天我设定了几个大的题目,一个是我相信每个人都会感觉到现在全球的经济危机重重,那不管是经济、金融,不管是在政治,或者说是在国际经济的面向上面来讲摇摇欲坠,那问题很多,那美国有美国的问题,欧洲有欧洲的问题,中国也有中国的问题。但是中国毕竟这 20 年的时间加入 WTO 之后,走上了全球经济整合的道路,中国在安然度过了 2008 年的华尔街金融风暴了之后,又经过了十多年的时间,中国仍然能够从这场的就是说全球经济危机里面全身而退吗?



唐湘龙 03:45 

第二个就拜登今天到了亚洲了,5月是拜登的亚洲月,除了前两天跟东盟的十国去了八国的领袖在华盛顿会晤之外,他这一次到亚洲,他担任总统之后,亚洲的首访第一站到韩国,那韩国喜出望外,给予拜登非常热情的回应跟欢迎。那日韩关系似乎越走越近,美日韩之间的就是说这个东北亚的新北约的架构似乎八字已经有一撇了,那接下去拜登要到日本,除了跟岸田文雄见面之外,同时这个印太四国的Quad这个峰会,那今年举行,但是里面的印度的态度变得非常的关键。还有一个变数就是澳洲的Morrison很可能在这场选举当中失去它的执政权的地位,那澳洲跟印度对 Quad 会不会有影响?韩国如果加Quad,那从 4 边变成 5 边,对亚洲的情势会有什么影响?再来我们关注一下芬兰跟瑞典表态加入北约,虽然现在土耳其表态,有条件的反对,有条件的支持,那暂时把芬兰跟瑞典卡住了。



唐湘龙 04:55 

可是芬兰跟瑞典这两个波罗地海里面的重要的国家,通常我们比较注意芬兰,因为芬兰跟俄罗斯的边界线非常长,那圣彼得堡就在芬兰湾的湾头,所以芬兰加入北约,如果乌克兰加入北约对俄罗斯来讲如芒刺在背,那芬兰难道俄罗斯真的没有什么反应吗?会像是普京讲的这么的轻描淡写吗?这里面还包括了瑞典,瑞典跟丹麦共同的卡住了整个波罗地海的出海口。



唐湘龙 05:28 

波罗地海是一个非常封闭的海,跟黑海一样,非常的封闭,那这种所谓的陆锁之海,那被陆地所封锁的海洋,黑海跟波罗地海,确是俄罗斯这个国家欧洲认同,他认为他是欧洲国家的两个非常重要的海洋的洋面,那这两个国家如果加入北约对俄罗斯的安全感影响打击势必非常大。



唐湘龙 05:52 

那俄罗斯会坐视吗?最后如果有时间的时候我们再来谈,就是说在俄乌战场上面,那今天的清晨的时间,美国的参谋首长联席会议的主席Milley那跟这个俄罗斯的这个参谋首长通了电话。继三天之前Shoigu就俄罗斯的国防部长Shoigu和这个美国的国防部长奥斯汀通电话之后,那现在参谋首长也通了电话,那马里乌波尔似乎战事结束了,不过今天最新的消息是说里面仍然有亚速营不愿意投降的。那俄乌的情势的最新的进展如何?这场的战争是不是快结束了?开始出现降温讯号了?好,这几个问题待会请教王孟源,来,先介绍我们的来宾,透过视讯的连线,在我们线上的那人,在美国的王孟源王博士。欢迎孟源。



王孟源 06:49 

很高兴再上节目跟大家聊天,谢谢你的夸赞。一辈子就是专心做学问,因为为了自己的兴趣,也为了国家、世界、人类的抱负。



唐湘龙 07:04 

好,能够做到这样我跟你说就很了不起。人能够专心的做一件事情是非常过瘾,而且非常棒,最重要是能够做得好。好了,我们今天来看全球经济。就是,我在过去看王孟源的谈话的时候,我注意到你对于全球的经济,全球金融的谈论,那个是我很惊艳的部分,我一开始其实认识你这部分我总觉得你碰触到一些我没有碰触到的领域跟视角,那我不知道你认不认为目前全球的经济正在一个非常 shaky 的状态,你如何看当下的全球的经济金融情况?



王孟源 07:45 

这是有好几个维度可以来看。就是你如果是以传统的观点来看的话,那因为欧美是占据世界的先进工业地位,所以他们的经济状况是第一个要谈的。那欧美在过去这十几二十年都犯了同一个错误就是拼命的印钞票,他们的旧有的巨观货币理论,这基本上被抛出脑后,然后他们开发出一套新的所谓的 modern monitor theory,就是MMT。那这个modern monitor theory到底讲些什么呢?它这个最大的宗旨就是你只要是能够印主权货币,就不必在乎有没有赤字,反正你印出来的货币可以无中生有。这样的说法当然是政治上需要,为了他们实际上拼命印钞票证明而编出来的谎话,那这很明显只是暂时可行,而且这个所谓的暂时,就是因为它不断的消耗他们自己的货币的公信力,也就是他们利用历史传承的地位。



王孟源 09:06 

就是因为欧美占据人类世界霸权有 500 年了,所以我们的人类历史在过去这 500 年中的主要国际储备货币都是在几个欧洲跟美国的货币这样换来换去,那美元是大概在一个世纪之前登上这个宝座的。所以你表面上似乎所有的证据都,短期的证据都指向的确他们这个理论是正确的,实际上是因为他们有这个 500 年的历史传承,然后他们在消耗自己祖传的财产的信用,所以这一次俄乌战争我以前已经详细讨论过了,它其实是要利用俄国做一个练手的机会,团结北约跟其他的附庸国,像日本、韩国跟澳洲,然后准备下一次挑起一个借口,然后能够让全部的这些先进工业国都团结起来围攻中国,所以他们是志在必得。



王孟源 10:21 

他们认为俄国的经济体量只有比中国要小一个数量级,而且跟欧洲又经贸关系非常的密切,工业上依赖又很深,所以他们以为很简单的就可以获胜,可是他没有想到自己的衰弱,这个衰弱就是他们在过去这 40 多年已经经过严重的去工业化,所以他们的经济已经都虚拟化了,都是靠金融的。



王孟源 10:57 

那金融的最基本的那个基础是什么呢?就是信用,现代金融全都是靠信用。那你一旦去没收俄国的外汇,没收了 3000 多亿,然后不只是他们的国有资产,而且连他们富豪在国外的资产也都是没有经过任何司法程序就直接没收。像这样子等于是自己急着要挖自己货币霸权的根,这是非常非常不智的。



王孟源 11:37 

而且这原本不需要是打压俄国的一部分,你可以直接就局限在军事上让乌克兰人去当炮灰,然后在金融上面不必出手,因为在金融上面你一出手的话,俄国用能源来反制,欧洲承受不住。而且刚好在 2008 年之后,他们印钞已经印了 14 年了,这时候非常的虚弱。



王孟源 12:08 

我在两个月前曾经在另一个访谈中提到一个比喻,就是你在一个山头上面把那个树,十几年下来把它全都砍光了,然后现在下了一场大雨,它有土石流。你说这个土石流原因是什么?是你把树砍光了,还是因为大雨。哈哈,但我认为你人为可以预见的事情是由人的决定来算责任,所以是把树砍光,是真正的原因。



王孟源 12:45 

那同样的,他们这些中央银行拼命印钞票印了十几年,现在原本经济就已经很虚弱,结果他们还硬去挑衅一个世界一流的能源跟资源出口国,这真的是非常不智的事情,而且还主动的消灭了自己货币的国际货币的信用,所以我们现在看到的这个经济危机其实是欧美的危机,那只不过是因为他们的货币是国际储备货币,所以当他们出现经济危机的时候,尤其是一种通胀型的危机的时候,它第一件事就是提升利率。



王孟源 13:30 

当然那提升利率之后就会收紧银根,那些比较没有储备、没有外汇储备,本身经济体质比较衰弱的国家就会受到很大的影响,那这里面最典型的就是土耳其。另外一个压力是过去的新冠疫情,过去这两年多,对很多中小国家,尤其是依赖旅游业的、观光业的中小国家打击非常的大。那这里的典型就是Sri Lanka。



唐湘龙 14:04 

它已经崩溃了,它已经崩溃。



王孟源 14:07 

要濒临崩溃了。所以你说现在这个所谓的国际经济议题,其实是欧美引发的,就是因为他们自己的不智,但是偏偏的,我曾经计算过就是美元的国际储备货币地位使得它有一个 1: 6 的杠杆,就是它做蠢事、做坏事的时候,这个坏处就是通胀会被稀释6 倍, 6/ 5这个通胀效应是由世界其他国家所吸走的,这还没有考虑到这种我刚刚提到的从土耳其这种国家回流的这种效应。



王孟源 15:00 

所以我们目前看到的这个世界经济危机其实是美国在对世界吸血的一个过程。就是美国本身出了问题,所以他开始对外界吸血,那这以前曾经发生过好几次,第一次在70年代跟 80年代都发生过。70年代那个 bretton Woods system 被打破的时候,就是停止跟黄金挂钩。



唐湘龙 15:35 

金本位位制被打破。



王孟源 15:37 

对,被打破的时候是第一次,然后后来在 1985 年的市场协议,他们强迫欧洲的几个经济强国在一年之内对美元升值了百分之百以上,严重的影响了国际进出口贸易,那是第二次。但是真正比较典型的、我们到目前为止还在一直受影响的,是 1997 年的那一次亚洲金融危机。



王孟源 16:10 

对,因为那一次是最典型的金融搜刮,就是他藉我刚刚所提的这种抽紧银根的方式,然后故意的把你的这些二流的经济,釜底抽薪,把你的资金都抽走,等到如果你的汇外汇储备不够的话,你整个国家就实质破产,那国际的实质破产法庭是  imf,然后 imf国际货币基金会就进来,然后进来说我可以帮你破产重整,但是条件是你必须要把你的优质资产贱卖出去。所以那个时候,三星其实现在的那个多数股权是由美国人拿的,就是1997年发生的那事。



王孟源 17:00 

那我为什么说这件事情影响一直到现在?因为从那之后,所有的新兴工业国家通通拼命的累积外汇,整体来说这些第三世界新兴工业国的外汇提升了 10- 50 倍不止、不等,就是视国家而定。就是你如果看看中国的外汇现在 3 万多亿,那实际上什么时候开始巨幅累积?就是 1997 年到 2005 年之间这8 年,被吓到了,其他的国家也都是很类似。那问题是你这个累积这些外汇可以作为抵抗这种 IMF 讹诈式,就是我刚刚描述的这个吸血式的金融打击。但是他也反过来鼓励美联储拼命的印钞,这就是为什么过去 20 年美联储可以拼命印钞的结果。美联储印钞印多了以后大家不太放心,就会想要准备替代货币。



王孟源 18:10 

这替代货币大家看一看,当然首选是欧元,所以欧元也跟着是全世界都在抢手。那所以这个后果就是欧元银行也可以跟着拼命的印钞票。但是这一次由美国带头来打击俄国,首当其冲最惨的就是欧元、欧盟,整个不但在金融上而且在能源供应跟工业上面都受损,远远比俄国更伤的更重。



王孟源 18:53 

所以当然我为什么会特别去提这件事?因为这其实就是美国的内心的那个算盘,因为他们也是希望日本、韩国跟澳洲这样子来帮忙,牺牲他们自己,来削弱中国。然后我们再回头看看这一次的这个经济危机除了货币,这个我刚刚很简单的讲了一下,还有另一个层面,货币它的那个在经济上的作用是所谓的demand,就是需求,它刺激了需求。那现在这个需求必须要开始下降。但是我们到目前为止,因为他印了太多的钞票,所以这个目前我们感觉到的这个效果只是通胀。



王孟源 19:45 

还没有看到那个衰退,虽然美国的第一季经济衰退了 1. 4,但是那个现在还不能够确定就会一路摔下去。因为你看看第二季的预期基本上还是持平,就是零上下这样动一下,就是你即使是也有衰退,也就是衰退 1\%、 2\% 这样子,并没有说很明显的已经崩盘了。那这原因是因为美联储印了6万亿美元,这个都还没有收回,到现在还没有收回,所以这些钱还在外面。



王孟源 20:23 

那我刚刚说这是他的这个需求面,我们谈谈供给面。这一次这个比较特别的一点是因为新冠疫情,还有其他的一些因素也造成了供给面的很大的问题。所以目前这个通胀的情形会这么严重,就是因为需求跟供给面两边都在涨,把它推高。那我现在要特别强调的就是这个所谓新冠造成供给面问题,其实是美国人、美国的经济学者为了政治正确而拼命讲的话。你如果持平的仔细的去看这个细节,就会发现新冠其实只是其中的因素之一,而且是这许多因素之中的唯一一个例外。我说例外是什么意思?就是只有新冠不是美国人自己创造出来。



王孟源 21:27 

这个为什么这么说?我们全球化从 80 年代末期到 90 年代加速,这个全球化的背景是什么?它的背景的需求是一个和平、合作而且稳定的国际秩序,对不对?因为有过去这 30 年有这样的一个全球化的背景,结果是什么?结果是你的这个全球的供给链变得非常的长而且复杂。



王孟源 22:04 

同样一台机器或者是一个手机,它这个零件你可以追,这个可以跨越整个地球好几次,到最后才组装成一个成品,然后销售到用户的手里。那所以在过去这十几年,美国开始觉得这个全球化对自己不利,因为他觉得中国追上来了,所以他有意、故意的为了打击中国,主动的去想要脱钩。那我刚刚也谈到他这个拿俄国这一次,故意逼俄国去打乌克兰,这也是为了练手的一个企图,它这个企图就是要重新建立冷战的两个阵营,分开的那个态势,因为他冷战吃到甜头嘛,当初冷战的这个美苏对立,其实是罗斯福看得很清楚,因为 西欧 不管怎么样子永远都比东欧要先进、要富庶。所以你这个美国跟西欧联盟一定不会输给苏联的东欧联盟,而且真正威胁美国霸权,在二战之后威胁美国霸权的不是苏联,而是西欧。



王孟源 23:41 

因为这些传统的国家才有那些科技,才有那些人才来可能挑战美国的工业实力,所以你故意把它分裂成两块之后,这个西欧永远都是你的小弟,因为你一不小心打起来以后他们就会被打烂了,而且他们对二战这个把欧洲战场打烂的那个记忆犹新,现在他也是这样,他就是基本上他们的这个战略思想就是我们冷战占了那么大的便宜。



王孟源 24:12 

占的便宜是谁呢?占了西欧的便宜,而且他们还感谢的不得了,还要付钱,然后占了苏联的便宜,那所以他就是想要复制这个,而这一次是要把中俄统统关到一起去,然后重新拉拢北约,然后日本、韩国跟澳洲来搞一次新的冷战。那你一旦为了这个来搞脱钩、来分裂、来切割的话,这个全球化就搞不下去了。你看现在这个全球供应链到处出问题,其实长期的原因就在于这里。



王孟源 24:55 

过去十几年美国有意的跟中国作对,很多事情该合作的不合作,而且国际组织变成他们的白手套就是完全偏颇的。你看那个 WTO 的那个诉讼,有哪一个是合理的?对不对? WTO 事实上也瘫痪了,其实他是偏颇欧美,美国还是受不了,就是你只要是还要假装是客观的,对美国来说就是不够好,然后再加上他们这个货币超发其实对供给链也有影响,因为你的这个货币的汇率,美国这样子,每次每隔几年就到外面吸血一次。这个对进出口贸易都是有抑制作用的。



王孟源 25:51 

最近这几年白左的思潮兴起,他们一天到晚就是讲这个环保的议题:反核,台湾也有嘛这个东西,这个对供给链影响也很大,因为你的能源不安全,你这个能源的替换到新能源不是一个理性、长期尊重客观环境的一个过程的话,而是由老太婆在街上拉横条来抵制的时候,这时候这个供给链就不可能运转的很顺利。



唐湘龙 26:28 

这个孟源我打断一下,那现在的这种的全球的经济的态势,我可以感觉到最近这,特别这一两个月的时间,我觉得大家对于自己的资产的紧张的程度是这几年仅见的。当然因为看到的这个下行的趋势越来越明显,那中国大陆它本身刚好在这两年时间它也推动了一些的变革,这些的变革其实是非常结构性的。那中国大陆,尤其您刚刚提到就是说美国在冷战里面尝到了甜头。absolutely,那他现在的所有的这些的,不管是金融面、军事面或者外交面的打击的对象,其实对俄罗斯来讲只是练一练自己的这些身手而已。他真正的目标是中国,中国在整合到全球经济,然后在 WTO  20 多年的时间,他也签了这么多的FTA之后,他还有办法在这一波的全球的经济跟金融危机当中脱困吗?中国现在到底遇到的是怎么样的情况?



王孟源 27:38 

这有两个层面,一个是中国应该做什么,然后另外一个角度可以看中国大概会做什么。哈哈,很不幸的这个我先回答第二个问题。就是中国大概什么都不做,就是因为二十大快到了,而且事实上他们的中层的技术官僚很腐烂的,就是没有人敢,没有人有这个所谓的 incentive 就是有个人利害的动机。



唐湘龙 28:12 

 incentive 动机,想要主动的去做什么。



王孟源 28:17 

动机去为国谋福。对,大家都是当一天和尚撞一天钟。所以我对他们能搞出什么东西不是很看好。那至于他们应该做什么呢?我在我的博客上有详细讨论。这个要谈起来至少三四个钟头,所以我们还是不要碰的好。所以实际上会发生什么呢?如果中国什么都不干的话,这个你今天所提的这很多的在一开始你提的这些问题、这些议题其实都是在一起的,它就是未来这几个月、这个短期之内美国对外的外交态势还有经济态势会是什么样子。那这个呢取决于两点:第一个是欧美的经济衰退的速度,就是他们这个一方面有通胀,所以你必须要提升利率,但提升利率以后固然资金回流,但是你本身的这个资产泡沫也开始一个一个爆掉。



王孟源 29:25 

这个也是我在过去这三年反复预言过的,但是你不能够确定说它这个速度还有它的这个步骤会是怎么样。就是你说封面要过境,我知道这个周末会下雨,但是我不知道几点钟会下多少雨,这个这是没办法预测的。所以这是第一个问题。那第二个问题就是俄乌战争,这会打的怎么样?为什么这个很重要?因为既然美国是要拿俄国当示范,现在反过来俄国就成为中方的示范,让欧洲、北约还有日本、韩国这些国家看看当美国走狗的下场是怎么样?季辛吉有一句名言就是 to be the enemy of America is dangerous, but to be his friend is fatal。



唐湘龙 30:34 

当美国的敌人其实很危险,那美国的朋友更危险。致命的。



王孟源 30:38 

当美国的朋友是会致命的。哈哈哈。所以如果俄国能够把这一点再示范一次,然后提醒这些国家的领袖,那么很多中国所目前面临的挑战就会不解而解、自然而解,就是他们,反正他们大概也不会做什么真正的作为。那所以,其实中国一直是很运气很好的。你如果去看我的博客,我曾经讨论过,在 2001 年小布希刚上台的时候,他们就预见中国的经济发展会很快,所以原本他们很可能有计划在 2004 年的时候挑动。



唐湘龙 31:23 

他们准备提早 20 年对大陆动手。



王孟源 31:28 

对,挑动陈水扁去搞台独,然后故意创造一场台海战争,在 2004 年的时候,美国海空军要打赢一场台海战争是办得到的,这个大陆是不可能打赢的。但是现在已经时过境迁了,这个已经太晚了。就是所以,911救了中国,然后后来 2008 年他们美国人又自己把自己的金融泡沫搞太大,搞成那个次贷危机,然后现在他们又去找俄国的麻烦,结果实际上已经是碰了一个钉了。



王孟源 32:20 

我原本做访谈的话通常都是跟史东的八方论坛做讨论,所以他原本也是希望这个周末让我访问,我跟他说其实如果我们要谈俄乌战争的话,多等一个礼拜好了,因为我觉得俄国的这个军事已经快要开始滚雪球了。就是多等一个礼拜的话,这个局面可能就明朗化。就是他在过去的这两个月基本上都是用大炮兵去消灭乌方的人力,所以他的有生战力消耗的差不多了。



王孟源 33:04 

在过去这两个月,你基本上每一天平均俄方是占领一个小村庄,但是在过去的这三四天,他平均一天占领五六个,你可以看出这个是乌方已经后继无力了,就是他的这个职业军人已经打的差不多了,那个洞已经补不上了、填不上,所以我的感觉是,你如果即使中方都不干,只是继续的搞内部的反腐跟改革,然后确定它的经济不崩溃,因为事实上它这个在上海,被上海地方官僚搞了一个大娄子,过去这两个月也是很惨,封的也是很惨,很影响国内的经济发展,所以它必须要降利率。那他即使只专心的处理内政的话,我觉得也有相当大的机率会再次的靠运气熬过美国的这次打击。



唐湘龙 34:13 

好,我们把话题,因为你刚谈到了俄乌战争,我们倒过来谈俄乌战争,你刚说它已经到了一个滚雪球的阶段,一方面这个礼拜就是马里乌波这边基本上告一段,因为它其实已经不影响战局,它纯粹是一个新闻跟一个意志面的问题。但是在乌东的顿巴斯地区,因为现在乌克兰的问题谈了 3 个月之后,台湾新闻上面已经接近疲乏,所以很多人其实已经没有follow的这么紧,那对于主要的新闻来讲可能也没有太多的去关注的热情,可是其实在乌东的战事是有很大的变化的,就是说俄罗斯,俄罗斯似乎它的包围圈正在生效。所以你刚刚讲的滚雪球是说一个有关于俄乌的这场的战争的关键性的转折快出现了吗?



王孟源 35:04 

是,我认为,因为一个礼拜之前,有出现了一个转折,就是俄军打下了Popasna,Popasna是一个中型的城镇,但是它是这个乌军防线的南线的一个枢纽,是重兵防范,而且是有 8 年的工事,所以是一个非常坚固的堡垒,结果他被俄军打下来了。打下来之后在过去这个礼拜什么事情?这个俄军就投入了好几个旅,然后开始席卷后方了。那你这样一来,一旦它已经突破了你的这一点,然后开始席卷后方的话,你那些工事都没有用了,因为人家是从你后面打过来的。而且在其他的几个战线上你也可以看出那个乌克兰军队开始捉襟见肘,比如说三 天前的乌克兰大吹特吹的那个俄军渡河被消灭了一个旅的事件,其实都是假的。我跟你讲实际上。



唐湘龙 36:16 

为什么?为什么?我都以为这是真的。



王孟源 36:19 

就是照印英美的新闻。啊英美的新闻连 Mariupol 这些人无条件投降都要说成撤退



唐湘龙 36:29 

撤离。对,他说是撤离,他不是说投降,他是说撤离。明明就投降还说是撤离。



王孟源 36:35 

明明是投降。我儿子也是看了新闻就说,唉,听说这个 Mariupol 的人撤退了。我说是,是撤退到了战俘营去。



唐湘龙 36:43 

对对对,都被关在战俘营准备接受审判,这哪里叫做撤离?就是投降。



王孟源 36:52 

这个三天之前,那一场的确是留下了大概 20 多辆装甲车辆的残骸。然后的确是俄军后来撤退了,但是今天他们为了庆祝这场胜利,乌克兰请了 CNN 的记者去拍视频。结果大家一看不对劲,那些被摧毁的坦克不是俄国的坦克,而是乌克兰的坦克。为什么呢?因为他们是T64,俄军当然有T64,但是早就都退。



唐湘龙 37:32 

对,现在都是 t 72 以上了。



王孟源 37:35 

对,T72。 就是因为当初 T64 跟 T 72 其实是同一代的产品,但是因为 T64 是乌克兰的工厂出产的,所以在苏联解体之后,乌克兰选择针对 T64 来标准化,而俄国选择对 T72 来标准化,所以俄国军的坦克是 T72,乌军的坦克是T64。



王孟源 38:01 

大家一看,不对,那些被摧毁的坦克是T64,而且用的是乌克兰的标准迷彩,乌军的标准迷彩。所以这告诉你什么?就是,而且他们那个在一个礼拜之前吹嘘的时候说那些被摧毁的,你没有看到太多弹坑,就是不是用炮兵去把他打死。那你现在这样回想一下,这个不是用炮兵去打下来的,为什么不能用炮兵呢?因为俄军已经在炮战上面获得绝对性的优势,乌克兰的火炮一开火马上就会被消灭,所以他们的确是把俄军的一个侦察支队打回去了。但是打回去靠的是什么呢?靠的是他投入了装甲预备队,所以双方的装甲互有伤亡。我们看到的,他们一个礼拜前看到的那些照片还有视频有 20、30 辆装甲车的残骸。



唐湘龙 39:08 

在那个顿内茨河上面。



王孟源 39:12 

对,其实是双方都有的,双方的装甲车都有的。那但是你在东乌已经打了 3 个月,这个乌克兰在一个月之前大家就可以确定乌克兰在东乌只剩下一个旅的预备队,所以我认为这一次是他们把最后的战略预备队投到那个方向去了,然后损失还相当的惨重,留下了一大堆这个装甲车的残骸。



唐湘龙 39:41 

好,那这个就是说现在战场上面情绪,因为国际媒体,因为来自于俄罗斯的一些比较可靠的讯息,现在西方的媒体当中几乎都是被遮蔽的,那我们大量的依赖就是说其实在乌克兰战场上面许多属于宣传性的影片,但是如果你认真去分析的话,乌克兰现在在整个东乌地区的战况恐怕是非常危险的,它的主力部队正在被包围,而且就是这两天的时间。它的包围的范围以及可能会被包围的这兵力远比马里乌波钢铁厂要大好几倍。那这个主力一旦被包围被歼灭或者投降了之后,它的战力几乎就已经消耗完了。



唐湘龙 40:24 

这样的一个背景是导致最近我们看到美国其实沉默了将近 3 个月,都不讲话,没有释放出任何的态度跟讯号。可是就在这个礼拜,他的国防部长主动的找了俄罗斯的国防部长,他的参谋所长主动的找了俄罗斯的参谋首长,这个动作是很特别的,你认为美国在帮乌克兰求和了吗?



王孟源 40:45 

事实上就是求和,求的是停火协议。然后我刚刚讲这么多,我很高兴你能够举一反三。这个因为一个很自然的结论就是这解释了为什么美国急着要跟俄国的国防部接触,因为他知道普丁不会跟拜登谈话,所以只好从军方来接触。这个接触的目的是为什么呢?这个目的就是希望能够有什么暂时停火,让这个乌军能够有喘息的机会。因为就像你刚刚已经领会到的,这个Mariupol乌军的损失大概是1万人,但是在东乌还有 6 万多人。这些刚好这7万部队就是整个乌克兰最精锐的前线。而且现在我刚刚讲到那个Popasna的那个突破,那个突破现在很快的在一两个礼拜之内就会形成一个包围圈,这包围可能是 15000- 2万人。那你一旦从7万人变成到6万,然后从6万再到4万,这个俄军要切割包围的话就越来越容易,就是基本上他要突破的时候,你没有预备队可以投进去反攻了。所以我刚刚讲的这个,未来这两个月我们会可能会看到东乌的情势滚雪球似的演变,演变之后这个会有国际性的影响。你不要忘了就是欧盟在那边搞的什么油气制裁,到目前基本上是不行了,就是因为匈牙利的阻挠,他没办法做石油制裁。那你如果连石油都没法做制裁的话,那天然气更加不可能,因为天然气更难远长途运输。所以你说要从中东或者北美寻找替代的出口商的话,远远更为困难。所以你在经济层面挫败了,金融层面挫败了,因为现在卢布的那个汇率反而比战前还要高,然后你在军事上也挫败了。你想想看,你刚刚说为什么这个美国国防部拼命要跟俄军,俄国的国防部接触。那你有没有想过为什么北约拼命的要把芬兰跟瑞典。



唐湘龙 43:32 

瑞典



王孟源 43:33 

因为他们也看到这个,这一次这个乌克兰已经快要没救了。这个你至少要找个面子,就是这个芬兰跟瑞典加入北约,证明北约复苏。然后这个是一种挽救面子的做法,当然长期来说也有实质影响,但是问题是现在土耳其的Erdoğan要价要的很高,异常的高。所以能不能达到妥协还很难说。



唐湘龙 44:13 

好,我下一次再跟孟源来谈,就是关于波罗的海周围的情势,我相信孟源对这一定会非常感兴趣,因为它对欧洲未来的影响会很大。好,但是我们现在把视角拉回到亚洲,就是拜登到亚洲了,拜登前几天才把东盟的几个国家找过去,但是我高度关注东盟跟拜登的这场互动,我认为这场的互动是非常失败的。就是拜登很可能,美国很可能因为这场的会面的失败而跟东盟越走越远。好,那东盟的情况是这样,但是拜登的重点是在东北亚,尤其他这次先到韩国,然后把重点摆在韩国以及接下去的日本,包括Quad。你怎么看拜登这一次的亚洲行?它显然是在建构一个在亚洲地区的类似北约的军事同盟对中国的包围圈。那他这一次到韩国、到日本,它能够达成怎样的目标?中国该如何看这件事情?



王孟源 45:13 

我们先从客观的长途大局面来看一下。你说俄国的战力当然远远胜过乌克兰,即使是西方拼命把武器跟资金往乌克兰送,但是,俄国能不能跟北约直接开战?可能是没办法获胜的,就是你至少必须要让他们打进俄国,然后再想办法长期反攻。但是中国的海空军已经发展到这个程度,就是你如果在东亚也搞出这个局面的话,中国不必要退守到海岸线之内,它可以很简单的就封锁日韩的贸易线,更不要说台湾。所以这个,我们现在说这个欧洲的这些领导人有多么多么的不智,就是牺牲自己国家。



唐湘龙 46:14 

照亮美国,燃烧自己照亮美国。



王孟源 46:18 

然后这事实上对俄国也没有什么,严重的损伤就是皮肉伤,真正受了严重内伤的是欧盟。但你如果把这个脚本搬到东亚来,东北亚说的话,日本跟韩国比欧盟还要脆弱的多,就是一旦打起来,他们马上经济完全崩溃。所以这个你要说他们虽然亲美,要说他们很笨,就算假设他们跟欧盟的那些领导人一样的笨,但是这个局面对他们来说更加的不利,所以这个要说他们完全跳上美国的战车,我觉得也不能够保证。就是这一次拜登出来访问,当然明面上一定会有很多光鲜的宣言,然后表达什么同盟的,指天画地的宣誓同盟,但这实际上能有多大的作用?我很怀疑,尤其是。



唐湘龙 47:30 

你对韩国的判断。韩国现在新任总统尹锡悦上来了之后,他虽然只赢了不到 1 个百分点,不过他几乎把文在尹的所有的地缘政治的、这些战略的一些思考全部的反转。你如何看韩国?



王孟源 47:49 

这些政客都是有很强的非理性包袱的,因为你要搞选举,选举就是绝对非理性的一个过程,基本上是选网红的过程。



唐湘龙 48:04 

我们都不要当网红。



王孟源 48:07 

所以芬兰的这个总理就是一个典型的网红。这个非理性是非理性,但是真正要跟中国摊牌,我觉得他没这个胆量。就是biden叫他多孝敬一点钱,他会愿意,叫他搞小动作,他可能愿意,但是真正要做到严重损害中国利益的事情,我觉得他不敢。



唐湘龙 48:42 

所以你认为不会有一个美日韩的三国的同盟。



王孟源 48:48 

我觉得他们还不敢走到那个地步,就是日本跟韩国不敢走到那个地步。因为他们的贸易线实在太脆弱了。那事实上美国也不至于要求到做出那种明显的挑衅性的第一步,因为他现在要准备的是先让大家有心理准备,然后未来可能还是要找一个借口,这次来说是让俄国先动手,然后几年前在叙利亚是搞这个化学武器的栽赃,然后在那之前那个伊拉克是核武器的栽赃。所以它很可能是在针对中国的这件事情上,他要一个导火线很简单,就叫蔡英文去搞就台独可以了。



王孟源 49:46 

所以我觉得他可能会有一些联合声明,可能会有一个新的所谓的同盟,但是不是真正北约式的同盟,而只是一个比较松散的像AUKUS那样的同盟,就是可能是一个小的AUKUS,这是已经是最极端了。我觉得这已经是要把日本跟韩国的领导人想象得非常非常愚蠢才能够发生。



唐湘龙 50:26 

那在这次拜登到了韩国,当然韩国现在是喜洋洋,因为毕竟拜登这次也给尹锡悦很大的面子。尹锡悦也在这个 ipef印太经济架构下面很热情的回应,同时也释放出他有兴趣参与Quad,就是原来的这个四方安全对话,他也有兴趣参与,可是这个四方安全对话里面现在我刚提到出现了就是说两个变数,一个就是澳洲即将举行的选举,这个选举到现在为止看到的最新的民调,现在的已经执政了将近 10 年的保守党,那看起来现在是居于劣势。所以保守党很可能会败给工党,工党可能会重新执政,那因此工党会不会在这方面有态度上的修正,这是其中一个变数。



唐湘龙 51:15 

另外一个就是更值得关注的是印度,因为印度毕竟在这次俄乌战争当中,他不是官方当中在玩国际政治,而是你看到如果注意到印度的整个舆论,我觉得舆论比较重要,印度的整个舆论,印度整个的人心在这场俄乌战争当中似乎有非常清楚的偏向,比中国大陆更清楚,觉得好像跟俄罗斯更靠近,那更敢于去为俄罗斯发言,那因此大家就会觉得说因为印度这个国家舆论对于政治人物的动向是有很强的绑架效果的。那印度是不是会在未来的整个美国所操作的亚太的反中的这样的一个包围圈当中,会有一些新态度、新的表态,你怎么看印度?



王孟源 52:03 

这个我对印度的评价向来是、基本上是类似刚刚我们提到对美国的评价,就是当他的敌人有危险,但是当他的朋友是致命的。印度更加是这样,因为他就是比昂萨更像昂萨。而且,他这一次站到俄国这一边不是因为有什么正义感,而是因为他看出,他没有欧洲那么笨。他看出这次是昂萨集团在找炮灰,而他不愿意当炮灰,因为印度人认为是要讲炮灰的话,是别人为我当炮灰,我自己不会去当炮灰。这其实已经在当前国际政治上算是很难得,你说这像这种的态度已经算是很理性了。



唐湘龙 53:09 

没错。



王孟源 53:11 

你像台湾的话,求之不得,台湾有这样的。



唐湘龙 53:13 

台湾会争先恐后当炮灰的意思。



王孟源 53:17 

台湾的那些党派的政治、政客都是抢先要去当炮灰的,反正损失的是老百姓,就像现在乌克兰死的是一大堆老百姓。



唐湘龙 53:28 

对,真惨。



王孟源 53:28 

对,那个Zelensky已经收刮了5亿美元了嘛,就是在开战之前就已经知道,大家都知道他有5亿美元的那个私人资产,然后上个月英国,这真的是给了他家人的护照,就只有他没有拿到护照,不过你当然知道那不用给,他也一定是安全的,就是他的老婆小孩全都拿到。



唐湘龙 54:02 

都已经有英国护照了。



王孟源 54:03 

对,那这个的意思是什么呢?就是你把乌克兰搞死了,不要紧,反正你到时候来伦敦当寓公。这个,你必须要尊重,印度这一次的选择,就是他们至少是为自我利益优先的一个理性的考虑,而不是傻傻的那些英美的媒体随便撒个谎叫你跳悬崖,你就真的跳下去了。但是你必须要了解到,这并不代表他是你的潜在盟友。事实上哪一个人把印度当盟友的,哪一个人就最倒霉,一定会惹一生骚。哈哈哈。



唐湘龙 54:53 

好,那你怎么看?



王孟源 54:55 

必须要拿到的教训是美国争取印度争取了这么久,花了这么大的代价跟时间,结果他们拿到什么?你刚刚,阿里刚刚被迫打 05 折,对,005 折,把他在印度的资产卖掉了,你说这个不是受政治压力我不相信。然后小米莫名其妙的就被开了一个几亿的罚单,这还是印度人对你客气。欧洲的那些那个无线通话公司,过去这 10 年亏的一塌糊涂,你这个美国的汽车公司现在不是都撤离印度了?



唐湘龙 55:43 

没错没错。



王孟源 55:46 

你去了就是准备要那让他们宰。所以中国要是有人笨到可以把印度当盟友的话,那我也没有办法。这个只能够希望老天继续照顾中国的运气,因为这个中国崛起到现在绝对不是因为有什么外交眼光或战略准备,而是就是东碰西碰一个这样运气出来的。当初美国在 18、19 世纪兴起的时候,欧洲国家也是说,这个美国人哪有什么了不起,他就是运气很好。我觉得中国也是这样子,过去这二三十年,那个内政他们当然是管的相当好,但是在外交战略上面基本上都是靠运气。



唐湘龙 56:35 

好,你估计今年的就中国的经济状况会如何?因为它第一季当然比预期中还是要好一点点,不过它原来设定了 5. 5\% 的这个就是说成长目标,第一季当然是达不到的。那现在其实在国际的这些金融圈子里面,经济分析圈子里面,那看坏中国今年的经济表现的其实还蛮普遍的,你认为中国今年的经济能够达到它预设的目标吗?



王孟源 57:05 

我认为要达到 1. 5 以上的机率不大。,4 点多是比较合理的,就是有两个原因,第一个是上海这个搞砸了。



唐湘龙 57:17 

封城了。



王孟源 57:19 

对,那这个影响第二季影响的很厉害。第二个原因是我刚刚讲了一个小时,就是欧美快要摔下悬崖了,我们虽然不能确定他会在第二季摔下悬崖,但是你第二季、第三季、第四季全部加起来,他们摔下悬崖的几率很大。



唐湘龙 57:38 

就很大了。



王孟源 57:39 

对。那你欧美如果摔下悬崖来的话,这个中国会受影响。



唐湘龙 57:45 

好,当然了,就是说中国努力在保持的就是他如果在体质上面不要受太大的伤害,他正在转骨的阶段,看到这两年他做了很多预期以外的调整,当然那个调整的方向对不对?黑猫白猫让这时间去检验。不过现在全球的经济现在是在一个非常 shaky 的状态,它对于中国今年的总体的经济表现一定是会有影响的,因为时间的关系。我觉得其实刚刚我跟王孟源谈的几个问题,每一个我估计都可以跟王孟源谈一两个小时以上。我知道你意犹未尽。好,不过对我来说,我想对我们的观众朋友来讲可能都已经觉得,这个这些的观点,而且在整体上面来讲确实是很棒的启发。好,那我现在反正每个月我已经我都有一个小时的时间,就是说开给王孟源,这个是王孟源的歌让区,就是让王孟源可以畅所欲言。好,今天是 10 点半钟,那时间到来我要先感谢一下我们的听众朋友,有一些的留言的听众朋友让王孟源听一下,刚有 15000 多位的观众在线上收看我们的直播,然后来。对,让我看留言,现在线才交出来,您稍等一下。在的。好勒,我简单的。好,来彭论,谢谢。彭论应该是在。在黑战。好,黑战的,这在降落到那真正大鱼就是孙子兵法。西方终于发现中国干涉俄乌战争的证据。其实看起来在雅速钢铁场底下是还有戏的。那个戏还没有完哈,就说现在出来的人不会是全部再来呢。



唐湘龙 59:35 

polo 感谢在香港那 IQ MW Chan 好,这在台湾感谢,然后呢?烟雨辰,感谢,然后耳东陈,感谢。这应该是在美国有些小粉红真的很烦主持人和来宾,只要有一丁点不是说中国好棒棒就抓狂哈。没没没关系哈,我喜欢来宾能够就事论事的谈事情,没有任何的没有人让你感觉到特定的片子。你每个人都有感情,但是我是说感情是一回事情,能够理智的去谈事情,分析事情,那个才是你在公共谈论空间当中来讲的最重要的部分。



唐湘龙 01:00:14 

好,再来利分利安范,感谢在台湾,然后 Julian 泰,这应该在日本,感谢,好,感谢我们的听众朋友收看了这一个小时。那我通过的这个视讯连线,那岳阳专访仍在美国的王孟源博士,那我孟源长时间的这些的知识准备,新闻的准备是让我个人非常惊艳的,所以我在访问他的时候我几乎是不需要讲话,那我们自己讲就好了。好,今天的时间的关系,感谢孟源,非常的感谢。



王孟源 01:00:50 

好,谢谢你给我这个机会。



唐湘龙 01:00:52 

你看到我很高兴跟大家聊,你看到王孟源的就是说他在录音的装备都提升了,哈哈哈,就买了很多的新装备,好让大家在访谈的时候可以达到更好的效果,这点非常的感谢慕岩。好,那时间到了,感谢今天不管是孟源也好,或者我们所有的观众收看今天的龙行天下,跟大家说周末快乐,大家平安,感谢,拜拜。





\twocolumn[\begin{@twocolumnfalse}
\section{逻辑推论、乌克兰、内部管理}
\subsection{20220530}
\end{@twocolumnfalse}]王孟源 00:00 

中国共产党的这个制度,还有它的传统都是为了要为底层民众谋福利的,那现在这个沦为一个口号,非常可惜的事情,因为在盎萨的霸权之下,不只是中国的民众,全世界的第三世界民众,甚至欧美国家内部的底层民众都需要救赎,都需要外来的救赎。而当前最好的希望就是中国。所以我觉得这是一个非常非常不幸的事情,值得体制内外的有心人,大家一起努力。



史东 00:53 

各位朋友你好,我是史东。在今天节目中为您请到的是王孟源王先生,他三不五时地,每过一段时间总喜欢在节目中和我们打声招呼,打个照面,和我们谈一谈他心里想谈的一些事情。我们继续讨论之前,就先把孟源请入我们的画面之中。孟源先打声招呼,谢谢。欢迎。



王孟源 01:15 

很高兴有这个机会来跟大家聊一聊。



史东 01:18 

是的,我想这个今天我们就把整个麦克风交给你了,你准备谈些什么以及如何谈?



王孟源 01:26 

最近我想要再扩展一下我的读者跟听众群,但是要做这样的话就会遇到一个问题。因为在当前的这些公共论坛里面的政论者之中,我是独树一帜的。如果听众不了解这个差异的话,就没有办法正确地吸收跟理解我所说的事情。所以我先把为什么我觉得我自己所谈的事情有什么特别,先跟新的听众解释一下。



我的老读者应该都很熟悉了,就是我是唯一一个把社会科学的议题当做数学题来解的人。我的分析是完全根据科学方法,根据逻辑辩证的正确规则来做出的结论。所以我所得到的结论,不是我的主观意见,而是客观的事实,是论证的事实。你说我对我自己好像很有自信,这个自信不是因为这个结论是王孟源的意见,而是已经被数学上证明了。即使是一个初中生,如果能够把一个数学题正确地证明出来,他的这个证明的力量一样是百分之百的、绝对的。所以我的信心是来自事实跟逻辑。那当然这里有一点差别,就是在社科议题上面的事实跟逻辑运作,比一般的数学题目要复杂很多。尤其我们是受公开信息的限制,很多信息其实是不公开的,所以这个事实往往是不完整,逻辑推论也就没有办法做到绝对的严谨。所以我一般做推测的时候只能够谈可能性最大的一条结论,这时候自然也就没有办法做到像真正解数学题那样的绝对了。



举一个例子就是4个月前我上你的节目的时候,我很简单地说,我认为俄国不会在当时开战。但是后来证明我是错了,这个结论是错的。但是你如果回去仔细看我的分析的话,逻辑分析没有错。因为我的逻辑分析是这样的:我认为当时Putin立刻开战不是最优解,他应该再等 3-6 个月,继续再做外交上的努力。所以我的这个事实前提是普京会选择最优解,后来是因为这个事实前提错误了,所以结论错了,但是逻辑分析并没有错。所以我希望听众或者读者来接受我的解释的时候能够理解这一点:我所传授的是客观逻辑的分析,而不是主观的意见,不是联想,不是类比。



进一步举个例子好了,我觉得这是一个很有趣的例子。不是学数学的人大概不知道 1 + 1 = 2 是可以证明的。你如果去看数学的话,除了定义就是公理,如果既不是定义又不是公理的话,那么就应该是定理。定理就是要从定义跟公理来证明的。定义跟公理是不能证明的,但是 1 + 1 = 2 不是一个公理,也不是定义。在大概 100 年前的时候,有一个逻辑学家叫做 Bertrand Russell,他写了一本 1000 多页的书,里面到第 700 多页有一个定理,就是 1 + 1 = 2。如果要不同意1+ 1 = 2 的话,你不能够说:因为你认为你对 Bertrand  Russell 有什么人身上的攻击,或者是他的身份地位不够高,你就不同意。因为这不是主观意见。他已经花了 700 多页的逻辑叙述来证明这件事情,你可以不同意他,你也可以不去理他的论证,但是你不可以不理他的论证就径自地反对他的结果。 你如果没有时间的话,可以完全不理这个话题。你如果要反对他的话,也可以先去把他的证明从头到尾地走一遍,看他有什么逻辑错误。但是你如果没有那个功夫,或者是没有那个兴趣去做这个努力的话,就没有资格批评 1 + 1 = 2 的结果。我所做的论证其实就是类似那样的一个证明,只不过是应用在国际跟政治上的结论上面。



我希望我的听众如果觉得我说的没有什么意思,没有兴趣,那就不要理我,也不要来我的博客。但是如果你要反对的话也可以,请(鼓励)你到我的博客把我的逻辑推论从头到尾地看一遍。我为什么这样鼓励呢?因为我写博客除了教育群众(就是华语的群众)这些基本的人生观、世界观之外,我还想要传授基本的科学方法跟逻辑辩证的规则。博客里面有一篇很重要的文章,就是《读者须知》。这基本上是我希望新读者去看的第一篇文章。那这篇文章里面所列举的各种规则,其实就是现代逻辑辩证应该有的规则。最常被新读者违反的是第 8 条。这个有兴趣的读者请你再回去复习一下。这第 8 条是什么呢?就是我刚刚解释的,你如果有什么反对的意见,必须要...。因为我已经尽了我的责任,把论证的事实前提跟逻辑演绎全部都明明白白地写下来。所以如果我的结论是错,不是我引用的事实前提有错,就是我的逻辑推演有错。那反对的人有责任先把这些仔细的看一遍,然后指出哪里犯了错。这是读者须知第 8 条的要求。事实上是科学辩论还有逻辑辩证的重要规则。



王孟源 09:00 

这个讲完以后。为什么我今天特别要讲这个?除了是因为我最近有一些新读者不太理解这个这些基本的规则之外,我想指出,最近Putin把盎萨的这个霸权体系很明白地说是谎言帝国empire of Lies,谎言帝国,它的这个能够这么成功,事实上过去 500 年都是很成功的。这个自从英国还没有当霸主就已经开始了。在 16/17 世纪英国还在跟法国争霸的时候,法国已经注意到盎萨特别喜欢撒谎,那时候有所谓的 Perfidious Albion这个词汇就是法国人给他们的。那么他们之所以这么成功,靠的就是一般群众不懂逻辑,所以很容易欺骗,就是他们完全违反逻辑辩证的规则。



王孟源 09:55 

我进一步地再详细解释他们现在这个盎萨的逻辑宣传,这个宣传洗脑假新闻系统是怎么运作的,但一般是设定一个narrative,这个 narrative 有的时候是自然涌现的,有的时候是全是几个幕后的势力集团之一订立的,然后其他的势力集团也接受。那么这个就成为一个narrative。什么样子的narrative,中国是邪恶的,中国对新疆回族是有严重的种族迫害。这就是一个narrative,中国是极权 的。OK,那一旦你设定了这个narrative,他底下的几百家报社跟传播组织,他那个总编辑有的是有默契知道一知道要遵从这些 narrative;有的则是因为这些很多,这些narrative是成系统的,几十年甚至上百年传承下来,所以他们是真心相信,所以就可以很自然的做出一个 convincing 的表象。



王孟源 11:31 

比如说这次俄乌战争,明明是乌克兰完全没有打赢的可能,但是这个盎萨媒体在过去这三个月里面就是只要是从乌克兰出来的虚假宣传,他们一律就当做是真的照登,那你如果只看他们的这些主流媒体的话,你就会以为其实俄军快要被消灭了,事实上刚好是相反,俄军打得有板有眼不是完美,但是已经是算是可以得一个 A了或至少A-。那但是你去看一些这些主流媒体里面一些比较离谱的,像是最离谱的就当然就是德国之声 DW 或者Atlantic、foreign affair,是 counsel of foreign relation 自己的刊物,所以当然是也是最离谱的,那这个 point 这里就有点就是提供了。我刚刚讨论了两个类别的案例。 foreign affair 本身是就是权力核心的一个直接控制的一个机构,所以他们的那个编辑是完全了解这些内幕,了解有必要撒谎。但是这个 Atlantic 跟了 DW 你去看的话,他们特别狂热的原因是因为他们那些编辑真的就是最狂热,最没有逻辑能力的宗教信仰分子。信仰的是什么呢?就是盎萨这套宣传洗脑体系。 foreign affair 是是一个有益的机制。 DW 跟 Atlantic useful idiot。



王孟源 13:34 

我认为要破解这种宣传战的对症下药的最强的武器,就是科学方法跟逻辑辩证,所以我今天开场的时候特别去谈这些事情。那刚刚谈到这个乌克兰,乌克兰的战事,其实不只是欧美或者是盎萨的媒体有意地去扭曲这件事情。我去看中国的那个军迷的那些网路,他们也有很多偏颇,就是没有那么偏颇,但是也是颇为偏颇。这个偏颇为什么呢?因为他们也自己形成了一个narrative,而这个 narrative 不是像盎萨那样子,有一个核心组织定义的,就是他们的这个 narrative 就是我们中国的解放军比俄军强。我们的军工比他们强,他们有的问题我们没有,他们很烂,那么我们比他们强好一大堆。那这个呢?是人性的,人性的本能就是你总是想要自己脸上贴金,因为这些人都是这些所谓的军迷记者,他们靠的就是那些消息,因为军事消息就是封锁的最严厉的,而且甚至还会意识地放烟幕,所以你这些军迷去讨论这些军事消息的时候,最难得到的就是内线消息。所以你如果当一个记者,自己最关心的就是跟那些内部人员的关系联系和人情,所以你去看这些记者,即使是大机构的记者,他们所谈的人你都可以看出他们是在鹦鹉学舌。就是他们的认识的那些内部人员跟他说了什么,他把它转述出来。



王孟源 15:41 

比如说几天前,这个是真的是三四天前,在观察者网习亚洲写了一篇文章,里面我从他的叙述去看中国内部军事中层的军事人员,跟底层的军工人员所谈的,我就觉得不太好,因为就是很主观,完全陷入了我刚刚所说的这个自我吹捧的narrative,没有遵从我所说的科学方法跟逻辑辩证规则。然后我举个例子好了,他在那个文章里面谈到说俄军用了把苏联时期的 240 millimeter 的重自走迫击炮拉出来叫做tulip,然后来轰击那个亚速钢铁厂,他就说这个这种炮弹,这种迫击炮是老式的炮弹,然后这样轰显然也是不合时宜的,也不知道他们能有没有用那个激光制导。可是其实如果是真的内行的话,去看两个礼拜前这个重迫击炮被公开的那个视频啊,你可以明显地看出它发射的那个炮弹的形状就是激光制导弹。就是事实上,我们大家看到唯一的一个公开的信息就已经证明俄军已经把该做的都做了,就是他们去炸那个亚速钢铁厂的时候,已经用的是激光制导炸弹了。但是你中国的军方呢?就坐在那里觉得我们,我们中国有钱,我们的电子工业比俄国发达,所以俄军一定还是在用 40 年前的非制导的弹药。这个我觉得这种心态非常的不好。你一方面显示自己对细节不用心,那一方面也显示不够客观,让自己主观的偏好影响了的分析的结果。



史东 17:52 

你说他们这种结果是想当然尔。



王孟源 17:56 

想当然尔。对就是因为,人性本能就是觉得要自己往自己脸上贴金,因为这跟你自己的单位将来的资金有关系,所以然后我自己人在谈的时候就都是大家互相吹捧,那么因为这样,结果在分析的时候也想当然而地就去贬低俄军。那事实上流出来的信息,你如果仔细去看那个细节,已经可以证明不是你所想的那么烂。这个我觉得我其实那个时候就想要评论,因为我过去这两年,我因为觉得这些军迷他们的那个素质特别差,就是他们对客观的科学方法跟逻辑态度的修养特别的低,因为我不太喜欢跟他们牵扯。但是这件事情因为很明确,所以我原本想要就在观察者网留言,结果因为观察者网的网站它的那个程序写得很糟糕,一天到晚宕机,我的留言长往往有一半是进不去,这个留言也进不去,所以就没有在那里留言,在这里跟你说。



王孟源 19:06 

所以我想大家要在看现在这个俄乌战争的时候,要理解,当前的俄乌战争其实是有三个层面,就是军事、金融经济,还有宣传。俄国人其实是很高明的,他们这个 RT 之所以能够做到这么成功,就是我今天看刚好看到有一个评论者说的,所以我借用他的话,他说那个RTdoes propagandawith truth。



史东 19:40 

哈哈哈,好厉害。



王孟源 19:43 

好厉害,对不对?你看看你这个破解谎言帝国的关键,就是坚持真相嘛,对不对?坚持客观真相,对不对?



史东 19:53 

所以说这些西方国家、美国、欧洲必须要把它关掉。



王孟源 19:59 

所以过去这两年不只是RT,即使他们先把中国的那些新华社什么都关了,新华社他们根本一点作用都没有。实际上美国人根本不必怕他们,只是因为被 RT 打怕了,所以连带的话,中国的那个媒体关掉其实是被连坐的。真正成功的是RT。



史东 20:24 

那你觉得?话讲到这,我就顺便带一句,你觉得中国的这个人应该有什么可以向 RT 学习的?



王孟源 20:31 

觉得他们是有心,但是因为经验跟 expertise ,专业能力还不够,的确是还有很多学习的空间。觉得还好了,你可以看得出他们是有那个企图感。



史东 20:47 

哈哈, trying 是当然 trying hard。



王孟源 20:50 

所以同样的,他们的外交部也是这样子。你这个外交部,也是大概 2019 年才开始认真地想要主动地反击,那三年下来也是有可见的进步了,但是当然还没有像俄国人那么样针锋相对的,所以我觉得这个像是这些外宣呐,还有尤其是外交部了,其实不是中国的问题,因为这些人都是有心要改革了,真正问题大的是那些内部已经烂到不可以,然后还多方掩饰的。那这里面最糟糕的就是学术管理,比如说,昨天我又看到哪一个大学有学生作弊,结果还留校查看,你这个作弊然后考进保研被抓了,他们还威胁那个枪手。就是这个作弊的反过来威胁那个枪手,结果被抓了以后还是留校察看啊。这个真的是很糟糕,这个正确的处置是立刻把校长开除,就是你这个校长没有处置的话,我就处置你校长。至于他们这个教育部教材搞的这个事情,就是唉。



史东 22:18 

你说他们有个是不是叫黑教材?还是他们有一个词?



王孟源 22:20 

对这些黑教材的事?对。



王孟源 22:23 

这个我在我的博客上从 2014 年就已经解释过台湾现在思想的问题,就是当年教材改掉的问题。香港也是一样。然后你再看的,新疆自己也已经有切身之痛。然后你看得再远一点,这个印度为什么 Modi 能够有这么强势的政治成功?或者是 Erdogan 在土耳其崛起成为一个强人,他们靠的都是掌握了基础教育。这个我在博客里面都有专文解释他们是怎么做的,都是花了二三十年掌控了基础教育的那个管道。然后才有了全面的民意支持。这个他们说的是不是实话?不是实话。但是他们就会成功了。那这么重要的东西,我想中国中共自己有一句话,这个舆论高地不去占领,自然敌人就会去占领对不对?这怎么。他们将近 100 年前就懂得道理,现在全都忘光了?其实是很奇怪的事情。



史东 23:40 

对,我个人的觉得他所谓他们就是当局者,并不是不知道这件事情的可怕性和不可为,而是下面执行者是因为相当程度的利益的交割、利益的切割以及利益的这种牵扯,而是他们或者是自愿地去做这个事情,或者是被人家买通了去做这个事情。不管怎么说,这个危害,对国家的危害非常的严重的,不可以道理计的了。



王孟源 24:11 

在过去这几个礼拜,其实刚好跟我的博客的读者一连有一系列的对话,谈到这件事情,就是我们大家的共识,是因为因为信仰迷失了,所以大家,这些中层的官僚,他们进入体系其实是为了自己谋权谋利,而不是为了要为人民服务,所以尸位素餐或者是反过来以公谋私是很自然的结果。这个思想方面论于形式,以至于越是体制内的人越是没有信仰,越是打着红旗反红旗,这是中共当前很严重的一个问题。固然像盎萨这样子完全靠谎话来维持最高阶层的精英的利益,这个也不是一个长久之计。但是中共自己明明占着道德的制高点,然后弃守,然后任用每一大堆给自私自利的人,你这个中共,中国共产党的这个制度还有它的传统都是为了底层民众谋福利的,那现在这个沦为一个口号非常可惜的事情,因为在盎萨的霸权之下,不只是中国的民众,全世界的第三世界民众,甚至欧美国家内部的底层民众都需要救赎,都需要外来的救赎,而当前最好的希望就是中国。我觉得这是一个非常非常不幸的事情,值得值得体制内外的有心大家一起努力。



史东 26:07 

对,谈到这个事情,孟源我,我就把我的想法和你沟通一下,因为我对这个事情我可以说有相当程度的矛盾在里面,一方面这个我完全同意,就是对于这一次的这个教科书的事情的发现,我是非常的吃惊,而且非常的愤怒。而且我最觉得当局方面对这个事情能够忽视了这么久,这都是完全不可原谅的。在另一方面,我也不希望看到一种完完全全的单方面的一种价值观念的传播,这两者之间你有没有一个好的方法把它两全其美起来?



王孟源 27:01 

我相信正确的方法就是我一开始所讲的,根据科学方法跟逻辑辩证,用数学级别的严谨度,来证明政治上的道德需要,跟行为标准,然后这才是正建立健康思想态度人生观的关键。所以事实上我写博客写了 8 年,这也是一个很重要的方向,我一直都是在教导。比如说我一开始的第一年就写了一篇文章,叫做政府的第一要务,你这个因为盎萨到处传播什么民主是普世价值。这是胡扯!你必须要做逻辑辩证,必须要仔细地想清楚这个政府是干什么。所以你第一个要回答的问题是设立政府是要来干什么的?他的第一要务是什么?我在里面论证的结果是政府的第一要务是扶贫。你从这个论证出发,然后就可以解释为什么有理想、有关心公益的人必须要参与政治,然后确保公益被维护。这个不只是体制内,而且是体制外。而且今天才刚刚,今天又有人重新问这个问题,这个也是我过去四五年来回答了好多次的问题。那就是说,你不论怎么样,总是有一些坏人可以,那你怎么处理这些坏人?我的回答,我这个我已经我说过,我回答了好几次了,就是你面对这些这种一个群众,人性天生就有差别,你可以先假设这个一个大数目的人群里面有 20\% 是好人。什么是好人?就是是他对损人利己的这个诱惑有一相当的抵抗力的,相当天生的抵抗力的他有良心,然后有 20\% 是坏人。这些坏人是什么意思?就是你明明是有很严的防范,他还是会想办法去损人利己,甚至损人不利己,因为他们喜欢害人,他们觉得害人很刺激,但是中间 60\% 是那普通人。那你这个一个执政者,一个体制的设计者,他的这个关键的要点就在于奖励好人,惩罚坏人,然后通过这个奖惩的机制来鼓励普通人,也不要做损人利己的事情。你如果成功的话,这个就叫做清廉健康的风气。我相信这些道理习近平都懂的。因为他一上台的时候,搞反腐的时候,他讲的那些话基本上都是根据这个原则来做的。问题就是贯彻,那贯彻的话,他建立了这些奖惩机制显然还不够,而且比较基本的那种软件,就是思想整体的世界观跟人生观的建立,没有特别去掌握。。



王孟源 30:44 

你看过像是方方这种人胡说八道,但龙应台在台湾胡说八道,那是因为台湾已经堕落了。但是方方在大陆,台湾胡说八道,他一样吃香喝辣,他完全没有任何的后果。这种事情你在美国的话早就被 cancel 了,所以我觉得大陆的这种对思想的完全放任是非常不好的事情,你明显的错误的事情就应该惩罚。那你会说这个跟是不是就要重到文革时候的一家之言?我觉得不是,这个标准应该是我今天开场的时候就讲的,标准不是你反对什么,不让你反对什么,而是要求你反对什么的时候,你要根据科学的方法讲事实、讲逻辑来论证。你不能够无端地就说我反对。你要证明,即使你反对 1 + 1 = 2 都可以,只要你是照着规则来。说我认为你的逻辑推论在这一步走错了。这都是合理的。你看我的博客留言也是这样子,我那个删除留言不是看你的立场,而是看你讲不讲理。



史东 32:06 

你觉得讲理这个事情就是最重要的一个关键词。



王孟源 32:13 

对,那方方她有没有讲理?没有嘛。那你不讲理的时候写一些感性的散文或者是虚构的小说可以,龙应台也应该就是专注去写散文骗骗那个小女生。但是你一旦谈公共事务,这种是错误或愚蠢的决定,是要命的,甚至可以是要百万以上的命。你不能够这样容许这种胡扯,不讲理的人就没有资格讨论公共事务。



王孟源 33:00 

所以盎萨思想的之邪恶就在于这里。它这个所谓的只选民主制是普选,而是普世价值这个说法就是利用绝大多数民众没有理性思维能力的这一点,所以只要资本掌控了大众传媒跟基础教育,他们的地位就会一直地固化下去。



史东 33:43 

提起公共事务,你对上海最近的情况你有什么感想?



王孟源 33:48 

Well,这是整个思想腐烂,除了教育部之外,整个地区都有问题,就是从人民到官僚都有,都出了问题。这是骇人听闻,让人……



史东 34:01 

我觉得往好处说,这个事情经过之后骇人听闻。我非常同意这 4 个字,我希望有关当局这之后能够好好地反省,真正地反省,讲道理的反省,也既有你的话讲道理道的反省,好好地把这个错误在哪里,好好地把它改正。我觉得这也未尝不是一个好事。你和老百姓之间的交流,跟他的接触,跟处理他们的抱怨,跟处理他们的 complaints 的时候,我觉得在执行这方面这些人都非常非常的业余,就是amaterish。



王孟源 34:43 

我觉得不只是志愿者的问题,这个你如果追根究底的话,最终还是思想的散漫和信仰的缺失。



史东 34:51 

对,这个我完全同意,我只是从志愿者这层开始在观察,当然上面的这种…… 



王孟源 34:59 

上面不监督的话,你怎么能够指望志愿者我就做到完美呢?这是不可能的。所以志愿者做错了,而且是普遍地做错,这一定是那个中层或上层的问题。



史东 35:13 

对,观念的下传。



王孟源 35:16 

对,就像我刚刚讲的那个,你不能够……教育部长不能够指望说每一个学校遇到作弊或作假就严格处置,但是他们做出和稀泥的处置的时候,你不去处置那个校长就是教育部长的责任。



史东 35:36 

那你自己就是在和稀泥了。



王孟源 35:39 

你自己就是在和稀泥。对所以,现在这个问题已经暴露,就是整个上海的地方官僚体系是完全复旦。那这个我们就看二十大,今年年底二十大还有不到半年,看看二十大这个中央是怎么处理这些烂人。我一直都说习近平上来固然是一个非常必要的改革,但是他的反腐还有纪律的要求,问题不在于太严,而在于还不够严。像我刚刚讲的有关思想上、舆论上讨论的事情,你正确的原则不是——当然你不能因为哪一个人的立场而去处罚他——但是你必须要求他们都是讲理的,那些不讲理的的散布谣言的胡扯。现在的处罚还是不够。尤其是体制内负责传播思想的那些教授学术,这,你连政治方面的宣传的,党校里面的教授胡说八道,传播盎萨的那些洗脑资料都没有任何后果,你怎么能够希望在自然科学界,那些靠专业知识来诈骗国有资产的那些学法会能够有受到一样制约?不可能。



王孟源 37:22 

我认为中国的改革最大的困难就在于处理学术跟思想性的问题。诶,现在这个教育部还有这个上海的问题暴露出来,我觉得是一个契机,但是能不能够把握到什么程度,很难说,就是我希望能够至少把政治性的思想的纪律贯彻下去。这个纪律是什么?不是你一定要遵从党的立场或者是国家的立场,而是你一定要讲理,你不讲理的话就没有资格发言。请你闭嘴。你如果坚持不讲理在那胡扯的话,你就没有资格教授学生来毒化他们的思想。



史东 38:10 

对,我曾经好久以前我有这么一个想法,我说最好的一个国家或者一种社会是一个什么样的社会?我想的时候,当然我们跟我们今天讲的题目没有什么关系,但是我想出来在这句话跟今天我们讲的话有点关系,我所憧憬的是一个有理走遍天下,无理寸步难行的社会,一个国家。



王孟源 38:36 

对,就是大同世界。我过去这8年说来说去、翻来覆去地讲,就是你必须要根据逻辑、根据理性、根据科学方法来讨论公共事务,对不对?然后我示范了这套体系思想方法的,它的那个成效就是其实它用它来做策略、战术上的分析,或者战略上的分析,其实这还是小事;你治国还有社会,文化,各方各面,反正只要是公共事务都应该是要讲理讲出来的。有人说我这做这些分析,就跟那个鲁迅评论诸葛孔明一样,是多智而近妖,其实这不是重点,你这个多智而近妖不是重点,重点是你必须要讲理,才能有一个健康的社会,才能够大家群策群力贡献自己的知识,然后贡献不同的角度,来真正讨论对群体的最优解在哪里。政治就是寻求最优解的一个过程。但是如果你的原则就不容许讨,理性地讨论,你就不可能得到最优解。如果你的思想指导不是社会主义,你就根本不会想要去追求最优解。



王孟源 40:34

所以盎萨思想的之邪恶就在于这里。它这个所谓的只选民主制是普选,而是普世价值这个说法就是利用绝大多数民众没有理性思维能力的这一点,所以只要资本掌控了大众传媒跟基础教育,他们的地位就会一直地固化下去。





只选民主制是普选,而是普世价值这个说法就是利用绝大多数民众没有理性思维能力的这一点,所以只要资本掌控了大众传媒跟基础教育,他们的地位就会一直的固化下去,对,真相跟理性永远都是高于多数决的。你如果多数人都不同意,以前美国曾经投票嘛,要投票通过圆周率是多少,结果他们州政府马上就通过了,圆周率不是3.15926,(编注:参见1897年Indiana Pi Bill,,该法案认为圆周率是3.2,印第安纳州众议院零票反对通过了法案)。你说这样的社会真的是普世价值?



史东 41:20 

这种笑话很多很多,非常非常多,就是不同形式的方式出现在我们的眼前,所以说我觉得,我刚刚,有次机会我想稍微做一个小小的总结,就我们到目前谈到这样,你希望的你追求的并不是人们对你思维结果的接受,你希望的是人们对你思绪思维方式的接受。



王孟源 41:52 

对,因为我的结果不能保证是正确的。OK,我虽然是很谨慎让我的逻辑分析都是可能有 99.9\%的正确度,但是我还是必须基于公共信息,那这些事实可能有误报,可能有 5\% 10\% 甚至 20\% 的误报,那有这种就有 5\% 10\% 20\% 的结论会是错的。就像我当初预测这个俄乌战争在2月打不起来,那。



史东 42:25 

我顺便地好奇问一下,有你介绍了多少这种反馈,说人家说你怎么说错了,还笑你或怎么样,你收到了多少?因为你这个信息也是在我节目中也放出来,对不对?然后我也这边也收,但是我觉得我收到的并不多,的确有人有这种反馈,你那会很多吗?



王孟源 42:47 

我相信都是新读者,如果是我的博客的老读者还不明白这些我刚刚讲的这些道理的话,那他这几年都是白读了,对不对?这就是新听众,因为你这个是不是我的主要管道,还是我的博客,所以通过这些其他管道、次要管道的时候,总是会接触一些对我不熟的那个听众,那他们就会问一些比较奇怪的问题,所以我今天特别要把这件事情也提出来讲清楚,因为它其实是我们正确思想的一个隐性的核心跟基础。你要做任何这种公众讨论,就像是我们这样的视频,当然是有公共讨论的一部分,对不对?要做公共讨论就必须是理性的,理性的讨论就应该是数学性的,完全基于事实和逻辑所做的推论。你感觉怎么样、我们人是一个群居的社会你要想要交什么样的朋友?想要吃今天想要吃什么的菜?你买车要是什么颜色的,那个都是你自己主观感觉决定。你高兴,读龙应台的散文,你就去读;但是你不能够拿龙应台的散文拿来治国,因为那样的结果会很惨,有人会因此而死,你不能够任性,当有人命有关天的时候,你就不能够任性。



史东 44:27 

像我最欣赏的就是刚刚讲的,孟源,就是你提供的是一个思维的方式,而不在重它的结果是什么?因为你透过这个方式出来,你得到的结果大致上都不会差到哪里去。但是如果你专注这个思维的结果,而不顾你的思维方式的话,就是别人看你就偏了。



王孟源 44:52 

那就是买椟还珠了嘛,对不对?我真正要示范的是那套思维模式,因为你要建立一个正确的三观,其实只需要两点。第一个,你从儒家的人本主义出发。人本主义是什么意思? humanism,这就是人命跟人的快乐。至少从这一点很简单就可以用逻辑推论,正确的政治观是社会主义,对不对?因为你既然基于人本主义,你是要为人群谋求最大的福祉。那么当然就是要社会主义,因为要增进福祉,最简单的方法就是去扶植最弱势的那个群体,一旦知道这个方向,接下来你就会想要求最优解。而前面所讲的那个人文主义跟社会主义就定义了最优解是什么。最优解是对社会整体的利益最大化,人群的利益最大化。然后怎么样去求最优解?收集所有的事实,根据最严谨的逻辑来预测,就是我一再示范的那一套。拿数学的解题法来解政治跟社会问题的方法,就是为了求解,所以你先有基本的三观,理解到你必须要求为群体的最优解。然后接下来求最优解的方法是什么?就是科学跟理性跟逻辑。就这样子,很简单,我的博客写了 8 年,就是写这样。



史东 46:50 

我们稍微换一个话题,谈谈这个,这个战争,俄国和乌克兰的这个战争。往前看,你的看法是什么?



王孟源 47:03 

我的看法是乌克兰的兵员快要打光了,他们在过去这 8 年靠着北约的武器支持……其实现在证据已经出来,就是他们那些最先进的武器开始紧急运输进乌克兰。其实是2月初的事,就是仗是2月底打起来的,那美国的那个什么标枪导弹,那个英国的什么Ammo,那些都是2月初就开始送进去的,而他们是 2021 年的前半就已经安排好了这些管道,所以才能够一声令下,马上就送进去。



史东 47:44 

我不知道,这个消息对我来讲很吃惊,这也再度了,可以说证实了为什么普丁在那个时候要开战。提前要开战不是吗?这也是一个因素嘛。



王孟源 47:56 

我觉得是,是乌克兰准备在3月开打,我觉得是应该让乌克兰开打,然后你再反攻,就是你可以做好准备,你可能要让当地的居民是当地的那些民兵有几百人或者上千人的损失,但是你这个用兵就名正言顺。因为美国,这次美国跟英国盎萨集团搞,这次在乌克兰搞事,它的真正目的是什么?是为了要整合欧盟要重建北约的纪律,如果他们真正的目标当然是中国了,因为能挑战美国霸权的是中国而不是俄国。但是,如果直接在台海搞事的话,欧盟太容易就置身事外了。因为你在地球的另外那一边对不对?我干什么要去跟中国对干,而且德国还有那么大的商业利益在在中国所以他之所以先在乌克兰搞事,是因为那在欧盟的周边,理所当然的就能够裹挟欧盟一起去制裁。俄国然后你一旦建立了这个前例,再要求北约来支持美国在台海的任何作为,就容易很多。这是他的大战略的观点,这里的问题在于他们太贪心了。那至于他们为什么贪心,我也会详细地解释,他们太贪心了。就是你真正的战略目标是要整合欧盟,要重建北约的纪律。这个完全开战就可以做到了,就是象征性地做个自残就可以了,你完全不需要把 Putin搞下台。他们这个什么扣押外汇,踢出 swift 基本上就是要把卢布搞垮,然后搞垮俄国的经济,搞垮俄国经济以后,希望内部有政变,把那个Putin推翻。那这个完全都落空了,因为他们Putin已经准备了 8 年。他有一个很厉害的中央银行的行长 Nabiullina。



史东 50:31 

对,红得不了了。



王孟源 50:32 



很厉害。我从 2014 年一直关注他,就是觉得一直认为她是全世界最厉害的金融管理人员,因为基本上我想到的策略它都是很快第一时间就也想到,而且还执行了,而且还执行得完美无缺,真的是很厉害。所以在几年前我博客上就讲过他是个厉害角色。那时候我说,中国如果人民银行想不出什么办法的话,就直接去问Nabiullina,哈哈,该做什么照着做就行。Nabiullina 会建议什么?我相信她建现在建议的,也就是我3月的时候建议的那些东西,就是事实上最优解就是那样子,你只要有那个程度,有那个逻辑能力,自然就会得到同一个最优解。那因为Nabiullina的关系,现在卢布反而增值了。



史东 51:29 

比战前还高。



王孟源 51:34 

实际上这个金融战就西方就打输了。那另外一个就是宣传战,这个 RT 也是做得很好,我前面提过就是因为 RT 做得太好了,所以他被封了,而且还连累了中国的几个新闻媒体,几个完全无效的新闻媒体也一起被封。那这一次这个宣传战的俄国也很高明,就让那个盎萨拼命地搞一些假新闻,反正这些战事,他们认为三四个月就会急转直下,也就是现在开始急转直下,然后急转直下之后需要圆谎的是盎萨,俄方根本就不需要出来说什么。对不对?



我举个例子好了,他们那个外汇被扣的时候,大家都说他们被扣了3000 亿。这个 3000 亿这个数字,其实你如果去看当时的资料,俄国的官方根本都没有详细谈什么数字,他们只说我们被扣了很多。然后这个估计其实是盎萨媒体的估计。实际上现在有官方泄露出来了,当时大家以为 3000 亿里面有 2500 多亿是欧洲的中央银行扣的,美国扣得很少,其实刚好相反。现在真实的数字出来是欧洲只扣了 230 亿,美国扣了 1000 亿,所以全部扣的是 1230 亿,这个占的是大约俄国全部外汇储备的1/ 5。然后再加上现在这个油气的价格上涨以后,他们的外汇每个月的外汇收入反而加倍了,所以你根本没办法让他们破产,那你不能让他破产,那美国到最后就没办法,就是不让你的银行还钱,不让让美元的银行用美元还钱的话,什么时候会出现?就是下个月,还有三个礼拜,下一个支付会没办法支付,就是理论上没有办法支付,然后你就可以相信 盎萨的媒体会跳出来说这个制裁生效了,俄国的俄国破产违约了。可是事实上你如果去看那个合约,这种都是没有完全没有专业知识的人才会胡扯的事情。凡是签过商业合约的人都知道,这种商业合约里面一定有一个条款叫做不可抗力因素,对不对?就是在不可抗力因素之下,这个签约方没有责任,这很明显就是当初那个前面他们谈那个油气要求,用卢布付钱的时候,这也是很明显的可以(算作)不可抗力,为什么?你们把我以前卖油气的外汇扣押了?那我当然可以,那当然算是违约的事情嘛,对不对?所以我当然可以说这是不可抗力,所以你可以看到英美的这些作为,其实第三世界看在眼里,都是恨在心里,哈哈。所以在宣传战上面俄国真的是大获全胜,它的基本上就是反正我只抓实利,我抓这个实际上在军事上跟金融上跟法理上我占制高点,即使你把 RT 禁了,至少第三世界还看得到。而且事实上,你在三四个月之内,就会有严重地自相矛盾,到时候要怎么自圆其说是你的问题了。比如说一个礼拜前, Kissenger 刚刚跳出来说建议这个乌克兰割让一些领土,赶快和谈,OK,再早一点, 3 个礼拜前,美国的国防部长跟参谋总长先后地找俄国的对等人员来谈停火的事情。他们为什么要停火?如果乌克兰打得很好,他们还美国会要求停火?很显然的是这些专业人员看出,战事要急转直下。就是我也看出是急转直下,因为我也懂这些事情,所以我在两个礼拜前就已经跟大家讲说。对,眼看着要开始滚雪球了。 LLoyd Austin ,就是美国的国防部长去要求停火,是为了挽救乌克兰的局势;但是 Kissenger 出来出面是干什么?他出面其实是为要为了这些主流媒体,这些建制派的主流媒体做铺垫,我一直说这个谎言帝国里面,这个宣传洗脑体系,地位非常重要,它其实比政客的地位还要高,就是他们距离幕后的那些真正掌权的财阀核心还要更近,比这些政客表面上的傀儡政客还要更近。所以出了问题的时候,你这个锅可以丢给那些政客,但是主流媒体必须要自己想办法脱罪,要脱罪的时候,你可以现在就可以想象这他描述的是什么?描述的是出人意外的,因为欧盟不团结,送的武器不够快,送的资金不够快,大家不够……结果让俄国反败为胜。所以我们过去 3 个月前讲的那些都是真的,只不过是俄国反败为胜。所以到最后他们怪的会是谁?他们怪的当然就是德国政府,怪的是匈牙利,因为匈牙利不让你制裁那个俄国的油,他们会怪,甚至Zelensky都会怪,因为 zolanski 到时候可能已经政变下台。所以很适合背锅,但是绝对不能够背锅的是这些主流媒体,因为他们才是权力核心的喽啰,核心喽啰。





史东 58:21 

守卫者。



王孟源 58:26 

对,守卫者,这些都是很可以很简单地事先就看出来的。然后你再看俄方现在的那个反应,在头一个月他们做第一个阶段的时候,他们是真的指望乌克兰给吓得投降了,然后所以他们那个第一个月是真的很认真地在和谈,那个和谈是真的,他们的目的,他条件开出来是怎么样的?很简单嘛,就是你承认克里米亚是俄国的;然后你给那东乌两个共和国自治;然后你自己去纳粹化,去军事化,就这样而已,不要加入北约,很简单。但是这个条件现在还有没有?没有!不可能了!他已经打了金融战,打了被欧美无限升级的金融战、宣传战,在军事、资源,已经打了3 个多月。俄方的牺牲已经太大,而且已经……乌克兰的纳粹杀伤残害战俘的事情已经做得太多,他们不可能再接受这样的条件,现在能够接受的最低条件,最好的条件是东乌两个共和国把他们那两个周就是 Donesk 跟 Lugansk两个州,而不是他们目前占领的那一部分,而是整个州的全部通通独立。然后再加上俄国占领的南部沿海地带,也独立。就是就地停火这样子。





王孟源 1:00:32 

我想花几分钟谈一谈这个各国在乌克兰南部海岸线占领的那块地带。一般人可能不明白这一块地带,有一个巧合。就是2月 24 号刚刚开打的时候,俄国运兵长驱直入,除了在北部是直指基辅,这是为了吓唬人用的;但是他在南部也是长驱直入,然后但是这是真正要赶快地占领这些地方,所以他们现在占领的地方就是当初预定要占领的,是有意设计要来占领那你如果看这个边界线的话,它倒不是完全跟着目前现代乌克兰的州的边界,所以有点奇怪,这个问题的解决答案在于什么呢?你如果去看 200 年前俄国的地方的那个地缘的边界的话,当时他们现在这些南部海岸线的这一块,就是俄国占领这一块,再加上Crimea,刚好合起来是当时的一个省。



史东 01:01:22 

他们要恢复进去的是这个意思吗?



王孟源 01:01:23 

刚好就是一个省,我觉得这不是巧,哈哈,刚好是 200 年前的 Taurida 省的全部,然后而且更巧的是他们现在那个地方已换用卢布了。就是他们刚进的头一个月,因为是准备要和谈,所以当时并没有做这些事情,连那个乌克兰的旗帜都没有拉下来,就是他准备要和谈,然后这些地方要归还给乌克兰。但是一旦那个换旗帜就别提了,他们那个连货币都换了,而且连牌照都换了。最有趣的就是他那个牌照,上面写的就是 Taurida, Taurida 这个字已经有 200 年没有用了。



史东 01:02:12 

你想想看,这个对于Putin来讲,或者对于俄国人来讲一定是相当欣慰的一件事情。



王孟源 01:02:19 

就是他要重建 Taurida省,因为这些人都是真正的俄鱼区,这些人大部分的人口都是支持俄国的。而且 Crimea 本身没有淡水,你必须要从内陆的水库送水,靠运河送水过来。那这个 Taurida的在乌克兰本土的海岸线的这一块就刚好是屏障。这个运河的地段是有实际必要的。但是我认为既然Zelensky不可能马上投降,因为他实质他并没有实权,他的责任只是做表演。你是个演员,你就在……因为刚好那个盎萨媒体有需要,乌克兰的宣传机构也有需要,所以他可以表演,那表演的代价就是他可以随便贪腐。目前我看到有两种不同的估计,就是他摊了5亿或者是8亿美元,但是我没有时间去做那个会计的追查。



史东 01:03:25 

这可能是全世界待遇最好的一个演员了。



王孟源 01:03:31 

哈哈,Tom Cruise 都没有拿这么多。



史东 01:03:34 

对,不可能有这么高的代价对不对?



王孟源 01:03:53 

俄国拿掉以后,乌克兰不只是失去 1/ 2 的领土,还失去了 2/ 3的GDP,你剩下的这个僵尸,你说北约要重建,我觉得是 80 年都重建不起来。嗯,他们的真实的目的,是为了美国的政要到处全球搜刮。因为你实际上,你只要看看他们实际上执行的差别是什么。美国原本就在全球搜刮,但是在Neocon之前他们是有标准的,你去搞乱的对象必须是你的敌人,搞乱之后必须对美国有明显的利益。Neocon的这个差别就在于他不在乎这些,不在乎国家的利益,就是先搞乱了再说。



王孟源 01:04:56 

对,他不可能投降,因为现在这个实权就是他们的国安体系,都是亚速营。



史东 01:05:02 

他都不能投降,他不准他投。



王孟源 01:05:05 

都是亚速营那套人对在掌管的,那不可能投降吗?我觉得俄国,Putin会到什么时候才又有兴趣和谈?我认为是它目前这个边界是从海岸线还有它原本的边界推进了大概 100 到 150 公里,在乌东跟南部海岸线,我认为它还要再推进100- 150公里,就是这个条状会变成两倍的厚,大概两三百公里厚。那这里包含了 5 个大城市,俄语人口占多数,就是北部的Kharkov,中部的 Dnipro 还有Zaporizhzhia,然后南部的Nikolayev,还有更最重要的 Oddesa,是乌克兰的第二大城。这些都是,乌克兰的精华区,大家可能不知道,乌克兰东部不只是它的工业精华区,也是它的农业精华区。如果去看那个黑土地,乌克兰很有名的那个肥沃的黑土地,就是以前东北也有的那个黑土地,中国东北也有的黑土地。它的分布其实就是在东部跟南部,刚好就是这些俄语区。对,我认为Putin等到把他的下一步,就是再花两个月把东乌的现在的这个机械化兵团完全歼灭之后,它就溃散了,剩下的你人数再怎么多都是乌合之众,都是紧急动员出来的民兵或者是后备役,它很简单就可以再推进 100 到 150 公里,然后占领这些精华地带,到时候他就会有兴趣再来和谈。



史东 01:07:03 

对,所以说到时候,战后乌克兰就变成一个内陆国,没有海岸线。



王孟源 01:07:09 

是一个内陆国,而且是很穷。因为你不但丧失了工业区,你也丧失了农业区,那你还剩下什么?你这个基辅虽然是人口最多的城,它其实是一个政治的城市,就是都是官僚机构,本身并不生产什么东西。然后乌克兰的高科技工业,剩下像那什么马达西奇,它在哪里?它在走Zaporizhzhia,对不对?那些像样的还有一点价值的工业。还有技术全都在这些俄语区里面。嗯,俄国拿掉以后,乌克兰不只是失去 1/ 2 的领土,还失去了 2/ 3的GDP。你剩下的这个僵尸你说北约要重建,我觉得是 80 年都重建不起来。



史东 01:08:02 

最近不是说他那个和波兰之间有些眉来眼去的动作,你是怎么看这个情况?



王孟源 01:08:09 

因为那个波兰是有野心的,因为当初二战之后,斯大林当然是顺便割了一块波兰的领土给乌克兰。那事实上,不是太有智慧的事情,因为现在的这些乌克兰的纳粹,他们主要就是从那个地方发源的。我不能够预测,因为这些是很愚蠢的人。你没办法假设他们是遵循最优解,因为波兰的最优解就是不要下去掺和这些事。但是你不能够保证他们不干这种事情,如果波兰真的下去掺和,试图吃回以前的那一块领土的话了,他真的是会吃不完兜着走。因为你看很明显的现在欧盟的经济已经元气大伤,德国的 PPI 就是工业生产的通货膨胀,在最近的这个数据,4月的数据是百分之三十几。



史东 01:09:16 

他都活不下去了。30\%?天呐。



王孟源 01:09:21 

你连德国都这样都受打击的话,而且原本那个美国的这个经济货币超发,原本就会有一个全世界的经济危机,由美国引领的经济危机,那你现在欧洲自己找麻烦,反而是为美国输血。但是他自己损失的是 5 倍或者 10 倍。这下一来,我们这一波的全球经济危机的核心,就保证是欧盟。我们现在看到当前这个全球经济上面是乌云密布,这个很明显的有一场风暴要来,这场风暴的核心会是哪里?就是欧盟。你一旦这场风暴刮过去以后,你想波兰还会有有人为他撑腰吗?在这个政治、外交跟军事上为他撑腰来开疆扩土吗?这种局势,这么简单的局势都看不出来。



史东 01:10:26 

所以,你觉得,让我重新重复一下你刚刚说的话,是不是波兰现在以为他现在是欧盟和北约的一员?于是乎,他有恃无恐的有这种野心,希望能够拿回他从前他认为他自己在乌克兰的那一块土地。



王孟源 01:10:46 

因为现在美国通过他的带路人,像是欧盟的那些领导,还有德国的政府里面的白左,就是绿党的那些人,完全裹挟了欧盟的政策,所以波兰看到这个局势认为是一个机会,可以借机也揩一点油。既然美国的目的是要对抗俄国,打击俄国打到底,那么呢他也可以趁机去揩点油,趁机把他以前的领土拿回来。那这是非常短视而且不智的事情,因为到时候如果出了事情,北约跟欧盟都刚好是在经济衰退之中,他们很可能就会说我们的这个第5条协防协定,要求你是受到外来的攻击,而不是在你国界之外去攻击别人。



史东 01:11:48 

还有一件事情我想趁这个机会提出来,有关于这个在乌克兰境内发现了很多美国的这个生化研究室的这件事情,现在看起来,从片面的消息,我只能说片面,因为我不知道这些消息到底是正确还是不正确。片面的消息支持俄国似乎有一种想法,要是要做一个全球性的发布或者新闻记者会之类的,这种甚至把对他们所得到的细节公布出来。整个来说你对这个事情的看法,或者你对于事情有什么期盼没?



王孟源 01:12:30 

这很明显的是军事方面的生化合作,但是他有没有真正开始批量生产有效的生化武器,或者是针,尤其针对性的生物武器,我觉得这是没办法确定的。不论如何,这从俄国人的观点来说是抓到美国的小辫子,而且我刚刚已经说过,那是事实嘛,你只要讲事实,就已经。



史东 01:12:59 

就已经够了嘛,够了,对不对?



王孟源 01:12:59 

就够了。对,所以我觉得是合理的。你这个程度上有多恶劣可以再讨论,但是的确是真的。那你就抓着这辫子狠打就是了,对不对?要辩解是美国的责任。



史东 01:13:16 

你说到事实,我想到有这个,有一个人说了一句话,我一直觉得很有道理,天安门事变的时候有一个你大概是就是我现在一下想不起来。那个在台湾去的一个唱歌的歌手,他在天安门,他在回忆这件事情的时候,他基本上他是冲着那些天,天安门在那一天,发生了一些什么事,发生了事件,他说“我没有看到”,他说“难道说事实还不够吗”?这句话给我的印象非常深。



(编注,这里所谈的是,龙的传人的作者侯德健,采访原文摘录:很多人说广场上曾经有2,000人被打死或者是几百人被打死,在广场上有坦克辗压学生、撤退的人群等等。那么我必须强调,这些事情我没有看见,那么我不知道别人在哪里看见。我是六点半还在广场上,我一点都没有看见。我一直在想,说:我们是不是需要用谎言去打击那些说谎的敌人?难道事实还不够有力吗?那么,如果我们真的需要用谎言去打击说谎的敌人,那只不过是满足了我们一时泄恨、发泄的需要而已,那么,这个事情是很危险的事情,因为:也许你的谎言会先被揭穿,那么之后的话你再也没有力气去打击你的敌人了。)



王孟源 01:14:13 

对谎言帝国来说事实永远是不够的。



史东 01:14:16

你这话讲得很有道理。



王孟源 01:14:18 

我想解释一下谎言帝国是怎么形成的。OK,我们既然谈到这里,稍早我提到其实盎萨他们的天性就是喜欢撒谎,他们这个体制是由资本主导的,资本商人做广告,你怎么能指望他说实话对不对?他们原本就是天性,但是美国不太一样,美国是清教徒建立的,然后在 20 世纪初期,经过两个罗斯福的整顿,其实他有了相当的社会主义倾向,而且有一个强力的联邦政府。与其同时,他们也有几个很能干,很睿智的媒体人做了整顿,所以英国的媒体几百年来就是以撒谎为专业。美国的媒体在四五十年代的时候是有自尊的,至少在他们的新闻学院里面教的是,你必须要说实话,那出来以后,毕业出来以后,为了因为老板要求而说谎话,那是另外一回事,但是你至少在学院里面的时候有这个道德要求。这是什么一回事?当然是因为国家的体制的结构跟战略有转变,对媒体对外撒谎有了需要,所以他们才引入了英国式的不是不撒谎的这个习惯。



但是我说的这个政治结构转变是什么呢?其实一般人没有注意到,是所谓的Neocon,现在美国在国际上倒行逆施,搞的臭名远步全球,这个其实都是Neocon搞的。那我想,正因为他们主导了这些坏事,基本上每一件对外的坏事,最近都是在过去这二十几多年,都是Neocon主导的。所以你必须要有必要去了解他们是一个什么样的组织,什么样的体系。这个Neocon其实是 1960 年代发展出来的,开始的时候他们的三个大头都是犹太人,其中最有名的是Irving Kristol,他后来他的现在的这个 new account 的大佬叫Bill Kristol就是他的儿子。Irving Kristol作为一个标志性的人物,他们是很典型的,就是他们其实当时六零年代,你知道美国正在搞学运,所以是全面的那种左派的思潮。那这个Neocon这个词汇是怎么来的?就是那些极端左翼、极左派的完全反战的那些人,指控一些不够左的,像是Irving Kristol这些人,把他叫做Neocon。这里其实是在骂人,因为你那时候 60年代,你骂一个人是Conservative,是你这是一个反动派,哈哈哈,这意思,所以那就是说你们不是传统的保守派,但是你们是新的保守派,你们没有资格说你们是 progressive 或者是liberal。这是他们,其实是一个骂人的字眼。至于为什么要骂他们,他们其实有两点不同。第一点,他们是犹太人,而当时黑人跟犹太人有点对立。那真正的这些搞学运或者是搞自由派的极左的当然是支持反种族主义,所以他们是支持黑人的,那犹太人就不满。另外一个因素是以色列,因为他们为了要保护以色列,所以他们认为你不能够完全反战,你必须要支持以色列,所以他们在外交政策上,也是有不同的意见。所以这其实是,1960 年代左派中的犹太人为了自己的利益,被那些极左的自由派,指着鼻子骂而形成的。Neocon就这样子,然后接下来的 30 年,他们一直都不是什么主流,就是他们那边发杂志写文章,真正的转折点是什么呢? 1996 年的时候,有一个新的Neocon叫做Richard Perle,他写了一篇文章叫做 A Clean Break。这篇文章是为谁写的?不是为美国写的。而是当时Netanyahu当第一任的,他第一次当上以色列的总理,所以他就要求资助,给了 Richard Perle 一点钱,说请你帮我作为智库,帮我们做外交决策,做个建议。那 Richard  Perle 写出来的这个结果就叫做  A Clean Break。这个 A Clean Break讲的是什么呢?他讲的是说你不要管什么远交近攻,就是你不要讲着说试图像跟埃及妥协,然后集中火力对抗伊朗,这种事情你只要不管三七二十一,把所有的中东临近的国家都能搞烂了。打烂了或者颠覆了,或者是创造对立,或者创造革命,通通打烂了,你自然就会胜出。



这个决裂点(编注:即A Clean Break的中文意译 ),很快的就被美国采纳,也就是很快的就变成美国共和党主战派的主流。



史东 01:20:02 

这个和以色列在美国共和党之内的影响力有没有关系?



王孟源 01:20:10 

很大,以色列其实控制两党,这是为什么Neocon后来通吃两党的原因之一。OK。但是还有一个更重要的原因,为什么Neocon能够通吃两党。美国的政治原本设计来是党争重于一切,你能够看到有什么东西是两档左右逢源的?很少很少,除非你是钞票,这个钞票是两党都左右。但是这种是要花钱的一种的政策方向。能够两档左右逢源的,除了Neocon以外没有。这就很奇怪,你光是因为它是为了以色列设计的还不够,还不足以能够克服美国党争的这种天性。



王孟源 01:21:01 

你让我再解释下去。就是到了 2001 年,那个 George W Bush 上台之后,因为他智商不够,没有办法掌握实权,实际上这个政策的政策跟战略的决定权落到了 Dick Cheney 跟Rumsfield的手里。那这两个人都是Neocon的信徒,所以大幅地启用了Neocon。你很有名的几个的,Wolfowitz就是Neocon,但是后来影响最深远的,在当时还是一个很年轻很资浅的 Neocon,就是Victoria 

Newland。她拿到的职位是什么?他是驻北约的大使,也就是美国派到北约的监军。其实不是很高的职位,因为那个时候苏联已经瓦解了,所以北约并不是很忙碌,所以是一个闲职,就是中层的一个闲职。但是她后来就越做越大,现在是Neocon最主要的执行人。美国的Neocon掌权以后,它也就执行了这个 Richard Perle 当初在A Clean Break里面所建议的那个核心的思想。这个核心思想我再说一次,就是你不要搞以前的那种远郊近攻,就是有明显战略利益或者是国安威胁的时候你出手,其他的时候你就和气生财。OK,这个是传统的对外战略;Neocon的战略是你不管三七二十一,反正有机会你就去把它搞乱就是了。管他是敌是友,反正你就去搞,把人家搞乱就是了。这,就解释了他真正为什么能够通吃两党的原因。因为美国的政客在国内靠着 lobby 来拿钱的时候,为游说业拿钱的时候,是有很大的限制,而且要等很多年,你要退休以后;而且竞争者众,分到了你的碗里的其实并不多。他们最肥的资金来源,是到国外去掠夺,而且在国外掠夺和平时期你顶多就是为国外的财阀、银行或者是犯罪组织游说一下。拿个 100 万 200 万,这其实还不是太大的钱。



史东 01:23:35 

对,小钱。



王孟源 01:23:38 

小钱,真正能够大笔几亿的拿的,战争的时候,那个军费一开就是几千亿,那这里面这个回扣拿得就很爽了,他们吃了两个甜头。第一个是 1990 年代苏联解体的时候,苏联内部几十万亿的国有资产被以大约 1\% 的价值贱卖掉,一般盎萨的媒体句式,就是说这个苏联本地的土豪,就是Oligarch,把它趁机买了,但是你即使是用 1\% 的价钱也是几千万几亿美元,你这个钱到哪里去拿?其实出钱的是美国人,大部分的钱都到了美国财阀的口袋,那就是美国 90 年代经济一下好起来的原因之一。另一次很大的吸血掠夺是在 1997 年那个金融危机之前,就已经先对苏联搜刮了一次,那个Oligarch只不过是他们的手套,赚的只不是一些经手费,但是他们一样是变成billionaire。对,然后在 2003 年,这些Neocon真正的,把这套政策系统化的付诸实施之后,那更是离谱。伊拉克战争跟阿富汗战争为什么要打那么久,为什么每年几百亿的往里面丢?因为都是大这些政要的油水。他们 2003 年打进巴格达的时候,一些喜欢附庸风雅的政要甚至还组团雇了佣兵去伊拉克的国家博物馆,把那些 Sumarian 的古董通通抢回华盛顿他们家里去了,那些都是无价之宝,但是就这样子,没有人,没有人知道,没有人理,没有人追究。然后每年几亿的进账这种好事情,他们的甜头吃过以后就不可能戒掉。



王孟源 01:25:54 

所以 2008 年政党轮替,奥巴马上台,奥巴马是一个很清纯的局外人,但是他是很有长远的眼光,他知道他可以晋升这个幕后的权力阶级,就是势力集团之一,就像 Clinton 那样子。 Clinton 也是 8 年总统做下来以后成为一个派系,其实是民主党最大的派系,她那个他出来选的时候连 Biden ,副总统 Biden 都不敢,在 2016 年都不敢跟她对抗。为什么?因为她是最大的派系。Obama有这个Cliton的前例可循,他也是希望能够建立自己的派系。但是你要建立派系就不能够得罪人。对,即使奥巴马自己不想去拿钱,但是你也不能够去挡人家的财路,对不对?这就是为什么,我们以 Victoria Newland 为例好了。她是知名的Neocon,在共和党做了 8 年,而且这做了几年,然后到 2009 年卸任,你要记得当时 Obama 上来的时候,他的口号是什么,是要扫除旧有积弊。重新从头出发的结果。 Victoria 诺兰居然升官了,马上就成为assistant secretary of state,是助理国务卿。那当时的国务卿是谁?Hillary嘛?没有为什么,因为 Hillary 看上了那几亿的进账,当然要靠这些Neocon来为她开辟财源。她一开始是先做 Hillary 的Spokesperson,然后后来 Hillary 退休以后,她升到助理国务卿,然后她的朋友有一个,她的一个好友也是 Hillary 团队的,转正到Biden去当国安主管,就是大名鼎鼎的Jake Sullivan。



史东 01:28:09 

都是一伙的,一窝子。



王孟源 01:28:12 

都是一窝的。对, 这一窝 Neocon 或者 Neo Lib。不管你怎么叫都是犹太系的,他们的责任就是全世界放火,到处放火。然后让美国的政要能够浑水摸鱼。



史东 01:28:29 

对,就是我在这顺便插一句,就是我所理解的 Neocon就是他们的诉求,当不是全部的诉求了,就是在国际之间在外交上,他们的诉求基本上就是在,特别是在苏联瓦解之后,他们看了一个机会,他们就是美国。这是美国独霸全球的一个机会。



王孟源 01:28:51 

那个是骗外人的OK?哈哈,骗外人的?你实际的用意是要搞乱世界,然后浑水摸鱼。但是你总不能够公开地说我们的教义就是这样。一强独大,所以可以重整国际秩序。这个都是骗人的,他们的真实目的是为了美国的政要到处全球搜刮。嗯,因为你实际上,你只要看看他们实际上执行的差别是什么,美国原本就在全球搜刮,但是在Neocon之前他们是有标准的,你去搞乱的对象必须是你的敌人,搞乱之后必须对美国有明显的利益。Neocon的这个差别就在于他不在乎这些,不在乎国家的利益就是先搞乱再说。



史东 01:29:41 

而且他的这种手段也是前所未有的话,比如说什么 Preemptive strike 这种理论的。



王孟源 01:29:50 

对,最重要的还是,他以前的美国的宣传还要脸,很多的新闻去教育,Neocon上台以后,加时加码地工作,所以也就落出破绽,也就给了 RT 用实话来破解他们的宣传的一个机会。



王孟源 01:30:28 

如果中国的崛起的意义只在于把目前主宰世界的那几百个。白种的脸孔换成黄种人的话,那我没有兴趣做付出,我做出这些付出完全都是为了理想,这个理想不是为国家社会,而且也是为全人类。是因为我算过这是对全世界人类整体利益的最优解。但是在自然科学上面,美国的自然科学还没有腐烂到这样的假大空地步,结果中国一个新兴的国家反而比人家,比一个衰退的帝国,还要腐烂。我认为不但这是长久未来最大的隐忧,而且是我良心最过不去的一件事情。诶,我希望我的听众跟读者之中能够有体制内的关系,能够对传播散布这样的这个知识能做出贡献的人。也请你尽一己之力来做出这个贡献,因为我认为这是我长期内部改革最大的一个难关。



王孟源 01:31:56 

10 天前我讲了一句话,结果有很多新观众很不满,我说中国在外交战略上基本毫无作为,在过去 20 年有很多危难基本上是运气好,就这样子毫无作为的就顺利度过。有点像唐三藏这个,他的徒弟孙悟空急得满头大汗,那他自己什么都不干。 这一次的俄乌战争也是一样,我刚刚我稍早的时候讲过美国,要先从乌克兰动手,原因是只有在欧洲边搞事才能够确保欧盟会站队,然后欧盟站队以后才能够以及全球所有附庸的力量集中起来对付中国,当然这里的前提就是你必须要胜利才行,要是失败而且败得一塌糊涂的话,人家这就反过来而反作用嘛,对不对?那我刚刚说过他这个金融战已经失败了,宣传战也已经失败了,等到这个军事打到秋天,俄国占领了 Oddesa 跟 Khakov 跟Dnipro,然后乌克兰整个经济完全崩溃,财政经济完全崩溃,甚至很可能会有政变。基本上,一旦进入第二阶段,Putin不再想着要和谈的时候,Zelensky就没有价值了。怎么样对他?Zelensky,原本对他的价值是在于有一个名正言顺的谈判对象,一旦他不在乎要和谈了,就,Zelensky死活对他就没有关系了,对不对?然后你想想看,等到盎萨媒体需要找一个替罪羔羊来说,为什么你们拿了我们这么多钱,结果还打得这么惨?一是被你贪污了,的确也是被他贪污了,但是真正贪污最多的是美国的政要,对不对?因为你看他那 400 亿拿出来以后, 250 亿莫名其妙的用途基本上都是花在国内,那基本上就是给国内的那些。



史东 01:34:11 

对,你看了那个明细的,看了那明细表吗?



王孟源 01:34:14 

我看了那明细表,里面有九十亿,说是要填补那些已经送给乌克兰的武器,所以这些是给了 Raytheon或者是Boeing,这些做那些标枪或者是什么导弹的那些公司,对不对?他们也是要拿回扣的,对不对?你这个九十亿对自己国外价,而不是美国国内价 3 倍的价格,你这不拿点回扣不好意思嘛,对不对?然后真正说要拿来交给乌克兰的是 60 亿,60 亿里面还有我的武器给你之后,我派人训练,我一个小时要你的15万美金、所以你说,指望拿这60亿,能够买到足够的先进武器来扭转战局的,是痴人说梦,所以大家根本就不用理会这个可能。基本上这 400 亿是要趁战争急转直下,所以根本没有理由再继续扔钱进去之前,赶快地把 400 亿通过,所以他们才能够往自己的口袋里面塞钱,这才是真正的用意。



史东 01:35:30 

可以分赃,对不对?



王孟源 01:34:50 

可以分赃,对。然后,所以你从这里就可以简单看出,再过两三个月乌克兰被彻底打垮之后,他们有必要保护着 Zelensky 吗?  Zelensky 其实就是最适合背锅的人。可以确定 Zelensky 面临这个政变的威胁的时候只能自求多福了,不能指望俄国人帮他,也不能够指望美国人帮他。



史东 01:36:00 

我相信 Zelensky 一定把他自己的退路已经安排好了。



王孟源 01:36:05 

听说这个,当然他的家人现在都已经逃到以色列了,在以色列有豪华公寓,但是听说Boris Johnson 也给他的家人都发了护照,他的退路是已经好了就,但是大家不必指望说他做总统做到年底了,那个机会并不是太大,OK。



史东 01:36:29 

当然最可怜的就是老百姓了。



王孟源 01:36:31 

最可望有钱人早就已经趁机逃到欧洲了,现在就是当难民。那最倒霉的永远都是底层民众。所以我才说过第一个要有社会主义的胸怀,理解你这个一切的政治,要第一优先,都是要照顾己底层民众你如果真的知道这一点的话,你就不会去胡搞这些事情,你就不会放任自己国家打残了,你就不会放任贪污腐败,因为贪污腐败最大的受害者就是这些底层民众。



史东 01:37:11 

这个世界的动乱就是不管怎么样,它的乱源总是在一个地方,一个国家。



王孟源 01:37:19 

所以盎萨原本,在过去几百年是为了维持他的霸权是,无所不用其极,但是至少还有点智慧,Neocon上台以后变成政要捞钱是最优先,你想想看,连党争都比不过他更优先了,那个你更不要提国家利益了。所以你说中国取而代之是不是理所当然,是人类的一大形势。但是这个这里的前提是中国要能够整顿自己内部的思想教育,不能够放任像上海那样的,或者是教育部那样的腐败,或者自私自利的态度。要不然你只不过是把盎萨的霸权,换成黄种人的霸权,少数黄种人的霸权那没有完全没有意义。我以前在博客就说过,如果中国的崛起的意义只在于把目前主宰世界的那几百个白种的脸孔换成黄种人的话,那我没有兴趣做付出,我做出这些付出完全都是为了理想,这个理想不是为国家社会,而且也是为全人类。是因为我算过这是对全世界人类整体利益的最优解,不是出于我个人的什么国族的偏见。



史东 01:38:51 

有没有结论可以给我们做一个。



王孟源 01:38:54 

结论,就是我希望我新的读者,尤其是体制内的新读者。我在过去这 8 年谈的除了一个很重要的是外交战略,另外一个很重要的是经济发展,然后最近比较着重的是金融货币政策,然后最后就是学术管理。我为什么用这个排列?因为六七年前我开始介绍盎萨的邪恶的时候。我是一篇博文只讲一点点,你如果回去看,有十几篇、 二十 篇博文,每次只讲完全没办法反驳的事实。一直到 2017 年、2018 年,美国的贸易战打起来,我才跳出来做了一个总结,说盎萨是世界所有乱象的泉源,它对中国的用意是要谋杀,所以中国退无可退,一直到 2017 年、 2018 年我讲那些话的时候还是只有我一个人这样讲,而且大部分人包括体制内的所谓学者、专家都把我当疯子一样看,但是现在已经是完全是主流了。你现在要是敢说个我们还是要中美夫妻论,或者是要维系全球化了,这个你会被人家笑,所以这个我已经成功了,以后我就不会再回去啰嗦,就没有再啰嗦的必要。



同样的我谈这个经济管理,我再说美式的,这个 MBA 式的短视净利市场至上是错误的。这个月稍早我才看到上海交大有一个教授写了一篇很长的研究报告,而且发了论文金融系的,他就是基本上就是拿了我 3 年前讨论播音的一篇博文,然后加了一大堆数据跟比较新的发展,然后就扩充成一个学术文档,我认为这是很好的事情。你我的影响力没办法扩到整个学术界,有人愿意把我的报告,拿我的文章拿去扩展,然后广为传播这些知识,会很好的事情。所以我对商业管理还有经济的原则的这些事情也是已经成为主流了。你如果连学术界都拿来写论文的话,他当然就是主流,所以这也是我可以放心的事情,不必再多谈。



王孟源 01:41:32 

我现在着重的就是这个金融货币政策的事情,因为你有 Nabiullina的智慧,可以借鉴。可是人民银行就是坐这里什么都不干,然后刚好过去 12 个月有两个师长,因为贪腐下马,这真的是很难看的事情,不过我预期在二十大之后,高层会认识到这件事情的重要性,然后要求人民银行出手。事实上,现在可能已经在私下做准备。



王孟源 01:42:04 

真正让我最担心的还是学术管理的事,就是你一般的思想教育腐化,一般的人还看得出来,你只要把它掏出来谈,就能够获得群众的支持,然后就造成压力,他必须要改革。比如说你提的那个有关教育教材的事情,这件事我相信一定是在有限的时间之内就会有处理,真正难以处理的是他们继续纵容假大空这些事情,因为这是系统性的腐化。邓小平在文革后反思以后,决定要纵容学术界,就是尊重专业,他搞错了,你尊重专业其实是要尊重理性地讨论,而不是尊重带着专业头衔的人,你尊重权威其实刚好就是一种反科学的错误。但是问题是这里就是专业的壁垒,就是最难困最难越过的。我可以批评他们这些,他们做的这些假大空是完全是骗人的东西,但是他们可以开空白支票,我 30 年后才兑现, 50 年后才兑现,这很容易骗。所以你这个必须要从体着手,那要从体制上着手,就必须要有高层认识到这一点,但是要高层认识到这一点非常的困难,因为这些专业人员自己不会承认,当初在大对撞机争论的时候,这很明显就是完全没有实用价值的一样东西。但是除了我在批评,然后还有在美国的赵午教授批评之外,中国国内敢说实话的就只有杨振宁,杨先生。你那个高能所里面有没有明白人?绝对有,但是要站队,他们全都是站在高能所那边,因为这是他们要徒子徒孙吃香喝辣享用不尽的。



史东 01:44:11 

这是一个饭碗问题。



王孟源 01:44:12 

饭碗的问题,对。这不是为了他们个人的钱,而是为了整个行业的钱。对他们来说就变成我是为山头争取利益,而不是根据我个人利益,我在道德上也过得去,我的良心也过得去,这当然是错误的。你在政策上的公共事务的讨论,最严重的不是为了私人贪污,而是错误的政策。



史东 01:44:42 

这个造成的坏的影响会更大。



王孟源 01:44:47 

对,更大。所以既然没有内部的专业人员会做这种反思,你很难向高层解释这个问题的严重性,而且他有历史的遗留,是邓小平以来 40 年完全都没有人敢碰的一个原则。那,你说如果不了解那个必要性,然后他又有这么严重的历史传承。你怎么样能够说服在位者,要了解他的说,这是一个必须要做的改革。你想想看美国现在的这个政治外交已经是自己开枪打自己的脚,为什么?因为他们的政要需要拿钱,就是Neocon在搞的这些东西。其实。不是完全因为他们太笨所以这搞成这样,而是因为他们把聪明才智都优先用到,把钱塞在进自己的口袋里面去,即使有聪明人也不敢在内部讨论的时候做反对,所以在社会科学这一方面,即使是中国的社会科学家,完全没有科学逻辑的素养,他们尸位素餐,比起美国这种自己拿枪打自己的脚的,还是要略胜一筹。但是在自然科学上面,美国的自然科学还没有腐烂到这样的假大空地步,结果中国一个新兴的国家反而比人家,比一个衰颓的帝国还要腐烂。



我认为不但这是长久未来最大的隐忧,而且是我良心最过不去的一件事情。因为我一直很努力地认为,中美的霸权交替是对世界人类一件好事,但是在这件事上真的是好事吗?你换上一个新的霸主,他的自然科学研发体系效率比既有的美国体系还要低一两个数量级。人类的进步不是就是靠着科学研科技研发,如果新的霸主会减缓人类科技的进步,所以我在讨论其他的事情的时候,必须要专注在提倡对这方面做出改革。我希望我的听众跟读者之中能够有体制内的关系,能够对传播散布这样的这个知识,能做出贡献的人,也请你尽一己之力来做出这个贡献,因为我认为这是,中国长期内部改革最大的一个难关。\twocolumn[\begin{@twocolumnfalse}
\section{「德法义」若调停败,「五大城」将难逃战火!、}
\subsection{20220617}
\end{@twocolumnfalse}]六月十五,已完成,“xxxx”标注是听不清楚的地方。



唐湘龙 00:02 

那你自己可以抓抓图片。



唐湘龙 00:45 

好,欢迎收看龙行天下,我是唐湘龙,我在台湾,我在台北,今天民国 111 年 2022 年的6月 17 号,那6月 17 星期五的时间。我在很多的地方都预告好了的星期五的龙行天下,我访问王孟源博士,那从两个月前我主动的约了王孟源,因为我看他节目看很久,那他所有的有关于不管是国际的政治财经各个领域里面的谈论的时候,就我个人来讲,我自己在工作上面来讲 30 年的时间,可是我会认为许多的谈论的观点很有启发。



唐湘龙 01:36 

第二个,对于这些资料的准备跟跟自己的查证,我知道那是硬功夫。好,所以我主动的邀了王孟源博士,请他能够在龙行天下里面我愿意一块的割让区属于王孟源的好,所以每个月通常我会安排王孟源在第三个星期五的时间,在龙行天下的单元里面跟大家见面,接下去是一个小时的时间,是王孟源的时间,那我设定了几个最新的议题,那听听看王孟源的看法。那我会尽量就不讲话了。接下去都是王孟源时间。来在我们线上的人,在美国的王孟源王博士,欢迎。



王孟源 02:16 

很高兴再上你的节目跟大家聊天。



唐湘龙 02:18 

好,来,我们今天从还是从乌克兰情势谈起,上个月我在和王孟源在访谈的时候,孟源谈乌克兰情势,他说他估计在 2 个礼拜之内,乌东的情势会有很重大的转折,其实就在这一个月的时间里面,我想大家注意到了乌东的情势,其实还没有到最后完全抵定的时刻,可是其实战场的情势已经非常的明朗。



唐湘龙 02:45 

那不管包括乌克兰方面也不断的释放出他们在乌东军事受挫,而且受伤惨重,甚至泽伦斯基公开讲这个是二战之后欧洲最血腥的战争,那即使是美国西方的一些军事专家,在过去几乎都是乌克兰的拉拉队,都是不断的放大乌克兰的战果。但是我们会注意到,在最近的这段的时间里面,那包括这些过去亲乌克兰的挺乌克兰的拉拉队,大概也都开始改变了调子,乃至于包括了北约的秘书长,这个就是说Stoltenberg都开始释放出这种,就是说乌克兰要开始准备屈辱性的投降的那个味道固然会遭遇到泽伦斯基、乌克兰方面比较激烈的这个就是说言辞上面的反击,但是形势上面的认知似乎大家差距不大。



唐湘龙 03:41 

好,那今天乌克兰的形势又有一些一些新的进展。今天凌晨的时间最大的进展就是德国、法国、意大利三国,就老欧洲的三个主要的国家的领袖联袂从波兰到了波兰集合,结合了之后搭火车,他们搭夜车,晚上搭着火车砰砰砰砰砰砰开,坐了一个晚上,坐到了基辅,下车的时候空袭警报大作。但是我形容那个场景空袭警报大作,但是俄罗斯觉得很冤枉,说我今天没有空袭击辅,基辅为什么空袭警报大作?因此大家终于懂了,好,那这大概是乌克兰帮他们做的配音,让他们比较有战争的临场感。好,那因为这三国的领袖再加上早一天已经到了基辅的罗马尼亚的总统,这虽然他不是共同约的,但是四国的领袖都到了,他们先到了基辅旁边的伊尔平。之前我们小谈过,在外面就是布查,就之前引发所谓的屠杀争议的布查,伊尔平跟布查是在在基辅的大概西北边的这个位置上面,他们到了伊尔平,那接下去呢,就跟这个泽伦斯基见面。但是我们先来解读一下,就是为什么这三国的领袖在这个时候联袂,因为时间包括出发点。其实虽然知道他们可能要去,但是之前都否认,都不证实这件事情,一直到出发了才证实,他们要干什么?



王孟源 05:10 

其实内幕消息说他们会去访问乌克兰,这个已经传了一个多礼拜了,只是时间问题而已。而他们之所以现在才去,其实是动作慢半拍,就是因为民意还不了解这个转折。我上个月上你的节目的时候,我说乌东的军事、情势很快要急转直下。然后接下来就是美国的媒体会开始为自己脱罪,因为他们已经搞了 3 个多月。



唐湘龙 05:49 

的造神运动。



王孟源 05:51 

假新闻。比如说他们那个时候,在三、四月的时候还有一个很大的消息,我不知道大家记不记得,太多了,但是我举一个特别的例子,就是他报道说因为俄军放弃坦克太多,所以乌克兰俘虏了太多的坦克,他们甚至没有足够的兵员把所有的俄军坦克开上战场。你现在跟这个礼拜他们乌克兰刚刚提的那些假新闻搞了 3 个月,要一下子扭转的话会很难看。是,所以必须要一步一步的来,就是希望他们的那些笨蛋听众能够接受是原本乌克兰打得很好,然后忽然战势有了转折,这期间又分好几个步骤。



王孟源 06:43 

第一个步骤是在上个月,一个月前他们所说的还是,他们的目标还是很接近,就是还没有放弃原始的目标就是regime change,要推翻Putin的政权。但是很快的那个国防部长就出来说,我们的目标其实只是要把俄国削弱 weaken ,然后在一个礼拜前变成bog down Russia。就是这不太一样了,这不是要把俄国削弱,而是只是要把它拖住。



王孟源 07:23 

这个又是低了一级,但是我想,过去这两天,Biden最新的谈话已经又降了一级,说他们现在的目标只是要 get some leverage, so that they can negotiate the peace,而且这个 peace 一定会有 territorial concession,就是他们现在的目标已经只是在战场上获得一点谈判的筹码,这样子他在割让领土的时候少割一点。你看这一个月之间连退 4 步,这个很好笑。但是这个在一个月前之前就可以看清楚,一定会发生。那我刚刚讲到国防部首先搞,接下去就是主流媒体。这就是你看他们的那个,看他们的头脑灵光跟专业的知识的程度,对不对?然后接下去下一步开始找背锅侠的是拜登他自己,然后拜登开始暗示这个仗打的不好的时候,这是两个礼拜前。很快的纽约时报就登了一篇报道。是啊,他开始指责乌克兰没有提供情报,所以美国人完全不知道这个乌军的内幕消息。



王孟源 08:45 

这个很可笑的,美国人怎么可能不知道乌军的内幕消息?这个很明显的是情报单位,就是 CIA, 要脱罪,倒也不是因为情报单位本身就有真实的责任,不管他们跟政客怎么讲,反正那些国务院的那些neocon不会听他们的,但是他怕这些neocon到时候找他们来当替罪羔羊,说这些 CIA 的分析情报分析搞错了,所以他事先找个下台阶说我根本就没有拿到情报,这个乌克兰的人在搞什么都是他们自己搞的。



王孟源 09:27 

所以你可以看到这个鄙视链这样一直传下来,最后接锅的一定是乌克兰。这个我上个月就已经在你的节目也谈过了,最后结果的一定是Zelenskyy。那事实上今天刚刚Zelenskyy自己在乌克兰内部的媒体,他发表了一篇文章指责他的参谋总长没有做好职责,这很显然是Zelenskyy自己也。



唐湘龙 10:02 

在找背锅侠了。



王孟源 10:04 

对,但是你说国际上有谁知道这个参谋总长名字是什么?连我都记不住。哈哈哈,对不对?所以这个Zelenskyy是跑不掉的,这个锅是跑不掉的,那到时候他在下场是什么呢?他绝对是臭名千里,现在只能希望能够保住性命财产,也许到迈亚密去做寓公,因为他在迈亚密有一个很豪华的,这个是大家都知道的事,再不济也可以到以色列去,他在以色列也有一个豪宅。



唐湘龙 10:40 

他爸爸妈妈在那里。



王孟源 10:44 

所以你看这个链子是这,从那个时间,从他们那个转变话锋的时间,你就可以看出这些人距离实际权力核心、了解真相的距离是从近而远的。最先的是主流媒体,当然除了国防部,国防部他们是专业的。最先的是主流媒体,其次才是Biden,然后再下一步是情报部门,然后再下一步才是欧洲的这些人Zelenskyy,然后北约,北约也是最近这 3 天才开始转变画风的,美国人已经明白的说了,那个英国的媒体也是比美国的媒体晚了两个礼拜。就是这个礼拜才有gurdian跟telegraph开始谈这种乌克兰快要被打垮的事情。所以你从这个就可以看出那他们的那个内部的排名的高下。我觉得很有意思。那当然排在最底下的是谁?就是欧盟。von der Leyen到现在还没有转变话锋。



唐湘龙 12:01 

von der Leyen。



王孟源 12:02 

德国、法国、意大利的,他们这些人还没有转变话锋,他还没有了解这个乌克兰的军队快要被打垮了,所以他们现在去乌克兰。你看美国人是什么时候去乌克兰?三月的时候,三月、四月的时候,英国人也是,对不对?然后欧盟是到上个月才去的。von der Leyen是上个月才去的。然后现在这个德国、意大利、法国去那边,这根本就是已经搭上末班车了。就是局势已经都改了,他们还在那边傻乎乎的跑到那边去作秀。这是非常可笑的,就是 Zelenskyy 已经从抵抗强权的英雄快要变成狗熊了。他们还不知道。



王孟源 12:56 

原先的这个错误决定是一个错误的政治决定。这个错误的政治上的战术决定,他跟他的错误的政治上的战略决定是一样的。就是同样都是误判了乌克兰的战斗意识。所以我在开战之前就说这个俄国还没有到开战的最佳时机,但是Putin还是开战了。那他为什么急着开战、没有选择最佳时机呢?就是因为他误判了。他认为只要一出手吓唬一下人,这个乌克兰就会投降,那事实上是没有,而且损失了多少人?俄国人自己承认。



王孟源 13:37 

到前 4 个礼拜,他们第一阶段打了 5 个礼拜,前 4 个礼拜就是阵亡 1300 人。我认为他的第二阶段就是从三月底到现在阵亡的人数可能还不到 1000 人。就是你头 4 个礼拜阵亡的人数超过接下来十几个礼拜阵亡的人数,而且接下来十几个礼拜的那个战斗要远远更为激烈。那为什么我这么说呢?因为一旦他们从基辅那边撤军回去以后,他们打法就全部改了,又回到他们正统的doctrine,doctrine的中文怎么讲?



唐湘龙 14:27 

教条。



王孟源 14:28 

规则?哈哈哈,对,这个。



唐湘龙 14:32 

在军事上面叫准则了,按照他们的准则去执行。



王孟源 14:36 

对,准则。标准的俄军准则就是大炮兵主义。



唐湘龙 14:42 

没错,他步战炮的那样的一种系统的就开始运作了。



王孟源 14:46 

不是,他的那个步兵跟战车只是做所谓的 reconnaissance in force,就是强力侦搜。就是你看起来好像是攻击,可是它从来不攻击。它其实只是一个侦查,就是你用攻击的形式来做侦查。那如果碰到激烈的抵抗,损失了一两辆装甲车以后就撤回去,然后再用炮兵轰上几天,然后再试一次,你这样反复的试,然后整个 1000 多公里的前线,每一个地方你都这样子,每隔几天试 1 次,一旦有了对方自己垮掉了,你就前进,这个 reconnaissance 就变成真正的attack。



王孟源 15:43 

所以并不是传统的这种西方或者德国式的。因为美国的这个准则是什么?美国现在的准则是所谓的,空地一体化,这个是 80 年代创造。那空地一体化其实就是二战的时候,德军的那个blitzkrieg 闪电战,就是你步战炮协同,然后再加上空军,他的那个重点是突破重点,然后包围分割。但是俄军并不是搞这个,俄军并不急着去包围他,它的重点在于杀伤,就是缓慢的杀伤。



王孟源 16:29 

那你如果看看他现在这个,根据乌克兰最近这个礼拜所承认的事情,他们自己承认,他们在前线上每天发射 5000 发炮弹,俄军是他有好几个人有不同的说法, 20 倍, 40 倍、 100 倍都有。我认为我们取最低的好了,20 倍,俄军每天发射 10 万发炮弹,第二阶段打到现在 80 天,80 天一天 10 万发,800 万发炮弹,哪一个人敢跟我说这个 800 万发炮弹的后勤能力不够的话,我觉得他这个应该回去重新念军校的一年级,就是能够在不到三个月之内把 800 万发炮弹运到前线,然后把它真正打掉了,这个是很了不起的后勤。然后有人批评那个俄军的制导弹药不够,其实俄军的制导弹药很多,只不过是跟美军的不一样。美军的制导弹药是轰炸机扔下来的,轰炸机或工具机扔下来的重磅炸弹、自导式炸弹,俄军用的是巡航导弹。他一共发了发射了几枚?有人做了统计。我刚刚看到几天前有人做了统计,从开战到现在一共发射了 1500 枚。 300 公里射程以上的巡航导弹用了 1500 枚。这个 1500 枚是什么意思?你想想看,美国从 80 年代到现在开始有巡航导弹。美国人打了好几场超大型的战争,在阿富汗打了 20 年,伊拉克打了十几年,他全部用了多少巡航导弹? 2000 多枚,就是任何一场战争都只用了几百枚。俄国人在 4 个月里面用了比美军在阿富汗打了 20 年,还要多一倍的巡航导弹。你跟我说这不叫做后勤保补给能力之强?这是很恐怖的。所以从三月的时候,英国跟美国的媒体就在讲俄军已经弹尽援绝了。



王孟源 19:05 

这个他的军机绝对补给不上,然后四月初又讲一次,四月底又讲一次、五月初再讲一次、五月底又讲一次、六月初又讲一次,现在不讲了。因为那他的这个军工的供给链真的是没有底的。我现在这个离题讲一件事情,两个礼拜前我上八方论坛的时候,我批评了这个中国军迷界,因为他们都是采访一些退休的中下级军官,其实中级的军官还好,就是那些下级军官,校级的或者是那个尉级的,退休以后变成那个军迷里面的大神。那他们讲的这些都是似是而非,比如说他们一个在三月跟四月的时候,他们批评俄军,这个明明是政治判断错误,俄军本身的能力没有什么问题,但是他们就普遍说这个俄军这次打得很差,我们这个解放军...怎么怎么讲。其实这个比较都是胡说八道,尤其是他批评的俄军这个后勤能力不够。我跟你说,你自己去算算看,哪一个人,哪一个,除了俄国人之外,哪一个国家敢拍胸脯说我打仗打 4 个月,我能够打出 800 万发炮弹,打出 1500 发巡航导弹,而且还继续的打。



王孟源 20:37 

眼看着还要继续打两三个月,所以我那个时候批评了这个他们军迷界的一个网红叫席亚洲,这个我对席亚洲,我为什么挑席亚洲来讲?因为我以前是观察者网的专栏作者。



唐湘龙 20:57 

对,他们也都是观察者网。



王孟源 21:01 

他是观察者网的军事编辑,所以我发现观察者网的那个读者的素质有下降的趋势,而且这个水准最低的就是席亚洲自己拉拢的那票人,因为他本身就是胡说八道而且撒谎成性,就是 Boris Johnson那种人。所以这次,两个礼拜前我提了他的这个报道,对俄乌战争的报道失实之后,我为什么特别今天又提起来?是因为昨天我发现了一件事情,就是我批评了他之后,他看到了,看到以后找了一个很可笑的借口,然后我说你这个借口很可笑,就是那两三句话就可以批驳。然后我昨天发现的是什么呢?他不但是我自己留言会被删,我在观察者网自己留言被删。有一个读者在留言栏讨论这个德国经济的时候说王孟源说了德国经济怎样,结果那个留言也被删了。就因为他的留言里面有王孟源 3 个字。



唐湘龙 22:29 

这个问题我再引申 2 个问题,问你对乌克兰。



王孟源 22:37 

再给我一分钟。



唐湘龙 22:39 

好,你请说。没问题。



王孟源 22:42 

我之所以特别提起来是因为两件事情,第一个是我不知道你有没有注意到河南现在有一个赋红码的事件?



唐湘龙 22:53 

我有听,但我没有去注意细节。



王孟源 22:56 

这完全是一样的,就是所发生的事情就是在河南有一些乡镇级的这个银行,它出了问题了。出了问题以后就有人要去维权。那结果这些维权的人莫名其妙,他们的健康码。



唐湘龙 23:11 

健康码就变红码。



王孟源 23:12 

对,这个就是公器私用。这个席亚洲在观察者网也是公器私用。这个不是说你不同意我就不行,你不同意我的话可以跟我讲理,你不讲理的话要撒泼,这个网络上喷子很多,我也不在乎,但是你当一个观察者网的编辑,公器私用,利用你的职位来删任何有关王孟源的言论,那个是有关德国经济的一个评论,而且还不是我留的,是读者留的。这个真的是。那你从一连有两个这个事件,小到观察者网,大到河南省政府,就可以看到这个组织的风气有多么恶劣,对不对?这个你现在这个河南这件事情去追究下去的时候,他们还在打太极拳,就说这个是什么卫健委,卫健委怎么会有这个功夫去管理这些为了银行账户而维权的人,这很显然的是有人、有这个操控这些银行的这些土豪,找了省级的官员去施压、要求搞出来的,对不对?就像观察者网管留言的那个工作人员,他没事去删这种谈德国经济的留言干什么?这是因为席亚洲看到了我的名字,怒从心头起 、恶向胆边生,所以去要求他去删嘛,对不对?你想想看,这里的问题不在于这个土豪想要报复这些人,或者是席亚洲想要报复我,而在于这些中间的省级官员跟卫健委员,他们认为这是可以做的事情。



王孟源 25:20 

观察者网那个管留言栏的人,他认为这是可以做的事情,他认为总编辑不会把他怎么样。河南的省级官员认为习近平不会把他怎么样,他可以为所欲为。你想想看,习近平已经当了 10 年的,十年的主席,他们的官场的风气还是这个样子。河南的省级官员还认为说可以这样搞,不会有后果。你说这个是不是很可怕?



唐湘龙 25:49 

对,当然这个健康码就是说全面红码的这个事情,其实它是有隐约在发酵的,就是说那个背后的那个逻辑。刚刚我王孟源提到了这个,其实我在注意这件事情,因为把你红码了之后,你就动弹不得,那这个动弹不得之后就使得你的抗争受到一个跟抗争无关的你的健康因素的影响,因为防疫防得非常紧,但是你会觉得你莫名其妙,我怎么会红码?但你会发现跟你一起抗争的人一挂人都红码。



唐湘龙 26:20 

好,但是我刚提到就乌克兰战争,因为三国的领袖去了再加上罗马尼亚,那我开玩笑讲说他们是去参加告别式的了,就是去跟泽伦斯基去告别的。但是在他们到访之前的时候,我们看到了就是美国,美国又拨了一批,这是第 12 批,同时也是以单一的一批的批次来讲,金额最高的 10 亿美元的



唐湘龙 26:45 

就是说对乌克兰的军援,当这个军援从你拨、同意到这些实际上面的装备的到货,弹药的都到货,那可能还要需要一段的时间,可是这个时候拨 10 亿美元的军援,他难道不表示美国对乌克兰仍然是有些的安全承诺?特别是在这三个领袖即将到基辅前的几个小时公布了这个政策,这不是一个将军的动作嘛,就是让这三国的领袖要空手来好像都很困难。



王孟源 27:17 

10 亿其实没有什么。他这里面真正的有意义的,连乌克兰自己也说他们真正在乎的是里面的两套Harpoon missile捕鲸叉导弹,其他的就是他们已经拿到的,然后再送两三套火箭炮、四五套海马斯。



唐湘龙 27:42 

海马斯。



王孟源 27:43 

这个根本就是杯水车薪,没有用了,但是真正比较新的就是被媒体拿来吹的是那两套捕鲸叉导弹。那为什么要捕鲸叉导弹呢?其实我觉得这个是那个美国国防部懂军事的人特别安排的,没有什么外交意义,就是实际上发生的事情是为了配合,就像你讲的,为了配合德法意三国的访问的,美国继续的做姿态,因为他们现在这个欧美的领导人,他们不治国了,他们就只是作秀。



王孟源 28:28 

那这只是他作秀的一部分,但是国防部就觉得我们还是及早把捕鲸叉导弹送进去,这个背后的意义是什么呢?其实我上个月也提过,就是接下来这个在东乌,再过一两个月他们把这个乌克兰的前线兵团全部消耗殆尽之后,俄国人应该会在前进 150 公里,然后占领接下来的五个主要城市,就是Kharkiv、



唐湘龙 29:06 

哈尔科夫第二大城。



王孟源 29:10 

Dnipro。



唐湘龙 29:12 

Dnipro中部大城。



王孟源 29:13 

Zaporizhzhia,Mykolaiv,然后Odesa。



唐湘龙 29:22 

奥德萨。



王孟源 29:24 

这里面最重要的是Odesa。



唐湘龙 29:26 

黑海。



王孟源 29:26 

Odesa是黑海沿岸的最大城。这占领了Odesa以后,这个乌克兰就成了内陆国了,而且Odesa是凯萨琳大帝始创的国家。这个Odesa的名字是Odysseus。是一个希腊神话里面的Odysseus。



唐湘龙 29:48 

Odysseus。



王孟源 29:50 

但不是希腊人建立的,而是俄国人建立的。他只不过是因为在启蒙时代他们很迷这个希腊神话这些事情,所以他取希腊神话的名字来建立。它是一个典型、传统、真正的俄国城市。他一直到 1922 年才被列宁把他分到乌克兰去。乌克兰本来是一个内陆国,是 1922 年列宁才把这个黑海沿岸的那块地割给乌克兰。所以在今年开战之前,二月初的时候,二月初的时候Putin给了一个演讲就提到这一点,结果也是被那些不学无术的美国人笑了一顿说,“你说什么?这个乌克兰是列宁创立的”,其实Putin讲的是乌克兰现在领土这么大,是列宁给的。在那之前是一个小小的内陆国。两个礼拜前还为了Odesa又吵了一阵子,就是英美的这些媒体又栽赃俄国说他们造成第三世界的饥荒,然后他们封锁了这些运粮船的出口。那俄国人就说,好,那很好,我愿意开放。但是这个Odesa的港口布满了水雷,这个是乌克兰布的,叫乌克兰。



唐湘龙 31:25 

自己清干净。



王孟源 31:27 

结果乌克兰马上反过来说,你这是想要骗我,让你能够登陆,绝对不干。其实俄国人也知道他们不会干。俄国人是因为要反驳美国人的栽赃,对不对?你这样就证明了这个不能粮食不能往外运是乌克兰自己的选择,不是德国人不让他们运。所以这个我刚刚讲到国防部为什么会急着送捕鲸叉呢?就是因为他认为他们还在担心俄方会做两栖登陆。



唐湘龙 32:02 

这个这地方我再补充问一下,刚刚王孟源提到的捕鲸叉导弹在台湾翻译做鱼叉反舰飞弹,那这个鱼叉反舰飞弹当然表示他在对俄罗斯的黑海舰队在海面的进攻在进行了某种的防范,虽然什么时候会拨付还不知道,但这确实是这 10 亿的这个说刚批下来的这个 10 亿的援助当中,比较会让觉得注意到的就是它里面,因为它叫反舰飞弹,所以跟现在的步战炮的这个这种的陆战的装备是非常不一样的。



唐湘龙 32:37 

好。但是刚刚孟源的提法就是说,因为最近在乌克兰方面也提到说他们不只是在乌东,在包括北顿涅茨克,就是说这地方他们受到重创,他说他们另外两个地方受到重创,一个是在哈尔科夫,另外一个是在科尔松的外围的部分。那这显示就是说战势已经不是只有在乌东,就是你刚刚提到的那五个城市,那五个城市在我的定义上面来讲,那就是它的第三阶段的军事行动了,所以你判断俄罗斯会有第三阶段的军事行动?



王孟源 33:14 

对。而且正因为你第一阶段的时候,我刚刚讲过他第一阶段的时候,他的目标是让乌克兰赶快投降,所以那个时候急着和谈的是俄国,乌克兰不急着和谈。但是谈到三月底为什么Putin咬着牙让他的那个后勤补给部队让乌克兰去打游击骚扰?因为他们还是一直在谈判,谈到三月底他们最后一轮的谈判是在土耳其进行的,两方面其实已经很接近了,就是乌方除了不愿意割让,正式割让克里米亚之外,其他的所谓承诺永久中立,这个都。



唐湘龙 34:01 

已经都谈好了。



王孟源 34:02 

对,都谈好了,但是三月底的时候发生了一件事,就是Boris Johnson跑到乌克兰,Boris Johnson跟Zelensky讲,你如果跟他和谈的话,我们北约绝对不会提供安全保障。就是你这个所谓的中立啦、还有安全啦,这个你还是要有第三国保障。他说北约不提供这安全保障,到时候你一毛钱不用想跟我拿,你不用想要在外交上得到任何支持。



王孟源 34:38 

那这样子Zelensky就很明白了这个意思,就是连我要去当寓公,到伦敦或者迈阿密去当寓公也都不行了,那所以他就退了,他这个马上Boris Johnson访问之后,Zelensky就撕毁了他们谈判的那个进程,然后马上Putin就改为第二阶段。这个第二阶段就不再是攻城略地,而是要杀人。就是,你看这个第二阶段打了 80 天,他的目的是什么?就是要杀伤乌军。



唐湘龙 35:13 

慢慢消耗你的主力部队。



王孟源 35:16 

消耗乌军。真的是,他们现在刚刚承认每天的死伤人数是 500- 1000 人。那好可怕,这个,这他们已经每天 500- 1000 人的死伤,你 80 天、100 天下来,即使是一个乌克兰这样的级别的国家承受不了,没有一个国家能承受的了。所以第二阶段开始到现在 80 天了,固然乌方不想和谈、俄方也不想和谈,所以俄方的第一阶段是想要和谈,第二阶段是想要杀伤,第三阶段才是真正去攻城略地。所以我们现在就是等着看再过一两个月这个第二阶段告一段落。第二阶段什么会告一段落?就是整个乌克兰的前线军团被大幅歼灭,他们现在其实因为他的目标是杀伤,所以他也不急着做大包围,他做的是小包围。就是我刚刚讲的做这种 reconnaissance in force,然后视防卫方的那个强力抵抗程度来决定哪一个地方可以前进。所以你不能够事先预定说我一定要攻这个方向,你就是每一个方向都尝试一下,哪一个地方弱你就会突出,然后就再继续尝试。这样子经过 80 天以后自然的形成好几个小包围圈,就是不是四面包围,而是三面包围。这个小包围,那我现在算一算至少有三个这个包围圈即将要收网,然后你可以看出这个乌军真的是强弩之末,就是因为他们有经验的军官跟士官快要打完了。你那个没有训练过的壮丁还有、还可以往前送,但是你没有职业军人来领导他们了,这样子还会有什么战力?你连那个普通的机关枪都不太会用。



唐湘龙 37:30 

这个接下去大概一两个月会看到的事情,所以我引用了就是说俄罗斯的前总统梅德韦杰夫,他前两天讲了一句话,他说如果这场战争再打两年的时间的话,他估计乌克兰就不存在了。这当然是有点在恐吓的味道了。不过因为我做新闻工作,我常常做察言观色的观察。我说跟 3 个月之前相比,你看到泽伦斯基跟普丁两个人的神情是有很大的变化的,那这三个月之前,泽伦斯基到每个地方,不管是演讲也好,或者镜头上的露脸的时候,他是很有气势的。



唐湘龙 38:11 

但是到最近这大半个月的时间,你看到的泽伦斯基都是气急败坏的样子,而且其实所有的言语基本上越来越消沉,越来越负面,那不断的跟西方国家发生这种口头上面的龃龉,在攻击美国,对北约的也很不爽,然后对于这些到访的这 3 个国家,你看到他在跟这三个国家的领袖见面的时候,其实肢体语言都是很疏远的。



唐湘龙 38:40 

他之前对德国很不爽,对法国马克龙觉得不可以羞辱俄罗斯这件事情他也很不爽,基本上面他在这种不爽的情况之下,勉强跟着三国的领袖拍了个这样一个定妆照,可是普丁是相反的,普丁之前西方的媒体都在炒作,普丁快要死了。可是你看到最近的普丁的那个神清气爽的,感觉比王孟源的气色都还好,那当普丁的气色这么好。然后他6月 15 号他又跟习近平通了电话,这个通电话是不在我预期之内的,因为事前并没有讯号。那这样的一个通话,事后的国际媒体的解读也很多,你怎么看这一次中俄之间的通话,它有代表任何的战略意函吗?



王孟源 39:30 

当然有了,这个俄国自己本身最先打胜的不是这个军事,这个军事虽然他现在胜算在握,但是事实上是最后一个阶段的胜利了。他先打赢的是金融战跟宣传战。你如果去看一个礼拜前这个Nabiullina就是他们的中央银行行长做的公开谈话,她上个礼拜她把这个标准利率从 11\% 调到 9. 5\%,对。



唐湘龙 40:11 

战前水准。



王孟源 40:12 

战前是10\%,就是他现在的利率比战前。



唐湘龙 40:16 

还要低。当全世界都在升级的时候,只有它在降息。



王孟源 40:23 

哈哈哈,它不止降息,而且它对这个美元升值了30\%,你说这是这牛不牛?而且你去看她的谈话,其实她是说,很明显的她是在说,我事先也没有想到会赢得这么爽,哈哈哈,我事先还很担心会是一场很艰苦的金融战,没想到对方会这么虚弱。这其实是我在,你如果去看我的博客,你就知道过去 3 年我一直在讲美国的这个金融跟经济非常的虚弱。他们经不起这种折腾。那这一次开战之前,这些neocon,国务院的neocon要搞、这样搞,其实财政部跟美联储都很明白的放话说我们经不起这样子、经不起搞,所以事实上是他们不管专家的建议跟警告去蛮搞,结果就搞出这样。



王孟源 41:30 

搞出这个大娄子就是金融战搞输了之后,就是 2 个月前就已经很明显的、他们金融战失败的时候,马上就有 New York Times 对Hunter Biden做爆料,那个时候我就认为很可能是华尔街的财阀对 Biden 很不满。所以去叫纽约时报去拆他的台。那这就代表着,因为你知道 Biden 其实是继承了 Hillary 的手下的那些。



唐湘龙 42:06 

没错。



王孟源 42:07 

好,这原因是因为你要说民主党里面的权贵政要捞钱,排队排在最前面的就是Hillary,然后Biden觉得他要插队,所以他就继承了这些neocon,所以他也要跟着去捞。那哪一个派系没有跟着去,没有去跟neocon搞在一起到全世界捞钱了,就是奥巴马这个派系。



王孟源 42:42 

所以你现在这个华尔街对拜登把美元的地位搞砸了、管理通胀搞得乱七八糟,非常不满。你想他们要找政治界的派系来跟他们联手,是哪一个派系呢?就是Obama。所以现在美国的政坛分析人员都在观望,看看在期中选举Biden、民主党,大输特输之后会不会有内部的这个 regicide 就是叛变,就是由那个由Harris代表Obama出来做反叛。因为美国这种总统制其实是比英国的这个议会制要稳定,这如果是在英国或者澳洲这种议会制国家,你的首相如果胡搞乱搞成这个样子。



唐湘龙 43:49 

应该已经下台。



王孟源 43:51 

马上就被自己党内的议员轰下台了。但是美国的总统是固定四年的任期,这个礼拜还有另外一个消息,我想你也会一定也注意到了,因为你的观察很敏锐,就是他们这个有民主党的一些议员站出来说,他们认为拜登不应该竞选连任。这个再引申一步,就是他们认为拜登不应该坐满四年,哈哈哈,这已经只差一步。



唐湘龙 44:28 

这个当然很难呐,不过从一个从美国政治史上面来讲,你要现任总统不连任,而副总统也不会是他们期待的人物,等于必须要推一组新的候选人,这个在美国政治史上面来讲也没有看到过,所以当然这个是表示民主党里面是开始有杂音。



王孟源 44:51 

有杂音。就是现在的问题就在于,美国这个总统制是让你要半途换马非常的困难。然后第二个阻碍是那个Harris本身人缘也很糟糕。



唐湘龙 45:06 

没错。



王孟源 45:07 

名声也很糟糕,所以这样一来反而保住了拜登的地位。我们再看看,不过这个内部的反对声浪是相当强大,你等到期中选举如我们现在预期的一样,他们大败的话会有很热闹的那个xxxx。



唐湘龙 45:30 

很好,我们回到前面这些,刚谈到就是说中俄之间的这个通话。



王孟源 45:36 

对不起,扯太远了。



唐湘龙 45:38 

你认为是有很明确的战略意函的,为什么?



王孟源 45:44 

这个俄国人他们原本准备了 8 年,他们为金融战跟宣传战准备了 8 年,但是他们原本的预期是他们能够存活下去,就是能够抵抗住而存活下去。他没有想到美国会这么衰弱。所以现在这么一打,天下苦美久矣,因为美国这个霸权,尤其是 20 年前neocon上台之后,真的是倒行逆施,全世界放火,这得罪的人太多了。那你一旦罩不住之后就人心散了。然后背景上他们为什么一直担心中国?因为中国提供了在财政、财务跟经贸上面的一个替代。你现在再加上俄国提供了一个在军事跟能源供应上的替代。



王孟源 46:52 

那这个已经是满了嘛,那你还要跟欧美打什么交道,对不对?你这个沙特也不听他的,Saudi也不听他的,那你这个其他的第三世界国家根本没有理会这个西方世界的必要。所以我想俄国跟中国在看到美国出乎意料的衰弱之后,他们都会想要进一步加强合作,就是这个分工合作,俄方进行这个资源作为资源供应商跟作为军事保护的提供者,而中国提供技术跟财政的资源,那这样就满足了所有第三世界可能的需要。



王孟源 47:47 

本周中国有一个智库发表了一篇文章,但是很不幸的我没有看到他们中文的原版,我看到的是美国媒体评论的文章,所以他们已经翻成英文了。所以我可能用的那个词汇不一样,就是他们翻成英文,我再从英文翻成中文。他讲到一个三环战略,这个所谓的第一环就是东南亚跟中亚,这是第一环;第二环是其他的第三世界;然后第三环是西方。那你这样的一来就是...,这我觉得很好玩。因为这你如果去看我三月初所写的那篇博文,就是刚刚开战的时候,我很快的赶快写博文以便尽快的做建议。你如果看最后的那 4 个段落,我的建议就是这样,就是中方去拉拢东盟跟中亚跟中东形成一个新的,当然还有俄国了,就是中国跟俄国合作起来,模仿当初欧洲共同体一样建立一个欧亚共同体。那我相信现在这个习近平跟普京一定是有在谈这件事情。就是怎么样进一步做这些合作。比如说我那篇文章里面建议了很多执行的细节,比如说建立一个替代 IMF 的机构、建立一个统一的货币、建立一个类似欧洲共同体一样的,欧洲共同体还有所谓的欧洲能源共同体、建立一个能源共同体,欧亚能源共同体、建立欧亚的交通线。



王孟源 50:02 

这其实已经是一带一路时代就已经尝试过,不过当时俄国并不愿意合作,所以你现在可以合作。像这些事情我相信都是习近平跟普京可以谈,因为你不但习近平有意愿,普京也有意愿,他们这一次仗一打起来就已经跟欧美彻底脱钩了,对不对?你连麦当劳跟可口可乐都已经撤走了。



王孟源 50:32 

对,所以那你还折什么?俄方做得多么决绝,他已经立法要剥夺所有这些西方国家公司的智慧财产权。这个是已经通过了,三月中就已经通过了。然后三天前他们的Duma就是他们的国会,又提出了一个法案要改变他们的教育体制,就是高等教育体制,要从西方式的再改回苏联时代的。这个意思就是说我们的学生不用去留学了,所以就不需要跟他们一样,什么学士、硕士、博士,这个我们自己搞我们自己的、苏联式的。这是真正准备破釜沉舟就是永远的割裂,老死不相往来的这个态势。那他一旦决定要这样子,就等于是确定要跟东方跟南方联手,东方的中国,南方的中东联手。现在我觉得中俄联手最大的问题在于印度,因为我相信从中国的观点来看,印度是所谓的老鼠屎,任何一个国际组织里面,这其实是一年多前在我博客里面也提过,结果后来已经变成大陆舆论的共识了。



王孟源 52:06 

就是大家都知道这个印度的确是,你就不要跟他们扯,他们如果要跟美国合作,就让他们去美国合作,反正头痛的,谁跟他走的近谁就头痛。那问题是我想俄国不是因为跟印度合作愉快,而是因为他还是怕中国的体量,经济体量太大。所以他还是会想要把印度拉进来制衡,这从中国来看,这不是怕被制衡的问题,而是印度真的是加入哪一个国际组织,那个国际组织就瘫痪了,对不对?我觉得这会是他们两个人谈判的最大的一个障碍。



唐湘龙 52:52 

好,最后一个问题,因为你在美国,在上个礼拜的时候,拜登很努力的开了美国第二次当主人的美洲峰会,但是这次的美洲峰会跟 1994 年美国作为美洲峰会的发起国,美洲峰会的总部也摆在美国的这样子一个美国当大哥的美洲峰会,我因为我在台湾,我不在美国,我当时看的是国际舆论场,看着拉美的这些媒体的消息,他们对这场的美洲峰会里面呈现出空前的借着美国在抗拒三个国家, 古巴、尼加拉瓜跟委内亚三国的这样一个参与,反而激起了美洲的这些很多过去被认为跟美国还算亲近的国家的激烈的这种反美理论。其中它代表性是墨西哥。



唐湘龙 53:49 

好那因此这场的峰会开完了之后,船过水无痕,坦白讲,我看不出这个峰会开跟不开,对美国来讲有任何正面的意义,没有留下任何实质的东西,来的就是给美国一个面子。大概21 个国家领袖来,但是没有来的才是大家真正关注的重点,这场美洲峰会到底对美洲政治未来会有什么影响?美国慢慢的在美洲失去了他的领导威信了吗?



王孟源 54:16 

其实刚刚我都已经谈过了。就是因为俄国跟中国提供了替代,就是credible、有目共睹、可靠的替代。而拉丁美洲别说第三世界了。就是全世界,就是中东跟拉丁美洲最了解美国霸权的丑恶,对不对?从门罗主义开始 200 年都是这样的。他们在中美洲进行殖民,一直到现在,他们那些天然资源非常丰富的中美洲国家到现在还是赤贫的,他们的人民很努力这些,为什么这些人拼命要越过边境来美国?我们这边割草的什么的,做劳力的都是这些中美洲而已。他们不是那种好吃懒做的人。



王孟源 55:14 

他们比美国人要勤劳的多了,为什么会搞了 200 年还这样越搞越穷?因为他们的那个土地还有技术还有政治完全都被美国财阀掌控在手里,一直剥削到现在。所以你说这些国家一旦有了选择的话,他们会选择美国吗?当然不会。所以这个其实是世界大潮流中的一个小反应而已。



王孟源 55:47 

美国现在也已经管不着了,为什么?因为他现在自顾不暇。就是我从 3 年前就讲美国的经济非常的虚弱,经不起折腾。而且如果是一个有眼光的政治家,他的第一优先会是控制通胀。一年前我还在我的博文里面讲,拜登最大的失策不在于跟着那些neocon出去乱搞,最糟糕的是他用了戴奇这个贸易代表,他的内阁里面最糟糕的,就是他的内阁里面都已经全是历史上最糟糕的内阁成员了,但是最糟糕最糟糕的还是戴琪,因为为什么?因为它的头号问题会保证它的这个政治生涯终结就是通胀。而戴琪就是“我不管你什么通胀,我就是要继续跟中国搞关税”。



唐湘龙 56:48 

没错,我在一年多以前其实我在节目当中谈过这个问题,我就说美国如果要解决通胀,它第一个动作应该是把中美之间的关税先打掉。先打掉之后。



王孟源 57:03 

最简单的,那最简单的。



唐湘龙 57:04 

对,没有错,但是戴奇死命的在抗拒。所以一年之后你看到戴奇现在成为拜登内阁里面最难处理的,而且他跟其他的阁员之间的意见的尖锐对立,所以我怀疑戴奇到底还能够做多久?拜登到底还能够挺他挺多久?好,那附带提,就拜登的下个月他要去中东,你刚也提到中东很重要,他这次去中东难道能够让自己在中东重振雄风吗?



王孟源 57:33 

我觉得是,他是去求沙特增产。那主要的问题还是这个通胀的问题。那其实能源只是通胀的一部分,因为你过去这半年石油价格上涨了50\%,但是美国的柴油价格上涨了100\%,所以这个其实不是问题的根源,问题的根源在于美国在过去这 30 年,他们的经济已经完全寡头独霸。



王孟源 58:10 

比如说你的这个,它的这个大众媒体,全国性的媒体在 30 年前有将近 1000 个不同的公司,现在 95\% 以上的这些全国性的媒体公司掌握在 6 个财团手里。现在刚刚这个奶粉供应不足。为什么?因为只剩下 4 个供应商。你如果住美国的话去买肉,就会注意到过去这两年肉类上涨最多的是牛肉,猪肉反而没有问题。为什么猪肉没有问题?因为最大的猪肉企业是中资的。



唐湘龙 58:55 

没错。



王孟源 58:58 

这个牛肉上涨了百分之百,猪肉没有上涨,那为什么牛肉会上涨百分之百?因为美国的屠宰商兼批发商只剩下 3 家。



唐湘龙 59:13 

都已经很非常寡头垄断了。



王孟源 59:15 

都是寡头的。那你这样,这种寡头他会管你?他第一步增加利润就是先把产量消减到刚刚够,勉勉强强够,这样子平常是节省你的费用。一旦出了什么问题,他的这个售价马上就冲上天了,对不对?其实美国在能源上,在油气能源上是已经是出口国了,它是一个净出口。所以这一次这个能源价格上涨,美国整体来说是获利的,尤其上涨的幅度最大的是欧洲。美国原本出口到欧洲的天然气很少,但是到上个月 80\% 的天然气是出口到欧洲去了,因为欧洲的价格最高,所以你这种其实是在刮欧洲人的钱。那问题是你整个国家获利,但是这个获利的全都是那些大石油公司,那4 家石油公司。



唐湘龙 01:00:31 

没有做。小小老百姓。



王孟源 01:00:32 

国内跟着倒霉,国内的老百姓跟着倒霉。那我觉得很有意思,就是上个礼拜这个德州,美国最大的一个液化天然气的港口爆炸、起火之后,欧洲的天然气价格上涨20\%,美国的天然气价格下跌20\%。为什么?你如果天然气都拿去外销了,你的国内的天然气自然就跟欧洲的天然气的价格平衡、一样,对不对?那我就觉得这样,这个很好玩,你这个爆炸以后,石油公司有没有损失?因为它卖给欧洲的气虽然少了一点,但是增加了20\%,价格上升了20\%。这个卖给美国国内的天然气虽然价格下降了20\%,但那他多卖了,对不对?所以真正占便宜的是美国百姓,倒霉的是欧洲百姓。那这个真正占便宜的是谁呢?好像是Biden嘛,因为急着要控制通胀的是Biden。



唐湘龙 01:01:49 

没错,好,当然这些的情绪都还在变化当中。那我每个月会透过电话连线,因为疫情王孟源的工作的关系,他人在美国,所以我们透过网路、透过视讯访问王孟源,那今天一开始的时候讯号有一点点不稳定,跟大家说明一下。但是我们随着每个月的这样一个访问,我们可以去 follow 一些国际的大的趋势,你可以去关注那些大的趋势,但重点不在于就是说要证明着我们的判断是不是一定准确,一定精准、如何,但是那个趋势的演进,那训练我们自己去理解跟去推论当某个事态发生的时候,它背后到底代表什么含义?以及这个事态发生了之后接下去会发生什么影响?这个就是在国际时事观察时候非常重要的部分了。



好,因为今天时间的关系,来,我感谢几位的观众朋友们。来,从 Po 伦开始, Po 伦,谢谢谢。然后小王他说麻烦你说话的时候把嘴里的东西关在什么什么,,好,那没看到。好,再来。李道阳,感谢,然后托尼红,感谢,然后圈份与感谢,然后圈光力感谢,然后也是圈光力。这个支持相同。大哥跟王博士刚昨天忘了打字,没关系,都跟大家说说,谢谢。今天的时间到了,感谢人在美国那透过电话连线接受龙行天下访问的王孟源。孟源,谢谢。



王孟源 01:03:20 

抱歉,讲的有点超时间,有时候管不住。



唐湘龙 01:03:22 

哈哈哈哈,不会,我也是。好,我们下个月见,来,谢谢孟源,拜拜。



\twocolumn[\begin{@twocolumnfalse}
\section{逻辑辩证、昂撒理论基础、观察者网、俄乌战争、通胀}
\subsection{20220621}
\end{@twocolumnfalse}]Credit: Fanboy



史东:00:13 



嗯。各位朋友你好,我是史东。在今天节目中为您请到的是王孟源。每次王孟源其实上节目呢,都有很多很多非常重要的事情要跟我们谈。我想今天呢一定也不会例外。讲到这儿呢,就先和孟源打一声招呼。孟源,谢谢,欢迎。



王孟源:00:32 

大家好,很荣幸啊。



史东:00:35 

我们今天要谈些什么?



王孟源:00:38 

嗯,有一些话题是继承三个礼拜前我上一次上节目的事情。但是真正最重要的核心是有关当前美国经济的走势,还有通货膨胀的严厉程度。这件事情是在过去四十年,这个经济学界跟财阀集团有意的遮掩之下呢,一直是很难精确地断定通货膨胀到底有多严重。但是我一直到这个礼拜才终于找到足够的资料,可以提供一个明确的一个概念picture。所以想跟大家分享一下。



史东:01:21 

好,非常好非常好。



王孟源:01:23 

那我们先从先继承上一次谈话的一些论点。我想先一开始先做一些理论的基础。就是上一次我谈到逻辑辩证的规则。为什么它是很重要的。事实上它的核心重要性在于求真,就是科学方法。逻辑辩证是科学方法中的一部分,一个很重要的成分。那什么是科学方法呢?科学并不是一个学科,并不是说你去做物理或者做数学计算,就是科学。你事实上同样的一个学问,有些人可以当做工匠来做,知其然而不知其所以然。那有些人可以完全死背,连其然都不是真正的知道。这些学科之所以称为科学呢,是因为理论上,理想里面你应该是以科学的方法去做的。而且历史上是以科学的方法做出结果的。那科学的方法是什么呢?就是我刚刚提到的逻辑,辩证逻辑,辩证的最简单的总结就是你正方要提供明确足够的事实,以及完整严谨的逻辑叙述。反方则针对正方所提供的这些事实跟逻辑呢来挑毛病,来指出其中的漏洞空隙。那你的事实呢,如果是包含一些不确定性的话呢,那反方可以指出来。然后呢,双方可以讨论这个不确定性到底有多大的危害,到底是不会影响你的结论。所以这一套方法是人类经过过去一万多年的过程中呢,发现要去追求真相的唯一手段。事实上没有其他的手段可以追求真相。



03:36 

有些人认为,这种理性逻辑只能够用来研究科学,就是自然科学。那你谈到一些非理性的对象,比如说人本身就是非理性的。那他们就会说“你这个不能够用理性来衡量”,这其实是错误的。这个从观察到接收分析其实是一种信号。你观察的对象如果是非理性的话,它其实是制造一些噪音,就是不确定性。你要解除这个不确定性呢,要你要就是所谓提高信噪比。你要从这些不确定性里面抓出其中的主轴呢,你的理性分析反而更为重要。就是如果你的麦克风去接(听)一个人的,录一个声音的时候,如果那个声音有很大的噪音的时候,你不能说那我这个麦克风就用最烂的。其实你这时候反而需要最好的麦克风,这样子才有可能设法事后尽可能的隔离那个噪音。我这里想用一个比喻啊,就是就是但我真的没有想要侮辱人。所以先跟大家解释一下,就是你用科学方法,用逻辑辩证去研究自然现象的时候,像是一个人,一个医生去看一个病人,你们两个完全可以对话。但是你用科学方法去研究社会现象或者人性的时候,这就有点像一个兽医要去治一匹马或者是一头牛。这个时候,你的过程就没有那么精确,那么便利。因为你至少没有办法问他,你哪里痛,你这个感觉是怎么样子,所以你就这之间就会有一些噪音,有一些隔阂。但是这并不代表你可以放弃理性。而反过来说要医马的话,那个兽医本身也必须是一匹马,或者你要医牛的话,就必须要是一头牛,那个是不对的,完全是错误的。



05:47

所以,我想跟大家讲一下,但是这并不代表呃不是理工科出身,没有逻辑辩证呃经验和习惯的人。他的作为一个人的价值有任何的损失。人的个人的价值不在于这个理性,人类整体的价值在于理性,国家整体治理的价值在于理性。但是个人不一样,嗯,一个世界头号的舞蹈家,他可能是全人类的珍宝,但是他完全不晓得不需要知道逻辑辩证。这个逻辑辩证它有用的地方在什么呢?就是任何问题你需要求真的时候,你就必须要认真的逻辑辩证。



06:30

那自然科学是大家都知道应应该用科学方法来研究的。其实公共事务就是跟政策有关的,跟大众治理有关的,也都需要求真。你如果不求真的话呢,就没办法评定是非对错。因为是非对错被扭曲;几乎百分之九十九点九九都是扭曲真假,因为撒谎是很容易的事情。你这个如果是坏人,做了什么错事被抓的,他第一个想的就是要撒谎。所以在一个社会里面有好的公共政策,要把公益。最大化的话绝对必须要能够分辨真假。而分辨真假的最佳方法就是科学方法和逻辑辩证。



07:30 

那然后我再试图用科学跟逻辑来研究社会科学,或者是应用在实际的社会跟政治议题上的时候,还会遇到另外一个反的说法,这个说法就有他的可取之处,跟刚刚前面几个例子不太一样:他说的是你人不住大陆,事实上我从来一辈子从来没有去过大陆啊,我唯一的最接近的一次是在上海过境机场过境,那他们就说你这个人不在大陆社会不可能了解这边的事实细节。所以你不可能得到正确的结论,这个说法是对也不对。就是你要是当地的地方性的这种政治社会议题是什么的性质。如果是有关哪一个人,哪一个机构内部运作的详细情况的事情,那我当然不会有那个消息。那不过事实上你如果去看我的博客,就会知道我一向遇到这种问题,我就说我不是适合评论的人。另外还有一些专业是我没有专业能力的,或者兴趣或历史研究过的。比如说税制,或者中国的农业政策这个这种事情,只要有读者在我的博客发问,我马上就说超出我的能力,无可奉告。所以这个呃点我是承认是有道理的。但是我在博客上已经很小心的避免了。但是还有一些议题,尤其是例如国际战略的议题,这个是不需要知道内部组织的细节,也可以找出最优解的。那我很喜欢用的一个例子是,假设我是一个外星人,我一辈子都在太空真空无重力状态生活长大。然后相对的你是在地球啊出生长大,你一辈子都在重力场里面。但是我很可能我对爱因斯坦的重力方程式很熟。但你一辈子每一秒都生活在重力场,你能写得出爱因斯坦的重力方程式吗?不见得,对不对?所以你要看这个问题是什么样。



10:00

这些是我想补充一下上一次所谈有关科学方法跟的逻辑辩证的应用,在应用的时候该注意的一些基本原则。然后,我想在实际应用在社会跟政治议题上的时候,我们必须承认这个光是逻辑跟科学是不够的。因为它只是一个方法,它只是帮我们求真这个求真。但是哪一方向的真,哪一个问题的真实答案,你要求最优解这个最优的这个所谓的utility 方程式是哪一个?这个逻辑并不能够提供。你必须要有一个从外加的条件,也就是在数学里面叫做公理。就是你的基本的价值定义。其实这就是所谓的哲学,尤其是政治哲学。



10:56 

那中国在春秋战国时代就有一有过一次很热烈的讨论,百花齐放的讨论。其中我认为有两个很重要的门派啊,一个是儒家,一个是墨家。那这两个都有一个共同点,他们都是理性的。人本主义就是他一方面尊重理性,尊重科学,尊重事实。但是呢他们所加入的这个前提,这个公理是什么?是人本主义,就是以人为重,这个人是指所有的人民。他们的差别在什么呢?儒家强调依托既有的组织,也就是家庭、社会、国家天下来提高效率。然后在这个组织的效率之下呢,再加上道德。墨家不一样,墨家不强调这个既有的政治组织,他们完全就是凭道德热诚这种volunteer 志愿者的这种热诚。所以他们的这个偏重不一样。但是你如果看的话,他的同样都是用理性来考虑实际上的战术。但是他们的战略目标都是人本主义,就是要为人民的公益最大化。这个理性的人本主义如果用在二十一世纪会产生什么样的版本呢?很明显的就是社会主义。我们现在说的所谓的社会主义,就是理性的人本主义应用在二十一世纪的工业社会里面。有一换句话说,我认为如果孔子复生的话呢,他现在也会是一个社会主义者,而且是一个有着强大政府监管的社会主义,也就是类似现在中国有的体制。那中国这个体制,不足以自行;他这个内部的好坏,文化、风气、纪律,那都是另外一回事。我们并不是说你有了这个理性的人本主义,就一切解决了。但是我们必须要了解,事实上当前中国的这个体制跟儒家的理念是大致符合的。就是依托社会跟政府组织发挥那个效率,然后再加上公共道德来为全民谋求公益的最大化。那这个过程中尊重专业、尊重理性、尊重事实,尊重科学。那现在这个中国最大的问题可能是那个道德层面。



13:38 

我想说的是,在西方有没有理性的人本主义呢?也有。因为这个人本主义是一个非常合理非常自然的公理,一个非常自然的价值观。所以古希腊如果去看他们的哲学思想,也是理性的人本主义。但是正如儒家跟墨家有不同的着重,他们也有他们自己的偏重。他们其实更接近儒家,这个就是苏格拉底跟柏拉图的的说法。但是我要强调的是,苏格拉底一辈子最鄙视的就是民选制,最后是被雅典的全民公投判处死刑。因为他的啊因言获罪。我们说希腊罗马的文化,我们要先想清楚啊。这个他的这个精华到底是苏格拉底的理性、人本主义,还是那个公民普选制呢?很明显的,苏格拉底认为他是糟粕。那个普选制是个糟粕,而且,我想任何理性的评估也会发现他是糟粕。因为他们一旦遇到挑战,几十年内他们的霸权就消亡了。现在我们现在会认为,西方文明是继承希腊、罗马的民主自由思想。这个什么普世价值啊,什么西方文明的珍宝啊,这其实都是盎萨在过去两百多年三百年编出来骗人自欺欺人的话。因为如果是普世价值的话,这个必须是演化出来的。就是在演化过程中有价值的。那人类在演化过程中绝大部分的时间是生活在所谓的啊小部族里面。那这个小部族的规模呢是大约五到二十五个人,基本上都是家族。你不是兄弟姐妹,就是表哥表弟。那你这个有什么民主可言,怎么可能是一个演化出来的普世价值?反过来说酒色财气这些东西,反而是真正的普世价值。所以我觉得盎撒用来虚伪宣传,来自欺欺人,来宣传洗脑这种东西,对他们非常有利的。他们对内要维持财阀巨富的地位,对外要为他们的genocide、种族谋杀跟奴役来证明。所以有这个需要。那其他的人是被奴役的,你去相信这个干什么?真的是,我觉得是非常可笑。



16:39

事实上我个人认为,你如果仔细去看英国的盎萨的的历史的话,你会发现它继承的根本不是希腊的哲学,也没有继承希腊的民选制。他模仿了希腊的民选制,但是它的内涵是完全不一样的。我为什么这么说呢?盎萨是Anglo-Saxon,他们的历史是什么?他们原本是介于丹麦跟荷兰之间的日耳曼部族,就是有一点北欧血统,也有一点日耳曼血统。Anglo是在五世纪的时候入侵,就是当西罗马帝国灭亡之后,他们入侵英格兰;然后Saxon在六世纪的时候去。从八世纪、九世纪到十世纪,其实大不列颠里面经常受到新一波的北欧侵袭,就是所谓的维京人Viking。那Viking事实上当时还占领了相当大的一块苏格兰的土地。然后到了十一世纪,英格兰终于被统一了。谁统一的呢?诺曼人, 就是维京人先到法国定居了几十年之后自称为Norman。所以呢它其实是一个北欧海盗跟法国封建制度的综合体。然后呢,英国的上流社会,所有的地主,还有贵族,他们全都是诺曼人后代。他们的文化,他们的政治规矩,都是继承北欧海盗。所以呢这个他们的这个民主,其实是一个分赃制的民主,而不是希腊雅典的那种城邦公民式的民主。



18:25

跟你讲一个数据就可以证明这里面有很大的差别。在1803年的时候,也就是拿破仑正在横扫欧洲。然后英国准备要把它打垮,然后建彻底建立全球霸权的时候,你知道选英国有选举权的人占人口的多少吗?3\%,这其中还有很多是贵族的亲近佣人。他们虽然有投票权,其实他们是贵族的会计师,贵族的律师,他们投票都是听着贵族叫他们投什么,他们就投什么。你说这种真的是民选制吗?就算他后来真的是变成民选制,这个民选制跟他成就霸业啊的成功有关系吗?不可能,你除非是先果后因,因为他的霸权在1803年的时候已经奠定了,就是他只需要再过九年把拿破仑打下来,就再无法怀疑。但是事实上在1803年的时候,已经完全奠定。我为什么挑1803年,因为一八零三年的时候他们扩大了投票权,然后慢慢到1830年的时候,终于有百分之十几的人有投票权。但是你不论如何,你不能说他这个强大是因为他民主。



20:25 

所以我刚刚这个讨论这个历史的细节,是要强调他们的这个民选制跟他们的霸权完全没有关系。然后你现在反过来看,为什么大家都相信民主?因为十九世纪的霸主英国跟二十世纪的霸主美国都在推销,但是他们称霸并不是民选制的结果;而是民选制,反而是他们称霸的结果。而且事实上他的这个民选制是拿着希腊的前例,而且是希腊核心哲学认为是糟粕的一个前例,用来掩盖修饰他们极度的阶级不平等。所以我觉得我们读历史必须要用逻辑辩证仔细的去检验一个细节,仔细的去考虑那个因果关系。如果比如说时间有先后的话,后发生的事情就不可能是先发生的事情的。因那你这个就可以得到正确的结论,而不会被简单的印象影响。



21:21 

这些简单的错误印象就是比如说像日本在明治维新现代化的时候,他们以为什么都照抄就可以了,对不对?日本人照抄的那个的程度,在1950年的时候,他们从美国进口了几套发电机。二十年之后需要维修的时候,请美国原厂的工程师去看你这个发现你有什么问题。他们发现日本人自己仿造了几台,发现这些仿造的上面都有一些额外的钻孔。那个工程师就说,哎,奇怪,我们的图纸上没有这个这些孔。这个日本人就说,我们是照着你的那个原件仿制的,怎么可能有错呢?那个工程师就去原件看了一下,然后看到那个钻孔的附近有一些螺丝拴紧的痕迹。哦,他说这是当初用轮船运过来的时候,为了要在固定在轮船上,所以特别开的孔,结果你们也是照样钻了。那我觉得这个民选制呢,就是我们当初引入西方启蒙时代的理性的人本主义的时候,顺便引进过来的钻孔。

22:46 

另外我一个想要谈的就是刚刚提到这个英国其实继承的这个文化核心是北欧海盗。如果是我博客的长期读者的话,应该会注意到两年多。三年前我特别突然的写了一篇博文去讨论瑞典的文化。然后我想至今绝大多数的读者还是摸不着头脑,为什么为什么我会去讨论瑞典的文化啊?因为瑞典就是北欧文化的代表,它不但是盎萨文化的起源,而且在一九六零年代,白左刚要滥觞的时候,他也是引领潮流。所以你即使是在现代二十一世纪的这个政治社会发展,去了解瑞典的社会跟政治,还有他的思想文化还是很重要的。



23:47 

我这里提一个笑话啊,就是两个月前有一个在乌克兰报道的智利记者叫做Gonzalo Lira。他被乌克兰的纳粹政府给关起来了,因为他老是说实话,不给人家情面了。他被关了两个礼拜。曾经有一段时间大家很担心他已经被处死。但是后来他因为他的,外国国籍的关系呢啊得到释放。我们大家都很高兴。那释放之后呢,他仍然是继续的做视频,到上个月大概三个礼拜前,他发的有一集视频,他也是了解瑞典重要,他是为什么谈瑞典呢?因为瑞典忽然要加入北约了,他开张明义地就说我旅行遍布全球,我对很多民族都有所了解。在所有这些民族里面,天生生理基因智商平均智商最低的远远最低的就是瑞典。然后我就笑歪了。事实上你即使在美国的话呢,你如果在美国住久的话,也会知道他们有很多这种笑话。就是所谓的blonde Joke。就这个美国有一系列这个笑话,就是说这个有一个金发的女人,她有多么多么多么笨。那事实上这个暗指就是瑞典人。



25:05 

当然,我必须要声明,我觉得他的这个论断是有点过分的。就是事实上在一个族群里面,个体的差别远大于族群与之前的之间的差别。那即使瑞典真的他的平均智商比其他的民族低一点,他们这个族群内部仍然会有很多聪明人。不过我我在这里对学历史或者是学外文,或者是西方社会学、政治学的啊,读者建议你们啊去研究一下。瑞典他们北欧海盗的文化,他们的那个封建制度跟文化,他对昂萨后来的制度有很大的影响。然后另外一个是一九六零年之后,他真的是当代白左文化的缘起之处,并不是他有意要用这个白左来污染世界,而是盎萨的左派财阀,认为他很有用,可以用来利用来扭曲社会主义的思想,就是把阶级之间的矛盾转移到其他这些鸡毛蒜皮的小事上。那所以他们特别引进了,就鼓吹了这套白左的思想。那现在当然是对世界的影响很大,所以我觉得值得去研究一下。好,这是第一部分。



史东:26:35 

我一直在听你讲,我一直在想这个事情,有关于民族,有关白左,有关于他们的价值观念这些事情,你也谈了一下。我的问题是你觉得这是一种自然发生、自然产生、或者自然演变出来的一种今天我们看到的结果,还是有一种力量在后面操作而成的今天我们所看到的这种结果。



王孟源:27:04 

我认为他是自然涌现,就是有一些傻子自然想出来。但是那些呃财富权力阶级认为他有用。你这些刚好六零年代学运的时候,他们喜欢去示威闹事嘛。你要找一些事情满足这些年轻理想主义者的冲动。而且,他一开始最大的作用就是要让他们忽视阶级矛盾。因为在国际上跟盎萨的这种资本主义社会,资本主义市场,自由市场社会之下,最基本的矛盾都是阶级矛盾,就是财富差别的矛盾。要缓解这种阶级矛盾的需求呢,就是像是一九二零年代三零年代的工运,那个都是所谓左派,就是二十世纪左派的那个骨干。那到了六零年代,到了七零年代,那些财阀开始做这些智库,想要污染扭曲这个整个国家的文化跟思想方向的时候。他是两条腿走路。右派的话呢就相信自由主义。自由主义就是两个简单的信条。第一,政府是问题之源,这是雷根的口号;第二个是greed is good,就是自私是一个好事。对,贪婪/自私是好事。这个就是定义了现在美国的右派。然后阶级矛盾,最大的威胁永远都是来自左派社会主义者。所以你也要给他们一些事情去做,让他们消耗他们的精力时间。但是更重要的是,在中间选民,一般老百姓跟蓝领阶级眼中的公信力。那你没有比反宗教,然后去搞一些什么性别,七十二种性别这种事情,更让蓝领阶级反感了。所以现在这个Trump当选其实靠的是蓝领阶级的票,都是很奇怪的。蓝领阶级去支持财阀。你可以看看这个白左洗脑是有多么的成功。



史东:29:32 

我刚刚问你这个问题,很大一部分就是你最后这句话。这些人,就是我们观察一个事情,我们就知道后面一定会有时候会有一些黑手。我想知道就这个黑手是谁,因为这个黑手的功力跟这个黑手的这个怎么讲这种创造能力,制造议题的能力。从一个我的这种做传播的人来看,是一个非常了不起的能力。只是他用错了方向而已。



第二节预告



“

王孟源:

他是我在哈佛遇到的中国人里面,私德最好的,这个是家教的功劳。所以我后来带我小孩的时候,也是希望能够在品德教育上做到同样的程度。当然我还教我小孩纵横术这些东西。我觉得陈宇明可能不懂昂撒的这套谎言骗术。但是我希望我的儿子能够懂。

他们没有那么强的先见,就是他们慢慢的尝试嘛,因为资源在他们手里对不对?就是有不同的option 过来,不同的选项过来,然后他们挑选,然后尝试一下,哎,发现很有效,很有效了以后就加码。这个白左呢,首先攻击教育界,然后从教育界再到一般思想界,媒体这样子。然后最后就变成铺天盖地,那自然就产生反动。而这个反动对立越强烈,这种为这些莫名其妙的社会议题,像堕胎什么的,越是强烈,大家越是没有时间去考虑这个贫富不均的问题。事实上,贫富不均才是一切问题的根本。

”



史东:31:49 

谈到这个这个议题啊,今天是一个很特殊的日子,black life matters 这个运动我今天才知道,就它有一个纪念日,就是今天。这个运动已经到有一个属于它的纪念日了,我想点出来的是这个。



王孟源:32:11 

你说的是Juneteenth。



史东:32:12 

是的,对,然后我听着我很吃惊,然后我去看了一下,大部分的美国州,并没有adopt这个Juneteenth。



王孟源:32:26 

他们越是这种强推,反动越厉害。就像五十年前硬是把堕胎合法化,你这种事情应该有了全民共识以后,通过国会立法来做。结果呢,你是用走后门把他强推过去。强推过去以后呢,反而凝集反对力量,那成为永久的窗口。

这个是富豪,巨富需要障眼法,所以才有意制造出来的。一样的,其他的白左的议题。所以你看我们到现在还在承受这个后果。就是拜登一旦通胀起来,然后搞得民不聊生,马上就是想办法把这个堕胎的事情又变成一个议题。他那个大法院还没有宣布,他们就想办法泄露,这泄露的是谁呀?一定是民主党的,就是为了要动员民主党。你这一方面共和党可以用来动员他们的群众,民主党也可以用来动员他们的群众。所以,两个党内的权贵、政要都不想要解决这个问题。问题很简单嘛,你就达成共识以后,在国会里面用民主的方法解决了,他不要



史东:33:56 

这个其实说回来就是一种民主制度的问题嘛,对不对?就是因为制度才造成这个问题,他们需要用这种issue 来解决他们眼前的困难,并不是解决他们的问题,而是他们眼前面临的困难。



王孟源:34:15 

他们这个民选制度就是我刚刚讲的,从历史上一开始就是十九世纪一开始就是为了遮掩财阀占全世界便宜的这个事实。在经济上呢,是从殖民地掠夺来满足下层阶级跟中产阶级的一些基本需要。在政治上呢,则用这个投票的障眼法来欺骗他们,让他们以为自己当家做主。刚好三天前我的小孩在跟我打电话的时候,对我发了一个问题。我对我自己的家教是很自豪的。我的小孩长大是一个纯正的社会主义者,二十岁就已经是纯正的社会主义者。他就问我说他忽然想起来,西方的社会学还跟历史学里面的有一整套,几千本书,几十万篇论文讨论为什么新教跟资本主义优于旧教,就是西班牙。因为他们举的例子就是专注在为什么英国跟西班牙在殖民里面胜出。他们为什么会花这么大的功夫去创造那些那几千万、几亿的文字呢?事实上有需要,因为你必须要解释为什么你成为世界霸主,能够到处殖民。人家第一个反映的就是事实真相,因为你去殖民嘛,你去杀人奴役当地人,你必须要找一个借口,冠冕堂皇的借口。这个冠冕堂皇的借口呢当然就是我的制度优势。那你说这个英国的制度是不是比西班牙优越呢?首先我要说你看西班牙的殖民地的话,西班牙人去殖民的时候他是奴役当地人。但是呢,英国人进入美洲的时候,他选的地方是人口密度低的。所以直接就把那些印度人杀光了。你说这是真正的优点嘛,他不能够把这个拿出来当优点嘛,对不对?所以他必须要去谈那个宗教啦,还有市场制度的的差异,对不对?

然后我最后一个论点是,这也是完全独立的论点,就是你即使真的相信英国的制度优于西班牙,这依然不掩盖他们致富称霸的本质,动力是来自于殖民。为什么?你看全世界当时能够殖民的,很自然就只有西欧的几个国家,在西欧的几个国家之中,像瑞典、丹麦、荷兰太小了,葡萄牙太小了,他们一定是不可能胜出。真正大规模的只有三个嘛:英国、法国跟西班牙,所以英国就算胜出,其实也只是三个里面最优秀的。你这三个里面最优秀的话,举个例子啊,这个biden 在2020年的大选的时候对trump 胜出,你能够就用逻辑推演,他是全美国最适合最有资格当总统的人吗?说他比拜登优秀都很勉强。你即使接受他比拜登的优秀。



史东:37:45 

你说的是biden比trump 优秀?



王孟源:

biden比trump 优秀。他也只是两个人之中比较优秀的那个,不是3亿多人里面最优秀的那个,对不对?所以我们实际观察到的事实,你如果用严谨的逻辑去推论的话,得到的结论就跟盎萨宣传媒体、思想界想要给你的印象南辕北辙。他们为什么要给你这个印象?为什么花了200、300年,多少人花了一辈子整个学术生涯,建立在演绎这些说法上面?因为他有实用。为了他的国家跟制度涂脂抹粉,所以自然就会有财阀去资助他们。



史东:38:34 

对,有钱能使鬼推磨啊。这种事情,历史上事情也多了。只是站在我个人的,可能是因为职业上的这种敏感度啊。我一直对于这种西方啊,你说昂撒所使用的这种手段而使他们达到今天的这种地位啊。我觉得需要有很好很好的研究才行,这些手段到底是什么手段,他后面的理论基础,就你刚刚讲到就是创造一些理论基础来justify他们的行为,他们的行动。这整个这个历程我觉得是一个很好的一个观察跟学习的一个案例。



王孟源:39:19 

他们这里借助的是另外一个普世价值,或者说普世现象,真正的普世现象:就是愚蠢。



史东:39:30 

你是说被他们说教的这些人的愚蠢?



王孟源:39:35 

对,因为绝大多数的人天生就不可能建立学习科学方法或者逻辑思辨的能力。所以呢才有这个空隙。那常常有新来博客的读者说你这个王孟源很可恶,一天到晚骂人笨。这个不是我喜欢骂人笨,而是深思熟虑之后,这个禁区就是【你的群众是不是愚蠢】这个事情是昂萨整个宣传体系的避难所。他一旦出了问题,有什么质疑,真正的质疑,他们一定最可靠的避难的地方就是说,我们有选举啊,所以你看这个民众的多数已经这样选了,所以一定是最优的。你唯一能够打破这个避难所的,就是公开的承认群众是愚蠢的,很容易就被外来的媒体洗脑宣传手段而控制。你只有先接受这一点才能够真正地、客观理性逻辑地、讨论政治学。否则的话,盎萨宣传体系只要抛出这点:我们有民选,民意是神圣的。那其他都不用谈了。那如果投票认为圆周率是4的话,那你也必须接受。



史东:41:00 

我觉得你说的这个圆周率这个事情啊,是一个很很尖锐的一种写照,就是对现在的所谓的民主制度的你说攻击或者指责都好。我觉得这是一个太棒的一个一个写照。因为现在...



王孟源:41:25 

这不是我编的啊



史东:41:28

我知道。现在我们所看到全世界的,包括我们关心的在台湾的所谓的“民主”,就是一个投票来决定圆周率的过程。对不对?就是这样子啊。



王孟源:41:35 

事实上整个是一个骗局嘛。因为候选人是谁决定的?是财阀跟土豪决定的。



史东:41:43 

这以后我们专辟一个时间,好好的把这个事情,深入的谈一下啊。



王孟源:41:49 

所以我们必须要知道,这个制度有多重的保护,确定呢你选不出真正。



史东:41:58 

对,所以我刚刚说,他们从哪里得来的这些方法、这种方法论、教条、理论。这是他们的实践,对不对?能够使他们完成这么样子的一种荒谬的情况存在我们眼前,这是不可否认的,这是这个荒谬的情况。然后反过来说孟源,你有没有感觉到最近几年,特别是这次俄乌战争之后,这个世界在改变你可以看到外交人员,不同国家的总统在不同的国际的场合,开始有一个和我们今天讲的所谓的盎萨的这种这种论调,开始有一个新兴的一种言论出来。



王孟源:42:45 

正是如此。其实我讲的这些呢,putin他都已经知道了。但是你如果去看他过去这十几年的演讲,他其实说的很委婉,一直到今年2月开战然后4月的时候他金融战确定获胜之后,他才出来讲实话,你如果去看他前天所做的那个长篇演讲,他讲的就是这个道理,他没有讲到这个内部的制度,他着重的是他们对外交战略上的,对全世界的压榨,就是对外的那一套骗术,那我刚刚强调的是对内的那一套。



史东:43:22 

对对,这个事情我们继续的,会不断的讨论下去。我在其他的节目中也不断的陈述我自己对所谓的民主制度这个事情的失望,到现在的反感,而且你刚刚也说了那句话。我曾经也用过,那是一个骗局。



王孟源:43:44 

从一开始就是设计来骗人的



史东:43:46 

我是后知后觉。我一开始我还是,我也曾经拥抱过民主制度,你知道吗?我也曾经拥抱过这些所谓的美国价值这些东西,我是个后知后觉的人。



王孟源:43:58 

我也曾经是将信将疑,你要说过去,我现在跟你讲一个故事。其实我年纪也不小了,所以年纪大了就会喜欢讲年轻的故事。我刚来美国的时候也是觉得,这是民主的灯塔、世界的人类的终极啊。然后呢,在1989年跟1990年的时候发生了好几件事情,首先是89年夏天的那件事情,那个在中国发生的事情。然后发生了之后呢,你知道那时候能出国到哈佛念书的中国人,那真的是顶尖中拔顶尖的,都是全国挑选出来最最聪明的人。但是呢,他们一致就是反对中共。那我也就这边听着,反正这种是主观意见嘛,主观好恶的话,个人有个人的偏好,我也不能够置喙。但是呢,后来事件结束之后他们就在那边讲,而且这是好几个五六十岁的老头,就是我们现在这个年纪的,口墨横飞的说中共在两个月之内一定会倒台,那时候我才23、24岁,我说奇怪了你们年纪不小了啊,怎么连主观的意愿跟客观事实都没办法分辨,就是你们完全没有提供任何事实跟逻辑,就是我刚刚说了半天的那个逻辑辩证来达到这个结论。这个结论就是凭空抓出来的,而且我可以看得出是从哪里抓出来,就是你的私人的意愿。那个时候就教训到给我说,你不要迷信这个权威,尤其是社科界的权威,很可能他们的科学理性思维能力是0。



45:56 

然后到了1990年海湾战争开打,美国的那个媒体也是各种军事专家在说这又是下一次越战。这个伊拉克的是百战之师,然后比越南要强大的多。我那时候跟我的同学说,奇怪了,越南是背靠着中国跟苏联的补给,而且因为美国不想像韩战一样无限扩大,所以永远不会越界、攻入北越的本土。而且再加上越南是热带丛林,你往里面一钻就什么都谁都看不见,伊拉克这些条件都没有,刚好就适合你美国狂轰乱炸,你怎么能够会说这会有什么困难?他们就说,哎这个你不懂,我们这些做军事专家。然后事后证明这些军事专家都是骗人的,所以我那次得到的结论是什么?就是这些所谓的大众媒体,在这种西方世界里面的大众媒体其实不是,完全不是学术性的,学术性就是求真的,他连所谓新闻性的都只是一个假象,他的本质是一个娱乐性的东西,就是胡猜乱盖,找一些比较上镜头的专家,其实他们只是有头衔的演员,就是说的话真的是经不起推敲。你只要能够用逻辑辩证的规则去,就会发现里面全都是空的。



史东:47:38 

所以说他们不能有,他们不能容许像你这种人太多数量太多的话,他们生意没法做了,节目没人看。



王孟源:47:49 

就是大众媒体先天就不会挑选真正做逻辑辩证的人去谈,对不对?你这个逻辑辩证多么耗脑啊,

而且讲出来一定是得罪财阀,哈哈不可能。



史东:48:02 

这也是他们,我现在讲他们就是,这就是也是资本主义一种必然的现象,他会控制人

控制人买通你,讲的好听是买通,讲不好听就威胁你,就是你不照我的话做的话你就

小则失去工作、大则不知道什么事都会发生。对不对。就是这样子来控制你的言论,然后控制这个社会、控制这个社会的价值观、价值感。然后透过这价值观再去引导他们去做一些,或者是行动或者是赞同一些符合你刚说的寡头利益的一些事情。



王孟源:48:47 

对,其实不只是寡头利益凡是有一点点权利的小官僚或者小职员,他们都可以自由的去搞这些东西,对不对这个是跟体制没有关系的。差别只在于这只是腐化,腐化在任何体制里面都会有。差别只在于资本主义的体制是资本凌驾在政府之上,中国的体制是政府凌驾在资本之上

对,那你政府凌驾在资本之上的话,你至少民众还有办法去反映他们的需求,政府还可能去管他们,资本的话...(?)



史东:49:26 

我想到的就是,这是我很早很早以前,大概是30、40年以前,很年轻的时候我悟到一件事情,我说在某些国家,在美国啊,你当你有钱之后你就有权。在另外一些国家啊,当你有权之后你就有钱



王孟源:49:45 

差别都是政府跟资本哪一个在上面。



史东:49:51 

我才知道。我那么年轻我就悟到了这一点



王孟源:49:59 

1990年的时候刚好是李登辉上台要夺权,他也是讲这一套民主。那时候他派钱复,钱复那时候是外交部长,他也是四公子之一嘛,然后跑到哈佛来给演讲,那时候国民党的那些党工呢都还在,就是钱复来演讲的话自然有一些党工动员起来,然后在事后用力鼓掌啊之类的啊。但是呢因为是在哈佛,所以他还真的是就问听众有哪一个要发问。我站起来发言,我说李登辉很明显的并不是真正要推进民主,否则的话他党内的制度也会民主化。那他这个所谓的民主呢,很显然的就只是为了要刨掉国民党既有的势力,那我认为他很可能是想要搞台独。那对我一个24岁的人都这么明显的事情,钱复气的要死那真的是把钱复气坏了。我不晓得他事后记不记得,那个不知好歹,老是说实话的人,老是说大实话的人就是在下,区区在下啊。



51:16 

然后我跟四公子的打交道还不止这个,沈君山是清华的校长这不用说了。我也是在1990年我遇到一个年轻人,也是台湾来的同学,哎,印象很好,觉得他很诚实、很诚恳,就是人品好你很快就可以看得出来啊。那个年轻人比我小一两岁,然后跟他深谈以后发现他是陈履安的儿子,陈履安的二儿子叫陈宇明,在哈佛念法律系。聊了一下以后,我知道他家教真的很严,而且他爸爸是真的是个老好人那时候我就可以确定他绝对只是被李登辉当棋子用,这个政治前途马上就要完蛋了。所以我那基本那个时候我就可以确定,因为沈君山是呃是个散人,他不是搞政治的人。那钱复的话呢甘心让人家利用,但是他不懂这些纵横术,所以他也不可能站出来。所以那时候你四公子算一算就一定是连战会被李登辉提拔,那果然陈履安没有多久就退休了。后来啊,我跟陈宇明大概也就是聊天,偶尔聊聊天这样子,就普通朋友这样有几个礼拜几个月,但是呢我印象很深刻就是,他是我在哈佛遇到的中国人里面私德最好的,私德最好的,这个这是家教的功劳。所以我后来带我小孩的时候,就是也是希望能够在那个品德教育上做到同样的程度,当然我还教我小孩纵横术这些东西。哈哈我觉得陈宇明可能不懂昂萨的这套谎言骗术,但是我希望我的儿子能够懂。



史东:53:22 

陈宇明现在在干什么你知道吗?



王孟源:53:24 

他在美国当大律师很有名。所以你如果有法律需要的话,我很鼓励你去,我想他大概是美国几百万个律师里面,品德排在前10名的



史东:53:37 

他人在东岸还是西岸?



王孟源:53:41 

好像在西岸吧,我不晓得我只知道他还在美国,还在当大律师



第三节预告



“

王孟源:

席亚洲要打击我的时候,他一直避免去谈是非对错真假的问题,他去引用他们那位姓毛的开国元勋的话,但是呢很好笑的是这个这位姓毛的开国元勋我可以确定他一辈子没有一秒钟想着我们讨论的那个迫击炮的问题。他一辈子没有一秒钟想着要支持席亚洲、一辈子没有一秒钟想着要批评王孟源。但是呢,他就把这个抬出来,其实这个是一种情绪上的暗示、联想,就是你这个王孟源是台湾人,而且是人在美国,我是大陆的,我是军事论坛的坛主,所以呢你们要爱国,你们要支持我。

”



王孟源:55:16 

我想谈一谈话题就是3个礼拜前跟你谈到那个迫击炮的事情,结果扯出一个论战。其实我不想为这种鸡毛蒜皮的事情跟几个混混扯在一起。不过后来扯大了很麻烦,我想让大家了解到为什么是真正成为一个困扰。这个争议本身是很简单的技术性质的事情,那个两个段落就可以解释清楚。所以我已经在我博客解释过了,有兴趣的人自己去看,那到目前为止对手都是顾左右而言他,就是搞一些情怀的什么东西,他们就是不正面回答。真正的事情的本质就是,首先第一个:那个炮弹到底是我所说的那个制导弹,还是他所说的增程弹;然后第二个是:我说他事后为了报复,把我列入这个,因为大陆这种媒体审查很很严格,其实尤其最近几年反而更严格,更严格的话呢,本意是好的,但是就是公器私用,很容易公器私用,就像我也想谈谈河南赋红码这件事情,也是公器私用。你这个健康码本来是一件好事嘛,但是就有实际执行的人员发现可以用来惩罚自己的敌人,席亚洲也是这样子。现在那个观察者网上面,只要王孟源这三个字出现在留言,那个留言自动会被屏蔽掉。我觉得这个是很不幸的,因为我跟..



史东:57:08 

你有没有跟观察者网站其他的人联系呢?



王孟源:57:12 

那一定否认的。我跟他们你交往主要是通过一个科技编辑,但他科技编辑2年前离职了,离职了之后我就懒得跟他们打交道,因为其他人对我的文章很喜欢鸡蛋里面挑骨头,我的文章不是太长就是太短,太诚实或者是太空洞啊。我也不晓得他们这个借口是哪里来的。所以那个编辑离职之后我基本就不写专栏了,我前面我不是提到说我人不在大陆,我怎么知道大陆的事情?观察者网还是我观察大陆内政的一个重要窗户,就是除了敏感的事情他不敢不能登之外,要另外找讯息之外,凡是官方许可的议题,我基本上第一个去找的都是观察者网。然后呢,观察者网也有其他几个比较资浅的编辑喜欢来看我的博客。我不是真的,他们来博客还跟我打招呼而是很明显的他们去写其他的文章或他们自己写文章的时候都是引用我的论点。那这个引用我的论点刚好就是我写博客原始的目标之一啊,我跟你解释一下。



58:41 

因为我的博客跟所有其他博客都不一样,就是大部分的论点都是原创的,而且里面真的有几千个原创的论点,所以你只要发现了这个博客,你就很容易的去转述,然后去假装这个是你的。那我其实我最喜欢的是观网的读者来我的博客找到一个观点,然后观网有相关的文章的时候他们在底下讨论。因为很明显的读者留言,他们不是假装这是自己的意见,而是他们在别的地方读到的,那即使他们不说这是从王孟源的博客来的,他们也没有企图侵占我的智慧财产,这对我来说这是完全OK的事情,而且事实上是很好的事情,因为他把正确的认知传递出去。



59:39 

然后呢,观网的小编辑写自己文章用到我的观点呢,这个我也很欢迎因为他们写的这些文章其实上都是所谓的综合报导之类的,对不对?原本就是引用其他人,他或许没有讲说:我这个是从DW来的、那个是从nytimes来的,但是呢,大家也知道不是他们原创的。所以他也不是在剽窃我。即使他没有说这是,他们通常都不会说这是从哪一个博客来的啊,但他们即使不讲,事实上他们也没有剽窃,实质上也没有在剽窃。



01:00:18

然后偶尔有学术界的人写面对大众读者的文章呢,他如果只是引用一个论点的话,我也不是有反感,因为他这不是学术论文,他没有citation的的责任,然后尤其如果那个论点看起来不一定是我原创的,我待会会举一个实际的例子让你知道,就是这些学者也许觉得我王孟源也只是

在哪一个英文报纸上看到这个消息,所以我转述一下,转述一下英文记者的结论,那他觉得既然我是转述别人,那他也可以转述我,这都是而理自然,我都没有反感。



01:01:00

真正有反感的是,你把我的原创论点,很明显是我的原创论点拿去而且还有意假装是你自己的原创。那即使是这样子我也通常,不是通常,一直8年来一直都没有去追溯。因为我写这些文章原本就不是为了名利、不是为了赚钱,我去维权没有意思,浪费我的时间。我宁可阅读跟思考。我举几个实例让你知道。比如说2014年的时候我写了一篇有关..



史东:01:01:40 

多少?你再重复一遍。20哪一年。



王孟源:01:01:42 

14年。



史东:01:01:43

2014年。



王孟源:01:01:44

对,就是我刚刚开始写博客,我就写了一篇有关印度的文章。那在这里面我解释印度你不能够当成一个正常国家来看,这不是他们的基因有问题,而是他们的文化有问题。就是他们这个印度教呢,教义本身就认为世界是虚幻的,他们是唯心论,印度教是一个唯心论,所以呢他们是绝对脱离现实,你跟他们没有什么逻辑辩证可谈,因为逻辑辩证是基于客观事实,他们连客观事实都这个概念都没有,那这篇文章发表之后呢,我收到一个内地的教授,他想方想方设法跟我联络上说非常感谢你写的这篇文章打破了我们以往的禁忌,让我们能够公开批评印度的民族性。OK,我不晓得这个,我那时候还没有开始看观网,也还没有开始真正注意大陆的事情,所以,我的那篇文章对大陆到底有多少影响我也不知道。



01:02:55

我一直是到2019年才开始注意自己对大陆舆论的影响。而这个观察的方法就是我去观察者网。只要那篇文章跟我以往的博文有点相关的话,我去看他的留言,就是他的留言讨论有没有用到我博文里面所含的原创论点,这个习惯是怎么来的?是我2016年那个Brexit之后英国脱欧之后呢,我关开始关心英国政局,因为我那时候判断这个英国的政局会对国际未来走向有影响,那为了有更深的了解,我甚至去每天去看daily mail的读者留言,就是,我不只想知道那个记者在宣传什么,我还想要知道这个受众读者受众,他们的想法是什么。那就这样读了三年,到了2019年boris johnson上台,大势抵定,那时候我还上你的节目,而且写了两篇博文,仔细解释我认为接下来发展是怎么样的。那就没有必要再去看那个,那时候我就把这个时间转过来看我自己博客。



01:04:18 

另外一个原因是2019年,我不晓得你记不记得。请你帮忙特别把把美国陷阱,就是那个有关法国阿尔斯通的事情介绍给华语世界的时候呢,我想突然我的声名又上了一级,就是为了那一件事情,因为很明显的这个介绍这件事情给华语世界的是我。所以我也想看看那些新读者,就是因为那个事件所吸引的新读者,到博客来以后,他是不是能够去作为那个正确认知的传播者,在观察者网的读者留言来去谈这些事情。



01:04:56 

那我这样看一看以后,到了2019年没有多久好像是陈平教授,有人在问他说他中国对印度的政策有什么看法。我记得是他说印度必然是中国的战略对手,所以必须要留一手,要谨慎不要去帮助他们,我不知道他说的细节是怎么样,就有人来问我说陈平教授这么这么说,你觉得怎么样?然后我就说这个私营的公司自负盈亏,他们要去投资的话你也不能够禁止,你禁止的话就撕破脸了,所以你这个私人的公司要去印度投资你就让他们去,但是国营的企业基本上都是做基建的,这个呢他们其实是拿了国家的补贴来让基建有低成本高效率,那其中有很多隐形的补贴,比如说那个员工的福利等等。你这个东西让国营企业到印度去搞廉价搞基建等于是中国的人民用他们的税收去补贴印度人。所以我跟他们说这个,正确的政策反应是停止让国营企业去印度做基建。两三个礼拜之后这个基本上,这个这个判断在官网就传开了嘛。



01:06:40

然后到了……我想2020年,又进一步,又有人就是因为2019年我做的那个回答,然后2020年又有人来问下一个问题,那个时候美国开始拉拢印度要成立一个印态的反中联盟。有人来问我说问我的意见,那我说这个问的很好,这个其实呢印度是一个很独特的国家,他非常的自私,而且呢无可理喻,就是我以前讲过的。那你任何一个国际组织只要他进去,立刻就失灵了。所以,中国应该鼓励他加入美国所成立的反中组织。那反过来说,中国千万不要妄想自己想要主导的国际组织,千万不要让印度加入,那时候还是在谈RCEP,我就说你这个东西直接就把他踢出去。不过还好后来是印度自己退出了啊,中国没有做出什么决断。那这个这件事情也是后来就变成那个很快就变成观察者网读者留言的标准回答,我详细的讲这些事情是让大家知道我刚刚讲的说,我的一些个人的判断是怎么样传播出去的。



01:08:00

那最近又发生了一件事情让我很敏感,其实就是3个礼拜前我上你节目也提过,就是我也是三年前谈波音的一系列文章被上海交大的一位教授选择了其中的第三篇还是第四篇,好像是第三篇跟第四篇一起拿去,然后把那个扩充然后增补以后呢写成一篇论文,这个这是好事,但是我希望他能够提供博客一个citation。那另外一个遗憾是什么呢,就是他没有来博客仔细的把所有的相关意见全部读完,我一连写了四五篇文章,其实是一系列的从从浅到深。一开始先讲波音737 max他的那个物理机制为什么会坠毁,然后接下来讲他的设计跟制造的问题,然后再讲波音文化是如何腐朽的,然后最后追究到波音这个文化腐朽呢是取自GE的Jack Welch。



01:09:18

你如果去看那个把波音搞坏的那几个高管,都是从GE出来的校友;接下来最后的结论是什么呢?就是美国的商学院在过去这三四十年所教的这一套,就是GE的那一套。所以这一套你商学院训练出来那些MBA,他们所应用的东西呢,刚好就是毁灭美国国内民族工业的最有效的手段。那很不幸的是这位教授,他只讨论了波音衰败的过程,就没有办法,没有去讨论那个GE跟Jack Welch;然后当然,就更没有推到整个商学院的教程的问题。那我想这也许是有意也许是无意的,有意的原因是因为他是商学院的教授,他也许不想得罪中国的商学院。都是都是照抄美国商学院的嘛,对不对?但是这就让我觉得,说这个我应该吸引更多的学术界人士来我的博客啊,直接获取所有的完整的精确的意见。所以,读我的听众们,现在跟大家说声好;然后如果你们认识学术界中的人的话,请你鼓励他们来我的博客看一看。因为这种转述过去的,很可能是片段或者扭曲的,那不是很可惜?



01:10:52

那观网现在把我禁了这个也是有点讨厌。就是一两个月前,我去看观察这个读者回馈的时候,注意到他们有一篇文章是讨论英国的政局。证据那我刚刚跟你讲过,英国的政局是我从2016年就一直关注到现在,其实2015年他们还没有脱欧公投,我就开始写博文一直到现在,而且花了几千个小时读过几千篇文章,几万个留言。才综合得到的一些消息。然后我即使三年前就预言这个Boris Johnson做不满他的5年任期,然后在一年前的时候,我预言他这个难关会出现在今年5月的地方选举之后。所以,观察者网就是有一篇文章,讨论这这件事情。然后我去看,读者真的是有说有人已经预测了这个Boris Johnson会有下台的危险。但是,他不是说那是王孟源的;他说那是一个苏师傅。苏师傅是谁呢?就是席亚洲最好的朋友。所以这个整个事件,对我来说就是很可笑的一件事情,很反讽的一件事情。这位苏师傅就是典型的把我原创的洞见,拿去假装是他自己的;然后偏偏他又是同一个团队,把一些读者反馈能够说“这一个意见是从王孟源博客来的”禁掉了。所以你说这是不是很可笑?



史东:01:12:41 

这个观察者网站他不再和你合作,他是明讲呢?还是说他只是停止交稿?而且还是怎么回他做怎么做?



王孟源:01:12:52 

怎么做?那个编辑跟我联络嘛?



史东:01:12:56 

就是现在离开的那个编辑?



王孟源:01:12:57 

离开的那位编辑。对,他因为他是我博客读者。所以他对我很信任嘛。他如果有什么题目的话跟我邀稿。不过大部分时间是我心情好了,写了一篇文章,然后我觉得应该对大陆的知识分子推广的时候,我就会跟他联络说你看你能不能用。大部分时间他会想办法用。但是有的时候总编辑不管。就是我……我一直感觉那个总编辑不太喜欢我,大概因为就是我不是体制内的,OK?所以我不需要讲中国这个人情;我讲人情我也不可能升官,对不对?那我的影响力就是说实话,所以我如果为了人情而不说实话的话,这完全没有正面(作用)。



史东:01:13:54 

那你就不值钱了嘛。



王孟源:01:13:58 

对,而且从巨观的观点来看,中国什么时候这个所谓的平均情商最高呢?就是晚清的时候嘛。每一个人都是讲人情的高手。



史东:

但是你的情商是可以……



王孟源:

晚清的时候嘛,对不对?哪一个人不是先转着圈子想清楚,这个人情世故是怎么样子,为了自己的利益,还有对方,还有集团的利益。所以想先想清楚。那你觉得现在的中国社会需要的是这样的人?现在的中国社会会缺这样的人吗?



史东:

这是一种……



王孟源:

好像我觉得缺的是能够讲实话的人。



史东:

哈哈哈对!我一直觉得中国是一个这样的:这个社会,我一直觉得中国是一个……有一个情理法

,中国人讲情理法,情字放在最上面,对不对?



王孟源:

那是不对的。



史东:

因为这是向着这种观念的,而且这是我不知道是几百年还是千年下来形成的这种格局,或者这种习惯呢?就是你人和人之间,人和社会之间之间的相处,他一定有一个先后顺序,就是就是三个字嘛。情理法就是这个三个顺序。



王孟源:

对,所以我一直觉得中国不是真正的儒家社会。儒家是理性的人本主义,所以这个理字必须要放在情字上面。那中国实际上,你至少实践出来的这个……



史东:

你觉得正……不是说正常;

你觉得一个正当的国家应该是法理情?还是理法情?还是怎么一个排列?



王孟源:

短期内,立即内是法理情,长期来说是理法情。就是用公理,用理性来修正法律;但是法律没有修正之前,它优于,它的效率优于理性。那你觉得合理吗?



史东:01:16:11 

我不知道,我现在不知道如何回答这个问题;我只知道情理法是值得商榷的。这是我到目前为止这个,我的感觉。就是说如果什么事情从情开始着手的话,这个事情会搞得很复杂。我觉得中国现在很多的这种社会上的现象,或者是国家里面的现象搞得很复杂,就是情的成分太多了。



王孟源:01:16:42 

李敖说这是酱缸文化,他说这是旧包袱。其实不是儒家包袱,是另外演化出来的人情文化。儒家就是我讲的,是理性的,讲究理性的公益为上。上一次我们聚会,三个礼拜前,我们那时候刚好有毒教材的事;然后,其后有唐山的治安问题;然后有那个接下来是郑州的赋红码,事实上这三件事情一连爆发,我觉得是件好事。就是让大众也理解,中国现在的社会文化跟政治文化,病得不清。习近平所做的反腐,任重道远。你记得习近平推行反腐的时候,他说他的目标是什么?“不敢腐”,然后做到“不想腐”,对不对?你“不敢腐”的话,是你那个纪委的责任。。



史东:01:17:40 

还是被动的。



王孟源:01:17:42 

对,但是你这个“不想腐”是什么意思?是这个文化,就是排除了这个可能性。那你这个,像这种公器私用,以私害公这种事情,如果连观察者网的一个小军事编辑都觉得可以堂而皇之的去干,然后事后抵赖没有任何后果的话,这你绝对就做不到“不敢腐”,对不对?这个文化风气就永远没法纠正。你光是纪委在抓老虎的话,没有用。风气是必须要整个社会扭转。中国现在社会还有另外一个腐化的现象,其实习近平也已经动手处理过了,但是我觉得还处理得不够彻底。就是所谓的饭圈文化。



史东:01:18:41 

饭圈文化。



王孟源 01:18:43 

对,他这个饭圈,据我所知是翻译英文的那个fan。所以跟米饭没有关系。它是那个,Fandom就是英文Fandom的翻译。



史东:01:19:00 

粉丝文化吗?



王孟源:01:19:02 

是粉丝文化,对。习近平现在靠的是管自媒体,但是这是外加的。那我这次遇到麻烦的这个席亚洲,他其实就是一个网红。然后他靠的是用一些非理性的论述,来鼓动他的饭圈他的粉丝,来造势。所以,很有意思的是我们平常在那边笑台湾的绿卫兵,或者是香港的黄卫兵,他们是非理性的;然后被人家用一些傻蠢话,怂动之后就冲在最前面。但是,其实大陆的人完全是一样的,他们的这些饭圈文化就是同样的,跟这些黄丝跟那绿卫兵一样的心理。席亚洲要打击我的时候

他一直避免去谈是非对错真假的问题,就去他去引用他们那位姓毛的开国元勋的话;但是,很好笑的是这位姓毛的开国元勋,我可以确定他一辈子,没有一秒钟想着我们讨论的那个迫击泡的问题;他一辈子没有一秒钟,想着要支持席亚洲;一辈子没有一秒钟想着要批评王孟源。但是,他就把这个抬出来,然后,其实这个是一种情绪上的暗示联想:就是你这个王孟源是台湾人,而且是人在美国;我是大陆的,我是军事论坛的坛主,所以,你们要爱国,你们要支持我。



01:20:53 

这让我联想到是什么呢?你记得两年前,Trump在竞选的时候,跑去一个美国的国旗然后把它抱住。你记不记得那个?所以你说,我稍早的时候说,你必须要先承认群众是愚蠢的,否则你这个公共事务的讨论无解。因为这是事实,而且是坏人最喜欢用的一个避风港。因为他们的这个宣传洗脑已经控制了大部分的选民,然后随便哪一个小小的网红,都有一大堆粉丝可以驱动,那更不用说像比如说像法轮功,那个驱动他们的,那些迷信的教徒。你必须先接受群众就是愚蠢的,然后才有可能采纳理性的政策,才能做到高效的治理,才能做到公益的最大化。



史东:01:21:59 

群众是盲目的,群众是盲从的,这是容易被人家利用的。那今天我们讲民主制度,很多就是利用的就是老百姓的盲,盲目跟盲从。对不对?



王孟源:01:22:13 

这是反驳民选制最简单的一条逻辑道路,您只要接受群众多数群众是愚蠢的;那你凭什么让他来?就是由金字塔,最下底层的这些人来决定国家的前途?事实上不可能让他们来决定。如果让他们来,真正让他们来决定国家的前途,国家马上就垮了。事实上大部分时间是假装让他们做橡皮图章,就是两个都是代表财阀利益的人,让你选橡皮图章,而且被人家做了橡皮图章,自己还不知道。



史东:01:22:50 

对。而且自己还认为自己这个图案是真的,



王孟源:01:22:56 

成年载月的花时间花精力,把它视为人生最重要的事。你想想台湾有多少人热衷于政治讨论跟政治事务?其实都是虚功,而且不但是虚功是负功。人类历史,以来最大的骗局。



史东:01:23:21 

你讲的这个比我还更伟大,哈哈哈。



王孟源:01:23:57 

我觉得,唐山那件事情,还算是普通的贪腐,这个例行处理就可以了。就是他的那个帮派很久没有处理,这是公安局懒政的一个结果,这由地方处理就可以了。这个郑州的赋红码这性质非常恶劣,我相信省纪委跟甚至国家纪委会介入,然后会交代一个水水落石出大家要了解,一方面这真正是公器私用的典型,必须要追究到底;另一方面,你在日常生活里面看到有人占公司的便宜,比如说席亚洲占公司的便宜,这一样也是公器私用。如果是你自己的……



史东:01:24:54 

又回到了我们刚刚讲的,情理法。

王孟源:01:24:59 

如果因为人情而放他一马的话,那你就没有资格批评这些真正的腐。那我曾经说过,我在博客曾经说过,大部分人不是反对侵占公益,而是反对别人侵占公益;他自己或者是自己的熟人,侵占公益是完全可以的。我说的就是这些人,然后我也说过黑人并不是反对种族歧视;他们反对的是被种族歧视。犹太人并不是反对压迫,而是反对被压迫,对不对?



史东:01:25:37 

我们都是发生在别人身上我不反对,发生在我身上我就反对,对不对?



王孟源:01:25:43 

那一个理性的人本主义的社会不应该是这样。但是你如果把人情放在法理上面,理跟法治上的话,这是必然会发生的事情。有关那个毒教材的事情,这是我最担心的。因为中国过去10年的改革,虽然是很深刻很艰巨,但是只要碰到教育跟学术,基本上就无疾而终。所以这个我对毒教材的事情是最悲观。其实现在因为有,唐山跟郑州的事情出来,民意的监督已经消迷下去。所以事实上民意跟媒体不可能提供高效的监督,因为太容易有新的新闻来转移注意力。大家的义愤永远都是5分钟的。这个必须要上层有系统而且是有组织的去处理。但问题是他们过去这10年建立的纪委跟监委,都不进入教育界跟学术界。所以等于是给他们一张空白支票?这是很糟糕的一件事情。



史东:01:27:00 

那这件事情为什么当初会形成这种样子的?这是如何开头的?因为这个事情,你一个平常,中等资历的一个老百姓,都知道这种事情不应该朝这个方向走,不应该这么做的;而且他就朝这个方向走,而且走了很久很久了。然后所以这个问题就是说,当初这个事情怎么样走上这条路的,这我觉得,这是一个更重要的一个问题,必须回答的,不是吗?



王孟源:01:27:34 

我认为是,我稍早半个多小时前提到,1989年的时候,我在哈佛遇到那些中国精英,他们是如此的非理性的仇视自己的国家,而膜拜盎萨的制度。那我认为这不是只限于那些出国的人,那些留下来在体制内的人,也有同样的信仰,也有同样的偏见。这个事件反映的是那一代人,非常普遍的偏见。也就是其实3个礼拜前我就说过了,打着红旗反红旗,越是体制内的越是这样。这个非常的普遍,那你也可以了解这是人性自然的结果,你改革开放之后,跟国外一见识到国外的真相。你发现这个贫富的差距那么大,你不会停下来想一想,这个是人家殖民几百年累积的财富,你第一个反应一定是:我们的制度不如人家。所以这些人丧失了信心。那这些人最糟糕的是什么呢?

就是50年次的那些人,我说50年次不是中华民国的50年次,而是那个西元的1950年。那还好这些人在今年年底基本上届满退休的年龄,他们会从政治局跟中央委员会退下去。那这些人退下去以后,我想60年代出生的人他们的人格形成,是在1990年前后;那就已经会稍微好一点。就是他没有经过,改革开放的时候那个震撼,那个时候他们还太年轻。然后当然再过五年十年,在下一代会更好一点;但是与此同时,你不能够让国家就这样烂下去。不能够说,我们等10年之后那一代来解决问题。必须要从最高层出手,针对教育跟思想来整治。



01:29:58 

其实我注意到中国现在问题还是很普遍,而且就是我说的年纪大,就是60几岁的那一代最糟糕。你去看清华大学,他有着全国最好的学生,但是他有没有全国最好的教授呢?至少在社科方面是没有,中央党校里面的反党叛徒多得很。都是很可笑的,真正比较像样的大学。我只(知道)有一个大学,我觉得我的印象很好就是人民大学。他有像是温铁军这样的人,这个是真正有见识的。别的大学我的印象都不太好。像中科大我是一年前我跟你讲,那基本上整个变成一个诈骗集团,高能所也是一样。很不好的事情。

的而做出了一些,他们可能自己都不自觉,或者他们自觉了,但他们还是这样做的一些损害自己国家利益的事情。毒教材实际在教育方面,我对毒教材这个我更怕的是,就是外国主动的在用他



史东:01:30:57 

我对这个毒教材的事情,我一方面同意你的观察,就是有很多人基本上就是崇拜西方的价值观念们的影响力来影响中国的教材。我们在台湾看到,我们在香港看到;我们现在中国看到这个现象,所以我不得不讲这是一个organized action; 而不是一个,而不只是一个国内一部分的人,他们对西方的价值观念的憧憬,而做出来的事情,你有什么感觉?



王孟源:01:31:55 

我认为这是在恶劣程度跟潜在影响,都是我们刚

刚讨论的三个事件中最深远最严重的一个;偏偏也是最可能不会有真正改革的一个。对所以非常的不幸,那我在一年前我就说过我会尽全力来推动中国在学术跟教育上的改革。我在我博客上曾经很不客气的说,国务院有两个大毒瘤一个是科技部,他的学术管理很糟糕;另外一个是教育部,他的学校教育管理很糟糕。那怎么说呢?你只能希望如果社会有了这个共识,或许能够传达到

您如果这边听众有体制内的亲戚朋友的话,请你推荐他们来我的博客看看,来学,来了解为什么我会把科技部跟教育部称为两大毒瘤。这绝对是情商为零的说法,但是我觉得在当前的社会局势之下是有必要的。高层了,对不对?





史东:01:33:16 

对,必须要面对事实。这是一个开始。



史东:01:33:35 

无条件投降之后会是一个什么样的状况?



王孟源:01:33:38 

要求成立新政府、立法禁止纳粹、宣布永久中立、然后割让东部的一半一半国土给俄国,重归内陆国,也就是1922年,刚好100年前列宁把沿海地带划归乌克兰之前,乌克兰是一个小型的内陆,基本上就是回归100年前的那个状态。这个西方的衰弱被暴露无疑,而且他们的邪恶跟双标也被暴露无疑,那俄国打了一场非常漂亮的金融战、宣传战跟军事战。这样一来全世界第三世界被剥削的国家就全部揭竿而起了,但是我其实已经开始担心,就是我们其实正在面临一个全球战略结构的转型期。这个转型的程度,是二战以来仅见的。但是呢你看二战后的时候重建新的国际组织跟结构的时候呢话语权在谁的手里?在战胜国的手里。所以我从一开始就说中国必须要积极参与主动出击



王孟源:01:35:25

我想我们可以谈一谈下一个话题就是俄乌冲突的新发展。这个战争呢,俄国人现在很明显的进入第二阶段之后,他完全不急,他就是完全以杀伤为目的,而且也很成功。他们以天量的重炮火力来杀伤乌克兰的防守。原本一开始的时候我也以为他们会采取德国闪电战或者是美军的那个空地一体战,也就是步战炮协同单点突破两面包抄,然后形成包围这种战法就是运动战。但是呢俄军选择做的是阵地战的杀伤战,我后来想想觉得其实他们是很聪明的,他们的战略目标是要尽量减少自己的伤亡而增加对方的伤亡,然后在战略上呢,他的时间是在他这边,所以完全不急,就是他愿意承受对方因为得到额外时间所做的金融战跟宣传战的那些额外的打准备跟打击。那一旦你不在乎对方增援或者是造假,或者是在经济经贸方面做出的额外的措施的话,那你完全没有必要着急。而且在战术上,21世纪的战斗,而且乌克兰是有整个北约来提供情报跟后勤资源的,所以你如果要做这种突击的话不可能有突然性,就是因为美国的电子侦察飞机跟卫星在第一时间就会观察到,对一目了然,所以你没办法有所谓的突然性。然后呢,现在他们又支援了大量的单兵导弹,就是最有名的就是标枪吗,标枪其实是一个很好的武器,但是他不是为了城市战设计的,他是在开阔战场设计的。



而且刚好昨天有比较可靠的数据出来,让我们知道在过去这4个月北约到底提供了多少武器。我这里有一个清单我简单念一下啊:640个火炮、240辆战车、600辆步兵战车、1,500枚防空导弹、9万多枚反坦克其中包括6,500枚标枪导弹。你想想看他们有9万多枚反坦克导弹,你如果搞那种闪电战的话,你的那个坦克先锋一定是伤亡全部伤亡殆尽,



史东:01:38:13 

这也是不是表明了当初乌克兰跟美国北约方面也想着俄国会用闪电战术,



王孟源:01:36:03 

我觉得事后想想俄军其实是非常的聪明,因为你这个代价是什么、你慢慢打阵地战的这个代价是什么?你的代价是让他们有时间创造一些假新闻,然后让他们做额外的后勤资源、让他们打经贸金融货币战,但是因为俄国已经准备充分了,他们过去这八年其实不止八年在2014年就已经开始准备了。所以他们完全准备要打一场全面战争,你如果看他们的历史的话,这个所谓的全面战争也就是俄国从拿破仑战争开始一贯的大战略。所以既然他已经准备要打全面战争了然后又面对着北约的先进情报跟后勤系统还有先进的单兵武器。你事后想想看,他们的确是选择了最优解,现在一般的共识是俄军一共阵亡了3,000人左右,然后一般认为乌克兰军阵亡4万到5万人,相当成功。而



且这这里阵亡的是包括了第一阶段打心理战那个时候是以战逼合,第一阶段是以战逼合、第二阶段的目标才是要创造伤亡,所以非常非常成功。我认为他的这个战略跟战术选择在未来历史都会成为反复被研究的一个一个范例。



当然我上一次已经跟你提过接下来的等到他杀伤足够了之后,目前他们已经杀伤了大概1/4的职业士兵。现在的伤亡总数是阵亡的是大概4-5万人,那重伤基本上是再加一倍,所以伤亡的话大概是9-10万人。原本乌克兰的职业军人是20万,所以你有一半是说伤亡了,一半伤亡以后这个壮丁普通的士兵可以补充,那军官跟士官你如果损失一半的话,是不可能在4个月内补充起来,那这个就是乌克兰目前最大的问题,那而且每天还在被恶军继续消耗下去。



另外公布的数据还有重装备就是,结果我看了一下发现,基本上我刚刚说北约补充的那些装备呢基本是乌克兰开战时候所有装备的20\%-30\%。但是呢,现在乌克兰剩下的还有的还可以用的装备呢却只有10-20。等于他们开战的时候的装备全部打光了,然后北约援助的还损失了一部分基本上是随时都可以崩溃,我们在等着看嘛,反正俄军也不急,俄军:我如果能多等一个礼拜而少死几个人我愿意多等一个礼拜

史东:01:41:36 

就是你讲的时间在俄国这边嘛,对不对?



王孟源:01:41:39 

我从一开战就说他如果要调停的话,必须要等法国大选告一段落,那法国大选是在4月,但是呢接下来国会选举是6月,就是昨天才刚刚结束,结果那个马克龙虽然是连任了,但是他在国会大败,在国会大败



史东:01:42:02 

这个事情对整个局面有什么样的影响?



王孟源:01:42:05 

短期内没有影响。因为在第二阶段跟第三阶段俄国都不想和谈了,他的第一阶段是俄方急着要和谈,他们以战逼和,但是第二阶段跟第三阶段都没有想要和谈,什么时候会想要和谈呢?他完成第三阶段。第三阶段就是要把接下去那5个大城占下来,乌克兰中部的那5个大城占下来。

那这5个大城占领之后他就会又想要和谈了。



史东:01:42:36 

你刚刚提到那个(Gonzalo) Lira。他有一集视频,就是昨天还是前天,我看到他讲这个结局啊,可能的结局。他说如果俄国现在追求的是无条件投降



王孟源:01:42:51 

对,无条件投降。无条件投降才能说我占领的就是我的。但是Gonzalo Lira认为俄军可能一路打到Lviv。这个我不太赞同,因为那个地方是纳粹的大本营,是反俄的民心很强,所以你如果去的话会有占领的问题,战胜是一是一件事情、占领是另外一件事情。所以我觉得如果是我的话我会只要求无条件投降然后个不去占领全境。



史东:01:43:30 

无条件投降之后会是一个什么样的状况?



王孟源:01:43:34 

要求成立新政府、立法禁止纳粹、宣布永久中立、然后割让东部的一半一半国土给俄国,重归内陆国,也就是1922年,刚好100年前列宁把沿海地带划归乌克兰之前,乌克兰是一个小型的内陆,基本上就是回归100年前的那个状态。



史东:01:44:02 

你觉得这个结果如果是这个结果的话,对于和中国的关系,乌克兰对于和中国的关系会有什么改变吗?



王孟源:01:44:11 

你可以看出从一开始普京筹备这场战争就已经筹备了大概20多年,大概从他一上台1999年上台

他就开始想着这场战争,至少在2004年开始他就开始真正动手准备,为什么说是2004年,2004年就是北约东扩的那一年。今天刚好有人在博客问这些这件事情,我给他一个很明确的回答就是,我认为他2004年知道终有一战之后呢,就开始积极做准备,但是当时俄国是经过90年代的那个衰败的过程,所以根本还没有准备好,而且美国如日中天。所以呢,他一方面鼓励美国继续再留在阿富汗跟伊拉克。



然后等到他总统做满之后,2008年他跟Medvedev换手,换了4年对不对? 那个时候刚好Obama也上台,Hillary当了国务卿,那你记不记得那时候Hillary就特别想要跟Medvedev做reset,就是想要和好。和好的意思就是说,你乖乖的做我们的走狗,回归到Yeltsin时代让我们的资本进去掠夺的那个状态,普京在2004年到2008年虽然还不能够开战但是他已经开始口头抗议,而且已经开始积极整治他们内部的那些寡头,就是当初美国资本进去掠夺的那些人。



史东:01:46:00 

他基本上是一个民族主义者



王孟源:01:46:03 

他是一个爱国的民族主义者,而且是一个很成功的纵横家。然后2008年他因为宪法必须要下台

换了梅的耶夫,但是呢实际上梅德耶夫是比他还要激进的,对一直到最近才被证实,很多人有怀疑。他们是因为有一个扮黑脸、一个扮白脸的需要,那既然Putin已经被西方认定是黑脸,那Medvedev即使他是更激进,必须要扮白脸。所以在那4年里面他必须要强颜欢笑,在国际上做出各式各样的退让。那你要记得在2008年的那个金融危机,真正受打击最重的,虽然这个引发的是美国的金融系统,真正受打击最重的是欧洲,所以俄国也跟着遭殃。然后接下去到2010年

希腊又发生了财政危机对不对?后来这个欧元这样子拖到2015年才算是勉强解决,那一段时间因为俄国跟欧洲的经济是结合的很紧密,所以你如果去看他们的经济成长率也是很受影响。国力真的没有恢复的像Putin原本所希望的那么快,所以他回锅当总统2012年马上美国又回头去搞颜色革命了,然后2014年,其实2013年就开始搞乌克兰的颜色革命,到2014年成功,成功的时候呢Putin觉得还没有全面开战的本钱,所以那是他为什么他只占领了克里米亚,而没有进一步的把乌克兰吃下来。但是呢接下来8年他继续的卧薪尝胆,等的就是这一天,所以我现在事后想想他之所以会这样子急匆匆的开战其实可能是因为真的等太久了,真的忍了太久了,这个你会影响自己的的理性判断,然后事实上他也已经准备充足了,所以你这个战术上的延迟,是否再延迟三四个月其实无关紧要,他觉得反正一定可以存活下去。



他的目标就是:在西方的全面经贸打击之下国家要能存活下去。他在军事上战胜他是一直都是很有信心的,他在大概8年前就有信心了,就很成功嘛。 Nabiullina自己也觉得很吃惊,这个美国跟欧洲的经济比他想象的还要衰弱,这样一来这个西方的衰弱被暴露无疑,而且他们的邪恶跟双标也被暴露无疑,那俄国打了一场非常漂亮的金融战、宣传战跟军事战,那这样一来全世界第三世界被剥削的国家就全部揭竿而起了。而且新冠刚刚好过去这两年,打击最严重的就是这些第三世界国家,斯里兰卡第一个先倒霉,而且在每一个国家之内也是底层民众先受灾,那以往有这种全球性的灾难的时候,而且往往这个所谓的全球性灾难就是来自于欧美自己内部的问题,然后扩散成全球性的灾难,都是由全世界买单,然后呢共同买单的结果呢就是因为欧美有较高的财富累积,他们能够大家都在抢着要买解决方案的时候谁能够负担得起?当然是美国很欧洲的负担得起,其他的人都负担不起对不对?这就是为什么他们会有“体质优越性”,实际上是因为他们有累积的财富。



史东:01:50:30 

有一点啊稍微偏一点的一个题目。在这次战争结束之后你觉得有多大的可能性北溪2号会被重新使用?在整个的这个局势变化之后。北溪1号现在已经已经他的那个输送量已经减少了嘛这是现在的情况嘛



王孟源:

减少40\%



史东:

战后北溪2号会不会是会不会重新使用?因为这个我觉得太重要了对于欧洲来讲。



王孟源:01:51:01 

事实上Putin在这个对欧的能源供应上有两个选择:第一个是他马上全部切断,就是立刻完全撕破脸,然后呢等着看欧洲的那个经济社会崩溃,但是问题是他知道他在宣传站上还没有强到那个地步,他只在第三世界的宣传上面占优,在欧洲本土上面很容易就被欧盟怪到他头上。如果你是真正一下子把所有的能源切断的话,很明显是因为俄国的作为所引起的这个经济混乱嘛。那所以他选择的是刚好相反就是你要什么我就配合你,对不对,那你还是出问题那就不是我的问题了因为我都配合你。他的方案就是这样一来呢欧盟的经济出问题可以多拖几个月,但是拖到最后的时候民众这个仇恨的矛头会指向他们自己的政客,而不是俄国。



史东:01:52:16 

当然就像你讲的这有很多的因素是取决于俄国的行动是什么样的行动,他的行为是什么样的行为。



王孟源:01:52:24 

因为他的大战略从一开始就是全面战、持久战,所以他才能够选择这些很奇怪的选择。你事后你看一看完全都是符合逻辑的,但是要整体放在一起看,尤其是他从一开始就是准备打持久战



史东:01:52:49 

我个人是觉得我是希望看到有一天这个北溪2号跟,北溪1号就不谈了,北溪2号能够能够重新开启,然后我觉得俄国跟欧洲之间的关系,不必要的再向恶劣的方向去走,我觉得这个,如果俄国跟欧洲的关系,欧洲能够更倾向俄国这边的利益的话,我觉得是件好事情不是吗?



王孟源:01:53:15 

我觉得那是短期内最不可能的事情。事实上刚好相反的的担忧才是我现在最啊最忧心的。我跟你解释一下欧盟跟北约的领导都是欧洲版的neocon。



史东:01:53:43 

就是现在台面上的人都是这样子



王孟源:01:53:45 

对,他们国家级的领袖呢核心的三个国家都太软弱了,都不敢站出来维护国家的利益,有三个国家就是德法意,这个问题是你必须要撤换国家政府而且撤换欧盟政府才可能有根本的政策改革。



那在过去这一个多月我们看到的是先由美国国防部出来、然后Biden政府出、然后美国情报单位出来,当然在这之前是美国的媒体,然后接下来是英国的媒体,然后最后是北约出来开始准备甩锅,就是准备接受战败的事实,然后把锅甩到Zelensky头上。看样子好像是美国准备承认这一次的失败,然后专注解决他们内部的膨胀问题,到两天前有报新闻报导说Biden要求Blinken跟Lloyd Austin都停止公开发表仇俄的言论,那更加这落实这是他们已经既定的政策,但是呢事情忽然发生了转折,因为事实上我也觉得this is too good to be true,就是neocon向来没有放弃的的习惯,就是他们一向是要一头钻到底,就是反正世界跟国家搞的越乱对他们越有好处,我说过他们这些人neocon本身都是非理性的疯子,像是你Victoria Nuland她是有家庭仇恨的就是她的祖父是乌克兰来的犹太人,她祖父当初被踢出乌克兰就是被俄国人踢出去的,所以这是家庭的仇恨。



01:56:00

但是他们能够掌权不是靠他们自己,而是幕后的权贵掌权的权贵政要需要他们来搞乱世界、搞乱国家,然后才能混水摸鱼,发动这样的战争瓦解一个国家,他们捞钱的方式是什么样的?有三个步骤,第一个步骤是你花的那个军费本身就可以拿回扣,因为这些军工企业100\%、200的利润到最后都是要跟你这些权贵私底下分的,但是这还只是啊定金,他们已经付了400亿吗?对不对这个就是定金。第二步是你这个对方解体,你冲进去抢他们的资源,抢能源的单子、抢他们的博物馆里面的文物、抢他们的国营私有化的时候的那个股权、最后一步是比较长远的,但是也是利润最丰的,就是继续控制他们的资本市场、继续的炒作他们的资产,比如说1997年他们把韩国搞破产了,所以可以贱价买下三星了对不对?但是这一次你看失败了,失败的是美国,但是那些权贵有没有失败?没有失败。为什么?你看那个第一步那个武器的那400亿不是已经通过了吗?所以第一步拿到了,对不对,不管这个战争结果怎么样,他们总是武器订单是拿到,resource就是本来要去抢他们的油田结果没有抢到,这是有点失算。但是呢,美国是一个已经是净能源出口国,所以这样图利他们的能源财阀,也算是安慰奖,对不对?你看现在他们现在的炼油的利润毛利率是62\%,这种暴利



史东:01:58:10 

从原来的20\%到现在的62\%



王孟源:01:58:13 

一夜之间。我说过了嘛,你有没有注意到你的柴油的价格涨了百分之百,半年之内涨了百分之百,原油价格可只是涨了50\%,最后一个去抢那个低价的资本市场,就是那个对方的企业支持不下去了需要破产注资的时候,哇这下没得抢俄国的,但是有更好更多的,就是德国跟欧盟其他国家的企业。他们眼看着欧盟的的那个经济要垮了嘛,首当其冲的就是,其实欧盟的实力在于什么?他的这个向心力的来源是什么?向心力是拿着德国的钱去补助周边的那些小国,这样周边的小国才会想着急着进来,但是问题是德国付了钱没有掌握到对欧盟的控制,原本merkel在的时候她还可以管得住von der leyen,但merkel一下台von der leyen就自由了、就成为美国人完全是美国人的走狗。所以德国出钱让美国人去把欧盟跟俄国里面打垮一个或者两个都打垮,然后可以渔翁得利。这就是Neocon的如意算盘。



01:59:44 

我说这个渔翁得利不是,我说美国其实指的不是美国这个国家。因为你看美国付出了什么样的代价?他这个军费是美国国库付的、美元的地位因为扣押外汇额而一落千丈。这所谓占便宜的这个美国是美国的权贵阶级,也就是Neocon幕后的老板。所以你不管怎么弄都是他们赢,唯一能够停止这种吸血的就是打倒昂萨霸权,现在因为俄国这样成功了然后旁边还站着一个中国,中国可以提供技术跟资金跟基建,刚好跟俄国互补起来就成了整个第三世界的一个替代选择。你不再需要美国跟欧洲来提供这些服务,所以我们会现在看到全世界揭竿而起的现象是很合理的。



02:00:27 

但是我其实已经开始担心就是我们其实正在面临一个全球战略结构的转型期。这个转型的程度呢是二战以来仅见。但是呢你看二战后的时候重建新的国际组织跟结构的时候呢,话语权在谁的手里?在战胜国的手里,所以我从一开始就说中国必须要积极参与主动出击。我一直说这是确保胜利的必要手段,但是呢你就算胜利已经确保了,你不需要中国出手也会胜利的话,是不是中国就应该袖手旁观呢?你到时候建立新的国际结构的时候,你的话语权是不是会要看你贡献多少,你在做设国际结构设计的选择的时候,中国如果现在不出手,将来的话语权就会降低,这是一种很明显的无形的损失。所以我觉得很不幸的是现在他们因为20大的问题呢,现在没有啊积极出手的这个余裕,这是有点不幸,那我们只能够希望俄国作为一个主导者,不会像昂萨那样吃干抹净,我想他不会像昂萨那么糟糕了,就是最起码他不像昂萨那样喜欢来阴的。明枪易躲暗箭难防嘛,对不对。我觉得有很大的几率是昂萨霸全在这一次这个危机就会倒下去了,但是呢这个取决在他们怎么处理这个通胀,所以呢接下来我们应该讨论通胀。



史东:02:02:52 

OK讨论一下通胀



王孟源:02:03:11 

我觉得有相当的几率是欧盟垮掉、成为第三流国家,甚至解体。然后呢美国熬过这个??

我认为其实这是最大的几率。那当然如果运气好一点的话就是欧盟也蒙混过关,那再严重一点的话就是美国也跟着垮,所以现在的你的par,就是最可能的scenario是欧盟整个崩溃。我说崩溃是经济崩溃啊,他的政治也可能崩溃,但是那个比较不确定。然后美国靠着对欧盟吸血而勉强过关,然后我们再静等下一场经济危机。那与此同时呢,俄国跟中国以及其他的第三世界开始重新整合建立一个新的国际秩序。



02:04:26 

美国通胀最常用的标准尺度是所谓的CPI,这是由劳工部所编纂的,但正因为它是这么重要、而且是官方订立的一个尺度呢,所以有很强的动力要假造数据或更改数据。因为social security是小罗斯福建立的,那建立的时候他就有这个所谓的cost of living index,也就是你的这个福利金自动的随着通货膨胀而提高,那这个通货膨胀是法律上明文定的,就是劳工部测量的这个CPI,那你想想看他们当然是一开始的时候这个social security弄得很慷慨,所以长期下来呢就负担不起,而且历任的政客都会都是忙着减税嘛,对不对。



02:05:25 

你既然不容易去修法、也不容易去征税,那最容易的解决方案就是去改变算法,所以历史上呢有两次大改:一次发生在1983年一次发生在1995年到96年之间,每次的贡献呢都都降低了将近0.5\%的幅度。但是呢,一直没有权威性的结论,你或许会觉得奇怪这么重要的问题怎么没有经济学者去研究呢?因为美国的经济学界是财阀的打手,所以没有人会自讨没趣去讨论这种惹人嫌的话题,1983年的那个改动是什么?他这个改动是把以前你这个房屋房价的波动排除掉,那这样一来你就可以动手脚把这个通货膨胀少算一点。然后呢到了95九六年呢他又动手脚,这次这个呢是所谓的substitution effect,举个实际的例子吧,就是你如果经济好了发展了,那以前吃土豆的人现在觉得要吃牛排了,你不能够说他的这个生活费用一下从土豆的价钱涨到涨到牛排的价钱,你应该说他就是多买一些牛排啊,这个牛排的价钱呢还是跟以前他没有吃牛排的价钱比,也就是没有通涨。你的这个生活虽然优化了,但是费用随之上升,但是不算进去,这个就是所谓的啊substitution,因为你消费的物品被substitute、被替换掉。



史东:02:07:30 

就是你消费的品质增高了。所以说因为消费品质增高而增加的这些消费的数目就不存在。



王孟源:02:07:39 

同品质的消费品的消费数量波动的时候,他们选择能够制造最低通涨数目的那个方法。



史东:02:07:51 

就算是低标了,最低标准了



王孟源:02:07:52 

对,他们这两个方法呢,不是造假的最大贡献。造假的最大贡献还没有改在一开始就已经在里面了,是什么呢?就是这个电子设备。我现在用的这台电脑比20年前的那一套电脑快一万倍,但是呢它的价钱是一样的,对我来说我同样都是打打字,他的这个对我的用途跟我的价值都是一样的,但是呢在劳工部的算法里面他是有很强的通缩作用,因为他的速度快了1万倍,所以你的这个价格呢照理说应该是涨1万倍,那你还一样还是1,000块的话呢,那就算你这个通缩了1万倍。



史东:02:08:49 

这个厉害,这个很厉害



王孟源:02:08:51 

这才是压低通胀最厉害的效应。但是呢他不需要改,从一开始就有。现在要看美国当前的通胀有多么严重,就是最新的数字是8.6,然后历史上最高点是1980年3月14.8,你说好像还差一大截对不对,但是呢 1980年14.8用的是两次大改之前的那个方案,就是高估的那个方案,然后呢现在的这个8.6用的是两次大改之后把它压低的方案。所以你有必要回头到1980年用现在的这套方案来计算当时的CPI,对不对。这一直都有一个博客他在过去十几年已经要钻研这个问题

,他的估计是这个估算呢误差是7个percent,就是他认为现在这个8.6的



史东:02:09:58 

在加7个percent?



王孟源:02:10:00 

在加7个percent。是事实上已经是创造新高了。但是呢他的这个计算我曾经试着去复制但是我复制不出来,就是他没有给足够的原始数据让我去复制,而且他这个不是正式的学术论文没有其他的学者检验过,所以我就一直存疑也就一直没有讨论,那这个礼拜新发生的一件事是什么呢?Larry summers就是Clinton时期的财政部长,后来退休以后当了一会哈佛校长啊,这也是一个传奇人物,所以我花几几分钟跟大家聊聊啊。Larry summers是一个学二代,他的伯伯是paul samuelson,1970年的诺贝尔奖得主,他的舅舅是kenneth arrow,1972年的诺贝尔奖得主,所以呢他在博士生时期呢就有内线关系,然后就有一些裙带的介绍啊,因为你知道美国的社会是推荐信嘛,对不对,熟人推荐是很公开的。但是呢事后证明他其实是美国经济学界最诚实的一个人之一。当然这以这个美国经济学界的标准来做考虑诚实的话,他的其实绝对上并不是特别诚实,但是相对上来说是



史东:02:11:44 

你说标准比较低是吧



王孟源:02:11:47 

标准极低。哈哈他整个就是一个诈骗集团嘛,一个宣传集团。我记得那个asimov在80年代写的,那时候是Ronald Reagan推行那个supply-side economy又称为voodoo economy。他写了一个笑话,他说这个总统想要知道2+2是多少,不是1+1,是2+2,请了一个数学家,数学家说2+2等于4。总统说不行换一个,然后找了一个物理学家,物理学家说我测量了半天都在4的附近,总统说不行换一个经济学家,那个总统说2+2等于多少,那个经济学者把门关上然后悄悄的说:“你想要他是多少”



史东:02:12:46 

这个笑话我听过,传神



王孟源:02:12:52 

美国的经济学界就是这样啊。我必须说了这个中国的社科界,中国的经济能够维持到现在还有相当的实力,而且实业一直发展,他的那个经济学的理论没有去盲从美国是一个好事,一件非常难得的事情。在80年代90年代他们美国的那个整个经济学跟商学院腐化的过程中,中国只模仿了他们的商学院没有全面模仿他们的经济系,这是一个很好的事,很了不起的事情,我不知道其中的细节但是呢他对国家的发展是有很大的正面影响。



02:13:44 

啊好回头讲summers(口误)。我这里想提的就是你看他是一个学二代,他最后呢成为一个美国经济学界目前最诚实的人之一,这是一件好事对不对,为什么?因为你这个养尊处优的公子

固然可能成为纨绔子弟,也可能因为他不必担心汲汲营营的去赚钱,而能够修养出好的品德来,我前面提到陈履安的儿子这就是另外一个例子。所以大家对官二代跟富二代不要有自然的反感,就是他个人的素质才是你应该着重的事情。当然整个社会体制里面应该有充足的阶级流动性,不能龙生龙凤生凤,因为天生的智商、品格还有教育的品质都会有落差,所以你不能够变成一种实质的封建制,现在美国的资本是在什么?就是封建。回头封建。只不过你这个贵族呢不是看你的头衔,而是看你的billions,看你有多少个亿在银行里面,然后呢他们要锁定他们的优势。锁定优势最好的办法就是控制高等教育的管道还有就业的管道,那现在美国的白左横行造成高等教育非常的反常,我儿子在现在是二年级升三年级,他上的学校还算是名列前茅的但是呢他去做校际比赛的时候观察到一个很奇怪的现象,就是从哈佛耶鲁来的反而是最差的,夸夸其谈的空心大萝卜,最强的反而是从没有听过或者是什么州立大学、乡下州立大学来的,最用功最努力最踏实的,他们大部分都是因为没有那种白左的加成,就是没有少数族裔义或者是性别取向加成,而且是贫家子弟,所以很多他们是因为只能负担得几州内的学费,所以就只能够就近读州立大学。这个现象是什么时候开始的呢?就是刚好就是我来美国之后的第二年,也就是我前面讲的1989年,我那个时候的室友是一个MIT的研究生、物理系的研究生,他有一天气冲冲的回来,我说:你在MIT遇到什么什么事情这么让你让你这么不高兴?他说他们这些新生啊,??分之一的女生变成现在1/3是女生,然后徵收的那些女生啊真的是笨到极点了,那个就是刚好1989年就是MIT改变入学标准的那一年。就是为了要满足quota、要有1/3的女生,而放弃了统一的标准。我在这里不是要侮辱女性,但是呢你这种根据身份政治来做quota的结果呢一定就是race to the bottom、素质越来越差,正确的做法保护弱势族裔的做法是给予他们机会。你看现在凡是搞白左的社会,他们那些出头的那些少数族裔都是unqualified的,通通是不称职的,欧洲的von der leyen女人、lithuania的总理女人、Estonia的总理女人、Moldova的总统女人、芬兰的总理女人、瑞典的总理女人、德国的外交部长女



人,刚好就是这次把对俄外交搞砸的主要推手。你看在俄方战胜的最大功臣是谁,一个是他的中央银行nabiullina,另外一个是RT的总编,两个也都是女人,为什么会有这么大的反差?因为俄国的提拔唯才是用,他

真的不是只因为你的性别就提拔你,你这样一来nabiullina凭自己的本事升上来,底下旁边的人才会服气、才能够真正做出对名字的政策选择。否则你一个笨蛋坐在台上,你这个组织马上就烂掉了,不管手下有多少个能人。而且事实上,在美国的组织那个政府组织里面也是腐化的很快。美国的腐化是全面的,这个白左化只是其中之一,你同时是财阀搞裙带关系,你看起来好像财阀是共和党右派、然后白左是民主党左派,其实它是共同的,就是选拔对他们私利最忠诚的、不称职的人员,正因为他们自己知道自己不称职,所以不会择善固执不会在公益上坚持,他们知道他们的地位完全都是靠幕后的财阀所赐予的,所以幕后的才华才能够放心的欲去求。太多了,这个通道太多了,但是不是他们设计出来的,就是他们自然演化出来的,因为他们有几十年去发现了实验跟发现。



史东:02:20:01 

所以再带回到这个通膨这件事情



王孟源:02:20:04 

扯的太远了,对不起。summers的这个新的论文呢,他发现什么呢?



他发现也就是1980年的那个高峰、3月的那个高峰,其实用现在的算法来算会是9.8\%;而 2022年5月的通胀数字是8.6那真的是只有半步之遥,所以事实上现在的通胀已经非常接近当年1980年的高峰。而当时这个高峰是怎么在三年内就掉下来掉到2.5的呢?是美联储主席volcker把它的贴现率调高到20 percent,美国近代最严重、最漫长的经济衰退。



然后summers的论文还进一步的去探讨说,你这个20 percent的贴现率对降低通胀到底贡献了多少?那我刚刚讲说用现在算是9.8,那实际上我们统一用1983年的那个算法是10.3,那降到1983年实际是2.5,所以它降了7.8\%。summers发现20\%的贴现率三年其实呢只贡献了5\%。就是你提高利率就是收紧银根之后需求面就收缩,需求面收缩以后就会有消除通胀的效应,这个通通的效应贡献是5\%。那全部是7.8,那你另外的2.8是哪里来的?是你供给面改善了。就是主要是能源的价格下去了,就是1979年的石油危机过去以后,那个刚好是苏联的石油开始增产,然后把那个石油价格带下去。



他的这个结论就是,你可以看看现在2022年这场通胀呢,其实供给面的贡献呢会比40年前还要高。因为你可以看到现在什么东西都有短缺,人手短缺、原材料短缺、运输短缺等等的。那美国现在缺货的东西越来越多像是奶粉之类的,那这原因呢是什么呢?就是我以前说过的寡头独占,所以他们的预备的余量不够。出了什么意外,(而且最近这两年意外很多),那供给链就不顺了,那这样一来你这个即使假设现在美联储把贴现率提高到20\%呢,三年之内降低可能也还不到5\%,那现在的通胀率是8.6\%降低5\%以后也还是3.6,所以呢你可以看出这个供给面上面

还大有可为。所以我从3月这个欧战争一开始,我就说中国应该采纳专心设计的金融跟货币政策来提高美国供给面的



史东:02:23:36 

提高美国供给面的成本



王孟源:02:23:40 

对,这样就会加速美国的金融危机。这是从战略角度出发的一个可能性。但是呢,事实上现在已经很明显了中国不会主动出手做什么,所以大概要等到年底吧。与此同时呢我们刚刚谈到俄乌战争的时候,还有一个新的发展我忘了提,就是立陶宛,lithuania。



史东:02:24:08 

啊对那个事情很奇怪



王孟源:02:24:12 

我认为这其实是Neocon的反扑



史东:02:24:17 

美国Neocon的反扑



王孟源:02:24:20 

对。权贵跟Neocon一起合作发动这场俄乌战争,但是现在有些权贵包括biden本人开始心了。他怕这个升级到核子战争,所以想要止损,所以过去这一个月才会有biden还有其他的权力核心站出来暗示乌克兰应该割让国土来求和。



史东:02:24:50 

这个跟立陶宛有什么关系?



王孟源:02:24:52 

两天前biden,有消息报导biden正式要求国务院跟国防部都不要再继续发表仇俄的言论。但是呢同一天的华盛顿邮报,就是目前的Neocon的大本营、Neocon在美国的大本营,又登了一篇战书。基本上是说这个俄国是世界公敌,你即使让全世界的第三世界都饿死了也是一个值得的代价,我不是开玩笑他真的说的就是这样,然后10天前已经出来建议乌克兰求和的北约理事长

那个挪威的Stoltenberg站出来说我们会支持乌克兰到底,然后von der leyen站出来,这都是过去这24小时发生的von der leyen站出来说只要乌克兰继续愿意死战,我们就愿意支持你到底。

这很可怕。这个是那个neocon,就是永远不愿意放弃的那些neocon,即使搞出核子战争也要继续的升级下去,这个lithuania的搞的这个东西呢,就是他们去刺激俄国,让俄国出手的一个手段。但是我认为很明显的,俄方呢现在已经胜券在握了没有必要理会他们,要专心的把乌克兰打掉然后在lithuania那个方面呢,等到乌克兰告一段落再去收拾他。



史东:02:26:39 

这个立陶宛的这个禁运,对kaliningrad这个地方会有多大的影响?



王孟源:02:26:46 

不会。因为你从那个你从波罗的海用那个就是圣彼得堡用海运呢,其实也很方便,只不过是因为用陆运的话有些东西更快捷、更廉价。所以事实上只是象征性的挑衅,那俄国很简单就可以不理会、暂时不理会,等到把乌克兰解决了以后再回头。



史东:02:27:12 

所以所以这个这个小小的挑衅基本上也是个non-issue,对俄国来的讲



王孟源:02:27:17 

他们就是想说如果俄国不理的话,他们又有了一个宣传上就是作秀;成功如果俄国升级的话这些neocon原本就希望继续唯恐天下不乱。



史东:02:27:35 

再回到那个通胀这个事情,做个结论。



王孟源:02:27:39 

总结论就是当前美国面临的局势呢,其实是至少跟1980年代是同一个级别的。而那美国的内部他的实业呢,远比40年前要虚弱很多,所以我相信即使中国不出手俄国也会设法下死手把他打倒,免得他卷土重来,因为这个已经撕破脸了,俄国已经摆明了就是要跟你完全决裂。如果有这个机会的话俄国会主导来把美国跟欧洲彻底打趴。现在这问题是因为欧盟自己主动的牺牲自己,所以变成欧盟成为美国去吸血的一个对象。我刚我刚刚提过当然德国经济垮了以后美国人可以用他们的那些美元废纸去买德国的实业BASF,或者其他的那些真正有价值的实业。所以说不定我觉得有相当的几率是欧盟垮掉成为第三流国家、更是解体,然后呢美国熬过这个病。我认为其实这是最大的几率,那当然如果运气好一点的话呢,就是欧盟也蒙混过关,

那再严重一点的话就是美国也跟着垮。所以现在的par、就是最可能的scenario是:欧盟整个崩溃,我说崩溃是经济崩溃,他的政治也可能崩溃但是那个比较不确定。然后美国靠着对欧盟吸血而勉强过关,然后我们再静等下一场经济危机。与此同时呢,俄国跟中国以及其他的第三世界开始重新整合建立一个新的国际秩序。\twocolumn[\begin{@twocolumnfalse}
\section{美式经济学、通货膨胀}
\subsection{20220715}
\end{@twocolumnfalse}]准备部分省略



唐湘龙 0:56

…以我个人的立场我跟大家稍微地declare一下,我的来宾我都没有预设他们发言的内容跟范围,就是基本上,基于他们的专业,他们自己的功课努力,他们所提出他们的观点,既没有要给大家洗脑,同时也没有要去附和哪些特定人的观点。所以如果你愿意去去倾听、去接受,把它当作是你接受讯息跟观念的一部分,那我非常非常地欢迎。



好,今天在我们线上,你看到我的标题题目,基本上就是我在关注的货币区,那从美元从美国,从欧元区、从人民币那中国区,以及如果有时间,我们再谈到了俄罗斯所代表的卢布区。其实卢布区,以卢布过去的这些货币交换的为主要系统的国家还有好几个。



那它代表的到底全球经济的板块的现况是什么样的?大家现在对全球经济的现况都很焦虑,第二个就是说它的未来会怎么样的?尤其当欧元现在兑美元1比1之后,大家觉得欧洲是不是已经完了?是不是沉下去了?



而中国的经济,习近平公开地表达,希望今年还是能够达标 5.5\% 的中速经济成长的下缘目标,这个目标...如果你把你把现在全球经济的现况,以及中国在第二季的封控所导致的经济的迟滞,你会觉得下半年,哇那个难度高了。



但是全球当然关注美国,美国现在的经济状况是乱七八糟的。前两天通膨来到了 9.1\%,它竟然超过9了。不但没有降温,年率反而更高了。那这么高的年率,让拜登现在即使在海外,但是美国的国内炮声隆隆。



每个区块它不只是货币,或者某一些的经济数字的变动,它背后,是不是隐含着全球经济板块巨大的位移?



好,介绍我们的来宾,在我们线上的,那人在美国透过电话连线...我很开心可以认识王孟源,我希望就是比我聪明的、比我更棒的人都能够能透过我的频道能够介绍给大家...在我们线上的王孟源王博士,孟源欢迎



王孟源 3:55

非常高兴再跟大家聊天



唐湘龙 3:58

好,今天因为你人在美国,我们从美国的经济开始。虽然美元非常的强势,美元指数相当于108点多,去年的这个时间,我check了一下,最低到88、89。那这个一篮子货币所计算出来的美元指数,大致上面也就是反映了全球货币大概的位置。你回头一看,美元这一年时间涨了两成,涨了两成表示美元所进口的商品其实都打八折。可是美国却深受通膨所苦,大家哇哇叫,经济的情况看起来每况愈下,为什么?美国的经济现在的问题在到底在哪里?美国的经济是短时间的困境,还是长时间的滑坡?



王孟源 4:50

两者都有。不过你要真正了解这背后的脉络的话,我想深挖几层,然后退后几步,从19世纪开始讲起。



首先你在节目开始前跟我说,我对经济的看法跟市面上的一般人有很大的不同,我想先解释一下。原因是我27岁的时候拿到物理学博士,然后 29 岁转行做金融,我到 30 岁才开始研究经济学。一般的经济学人是十七八岁进了大学以后开始学的,所以那个时候年轻,根本就没有抵抗力,教科书上怎么写他就怎么学,这是一个很大的问题,你如果是自然科学的话,那种都是可以实验验证的,所以谁写教科书其实不太重要。但是经济学这个东西,不但本身是社会科学,所以太过复杂没办法做精确的实验,没办法精确地认定事实,而且因为它事关政府的经济管理,所以财阀有很大的动机去扭曲它的理论、它的整个思想架构。



而事实上,美国的经济学研究史本身就是一个财阀扭曲的历史。我想简单解释一下,很多接触过英美经济学的人,听过一句话就是 Economics is a dismal science,经济学是一个惨淡可悲的科学。这句话是什么时候说的?是200年前英国人说的。说这句话的人,我认为他当时说的时候是太过分了,是 100 年之后这句话才适用。为什么?因为 200 年前这句话刚刚发明出来的时候,他们争议的论点其实是英国的蓄奴制,因为那个时候英国比美国要早几十年废除奴隶制度,所以那时候经济学界就出来说,我们认为每一个人都有充分发挥他们能力的潜能,所以我们支持废除奴隶制,然后就有一个老古板,学历史的作家站出来骂他们,说你这些经济学人都是胡扯。我们现在回头去看,那当然是他的这个论据的事实根据是错误的,那时候英国的经济学家的看法其实是正确的。我个人的看法是:英国的经济学从 19 世纪的所谓古典经济学,一直到 20 世纪早期的凯恩斯经济学都是持续地在正确的方向发展,真正所谓 Dismal science 这个标签什么时候变成合适的?其实是二战之后,美国的经济学才真正是 Dismal science。



这里你可以看到当时英国的学术界,还继续地发展他们的社会科学还有经济学,最后到 Keynes ,我认为是集大成的。Keynes 他其实是理工科的人才,他是用工科的态度去做经济学的,就是做英文里面的所谓的 Rule of thumb 就是简单的经验准则。他承认经济学研究的人群跟社会太过复杂,所以你不能够说我一定要建基本的模型,把人当成原子跟分子一样来做统计。



但是后来的美国人不管,为什么他们可以不管?原因是美国的经济学从 19 世纪末刚建立的时候就是财阀资助的。一般人不知道美国现在名气最大的芝加哥学派,他的大本营是芝加哥大学。芝加哥大学当初设立的时候就是拿了Rockefeller 的钱,Rockefeller 设立芝加哥大学就是为了那个经济系。他为什么要建那个经济系?就是要让那个经济系来鼓吹绝对自由市场主义。你政府都不要监管,我们市场自然是万能的、是完美的。那这一套一开始还很成功,一直到 1920 年代,这个所谓的 Laissez-faire 绝对自由的这个经济学思想在美国是主流。



但是到了1929年的时候,他们搞出那个大萧条。到了1930年代,他们整个名声都臭了。所以,1930、40年代,其实是美国国力凝结(?),而且上升崛起的一个高速时期,为什么?因为他们终于摆脱了资本在思想上的控制。然后在20、30年代有工运,所以他们那时候也是很快地建立工会。然后罗斯福上台以后,这个新政本身的刺激不够,是后来加入第二次世界大战以后,才真正有足够的刺激让他复苏的。但是他建立了一个联邦的监管机制,在以后的几十年都一直是很有用的。我们现在印象中美国的食品就是比较干净、不会有什么污染,这些问题都是在罗斯福任内解决的。那个 FDA 是老罗斯福任内成立的,但是其他的一些主要的监管机构是小罗斯福建的。如果回头到100年前的话,鸡蛋都已经坏了然后在超市以次充好这种事情,在美国是家常便饭,是后来的监管机制、国家强力地执行以后,才改变了社会的习惯。



那问题是到了二战结束之后,美国成为全球的霸主——至少是资本主主义西欧一系列国家所追随的霸主。这时候,大家对大萧条的惨痛的苦楚已经开始淡忘了。1950 年代时候,美国进入现代最快的工业发展期。因为那个时候欧洲跟日本还没有重建,全世界的工业产能基本都在美国。这个时候芝加哥大学就借机借尸还魂,刚好这时候他们找到一个传销界的传奇天才叫做 Milton Friedman,他重新推销这些绝对自由主义的思想。



我们最常听到的一个就是所谓的 Efficient market hypothesis 有效市场假说。这个就是假设,每一个市场上的每一个参与者都是全知全能、绝对理性的,就是基本上这是一个三亿个上帝的经济。那这跟真实世界一点关系都没有,但是就是因为当时他拿到了太多的资助,就是 Rockefeller 的基金还有一大堆人其他基金,例如福特跟卡耐基的资助。然后他去整个美国的教育界,凡是最聪明的、刚毕业的博士生,他出三倍的价钱。芝加哥大学那时候经济系的薪水跟补贴是其他大学的3倍。所以你可以决定去别的学校当一个穷教授,或者你可以去芝加哥大学吃香喝辣,唯一的条件就是你来这边必须要写推销有效制市场假说的论文。所以我们后来看到的这些教科书,都是这样来的。就是最聪明的经济学学者、年轻学者为了自己的生涯、自己的前途,必须要昧着良心说谎话。



第一代的时候,在80、90年代的时候,美国两岸的传统学校、精英学校,东岸的话是哈佛和耶鲁这些学校,西岸的话是斯坦福,那个时候还做了一些抵抗。就是一些老教授说,你们这个胡说八道,我们在30年代的时候,已经都知道你这个是骗人的东西,你现在又重新要来骗人。这个暂时抵抗的结果就是美国的经济学界曾经在80、90年代分裂成两个分支,他们自己内部开玩笑叫做淡水经济学跟咸水经济学。淡水经济学就是芝加哥大学的那一套,因为他很快就散布到美国中西部那个淡水区,然后东西两岸的咸水区说你这个东西太离谱了,我们不接受。但事实上到了90年代后期到21世纪初期,他们也是把咸水区征服了,所以现在整个美式经济学他们完全就是相信这套,已经统一了。



我一辈子绝对不会去读的自传...尤其是商人的自传。我只有军人的自传我还读一读,就是为了军事历史。商人的自传它里面绝对写的是多么的刻苦努力、白手起家、眼光精准。事实上它是先靠诈欺取第一桶金,或者是娶了一个有钱有势的岳父的女儿,然后在事后到处地贿赂,在竞争对手背后插刀这样上来。你想它自传里面会这样说吗?不可能的。所以你看商人的自传,他们千篇一律都是如何如何地刻苦这样做出来。同样的,你去看昂萨他们自己的社会史,还有他们的商业史,就是从一个国家的观点来看,你绝对不能相信他们上面说的东西,



唐湘龙 16:58

都是极端的自我美化的。



王孟源 17:00

对,都是绝对自我美化过的。在这里面,我刚刚提到这个 Efficient market hypothesis 当然有被自我美化(的部分)。那 Milton Friedman 还发明了另外一个东西叫做 Modern Monetary Theory,这个东西也是很典型的,就是被这样美化过了。你如果去 Wikipedia 去读它说的是什么,他会说,我们这个要考虑货币的发行权必须要在国家手里,它代表着主权等等,听起来都是言之成理,但是他实际上的真正的目的就是——你一旦带进白宫、或者带进财政部(的视角)——你可以无限印钱而不会有代价。



这个就像40年前,那个时候雷根搞这个所谓 Voodoo economics,所谓的supply-side economics 供给面经济学,但供给面经济学如果去看它的 Wikipedia 也是冠冕堂皇的一套理论,跟它实际运用起来一点关系都没有,因为那个都是编出来遮掩的烟幕。它的实际目的就是,我的政策要全面向大企业扭曲,因为你想想看什么是供给面,什么是需求面?需求面就是消费者,就是一般的消费者,供给面是什么?在美国是大企业。所谓的供给面经济学就是讨好大企业的经济学。就是我要压榨小老百姓来图利财阀,然后我用这个专有的名词,我不说我图利财阀,我说我是去照顾供给面。所以美国的经济学必须要这样读。



那为什么我能够一开始就看穿,因为我开始接触经济学的时候,已经30岁了,而且我自己是刚刚从高能物理界退下来。高能物理界应该是自然科学里面最数学化、最标准化、最可靠的,但是他们就在我做博士后的那一段时间,变成玄学了,就是他变成那个超弦,完全的原因就是他们实验做不下去了。他们在1974年建立标准模型之后,接下来所有的实验,其实一直到现在快50年了,所有的实验都没办法超过这个理论。你没办法超过这个理论的话,你有一万个两万个做理论的教授每个月都要发论文,那要发什么?就只能发玄学的论文,对不对?过去50年发的论文你回头去看,没有一篇是跟事实有任何关系的。那你想想看,那几十万篇论文不都是玄学吗?因为我自己有那个经验,而且我见过很多个名气很大的高能物理的诺贝尔奖得主在那边胡说八道,而且是台前说一套、私下说一套。所以我进了金融开始做经济学之后,我根本就不会相信他们的表面上给我的那一套台词,



唐湘龙 20:40

所以你认为现在美国的很多的经济学者,甚至于这些了,经济的官员联储会的官员包括Yellen这些人他们也都在胡说八道吗?



王孟源 20:51

Yellen 是真的是不行,这个我从三年前就说,下一场经济危机会是一个通胀的经济危机。你连我一个这种民间的一篇论文都没有...哦,另一个原因,我不必管他们这些经济学的胡说八道。因为我一辈子从来没有写过任何一篇经济学论文。就是我刚才跟你讲的:你要发表论文的话,这个论文必须是主流接受的,就是财阀资助的那些人愿意刊的。你如果要说实话的话,你的论文就发表不了。刚好我不在乎要不要发论文,所以我可以去追求真相。那Yellen的水准太差了。



但是我刚刚说过,90年代是美国经济学界对这套胡说八道最后的抵抗。其后就变成只剩下零星了。现在还剩下...我想我知道的知名的教授只有两个半还在抵抗:一个是Stiglitz,另外一个是Jeffrey Sachs,最后还有半个是Larry Summers。



Larry Summers是因为他并没有直接指着这些人的鼻子骂,说你们是在骗人,但是他自己静静地在做一些研究揭穿他们的真相。其中很重要的一篇论文是上个月出来的,他就揭穿美国劳工统计局所做的这个CPI,就是我们常用的这个消费物价指数。他说,这个通胀指数在1983年跟1996、97年那段时间两次被修正过,所以上一次美国的通胀记录是1980年,最高点达到14.8\%,但是用的是当时的标准、当时的算法。现在的算法你看,你说这个月是9.1\%。Summers是过去40年来唯一一个主流学术界敢去做这个研究的人,他研究的结果是:你如果用现在的算法,去算1980年那个最高峰的话,当时的14.8\%会是现在的9.8\%。换句话说,已经非常的接近了。



其实别说是三年前,就是20年前他们刚刚开始印钞票的时候——就是那个我刚刚提到的Modern Monetary Theory刚刚开始鼓吹说可以印钞票的时候——凡是脑筋正常的人都知道,你不可能凭空产生财富,最后一定有代价。那既然美国发行的国债都是用美元来定价的,国债本身绝对不会有问题,因为你到时候即使还不出来,反正就多印一点就出来了,所以偿还国债从来就不是问题,真正的问题就在于通货膨胀,你印太多以后就会有通膨。



唐湘龙 24:27

我打断一下。所以美国现在的通膨的问题是纯粹是金融面的问题吗?就是他只要收银根了之后,他终究是能够把通膨压下来,那对美国经济会产生什么影响?现在大家讨论的就是说,美国的经济会是会是联准会所期待的那种的软着陆,还是有可能会像是许多企业界警告的可能是硬着陆,他会是怎么样的情况?



王孟源 24:55

这个,你稍安勿躁,让我继续说这个故事,然后这是一个完整的...你要从历史上这样一路看下,要就像我刚刚讲的,我必须要去谈100多年前Rockefeller,你才知道为什么美国经济学是这样的腐烂,他为什么整个学术界就是实质的诈骗集团。



刚刚讲到印钱。美国的印钱其实一开始是Greenspan在90年代的时候开始,他当时不是直接地量化宽松,而是把利率压的过低。利率压得过低的话,很多人就去跟美联储借,然后再经过2000年的dotcom网络泡沫化,那一直到2008年的时候,你的利率那么低,那些大银行自然想办法要去跟美联储借钱,借钱的话就是等于美联储多印钱才能借嘛,因为美联储本身就是一个美元的黑洞,他就高兴产生多少美元就有多少美元,所以你去跟美元处借钱的时候,借的那些就是刚印出来的。到了2008年的金融危机的时候,美联储在Greenspan跟他的继任者的管理之下,印了还不到1万亿。超过半个万亿,但是还不到1万亿,但是2008年那个危机出来了,危机出来以后他们就开始拼命的印。这个时候就是正式的变成Quantitative easing,原因是你的利率已经砍到零了,那利率是没有办法变成负值的,至少不能够变成很负的值。



所以一但到了零以后,他们就决定,干脆就说我要印多少,就是把它正规化。美国的观念就是正规化,正规化以后,这也是另外一个话题。美国人喜欢把这种坏事或者是贪腐这种事情通通都正规化、法制化。其实因为他是资本至上,就是那个资本在政府之上,所以政府的管制里,如果一切正规化、法制化以后,资本要来收买政客的时候可以让他们竞价,用最低的代价获得最好的服务,这个基本上是把自由市场运用到收买政客上面去,这个是题外话,我不想多说。



我想说的是,当初2008年、2009年开始的量化宽松,实际上是出来声明说,我现在每个月要多印多少钞票。就是不再是...在那之前,我的利率是很低,所以你要来跟我借,我(美联储)是看你有多少需求我印多少。所以所谓的量化宽松,其实是这个差别。在(QE)之前的是不定量的只是定价,在(QE)之后是变成定量的,就是我不管你要不要,反正我每个月就印。那你这样印出来如果没有人要借怎么办?他就直接到市场上去买这个所谓的 Mortgage-backed securities 跟treasury,就是联邦债券跟房贷的衍生物。



然后这样子他印了六年,印到2014年,原因是因为到2014年之后美国的经济才缓过气来。然后他是在2013年、14年的时候 Taper 的。所谓的 Taper 并不是说他收回当初印的那些钱,而是他印钱的速度减低了,那叫 Taper。最后到2014年 Taper结束,就是说他整个完成了QE,到这个时候他一共印了多少?一共印了五万亿。



从2014年到2017年他没有量化宽松也没有量化紧缩。到了2017年他们认为可以紧缩了,他为什么要紧缩呢?因为当时还是有头脑正常的人,认为你这5万亿在这边缎面(?)漂浮的话,永远都会有通胀的压力。他已经开始担心会有通胀压力。稍微有点正常智商都应该知道,你不可能无限几万亿几万亿地印,然后不会有通胀压力。美元有1比6的杠杆,因为美国在国际贸易上所占的比例大约10\%,但是作为储备货币——就是人家累积资产的话——用美元计价占全世界的60\%。所以你这样一个是1比6,我印六块钱美元,其中五块钱会被全世界其他国家拿去作为储蓄、交易或者储备等等,只有一块钱留在美国国内来制造通胀。所以有通胀的话,也是其他的国家先出毛病。



而且不只是通胀,任何的经济危机,即使是美国自己创造的,到最后都是别人倒霉。即使是2008年的次贷危机,那些次贷、没有价值的金融衍生品是谁去买下来的?德国的小银行买下来的。最主要的买家是德国的小银行,因为他们都是买通了 Rating agency 就是风险评级的公司,把它买通好了之后说没有风险。那没有风险的话,德国人就傻乎乎地说:“好没有风险,那我花钱去买下”。美国人自己不相信,美国人自己知道这是怎么回事,美国自己不买。所以你在2008年出来以后,2010年就出现了欧元危机,为什么?就是因为德国那时候已经很虚弱,然后在希腊爆一次,所以就变成整个欧盟的危机。



其实欧盟本身也是占这个(储备货币)的便宜。你看,现在欧盟本身能源供应有问题,食物原料供应有问题,谁先倒霉?不是欧盟倒霉,欧盟到全世界高价收购,用以往五倍的价格去收购天然气的时候,谁没有天然气?第三世界国家没有天然气,



唐湘龙 32:47

第三世界国家没有粮食。



王孟源 32:52

对,现在埃及现在有一个粮食危机,巴基斯坦现在没有天然气了,为什么?因为他们出不起这个价钱,这就是美国跟欧洲在过去200年,这种国际自由市场制度下所占的最大便宜:就是你真正出现什么短缺,什么问题的时候,反正价钱上去了,只有我们负担得起我是必须要掏钱。有一个小的危机,我必须要付出一点代价,那别人的话就死光了。因为在欧洲是天然气价格上去,在其他的第三世界是根本就没有天然气,那这个差别就很大。你看斯里兰卡就是这样。为什么几年前会改去做有机农产?因为他们买不起肥料。



唐湘龙 33:49

前面半小时我们稍微整理一下,就说美国现在美元的价格非常高,我让大家看一下美元跟欧元的比值的变动。但是,(尽管)美国现在的经济成长率勉勉强强,可能第二季还维持着正成长一点点,可是通膨很高,美元很高,油价也很高,大家觉得美国的经济可能会出大问题,会吗?那欧洲的经济又会是怎么样的?我们先把美国的部分...美国当下的经济情况是一个短期问题还是一个长期问题?



王孟源 34:29 

都有。我刚刚讲的是长期的那一部分。那你要记得 2008 年的时候这个美联储印钱,美联储没办法发钱。美联储这个钱只能够交给大银行,或者是买联邦债券交给联邦政府。所以 2009 年的时候,你去看奥巴马的政策,他那些钱百分之百都交到财阀的手里。所以这是非常糟糕的事情,因为你这样一来贫富不均。



唐湘龙 35:03 

事情更严重了。



王孟源 35:04 

所以到 2017 年,我刚刚讲到美联储开始做紧收的时候,他努力了两年到 2019 年。事实上在 2009 年的时候,当时的Ben Bernanke就说过“ I wish 我宁可有直升机可以去直接撒钱给国民”,但是他没有办法,因为当时那个时候他只能够把钱借给美国联邦政府。然后奥巴马拿了钱以后马上转手塞给大财阀去了。但是到了 2017 年他们开始紧收,然后两年下来收紧了大概 1/ 5,就是还不到1万亿。然后到 2019 年就发生了所谓的 Repo crisis。嗯, Repo crisis 就是你这个现金不够了,你想想怎么会现金不够?你5万亿的钱只收回还不到1万亿,怎么会现金不够?



唐湘龙 36:05 

因为都在大财阀口袋里。



王孟源 36:07 

因为大财阀都拿去做长期投资跟国外投资去了,所以你这个美国国内的现金就不够。所以美国这一轮的量化宽松不是 2020 年新冠开始才起的,而是 2019 年9月出了那个 Repo 危机以后,他就开始当场原地转向,从量化紧缩转成量化宽松,然后过去这两年多印了多少?又印了5万亿。那这个 taper 是什么时候开始 taper呢? 去年年底开始taper 到今年3月的时候中止量化宽松。因为那个时候这个通货膨胀的的势头已经不太好了。你发你像是像Larry Summers,我刚提到Larry Summers已经开始站出来说我们这个通货膨胀,眼看着要是 1970 年代的重复,其实原本会是比 1970 年更严重的事情,我待会再解释,就是也顺便回答你刚刚的那个问题。但是这个我先讲它的国内,它的这个国内的因素是什么呢?美国在Reagan之后40 年,它采用我刚刚讲的那个 supply side economics 供给面经济学去图利大财阀、大企业,那这个大企业就把他国内的蓝领工作都砍光了,就是一些基础工业都砍光了,都是所谓outsource送到国外去。



唐湘龙 37:50 

没错,嗯,产业外移。



王孟源 37:52 

那这样一来,美国国内的这个劳工的收入有 40 年没有成长,有 40 年没有增长,就是连跟上这个通货膨胀都很勉强。那问题是因为美国在 80 年代把通货膨胀压下去,嗯,然后接着在九零年代苏联集团瓦解,所以美国的政治跟外交的霸权、军事的霸权确立。所以原本在 80 年代的时候,日本跟欧洲的国家曾经试图建立他们自己的货币地位,就是去分散美元的。



王孟源 38:43 

这个美元的地位,到了 1990 年之后马上又反转。就是你如果去看到 1990 年的时候,美元的储备货币地位是40\%,然后从那之后就马上反转,然后到 2008 年之前达到另外一个高峰,就是70\%。过去这 14 年、 15 年又从 70\% 降到60\%。就是我刚刚已经提到这个比率是你可以去看美元地位的一个最重要的指数。你看这个,它指数越高就是杠杆越高,美国的这个发行钞票的代价就越低,因为里面越大的比率是由全世界其他国家来负担的。当年在 1990 年为什么会降到40\%?原因就是在 1970 年代初期,尼克森打破了那个 Bretton Woods system,打破了金本位,然后已经在1970 年代乱印钞票过一次了,所以 1970 年代末期的那个通货膨胀也是一样,因为他们乱印钞票,而且是六七年之后这个后恶果就出来了。



王孟源 40:08 

然后出来之后为什么当时的那一场通货膨胀会那么严重?会来的那么快?原因是因为当时的日本跟德国还有欧洲的那些经济还有点骨气,他们发现被美元欺骗收刮之后,就开始减低使用美元的那个,他们自己的之间的贸易就开始尽量的想要取代美元。这就逼迫美国在 1985 年去搞广场协议,就是真正用政治力把你们叫来,说你们不能够这样子,你们必须要,所以他们这个美元的占比降到40\%,其实是 1985 年的,然后从 85 到 90 就维持 40\% 左右。



王孟源 40:56 

那你现在看美国这个现在这一轮的通胀,它的问题就在于底层的劳工在过去这 40 年被剥削的太厉害。嗯,他的实质收入完全没有成长。那这时候你怎么样安抚他们呢?靠的就是你这个生产先是转移到东亚,然后转移到中国,所以你现在到我们这边百货公司去买的东西全部都是中国制造。



王孟源 41:28 

曾经有一段时间都是日本制造,然后台湾制造是韩国制造,然后现在也是中国制造,那接下去可能是越南或者是其他地方制造。但是你既然这些生活必需品,这些基本的生活必需品价钱没有上去的话,你勉强压得住。就是这些劳工,你虽然薪水没有上涨,但是你的花费也没有上涨,就是每年的薪水上涨 1\%、2\%,但是你的这个花费每年也是上涨 1\%、 2\%,所以这个可以压得住。压了 40 年,但是这 40 年下来,刚好遇到这个新冠,然后又有这个你滥发美元,美国我刚刚说过 70 年代滥发美元只有六七年,它就爆发了很严重的通货膨胀。那为什么 21 世纪滥发美元滥发了二十几年,到现在才爆发通货膨胀?因为现在的日本、欧盟的政治地位比 70 年代要弱得多,他们的这个金融外汇政策根本就不必让美国再搞一次广场协议。



王孟源 42:44 

就是美国日常一通电话过去,他们就乖乖的,乖乖的听话配合,所以才能够拖在那边。但是你拖并不是解决问题,你只是把问题滚得越来越大,以前是照理说 6 年, 7 年就应该问题爆发,结果这一次拖了 20 年,到现在这个问题变得非常的大。就是你这个西方的经济非要倒下一片不可。那我刚刚又讲到这个因为新冠的问题,这些劳工忽然发现我不但要廉价的付出我的时间跟精力,而且连我的命都要赔上去。因为你去看美国过去这两年死的那 100 万人都是什么?都是那个。



唐湘龙 43:37 

弱势的、老的。



王孟源 43:38 

弱势阶层,就是他们没办法得到好的医疗体系,没有办法吃健康的食物,所以都有糖尿病什么的之类的,然后还被迫要去出现在做零售或者是体力活,那个都是需要面对面的、(容易)被传染对不对?所以他们这样一来,这些劳工阶级里面,比较有资产的,比较有储蓄的就退了 4\% 或者5\%,那所以你的劳工的来源一下子缩减了5\%。你现在看到美国的这个。



唐湘龙 44:21 

失业率低。



王孟源 44:22 

失业率4\%,实际上应该是9\%。



唐湘龙 44:27 

就是许多人已经退出市场了。



王孟源 44:30 

退出市场。这就是为什么过去这两年你会看到一方面劳工失业率很低,没有人说需要救济。另一方面那个你去看那个企业的空额,就是要找人



唐湘龙 44:52 

要徵才。找不到人。



王孟源 44:53 

找不到人。就是相对,你去看那个两个资料根本就对不上,因为你的失业率是 3\%、4\%,然后那个企业空额是对应的失业率 8\% 9\% 的情况,原因就差在那 5\% 的人退了。然后留下的那 95\% 的人怎么样?他们说我受不了了,而且你现在这个通货膨胀是实际上明面上 9. 1\%,实际上是 14\% 点多,嗯,我的生活用品的价钱上去了 14\% 点多。我必须要要求涨价,要求我的那个工资上涨,这个在美国现在已经变成一股洪水了。



唐湘龙 45:39 

他势必会影响到就是说年底的选举,那到可能选举完了之后会是比较大的调整。可是那为什么现在感觉上面欧洲可能比美国更惨呢?



王孟源 45:54 

对,我就要。好,这个是一个很好的问题。其实你问的这些问题我在我的博客当中都解释过了,那今天我很快的总汇一下,一个小时其实不太够。我的博客已经有 1000 万字了,所以解释了大概几千个到或者一万个以上现代社会里面的问题,那真正有兴趣的人应该是花几个月去好好的研究一下。



王孟源 46:17 

不过我这里只是简单的介绍一下,你刚刚十几分钟前你问我说这个通胀的结构怎么样?我跟你讲,这同样是美国的经济学腐败之后,被财阀收买之后,他故意把所有的通胀就凝聚成一个数字。事实上通胀至少可以分成四个数字,第一个是你那个工业生产的用品,就是像...,第二个是你的劳动,劳工的工资。第三个是你这个工业成品。



王孟源 46:54 

第四个是你的资产,资产就是像房产或者是股票,你这 4 个东西完全没有理由说同一个方向走,而且它对社会跟政治跟对经济的那个影响完全都不一样的。所以你 4 个东西放在一个篮子里面,统而言之的话,这个是故意混淆视听。那你去看看的话,美国在过去这 40 年他干的是什么?他干的是,因为他有了页岩油的发现,所以它变成了一个能源的出口国。所以它暂时可以压制这个第一类的通胀,就是那个石油的价格事实上到目前为止还没有达到 1970 年代的那个巅峰。你算那个过去四五十年的通胀的话;那第二个是劳工,这个方面其实是你最想要的通胀,因为你这个工业经济发展之后,这个获得的这个成果应该是分配给全社会,广泛的分给劳工的,那这里的问题在于我待会提的一点,所以我待会再解释;那个第三点是这个工业成品,那既然是outsource,就是靠着中国在过去这 40 年把工业成品的通胀压得很。所以前三者都压低以后,这些通胀的压力全部都集中到资产去了。其实从 90 年代末期就是很明显的事情,就是你第一次的dot com是从股票,然后 2008 年那一次主要是债市,现在你拖了二十几年下来,已经变成 everything bubble,全部到处都是bubble,所有的资产都是冲上天的。为什么会有这个 bitcoin 这种东西?就是闲钱太多了,人傻钱多的才会去玩那种东西。



王孟源 49:12 

我刚刚说这个劳工的工资是一个你在考虑通胀的时候必须要最小心的一件事情。我在稍早一点又说这一轮是美国的劳工 40 年来第一次的反抗,要求他们的这个工资也跟着上涨去。为什么会有...我说这是什么意思?因为通货膨胀他有一个滚雪球的压力。就是从 70 年代的那一次通胀的经验,他们总结了就是通胀有一个自我实现的预言的那个效应。



唐湘龙 50:03 

就是越来越高。



王孟源 50:05 

大家都预期这个明年的通胀是 10\% 的话,大家都会要求工资提升10\%,然后企业也会把它的售价提升 10\% ,没错,对不对?那就叫做Self-fulfilling prophecy自我实现。



唐湘龙 50:18 

预期心理所造成的自我实现。



王孟源 50:22 

对。以往这 40 年没有这个危险,是因为这一个自我实现是两步走的。一个企业,一个劳工,那过去这 40 年劳工被压的死死的。但是现在不一样,现在劳工,现在你如果在美国的话就会发现所有那些底层的劳工,他们的那个薪资在过去这一年、两年都成长了 20\%、 25 平均,至少。所以你这个滚雪球的效应就会下去了。那这时候你怎么办?你这时候通胀其实是...,我刚刚讲的是通胀的来源有 4 个来源,但是你去控制通胀的时候有三个方向:一个是需求面,你可以去减低需求,那这个价钱就下来了,对不对?第二个是你提高供给面,你提高你的那个产品的供给,那这个价钱会下来。第三个是降低预期,那当年 1980 年的时候,Paul Volcker怎么把这个通胀压下去?他把短期贴现率提升到20\%。这个有什么效应呢?它用在第一跟第三个效应,第一个效应就是你的这个利率这么高的话就没有需求了。没有人敢借钱,没有人借钱的话,你就必须要开源节流,就必须要节省、不要再花钱,这是控制需求面,然后同时也控制第三点,也就是你的预期,但是他没有办法控制供给面。



王孟源 52:27 

那事实上Summers,我刚刚讲 Larry Summers他上个月的那篇论文他也特别去算, 1980 年到 83 年,他把整个通货膨胀下降了 7\% 点多、 7. 3 。他算了这里面有 2. 3\% 是供给面的贡献,就是因为像是能源增产,当时OPEC增产这样的贡献。只有 5. 0\%是Volcker提升利率提升到 20\% 才对需求面消减,然后才把预期压下去。基本上你可以说这三个方向,各占 2. 5\% 左右。



唐湘龙 53:13 

因为今天时间的关系有其他问题没办法谈,那如果说是纯粹从美国的经济的控制来看,如果要降低通胀的预期同时要增加供给面,那美国为什么对于就是说打掉对于中国的进口关税?因为关税本身就是一个人为操作,打掉中国的关税表现的这么犹豫,拜登会大规模的取消对中国的那个惩罚性关税吗?



王孟源 53:44 

拜登上台之前就是两年前,一年半前,我在博客上就说这个是最简单也最紧要的。我说拜登上台以后最重要的国内的任务是控制通货膨胀,结果他干了什么?他推了两个万亿级的法案。但我必须要说这个,他做的这个事情在经济学上是白痴,就是他们这个 Modern Monetary Theory自欺欺人,他的那些所谓的专家也都被骗到了,尤其是 Yellen 。



王孟源 54:24 

但是至少拜登的这两轮,他所印的 2 万多亿,总共 2 万多亿其中是分配的比较好,就是 1/ 3 到了大财阀, 1/ 3 给其他的地方政府或其他这些组织,然后有 1/ 3 直接发给老百姓,就是比 2009 年奥巴马做的百分之百都给财阀,还有Trump任内百分之百都给财阀。



王孟源 54:54 

老实说他这一次做的是在分配上是比较好的,但是他这个时机的选择非常的糟糕,就是他上台的时候, 2021 年的时候已经是控制这一轮通货膨胀的最后一个机会了,但是他还去继续的印了两轮的钞票。那我到现在才把国内的事情讲完,我们还有 10 分钟左右,我赶快把国外的事情。



唐湘龙 55:20 

你说。比如从欧盟的角度来讲,欧元当然从它的实质开始的发行,大概 2002 年到现在为止,欧元兑美元的汇率已经跌到了新低点,甚至于一度出现了就说 0. 999 兑 1 美元。那这样的一个趋势,大家会认为说那代表的欧元,不只是欧元的短期现象,而是整个欧盟区经济的长期的弱化。是这样吗?



王孟源 55:54 

是的,事实上一直到一年前,就是 Merkel 还没有下台之前,整个局势看起来是欧盟跟欧元崛起的大好机会。



唐湘龙 56:07 

没错。



王孟源 56:08 

因为你这次 2021 年、 2022 年美国的通货膨胀自我崩溃implode, 比起 2008 年有过之而不及,而且因为它是通胀性的衰退,所以它这个汇率非常的重要。那美元占有 60\% 的国际额份,下一个是谁?就是欧元占20\%。那你看美国最近这 20 年搞得天怒人怨。中国、俄国、其他的国家要去美元化,第一选择是什么?都是欧元。所以在一年之前我还在说这个,如果  Merkel 能够继续做下去的话。这一次是欧盟崛起的最佳机会。因为你只要在美国通货膨胀明显化的时候对美元落井下石,因为这个美元如果贬值的话,它进口的这些东西,美国有很大的贸易逆差。



王孟源 57:15 

你进口的这种东西忽然还要再加上你这个外汇率的变动,那这下就变本加厉,那人家就会造成恐慌,那等到人家抛售你这个美元从 60\% 一下要降到 10\% 或者至少20\%,就是比较合理的额份的话,过去这七八十年美元靠着作为国际储备货币所占的红利。



王孟源 57:45 

就是我刚刚讲那个杠杆效应,通通必须要吐出来还给全世界。那这一下子美国就会整个崩溃,因为他们没有办法,整个经济的问题就完全爆发,没有办法再支持他的军事或者是外交上的野心,但是问题是Merkel 刚好退休了,对不对?所以他(欧盟)反其道而行,你去制裁俄国,冲在最前面,而且制裁的手段是对俄国效应很差的,断绝能源进口,但是对自己的损失却是十倍于俄国的。



唐湘龙 58:33 

没错。



王孟源 58:35 

这样一来反过来变成欧洲的经济眼看着要垮了,而且很快就会垮。到今年冬天的话,那个这一次欧洲会经过一次比 2010 年还要严重的经济危机, 2008 年、 2010 年还要严重,就是原本应该是美国去经历这次经济问题,这么严重的经济,世纪性的经济危机。变成欧洲跟日本,因为日本也自愿的把它的那个它的汇率下调,所以等于是这两个地区被收割了,那就是收割之后缓解了美国的通货膨胀危机。那详细的讨论我们没有时间,但是我简单的说就是原本应该是全世界合起来,挣脱美国的这个经济奴隶制度的大好良机,然后其中最占便宜的就会是欧元,而且简单到你不要说是有Merkel 了,你就算欧盟的所有领袖在一年之前全部都中风、瘫痪什么事都不干。



唐湘龙 59:53 

也不会太差。



王孟源 59:54 

现在的欧元也会是对1.5 美元。就是他们什么事都不要干,就是不要去制裁俄国,什么都不要干,结果他们反其道而行把。欧元这下子完蛋了,没有人要会再用你欧元,因为你当初去制裁俄国的时候、去偷强拐骗俄国的国家资产的时候,欧洲也是抢在美国的前面,那谁还敢信任你的银行?谁还敢把资产放在你那里?现在眼看着关键的点,现在唯一不确定的一点,不是会有替代美元跟欧元的货币,就是一定是金砖国家组织会要推出他们新的合成货币。这个其实是战事一开始第二个礼拜我就写了文章说一定是...,这是正确的道路。那唯一的不确定的地方就在于沙乌地阿拉伯会不会加入。



唐湘龙 01:00:58 

他现在已经提出申请了。



王孟源 01:01:00 

已经在讨论了。但是问题在于第一个加入的是谁?是伊朗,那现在沙乌地阿拉伯。



唐湘龙 01:01:08 

会不会跟着进去。



王孟源 01:01:10 

沙乌地阿拉伯最讨厌的国家就是伊朗,但是你又不能让伊朗进去,为什么?伊朗的重要性在于他是俄国从西边退缩回来,向东向南,他向东当然就是对中国,向南是对印度。俄国到印度的话,最好的交通线是从俄国到伊朗然后再到印度。这一条他们在两个礼拜前公布了,叫做所谓的北南交通线。而且在宣布北南交通线的前一天,他们宣布要接受让伊朗进入金砖国家,你说这是不是一个很巧的事情?伊朗同意这个北南交通线就是。



王孟源 01:02:03 

伊朗虽然跟俄国没有直接接壤,但是他们同样是里海的国家,所以那个交通线可以,还有那个管油气管道都可以直接经过里海,不经过其他的国家,直接从俄国到伊朗,从伊朗再走波斯湾到印度。因为俄国的战略需要,所以你必须要让伊朗第一个吃螃蟹进金砖国家。这下一来我就有点担心沙乌地阿拉伯愿不愿意,就是要看,我认为大几率它还是会进去的。就是因为事实上中国也建立了一个很好的名声,就是虽然中国在过去的二三十年在外交上基本上是很消极,很保守,但是他有好处是他建立了一个我只做生意,我不干涉你....。



唐湘龙 01:02:56 

这个对这个第三世界国家来讲真是太好了。



王孟源 01:03:01 

对,这个跟美国是一个很大的对照,所以这是中国现在搞这些东西,参与这些事情的一个很正面的资产。



唐湘龙 01:03:12 

就是每个人就管好自己家里面的事情,但是在彼此之间、在互动合作上面仍然可以展开。



王孟源 01:03:20 

我其实今天准备的材料只讲了一半。



唐湘龙 01:03:25 

没关系,我们下一次可以再继续,因为包括了就是人民币、人民币的未来以及中国经济,因为毕竟第二季的时候是一个因为疫情的关系,一个半封锁的状态。那因此今年的下半年的整个的中国的经济的包括经济政策都高度值得关注。另外你刚因为提到了金砖,金砖国家未来会不会有一个金砖体系的一篮子货币,



王孟源 01:03:57 

一定的。



唐湘龙 01:03:58  

就是金砖区,对,就是金砖区里面就是人民币跟卢布区,以及包括像这印度的 rupee等等等的,是不是就会整合进来,成为一个金砖区的一篮子货币,成为全球金融体系的另外的一个平衡系统?这个是高度值得关注的。



唐湘龙 01:04:15 

刚刚孟源提到过了,换到说人民币区跟卢布区的某种的整合,那他可能就会对于未来的全球的金融的曝险跟避险的思考就会产生了影响。不过今天其实可惜就是我们,因为下次我们再来谈,因为欧元区,因为欧元就像刚刚孟源一直讲太可惜了,欧洲、欧盟是不是错过了一个千载难逢的能够摆脱美国而崛起,真正能够让欧盟跟欧元独立的机会还是错过了这个机会之后一去不回头。没关系,反正下次的时候我们再找,因为孟源已经我跟他说你要做这个功课,他已经做很久了,所以我们再让孟源再找时间把它讲完。



唐湘龙 01:05:01 

感谢Donate部分省略





唐湘龙 01:06:27

拜登还有可能去掉大部分的中国关税。这个问题刚刚孟源没有直接回答道。你认为拜登最后会把中国的这些的惩罚性关税都打掉吗?因为很明显的这个对美国的通膨是有利的。



王孟源 01:06:41 

早就应该打掉了。嗯,我一年多前他刚上台...。



唐湘龙 01:06:44 

但是他会不会这样做?你认为他会这样做吗?



王孟源 01:06:47 

目前看来是他瞻前顾后,因为主要是他的幕僚水准太差了。就是他们不了解轻重缓急。而且太多了,他重用了neocon,这个neocon就是对一切敌对到底。所以我认为最可能的 scenario 是它会减低一些关税,但是只是部分减低,不会全部撤掉。这是最可能的。



唐湘龙 01:07:16 

好,那今天因为时间的关系,大家可以听到王孟源刚刚只讲了一半而已,后面还有一半就是美国以外的这一半。那下回再请王孟源继续把他的一些的观察跟他的预判分析给大家听。今天透过电话,透过视讯的连线,感谢孟源,感谢



王孟源

很荣幸。



唐湘龙

那也感谢所有的我们观众朋友,好朋友的收看。下回见。周末快乐,拜拜。\twocolumn[\begin{@twocolumnfalse}
\section{欧盟还有没有春天?}
\subsection{20220722}
\end{@twocolumnfalse}]Credit: 栗子



唐湘龙 00:00 

好,欢迎来到龙行天下,我是唐湘龙。今天星期五的时间 9 点半到 10 点半了,一个小时的时间,当初我设定这个主题的时候,我就是希望能够能够畅行天下,那不是只从在台湾谈一些细细琐琐的事情,或者台湾的视角,那我希望一些我直接认识的,或者说就我就算不直接认识但是我觉得很棒的朋友们,那我能够邀请他们到这个栏目里面,那借着这个栏目那开放给这些很棒的朋友,他们可以畅所欲言分享他们对很多事情的观察判断。



唐湘龙 01:06 

好,那但是即使开这个栏目这么长的时间,从来没有,像是这个月我几乎把所有的时间都割让给在我们线上的王孟源。这两个原因了,第一个就是说因为我问城市讲我的感觉,我倒不是吹捧我的来宾,而是我觉得王孟源肚子的货很多,那再怎么讲也讲不完。第二个因为我之前也希望王孟源把我过去看到他谈的一些的国际的总体经济,金融方面的那些分析跟判断的能力,可以在当下 2022 年全球一团乱的情况下面,每个国家都遭遇到了不同情况的经济跟金融的危机。这个时候我很想听听看王孟源的看法,同时把这个看法让我们的观众朋友分享。那因此从上个星期开始,上个星期本来我想着一个小时让王孟源谈经济就够了,不,他只谈了三分之一。



唐湘龙 02:07 

好,那上个星期我们谈比较多的美国的问题,美国当然有很大的问,那是一个非常大的泡泡,这个泡泡接下去要怎么收?那个难度是很高的,它绝对有重落地的风险,大家会觉得软着陆,但是你要提防它绝对有重落地的风险。但是 2022 年我觉得可能最困难的会是接下去。



唐湘龙 02:31 

今天我们要谈的在欧元区,在欧盟,但欧盟跟欧元区并不完全重叠,欧盟里面大概只有 19 个国家是使用欧元的,就是就把欧元当作是国币的,它不完全重叠。不过大致上面我们是这样的,认为那欧盟今年所遭遇到的困难,一方面跟美国的过度的连接,第二个当然跟俄乌战争是有关,所以欧盟遭遇到非常严重的结构性的挑战了。



唐湘龙 02:58 

好,那欧盟现在的情况我想天天的新闻都会看得到,德国也好,法国也好,英国也好,或者这两天的意大利其他的一些那些小国家更不要说,几乎没有一个国家是好过的。那怎么办?那欧盟是短时间的问题还是结构性的问题?是一个长线的欧洲的时代的结束吗?好,那因此今天我们所设定的主题,欧盟还有没有春天,我可以先透露一下。



唐湘龙 03:26 

王孟源是悲观的,好了,先介绍呢,我们的来宾,人在美国的老朋友王孟源,欢迎。好来,我们延续上个礼拜我们没有谈完的另外的一部分来,我们今天把重点摆在欧洲区。我刚刚卖了一下关子,我说对于欧盟还有没有未来,还有没有春天,王孟源是比较悲观的,为什么?



王孟源 03:53 

其实一年之前我是对欧盟非常的乐观,就Merkel退休之前是非常的乐观的。为什么呢?因为 Merkel 虽然没有站出来明讲,但是他治下的德国,还有他影响的欧盟的主导官员已经把大战略画出来,而我认为这个大战略是很合理的,很适合当时的国际形势。它这个大战略是什么呢?我们先讲讲欧盟本身的长处跟弱点。欧盟的长处是它是老先进国家,所以它有一些累积的高科技工业,还有他的这个法制跟政治制度都是经验比较丰富的。那所以当然财富累积也是一个资源,那我上一次已经谈过。



王孟源 04:49 

那当时,但是他也有很大的问题,比如说跟美国相对来讲的话,它本身的资源,能源来说,比如说能源这些东西,就没办法自给自足必须要依赖很多进口。但是它又没有完全金融化,尤其是德国对实体经济的保留相当的好,可以说是先进工业国中的典范。然后再你再可以又反过来说,但是它保留的这些实体工业是旧工业,就是 100 年前的工业,钢铁、汽车这些东西,化工这些东西,那在新的这个高科技,就是半导体或者是软件这些方面,软体这些方面它落后了。尤其在互联网,它根本就没有像样的互联网产业,所以这是一个隐忧。然后他在本世纪的头 20 年,他的金融获了一个xxxx受了很强大的打击,就是被美国集团明的暗的来的,暗杀了他们的金融力量。



王孟源 06:07 

这其实是一个很长的故事,所以我就不详细的讲了。但是你如果看欧盟、欧洲,欧盟再加英国所占世界 GDP 的比例,你会发现它最高点是在大约世界 GDP 的 30\% 左右,那这第一次占到,就是在二战之后,它复苏之后,第一次上升到 30\% 是在 1980 年前后,就是美国经过 1970 年代那些能源危机,还有它的那个通胀危机之后。然后美国经过 80 年代的Reagan的那个政治,用他的广场协议做货币霸权,他那个稍有衰退。



王孟源 06:57 

但是在 90 年代,虽然苏联解体,它的那个利润主要是给昂萨拿去了,但是贸易的利润,就是金融的利润被昂萨集团拿去了,但是贸易的利润其实很多进了德国跟意大利,甚至希腊这些国家的手里。所以你如果去看他们的GDP,到 1990 年,然后又刚好又有东西德合并,到刚到 90 年,其实是巅峰了,到了 30\% 左右,然后整个九零年代一直都不错,然后在 2000 年那十年有点问题,但是还勉强能够维持30\%。



王孟源 07:40 

它真正是掉下去就是 2008 年之后,我上一次已经提过了,那一次的那个金融危机,德国的中小银行受伤最重,然后接下去两年之后就出了希腊的财政危机。对,从那之后欧洲就没有再达到30\%,就一路往下滑。那这个额份当然就是让中国拿去。那你可以看出,从这一段简短的历史你可以看出,欧洲其实是在一个衰退的过程中,而且已经开始了一段时间、正在加速。在金融方面,原本欧元在 1999 年开始,然后到 21 世纪正式的全面替换国有货币之后,曾经有一段很风光的时期,但是也是同样在 2008 年到 2010 年那个时段,它的那个额份就卡在20\%。这除了我刚刚讲的这个经济的底气不够之外,还有一个原因是你这个货币的力量必须是要靠你的金融机构在国际上打下来的,就是你的那个你的银行,你的大银行在国际上做金融交易的那个额份有多少?那刚好就是在 2008 年到 2010 年之后,也是一个转折点,欧洲的大银行在那之后都被整得很惨。



王孟源 09:24 

我说整的很惨是有人在整他了,这个整他的是谁?就是美国的联邦政府。这一下子 80 亿,一下子 90 亿的罚款,然后就是说他们这个他们违反什么什么美国国内的法怎么样的,那你如果去看那个这些大银行,欧系银行的市场额份的话,它降低的比欧盟的那个 GDP 的额份降低的还要快。所以事实上欧元现在还有 20\% 其实是因为美国也拿着这个制裁的大棒到全世界敲诈,所以中国跟俄国还有其他的一些第三世界国家原本就一直指望着能对美元有一个替代,所以才维持着欧元一直在 20\% 的左右。



王孟源 10:19 

那么你看这个情势,如果一年前你是欧盟的领导人,欧盟的实际领导人当时是Merkel,如果有一个人能够说我们的大战略应该转向,而整个欧盟就会真的转向的话,那个人就是Merkel。那当时这个战略当然不是一下提出来的,因为 Merkel 并不是一个所谓的战略家,他只是一个务实的政治人物,因为他毕竟是当年苏联体系下学物理教育出来的。那都知道苏联教育体系的理工科是很扎实的,比那个西方的教育要扎实的多。那所以他虽然没有那个战略眼光,也没有那个野心,但他是很务实,他愿意听专家的意见。所以比如说他对工商业怎么发展,他就去问工商业领袖,所以会在乎他们的观点。那这些 CEO 一些是有观点的,比如说大众。他就很明显的站出来说我们必须要跟中国进一步的绑定,当然这是因为从中国赚钱赚了很多的关系。但是你不能够说他们是为了私利,你就不听他们的,因为事实上大众汽车的利益就是德国的利益。我们五六十年前在通用汽车顶尖的时候,曾军他们的 CEO 讲过一句话,通用汽车的利益就是美国的利益。你现在也是一样的,大众汽车的利益就是德国的利益。那所以Merkel虽然没有这些战略眼光,但是他慢慢的,你给他时间,慢慢的会有专家来跟他提意见,然后他也有那个智商来分析,来挑选合适的策略。



王孟源 12:23 

那当时欧盟已经准备好的策略是什么呢?我其实是非常非常的乐观的,为什么呢?因为他这个,行得通,行得通的原因是什么呢?英国跟美国都在自爆。那英国首先自爆,因为英国脱欧之后,英国脱欧主要的原因是因为他当地的土豪为了要能够继续逃税,必须要脱欧。因为当时欧盟在 2016 年的时候提起了新一套的查税的规则,很严格,就是你以真正那个亿万富翁用来逃税的那些花样被禁掉。但是一旦欧盟英国脱欧之后,英国原本是欧洲的金融中心。



王孟源 13:18 

对,而且不只是金融中心,还是整个服务业的中心。服务业是什么意思?举个例子来说,我们现在的这个国际社会并没有一个世界政府,也没有真的完整的国际法。因为你没有世界国会,到哪里去提国际法?这些国际法其实是靠很多条约的,那这些条约当然是有局限性,因为条约往往是双边的或者十几个国家的多边条约,而不是普世性的条约。所以比如有很多国际上的贸易需求,必须要靠第三国来提供服务。



王孟源 14:02 

我举一个确实的例子,我现在两个国家的大企业要签一个国际契约,这个契约要用哪一国的民事法、契约法来规定?你必须就必须要挑选。那英国就是一个很重要的供应,这种法律的供应国。很多欧洲、俄国甚至非洲的国家,他们的大企业在签国际性契约的时候,选择英国作为它的法律背景国。那这样一来他的意思就是说,你签这个契约的时候雇的律师事务所必须是英国的,然后出了问题的时候打官司要到英国的法庭去打,在契约里面就明显的写的,我们这个契约是根据英国的法律写的。



王孟源 14:55 

那英国一旦脱欧之后,这些服务业,不只是金融,金融当然是其中最大头的。这个法律服务业、会计甚至建筑设计其他的一切东西,这种东西都是虚拟的,虚拟的话就是你凭空选择一个,那欧盟当时并没有手软,他们把那个能够抢的东西都抢过来了。所以一旦在 2020 年脱欧没有多久,全欧洲最大的股市就从伦敦跳成阿姆斯特丹,就是你算交易量的话,然后有很多的这个银行转了几千个那种高收入的银行雇员,从伦敦到法兰克福或者是到巴黎,或者到阿姆斯的胆,甚至到马德里或者米兰,所以这是英国自爆的一个结果,就是欧盟可以靠着吸收英国的养分就活得很滋润。



王孟源 16:04 

因为欧盟的这个经济阶段,它的发经济发展阶段刚好就是跟英国很适配,他们这些东西都是高收入的、白领阶级的然后虚拟的那个服务业的。那另外一个是美国,我从 3 年前就开始说美国下一场这个经济风暴会来,而且是会通胀性的。这个通胀性的经济风暴是过去 50 年来第一次,就是从 70 年代对的膨胀,然后到 80 年代初代把那个通胀压下去之后,过去的五六个经济衰退都不是通胀性的经济衰退。所以我们这两代人都不记得这个通胀性的经济衰退是怎么样?那通胀性的经济衰退有一些特点,我待会再仔细讲。



王孟源 17:01 

因为我希望在这个节目结束之前能够给大家讲一些,我们今年还有 6 个月,预期这个经济的态势会是怎么样、还有明年的态势。但是这里必须要了解的是,通胀对美国最大的危险就是威胁美元。因为这个通胀上去之后,你美元的资产贬值的就快、实值贬值就快,那一旦美元资产向外逃的时候,美元的汇率下降,那这样一来更增加通胀的速度,然后就是变成一个恶性循环。这是当时我从 3 年前就一直解释美国将面临的最大的危险。那如果欧盟什么都不干,就坐在那边等着这个危险的话,他可以接手什么?就是接手这个国际储备货币的头衔,因为原本中国、俄国第三世界就是准备把欧盟的欧元拿来做美元的替代,美国在过去 20 年对外的长臂管辖跟制裁越来越离谱。



王孟源 18:15 

原本大家就很急着停用美元,对不对?你这个,为什么他后来说这个四五个月前当这个乌克兰的仗打起来的时候,大家急着谈Swift。很多人不了解 Swift 其实是一个欧洲系统,把俄国踢出Swift,它的意义在于不让俄国用欧元,而不是美元, Swift  其实绑定的是欧元。



王孟源 18:49 

所以我认为当时的一年前欧盟的大战略很简单,你就是等着英国自爆,然后去吸收它的金融跟服务业,然后再多等一年、两年,等着美国的经济自爆,这个通胀再加上货币危机,然后去接收它的这个货币地位、国际储备货币。那你知道这个国际储备货币地位,它的好处就是你拿着你印的钞票去买别人的实体的东西,这刚好非常适合欧盟的需要,因为欧盟就是本身入不敷出,他的这个社会开支太多了,然后他的新产业没有竞争力。那如果能够有的这个注血进去的话,你想想看,美元的额份是60\%,欧元的额份是20\%,如果从 20 涨到60,这个就是40\%。真的是。



唐湘龙 20:03 

不可想象。



王孟源 20:05 

几十万亿的财富凭空掉下来,欧盟可以至少再吃 10 年、 20 年,而不必担心他的那个经济问题。



唐湘龙 20:16 

我这个地方我打断一下,我补充一点东西,当然第一个就刚因为孟源提到了英国当然很重要,因为牛人控xxxx伦敦是全球的金融的中心,一直到二战之后,因为美元的崛起,纽约的崛起才稍微的平衡。那在英国脱欧了之后,这个伦敦有多重要?你从即使是俄罗斯,即使开战到现在英国对乌克兰战争的介入,跟俄罗斯的敌对性这么强,可是你要知道俄罗斯的一些大富豪都躲在伦敦,甚至于都还是保守党的主要的金主在伦敦的很跃吞,那些俄罗斯的富豪几乎都是在英国过日子。



唐湘龙 20:58 

第二个就是说我不太理解,就是说法兰克福有机会取代伦敦吗?第三个刚刚孟源,孟源提到的就是通胀,对,就说你如果看过去十几二十年诺贝尔经济学奖得主,已经没有人再研究通胀的问题了,因为为什么?因为通膨是个冷议题,就是大部分的经济学者,上个星期孟源谈到了很多的经济学者的虚构,那这其实他们都只不过是这些这些财团所豢养出来这些的润笔而已。



唐湘龙 21:33 

那你看到的这些经济学者他不研究通胀,甚至于给世界一种感觉就是说通胀已经不会再回来了。就说这个事就是说通胀已经是过去式,所以就没有经济学者在因为通胀的研究而得到诺贝尔奖,因为它根本就是个冷议题。可是现在他回来了。好,那我刚问的那个问题就是说欧盟虽然它好像失去了一个取代美元的机会,但是它会削弱今天全球在碰到美国的时候,觉得被美国绑得很紧,想要去美元的那个企图会因此而泡沫吗?



王孟源 22:07 

不会,不会。但是现在变成,你想想看这个你刚刚提到的,比如说那些俄国的富豪,这是私人的财产。他们现在都被扣押了,就是没有法院的那个裁判,就直接由政府扣押了。这个你这样子,其他第三世界的亿万富豪还敢把钱放在英国吗?还敢把钱放在欧洲吗?原本欧盟可以就坐在家里面躺平,就是这个昂萨集团会主动的挖自己这个金融霸权的墙角,但是他兴高采烈的去冲在最前面,结果制裁俄国比英国跟美国还要厉害。



王孟源 22:57 

你看事实上美国是,美国跟英国都是公开宣布讲得很恶劣,可是实际上在做的时候就慢吞吞的,或者甚至是睁一只眼闭一只眼,对不对?但是欧洲人不懂,他们就是真的冲在最前面。那我刚刚讲到长臂管辖等等的东西,那个欧盟其实还有另一个进账:就是国际管理。因为你从美元换成欧元这个目的就是不要让美国人长臂管辖。所以这个国际管辖的这个权利,这其实也是一门很大的生意。你看像那个 IMF 基本上就是国际破产法庭。



王孟源 23:43 

还有你的那个贪腐,国际贪腐案,那要谁来办?都是美国自任,自己任命自己当成警察,跟那个法官跟检察官嘛,对不对?这些事情都可以由欧盟接手,尤其最重要的原本就在欧盟境内的像是 w h o 、w t o世界贸易组织,这些都是真金白银的,你要是控制了这个 WTO 的规则的话,对你自己的这个贸易的那个进出口平衡有很大很大的好处。然后更不要说什么人权法庭,国际人权法庭在海牙,那个就在欧盟里面,都是欧盟已经拿在手里的,对,只要把美国踢出去,而且事实上中国俄国跟第三世界都很乐意把美国踢出去。你看在 w t o 现在不是因为美国威胁的要退群,整个瘫痪。那你现在如果欧盟利用这个机会转过来跟中国俄国跟第三世界重新修订一点点规则,就是我的重点就是说,一年前这个战略很简单,而且他们的确有些智库已经讨论了,而且写下来了。



王孟源 25:07 

那写下来之后当然是美国人看得很不爽,对不对?哈哈哈,所以我认为美国从去年开始积极的挑动这个俄乌战争,其实原本就是要他们两败俱伤,俄国跟...不是乌克兰,乌克兰没人在乎,俄国跟欧盟两败俱伤。那你如果回头去看,当时他们那个负责乌克兰方向的这个neocon,就是新保守主义者,就是他们的国务次卿Victoria Nuland所讲的话。他们的确是在去年下半就已经在积极筹备这件事情,而且去年下半还发生了一件事,我不晓得你记不记得,就是他们从阿富汗撤军了。这件事的蹊跷不在于这个从阿富汗撤军对不对?



王孟源 26:08 

因为你事实上在阿富汗那边就浪费财力、物力跟生命,对不对? 20 年的时间,这个对国家的好处是什么?我这个不是我要跟大家讨论的事情,真正的重点在于美国的政策决定不是根据国家利益来做的,而是根据大财团利益来做的。那控制国家利益的三个最大的财团是什么呢?第一个是军工财团,第二个是能源财团,第三个是金融财团。



王孟源 26:45 

那你从这个阿富汗的战争打了 20 年。为什么打了 20 年?其实早就知道不可能打赢了。但是为什么硬是每年几千亿、几千亿的这样花下去?花了他们现在估计是8万亿,总共 20 年花了8万亿。这个你为什么会拖这么久?就是因为军工集团不让他撤。所以你想想看,如果 20 年来都不撤,是因为军工集团不让他撤。那拜登敢撤军,然后能够安抚军工集团。这是什么意思?



唐湘龙 27:29 

再开一个战场。



王孟源 27:32 

就是他们已经跟军工集团讲了,我们会开一个更大的。



唐湘龙 27:36 

没错。不要担心,这个战场我们关了,我再开个新的,买个新玩具的一个给你。



王孟源 27:42 

对,而且事实上这不只是因为你会在乌克兰有战争,所以可以从阿富汗撤军,你必须要撤军,为什么呢?因为阿富汗跟俄国接壤,而且俄国跟这个阿富汗的这个塔利班,交情不错。你到时候如果没有从阿富汗撤军,结果乌克兰打起来,俄国真的派人,他们几年前就已经开始撒谎说俄国有那个悬赏美军的人头,他们怕如果他们既然已经造过谣了,怕俄国真的就去那样做,那这个他们的颜面上就不好看。



王孟源 28:25 

所以这个你光从阿富汗撤军的那个时间,然后再接着大概一两个月之内那段时段,就是去年年中大概还不到一年前,那个时段Victoria Nuland所讲的话,你就可以看出他们其实这个事从那个时候就已经计划好了。那这个计划好了之后,实际的执行细节我在 3 个月前已经, 2 个月前、 3 个月前已经跟你聊过了,所以我不再仔细谈,现在纯粹谈这个经济层面。这个经济层面的问题在于你这个neocon它的特点就是不管三七二十一,反正先到全世界点火。那他们用的这些专家就是当国务次卿或者是助理国防部长这些人,他们的背景往往都是跟你这些目标国家有深仇大恨。比如说Victoria Nuland,他的祖父跟父亲是 100 年前被俄国从乌克兰踢出来的犹太人。那他的这个当时他们的家产被没收了,所以这个是有世仇的。



王孟源 29:47 

我曾经开过玩笑说这个在美国反对中国政府的人都是有仇有瘾有病,或者是,哈哈哈哈,有仇有瘾有病,或者是忘了一个,年纪大了,那这里就是典型的有仇。那你现在加拿大为什么现在对这个俄国也是跳的这么欢?你加拿大,大家印象中是昂撒集团里面一个比较爱好和平,就是有点像。



唐湘龙 30:22 

相对温和的。



王孟源 30:24 

New Zealand。就是刚好因为他的这个外交部长Chrystia Freeland。他是天主教的。他的爸爸也是乌克兰人。



唐湘龙 30:35 

OK。



王孟源 30:36 

而且他爸爸就是乌克兰的纳粹。我们现在在讲乌克兰从二战的时候不是有一个纳粹政权杀了多少多少人。



唐湘龙 30:42 

没错,现在被当成是乌克兰的英雄一样。



王孟源 30:50 

对,他的爸爸就是那一个党里面的管文书的。是当时乌克兰纳粹报纸的一个编辑。一般人不知道这段历史,就是当年乌克兰的那些纳粹,甚至成立了一个党卫军的师,那个党卫军的师跟那些白领的、就是做宣传的那些人,在二战结束前就是4月的时候、5月的时候退到什么地方?退到奥地利维也纳。所以就是有大概 2 万多个乌克兰纳粹退到维也纳,因为维也纳是靠 西欧 的,所以他们是向英国人投降,而不是向俄国人投降。那因为英国那个时候就已经非常的反苏,所以这些乌克兰纳粹就被统统运到加拿大去了。



王孟源 31:48 

所以你现在加拿大特别反俄,尤其对乌克兰这个这件事情特别跳得很高,非常之xxxx,原因就在于这些乌克兰移民在加拿大它很有地位,已经很有地位了,甚至当上了外交部长。就好像美国一直要制裁古巴,是因为有很多当年卡斯楚革命的时候逃出来的古巴难民,他们是反卡斯楚的,所以不容许跟他那个和解,那这个加拿大对俄国的外交政策尤其是有关,也是因为他们内部有大票的乌克兰移民,所以不可能。那我这个是想跟大家讲一下这个,这些实际的,这些人能够掌权并不是光凭着他们的仇恨,而是因为他们对财阀有用,那对财阀有用。这个neocon背后有 4 个主要的支持,其中 3 个是我刚刚提到的那三种大财团,一个军工,一个能源跟金融,最后一个是政要集团,这个政要集团形成一个在国外搜刮的体系,是从克林顿开始的,是一个相对新的现象。然后政党轮换以后Dick Cheney上来。才正式的把这个neocon引进到政府里面去,在 Clinton 的政府手下的时候,他们叫做 Neolib,就是这个 Neolib 的差别就在于他们是用和平的手段,就表面上和平的手段去掠夺全球,但是neocon认为你可以用军事手段去掠夺全球,这是他们的主要差别。



王孟源 33:44 

那neocon的圣经是 1996 年一个美国的犹太人为以色列写的,这个圣经的主旨是什么?就是你不用管以前那种外交战略、远交进攻的东西,反正就是不管三七二十一,不管是朋友或者是路人或者是仇敌,一律搞乱。那搞乱了以后你就可以去浑水摸鱼。那你可以看到我为什么特别提这一点,就是Bush政权轮替之后,欧巴马上来,照理是说他竞选的时候是说要change,要改变,但是那些neocon他照单全收了,不是奥巴马照单全收,而是 Hillary 照单全收。 Hillary 不是当了国务院国务卿,那个国务卿吗?对,他把这些neocon从共和党的政权里面转到民主党国务院里面去。 Victoria Nuland 就是那个时候转进去的。然后我们现在看到的这些neocon反而是在民主党系里面比较多。就是拜登,等到那个等到 Hillary 退休之后, Hillary 是在 2012 年还是 2013 年退休之后,这些neocon就由 拜登接收了。为什么呢?因为 拜登也想成为这些政要,他看着 Clinton 那样的很眼红,尤其你看他的儿子,他的儿子是一个纨绔子弟。



唐湘龙 35:22 

是,hunter Biden。



王孟源 35:24 

hunter Biden,他非常喜欢赚这个,为什么呢?原因是美国是一个资本主义社会,那资本主义社会的第一原则、最高原则是资本的利益至上,所以如果资本的利益至上的话,你就必须要用遴选的这样子资本才能够控制政府,因为唯一能够挑战资本的势力,你如果有自由市场的话,唯一能够挑战资本的势力就是国家机器。所以你必须要控制政权。那控制政权最简单的办法就是由愚民来投票。那这样的这个愚民投票选择的当然是靠媒体宣传,谁能控制媒体当然就是资本了,对不对?那这个资本,美国的这个资本主义,因为资本凌驾于政府之上,所以政客其实是为资本打工的。



王孟源 36:25 

你看他们现在这个所谓的游说业,为什么是把贪腐正规化?收买政客正规化、法制化,就是为了方便资本来收买政客,为了服务着,因为这样子不但免除了他们被追究法律责任的危险,而且能够靠着政客之间彼此的竞争、竞价,压低他们所买的价格。所以这些政要在国内替这些资本做事的时候,拿的那些钱其实都是小钱,你辛苦了两三年,拿到多少?几十万。但是你一旦到了国外,美国的政治就凌驾在外国资本之上,对不对?你美国的资本凌驾在美国的政治之上,但是美国的政治当然凌驾在外国的资本之上,这时候就可以好好的敲诈。你看那个hunter Biden或者是那个Bob Dole,他们只要挂个名去当那个董事的话,每年一个小时就可以拿到几百万。



唐湘龙 37:40 

好,那孟源刚提到的就是说,在这个梅克尔之后,默克尔之后,欧盟似乎已经失去了他的那个战略的机遇期。你刚提到美国今年开了乌克兰的战场,这个战场是需要美国之所以持续的需要战争,就是因为刚刚孟源提到的后面的这些的真正在主导美国政治的力量,这些的军工团体,这些的这些金融,这些的能源,乌克兰是一个刚好复杂的能够把这些都摆进去的,能够满足大家需要。同时把俄罗斯跟欧盟就是二桃杀三士。



王孟源 38:24 

最重要的不是俄国,其实他们宁可是拿中国当目标的,但是这里的重要考虑,选择乌克兰的重要考虑是欧盟,因为只有乌克兰能把欧盟扯进去。你如果是在台湾扯事的话,欧盟拍拍手说好。你们去玩。



唐湘龙 38:44 

当然欧盟就没事了,所以他将来如果要搞中国一定是在台湾身上动手。



王孟源 38:51 

对,所以这个之所以选择乌克兰是考虑到欧盟的关系,而不是因为他们非要整俄,优先整俄。那你看过去这 4 个月,他们这些利益集团拿到的利益是什么?这个 400 亿基本上是交给军工集团了,这个紧急预算,然后这还不包括他们资助乌克兰政府,乌克兰的政府现在已经基本没有税收了,他们这个全部靠欧美来援助。他们原本开口是每个月 50 亿,现在已经上个月已经提高到每个月九十亿了。那这些钱都是去那边转一圈以后马上又拿回来分成的。然后你比如说捷克斯拉夫刚刚通过要买 24 架 F 35,美国自己通过要买 300 亿的 F 35,这些都是因为这个战争。



唐湘龙 39:56 

所以真的多好。



王孟源 39:58 

国会就不会去审查这些事情。



唐湘龙 40:01 

所以军火商的订单都是天上掉下,只要你敢打仗,订单就掉下来。



王孟源 40:07 

对,能源的话我不多说了,大家都知道现在这个天然气能源价格涨到哪里去,而美国是一个能源出口国,现在美国的能源价格涨成这样,为什么?因为他们把这个欧盟的能源价格涨上天了。



唐湘龙 40:22 

当然卖到欧洲赚比较多。



王孟源 40:25 

他把它全部拿去卖到欧洲。



唐湘龙 40:26 

所以美国的能源当然就贵。对对。



王孟源 40:31 

那这个也是同样几百亿几千亿的。那大家还没注意到的是金融。两个月前美国最大的对冲基金Bridgewater看出来。做他们的年度报告的时候简单的提到了,说我们现在正在做空欧盟。



唐湘龙 40:53 

没错,哈哈哈,他一直在放空。对。



王孟源 41:00 

对,就是美国去搞这个乌克兰战争,意思是要让俄国跟欧盟两败俱伤。他事先没办法预期哪一边会伤的比较严重。当然他原本你事后可以看出来,他原本是预期俄国会伤的比较严重。



唐湘龙 41:15 

对,但没想到伤最重的是欧盟。



王孟源 41:19 

他不在乎,因为欧盟可以掠夺财富反而比俄国更多。



唐湘龙 41:23 

当然爽,爽翻了。



王孟源 41:25 

爽的,真的是赚翻了。你现在这个,现在欧盟的经济即将崩溃,他这样一路的短线操作,这样做下去,做 short 就是卖空,这个赚了几百亿只不过是开胃菜,你等到欧盟在未来两年三年整个彻底崩溃的时候,他进去抄底的时候,那个时候赚的才真的是 20 倍、 30 倍的赚。



唐湘龙 41:53 

好,我们看一下我们秀出来这张图,让你看到,因为欧元,当然欧元并不是全部,但是欧元可以让你看得出来,就欧盟区现在的总体的情况。那欧元兑美元的价格最高的时候出现在 2008 年,就是华尔街金融风暴的时候,那个时候欧盟的欧元成为大家的避险货币,所以欧元对美元来到 1. 599,将近 1. 6: 1。好,但是你现在都已经要破 1 了,而且它是从 2008 年初开始,长线的一波比一波低的在往下走,从欧元的角度来讲,当欧元不好的时候,欧盟就意味着欧盟已经完了吗?你为什么会这么悲观?



王孟源 42:38 

这个你观察欧盟经济前途的重点有两个国家,首先是德国,德国当然是欧盟的经济核心。那事实上整个欧盟的这个架构是什么呢?拿着德国还有北欧几个小国,荷兰这几个类似德国的经济的,但是他们都很小,他们跟德国比起来,他们加起来还没有德国的分量。那这些是所谓的强势工业国,拿着这些强势工业国的钱去收买南欧跟东欧的这些落后弱势国家,然后要求他们做政治的退让,这是基本上整个欧盟内部治理内部的一个基本原则、基本方法。



王孟源 43:26 

一个基本架构。那你如果德国出了问题,这个利益的传送链就断,整个欧盟的这个根基就垮,那这垮不是说马上就垮,这只是他一个长期的问题,这是一个长期的问题。那我们还要看德国这一次因为他们制裁俄国,所以他本身的通货膨胀。我想我上个月提过,就是德国上个月的 PPI 是33\%,这个月的 PPI 稍微降了一点,是32\%,但是这个都是你通货膨胀达到 30\% 几,这个都是史诗级别的东西。对它的工业,尤其是化工,你一旦这个化工垮了以后,你汽车的那个油漆,还有其他的那个化学材料,塑料什么的,你都不用谈,对不对?



唐湘龙 44:29 

PPI 是原物料价格了,基本上就是躉售物价、大宗物价。



王孟源 44:36 

对工业品的通货膨胀率,就是工业材料的通货膨胀率。



唐湘龙 44:42 

就可以想见他现在的这些的制造业的生产成本被垫高到怎么样的地步。



王孟源 44:47 

对,而且这还是在俄国根本没有出手的,是不是?那俄国之所以没有出手,刚好就是因为两个原因。第一个,他原本定的就是长期战略。因为他怕美国,他如果全力投入到乌克兰的话,他怕美国跟北约偷袭他,所以他才只投入了全部军力的 1/ 5 还不到。那你如果投入这个军力,事实上跟乌克兰的军力比是 1: 2,你原本照理说你进攻方应该是 3: 1,你变成 1: 2 的话,那么这么一来你就不可能速战速决。不可能速战速决,所以俄国原定的战略就是要长期消耗。



王孟源 45:33 

那这是经过 20 多年准备,尤其是 2014 年之后,过去这 8 年是积极的在准备,卧薪尝胆的才做得到。那正因为俄国是准备要长期战略,所以他没有必要马上就掐断这个德国的咽喉。我想他事先也没有想到欧盟跟德国的领导人是如此的愚蠢。就是很多这个制裁都是做得非常过分,你照理说是应该自损800,杀敌1000,结果变成是自损1000。



唐湘龙 46:10 

杀敌800。可能还还不到八百,大概两百。



王孟源 46:17 

但是他们也没有想到欧盟的那个领导人会是这么的愚蠢。那另外一个原因就是俄国其实在乌克兰打得很顺,完全就是头一个半月是失误,就是他以战逼和失误,但是进入第二阶段以后就打得非常的顺,就是他的那个伤亡率一下子就掉到很低。



唐湘龙 46:43 

没有错,到最近都还是一样,但是因为大家已经都不太关心战争了。



王孟源 46:49 

对,基本上它可以无限的持续下去,那既然打得很顺利的话就没有必要下死手。反而是如果你把乌克兰,事实上北约再怎么送也不可能把乌克兰军起死回生,因为他的那个军官和士官已经死的太多了。但是我是说假设的话,假设如果这就是一个很反讽的事情,如果北约真的把乌克兰军起死回生的话,反而是俄国有动力下死手,把那个德国的经济当场掐死。现在像今天他们那个北溪一号。



唐湘龙 47:30 

北溪一号的天然气,要恢复供气。



王孟源 47:33 

这其实都是很正常的。他只是跟他像那个猫玩弄老鼠一样这样吊着玩一下,并没有真的要把它宰掉。



唐湘龙 47:42 

对,因为今天的恢复通气对德国来讲真是松了一口气,要不然他连这个冬天都不知道怎么过。但是其实你正常合理的就是,就像王孟源刚的分析一样,你只要稍微正常理性的去判断,俄罗斯不会去掐断北溪 1 号的。



王孟源 48:03 

因为他如果要掐断的话早就掐读错,不需要等到现在,哈哈哈。像他这个基本上是因为那个西门子的那个涡轮机的事情很怄气,所以要利用这个机会教训一下德国的那些政客,不论如何,我们现在一个未定的因素是今年冬天德国的工业会有多惨。我说的不是消费者,你消费者这些所谓的先进国家每年冻死一两万人,这是正常的事情,这个你到了博客去看,就可以看到英国每年平均冻死 7000 人。



唐湘龙 48:41 

不可思议。



王孟源 48:42 

每年冻死 7000 人,这个是过去十年的平均。不是某拿一个特别寒冷的冬天,这些所谓的先进国家,其实对他们的这个弱势群体非常的冷血。这个反正你已经有投票权了,你还在乎生命权干什么,对不对?所以德国今年冬天我们必须要观察的是现在这个 30\% 几的 PPI 对它的工业有很大的杀伤力,像 basf 德国最大的化工厂,就已经公开的在谈要把它的产能转移到中国去了,这个转移过去以后就不会回来了,对不对?所以这是我们看欧盟问题的第一个重点,第二个重点是意大利。



王孟源 49:34 

为什么?因为德国即使是整个垮掉,经济垮掉了以后,欧盟只是会长期的不断萎缩下去,要整个短期内就崩溃的话,还必须要有一个导火线,那这个导火线呢?光是希腊是不够的,我们在希腊,在十几年前已经经过一次的,就是希腊的这个额份,它的 GDP 只占欧盟的3\%,这是不够的。这个意大利的 GDP 比希腊大四五倍,这个就,如果意大利垮掉,就足以把整个欧盟或者至少欧元拖垮。那为什么今天 Draghi又赶快去辞职?因为他是经济专家,他是前任。



唐湘龙 50:26 

的欧洲央行总裁,他是前任欧洲央行。



王孟源 50:32 

他知道这个是没救了,他不想,哈哈哈哈哈哈,你这个美国的那个俚语是说when shift hit the fan 就是你天花板上不是有一个那个旋转风扇?如果你知道大便会掉到那个风扇的时候。



唐湘龙 50:50 

那就完了。



王孟源 50:52 

你最好是不要在这个房间里。这就是Draghi为什么要赶快辞职的原因。那意大利的财政跟金融问题之严重不是一天两天的事情,从希腊出问题的时候,意大利就一直是一个很大的隐忧,然后到了 3 年前,他们又有了几个银行出了问题,几个主要银行出了问题,那这一次的新冠其实是挽救了意大利,因为欧盟一下子大开钱包来做财政补助,反而是挽救了意大利。



王孟源 51:38 

那所以这一次的这个转折,问题就在于,你一下子如果德国没有钱了,欧元又被中国、俄国第三世界抛售,没有人拿你的这个印出来的钞票,那这个时候你的汇率继续跌下去,你又不能够继续印,否则的话它跌得更快。那这个命运就是原本一年前我认为会发生在美国的,现在严重的威胁到欧元区了。那这个如果出现在欧元区的话,第一个可能倒下去的可能还是希腊,但是第一个足以把整个欧元区拖垮的是意大利,所以我们必须要仔细的看意大利的这个经济跟财政跟金融的发展。



王孟源 52:33 

你在一开始的时候说,我对欧盟的经济很悲观,悲观就在这里。因为它的强势的支柱已经被拆掉了,它的这个弱的窗口正在被撕开,所以你这两件事同时发生,而且一直到现在他们还不知道悔改,还要再,还再打算要做第七轮、第八轮的制裁。这个真的是,你真的是很难想象这个国家级的领导人会蠢到这个地步。但是我们就活在一个很有趣的时代,就是可以观察这些很有趣的现象。就是你一个普通的公务员在家里面看我们的节目,或者看我的博客,可以很简单的笑着说这个国家或者整个欧洲的领导人,怎么都是如此的蠢蛋,这个历史上很少有这样的时期,尤其在现代。但是我们现在的确是在这样的一个时间点,所以是非常的有意思的。那你看现在美国的经济是怎么样?就是靠着从欧洲输血,你看现在,我刚刚提到这个对冲基金正在做空欧元来赚取利润。所以你像现在美国的高科技还有银行,他们虽然是缩减雇员,但是还没有裁员,就是说他现在已经两个季度它的那个成长是负成长,照理说这应该已经就是个经济衰退。不过我必须要特别讲一下,这个两个季度负成长就是经济衰退,只是一个非正式的。



唐湘龙 54:26 

没错,经验,经验上面是这样子认,经验值。



王孟源 54:28 

这个至少在美国你是不是有经济衰退,它有一个专门的、国家设立的非营利组织来事后确定,就是隔了一个季度以后,他会说我们经过一个衰退,或者我们没有经过一个衰退,那它的这个论断可能跟你的这个两个季度的 GDP 负成长不太一样。那我现在认为第二季度的数字还没有出来,但是我认为可能是负成长。那就是用这种传统的定义的话,会是已经有一个衰退,但是我怀疑这个组织会说没有衰退。为什么呢?因为它的这个失业率没有上去。



唐湘龙 55:15 

没错。现在还没有。



王孟源 55:15 

这失业率没有上去。因为你通常遇到一个衰退之后,这个影响消费最严重的失业是什么呢?就是银行跟高科技公司马上裁员。这些都是高消费人群,因为他们的薪水都是最高的,他们一旦失业以后就影响的总消费金额最快、最大。但是你如果去看现在美国主要银行,他们所宣布的基本上是,甚至没有 hiring freeze 就是冻结雇用就只是减缓,减少雇用。



唐湘龙 55:53 

减少新聘人员。



王孟源 55:56 

推缓雇用,而没有,那更不用提裁员。所以美国这个是凭什么?这个高科技是靠着跟影子银行借拿钱,就是他们继续拿投资,那影子银行跟银行界的钱哪里来,从欧盟那边拿来的,从欧元区那边你做空再赚你们一些钱。



唐湘龙 56:19 

好。最后一个问题就是说,因为当如果你对于欧盟、对欧元的未来是悲观的,那我们刚也提到就是说其实现在的全球的金融体系里面,除了美国以外,大家都在寻找一个可以去美元或者削弱美元的替代强势货币,那这是欧盟区出现了之后,其实全世界的一盏明灯,觉得终于找到一个经济 base 也够大,那同时也先进,大家的认可度也高。再加上有伦敦的金融中心等等的这些的背景。



唐湘龙 56:57 

可是当欧元,欧元现在在长线下滑,它的占比不但没有提高,现在反而是美元更独大了。你看到美国现在的美元的指数一直都撑在 107、108 以上,欧元一直节节下滑,在这种情况下面,你刚提到比如说俄罗斯,比如说中国,甚至于金砖国家好了,天下苦美久矣。大家最受不了的就是美国用它的在金融当中的独特地位,恶意的去操作全球,把所有的它所产生的这烂摊子,那都用这种以邻为壑的方式,让大家去承担它的经济衰退,承担它的通膨。但是大家想要让欧元起来,可是欧元起不来了之后,是不是意味着这种的寄望?像包括人民币也好,包括了卢布区也好,金砖国家也好,他们就找不到其他的替代货币了?



王孟源 57:57 

其实那个那场战2月底一开打,我第一时间就写了一篇文章说,现在是第三世界搞自己的货币的时间了。就是一个篮子货币,就是。然后你在过去这四五个月,很明显的,是中国跟俄国跟沙乌地阿拉伯跟印度,他们都有这个意图。所以现在看来这个架构会是基于金砖国家。这也是为什么伊朗跟阿根廷都急着。



唐湘龙 58:32 

要加入金砖。



王孟源 58:34 

加入的,因为他们如果没有这个集团来为他们撑的话。这些国家刚好就是在这一波世界经济衰退的浪潮中,很可能要被美国收割的最惨的,就是跟欧盟一样的惨的国家。现在已经破产的国家是什么呢?斯里兰卡,黎巴嫩,Suriname是南美的一个国家,Zambia非洲的一个国家,Ukraine乌克兰,这 5 个国家是已经破产的。然后接下去还有大概 10 个国家是即将破产,阿根廷,Ecuador、El Salvador、巴基斯坦 Tunisia、Ghana、Egypt、Kenya、 Ethiopia 跟Nigeria,这 10 个国家是如果这个经济继续这样下去,然后美元继续的从国外吸血的话,在未来一两年有很高的破产的机率。就是我说破产,当然国际法里面没有破产这个词,就是他们付不出、还不起外债。



唐湘龙 59:48 

债务要重整,国家债务要重整。



王孟源 59:51 

对,债务要重整。在这个欧美的西方体系下,这个你在国际上国家破产的话,是由 imf 国际货币基金会来拯救你。这个你可以从 1997 年他们去拯救韩国的那个过程,就可以看出这是怎么回事。



唐湘龙 01:00:09 

他现在在处理斯里兰卡,就是这样。



王孟源 01:00:13 

对啊,你把三星拿出来卖给美国的财团,我们才把钱给你,哈哈哈,就是这样,三星现在是主要的股东都是美国的财团。



唐湘龙 01:00:24 

80\% 的普通股都在美国手里。



王孟源 01:00:28 

对,你基本上是美国对外掠夺,就是我刚刚讲它原本这个neocon要掠夺的最后一步,为金融集团做服务的那一步。那所以这些国家他们都急得要加入金砖,为什么?他们指望的就是那个金砖货币。因为这个金砖货币必然会带来一个国际破产机制,一个新的国际破产机制,那这个破产机制或许会要求他们背上债务长期偿还,但是至少不需要把民族工业,他们的这个国家的瑰宝拿出来贱卖。因为当初韩国卖那些资产的时候,是以 1/ 20 的实际价格卖给美国财团的,你基本上一年之内就赚了 20 倍。那这种事情就是这些国家急着要避免,所以我认为这个新的货币应该会搞得成。这问题是在今年年底还是明年年初宣布?我们还必须要观察。



唐湘龙 01:01:38 

好,我们把节目再延长 10 分钟,因为我觉得这段很重要,就是刚刚讲的,就是因为现在全世界其实任何你如果有做货币避险的,大概现在都非常的困扰,因为美元已经非常高了。那你说这个时候去找美元避险,好像最好的时间点也过了。那你说找日元,日元现在是稀里哗啦剩一半了。那如果你找欧元现在刚刚破底,那人民币因为它的自由度是不够的,所以大家都会想说糟糕了。



唐湘龙 01:02:17 

那现在所有的你数的出来的,在银行挂牌的强势货币里面,你还真的没有什么是你觉得你可以去换了,然后摆在口袋里面会安心的。那在这种情况下面,金砖货币你为什么会看好?因为它很复杂。这些国家其实除了中国之外,其他的经济实力并不强。我说只有中国算是金砖,其他的连镀金都谈不上。那在这种情况下面。



王孟源 01:02:44 

俄国也是,俄国的财政跟贸易已经顺差了 20 年。



唐湘龙 01:02:49 

对,好,那在这种的情况下面,人民币可以扛,如果金砖币能够成立的话,人民币当在底下、当它的重要的地基是绝对的。那人民币撑得起来吗?



王孟源 01:03:04 

撑得起来。而且我刚刚讨论了,论证了这么久,重点就在于全世界对一个新的货币的需、国际货币的需求是如此的强烈,如果在两年之内他所占的国际储备的额份没有到20\%,我会非常的意外。就是取代欧元成为世界第二大储备货币。



唐湘龙 01:03:28 

两年内。



王孟源 01:03:30 

两年之内,就是事实上,20\%事实上是说的太客气,哈哈哈,我是希望他能够成长到 40\% 或者 50 的。



唐湘龙 01:03:44 

那如果照这样比例的话,那欧元就不见了。



王孟源 01:03:49 

对,欧元不见了,这个我认为 99\% 的机率欧元没有人用、没有人会用。因为你当初没收这个俄国外汇储备,或者甚至俄国人的私人资产的时候,欧元抢在前面。而 Swift 本身就是我刚刚解释过欧元的贸易通道。现在大家都不用、不想用Swift,对不对?因为你出了问题,第一件事情人家就是把你踢出Swift,那你等于是自己自愿把一个绳子绑在脖子上,哪有国家愿意这样?这里最大的关键,金砖这个集团能不能搞得起来,我认为很乐观的, 90\% 以上可以,剩下的这个 10\% 的问题在哪里?我认为是出在Saudi的问题上,因为Saudi真的是能源供应举足轻重的一个国家。那这个金砖货币不只是靠人民币来做基石,也是靠xxxx能源。因为你如果有俄国跟阿拉伯联合大公国,再加上这个沙乌地的话,基本上,而且原本就有委内瑞拉跟伊朗,然后巴西跟那个阿根廷负责提供农产品,这些都是资源出口大国。他们或许他们的贸易逆差很糟糕,财政逆差很糟糕,但他们都是资源大国。那你在一个分崩瓦解、全球化逆行的环境下,资源就是硬货币。那这个,所以绝对不是只有人民币能够撑底。中国贡献的是一个工业产能,就是这些资源供应国,他们还是需要先进工业,还有那个工业能力,还有消费能力、市场,对不对?中国跟他们是互补的关系,但是绝对不是单凭中国就能搞得起来。



王孟源 01:05:50 

所以这件事情是互利的一件事情。那其中最关键的就是沙乌地阿拉伯,因为沙乌地就决定了这个能源的供应是掌握在这些革命党的手里,我认为这些基本上是第三世界在革,昂萨政府、昂萨集团的命,这个革命。因为昂萨集团事实上是世界政府,我说没有正式的世界政府,但是在过去 200 年是实际的世界政府。那事实上这个这一次,是一个第三世界由俄国开了第一枪的一个革命,那这个革命最重要的重点就是中国已经加入了,金砖国家已经加入了,现在重点是能不能争取到沙乌地阿拉伯,你争取到沙乌地阿拉伯,你就控制了世界的能源供应。你这个革命就一定搞得成。那沙乌地阿拉伯,他在政治、外交、经济、贸易、科技、工业上面都绝对愿意加入。他唯一的考虑是什么?美国的驻军,对不对?所以这件事情就是看美国人愿不愿意拿驻军这件事情来做筹码,威胁要搞政变或者之类的东西。



唐湘龙 01:07:13 

好,当然刚刚的那个分析简单讲就是说,将来的这样的一个新的体系的整合,可能是一个 OPEC Minus,再加上金砖plus。我说的 OPEC Minus,因为现在大家都在谈 OPEC plus,但是 OPEC plus 很大,但是OPEC Minus是OPEC里边的核心,刚讲的沙乌地,然后阿联酋,然后俄罗斯,然后如果再加上伊朗跟委内瑞拉其实就够了。其实老老实实讲,在所有的OPEC系统里面来讲,最主要的就是这几个。那如果金砖plus、金砖再加上了几个国家了之后,它确实是一个非常庞大的就是资源跟一个生产体系的整合。



王孟源 01:08:03 

对,然后他们这方面这些革命党只要等上三五年,让欧盟跟日本自然衰退、自爆、所谓的implode,英国也是要自爆。你最后就只剩下北美洲,因为北美洲是有很多能源,北美洲跟澳洲是有很多资源的,他们可以自给自足,他们也有那个科技,所以到时候,这个东西搞成以后,三五年之后等着欧洲、英国跟日本自爆,自爆之后就变成亚非拉对抗北美,那你想想看这个实力的悬殊的,这是打都不用打。所以我预期只要能够到那个地步,美国人就会认输。



唐湘龙 01:08:52 

好,今天时间的关系跟王孟源今天还不错。今天王孟源表现不错。我要谈的东西大概都有谈到,因为我怕王孟源有很多东西都没有讲到。好了,我先感谢一些一些观众朋友的支持,从这个坡论开始。坡 t s e。好,那的他用了。龙哥早询问英国行政法院就蔡英文博士论文对儿子一发出违抗行政命令正式通知,有用吗?会,我的简单讲就是说会有用,因为英国的这些学术机构是守法的,那我们就是看他会不会把那个论文拿出来让大家接受。



唐湘龙 01:09:37 

就是说以英国所代表的,就像是刚刚王孟源提到的,他们基本上代表了当下的国际管理的一套系统,包括学术圈子里面,那它能够发生什么作用?好,再来彭论,谢谢。好在一日不做。那台湾岛如果不在中国边上,你们试一下五眼联盟会多看你们一眼吗?好, Grace 力。



唐湘龙 01:09:58 

好的,非常的谢谢 Grace 力就留个图感谢来路易斯良来宾。讲解的真的很深刻。那我的对没有错,他说上星期没赶上直播,然后他看回放,这一次赶上了,必须要懂内,感谢一下,谢谢。那 w m 秒秒这在新加坡,他说这一期龙哥又请到了王博士,很惊喜。那谢谢两位带来精彩的节目,他上一席他听了两遍,好好继续。这,这是对吧?我坦白讲,我即使很专注的,你会知道感觉我很专注的在听我王孟源讲话,但是听完的时候我往需要回头再去完整的听一遍,好,再来宇宙真理,好,再来日本。感谢。



唐湘龙 01:10:42 

然后我是你哥哥在美国,他说麻烦唐先生问一下,为什么蔡琴的那一期不见了,因为因为有一些版权这种方面的问题。我,他倒不是不见了,而是蔡琴的那一期我们就把它先锁住了。那反正家居要办演唱会什么的,你可以留意一下 Candy Chan, Candy chain,他说今天超级战,这在肯定王孟源,谢谢苏珊s,这在美国听君一席话胜读十年书。他说我这个淡水经济学派的门徒受教了,每次听都收获满满。好,这个,这不算是对王孟源的意美,这也是为什么呢?我觉得只要王孟源愿意,我就尽量把时间都开给他。再来凯瑞迪在香港,他说王坚你好,希望能够多上节目。他说我是在上海的成都人,我有一个红旗下长大的愤青知识分子父亲,那滑稽的事,我从小愤青到小粉红全是因为受到几个台湾人的影响,一个是王牧原,还有一个叫薛仁。这薛仁明我不认识。好,那香龙跟凤心的节目比我党的宣传文章还要好。对,我也觉得好。那谢谢你们,虽然我们都是底层的,逻辑可能不一样,但我们都是中华文明的子孙。



唐湘龙 01:12:05 

好,李道阳,谢谢旅游,谢谢。在美国那语出惊人。谢谢。语出惊人说梅克尔没有把冯德莱恩踢到联合国是总理生以来的一大败笔。对,我觉得冯德莱恩,欧盟现在就是没有像样的大人,这个很麻烦,就他没有leadership。那你说那少了一个梅克尔,少了一个梅克尔真的差很多,因为梅克尔有那种在精神上面给稳盘的效果,它有leadership,那现在欧洲大家都想跑,你想连德拉吉都想跑,因为没有大人,你会觉得真的出事情的时候你没有帮手,所以大家敢宁可选择最糟糕的情况就是就摆着烂吧,撒手不管,烂到底的时候大家再来想想办法。



王孟源 01:12:51 

意大利总理的分量不够。就是Draghi知道他自己孤掌难鸣,就算联合法国的Macron都不够。因为这个事情必须要德国总理站出来才有可能。



唐湘龙 01:13:02 

没有错。但是现在的德国的总理其实本身的威望是很差,而且我发现他没有什么国际观,对国际议题几乎讲不出什么什么有关系,



王孟源 01:13:10

他是律师出身,



唐湘龙 01:13:14

对,或者会让人觉得说,唉。他这次讲得很好,我们可以试试看。就是你会发现肖芝跟梅克尔真是天差地别的两个人。



王孟源 01:13:22

对



唐湘龙 01:13:02

好,再来张澜城,感谢一日不做谢谢,然后就能配感谢孙泽敏在日本,谢谢好时间的关系,今天还是一样非常的感谢王孟源。那什么时候能看到我?王孟源,下个礼拜,下个礼拜我还是给王孟源,下个礼拜我们就把财经的问题放一边,最近的国际的总体经济的形势的变化是很大。



唐湘龙 01:13:46 

还有几位来威廉姆one,感谢这在澳洲,然后力旺,这在美国,他说王博士,请问你对于中美两国货币汇率的未来好待会我请教一下再来呢? l v a Chan,这在呢香港,它是非常有营养的内容。好,我请问孟源刚刚这个 力旺问的就是你对中美两国货币汇率的未来的走势,你怎么看?就人民币对美元,你认为人民币会升还是会贬?



王孟源 01:14:17 

我觉得人民币,在这个金砖货币出来之前人民币基本上就是以稳定为准,所以不会大涨大跌。基本上他还是盯着那个篮子货币,就是美元加欧元,那因为美元相对欧元大幅升值了,所以人民币相对美元也贬值了一点。但是我想当前短期的策略就是以稳定为主、以维稳为主。



唐湘龙 01:14:49 

我的判断是到最少到今年底大概 12 月左右,我认为人民币会有一波的急拉的走势,这跟他的就经济成长率的计算是有关。



王孟源 01:15:02 

那是下一个阶段就是金砖货币要出来的时候,这些货币都会获得大量的支持,因为他们会忽然占领很多国际储备货币的额份。



唐湘龙 01:15:13 

好,来,最后又多了两位皮特和感谢这代纽西兰,好一阵子没看到纽西兰的朋友再来 c y c s 旺,他说感谢了孟源先生深入精辟的讲解,还有过人的洞察力。好,那时间到了,其实已经超过了。感谢人在美国那很热情的那参与龙行天下,能够让龙行天下的观众朋友们能够在和王孟源听王孟源说话的过程当中,能够有更多的收获跟成长。感谢孟源,感谢,好,周末快乐。来,下个礼拜见。拜。



\twocolumn[\begin{@twocolumnfalse}
\section{「习拜会」后的大国政治!}
\subsection{20220729}
\end{@twocolumnfalse}]Credit: Aaron



唐湘龙 00:00 

享受的就是反正除了要跟你讨论题目,然后要因为我大概知道你说话的那个format,他必须要有一个比较清楚的框架,跟那个跟内容跟方向性,所以要定题目真的不容易,所以我就尽量就不丢那种 1014 题去考验你,因为那种是 14 题,其实就是跟着新闻走。那个如果做对我来讲是驾轻就熟的,可是对你来讲并不适合。可是我觉得只要题目定出来了之后,我对你的谈话的内容是非常有信心的,好来准备。



王孟源 00:41 

我其实不喜欢第一时间去讨论新闻,因为第一时间出来的消息常常是错误的。



唐湘龙 00:47 

没有错,所以,所以那个的那个。



王孟源 00:53 

我做评论...今天其实最重要的消息是前工信部的部长被抓了,整个主导半导体工业的那些人被抓了一票,大概四五个人被抓起来了,这是今天最大的消息。



唐湘龙 01:08 

这个对,接下去的,因为大陆的半导体现在正在拼,所以这方面的讯息这两三个礼拜其实有关半导体市场的,中美之间的半导体的讯息量都很大,而且都非常的丰富,表示半导体业的那个这个大战正在一个最高潮的时候。好,来,我们准备开始来。好的,好,欢迎收看龙行天下,我是唐湘龙来,今天星期五接下去的一个小时的时间,我之前预告过,在我做节目,尤其在开了网络的栏目之后,从来没有过就是一个月几乎都包给一个人,他这个人叫王孟源。好,那前两个礼拜的时间,那借助于就是说孟源的他的过去的学经历跟他的专业,尤其他对一些的资讯收集之后的分析研判跟组合的能力是非常好的。那前两个礼拜我们谈了美国的总体经济当下的困境,以及因为俄乌战争所引发的欧盟跟欧元的一个重大的危机,这个对整个的国际的经济跟金融都有非常深远的影响。那前两集的内容我看到很多的留话的朋友说他不止看一遍,他看两遍,那对孟源的分析跟解读都是赞不绝口的,所以大家可以自己去找前两个礼拜的视频。好,今天我们回到在之前跟孟源设定的一些的大的一些的国际政治的框架,今天可能会主要包括了最新的习拜会,因为毕竟昨天晚上,而且中美这次的对话其实应该有很多可以解读的部分,以及牵涉到的,就是说佩洛西众议院的议长因为孟源人在美国,那美国今年底再过 3 个月就有地方的选举,那我相信在生活在美国大概也都会感觉到这样一个选举的气氛。



唐湘龙 03:07 

另外芯片战争以及俄乌。俄乌战场其实是有些变化的,这个变化我称之为俄乌战争 3. 0,就是因为乌克兰开始集结了比较多的西方所供应的装备,开始酝酿战争以来一波比较大的反攻,那它的反攻的重点从奥德萨往外头推,他现在把重点摆在了克尔松的这个地方,那这个大概跟黑海的黑海的控制权是有关系的,同时要去切断就是克里米亚的补给线。



唐湘龙 03:39 

好,那最后如果有时间的时候,我们再来看欧洲如何准备要过冬,这个冬天对欧洲来讲,天然气的问题是无解的,我所有欧洲的朋友,不管在意大利的、在德国的,在英国的,每个都叫苦连天。好,不过我们先回到呢,在我们的就是说节目的连线,来介绍在我们线上的,那今天的龙行天下的来宾也是这个小时的主讲人,接下去我就尽量不讲话了。王孟源,欢迎。



王孟源 04:07 

大家好,非常高兴再跟大家聊一聊。



唐湘龙 04:10 

来,我们的听众朋友 stand by 的,但速度非常快。来,我们先来关注一下昨天晚上,在亚洲时间是晚上了,就说 10 点钟左右,那在美国的时间是上午,拜登最近这 10 天的时间两度公开讲话说他 10 天之内要跟习近平通话,但是大陆方面来讲一直很沉默。好,那昨天晚上的通话了,这个通话的结果也大概两边都各有消息,都曝光了。我相信我们也大概都做了一些美方释放的消息,中方释放的消息的比对,你如何看?这个习拜会。



王孟源 04:49 

是这样子的,这个美国人对元首交谈、电话交谈,事后发布的笔记都是非常的不诚实的。我说的是最近这 10 年的趋势,为什么?尤其是拜登,因为拜登本人基本上老年痴呆了,他根本就没有自己主控政治决策的能力,所以这些,这些美方发表的会谈记录基本毫无价值。你过去这十几场,我去看了十几次了,那基本上我做阅读的一个工作其实是很快的浓缩里面的重点,就是事实跟逻辑的重点。那美国所公布的这些记录向来它的信息含量都是0,但是中方这一次我还没有看到他们的详细版本,只有一个简单版本,那这个简单版本本身也没有什么含量,唯一的一个有信息量的一句话是应邀来做这个会谈。所以这个从这一点你可以看出,并不是习近平想要跟,有什么话要跟拜登谈,而是美方要跟中方谈,那美方要跟中方谈,从过去这一年多拜登政权的这个案例来看,从来都不是拜登本人想要说什么,而是他的国安会主席Sullivan想要去预先擦屁股,所以叫拜登跟那个习近平打电话,打电话里面讲的都是客套话,而且都是假话,那事实上上一次就是两个月前、还是一个半月前,那一次事后你如果去看中方的记录的话,那基本上习近平是以外交辞令来直接说你们跟我撒谎撒了太多次了,我不相信你了。你如果去总结那个上一次。



唐湘龙 07:01 

上次通话是3月 18 号的时候。



王孟源 07:04 

噢噢,对,那两个多月前,三个多月前,那次记录的话,它事实上它的核心信息就是:每次你都在跟我撒谎,而且都是一回头就证明你在撒谎,所以我根本不想理你。就是,那是它里面的信息,那所以这一次基本上还是依照外交礼仪。所以既然美国总统要求要讲话,那就浪费一个小时跟他打打屁,反正事后也是转过头来就忘了。那这里面 拜登讲什么呢就毫不重要。那美方能够对中方做出什么呢?那这里就,事实上真正美方现在正在筹备在做的就是 Pelosi要到台湾访问这件事。那这个就很奇怪,你如果是照他们原先的计划,在让俄国跟乌克兰打起来之后,北约大获全胜,那个俄国的经济崩溃靠制裁。他们原本计划不是在军事上胜利,而是在军事上拖着,然后在经济上跟贸易金融上打败卢布,然后导致俄国的经济崩溃。事实上在 2008 年的时候他们就很惨, 2014 年的时候准备比较充分,但是也是有一波很严重的资金外逃的问题,但是你看这一次就是因为Nabiullina又有了八年来做准备,就是他们的中央银行。



唐湘龙 08:35 

他们的央行总裁。



王孟源 08:37 

对,又有了八年来做准备,这一次是反过来,他们那个卢布从 70: 1 升到 50: 1,现在是Nabiullina反过来想方设法要把卢布再压回 70: 1,那现在他连 60 都做不到。就是现在俄国的经济情势跟贸易情势是如此的好,财政情势是如此的好。那虽然Nabiullina拼命的想要压低货币,它现在利率已经降到 8\% 了,在开战前利率是10\%。即使这样子,但是因为做国际金融贸易的人都可以看得出来这个卢布才是硬货币,所以他说什么就是压不回去,压不回去的话,俄国现在其实是在经历通缩。



王孟源 09:31 

你如果去看的话,全世界都在通胀,只有俄国在通缩。那在这种情况下的话很奇怪,你这个为什么他们还把原本计划好的下一步挑衅?就是你原本如果是把俄国打垮了以后,你又整合了欧盟进入北约。这个时候在台海搞事就很有道理。因为这时候欧盟已经上了你的轿子了,对不对?跑掉不了,那这个时候就变成美欧澳日英这些西方的所有同盟、一个广义的北约,合起来要围攻中国。那所以我相信这些事情,其实Pelosi的访问原本几个月前就谈过了,后来他因为有新冠所以没来。那个其实是在俄乌开战之前就已经计划好的,就是下一步,就是用这个做引子来调起台海战争。那现在很奇怪的是你俄乌战争打输了,而且输得很明显、很难看,现在欧盟基本上是要自救而不可得了。你如果看到他的第七轮的制裁是什么意思?他们现在制裁了7 轮,对不对?到第七轮的时候,你去看他第七轮的制裁的条文的话,唯一的信息、实际信息,新加的制裁是不得从俄国进口黄金。俄国已经都没有办法用欧元跟美元了,他把黄金卖给你,换成欧元跟美元干什么?对不对?原本俄国就没有,就完全停止再继续对外输送黄金了,那你去制裁那个。所以现在基本上欧美,就是欧盟本身跟美国,现在他们所做的这些制裁还有进一步的所谓的新政策,其实都是为了广告。就是宣传洗脑用的,那一个对内,一个对外。这个对内是什么意思?欧盟是纯粹对内的,就是他要忽悠自己的百姓,说我们还在继续着制加码制裁俄国,我们还有希望能够把俄国打败,就是只要坚持到冬天,俄国人的经济就会崩溃,他的那个卢布就会崩溃。



王孟源 12:16 

美国人的忽悠除了对自己的百姓,做出这样的假象,维持这样假象,因为他们毕竟是过去几个月,过去 5 个多月,就是这一套主流宣传,众口一声的这样讲。但是他还有另外一个对象,其实就是欧盟,因为他也需要忽悠欧盟的政客,欧盟的政客之所以一开始会制裁的那么高兴,这不是推测喔,这个照理说以往...,到过去这个礼拜为止都还只是逻辑推测,但是刚刚有那个有英文的媒体就是泄露出来说的确是那个 CIA 跟MI6向欧盟的领导人在俄乌战争开始的那个时候做了简报。保证、拍胸脯保证说你只要做制裁,那个俄国的经济就会垮,是这样子他们才上轿的,上了轿以后当然现在是下不来了,被骗了、现在下不来了。那下不来的话,这个变成我上个礼拜已经谈过,现在的局面看起来就是欧盟要自己变成人家刀俎上的鱼肉,而这个人家就是美国,对不对?因为现在美国正从军工、能源跟金融三个方面去吸欧元、欧盟的血。那么我说过这对美国的经济是很有帮助的,因为第一个你这样子美元坚挺的话,至少它进口的货品的那个价格就不会上去。



王孟源 14:21 

我两个礼拜前也说过,事实上你算通胀的话要有四个方面,第一个就是原材料,第二个是工资,第三个是成品,最后一个是金融资产。那这一次的这个,我们这次的经济危机,一开始就是金融资产先下来,因为这是一个bubble、一个泡沫,那像是股票这一类的,它下降了大概25\%。但是问题是为什么现在还不形成一个危机?光是在资产这一方面,你可以看到美国现在的地产虽然交易量下去了,但是它的价格还没有下去。而且那个交易量不是0,就是还是有交易,而且那个价格还在有些地方还在往上爬,这就很有意思,这是为什么呢?因为这是一个通胀性的的衰退,而且现在这个通胀的压力主要是来自劳工工资。



王孟源 15:30 

原材料是过去这一年、原材料上去了,但是现在这个石油价格在 100 多块美元一桶,基本上稳定下来了,所以它那个推力消失了。就是过去这半年多,原本能源价格是一个很重要的推力,但是现在已经稳下来了。接下来没办法稳下来的是工资,我上个礼拜也谈过,就是美国的劳工经过 40 年的压榨,到现在真的是忍不住了。



王孟源 16:04 

尤其你被那个能源价格,然后再加上新冠的问题这么一逼,他们的生活费用上去以后,他们不得不开始加价,然后一加价发现有劳工短缺现象。你加价了以后大家还是要买。就是比如说我现在我家在过去这两个礼拜,我刚好也忙,忙什么呢?就是我要把这个房子改装太阳能电池板,然后我原来是烧柴油的,这个柴油一年前我买柴油的时候是 2 块钱一加仑,现在是 6 块钱一加仑。哇。



唐湘龙 16:47 

啊,涨 3 倍。



王孟源 16:49 

对,涨 3 倍,所以我就要换成那个太阳能的heat pump了,就是热泵,那个双向的冷暖气机,用电然后再用太阳能来驱动,然后我去问了价钱,就是跟 6 个月前相比已经涨了 30\% 了。就是这个太阳能板还有装热泵,但是我还是急着要,而且是急着。事实上他们跟我讲,我这个月底如果没有办法签订单的话,下个月就要涨 15\% 。哈哈哈,下一个月就要涨15\%。



唐湘龙 17:24 

什么都在涨。



王孟源 17:29 

你看这半年涨了30\%,然后下半年大家都看着大概至少还要再涨30\%、还有50\%。所以还是有储蓄的人、还是有资产的人,他们反而是急着要消费。这也包括去买房地产在内。如果你真的是需要一个房子住,然后你又负担得起的话,即使是现在你的这个房贷的利率已经上去了,这些上中产阶级的人还是赶着反而是加码去消费。



唐湘龙 17:58 

因为他预期接下去会涨更多。



王孟源 18:01 

对,因为你预期是长期通胀,所以在开始的阶段反而消费有那个相对有钱的那些人支持着,因为他们有剩钱,那这些剩钱如果被通胀拖上个 5 年、 10 年都不值钱了。还不如现在就直接拿来买一些实际的资产,比如说房产之类的,或者是像我这样的装修我的房子。那所以你现在看到美国第一季度它的经济衰退是 1. 2\%,然后第二季度零点几。



唐湘龙 18:40 

连续两个季度都衰退。



王孟源 18:44 

对。其实是很温和的。而且我上个礼拜也跟你提过,美国正式的经济衰退不是这种简单的rule of thumb,这个规则是给新闻记者用的,不是经济学家、专家或者是政府用的。政府正式的经济衰退是有一个 non profit organization叫做 Bureau of economic research。那这里面是美国几个比较知名的经济学家在那里面,他们负责来宣布说我们正式进入衰退。那目前的预期是,虽然我们有两个季度的负成长,这个 NBER 就是 National Bureau of Economic Research,不会宣布衰退。



王孟源 19:37 

所以你看,事实上拜登还是他的高官,还是Yellen在几天前也出来说,好像是Yellen。对,我想起来了,大概三四天前Yellen也出来说两个季度的负成长并不代表有衰退,他的意思就说,因为我上个礼拜也提过这个,你除了这种消费之外,另外还有就是失业率。那现在你还有劳工短缺的问题嘛,那些高收入的白领工作,像我上个礼拜谈的,银行跟高科技他们也都还在雇佣之中,就是雇人的速度减低了,但是还是在雇人,而不是在裁员。



唐湘龙 20:23 

裁员的时候。



王孟源 20:25 

所以的确是很可能他们不会把这次算作一个衰退,要等到...。所以你可以想着现在美国的经济所面临的问题是什么呢?它有一个很温和的衰退现象。这个温和的衰退现象为什么会这么温和呢?一个是因为欧盟让他吸血,第二个是因为这是一个通胀性的衰退,所以过去这几年累积的这么多现金,民间的现金还在拼命的花,就是怕这个通货的那个实际购买力继续的下降。我像我真的明天就要签约了,我明天不签约的话,哈哈哈,下个月就要涨 15\%。



唐湘龙 21:11 

好,这个是一个很好的case。好,那我们刚刚一开始讲到的就是说习拜会,当然习拜会你刚刚提到就是说看起来是很空洞的,而且美国方面所公布的这个报告很有限,而且基本上说谎的机会多。但是习拜会之后,你刚稍微提到了Pelosi,Pelosi在习拜会之后,你认为他仍然有可能来台湾访问吗?如果来的话,我当然。



唐湘龙 21:38 

从上个月开始我就注意到经过一整年大整修,完成了里根号航母,到了南海活动,他离开了南海,到新加坡去休个假,本来说要去越南,但并没有去。他到新加坡休个假,当这个礼拜当 Pelosi 就是说出访的新闻炒热的时候,里根号回来了,他带着他的战斗群现在回到了南海,在那边 stand by。那大家就在想说,难道中美之间真的有可能因为 Pelosi 如果坚持来亚洲到台湾访问,有可能发生军事冲突吗?



王孟源 22:15 

有。是这样的,我刚刚已经提到,就是 10 分钟前我已经提到。你原本照着他们的如意算盘是俄乌战争把俄国打垮之后,用金融战打垮之后,接着用 Pelosi来做一个引子来挑起台海的战事,然后就可以欧美合起来制裁中国,对不对?这样不是很好的一个局面?



王孟源 22:45 

这当然在战略上是有问题的,就是有点不自量力。但是你至少在战术上是很合理的。但是现在已经知道俄国反过来大获全胜。你连战术上的理由都没有了,那他为什么还在继续这样子?这个就必须要谈到,而且我先讲几句,就是大家不要说我自以为是,说现在这个客观的局势不应该从美方的观点来看,不应该再由 Pelosi 来挑起事端。你事实上你去看美国目前内部的政坛,民主党的人包括Sullivan都说不要去。现在拱火的是谁?都是共和党。



唐湘龙 23:40 

今天早上的时间,即使华盛顿邮报都公开的社论说 Pelosi 未来可以去,但是现阶段不合适。可是你发现只要是共和党的媒体或者共和党的政治人物都说去,如果你不去,那就是向中国示软,那美国就没救了。那共和党是这样子,可是那拜登能够去处理 Pelosi 吗? Pelosi 为什么感觉上面态度还是很强硬,好像非来不可。



王孟源 24:12 

就是客观的、从国家利益的观点来看是没有理由再来,因为你现在美国已经连俄国都打不下来,你再跟中国挑衅的话是自找苦吃,但是这个现在他是有点骑虎难下,因为美国国内的政治正确,这变成给了共和党人一个很好的借口。就是“你去啊你去啊”,你要是..因为他们的计算是你去了以后挑起问题,反正还是 Biden政权的问题嘛,对不对?如果没有挑起问题的话,那你可以在选举的时候说民主党人都是孬种,对不对?所以不管怎么样,你反对党在这里面都是有利益可以捞的。那Pelosi本人也是这样子,因为Pelosi他们家族贪腐的很严重。



唐湘龙 25:06 

她老公这两天又卖股票了。



王孟源 25:09 

比Biden还严重。对,Biden至少还是她儿子在贪腐,这个Pelosi是她自己跟她先生。不止她的儿子,不止她的那个子女,就是连他自己跟他的先生都是很有名的股神,就是很喜欢拿内线消息去炒股的。那事实上一般人没有注意到一个小新闻,就是几个月前有一次 Pelosi去做 town hall meeting,就是到镇上、地方上去跟民众交谈的时候有一个,这是当然是民主党的,那有一个民主党的民众站出来说,为什么我们现在的政治正确还是你站出来说我是社会主义者就会被人骂?就会连民主党内部都不接受?这是因为像 AOC 或者是Sanders,他们都想要说自己是 social democrat 甚至 democratic socialist,因为这美国的玩弄字眼玩的很多,就是你稍微有点社会主义倾向,其实根本跟社会主义扯不上边,只是比较关心民众的利益的话就叫做 social democrat,你如果是真正有碰到一点社会主义的边的话就叫做 democratic socialist。



王孟源 26:43 

民主社会主义者,那是在美国的现阶段文化跟社会环境下,你能走到最左的,事实上根本就不是,还没有碰到真正的社会主义,但是那已经是最左的就叫做 democratic socialist ,那 Sanders 他自己就想要称自己为 democratic socialist了,那但是他不敢,因为你这个拿出来会被美国全国的骂,包括民主党内。那所以就有一个民众他大概是Sanders的支持者,他就说为什么民主党的主流,比如像你这个国会议长不支持像Sanders这种站出来就说我们是为了民众的利益而考虑,所以我们是 democratic socialist。



唐湘龙 27:31 

这是一个非常好的问题。



王孟源 27:32 

对,然后 Pelosi火冒三丈,当场的破口大骂说我们是资本主义者, we are capitalist through and through。



唐湘龙 27:49 

那这也很诚实啊。他确实是。



王孟源 27:53 

很诚实,对,因为那个不是大新闻事件,是那个在乡下搞一个跟民众的交流,所以他没想到..。



唐湘龙 28:02 

众议员都要回到选区去做这些事。对。



王孟源 28:06 

她没有想到有人会注意到,但是我注意到了,就是这种事情你只有在网络上的自媒体才会看得到,主流媒体不会报的。那他的意思就是在资本主义这种社会达尔文哲学之下,“我是成功者,所以我就有权,我是有钱、有权、有势的人,所以这个我就能够为所欲为,我高兴炒股就炒股,我高兴拿内线交易来炒股,这是天经地义”。



王孟源 28:43 

他的心态就是这样,那所以就是在这种自私自利、每个人都为自己的那种文化心态下,你要记得虽然他是民主党人,照理说应该是左派,而事实上这种所谓的主流民主党人也是一样的,都是自私自利,有机会侵害国家社会利益而为自己捞钱的时候,他们绝对不会犹豫。



唐湘龙 29:14 

所以你认为他会来亚洲挑衅。



王孟源 29:17 

她来亚洲挑衅就是为了他自己的政治利益,因为这个这种跟中国敌对是争取选票最好的方法。那他的选举,他这个地方选举是不可能输的,因为他的那个是铁铁票区,但是他的那种结党拉夥的那种威望就是必须要靠出去闹事来建立。那他如果去台湾的话,他就想着纽约时报会把我怎么捧、CNn 会怎么样称赞他想的是这些事情。



唐湘龙 29:59 

对。



王孟源 30:02 

会不会引起国家的损失?会不会冒着产生核子战争的危险?这个他是完全不管的。



唐湘龙 30:12 

那我们看到就是说,比如说大陆方面来讲,除了昨天的习拜会,其实几乎都在谈台湾,那这次习拜会的一大特点,但是纯粹就 Pelosi 的来亚洲、来台湾这件事情,昨天习拜会最少公布的文稿里面并没有特别谈到。不过大陆的外交部已经第五次的强硬的声明,就说一定会做强硬的反应。那我刚提到就是说美国除了它的印太司令部以及我已经注意了两三个礼拜的,就是里根号的动向一直在这地方。那再来就是大陆方面会怎么回应?大陆方面我觉得比较明确的讯息倒不在于最近是不是有军机,然后穿过防空识别区,或者说是做一些小规模的演习,而是在福建当面的龙田空军基地,它很明显的最近的龙田的空军基地的扩充跟战机的进驻是很明显的,但是我从我们的角度,台湾人可能会觉得说这个是一个虚张声势的动作,你怎么看龙田基地?



王孟源 31:21 

这倒不是跟现在的这个消息有关系,因为这个已经进行了两三年嘛。只不过是刚好几天前完工了,所以那个新的远程战机进来,就是歼 11 跟歼 16 。比较重要的,我觉得你在考虑 Pelosi 访问台湾这件事的时候,比较重要的消息反而是一年前胡锡进所发的一些消息。我不晓得你知不知道胡锡进是谁?



唐湘龙 31:50 

我晓得,当然,他是环球时报的前任总编辑。



王孟源 31:54 

对,他退休以后反而说话的自由更高了,所以更激进了,但是它反映的事实上是官方内部的讨论,就是不是官方内部的决策。而是官方里面讨论可能。



唐湘龙 32:14 

那个细分。



王孟源 32:15 

就是内部讨论,不一定是已经决定了,但是考虑的一些选项。你可以看得出来。那他从去年就开始谈说,如果美国如果对利用在台湾搞事以便收编欧盟一起做经济制裁的时候,你可以用逐步的派更多的军机进入台湾领上空,而且越来越靠近台北。那最终就是到台北定期每天绕行嘛,对不对?这现在的问题是你这样就把这个所谓的英文里面就是把ball kick back to your court,就是这个烫手山芋又丢回去给你了,你敢不敢打?你敢不敢用飞弹打?你用飞弹打的话就是你挑起战争,而不是中方挑起战争,对不对?那他公开谈这个谈了好几次。那你想想看这个解决方案合不合理?非常的合理,就是在法理上是完全站得住脚,因为你现在理论上台湾上空也是中国的领空。



唐湘龙 33:34 

没有错,在法理上面是。



王孟源 33:36 

站得住脚。而且他只是巡逻,对不对?因为事实上他们从来没有签过什么协约或做过什么声明说你不会跨过台湾海峡中线。



唐湘龙 33:43 

没有。



王孟源 33:46 

这个都从来没有的,所以在法理上完全都是站得住脚。那第二个就是他也没有发射武器,就是来这边绕一圈。那现在这个问题就是你台湾要不要打,对不对?那这个难题就变成你打了以后打不过,不打的话,哈哈哈,它变成定期巡逻。他就是事先把这个风声放出来,跟你讲说我们的解决方案就是这样子。我也不需要跟你故弄玄虚。那看你们能不能破解我的办法。要是破解不了的话最好就不要升级。那现在,所以我刚刚说了半天说这个,从美国国家的观点来看,他们对这个并没有破解的办法。因为你现在连拉欧盟过来一起制裁中国的这个愿望都不可能实现,欧盟现在自顾不暇了,对不对?那你去升级,然后中国又有一个完美、很好的反击手段。至少一个,我刚刚只是举一个。其实还有好几个胡锡进也都讨论过了,比如说定下禁飞区,或者甚至就用军机去拦截,不过用军机去拦截 Pelosi 的飞机就可能搞成Pelosi偷偷摸摸把那个 transponder 关掉,就是雷达反应器关掉,然后偷偷摸摸想要偷渡进来这样的问题。



唐湘龙 35:20 

这不可能,技术上面,以现在的大陆的解放军对于第一岛链范围之内的空中的监控他办不到这一点。



王孟源 35:29 

但是他可以假装是普通的民航机。



唐湘龙 35:32 

对,但是我想他的所有的动向一定都是被监控的,我个人认为的就连假装都很难。



王孟源 35:40 

但是他要是偷偷摸摸的上了一架日常的。



唐湘龙 35:44 

那最有可能的。最有可能。



王孟源 35:48 

美国人现在就是会玩这种很低级的中学生的把戏,他们现在就是在玩这些东西,然后,我不晓得今天有没有时间,可能没有时间了,那就不要谈了。不过我想说的是美国现在的精英像是Sullivan啊或者是,他们这些幕僚水准,你不要说是Kissinger那个水准了,就是 90 年代像 James Baker那样的水准。



唐湘龙 36:17 

当过国务学。



王孟源 36:19 

连认真的搞政治的这个的水准都到不了,他们玩的就是美国高中里面最流行的那种所谓的霸凌的把戏。就是我看你不爽,我就搞一些花样来霸凌你。他们现在把高中霸凌同学的花样拿来做外交战略的斗争,这个是非常非常可笑的。没有时间不能够再多讲。这其实也跟,你如果去看哈佛的那个入学的标准越来越偏向就是会耍花招的,然后还有他们的学费在过去 40 年涨了十几倍。这些事情统统是连在一块的,就是它的整个社会腐化掉了,所以他的精英阶层的那个智商也就下降。因为你不再是挑选聪明的贫家子弟来做精英,而是纵容富贵阶级的子弟,那这些都是脑中空空的,只会在学校里面玩霸凌花样的人。



唐湘龙 37:34 

我称之为现在美国的大学的教育,现在招进来的都是有钱的纨绔子弟,(对)他甚至于在接下来从政之后,像是Sullivan这些人,他们就是政治上的纨绔子弟。他的重点是他是纨绔子弟,政治对他来讲这只是一种 game 的部分而已,他没有什么的高贵性跟崇高性跟严肃性在里面。好,那我们来,我们约时间来聊,我们回头,因为这个问题是我一定要问孟源,因为我们从一开始我们就来关注俄乌战争的进行,从二月 24 号开打了之后到三月底的时候,本来认为有机会和谈,结果因为布查事件那个时候对乌克兰来讲反正突然态度翻转,然后战争就开始进到了所谓的第二阶段。



王孟源 38:27 

态度反转不是为了那件事情,那件事情是乌克兰要栽赃的,因为他们知道他们说什么英美的媒体就照印。真正翻转是因为在三月底的时候,他们在土耳其达成了一个协议,那俄国认为是一个很接近、可以接受的协议。就是当时他们已经承诺不加入北约、承诺非军事化、承诺去纳粹化,唯一的问题就是领土的问题,领土主权的,那俄国认为这个可以继续谈,但是那个代表团一回去,从那个Istanbul一回去,泽伦斯基就接到Johnson的一通紧急电话,说你千万不可以继续,你要是继续的话我们要切断所有的资源,你不要想再从北约拿到一杆枪,不要再想拿到外来的一块美元的资助。然后一个礼拜之后Johnson就亲自到基辅去会谈,然后就是因为那样子,所以他才退出和谈,退出回谈以后,Putin就很生气。很生气以后,就是3月底他退出和谈之后,就结束了第一阶段,因为我说过好几次,他的这个第一阶段的军事是宁可牺牲俄国士兵的生命,也要...



唐湘龙 40:07 

尽量不伤害乌克兰的平民。



王孟源 40:11 

不是,这是到现在为止还是这样。但是在第一阶段的时候是他要假装包围基辅,就是做威胁,威胁要斩首,这个威胁的目的就是要以战迫和,那既然以战迫和是不可能之后,它就结束了第一阶段,然后改成第二阶段。第二阶段就把这些基辅的兵统统撤回来,撤回来然后投入了东乌。这里的问题就是我已经讲了好几个月了,就是他为了防备北约出手偷袭,俄国的原则就是我只派 20\% 的兵力在乌克兰,那既然这样子的话,它的兵力比乌克兰的兵力还要少,乌克兰一开战的时候正规军有 26 万,俄国连民兵,连那个东乌的民兵,再加上那个后来车臣的民兵进去也才只有 19 万,所以你进攻方反而比防守方还要少。



王孟源 41:18 

而且这个防守方是北约经过 8 年重建,完全是北约的战法跟组织,就是他的武器虽然还是苏联系列的,但是它的战法、指挥系统,然后还有美国跟北约的情报侦搜手段,什么卫星,什么电子战、飞机,都在罗马尼亚黑海那边绕来绕去这样盯着你看。所以在这样的情况下,你用比防守方还少的兵力去打,唯一的方法就是慢慢的用优势火力。



王孟源 41:56 

把对方的那个兵力消耗掉。那这个消耗的目标不是士兵,而是你的军官跟士官。你这些受过 8 年训练的军官、士官消耗掉以后,没有两年没办法再训练出来,对不对?所以既然这个两年是他补充的时间,那刚好你就没有必要急着打,反正你可以打上两年,他也没办法补充回来。



王孟源 42:27 

事实上,我想俄军的作战计划,第二阶段的作战计划是以大约半年为期的,就是他预期到今年秋天结束第二阶段。那我们其实已经是进入第二阶段的末期了,就是已经开始要收网,那为什么?我在这里要讲的详细一点,就是两个月前他们把那个Severodonetsk 跟Lisichansk打下来之后。



唐湘龙 42:55 

利息昌斯克。



王孟源 42:56 

Lisichansk打下来之后,他有两个选择,一个是就是趁机的席卷,加码席卷,另外一个是你停下来继续的轰。因为它的这个战法就是先用火力把你轰到你受不了了,你基本上兵力都散了、都死伤了,然后才慢慢推进一步,然后再继续的轰。反正你现在,因为他们是北约正规训练,所以都是纵深防御,你这光是突破一线以后并不保证你能够突破第二线,所以乌克兰的军队也都是训练有素的,北约的准则嘛,对不对?结果俄国依旧不急,所以他并没有加派部队,而且开始轮换,就是打下 Lisichansk之后,刚好那些胜利的部队就让他们家去放假,放了几个礼拜,然后再回来。所以现在刚好在这个礼拜我才看到有这些部队重新投入战争。



唐湘龙 44:03 

因为最近的,最近在俄乌战场上面,虽然我知道在国际舆论当中来讲的,现在对乌克兰疲劳却确实存在,可是俄乌战场最近有一些的,有一些战略层面跟战术层面的改变。战略层面我们看到,普丁最近比如说跟土耳其的关系走得非常近,他跟伊朗跟土耳其关系,土耳其8月 5 号埃尔多安还要再到莫斯科单独跟Putin见面,两个人两个礼拜见两次,现在北约非常的不高兴,觉得你羞辱了北约。第二个,我们看到南北韩竟然都表达了他们要投入到乌克兰战争,韩国也是要帮乌克兰,然后现在朝鲜说他要去帮俄罗斯,这个对战局会有影响吗?第三个,就是最近的乌克兰,你看到泽伦斯基除了跟他老婆拍 Vogue 的那种的,我现在不知道该怎么形容这一幕,那种的时装照以外,但是它对于在战场当中来讲,似乎在酝酿的一波的反攻,有组织性的、战术性的反攻。这是第一次,那在南部、在黑海的边上来讲,开始对科尔松发动了一个比较全面性的反攻。这个反攻会有效吗?会改变战局吗?



王孟源 45:24 

我先回答你最后一个问题,这个问题回答以后,你就自然解答前面的问题。事实上是北约以己度人,以为是俄国在打下 Lisichansk 之后就会赶快的加码,然后席卷整个东乌,但事实上俄国没有,他很有耐心,因为反正拖到冬天的话,我的那个能源。



唐湘龙 45:48 

天然气就把你活活冻死了。



王孟源 45:50 

天然气更强。所以本来在四月、五月的时候,这个连北约理事长、秘书长更不用说 Biden这些人,都已经在准备找替罪羔羊了,就是他们怕整个乌克兰的这个战局垮掉、很快垮掉,很快垮掉的话他们就必须要找人来当垫背。那最理想当然就是zelensky,所以他们那个时候都已经开始做铺垫,结果后来没有,发现俄军并没有加码,就是还是继续的、慢慢的、一步一步的往前进。那接下来,那这样子他就缓过气来以后发现,他们真正的利益在于对欧盟吸血,所以他们的战术指挥就转变成要制造这个乌方有胜利的这个假象。



王孟源 46:52 

又回头,就是不再是准备承认失败然后要指责zelensky,而是反过来又要说乌方其实是在胜利的。我前面其实半个小时前我已经讲过,他们这个,对美国来说,这是一石两鸟之计,一方面安抚国内的选民,对今年年底的一个选举可能有好处,另一方面继续忽悠欧盟,继续的不要跟俄国和解,继续的让自己的经济崩溃,这样他们的美国的军工、能源跟金融机构能够继续的去吸血,然后美元能够继续坚挺,美元坚挺才能够保障成品的进口价没有上涨。



王孟源 47:39 

你现在美国劳工资已经涨得像天了,然后这个能源的价格也依旧是居高不下,你要是在成品的价格也真的涨上去的话,它这个通货膨胀撑不下去。所以为了这个考虑你必须要创造一个胜利的假象。那刚好,这个我刚刚说过,俄国的这个它的重点是要杀伤正规军,但是乌克兰已经全民动员了,而且甚至已经开始动员女兵了,就是你路上看到一个年轻的女孩子也有可能被抓去当兵了,那这样一来,他这些没有训练、连那个枪支分解保养都不会的士兵一下子有了将近 100 万。就是这个炮灰、典型的、真正的炮灰。那这个炮灰,因为乌克兰的这个战略指挥,事实上是北约,北约的军官在指挥的,尤其是美国的军官在指挥,这是公开的,他们有一些那个校级的军官在乌克兰,然后真正的将领是留在华盛顿的遥控指挥。



王孟源 49:09 

所以他们为了创造一个胜利的假象,发现因为俄军的兵员不足的问题,他一次能够进攻一个地方,然后他继续只能够真正在东乌压力,而东乌刚好就是乌克兰剩下的正规军还有、还在的地方。所以你从俄国的观点来看,这是很适合继续的用火力去打,继续的每天杀伤两三百个正规乌克兰军。



王孟源 49:41 

但是这样子就代表他在北部的Kharkov跟南部的Kherson这些战线都是很松的,就是依防御所需的、防御正面最低的标准。比如说你在进攻的时候,一个营战斗群,合成营战斗群,进攻的时候正面是 5 公里,防守的时候正面就是 25 公里了,那你刚才Kherson就是用这种 25 公里的正面来防御,对不对?那以这种非常松散的防御的时候,俄国人的逻辑也很简单。



王孟源 50:20 

我在那边我也不计较寸土必争,你要来的话我就让你前进几公里,两三公里甚至 5 公里都可以,反正你来的这条路上我炸,我现在已经掌握绝对制空权了,然后在炮兵火力上也是 5 :1、6 : 1 甚至10: 1 的优势。反正你来的路上我把你炸一顿,等到你要逃的时候我再炸一顿。你这样子,所以事实上乌克兰现在为了制造这个在Kherson进行攻势的假象,可能累积了 6: 1 甚至 10: 1 的兵力比,但是这些是乌合之众,你不可能攻下Kherson。那俄国人也没有说非要寸土必争,反正你要来就来嘛。我让你来,反正我就几公里之下,你就一定会炸的战力全无,然后你又逃回去,逃回路上我再炸你一个不剩。那目前的战局就是这样子。现在如果你去看英美的主流媒体,他们当然是说在这个Kherson战线取得进展、准备要怎么包围...事实上你去看他们每隔 2 天的说辞,细节上就要换一套,为什么?两天前讲的那些都是谎话,包围了 2000 个俄军,没有的事。



唐湘龙 51:49 

如果真的包围,那真的是一件很大的事了。



王孟源 51:54 

没有,没有发生。这因为他们现在唯一真正占优势的就是英文媒体,所以他们高兴怎么编就怎么编,反正西方的百姓蠢的不得了。这个记性是不延迟超过 48 小时,你 48 小时后讲、跟 48 小时前完全相反的话,他们都不记得,而且一连发生一两百次,他们都不会学乖。对不对?我们过去这 5 个月,西方的谎言至少是上千次的,其中至少有 80\% 已经被揭穿了。但是你即使是,比如说昨天那个 pbs ,美国的 pbs,这可是跟纽约时报或者是 cnn 相提并论的民主党系的的主流媒体,他昨天站出来说四、五个礼拜前不是乌克兰说俄国用巡航导弹把一个公寓给炸平了,然后杀了 100 多个人,俄国人说那的确是普通公寓,但是在一个礼拜前刚刚被乌军占用作为士兵宿舍。好,你可以说俄国人在撒谎,但是昨天 pbs派了记者去,然后刊出一个视频说我到那边发现那个残骸还没有清理掉,残骸里面都是军用物品、军靴、军衣。



唐湘龙 53:27 

就是证实了俄罗斯的情报证据,。



王孟源 53:31 

证实俄罗斯的说辞是正确的,但是西方老百姓在乎吗?台湾老百姓在乎吗?不在乎嘛。这些东西、这些谎言被揭穿,你如果是在乎自媒体、主流媒体之外的消息的话,你可以证实俄国人从来没有有意的撒谎过,就是这些战场的细节没错,但是乌克兰方已经撒谎了 1000 多次。



王孟源 54:00 

你即使只看主流媒体,偶尔也有揭露的,揭露的也已经超过 100 次了。但是这些愚民他们就是不管,第一个就是不看、不听、不想。那这是没有办法的。为什么英美创造出这种民选体制?就是要靠你群众愚蠢、选民愚蠢的这个天性嘛,对不对?然后再扭转教育跟媒体、大众媒体来让这些群众更加的愚蠢,比天性还要更加的愚蠢,这样子他们资本的掠夺才方便,这资本主义的一个自然现象。



唐湘龙 54:46 

这个战争可能最后是到冬天由有老天爷来收场。最后这个问题我还问你,因为你在美国,昨天美国的参议院通过了芯片法案,今天凌晨的时间,就在我们连线前 3 个小时,美国的众议院也通过了芯片法案。这个案子过了那520 亿美金的这个补助,然后送到拜登的办公室,拜登一定会签署这个案子。这个今天法案是最近的,我们一开始提到了,最近有关于芯片半导体的讯息量是非常多,非常丰富的,他在告诉你,他在一个芯片的市场以及半导体的这个叫卡脖子战,它到了一个转折点上面,你如何看这个芯片法案?它能够强化美国以及之前Yellen到亚洲推动所谓的chip 4,就是这个就芯片的四超人的联盟,然后要进一步的去打击中国的芯片业。中国当然最近最大的新闻就是中芯半导体的 7 纳米制程的突破,我想这个呢,因为孟源本身是学就学理工的,你怎么看到中美的芯片的对抗,中国有脱困的机会了吗?



王孟源 55:54 

首先在八年前我开始写博客的时候,头两年我写了很多文章,你可以看出当时我认为产业升级是真正的战线,也是一切内外大战略的最终目标。但是那个前提是你还在全球化的环境之下,一旦双方撕破脸。这个前提假设在  2017 年、 2018 年Trump开始挑起贸易战之后,就被打上一个问号,然后到拜登上台,然后继续加码来封锁中国的时候,就变成两个问号。然后在俄乌战争发生之后,全球化已经死亡了,这个已经是断气了,没有什么好再讲的。



王孟源 56:51 

所以产业升级的这个问题已经不是主要战线,现在的主要战线是在于外交、军事,外交就是争取盟友,军事是保障不受直接打击,然后跟货币了。那这 3 方面稳定以后,就变成争取新国际架构的主导权。那所以在中方来说,反正你这个昂萨系跟第三世界之间的这个裂缝已经无可弥补了,那你不可能再依靠美国所提供的芯片的软件跟设备,所以这个只是一个,现在已经是百分之百的投入要建立自己的芯片工业。而事实上在整个第三世界,中国是在芯片上面远远领先的,因为排名第二的是谁?俄国,那俄国这个他最弱最弱的科技就是芯片。



王孟源 58:04 

那个落后中国,比中国落后。



唐湘龙 58:06 

那个更远。



王孟源 58:08 

还远,哈哈哈,那你再看看顶尖的芯片,它的意义在哪里?顶尖芯片其实都是消费品,手机还有那个笔记本电脑,这种东西就是在 GDP 上面很好看、在全球化的环境下很重要,但是一旦你这个世界是割裂了,新冷战来了,这个就变成不重要了。你像是什么名牌的皮包,一个名牌皮包可能是1万块美金,你真的是比 100 桶的石油值钱吗?你在全球化环境下,它是比 100 桶的石油值钱,但是一旦是在一个新冷战的的条件下,你说一个皮包能够跟 100 桶的原油相比吗?不可能。



唐湘龙 59:09 

这个观点很重要。



王孟源 59:12 

对,所以我先把这个大局的观点讲清楚。就是事实上中方芯片的技术的落后虽然没有因此而被弥补,但是事实上它的重要性一下被消弭掉了。因为你现在中方早老早就能够自制 14 纳米,现在这个 7 纳米已经开始试制了,这个是完全够用, 99. 9\% 的实体经济已经够用。那从美方来看,为什么美方这次会拿出这么多钱来资助?其实他担心的是被中方反过来掐脖子。



王孟源 59:56 

你想想看,如果在台海发生战争的话,台积电,这个全球领先的制程的那些设备还有人才,会归到哪一个阵营去?然后与此同时,美国有将近3万人在韩国,有一个反导的基地,那个会不会被遭到打击?如果美国参战的话一定会被打击的,打击的时候会不会顺便有一些炸弹莫名其妙的落到三星的晶圆厂了,这很难说,对不对?所以说不定反正已经开打、已经撕破脸了,你就顺便就炸一炸嘛。



王孟源 01:00:39 

那这样子是对美国美方昂萨系的那个的经济做打击嘛,对不对?那我不是说中方可能会这样做,我的意思是说美方这种以己度人,以小人之心度君子之腹,他们觉得如果我是中国我会这样做,所以他就必须要防备这些事情。为了防备同时损失三星跟台积电的那些晶圆厂,所以才投这个资金赶快把这些先进的制程引进到美国来,然后这是作为备用。他其实是怕被中方反过来掐脖子,所以我先把这件事解释清楚。就是这个,目前的这个争芯片的事情,你从技术上来看,中方要解决比如说 EUV 这个制程,就是欧洲荷兰所控制的那个最先进。



唐湘龙 01:01:48 

ASML光刻机,先进的光刻机。



王孟源 01:01:51 

我自己估计是要大概要 10 年了。 10 年这个数量级才能够自制。但是问题是在战略上它的重要性一下子不见了。就是以前你是在,以前的斗争是经济跟产业上的斗争。现在不是了,现在是外交。因为你一旦有这个新冷战,然后有这个两个阵营的裂缝的时候,俄国的芯片要跟谁买?伊朗的芯片要跟谁买,对不对?你即使中国还是只有 14 奈米的制程,他们还是会抢着要,对不对?那你既然有了这么大的世界的市场,这个最先进的制程就不再是太重要了,对不对?就是以往这个经济跟技术跟产业是最重要的。那是因为政治跟军事跟外交还没有撕破的脸。现在已经撕破脸了,这个是完全不同的故事。



唐湘龙 01:02:56 

好。当然现在刚刚孟源讲到的最后的这一部分,因为全球化的断裂,在过去美国,美国一直在向他的盟友包括台湾在诉求一个去中国化的一个产业供应链的逻辑,如果在那种的环境当中来讲,在一个全球化的框架里面来讲,那这种的半导体的锻炼对中国来讲可能比较有威胁性。可是我最近在看这件事情的时候,因为孟源刚提供了一个很好的视角,就是因为全球化已经崩解了。



唐湘龙 01:03:35 

老实讲,今天中国如果美国努力的要把华为给活活的闷死,华为的手机的销量近乎已经不只是腰斩,近乎快归零了,除了大陆市场以外,那当只有像华为或者说是一些的尖端的手机,它需要很先进制成的芯片,当然像华为这种的都自己都快被你闷死了,中国对先进芯片的需求或者全球这方面的需求其实并不大,真正的关键是在成轴制成的芯片这一部分来讲,中国正在快速的取得主导性。尤其对世界第三世界国家来讲,将来许多第三世界国家的发展,它的产业的发展,恐怕会高度依赖中国的成熟制成的芯片的供应。因此在到底未来在整个芯片产业,到底是美国在去中国化,还是中国在带动去美国化,我相信大概在未来一两年之内,那个趋势就会变得非常的清楚。



唐湘龙 01:04:33 

好,今天跟孟源从我们一开始讲的习拜会close,然后呢?谈到了就是说俄乌战争的情绪当中的判断,这个判断是非常重要的,就是我觉得美国还在打烟雾弹,还在意图尽可能拖延,因为对于美国来讲,我,我个人认为了莫云可以参考我。我觉得拜登会很难很难去说服美国的国内舆论。去年的8月我们丢掉了阿富汗,而今年的8月我们丢掉了乌克兰。如果说在每一年我们都丢掉一个重要的战场,那拜登真的就不用混了,而美国人那些信心恐怕会受到非常严重的打击,这可能才是现在美国对乌克兰战争当中不知道怎么收场的真正的原因。



唐湘龙 01:05:24 

好,来,我要感谢的我们的一些的观众朋友来卸票的,今天刚有两万位的观众在我们的线上看孟源的直播,从张洛图,谢谢在香港来 Alan 唠唠落唠唠。好,非常好的节目,观点都是中立客观的,谢谢那黑战 10 年韩庄一见了,那今朝出鞘是锋芒,佩洛西在拜习会对话,还执意要去台湾,解放军一定会给他一次难忘的经验。好,那朋友论谢谢你。然后 WK 马谢谢赵宏宇,谢谢他说他在日本念社会学,但是我们的节目跟着王梦远的节目对他帮助很大。我我我,常常听到我们很多很高端的在海外求学的知识分子在收看我们的节目,这个余有荣焉了,但当然希望大家学业都非常非常的顺利,希望真的能够帮上大家忙。他说他暑假终于赶上直播支持一下。



唐湘龙 01:06:23 

好感谢宏宇来 historical dynamics。他说裴洛西这一次一定会访问台湾,为了应对裴洛西访问台湾,解放军是按照 w 在第三次世界大战的顶级规格在准备的,美军还没有准备好,但是又爆发战争。我生活在台湾,孟源也来自台湾,我们一点都不希望看到这样的情况。好了,张宏文他在台湾,他说其实台湾跟西方政客可以打嘴炮,打的太习惯会以为中国也跟他们一样。如果观察近 10 年的中国,会知道中国是干实事的,从基层的建设,国际金融、防疫政策乃至恒泰的发展都是造表超客来达成目标的。所以不要以为中国在开玩笑,来,再来 w m 渺渺谢谢谢你在新加坡,然后 Heran Shen,感谢在美国,然后吉娜华,他说我先生皮特是澳大利亚在澳大利亚的湘龙和孟源的忠实粉丝,因为工作不能够看直播,特别打电话给他说哈哈。他现在在澳大利亚,然后打电话跟他说,你要懂内他好。



唐湘龙 01:07:32 

他认为湘龙跟王博士的访问节目是中文和英文世界震惊节目里面的天花板。他说王梦云是奇才,那奇知其精准深邃的洞察力,才知其广博科学的知识库,通过相同的节目认识了王孟源博士,进一步查看王博士的博客,获取了许多看穿震惊变化的视角。那真心希望这样的访谈呢再多再长一些。所以我找王孟源是有原因的,因为我看的感觉跟你的感觉是一样的。



唐湘龙 01:08:04 

好,那谢谢那君娜也带我。谢谢先生好,他在澳洲的等内好那语足惊人。他说,小英如果一直保持沉默,那希望我们的国军弟兄到时候也保持沉默,置身事外,不要为了不敢表态的统帅作战,一点都不值得。 Julian k 在日本感谢,然后肖西,王,好,那这在美国俄罗斯这次真的替我们挡了一炮,比当年苏联援助更为关键。



唐湘龙 01:08:36 

再来马克CW,那倩在香港感谢于露,好,他说谢谢湘龙和王博士的分析,那希望有机会能不能邀呢?温铁军呢?跟翟东生呢?也来上您的节目。好,我来试试看,好,时间的关系,还有后面还有,是不是后面还有。好,还有两位没有,我希望每位的听众朋友我都能够感谢到。好,好,来凯润,来凯润利在香港谢谢湘龙每周让我们能够见到王博士,爽,我也觉得很爽。我,这这,这不是恭维王孟源,我私下我就要跟他说好,再来张洛图,谢谢,在这香港好感谢我们的所有的观众来收看今天的龙行天下的王孟源的时间,那孟源提供的观点那希望大家心领神会之后对大家理解国际的震惊时事有所帮助。当然也特谢特别谢孟源在美国跟我的时间是相反的,但是能够配合龙行天下的时间提供的非常精彩的内容,感谢孟源。



王孟源 01:09:41 

很谢谢你让我有机会接触更多的听众,我希望心怀国家跟人类福祉的人都能够接触到事实真相,然后为传播真相,然后选择正确的政策做出一点点贡献。每一个人都开始传播真相以后,我们才有可能打破昂萨现在主导的谎言体系,因为他们真的是一个谎言帝国。



唐湘龙 01:10:15 

谎言,真的好,那因为孟源现在可能会很困扰,因为他应该会发现到现在华人世界里面的认识他的人指明率越来越高了,在生活上面是会有一些干扰的。好,不管怎么说,那感谢收看今天的龙行天下,周末快乐,下回见。拜拜。



\twocolumn[\begin{@twocolumnfalse}
\section{全球能源战,中美一盘棋!}
\subsection{20220819}
\end{@twocolumnfalse}]Credit: Aaron, Anonymous



唐湘龙 00:00 

湘龙,你迟到了,我不是迟到,是线上作业。不好意思,我们的听众朋友稍微的 stand by 一下,要开始了,来,准备。



唐湘龙 00:28 

龙行天下。好,欢迎收看龙行天下,我是唐湘龙。好,那今天星期五的时间, 9 点半的时段,那一小时最少一小时,但是我必须要再多预留一点点的时间。那因为今天龙行天下的来宾是王孟源,不管你过去认不认识王孟源,你也可能像我一样的是这两三年的时间才会注意到王孟源的谈论。当然有很多人说,我注意王孟源很久,偶尔我会遇到这样子的听众朋友,但是我高度推荐王孟源是因为我不只是我现在访问他,而是在过去我在看到王孟源那些视频或者他的博客的时候,我觉得第一个他比我聪明,第二个他在谈事情的时候比我有深度。所以我就主动的邀请王孟源成为龙行天下的固定来宾,把每个月最少 1 小时的时间割让给王孟源。



唐湘龙 01:32 

好,当然在结束的7月,7月的时候我邀了王孟源三次,几乎把7月都给了王孟源,让王孟源对于国际的政经情势可以做一些比较总体式的谈论。因为在地缘政治板块上面,随着俄乌战争的进展,随着全球的这种的地缘板块的破碎化,那这种国际政治的问题越来越显得复杂,那政经交错一般人不是太容易掌握到脉络,这个时候王孟源的分析非常的重要。好了,先介绍我们的来宾,来在我们线上的那透过视讯连线在美国的王孟源,早安。



王孟源 02:15 

大家好,我很高兴再上你的节目。



唐湘龙 02:18 

好,我说的早安是亚洲时间了。好,对孟源来讲现在是晚上。好,那今天我们基本上谈谈三个主题,那后面我们会花很多的时间谈谈全球的能源问题,包括了习近平,现在虽然还没有大陆方面还没有正式宣布,但是整个国际社会大概弥漫的那种,习近平即将要到沙乌地访问,而且会引起全球能源政策、中东问题、中美关系、地缘政治板块角力非常非常剧烈的变动的这种想象跟讨论。



唐湘龙 02:52 

另外在现在已经是8月底了,我们之前讲过,就是说到了冬天的时候,到了九月份,其实像现在高纬度,像日本最近的秋凉的感觉已经蛮明显了,那已经即将要进入冬天,一旦到了9月,像乌克兰这种地方,到了9月之后天气冷的速度非常非常快,那个时候能源天然气的问题都会变得非常的严肃。而不像是像夏天的时候,好像大家吵一吵,叫一叫,不太当一回事。



唐湘龙 03:21 

好,那这些我们摆在后面,但是我们今天既然待会要谈到俄乌战争什么时候收场,它导致整个的欧洲的能源问题的大混乱。我们先来看看俄乌战争,因为很久大家都在关注其他的议题,包括台海的问题,那乌克兰好像大家都快忘了,就是说 peak 到了一段时间了之后,但是如果你偶尔有注意到,我相信孟源有注意到,就是最近从俄乌战场上面所出现在讯息,尤其西方主流媒体的叙事,认为乌克兰正在酝酿一些反攻,甚至于颇有斩获。



唐湘龙 03:58 

比如在克里米亚这个机场的爆炸的事件,对俄罗斯来讲,虽然俄罗斯就闷着头,也不说是谁干的,那也没有喊痛,可是那个卫星画面显示那个机场受损的情况确实蛮严重,所以大家就说唉,乌克兰有戏了。俄乌战争现在是一个怎么样的情况?他到了乌克兰的反攻的时刻了吗?



王孟源 04:24 

我其实上个月已经提过了,这个反攻完全是宣传上的反攻,实际上虚构的。因为事实上乌克兰能够撑到现在,就是它的在火力上是只有  10 分之1 或甚至 20 分之1。那他唯一能够撑到现在的原因就是两个原因,一个是他已经全民动员,不断的抓壮丁往前线一塞,又把它塞到那个壕沟里面去。那你在壕沟里面总是有点隐秘的作用,我在过去这半年不知道看过多少,真的是上百个视频,是一个很简单的班阵地,就是乌军的班阵地,周边有 100 多个弹坑,就是俄国的那个火力是强到这个地步,那他们第一个够撑下去就是靠全民动员,所以它事实上总动员的兵力应该是在 100 万左右。



王孟源 05:30 

那你知道 俄军连当地的民兵还有车臣的民兵,真正在乌克兰作战的人还不到 20 万,所以他们从来都,一旦进入第二阶段以后就不会再做什么强攻,因为如果再做强攻的话,我前两个月也都提过,就是事实上已经没,有闪电战、二战闪电战的那条件。因为现在的这种卫星侦查、无人机侦查手段太多了,所以你没有战术隐蔽性?那此外美国所领导的北约同盟在事先把很多,就是有 9 万多件的反坦克单兵武器,还有 2 万多件的单兵防空武器事先送到乌克兰,那基本上世界上任何这一个国家能够面对这么多的反坦克跟那防空武器,还能够进行传统的机械化冲锋的,这是不可能。



王孟源 06:38 

所以俄军基于两个考虑,一个是战略上,战略上就是他必须要保留绝大多数的预备队,然后准备防备北约进行所谓的陆空一体战的突袭,那他不能够派出超过 20 万的部队到乌克兰作战,那另外一方面他也面对着北约的情报资源以及天量的反坦克单兵武器跟防空。所以你不可能再做二战时代的那种兵种协同,然后先轰炸,然后就由装甲突袭,然后的步兵跟上的这种传统作法。所以他现在基本上就是钝刀子割肉。用那个,用压倒性的炮兵火力来慢慢的杀伤乌克兰的兵员。那这个计划他们很显然的是原本就计划要进行钝刀子割肉。最新的证据说他们是原本计就计划这样打,是他们在3月初占领了Kherson跟Zaporizhzhia省份之后开始组织民兵,就是像Donetsk 的民兵那样,那是 8 年前组织出来的。他们在3月占领了这些新的地方以后,就从那个地方开始组织民兵,而且不像乌克兰是直接拉壮丁训练两三天就直接到城市那充炮灰,他们训练到现在才开始毕业,准备向前布置。



王孟源 08:23 

就是说如果你花了五六个月才训练民兵准备投入,那原本必然就是计划要打长期持久战,就是原本就计划要打上几个月,所以这是很明显的证据。那一旦这些民兵加入以后,这个俄军的前线总兵力就有可能超过 20 万,到时候你就有可能看到比较加速一点的步调。另外一个比较有趣的新发展是大概一个多礼拜前,一个中东很有名的资深新闻记者,就是他是有信用的、有名声的说他从伊朗得到消息确认,伊朗将出售 1000 架长航时的察打一体无人机。这一次俄军的进展其实我认为最大最大的战术弱点就在于它没有这种长航时的察打一体无人机,就是它有很多无人机,但是都是短航程的、小型的用来做炮兵校正。这个有一个问题,就是他对反炮兵有点问题,尤其是像HIMARS这种射程达到 80 公里的,你必须要有长航程的无人机,长期在上空监视,没错,然后看到直接就用导弹直接击毁,否则的话等你再送回讯息,然后你自己的长程飞弹过来,或者你的火箭弹过来的时候已经有三五分钟了,它可以跑掉,对不对?。



王孟源 10:12 

所以这个HIMARS现在美国送了 16 门俄军宣称击毁了 12 门,乌军宣称一门都没有损失,这个我们现在这个还是不知道,但是我觉得俄军在战术上,就是真正在战线上有点被动,这个最大的原因就是它的反炮兵的能力有点弱。就是虽然它达到了压制性,数量上的压制,但是他没有办法把最后的那 10 percent 解决掉。那尤其是像HIMARS是这种长程火力又很精确的,就变成一个很讨厌的蚊子,就是并不影响战局,但是刚好现在乌方他们所需要的就是外交宣传上的的把柄,那这个就对他们来讲很合适,因为乌克兰现在每个月需要 50 亿美元的资金援助。



唐湘龙 11:16 

没错。



王孟源 11:18 

不是说军事,而是光是政府的开支,购买能源什么之类的,每个月要 50 亿。那但是在过去7月里面,欧洲的主要国家,六个主要国家的总贡献是0。你如果再这样下去的话,美国也是不愿意为他买单,所以你现在你可以看出它很积极的在运用这些手段来激起欧洲国家牺牲贡献、继续投资的这个热情。那HIMARS是一个借口,就是不管实际上有什么战果,至少能够登上头条,就是你必须要占据新闻的频道。这就是他们的重点。事实上俄军(编注:结合上下文,应为乌军,次数口误?)即使牺牲几千人也是要吸引这个、也是要吸睛,也是要在国际媒体宣传上继续占据欧洲,至少欧洲人的注意力,这样子才能够保证后续的资金能够一直到来,否则他们自己先在金融跟财政上都撑不下去?,所以过去这一个月有两条大新闻。一个是他们在攻击Zaporizhzhia的。



唐湘龙 12:43 

扎布罗热的核电站。



王孟源 12:46 

对,这个另外一个是在Crimea的几个机场的突击。我先谈那个核电站的事情。这个核电站是俄军在3月初就占领,到现在已经 5 个多月。但是这个被炮击之后,你如果去看美国的媒体的话,一开始是先说这个是俄军炮击,那就很奇怪,你俄军自己占领的地方被自己人炮击。



唐湘龙 13:15 

没有错,没有错。



王孟源 13:18 

这个就是他们胡说八道,就是骗美国跟英国跟欧洲的那些愚蠢的老百姓,而且美国媒体把自己的老百姓当傻子,英国的媒体是把自己的老百姓当白痴来看。为什么我说这个差别呢?这个美国人像New York Times纽约时报,至少他们头两天说这个是俄军在炮击,他后来改口说是乌军在炮击,但是因为那里有驻俄军,他们至少不是同一天说这两个自相矛盾的说辞。而英国的媒体我去看了,在上个礼拜有同一家媒体,同一天有两篇文章同一天登出来在头条,一篇说是俄军在炮击那个核电站,那另外一篇说这个乌军炮击这个核电站,因为俄军利用这个核电站作为炮兵基地。你看这就是把英国老百姓当白痴,那真的,而英国人真的就吃这套。那我顺便提一下那个他们现在的这个宣传口径,已经基本上第三世界的人都觉醒,都知道他们这个盎萨是一个谎言帝国。



王孟源 14:40 

但是还是有一些仆从国家,例如像我们都很熟悉的台湾,他们根本就连白痴听众都不够格,就像一只小狗一样是在旁边当家畜?,但是他们听了以后反而群情激奋,然后说这个俄国是如何如何邪恶。你要知道 4 个多月前那个所谓的布查惨案,Bucha。他也是说这个俄军在占领之后撤退,然后撤退之后第一阶段作战结束之后撤退了,留下了几十具尸体,说是俄国人杀的,可是后来一个多月之后乌克兰自己起了验尸报告,说这些尸体都是死于炮击。那你这个就很奇怪了,俄军在那边占领了一个多月,他们自己炮击自己的步兵干什么,对不对?但这个这么简单的事,就你光看他们自己的宣传报道就可以看出事实真相是什么。但是全世界就是有些,大约 15\% 的人口,愿意相信这些谎言,这 15\% 的人口是什么呢?就是他们所谓的 golden billion,这个你如果去看Putin的演讲稿的话,就会听到他说,这就是昂萨在过去二战以来所建立的一个国际体系的一个仆从系统,大约有四个阶层,最核心是美国,第二层是其他的英文国家,英语系国家,第三层是欧洲国家,第四层是其他的附庸,就像日本、韩国这样子。



王孟源 16:31 

那他们现在,这golden billion的就是全世界 70 多亿人中有这 10 亿人是美国认可的先进工业经济,就是他觉得有资格参与先进工业生产,其他的国家你如果试图发展经济的话,美国就要不择手段的把你打击掉。所以我们现在看到的就是,他们现在这个宣传体系就是针对这 10 亿人,所谓的黄金的10 亿人。但是你不要以为自己是黄金就很了不起了,因为事实上美国内部的腐朽已经到了一个地步,他必须要靠吸外界的血来维持自己的经济。那我上个月已经详细解释过了,就是现在的美国的通胀原本会是一个足以推翻他霸权的严重的衰退,但是因为欧盟自己愿意出来挡子弹,就是Scholz的这个新的政权愿意自己出来挡子弹,所以现在变成德国优先死亡,当然英国死得更快,那是因为英国选择脱欧,事先选择脱欧,那跟过去这一年的事情没有太大关系,待会详细分析德国今年冬天天然气的问题,我们先转回话题来谈。



唐湘龙 18:17 

刚刚孟源提到,因为孟源现在麦的声音,我可能这边还要再调一下,因为有的时候会有点断断续续的情况。但是我们刚在讲到就是说俄乌战争的时候,当然乌克兰最,近因为这个消息大概是确定了,西方的媒体有报道,就泽伦斯基把他的一些一些军令的指挥系统,包括总司令等等的换了,那之前把它的安全部门的人也都已经换了。看他周围的人,第一个不管是不管是就是说指挥的能力,或者说是或者说忠诚度大概都有很多的怀疑。



唐湘龙 18:53 

那我一看那天比较平时的报道是华盛顿邮报,对于乌克兰在返攻这件事情,华盛顿邮报就到了科尔松的就是前线的位置,那华盛那个报道就说他看不出来乌克兰有在反攻什么,就说他前线基本上面就完完全全都还是俄罗斯在控制,那乌克兰也没有在那地方进行怎么样的军事的部署跟反击。



唐湘龙 19:20 

所以我们看到的不管是在克里米亚的这些爆炸的事件,或者像是扎布罗热的核电站事件,它可能都很可能是单一的事件,而且是被拿来当做新闻事件的处理。当另外一条没有办法被证实的,就是说普京是不是也把他黑海舰队的司令给换了,那这件事情可能跟他的黑海的这个指挥的体系是有关的,但重点是说俄乌是不是就在这个地方就僵持住了,还是说战事有可能会在未在某一段时间会出现比较戏剧性的转折,因为俄罗斯似乎觉得他胜利在望,可是在战场当中现在的讯息量越来越少,因为冬天快要到了。我记得之前我们也讲过,历史上面许多这种,尤其在欧洲战场上面,许多大规模的战争最后都是老天爷在决定的。冬天快要到了,俄乌战事会有转折吗?



王孟源 20:18 

对其他的国家冬天或许是一个问题,但是你不要忘了我们现在在谈的是俄罗斯。



唐湘龙 20:22 

俄罗斯就不是问题。



王孟源 20:26 

事实上是这样子,这个我刚刚已经提到,现在乌克兰唯一他努力的方向就是制造新闻话题,所以这个之所以会去没事找事,然后故意去轰击这个核电站,然后把联合国、北约、 g7 统统找进来。其实它有两个理由,一个小的理由是希望能用这个借口,然后让外部的维和部队进来,就是北约的维和部队,当然俄国不会同意,所以我说这是一个小的愿望,真正大的愿望其实就只是为了上新闻。因为就像你刚刚说的,现在越来越明显,这个没有新闻的原因是因为乌方的话,他还是天天想方设法的制造新闻,但是欧美方,欧美这方面的民众有点疲劳了,就是因为你都是假的,然后事实上战局就是胶着。那至于俄国,反正他是原本就是讲实力,他这次作战的这个算分数的时候,他的分数是算什么?他不是算你有多少公里的土地他占领,而是有多少的乌军被杀伤。



王孟源 21:55 

三、四个月前我刚开始上节目之后,我就估计过说这个乌克兰可能每天的伤亡总是达到 1000 人。那在两个多礼拜前,乌克兰国防部有一份文件暴露,泄露出来以后,他连否认都懒得否认。所以你知道他们对撒谎向来是没有什么顾忌的,但是连这个连否认他们都不敢否认。可以事后,我觉得基本可以确定是真的,就是到月底,这个文件是他们的一份伤亡报告,就是说乌克兰的部队国防部直属的正规军,到7月底,总伤亡人数是 191000 人。



唐湘龙 22:45 

191000 人。



王孟源 22:47 

你如果算一下的话,那折合一天 1200 多人。



唐湘龙 22:51 

差不多。



王孟源 22:53 

对,所以跟我的估计,比我的估计还要高。而我想现在真正有能力的、有军事素养的观察家都注意到,在过去这个月,事实上俄国的杀伤,俄国对乌军的杀伤速度其实又上去一级,就是很可能以往是一千人或一千二,现在可能是达到 1500- 2000 左右每天,这原因是因为乌克兰的正规军已经死伤殆尽,现在上场的都是临时拉出来的民兵。所以他们的军事素养不一样。



王孟源 23:33 

然后其次是在Kherson那边,他们其实是有尝试要进攻的,但是你尝试的顶多只营级或者旅级的进攻,但这个营级跟旅级的进攻是我刚刚十几分钟前有提过就是,他们现在能撑这么久主要是靠的工事,就是藏在壕沟里面,跳出工事出来,然后高高兴兴的往前冲的时候,这个时候炮火来了,你连躲都没有地方躲。所以你越是试图做进攻,你所损失的那个伤亡人数就越大,而且为了准备这场攻势,他们把在 Donetsk 正面就是防守攻势设建了 18 年最坚固攻势的那个地段的炮兵,全都撤到Kherson那个战线区。结果你一旦没有炮兵资源,俄国的武力侦查的部队就可以很轻松的去摸他们的前线,然后可以更精确地引导俄国的炮兵。就是这样子。所以他们在过去这 4 个礼拜,其实在最硬的,就是那个壳子最硬的地方有了突破,就 Donetsk 的正面反而有了三四个突破,这个有突破的原因就是因为他们的重炮兵跟那个最重要的正规军资源被抽掉。



王孟源 25:12 

那你想想看,在这种正面战场上每天伤亡到了 1500 人或者 2000 人,你忽然派一个特战部队到Crimea,去偷袭有什么意义?没有意义,就只有宣传上的意义。那事实上俄军到目前,我现在算到是有三个损失,那其中第二次的攻击是战果最为辉煌,一共摧毁了 9 架俄军的前线战斗机,其中包括 Su-30  就是,那是相当先进的,大概过去几年才生产的。所以你不能够说俄国没有什么损失,但是这只是皮毛之伤。他们换一个卫戍司令也就算了,因为黑海海舰队司令事实上就是克林米亚的卫戍司令。整个克里林亚半岛是由那个黑海舰卫司令主管的。



唐湘龙 26:07 

没错。



王孟源 26:09 

那他之所以会放松是因为他那个地方距离前线太远,超过 80 公里,连HIMARS都打不到,所以他就没有做那些防范的工作。其实你刚刚提到的那个卫星照我也是,一出来我就在看。我看的时候是在找什么?找弹坑,但是我没有找到大的弹坑,只看到一片草原,那个草地被烧掉,这个是油料跟弹药露天所被击毁燃烧的结果。



唐湘龙 26:45 

爆炸燃烧的结果。



王孟源 26:47 

爆炸燃烧的结果,对。



唐湘龙 26:48 

它不是被炮击的结果。



王孟源 26:51 

对,而不是被重型的导弹钻到地里面的弹药库,把弹药库给掀翻。不是那样。所以那个时候你只要看那个卫星照片,马上就可以得到一个结论,这不是意外的话,就是用无人机或者是迫击炮这种东西小口径。刚好打到他们在补给加油的飞机上面引发连环爆炸。因为如果是重型的导弹的话,它一定会是针对既有的工事,也就是那个弹药仓,估计那会留下一个很大的弹坑。



王孟源 27:36 

好,那事实上大家也许不记得,但是两个多月前其实乌克兰就试着用无人机,一个自杀式的无人机去攻击俄国境内,就是这不是在南边的Crimea,而是在北边的Kharkov,北边的俄国境内的一个油料站。哈尔科夫北边的,对,那边他做了好几次攻击,有一次是用直升机突击,有两次可能是爆破分子、特战部队去用炸药包炸。但是有一次很明显的就是他那个视频被录下来,而且传到互联网上,是用一个自杀性的无人机这样俯冲下去,砸到那个油罐,然后那个油罐就爆炸。那这就很奇怪,因为那个无人机在视频里面录得很清楚。那大家一比的话,它不是军用的无人机。结果后来这样比来比去,发现在阿里巴巴上面找到了,是中国制造的民用无人机。



唐湘龙 28:41 

民用的无人机当做军用攻击。



王孟源 28:44 

对,叫做 skyeye 5000。我今天想要上节目谈这件事,我就想会不会是他们又用 skyeye 5000 来做这个自杀性攻击那个那个Crimea的基地?所以我去看了,结果发现skyeye 5000 是他们在欧洲的经销商的商标。这个真正的制造商是厦门的云轮,厦门云轮科技公司。那他们的那个生产的那个产品线叫做 Mugin UAV ,而且刚好有两个产品线,一个是正常起飞的,就是普通的小飞机这样子从跑道上起飞的,那两个月前可能是用了那个,但是它还有另外一个生产线是垂直起飞的,就是有那个小的,小的像大疆无人机那样子,可以先垂直起飞,然后再转成普通飞机这样飞。这两型飞机它的载重分别是 25 公斤跟 15 公斤。你如果是 25 公斤跟 15 公斤的高爆药那就很厉害,那就相当于 3、5 发的重炮的炮弹,那就足以引发那个在加油加弹的时候飞机这样的连环爆炸。



唐湘龙 30:11 

是,所以无人机,就算是民用的无人机,经过小小的改装之后,做这种就小规模的有杀伤力的攻击是办得到的。



王孟源 30:21 

对,大家如果有兴趣的话可以到阿里巴巴上去,这个一台才...。



唐湘龙 30:26 

哈哈,好,这个大家自己去张罗去,好。不过来我们,因为乌克兰的战争看起来接下去,有一点打不完的味道,但是乌克兰战争,刚刚王孟源提到他最大的受害者显然就是欧洲国家,那欧洲国家现在不止挡子弹,而且我觉得德国的总理肖兹,我对他评价真的烂透了,他基本上面既无能而且也没有方向感,最近甚至于连处理巴勒斯坦的问题都搞得满面都是豆花。



唐湘龙 31:07 

可是德国终究是欧洲的强权,最大的经济体,全球全球第四大的经济体。但是德国现在的能源遭遇到了非常大的问题,它的核能也延后,延后除役了,它的燃煤也开始烧了,它的电价创高峰。那现在连北溪一号都有一搭没一搭的,北溪 2 号被美国已经阉割掉了,那德国的能源出问题,基本上就代表了欧洲的能源出了大问题,那这个冬天德国的能源会是一个怎么样的状况?



王孟源 31:40 

这个也是最近有好几天报道、主流媒体的报道,而且有点互相矛盾,所以我把事实真相跟大家解释一下。这个原本德国他们是在 Merkel 任期 15 年最大错误我认为就是放弃核能。这其实也不能够全怪他,因为德国反核是有很长的传统。



唐湘龙 32:17 

没错。没错,三十年以上。



王孟源 32:23 

绿党的第一党纲就是反核,所以你一旦绿党进了政府这是没有办法的事。那现在被逼着最后的这 3 家核电站要延迟退役,已经算大的退让,干嘛退让?为什么?为什么绿党会退让?是真的被逼着没办法,他们现在有多惨,我仔细跟大家讲一下,他们在这个二月俄乌战争开战之前,他们已经基本上德国的经济已经转向使用天然气为主要燃料,而它天然气的来源有大约 55\% 到 60\% 是来自俄国,有 3 条管道,一条管道是从捷克那边南线的进来,那经乌克兰、然后经捷克进来,那中线是从白俄罗斯到波兰,然后从波兰进来;北线就是那个nordstream 1北溪一号。那当初Merkel之所以要建北溪 2 号,就是因为你一旦建了北溪 2 号,中线跟南线就可以不管。



唐湘龙 33:45 

摆脱美国跟这些国家的控制。



王孟源 33:50 

对,当然这一旦俄乌打起来以后,美国就立刻鼓动他们在那个在德国的大本营,包括他们的整个新闻体系还有绿党,要求把北溪 2 号封存。好,那封存了以后,俄国一开始还没有限制天然气,为什么呢?因为俄国知道要第一个优先是先要在经济战、金融战上面求生存。那俄国一直到5月才能够完全确定自己不但会存活下来,而且会活得很坚强,比欧盟还要坚强。所以他真正开始动手去削减天然气,是到5月以后的事。



王孟源 34:43 

5月底以后的事。一开始还是很客气,因为俄国人他们也知道德国这次是真的运气不好,遇到一个笨蛋政府,这笨蛋不是我用的词。事实上 Medvedev 昨天才刚刚,Medvedev就是他们的前总统,昨天才刚刚出来说了一句话很有意思,他说欧洲人民必须站出来搞掉他们的白痴。他们的白痴是谁?他们的政治领袖,哈哈哈哈,他用的是白痴那句话,所以我相信我用笨蛋还算是相对的比较客气。



王孟源 35:31 

一直到5月、6月他才开始削减,而且这个是在波兰先启动关闭了中线,然后乌克兰把他的南线关闭了一半之后,俄国才开始对北溪 1 号动手,然后真正俄国的火气上来还是因为那个西门子的涡轮机到大陆xxxx。这个东西明明是你德国需要的东西,我俄国不在乎,因为我们的金融战已经胜利了,我不怕你,但是他们还在那边摆摆姿态,一副矫情扭扭作态,我想Putin是有点火气上来就说“好,既然如此的话,你们除非把这个燃气轮机修好,用双手捧着送回来,否则我们不会接受”。哈哈哈哈,结果德国是占了便宜还要卖乖。就是不但要北溪 1 号继续重启,而且还要俄国人来跪求他。那这样子结果就变成现在僵持的问题,所以就一路那个供气量又降到40\%,然后现在降到20\%。那所以我跟你讲一下这个实际的数据,因为我今天下午的时候觉得这个很有意思,所以我去查了一下,结果发现德国经济部手下的那个Federal Network Agency,就是管能源跟网路邮政的那个单位。



王孟源 37:27 

其实从4月开始就有公布这些详细的数字,所以你都可以找得到。如果有兴趣的人可以去 Bundesnetzagentur 的网站, Bundesnetzagentur他们的国家网络站去找第一手的消息,就是在2月的时候,就是战事开始之前的时候,他们每天进口 2. 4 Terawatt hour (TWh)的天然气,从俄国,然后从所有其他的管道进也进口 2. 4。但是到一个礼拜,从俄国进口的每天是 0. 5 ,就是从 2. 4 掉到 0. 5。



唐湘龙 38:14 

那差太多了。



王孟源 38:17 

其他管道从2.4 增到 2.8,所以它这个总量,就是从每天 4. 8 降到了 3. 3。但是问题是几天之前,就在我想大概三天前吧,有新闻说德国的这个天然气库存量还是达到了77\%,这个跟过去五年的平均值差不多,就是每年8月中的时候那个储存量累积到77\%算是正常,那这是什么意思?我去查了一下,他的这个德国的天然气总储存量是 243 TWh,这相当于冬天的时候一个半月的使用量。



王孟源 39:10 

所以事实上它的夏天的使用量是每天 1. 5,所以事实上每天进口 3. 3,但是你只用掉 1. 5,你还是 1. 8可以放到储存去。这也是为什么它能够累积下来。但是到了冬它的每天的用量是 4 或者 4. 5,这是一个普通的、平均的冬天,就是我后来计算,我把它计算一下,假设俄国人不把这个北溪一号完全关掉,继续的供应 0. 5,也就是德国继续的每天拿到 3. 3 的话,然后它又有77\%,就是 243 的总储存量现在有77\%,之后还可以继续累积。



王孟源 40:08 

到 10 月才开始拿出来用,在一个正常的冬天,它是可以熬得过去的。所以现在的问题是在于这个冬天会不会特别冷。OK。所以会不会用到北溪 2 号其实是一个问题,就是必须要这个冬天非常非常的冷,才会紧急到要把北溪 2 号重启去求Putin。基本上你要重启北溪 2 号的时候,你就是把面子都不要了,就是在普京大帝的面前跪下来求他原谅,哈哈。但是这个只有在今年冬天特别冷的前提下才会发生。



王孟源 40:54 

真正的这个天然气供应的问题在于什么?它这个原本的这个天然气的价格是大约 40 欧元每megawatt hour (MWh)。今年2月开始仗打起来的时候,它一下跳到 220 欧元每MWh,然后到了夏天又掉回到120、150。当然德国利用这个阶段拼命在全球收买各式各样,包括美国最贵的液化天然气,也是一样拼命的买。但是在过去这一个月,因为夏天即将结束了,然后跟俄国的这个态势,俄国在天气供应上的姿态越来越硬,所以这个价钱现在又回到了 220 欧元一个MWh。而且你现在这种天然气都是有期货,你可以去看那个期货的价钱,期货的价钱是预期到冬天会至少是 240 欧,然后到明年春天会是 200 欧。



王孟源 42:20 

所以我利用这个期货的价格我自己简单的算了一下,然后德国每个月所用的天然气的数量,我假设是跟去年的一样,就是有去年的资料,你可以把这个价钱跟它的用量全部乘起来,然后加起来,我得到一个数字,大约是 3000 亿欧元。 3000 亿欧元是什么呢?是这一场俄乌战争所导致,德国在天然气每年必须多花的钱就是从现在今年8月到明年8月之间,德国的经济必须拿出 3000 亿的美元额外的钱来买天然气。那这个也许是很抽象的,跟你讲一下这个 3000 亿美元占德国的经济的比例多少?德国的 GDP 是3万 8000 亿,所以 3000 亿美元大约是 8\% ,GDP 的8\%。也就是说德国在一年之内,它的能源的费用就是经济一下就凭空的不见了,因为忽然就是你所需要能源,就多出这些钱,多了多少?多了整体经济的8\%。



王孟源 43:57 

那让大家有一个更切实的理解,这有多少?我们拿中国来比,中国的经济大约是 20 万亿,如果中国一年要多花 8\% 的 GDP 来购买能源的话,那就是 1.6兆美元



唐湘龙 44:18 

1. 6 兆美元



王孟源 44:19 

1. 6 兆美元,它每年进口石油 40 亿桶,中国去年的原油进口量、年进口量是 40 亿桶。你把 1. 6 兆美元除以40 亿桶的话,你会发现这等于每一桶每石油要多花 400 美元。也就是说现在的石油实际上是 100 美元,如果它涨到 500 美元的话,对中国经济的影响就跟德国现在面临的这个天然气价格上去的影响一样。大家想象一下,如果石油的价格涨到 500 美元,忽然一昼夜之间就涨到 500 美元,这个中国所面临的这个打击就是,德国在俄乌战争开始之后这一年所面临的。



唐湘龙 45:20 

好。因为我们今天跟孟源在讨论了之后,我把题目就定为就说全球能源大乱斗,因为孟源刚刚已经提到了中国的能源政策的一个大的战略调整,德国我们再可以再观察个一两个月,大概就会知道今年冬天欧洲的这些个国家大概会乱成什么样子,因为对俄罗斯是没有压力的。



唐湘龙 45:51 

俄罗斯今年的第二季它的公布的数字,如果可靠,它今年的第二季它的外贸的剩余是创新高的,他现在的口袋越来越深,它的通膨率跟它的利率都已经回到了战前的水准,它的通膨,它的第二季虽然还是负成长,不过状况其实都比欧洲很多的国家要来的更缓和。



王孟源 46:17 

他们的经济同比是负成长,通胀同比是还是正的?但是你如果用环比的话,从一季到一季的话,经济是正成长,而且是……



唐湘龙 46:33 

没有错,就是它的经济动能正在恢复的过程,所以它现在几乎就是你如果跟战前相比来讲,你甚至可以说俄罗斯现在的情况即使不是毫发无伤,但是其实那个伤势远比方说乌克兰比欧洲国家来要缓和很多。所以德国有没有可能像日本,日本最近又开始偷偷摸摸的去买俄罗斯的原油,那你就知道就是真的到了困境的时候,其实是什么什么制裁都已经被放一边了,那国际原油因为全球经济状况不好,国际原油的期货价格现在反而在缓慢的下跌的,比俄乌战刚爆发的时候将近每桶 130 美元,现在大概是在剩下九十几美元。



唐湘龙 47:17 

不过当欧洲现在的能源困境,即使俄乌战争,不管接下去怎么样,这是整个欧陆欧盟非常严肃的课题,现在是无解的状态。但是我们看到最近国际在关注的可能会影响到越未来全球能源环境的就是中国,对于中东的经营是不是到了一个开花结果的时候?我们没有忘记,上个月的时候,拜登才刚跑了一趟中东,拜登去中东除了以色列,除了巴勒斯坦,重点当然是沙特,到沙特大家要关注了,因为拜登从上台前的时候对沙特的这个王储其实讲话就非常重,把他跟普丁摆在同一级的都是刽子手,都是杀人凶手级的。所以他去之前就很难堪,大家就在挖苦他,你要去见杀人凶手吗?你要呢?你要去跟杀人凶手去握握手吗?他没有握手,他说我们碰一下拳头。



唐湘龙 48:14 

大家认为这次拜登到中东的访问是非常不成功的,尤其沙特给他的态度基本上是一个冷待遇、冷处理,热脸贴冷屁股。那拜登的中东的访问当然显然想要,他有预期到中国很可能会在中东会有所突破,这个突破层面就非常多了,不只是能源的,包括金融的,就石油、人民币是不是会出现,以及在战略上面,中国会不会在中东找到一个强大的战略支点?因为中国大陆还没有说习近平去不去,但是全世界都在做这个准备。



唐湘龙 48:52 

我们做个假设,习近平如果去了,这代表什么?就是未来的全球的能源,中国和中东之间的关系会影响到全球的能源配置吗?好,现在的讯号,而现在的讯号是中断的,好像电缆线是中断的。好,所以现在梦云的声音我们是暂时是听不到的。好,但是我们现在再重庆把线路接上,以前的时候我们刚在谈的这件事情就是借这个冬天对全世界最严肃的考验,大概就是能有问题,欧洲的天然气以及在整个未来中东的石油要往哪里走?那美国当然希望在他自己成为产油大国的情况下面,他仍然希望其他的产油国是乖乖跟他配合的,继续做全球能源的主人。



唐湘龙 49:50 

可是现在看起来,中国方面来讲,对于中东的整个的部署,那个部署不是光说我有钱去买能源,去买石油,然后就能够过得了关的,显然并不是这样。中国毕竟我过去讲过,就是说中国的这样的一个三中关系的经营,从中亚到中东,他经过了这几年之后,我认为中国的中东政策,在过去如果三五年之前,我认为中国是没有一个明确的中东政策的,因为太难表达了。可是中国现在的中东政策,随着美国在中东的影响力逐渐的消退,中国在中东的影响力正在扩大,这个代表的意义是什么?我们过去讲过,就是说美国它的全球战略,它有三个核心的支点是它绝对不会撤退的,一个是北约在欧洲,一个在亚洲是日本,还有一个就是中东,那中国要在欧盟,要在北约的体系,你们要取代美国那个基本上不可能,要在中国的家门口要让日本回新的转译就是说从过去的托雅入欧重新回到亚洲,那认同亚洲这个大家庭那个难度也有点高。



唐湘龙 51:13 

但是我认为中东很有可能是破坏美国三个战略支点的最重要的突破口,因为沙特的态度越来越明显,当你说沙特真的有可能就从过去的这么亲密,跟美国的利益紧紧地绑在一起,然后就是因此根本的改变吗?短时间之内也不会。你知道国家,像这种体量的国家,它的战略的战略的转变了,那个就跟那航空母舰转弯一样,那个是一个很耗的耗能量,而且需要非常大的回转半径,同时必须要把速度控制得非常好,否则那个船上的东西都会不只是晃来晃去,船甚至都有翻船的风险。航空母舰的转弯就是现在在整个中东政策的一个重大的转变的一个开始。



唐湘龙 52:05 

好,那习近平会不会去中东访问?我认为一定会。习近平在北戴河之后,他到了东北,到了锦州,在锦州的讲话那大概出来了。虽然所有曝光的这些谈话的内容,我认为基本上都还是原则性的,可是他大概已经标志出习近平对于他的第三任及之后的这整个中国的经济的三个支点,那跟过去我个人在节目当中所整理的三个支点看起来是高度吻合的,就是香港的金融,新疆的新科技,以及呢跟中亚关系的这样的一个总体的融合,让新疆成为中亚经济的领头羊。



唐湘龙 52:55 

而东北所代表的是旧经济,东北所代表的是中国的铁锈区。但是随着俄罗斯的东向,俄罗斯开始从欧洲开始慢慢往东方转移,他注意到亚洲的时代的崛起,而俄罗斯终究有一半是在亚洲,俄罗斯开始回到亚洲,再加上就是东北亚的情势上面的转换,那对于朝鲜来讲,朝鲜终究要脱困的。



唐湘龙 53:26 

这个时候东北就扮演了一个非常非常重要的一个枢纽的角色,因此东北的设定跟整个东北亚是一致的,这个是在内部的部分,可是在外部的部分来讲,有关于能源,有关于科技的自保跟掌控,会是接下去习近平的第三任级当中最核心的战略。好,现在我们回到了,就是跟着王孟源现在重新连上线了今天的讯号,包括前面的声音到后来的影像。当然因为月孟源在美国,所以这个连线本身有一些对我来讲是节目当中不确定的风险,而且我们是 life 直播的节目,这里要特别跟大家说明他不是事后能够去简洁去修剪的好,所以这个当你在看现场节目的时候,你要听王梦媛讲话的时候,你难免就会有一些的风险。好了,我们回到,回到就是说中国和沙乌地之间的关系,习近平如果去访问沙乌地,第一个为什么?这个会有什么影响?



王孟源 54:29 

其实为什么?这个问题的解答很明显。就是俄乌战争一开始的时候,第一个礼拜我马上就写了一篇博文,欢迎大家再回去看。那里面最重要的论点就在于你必须要从货币上面去釜底抽薪,因为美国现在已经失去了在科技工业发展以及军事上面的绝对霸权,之所以还能够维持一点门面,其实都是靠着印钞票来维持的。那你如果能够把美元的无限滥发的能力抽掉的话,整个美国霸权就会从内部崩溃,那当然后来是欧盟自身自己愿意用欧元,还有日本用日元来当垫背,来当真的肉盾。



唐湘龙 55:31 

日元跟欧元都贬了很多。



王孟源 55:34 

对,而且其实第一个会死的我觉得是英国,因为他们从脱欧下来已经有 6 年了,对不对?在过去这三年本身元气已经大伤,更加承受不起这一场折腾,所以刚刚出来的数据是英国的通胀率 10. 1是g7 里面最高的。不过不管怎么样,就是我在3月初所发表那博文就讲得很清楚,中国在长期战略上的反击就是联合俄国以及其他第三世界国家去创立自己的新国际体系。那这个国际体系最重要的层面就是货币,那事实上已经有很多迹象,这个幕后有很多的谈判,在积极斟酌中,我想我两个月前好像有提过,我预期是在今年底、顶多明年初就会出现所谓金砖货币,但是金砖货币当然中国非常重要,俄国也重要,但是印度跟巴西跟南非其实一点都不重要,就是在货币层面上,甚至印度是一个负面的影响。真正需要拉近这个货币同盟的是谁?是沙乌地阿拉伯。



唐湘龙 57:08 

真的。



王孟源 57:10 

因为你一旦把沙乌地阿拉伯给拉进来之后,这个货币同盟就取代了石油美元。美元在 50 年前自己放弃了这个Bretton Woods system,就是二战之后的那个体系,是一个汇率的体系,等于是半金本位,因为当时不是直接的金本位,而是其他的主要货币都是固定跟美元做兑换,但是美元本身是金本位,那是 1971 年的时候,Nixon的时候主动放弃了,所以他当时也是想要无限印钞来占便宜。那无限印钞,这样一来就没有本位了,就变成所谓信用的。



王孟源 58:04 

那信用这个东西完全空的,那本身就是英文里面所谓的oxymoron自我矛盾的话,就是信用本身怎么能够当本位呢?对不对?所以当时就是他们跟沙乌地阿拉伯得到一个共识、签了约,要沙乌地同意让所有的原油交易都要用美元来定价,那这样一来的话沙乌地负责收美元,那其他的国家当然也就必须要、想要获取美元才能够买到原油。然后沙乌地累积了这些所谓的原油美元,然后再去投资到美国资产上面,大家都很高兴,尤其是美国特别高兴,但是这个循环的关键点就是沙乌地阿拉伯。那事实上目前世界两个最大的石油出口国家,头一位就是沙乌地阿拉伯,第二号是俄国。所以你如果能够有俄国跟沙乌地同时加入这个货币联盟的话,就保证这个联盟能够立刻推翻这个西方的系统。美元或许还可以苟延残喘,但是欧元跟日元会死得更快。所以这个我在3月的那篇文章已经讲得很明白,中方的一切外交努力重点是集中在沙乌地阿拉伯。那现在有这个成果也是,其实运气也很好,因为 MBS,就是他那个王储Mohammed bin Salman,他特别不喜欢拜登,就是你刚刚提到的。



唐湘龙 59:56 

拜登也不喜欢他。



王孟源 01:00:02 

那他现在在沙乌地国内的地位是绝对的崇高。他已经把所有的异己都该打的打,该关的关了。就是他已经是预备独裁的实质了,所以他才敢做这件事。但是我认为这件事当然是对沙乌地阿拉伯很好,对全世界的第三世界都好,只对美国不好。 但问题是美国在沙乌地是有驻军的,有 5 个基地, 5 个基地里面其中有两个就在利雅得。那我觉得 MBS 可能是年轻气盛,没有把这个东西想清楚。因为提供安全保障,偏偏就是中国的软肋,中国是从来没有提供安全保障的。



唐湘龙 01:01:02 

对,就中国很难对于其他的国家的政治人物提供一种离岸的安全保障。你如果来到中国当然没有问题,但是离岸的,如果你在国外的,因为中国就是个不结盟运动,他没有离岸的军力。

王孟源 01:01:20 

邓小平的政策,因为邓小平认为你要韬光养晦,其实在当时是完全正确的,以当时的那个实力是那个样子,必须要专注于发展经济。但是你现在到了已经问鼎世界霸权的地步,你还在自缚手脚的话。那这个问题就是美国的 CIA 不是吃素长大的, CIA 到现在只有 70 多年的历史,倒有 80 多件这种颠覆。



唐湘龙 01:01:50 

对,国际暗杀颠覆记录。



王孟源 01:01:55 

对,所以我觉得 MBS 跟习近平当然是可以谈得很高兴,但是问题是谈得高高兴兴,然后一年半载之后就忽然有个政变,然后沙乌地还有 2000 多个王子,随便找一个出来替代都可以,对不对?这个这局面其实是很危险,如果我是来替中方借箸代筹的话,我会说或许在沙乌地开一个基地是最理想,但是可能。



唐湘龙 01:02:32 

真的吗?好,但这个其实本来是我想要问你的,就是说我觉得中国当它走到今天的时候,尤其它的力量正在慢慢进入到中东的时候,就包括印度洋的周围的时候,它会不会慢慢的就是说走出,或者说不知不觉当中脱离过去的不结盟运动,然后开始有某种的结盟或者海外驻军的可能性。你觉得有这种可能吗?



王孟源 01:03:01 

我在我博客上也都讨论过。我认为正确的做法是仍然坚持不干涉其他国内政、而且不去全球驻军,但是反抗昂萨就是 mi6 跟 cia 去做颠覆、去做政变,你出这种安全保证是有必要的。要不然你三天两头就被人家干穿了,你这个生意怎么做?



唐湘龙 01:03:28 

这个其实是我觉得中国在海外投资做生意很大的风险,只要跟你走太近的时候,那个人被推翻了,你可能就血本无归。



王孟源 01:03:39 

现在基本上是必然的事。那我觉得中国当然没办法说几个月之内就在沙乌地搞出一个基地来。那你至少先把大使馆扩建一下,然后建一个直升机坪,到时候 MBS 如果要避难的时候还可以躲上几天嘛。这是最基本的事情,但是我觉得中国不会做。



唐湘龙 01:04:03 

孟源不用讲话不用赶,我可以把节目延长半小时的时间。我们的听众观众朋友,我平常很少建议我们的观众朋友来表态,如果你喜欢王孟源的,你可以在那留言板刷一波,让王孟源知道一下。就现在一万六千多位的观众,除了听王孟源讲话之外,如果你喜欢王孟源,刷一波让王孟源知道一下。好,那沙乌地有可能会接受人民币作为石油交易的主要的货币吗?就石油人民币会出现吗?



王孟源 01:04:37 

我认为是是金砖货币,就是我其实3月的时候我建议是另外创一个新的货币?,然后另外取一个名字,但是我想俄国人有点顾忌,因为俄国的经济跟中国比起来太小。如果就是俄国跟中国做主导的话,就是 g2 的话,他怕自己会被中国压倒,所以我想是俄国人去拉中国去搞金砖。那事实上这次对抗昂萨霸权的这场起义是俄国人冲在最前面,所以他的话语权益最大。那就因俄国的利益,所以这个国际货币就搞成金砖货币了。那我这个担心、这个担忧我其实在我们博文上也在半年前就已经讨论过了,真正的问题不在于其他那些国家,巴西什么的那个都能解决。最大的问题在沙乌地。



唐湘龙 01:05:38 

沙乌地。



王孟源 01:05:40 

在于沙乌地,你想要拉沙乌地,但是你怎么能够把沙乌地拉进来而不让印度进来?这个是很困难。那事实上现在很显然没办法拒绝印度进来。所以一个多月前开始有风声说这个新货币会是金砖货币的时候,刚好习近平做主席、做主场来召开金砖会议,你晓得你记不记得,他所做的演讲重点就在于我们要赶快扩充金砖集团。



王孟源 01:06:19 

这为什么呢?其实也很简单,你如果不扩充的话,这个印度就变成五大创始人之一,那这还得了,什么都可以蒙。哈哈哈。你要是创始货币的时候已经有十几个成员的话,那就谁都没有否决权,那就看谁的朋友多,那这样的中国至少还能够控制住局面,对不对?所以现在我觉得习近平去沙乌地,我想主要还是谈双边经贸合作。这个加入金砖这件事情事关重大,如果是我来建议的话,不会说这么急着宣布,而是继续的酝酿准备。但是利用拖下去的这几个月的时间,赶快把 MBS 的安全问题解决,因为这个沙乌地国防军的忠诚度问题很大,因为他们都是美国人训练的。



唐湘龙 01:07:31 

没错。



王孟源 01:07:34 

这个到时候要搞政变。我真的觉得 MBS 是年轻气盛,这个初生之犊不畏虎。



唐湘龙 01:07:42 

哈哈,你现在很为他的安全担忧对不对?



王孟源 01:07:46 

你这样的去搞美国的基本利益,你不怕被他用政变搞下来?这真的是哎呀。你想想看当初吳廷琰,越南的吳廷琰,他不是去反抗美国啊,他只是因为他是天主教徒,所以对佛教徒打压的时候过分了一点,美国人就为了这么一点部署xxxx就把他政变、然后枪决了。美国人搞政变的那个要求水平很低的人。这个 MBS 这样子搞真的是哎呀。



唐湘龙 01:08:30 

好,我们刚刚呢,我在跟我们的观众朋友要对王孟源表态的时候,王孟源又在专注讲话了,看不到我们画面上的留言板的跑动,那留言板你们一直跑。好,那我们的观众朋友因为我接收到的讯息就是喜欢王孟源的那个坚固的程度,忠诚度是非常高的,不,不是一天一夜,就是短时间看视频的热闹的,所以有高度肯定跟专注度的。好,那我们把我说了,我把时间可以延半小时,但是我还要花一点点的时间,你觉得中国有可能同时在中东搞定沙乌帝跟伊朗吗?你刚刚提到就是说在金砖体系里面来讲,就全球能源体系里面,俄罗斯跟沙乌地非常的重要,那这两个体系如果结盟了之后,哇那个如虎添翼,金砖体系的支配力量会非常惊人,其实我们还少谈了一个伊朗,那伊朗这个时候能不能够进到金砖体系里面,不只是美国的问题,而是伊朗和沙乌地之间的传统的那种世仇的关系。他们有可能整合到同一个体系里面吗?



王孟源 01:09:44 

你说到一个很有趣的考虑,事实上在我三月的博文里面,我特别说不要对伊朗太积极。这个考虑的背景就在于是因为伊朗跟Saudi 要和解,当然是有点迹象,但是你在促成他们和解成功之前,还是以吸收沙乌地为最优先,不能够为了伊朗。因为伊朗在世界原油供应能力,虽然它的这个储备很多,但是因为被美国制裁了 40 多年,所以它目前生产能力只有沙乌地的 1/ 3 左右,所以你不能够因噎废食。但是后来也就是我刚刚提到有关印度的考虑,是由俄国压倒了中国的考虑,那在伊朗上面这个样子。俄国,因为我刚刚不是提到说金砖会议,金砖联盟在两个月前开了一次会议,一次的网络连线会议,有中国做东,在那时候宣布的第一个要加入金砖的新国家就是伊朗。



唐湘龙 01:11:05 

就是伊朗。



王孟源 01:11:05 

为什么呢?因为我刚刚才解释过,从中国的观点来说,伊朗是可以缓一缓,当然是欢迎的潜在盟友,但是重点应该是先把沙乌地拉进来。但是从俄国的观点来说,伊朗是第一个必须要拉,为什么?因为俄国现在基本上已经放弃了对欧盟的幻想,在过去的这 30 多年,就是苏联瓦解之后,俄国的大政略一直是要希望融入欧洲的。一直到这一次俄乌战争才是彻底放弃这个幻想,那放弃向西融入的话,你就只能往东、往南找替代,往东的话就是找中国。但是我说过他对中国有点忌惮,因为中国的经济体量要大太多了。



王孟源 01:12:01 

那所以必须要搞印度还有中东来做制衡,这就是所谓的南向路线。那他的这个南向路线也是大概在5月底的时候公布了一个计划,叫做北南通道。这个北南通道就是从,就跟这个中国的一带一路有异曲同工之妙。中国的一带一路,比如说那个陆路是从中国到俄国,然后从俄国再到欧洲。这个俄国的北南通路是从俄国到伊朗,然后从伊朗往南到印度还有非洲,那所以伊朗基本上是这个北南通道的门户,因为伊朗就在波斯湾上面。



王孟源 01:12:51 

没错,你一旦到了伊朗就可以把那个原油或者是货物直接的送到海轮上面去。俄国跟伊朗之间虽然没有直接接壤,但是他们都面临里海,所以可以靠着里海,可以利用里海来,等同于直接接壤了。所以对俄国来说,他向南要到印度洋。如果是走伊朗这个路线的话,只有一个国家,一个中点国,那这是最简单、最直接路线。而且伊朗对俄罗斯的依赖非常的深,它被制裁 40 多年,本身经济很衰弱,军事上面急需现代化,那就是因为这样子,所以伊朗才成为金砖集团所吸收的第一个成员。那接下来我相信幕后Putin一定会积极的去斡旋,让沙乌地阿拉伯跟伊朗尽速地和解。那中国当然也可以加入,但是中国的外交体系没有这个传统,所以顶多就是说你们和解之后我会锦上添花帮你们做些场面。



唐湘龙 01:14:16 

没错,中国的态度。



王孟源 01:14:20 

但是真正的国安层面、战略层面去和解的话,我觉得需要俄国人去穿针引线。那我相信俄国人已经很积极的做了五、六个月的工作,只是没有公开而已。



唐湘龙 01:14:35 

好,当然。。。



王孟源 01:14:37 

我稍早提到说可能买了1000架。。。就是这个报道,我认为是可信,但不是完全,我们先姑且相信就是他买了 1000 架(察打一体)无人机。那这个等于是准同盟,对不对?就现在跟中俄之间的准同盟关系一样。



唐湘龙 01:14:56 

没错,好,因为在中东,你像中国、俄罗斯和伊朗之间都已经可以在波波斯湾附近举行小规模的军演,就是那个,那个就是隐隐性之间的,这种的同盟跟军事合作的味道已经摆在那里。但是你终究不能够忽略掉。我之所以说萨欧弟,当然我完全同意,就是说孟源刚刚讲的就是中国的未来的,尤其在美国的全面封锁打压之下,中国突围借着金砖手拉手去突围,比自己单打独斗要有用的多,而而所有金砖体系的突破的重点会在沙乌地,那美国难道不知道吗?美国当然看也看得出来沙乌地太重要了,所以他不可能轻易的放手。



唐湘龙 01:15:44 

那我们也不要忽略了,我刚刚提到了,就是说美国的全球战略,它什么地方都可以失去,但是有三个地方它绝对会守得非常死,一个是北约,一个是,一个是日本,就每日安保。另外一个就是中东。在沙乌地。好,那因此这三个支点,沙乌地很可能是中美真正在美国的核心利益的三个核心利益区域真正交手的第一块,那这一块中国能够达成怎样的突破?你不要忘了美国在波斯湾里面是有它的第五舰队的,他在那个地方的经营不是只有萨欧迪而已。好,所以慢慢来。但是最少如果真的习近平到萨欧迪访问,那个是一个破天荒的开始,世界可能就开始真的变得不一样了。好,因为时间的关系。来,我要卸票一下,来,因为我们的今天的,今天的听众朋友,那第一个在我就是说在回复留言之前再跟大家说抱歉,因为刚刚的讯号的中断的关系,刚刚有几分钟变成是我个人的 solo 那但是但是我知道所有人在这个时刻都是为了王孟源来的。那事后没关系,你可以再收看视频来感谢我们的观众朋友们破论,谢谢。



唐湘龙 01:17:00 

然后乌艾瑞斯那他说王梦宇老师,我至今还没有看到外交部对习近平出访沙特作出确认,内心十分焦急,非常不希望大陆错过这个机遇。对老伯大陆的有关于领导人的出访向来没有到出访这前 3 天他不会讲话的,他基本上都是这样的态度。好,再来接刘,感谢,再来给他 Grace 力,谢谢凯文令,谢谢他小小的斗内支持香龙的访问,王孟源感谢好了这个肖 k 王支持王孟源宇宙真理。他说好消息,统一进了统一教室是禁了统一我还不知道哈,统一还要再努力,大家都要努力 w m 秒秒那他说差点就错过了直播,他说王博士能不能和龙哥开个长期节目,我一直有在想这件事情,可是那这个我跟王孟源再商量一下来,我不能帮王王部门答应好再来呢。



唐湘龙 01:18:03 

汪王王仔,他说每周到关注香龙的各个节目,今天的音质不好,对了,不好意思,好的,麦克不给他调整一下,听的有点辛苦。对我问八九,我刚我也注意到,就音值上面来讲,一直都有一点破音的感觉,这个是在连线的时候我很难控制的部分,因为我们借助的是 y t,然后又是用连线好的凯瑞丽来。王博士向两位辛苦了。王博士昨天的微博有一条评论回复了关于这次大陆针对台湾的军事行动,好像并不认可,后面微博可能各种原因删了,能不能简单的说明一下?你要说明一下吗?



王孟源 01:18:42 

我觉得我在博客上面所说的事情比较精确,我可以停下来字斟句酌。好,所以还是请大家博客。



唐湘龙 01:18:53 

大家注意微博,就是有一些时候,王孟源如果不止在分析,不止在叙事,他认为讲话要非常精准的时候,他就会用文字的方式去表达,因为那个会比较精准,同时要做一些调整也比较方便。那我们做直播的人,平常讲话人,你始终有个风险,就是话一出口了之后你要再去修饰,变得非常的困难,有的时候真的会词不达意,或者很难用三言两语表达一个很复杂的观念。好,这个请大家要理解。再来,他说这个肖 k 王,他说王博士有没有微博以外的平台?他说我微博账户早就不记得了,现在也不好注册,你还有其他的平台吗?联合报。



王孟源 01:19:39 

微博是一位北京张先生帮我维护。他只是搬运我博客上面的,我真正的大本营是在联合报的博客栏,所以如果去udn 的话可以搜索我的博客,那才是我真正维护的重点。



唐湘龙 01:20:01 

好,那我们的这么当然,在内地的观众朋友,可能你就比较辛苦一点,你不见得能够看得到联合报,那联合报在台湾算是一个相对来讲我觉得不只是它的平衡感,而是联合报是我既做新闻工作联合告的。联合报的质感要比其他的台湾的媒体都要好很多很多,不管这政治立场如何,即使是立场接近的联合报也都要好很多。



唐湘龙 01:20:27 

好,再来的 90K 感谢在日本,然后哀怨城他说的请这个王博士继续来我们的平台继续讲。李连范,感谢朱婉怡,谢谢。然后 003 他说谢谢王先生的分享,杰利杰瑞赖他说喜欢王先生,了不起的人才,我看有些人对你的爱已经超过了我的想象,会让我嫉妒。好嗨,就立潘,感谢来胡少瑜,谢谢。然后戴维威,他说今年看一票欧洲政客各种花式的自残,已经心态崩溃,那通鹏这么明显也就算了,主要是现在在欧洲我看不到未来。



唐湘龙 01:21:14 

对,这是一个很麻烦的问题。张家慧,谢谢你,然后申泽敏,谢谢,这是在日本后面还有吗?我要怎么样?再怎么样点,点下面的对不对?好,深泽米再来尖叫,感谢。在加拿大索利亚利,感谢。然后神奇光,感谢。神奇光说一个月一次不够,能不能一个月两次或者开一个代妆节目?我,我说了,我,我很愿意去凹凹,凹到往后圆,我再跟他的研究一下好了。今天还是在跟大家 say sorry,有时候讯号我曝光有的时候很好,但是今天的讯号训话的品质不是太理想,但是我讲的精彩的内容,大家只要专注听的时候你还是可以可以听得非常的有收获,那当然最主要的感谢人在美国跟我们的十三十二个小时以上相反的时间,但是为为龙行天下听众的朋友们提供了非常深度的就是说这些解读的跟一些专业的判断的。王孟源,感谢。



王孟源 01:22:21 

好,谢谢大家的关爱,有空还是到我的博客去看,然后有更多的讨论议题,而且可以讲得比较精确。



唐湘龙 01:22:30 

就是好。所以如果有一些直接你要你觉得比较深度的问题,我觉得你直接开在孟源的博客也好,或者是在联合报你留话,那以我觉得的王孟源的个性,只要你有点到他的点,他一定会回复。感谢孟源,感谢周末快乐,也感谢我们所有的观众朋友线上的收看收听,周末快乐,大家平安,拜拜。



\twocolumn[\begin{@twocolumnfalse}
\section{俄乌战争反转了吗?}
\subsection{20220916}
\end{@twocolumnfalse}]六月十四,已完成,“xxxx”标注是听不清楚的地方。



王孟源 00:00 

声音,出去好,好买个。



唐湘龙 00:13 

沸腾。,重庆天下不,好,欢迎来到观点平台,欢迎来到龙行天下,我是唐湘龙,现在早上的 9 点半钟,龙行天下所设定的栏目里面。



原则上每个月最少会有一次王孟源大师上课。那在当我透过许多的管道找到王孟源的时候,一方面我自己会觉得有如获至宝的感觉,因为我已经注意他的一些谈论非常久了。第二个我也不意外,我们的观众对王孟源的粘着度、忠诚度非常高,因为他的叙事风格的独特性跟一致性,我相信这是许多喜欢王孟源、王大师的观众朋友们的主要的原因。好,那今天王孟源上课,我也是听课的学生。



那今天在跟孟源沟通之后,我们设定了三段的主题,当然最重要的是我们先从俄乌战争谈起。在过去几个月的时间,几乎每个月跟孟源在连线的时候,我们大概都会了解一下俄乌战争的最新的情况,这个情况不是一些单纯的大国战略等等,而是战场上面的讯息其实很有限,而且里面都会包含着很多很多真假不分、虚实末变,那有许多是官宣,有许多是被被网络上面刻意编造过之后意图带风向的假消息。那这使得你要了解俄乌战争真实的情况变得非常的困难,你要去追踪很多的管道,正反面的管道的讯息,同时要很有耐心的去做交叉比对,在比对的过程当中要确定一些的事实面,它到底代表怎样意涵,这绝对不是简单的事情,即使在战场当中,你现在即使身处战场你也不见得能够做得到好。因此跟孟源在谈这些事情,尤其在从上个月到这个月,我想在俄乌战狼最大的转变就是乌克兰反攻了。好,那乌克兰反攻,你看到整个西方媒体的叙事,以及在基辅泽伦斯基官宣的内容来看,都是一副越来越胜券在握,然后觉得自己好像,好像非常的强大,非常的强势,除了拒绝谈判之外也不断的放话就是我要打到莫斯科,我要收回克里米亚,那我绝对不跟俄罗斯谈判,等等。那西方似乎也都在呼应这样的主张,那也不断地/源源不绝地,比过去看起来更积极地在供应乌克兰的一些军需的要求,真的是这样吗?好,那这个是非常重要的一段了,比较完整的就俄乌战争的分析,现在在所有的媒体当中不太容易看到。



唐湘龙 03:25 

第二个部分我们再来看,因为美国即将要大选了,再过一个多月的时间,那这个大选本身,不管是其中选举的现在的气氛也好,或者是在其中选举的酝酿的过程当中所导致的美国社会的许多的价值观、意识形态的尖锐的对立,已经全面的台面化了。那王孟源除了做政治观察之外,他毕竟很长时间是在美国生活工作的,他自己怎么看这件事的事情,他有他的观察。最后我们再来看通膨,美国刚刚公布 3 月份的通膨的数字,公布完之后美股大跌 1276 点,那两年多以来最大的单日跌幅,它都告诉你,市场上面觉得为通膨并没有退,那美国的通膨仍然很严重。那国际上面的通膨的数字起起落落或许有一些,但基本上都在历史的高点的附近,那通膨到底接下去会怎么走?好?来,今天来先介绍我们的来宾,那今天在线上的是人在美国的王孟源,欢迎。



王孟源 04:30 

很高兴再上你的节目,跟大家聊聊天。你刚刚提到的开场白其实让我有点感触,就是怎么做信息的分析,尤其是在当前这个国际社会是在谎言帝国的统治之下,然后我们又要去看,尤其是俄乌战争这种先天就有战争迷雾的事件,怎么样去仔细的过滤错误的讯息或者虚假的讯息,我想既然你已经提起来,那我就先稍作解释,然后我们再去谈整体好了。其实你如果是我博客的老读者就会知道,我常常说我在一般的预测跟分析上可以达到大概 90\%的正确率,但是有一个例外就是军事议题。那这个军事议题为什么这么困难?是因为他先天就有保密的需要,而且先天就正正当当地需要造假。你看像这一次俄乌战争,俄国的国防部虽然是不轻易的撒谎,但是真正有需要的时候他们还是会撒谎的。比如说在开战前,他们就说我们绝对不会开战。



唐湘龙 05:55 

没错,讲了好几次。



王孟源 05:58 

对,然后那个,然后再加上另外一个问题,就是我们这次分析这个俄乌战争,所有的主要消息来源都是俄文,那这个你语文的熟悉度其实是你做这些分析的时候一个非常重要的考虑因素。因为你从那个语文的流畅、写作的结构,还有他对事实证据的论述,或者是那个作者本身的记录都可以看出,是不是可以。那你如果这个不是你的母语,或者是你对他不够熟练的话,这些很重要的第一步过滤你就做不到。



王孟源 06:40 

那这样说来,是不是这次俄乌战争它既是军事,而且又有语言的隔阂,是不是就是我们无法跨越的鸿沟?其实刚好相反,因为它有一个很特别的因素。我想跟大家解释一下,其实它是我所知道主要的过去几年来的主要军事议题之中最容易分析的,你只要用心就很简单可以分析的对,这个原因是什么呢?因为乌克兰它那一方基本上就是公然撒谎,他们讲的话 10 句里面有 9 句是造假的,跟自由时报一样。



唐湘龙 07:20 

哈哈哈哈,你那么有把握吗?为什么?可是我看泽伦斯基或者是他的参谋总长讲话那个讲得斩钉截铁,很有自信。



王孟源 07:33 

他们基本上Zelensky 的专责就是撒谎公关。事实上,他的最高级助手,他的所谓的国安助理,在我的博客上有一个link,可以去看他的视频。他公开的说,我们现代乌克兰建国的根基就是谎言,因为没有谎言的话,我们马上就土崩瓦解。那乌克兰那边是绝对都是撒谎。



王孟源 08:05 

噢,我刚刚提到自由时报,其实我的老读者也都会知道,我当初决定会想要写博客,就是因为 2013 年我在很久很久没有回台湾之后又回家省亲,因为我爸妈身体不好了,不能够再来美国跟我住。所以我后来决定,既然他们不能来,我还是回去看看他们。那去的那个暑假,因为我在离开台湾已经 20 多年了,所以有点陌生。那我住台南嘛、台南乡下,那几乎亲戚朋友都是看自由时报,我看得目瞪口呆,因为我在美国当然是习惯了假新闻,但是,但是那个密度跟强度还是把我吓得大开眼。



唐湘龙 08:57 

不好意思,越你讲的让我觉得太好笑了。好,你继续。



王孟源 09:04 

所以是我回了美国以后一直觉得这样子下去不行,所以我才自告奋勇的要决定,考虑了好几个月,然后最后到 2014 年开始写博客。所以在你们现在看到我、会知道王孟源这个人、会有这个博客,其实是因为我被2013 年那一次被自由时报的震撼教育。所以,那好,那我现在刚刚讲到乌克兰那一方的资讯完全不可靠,那俄国这边他们是有意的只给你很枯燥的、很没有什么好分析的一些信息。我去看那个他们的国防部每天的公报,那真的是,唉,干枯的跟柴火一样。所以事实上虽然有很多小道消息这样飞来飞去,但是你在乌克兰方有意造谣,而且他们会故意去假装是俄国人然后来造谣,这个情况下很自然的就涌现了出一个自清的集团,叫做Rybar,他们的简称叫Rybar,那他们自称是一个自告奋勇的一群志愿的俄国记者,那他们在过去的 6 个多月很快的就建立了很好的信用。第一个是因为他们几乎所有的话题都可以评论,第二个是他评论的时候很显然不是有主观偏见的,就是他虽然是从俄国人的方向来角度来看,但是如果有人说俄军今天打了大胜仗,在你们这个村庄,大家到最后等的都是要Rybar出来,说往往Rybar是隔了一两个小时以后跑出来说并没有完全占领,只不过是占领一个角。那然后大家就说,噢,好,然后事后都证明这是正确。



王孟源 11:10 

所以因为这个因素,因为Rybar这个因素,所以俄乌战争的讯息其实很简单,它就有一个瓶颈,你不管是谁要找新的信息,他的第一手资料都是来自Rybar。那所以你不论是华语世界或者是英语世界,或者其他语文的世界,你真正要追踪这些细节,最终的那个手段都是靠着自己阅读俄文,或者是去看Rybar的英文Twitter,他的那个 Twitter 的发表发言。那我个人追踪的是一群主要是欧洲的评论员,他们是德国人、白俄罗斯人、奥地利人等等,希腊人等等,但是他们基本都是都会讲俄语,然后他们基本都是去看了Rybar的的这些报告之后自己总结,然后整理出来。



王孟源 12:21 

那因为有这个瓶颈,所以你事实上都不需要去当第二手的人去看,直接去看。你可以再去看这些,你自己可以当第三手来看这些第二手的资料。因为为什么呢?因为正因为只有 Rybar 一个管道,一个可信的管道,而且他这个管道已经打开知名度了,所以不可能有什么独家新闻。所以你反而不是不需要说去过滤什么,你很简单的就是追踪几个知道那个是比较诚实的评论员,然后看他们之间有什么重复,就可以很简单的看出哪些是他们从Rybar拿来的事实,然后哪些是他个人的分析意见。那所以这对我来说是可以说再方便不过了。



王孟源 13:19 

那一般来说,一般的话题你做分析的时候是没有这么方便的,因为你的第一手资料本身就会很可疑,而且会有很多很多的第一手资料,再经过转述之后,第二手、第三手,更加的xxxx。所以在一般做研究的时候,一个很重要的是你尽可能的去直接找第一手资料,然后你要怀疑第一手资料,然后想办法的想办法过滤或者是还原事实。但是在这一次的俄乌战争中没有什么太大的必要。



唐湘龙 13:57 

为什么?



王孟源 13:58 

就是因为这个 Rybar 这个瓶颈,因为 Rybar 本身就已经具有:第一个它很完整,它基本上所有的战争细节它都报告。第二个它本身就很严谨,那其他的第一手的、那个当地的报道直接出来的资料跟他的质和量都完全比不上。你如果两有冲突的话,照逻辑来说都必须要先相信Rybar,那你就不必要去管其他的第一手。



唐湘龙 14:31 

那因为今年的从其实几个月前,当我看到当时的泽伦斯基辅不断的放话说8月要反攻,那我们也在看了,那最少从西方的叙事或者战场的变化来看,8月确实乌克兰去确实发动了一波的反攻,而到这两个礼拜左右的时间,在乌克兰为主的西方叙事里面,它就说就是重新夺回了 1000 平方公里、 3000 平方公里、 6000 平方公里。



唐湘龙 15:04 

到最近的叙事说夺回了 9000 平方公里,将近1万平方公里的土地,同时在哈尔科夫这个北方的主要的省份已经把俄罗斯给赶出去了,那战略要地Izyum收回来了,那在南部的这Kherson地区也开始在组织反攻,那也看到了泽伦斯基越来越强硬的讲话,而在俄方方面来讲,除了偶尔的这个就是说零星的这些,就跟您刚提到的那种很枯燥的新闻回应之外,俄方的讯息反而现在开始逐渐被怀疑。因为战场上面看起来他是在撤退。那到底现在的战场的实况如何?



王孟源 15:45 

其实双方都说了一些实话。,比如说乌克兰说他们占了几千平方公里,他们的确占了几千平方公里,俄军是不是一路溃败?是一路溃败了。但是俄国人他们说我们并没有什么损失,他们的确是没有什么损失,然后他们说我们事先就计划要撤退,这个不是完全精确,可能让你误解他们是故意的一个陷阱,不是陷阱,但是这个事先就准备有撤退的计划,也是在某个角度来看是实话,那这个我可以待会我再完整的解释给大家听。不过事实上是他们双方都各自说 2 分实话, 1 分谎话,那各取所需。为什么可以这样子?因为他们双方所寻求的东西是完全不冲突的。你也许说两军对战怎么会不冲突?他们都在追寻胜利,但是俄军我即使在过去 6 个月已经反复解释过了,他追寻的是杀伤,他并不在乎土地。



唐湘龙 17:01 

这个这里确实重复讲过。



王孟源 17:04 

对,然后乌克兰他追寻的是什么?公关。那做公关的时候这个报道是报道土地占领比较容易吸睛,还是报道伤亡比较容易吸睛?他根本就不在乎伤亡,因为伤亡这种东西是随口说了算,没办法证实的。但是你说我的部队占领这个村庄,这个是大家有目共睹,谁也不能否认,对不对?所以乌克兰这一次作战的目的就是占领土地这个,因为乌克兰现在别说是军事了,他们的这个人员基本上战前的那一批正规军已经快要消耗殆尽了,现在都是新动员的,现在它不但所有旧的武器也都消耗殆尽,像现在用的都是北约送的武器,它甚至连一般的文官官僚国家的正常运作,每个月 50 亿美金的资金都要靠美国跟欧洲。这个唯一能够激起这些捐款热情的就是你必须要有胜利的战报。



王孟源 18:16 

这个胜利的战报,那我又刚刚讲过,占领土地的胜利战报要远比所谓伤亡的要重要的多,因为你这个伤亡的结果随便撒个谎就可以否认了,所以事实上双方都可以各取所需,它并没有什么可以真正冲突的地方。那我现在开始正式的回顾这次的作战的过程。然后你可以看看跟新闻的报道是不是大部分吻合,只有细节上这个实情并不是很多主流媒体所给你的印象。这一次乌克兰进行的反攻其实是从5月就开始广告了,已经广告了四个月。



唐湘龙 19:07 

没错。



王孟源 19:09 

那他们为什么知道在8月底9月初可以反攻,是因为他们有一批新的征召的兵送到英国跟波兰去训练,就是我几个月前也谈过,就是他们的原本的精锐的野战正规军。我这里说正规军就是有能力野战,并且进行兵种协同,能够跟炮兵、战车,然后跟空军协同作战的。那乌克兰的民兵基本上就是拉夫过来,有的训练了两三天,有的根本就没有训练,直接一把枪交给你。那这样的人除了守战壕之外没有任何能力,你教他说往前进攻 10 公里,他们都会迷路的。



王孟源 19:59 

那他们原本的正规军有大概8万的精英,就是在东乌前线原本是准备要进攻的,一旦仗打开来,他们基本上就被定下来了,偶尔有一两个旅可以往后撤,但是这个整个部队的整体xxxx(推测为上前)。因为你一旦撤多了,那个俄国人就会乘虚而入,然后席卷整个战线,那是不可承受的。而且你要做机动防御也不行,因为俄军有完全的空优,我想我上个月才又解释过,这个他们能够撑到这么久,其实很大的因素是靠着工事(据“上月解释”,显然指工事非攻势。上月:乌军靠工事战壕撑着,一跳出就被轰),尤其是已经八年来, 2014 年以后建立的工事,所以你要机动防御的话就必须要放弃工事。在俄军有压倒性的炮兵优势之下,这是找死的。事实上 2014 年他们已经有一个很惨痛的经验,有两个机械化营在机动的过程中被俄军的炮兵打了,连一个敌人都没有看到,两个营就死伤大半,所以根据那一次的经验,这一次乌克兰从头就没有想要打机动战。



王孟源 21:21 

那既然原本的这精锐的受过几年训练的专业士兵都在东乌,那他准备这个反攻其实是靠着英国跟波兰帮他重新训练的部队,那这些实际上数字有多少,这个是没有办法确定的,因为他们不会公布,公布了你也不相信。 Rybar是唯一一个可靠的讯息。那他也不可能知道乌克兰那一边送了多少人,顶多就是估计一下,估计一下也是他的估计,那也是你估计你的,我估计我的,主观意味很严重,所以这个时候你就知道这些数字完全都是大家凭推理、逻辑跟经验来估。所以大家就英文是说 take it with a grain of salt,就是带点调味料来吃,不要把它的生吞活剥,认为百分之百的是纯的,那从这一次事后的复盘来看,他们大概是训练了 3 万多人。



王孟源 22:34 

那为什么我说3万人?因为他们第一波就是第一个礼拜是先在Kherson那个地方投入了 5 个旅,那乌克兰的旅一个旅是 3000 人,那五个旅就是 15000 人,那这五个旅基本上就是被拒止在防线之外,然后关起来打。唯一的例外是有两个旅是突破了那条 Inhulets  river,就是那条河,一条小河,然后很奇怪的是别的地方都没办法突破超过 3 公里,就只有这个要过河的,他们建了两个浮桥,过了河,很轻松的前进了 10 公里,然后忽然这个口袋就收紧了,他们到现在还被关在那个口袋里面,所以基本上第一波去打Kherson的,那个真的是踢铁板了。就是我其实上个月也已经预期、解释过了,就是他们进来的时候要挨一波炮火,准备要逃出去的时候还要再挨一波,然后隔了一个多礼拜,忽然的在Kharkiv-Izyum 这个方向有了新的攻势,那这个就是势如破竹了,也就是现在大家纳闷的事情。



王孟源 23:58 

那我跟大家解释一下,这个投入的时候一开始只有 3 个旅,但是后来证实后面还有跟着的预备队,有人说是两个旅的预备队,有人说是三个旅的预备队,那这样子就可以算他这一次反攻作战,乌克兰投入的一共是 10 个旅,或者是 11 个旅,那就是 3 万多人。那当然这些旅是所谓的正规军,就是能够机动作战的部队,他们还有很多民兵,就是刚刚拉出来的炮灰。那个是负责是你如果有前进的话,尤其是在Kharkiv那个地方,你如果进展顺利的话,后面的占领的责任是由他们来负责,对不对?他们是负责善后。那至于为什么俄军会一触即溃,是因为他的那个军力实在太悬殊了。这个他们进攻的这个方向是在从西往东,越过河去攻击Izyum北部的那个森林地带。那个地带我先跟你讲一下,我现在讲的那 6000 平方公里的地方,刚好是乌克兰北边人口密度最低的一块地方。Izyum,好像大家谈的耳熟能详。但是他的战前的人口在开战之前的人口也只有4万人,想想看4万人是多小的一个镇。



唐湘龙 25:36 

大概就台北市的一个里再大一点点而已。



王孟源 25:42 

对,然后开战之后往东边跑往西边跑的,看你的政治倾向,跑的都快光了,根本就没有几个人。顶多就是几千人留下来,其他的地方都是小村落,而那是因为那个地方刚好是那个乌克兰最主要河流上游的地方,有很多森林,还有那个河道弯曲留下一些旧的湖泊,所以是一个水箱,不能说是沼泽啦,但是有很多河流跟湖泊的地形,然后这个水边都是茂密的森林。事实上俄军开玩笑把一个其中最大的一个森林叫做 Sherwood Forest ,就是那个罗宾汉故事里面。哈哈哈哈



王孟源 26:43 

照理说,你说像这种森林地形还有很多水流,它是易守难攻,但是这个易守难攻的条件是你有步兵在那里,就是他能够抵消对方的装甲兵,但是他也能够抵消对方的炮兵,因为炮兵没办法看穿这个森林的树叶,他们比较容易掩蔽。那这样一来俄军的真正的优势就是炮火跟空军的优势就大致抵了,然后俄军又一直没有真正去花攻势,然后等到你的面临的攻势是 8: 1 或甚至 10: 1。现在有很多消息出来,估计大概是实际上的比率是 8: 1 和 10: 1,就是当地守军只有 1000 多人,而且这 1000 多人不是重装的正规军,而是武警、



王孟源 27:46 

警察部队就是所谓的警卫部队,完全轻装的只有机关,顶多只有机关枪的,那这种部队基本上就是去维持治安,然后做警惕,这为什么会这个样子呢?基本的原因还是当初一开战在3月跟4月,我不记得是在你这个节目上说的还是别的地方说的,就是投入的兵力实在太少。



唐湘龙 28:15 

没错,这个你已经谈过几回。



王孟源 28:18 

对俄军投入的正规军一般估计是不超过8万人,总投入连民兵跟那个佣兵加起来最多的时候有 19 万,现在大概只有 15 万或者 16 万左右。



唐湘龙 28:35 

就是孟源这地方我打个岔,因为这是我看不懂的地方,就是俄罗斯从 2 月 20 号开战,因为战争是俄罗斯设定的,是俄罗斯是发动攻击的一方。但是就像战场上面所看到的,孟源从一开始推断的是对的,就是俄罗斯的整个的部队是不可能发动全面占领的,他也没有那样子一个军事的部件,那打到现在这个规模了之后,他也不可能不知道乌克兰在正在酝酿一波的反攻,他一定知道他要酝酿一波的反攻。而这个反攻的位置,老实讲,以乌克兰的地形或者现在两军交战的形态来讲,你要去骗俄罗斯也不太可能。



唐湘龙 29:19 

但是我不懂的是,当乌克兰的反攻开始了之后,为什么所经之地感觉上面俄罗斯好像没有什么准备,没有什么主导?甚至我们看到的国际新闻,现在白俄罗斯会不会进场不知道,那这个也看到车臣的这个就是说这个领导人也表达了一些的不满,觉得那个是领导的问题,我们也看到甚至于连朝鲜都说他们要卖武器,他们要派军进来,连伊朗的就无人机也都进来了。因此他就形成一个印象,就觉得俄罗斯好像快不行了,连这个外围的帮手都准备要进来,连一些像是朝鲜的武器装备,除了核武器之外,你可以判断它的装备不会怎么样的,好像都准备要进场了。为什么会是这样的一个俄罗斯?



王孟源 30:12 

首先我先谈一下这个武器,它跟伊朗买的唯一的是那种长航时长察打一体无人机。



唐湘龙 30:18 

无人机。



王孟源 30:19 

这刚好。



唐湘龙 30:20 

是俄罗斯需要的。



王孟源 30:21 

这刚好是俄国这一次最需要,我从二、三月的时候就讲,他们这个装备的最大弱点就是长航时的,然后后来美国送了M777 还有HIMARS出来之后,这个需要就更迫切。原本俄国人还想着撑,咬着牙撑一撑就过了,结果你想想看,这个HIMARS的射程 70 公里,将来可能还送射程更远的。你去反击这种超长程的、机动性的火力投射武器,最好的方法就是察打一体的无人机,看到了就直接一枚小飞弹下去,他跑都跑不掉,没有时间跑。所以照理说世界上最先进的察打一体无人机三个国家就是美国、中国跟以色列。



唐湘龙 31:10 

没错。



王孟源 31:12 

但是我想中国不愿意明着参与这个战事,就是他的这个外交战略,所以退而求其次去跟伊朗拿,我觉得是合理的,因为它就是单向的,刚好是一个最大的弱点,然后刚好又遇到HIMARS这个需要无人机去克制它。那至于北韩这个消息还没有确认,但是你就算真的确认的话,他买的也不是武器,而是弹药。我想我也说过了,这个俄军消耗的弹药的数量真的是吓人的,这个是已经达到了二次大战同一个级别,就是他每个月是几百万、几百万的在打。那你想想看这个几百万发这个是什么意思?美国的陆军刚刚招标,要重新补充他们的炮弹,因为他们炮弹送去给乌克兰的话,因为现在都 M777 送过去了。他们认为美国全国的产能是多少?每个月是一万两千发,每个月,这是美国全国的产能一万二千发。但是俄国现在的使用率好像是一个月就一两百万发,对不对?那你说这样子,我想俄国的产能当然是要比美国高好几倍了,但是那个。



唐湘龙 32:54 

仍然是不够的。



王孟源 32:56 

还是不够用。那你去跟北韩这个便宜又好用,炮弹这种东西没有什么技术嘛。



唐湘龙 33:03 

没错,没错。



王孟源 33:04 

就是产能对不对?我觉得是说的过去的,大家不要把它太在乎,这个只是权宜之计,仗打起来的时候你总是会发现一些小东西,你这个没有什么太大价值,但是仗打起来以后忽然就有需要的时候,你那基本上哪里方便就从哪里找嘛。我觉得它的真正的问题是普丁从一开始就没有投入足够的军力,而事实上在开战之前,我还有全世界所有值得听的评论员都说仗打不起来。这个分析其实有两个路线,一个是从外交政略路线来看,就是你事实上应该尽可能的再拖上两三个月,看看能不能让德国跟法国回心转意,那是我那时候这个有明说了。



王孟源 34:01 

另外还有一个军事的原因,就是当时俄国朝前部署的那个部队就是十几万。你要打乌克兰的话,大家依正常的军事常识来估计十几万是远远不够的,你至少也要有四十几万才能够正经的打一场漂亮的战争,所以百分之百值得听的、认真的、有专业知识的观察员越是懂军事的,越是认为打不起来,因为俄军没有那么多部队在前面。结果他打起来了,打起来以后果然第一个月是很吃亏,对不对?然后现在问题是第一个月的时候他试图去威胁基辅,那这个当然是自找苦吃,后来因为要以战逼和。那事实上乌克兰也不可能同意和谈,因为他们那个领导人全都是被昂萨控制的。后来他就第一阶段改成第二阶段撤回来,撤回来以后他这个Kharkiv没有完全撤,即使Kharkiv那个时候他们就已经考虑要撤,但是Kharkiv没有完全撤。因为为什么呢?他觉得他们的这个 SMO 他们的所谓的 special military operation 特别军事行动,他的这个核心的目标、地理目标是什么呢?是Slovyansk跟Kramatorsk。



王孟源 35:40 

这个是在北Donetsk州, Donetsk州北部的两个大镇。这个是我刚刚提到乌克兰东乌军团的军团司令部所在,也就是他们最后的防线。那个时候俄国之所以会没有从Kharkiv外围全部撤掉,反而是派兵往南占领了那个小镇Izyum,就是这次被光复的那个小镇。原因是从Izyum,你可以从另外方向去威胁那两个重镇。就是 Izyum虽然是在Kharkiv,但是它是隔壁的,所以从那边你就可以多一个方向去威胁他。那这里的问题是从4月占领 Izyum 之后,4月、5月他们真的尝试着从  Izyum 这样子中央突破、四面开花,结果被挡住、打得不漂亮。然后到了6月、7月改成从Lyman,Lyman是比较靠东,就是直接在Kramatorsk北边的正北,而不是从西北下来,是从正北下去。结果就是在我刚刚提到那个 Sherwood Forest 被挡住了。因为乌克兰真的有的是兵,反正我刚刚说过这些刚刚这动员起来的人别的不能干、那个守战壕是可以的,对不对?就把那他这个叫三毫五步, 5 步一个、 十步在一个大的。那这样子就变得非常困难,尤其在森林这里面你很难使用重兵器。所以俄军原本在Izyum是有好几个合成战斗营,然后后来重点转化Lyman以后,Izyum还是有驻军作为助攻。



王孟源 37:50 

但是我们现在回头去看,在7月的时候,这个这个从Lyman向南的攻势受挫之后,其实他们就已经开始重新部署了,就是这些部队就被抽调,抽调以后就是唱空城计,因为我们这一次事后复盘,我刚刚讲过,整个Kharkiv几千公平方公里只有 1000 多个兵,而且是警卫兵。这个唯一的解释就是俄军的确是觉得这个地方食之无味,弃之可惜,那就干脆把这些,反正要打也定不了,那就把这些兵调到更重要的Donetsk前线去,然后派一些杂兵在这边守着,很少的杂兵。



王孟源 38:44 

那这个杂兵可以怎么样?如果乌克兰继续做袭扰,就是从4月Izyum 被占领之后,其实他们一直通过从西方的森林试图对Izyum做袭扰,就是小部队一个班或者是一个伍这样子做袭扰。那他们觉得这些警卫部队应付这些袭扰够了,但是他们可能没有预料到的是这次乌克兰的计划,因为他们的重点是要占领土地,为了占领土地去做公关。



王孟源 39:24 

所以其实在Kherson那个佯攻,就是只用了一半或者略低一半的兵力,大半的兵力保留在一次投入,投入到Kharkiv去了,那投入到Kharkiv去,那当然 1000 多个兵是挡不住1万多个机械化的重装部队,对不对?所以他们就掉头就跑了。当然这 1000 多个不完全是警卫部队,有一些东乌民兵,但是你后来事后复盘觉得伤亡顶多就是 300 人,你一开始就只有 1000 多个你能伤亡多少,对不对?然后我也曾经考虑过,他们为什么,其实 1 万多个进攻也不算是太大的一个。



唐湘龙 40:12 

没错,也不是挡不住,何况如果哈尔科夫就在俄罗斯的边境上面,他如果要做,资源是办得到的。



王孟源 40:22 

那你真的要挡的话,先投入一个旅去把它阻击一下,然后在两三个旅这样子上面。但是问题是这一次的这个 s m o 过去 6 个月,我们可以看出他们作战的最高标准,是什么?就是我刚刚已经说过的伤亡比。他们最要求最重要,俄军看的最重的就是伤亡比。,他们这一次决定如果逃跑的话,我损失 300 人,但是我可以用炮兵跟空军,创造乌军3000 人以上的死伤,对不对?甚至现在的估计是至少 3000 了。我说的是死伤,不是死亡,死伤,死亡跟重伤加起来有,从 3000 一直到 6000 都有。我说 3000 是最低的估计。那你这样的这个伤亡比是 1: 10,但是你如果是投入野战兵力,紧急投入的话,你的伤亡比就达不到 1: 10,那他完全就不在乎这些土地。我个人觉得你不在乎土地可以,但是你这些占领区的百姓,可能已经跟你俄军合作。



唐湘龙 41:36 

没错没错,这个是个问题。



王孟源 41:38 

那你这样,但是再反过来讲,有两个可以为他开阔的解释。第一个是当地原本就没有什么人口,我刚刚提过Izyum也只有几千个老百姓,对不对?第二个是跟南部的Kherson不一样,Kherson是一开始的时候也没有说要把他们俄国化,但是到了4月初和谈无望之后,他们就开始整体俄国化全部都换成卢布,全部都换成俄国,而且准备公投,但是他并没有在Izyum跟Kharkiv做这样子的事情,所以他们可能是觉得应该没有什么太大的风险,他愿意承受这些百姓被乌克兰人、乌军占领的这个危险。



王孟源 42:35 

当然我们说到底就是,Putin为什么不愿意多派一点兵?就是我已经反复的,我从2月讲到3月,然后一直讲到现在,就是我一直觉得他可以再多派点兵。那因为你即使不完全投入,拉着当预备队也可以有减少伤亡跟减少风险的好处,对不对?那像现在这样子,整个战线捉襟见肘,次要方向必须要唱空城计,其实是因为他在用兵非常吝啬的一个结果,必然的结果,这个你那个战场上的指挥官是没有办法,巧妇难为无米之炊,这是没有办法的事情。



唐湘龙 43:30 

那我有两个问题就是请教孟源。那也这个话题我们就可以先告一段。一个就是乌克兰的这一波的反攻什么时候会开始遭遇到俄罗斯比较正规的回击,那为什么你判断就是说俄乌战场的形势其实并没有改变,大家看到的都是表象。第二个就是说扎波罗热核电站被攻击这件事情,他有什么特别的战术意函。



王孟源 44:05 

先回答你的第一个问题,他们基本上,我说过这一次虽然俄军没有事先预期到这个攻击会是有 5 个旅或 6 个旅来攻击,所以有点仓促。但是他们的确原本就不在乎,原本就计划考虑,就是说没有确定要撤,但是考虑要撤。所以一旦面临乌克兰的大军过来,他们大概开了一个小时的紧急会议。就是我刚刚讲的,你有两个选择嘛?一个就是那 1000 多人全部赶快撤回来。尽量减少伤亡,反正让对方进来的时候,我用炮兵跟空军去打,跟地面上有没有步兵一点关系都没有。那第二个选择是你派紧急预备队,然后去跟他做地面战,那这样一来你没有攻势可以依托,对不对?完全就是进攻对进攻,这样子做运动战,这个的时候的伤亡比就很难看了,对不对?不可能好看。



王孟源 45:10 

所以他们选择的,所以我一开始的时候说过,俄国人说他们这是按计划撤军,既是实话也是谎话。实话是他们的确是有这个计划,而且事后也按照计划做了,但是这个时机的选择不是他们选择,那但是因为有那个计划,所以他们就很简单的退到更大的一条河后面,所以基本上现在乌军一旦前进到这条河,然后想要渡河去攻击Lyman的时候,就是又跟那个Kherson一样是踢到铁板了,就是完全都没办法前进了,就是这边依旧是次要方向,依旧是防守态势,但是正规的、认真的要守住阵地的防守态势,不是只有零零星星的所谓部队那种。



唐湘龙 46:13 

我们现在已在荧幕上面所秀出来的这张图,刚刚就是王孟源提到的更大的那条河叫叫Oskil River,他在Izyum的东方的位置,所以如果接下去乌克兰的军队再往前推进的时候,大概就会真的碰到了俄罗斯在的地方,所部署的包括以Lyman为中心的主力部队大概就在这附近了。



唐湘龙 46:38 

好,那前面呢?你看到的哈尔科夫的部分,看来为什么一直有种声音觉得俄罗斯似乎是在设一个空城计在请君入瓮,这是为什么市场上面就舆论当中一直有这样怀疑的原因,就是你乌克兰反攻的也太轻易了,而在战场当中的交手的这样一个炮火这么的低,而伤亡又这么的少,这不合理,所以认为那个是俄罗斯本身的战术思考的问题。好,最后就是我看不懂的就是说为什么大家会对核电站?这全欧洲最大的核电站,乌克兰毕竟是发生过车诺比尔的,对于核电站的风险核外泄,他们其实是比任何国家都有切身之痛的。可是在战场攻防之后竟然会去攻击核电站,他有他特别的用意吗?。



王孟源 47:31 

Zelenskyy,你觉得他在乎这些事吗?



唐湘龙 47:33 

对,这我从最坏的情况我就只能这样想,就是他根本就不care。可是他是那个很不可思议。



王孟源 47:41 

我跟你讲Zelenskyy唯一的任务、唯一的职务,就是负责搞公关、做表演。他是一个演员,现在还是演员,他从来就没有从演员的生涯退休下来。那这个为什么会专注在这个核电站呢?我其实上个月已经讲过了,就是因为核电站特别容易吸睛,特别容易吸引注意力。那另外一个因素就是这个国际原子能组织,也是一个完全由欧美控制的组织,去看他们发的那个报告,你如果去看那个英文的原文,像我刚刚我半个小时前讲的,你尽可能的去找第一首资料,然后可以字里行间去看它的这个用语怎么样的扭曲,那就可以看得出那个写作的人的立场。那他们你去看他们发表这个报告,那真的是所谓的contortionist,就是扭来扭去的,就是不讲这个炮弹是谁打的。



唐湘龙 48:49 

他连讲都不讲,我觉得这个太不合理。那你原能总署在干嘛呢?你人都到了现场,秘书长都讲话了,炮弹应该很容易判断是谁打的。



王孟源 49:03 

因为他们也是欧美的附属组织,所以这一次是一个政治任务。他们去那边其实是策应。那个他们附带的那一个整个旅行团里面就渗透了一大堆那个乌克兰的特种兵,被抓住了嘛?那些特种兵假装是随行的记者被抓住了,然后与此同时有 2 个连的特种兵想要渗透,也是被俄军守株待兔在那个河面上给打成蜂窝。



王孟源 49:37 

那后来事后他们还在继续的尝试,为什么呢?因为如果突袭得手,那些国际原子能组织的人就是保证俄国不能反攻的棋子,对不对?你如果,如果他们能够侥幸占领了核电站以后,又是一个大公关胜利。对,而且俄军没办法去反攻,你怎么反攻?那边有一大堆那个际原子能组织的人,对不对?那他们就是人盾,对不对?所以这是他们的如意算盘,而且我可以告诉你,人家事实上已经都发掘出这是谁帮他们策划的,是 m i 6 跟英国的国防部帮他们,不是美方策划的,就是是英国人策划的,因为他们事后发现那些被打死的特种兵是英国人训练的。



王孟源 50:41 

所以我希望这个回答你的问题,我想基本上我们时间也不多了,我尽快solo。唯一一个需要再谈的话题,有关这个俄乌战争需要再谈话题是未来会怎么发展。那这个不论怎么样,这个这件事情很难看,尤其是整个俄国国防部高层还有普京本人都在远东嘛,先是看一个演习,然后接着他去参加这个上合峰会。那他人不在那边,你如果是民主国家的领袖,绝对不敢做这样的,因为这个民意调查一定就崩了。但是在俄国的话,他们国内还是有反对派。我想我几个月前也讲过,普京其实在俄国是温和派、温和保守派,那他在俄国国内的主要反对派都是激进的国家主义者。那所以这件事情发生了以后,他们这个Duma就开会,开会以后有很多那个很多幕后的折冲。



王孟源 51:57 

那一时就有人有谣言说他们会开始宣战,会动员。我可以跟大家明确的说,这个宣战跟动员的几率非常非常的小,因为这个你看过去这六七个月铺垫,普京固然是在军事上我认为过于保守,但是他在外交上对欧洲做的一件事情我觉得是很正确的,就是他始终没有对欧洲的百姓作出敌对的态势,因为一旦一个国家进入战争动员、进入战争状态,老百姓就忽然的会进了一个断层,能够承受更高的牺牲。你连命都要投入了,那其他的那个生活的牺牲都那算什么?所以普京的很重要的一点是要保持欧盟在和平状态,它的社会必须要保持在和平状态,光是为了这一点,俄国就不应该动员,更不能宣战。,所以,但是他其实不动员也可以有志愿兵,也可以有民兵,增派5万人甚至 10 万人都是可以的,那这时候还有人建议说也可以把他这个所谓的 special military operation,他的这个任务就是帮助那里的东乌共和国自卫,把它提升成反恐。因为事实上乌克兰已经有好几次攻击俄国本土的记录,那是真的是名正言顺,可以说是反恐。



王孟源 53:46 

那你提升为反恐以后是什么意思?就是你可以更名正言顺地去打击基辅的政府建筑或者其他的一些民用的基础设施,比如说电站、水站。我觉得这个比较合理,但是也不是完全必要,就是事实上他们在这个Kharkiv大撤退之后,马上就把这个乌克兰的电厂全部炸了一遍,对不对?然后今天消息传出,他们把那个Krivoy Rog,旁边的一个水坝给炸了。然后那个Inhulets River就涨起来。我前面不是提到在Kherson前线只有两个旅是渡河去进攻结果被包围成一个口袋吗?他们那个俄军本来把那两个浮桥给炸了,但是因为Inhulets River的水位很低,所以你基本上可以赤脚过河,所以炸掉那个浮桥没有什么太大的意义。然后就很好玩的是今天他们把那个上游的水坝给炸了以后,那个水位一下上升了。那这下那两个旅真的是变成孤军,都是很有意思的发展。



王孟源 55:12 

不过言归正传,就是我预期普京不会大幅的升级,不会有宣战,不会有动员。它有一点可能,有一些可能会升级为反恐作战。至于是征兵吗?我个人认为应该征兵5万或到 10 万,但是目前还没有明确的迹象,可能他只征兵两三万也有可能,就是他们刚刚好成立了第三军,这个是原本就准备,第三军大概也是两三万人。



唐湘龙 55:51 

好。现在已经到了9月的中旬了,天气开始变冷了。那天气会决定在整个的欧洲东部的战场,从二战、一战以来都一样,就是天气会是一个非常重要的因素,所以在这个时候去征兵看起来也不合理,因为接下去天气会使得双方面的作战进入到了一个经验值可能胜于一切。那过去大家普遍认为就是说天气是俄罗斯的好朋友,所以当天气开始变冷的时候,对俄罗斯来讲,它的经验值会加权、会提高。这个下个月之后,下个月我再跟王孟源连线的时候,那时候天气大概就已经相当冷了,那个时候战场当中态势会变得更清楚。



王孟源 56:40 

还没有要两个月,差不多要到时候11月那个树叶都掉光的时候,那些森林的隐蔽效应就没有了。



唐湘龙 56:48 

那时候就更能够看得看的时候出来。好,最后我们还要再花点时间,我们来看,因为现在到了距离美国的期中选举剩下不到两个月的时间。我们看到在整个民调上面来讲,我注意到的美国的民调,第一个就拜登的民调从 低 档上来了,虽然还没有过半,但大概到 40\% 几的支持度,表示民主党的基本盘是在回流的。第二个就是说在现在可能改选的众院跟参院里面,那现在看起来民主党可能会失去众议院,可是他对参议院的主控权可能会扩大,但是这些都是选举的表面,而重点是这场选举里面我们看到了美国国内的意识形态的尖锐的对立,似乎更强烈了。那我从前两天我看到马斯克公开的抨击,这个就是说觉醒主义的这个 wokeness,他在抨击这样的一个觉醒主义,他认为这个是美国建国以来,他的意味着美国建国以来最大的可能会伤害到美国立国精神的乱流。你怎么看这个事情?



王孟源 58:03 

噢,其实这个话题很大。我们没有太多时间。



唐湘龙 58:07 

没关系,我们花个 25 、 20 分钟都可以,您请说。



王孟源 58:11 

我只讲其中我认为最重要的一点。我之所以会想这件事情,是因为最近这个 Amazon prime 出了那个 lord of ring the ring of power,就是魔戒。



唐湘龙 58:33 

魔戒,魔指环。



王孟源 58:34 

魔戒故事的续集,是有史以来最昂贵的电视剧集。然后 HBO 也刚好推出权力的游戏 game of thrones 的前集,那这里面其实在过去 10 年的美国的影视界,是这个wokeness还有白左政治正确的重灾区,所以你从这里面可以看出它的危害。可以看得最清楚,最有名的例子就是星际大战star war Lucas那个工作室,迪斯尼的那个Lucas工作室,他们当初那个Lucas,George Lucas把这个把他的公司卖给迪斯尼的时候,他们把原本的一个中级主管Kathleen Kennedy 提拔成为总主管。所以后来的这个续集,最新的那三集续集,也就是口碑极差,把整个知识产权给搞砸的那三个都是Kathleen Kennedy搞的,Kathleen Kennedy就是一个典型的 wokeness 的女权战士,所以你可以看出那个星际大战的续集的那个主角ray,那个女主角,是所谓的叫做 Mary Sue 天生无敌,就是智商 300 然后体力超过所有奥运冠军。



王孟源 01:00:22 

就是没有任何的缺点,全身到下没有一丝杂毛的那种,那这种故事完全没有戏剧性,违反最基本的戏剧常识、常理,但是他们为了政治正确,硬是这样强推,强推的结果就是把不止把老观众给得罪了,新观众也得罪了。那而且这样一来,这还不只是艺术创作上的失败,你再从生意的观点,迪斯尼也是自己摧毁自己的知识财产。



王孟源 01:00:56 

那为什么会这个样子?是因为在大环境下他们形成一个政治正确,那这些决策者、执行的人有必要这样做。你说损失了十亿、 20 亿的知识财产的价值看起来很大,可是对幕后的那些财阀来说、真正的权贵政要来说,白左是如此重要的一个政治跟社会控制工具。



王孟源 01:01:36 

这几十亿、几十亿的,反正是上市公司,那个是所有的股票投资者一起承担,对他们来说没有什么太大的损失,那他们赚的确实很好,确实得到能够撕裂社会,让这个国家社会不再专注在他们所做的恶行跟骗术,能够延续他们的统治。事实上他们对欧盟这次在俄乌战争的反应不就是靠白左的这一套催眠术?



唐湘龙 01:02:14 

没错没错,就那个催眠的效果非常强,它对美国的社会主流社会影响力是很大的。所以当他碰触到了言论自由、思想自由的这个领域的时候,大家警觉就说,哇,这个白左的这种的意识形态他真的可能已经碰到了,就是说了美国人的这种信仰的核心的底线的部分。好,但是在这场的选举当中的,选举以现在来讲还有一个多月的时间,白左跟白右的一场的大对决。那现在是一个怎么样的状况?选举之后会有什么影响?



王孟源 01:02:54 

我其实不太想谈这个大选的问题,因为我今天这个节目结束之后,我会贴出一个新的博文,那个是已经写好的,刚好就是仔细谈这个问题,所以大家有兴趣的人去看一下。我刚刚也提到这个Amazon跟 HBO 这两个剧,一个口碑很好,一个口碑极差,也是完全一样的,就是 Amazon做的那个魔指环的故事,他们也是刚好三四年前,当时Amazon负责他们的这个影视生意的人,因为性骚扰就是 sexual harassment 被开除了。那Jeff Bezos原本就是急着要加入这个真正的富豪权力核心,就是你光是有钱没有用,人家要接受你才行,对不对?所以他不但买了 Washington post,然后他这个 Amazon 的这个 studio 就是这个摄影厂也就顺势的交给一个白左,也是一样,跟那个Kathleen Kennedy一样的一个女权斗士,叫做Jennifer Salke。



王孟源 01:04:20 

Jennifer Salke一上来第一件事情就是终止了他们当时准备要做的,要摄的主要的一个计划,就是 Conan the Barbarian 他们原本计划要重拍Conan the Barbarian,而且他们的那个剧本是很受好评,但是 Conan the Barbarian 我不晓得你知不知道,就是当年阿诺斯瓦辛格的那个电影。



王孟源 01:04:49 

那个当然在女权主义底下是受不了的,一个肌肉男拿着剑,然后一个裸的美女在自己手里,他们怎么能够,这个怎么改都没办法改成政治正确,所以他就把他砍掉了。砍掉以后那个计划的主导人就跳槽到 HBO 去了,那一位先生,后来就证明他真的是很有才华的,他就是现在那个权游的前期的制片人,他的那个主编剧跟制片人。我觉得他这一次是复仇复得很痛快,因为他的这个两个很重要的,很昂贵的剧集对战,结果他把Amazon 给宰的、杀的溃不成军。我真正提这些琐碎的例子,不是说我想跟大家做影评,而是我想跟大家用经济的观点来看一下。



王孟源 01:06:08 

你这个用 wokeness 或者是政治正确来选拔人才、来决定资金配置的时候,你必须要想到他们跟真正的才能、跟最优解有没有什么关联?那事实上你想尽办法去用女性演员或者女性编剧、女性导演、女性制片人或者是黑人等等等等,这个可能跟最优解没有什么太大的偏离。就是说,你不是说那些黑人或女性,他们的演技就特别差,他编剧智商就特别低,但是你至少他不是专注在制作一篇精彩好看的片子上面。我们用数学来讲就是这个 correlation 不是很强的证实,就是关联性不是很强的证实。那这你或许觉得那这没有关系,我们反正就投钱还是去找最好的女性导演、女性编剧。



王孟源 01:07:30 

这问题是你一旦有了,你一旦用了一个,不是选择最优的,不是所谓的 merit based 的,不是看你的才华才能,不是选贤与能的标准之后,你就会遇到一个 adverse selection 问题。我不晓得中文正确的翻译是什么,这是一个经济学的术语,要 adverse selection 勉强可以翻成逆淘汰。这个意思就是说,比如说,所有的人随便哪一个进来我都雇用,然后一个小时给他 100 块美金的话,我保证进门来申请工作的人,他的工作的价值都不到 100 美金。凡是他的工作的价值超过每小时 100 美金的人,都不会来申请你这个工作。这叫做 adverse selection。就是你如果设定一个标准,它不是要求最好的,最直接最准确的标准的时候,人家自然是最不合格的人会拼命用漏洞。所以你一旦不是专心的选贤与能,不是专心的 purely one hundred percent merit based。这个时候越是无能的人越是会无耻的去强调他是女权主义、他是同性恋者。事实上我的儿子已经跟我说,他知道有很多他的同学假装是同性恋,所以才能够申请到那个学校。



唐湘龙 01:09:14 

它变成这种身份保护了,其实那个身份可能是假。



王孟源 01:09:19 

对,你这样一来反而是越差的人越没有廉耻的人,越会想尽办法钻这些漏洞。因为反正你不是要根据贤能、能力,跟品德来挑选嘛。所以表面上看起来好像你说,你这个白人改成黑人,为什么不行?你这个制片人,男人改成女人,为什么不行?理论上好像是没有什么关系,但是一旦你放弃只选择最优的、最有能力的人、最适合的人的时候,自然会有人来钻这个漏洞。那这就是美国现在面临的问题,他们为了背后的权贵能够忽悠社会,忽悠国际,他们自己腐化了他们的整个社会,整个从经济、学术、艺术每一个环节都因此加速腐败,这是非常非常糟糕的事情。



唐湘龙 01:10:30 

因为时间关系,后面的那个题目我们谈不到。孟源刚刚讲的就是说这个wokeness,就是现在美国在讨论的觉醒主义,这个觉醒主义本身它有高度的政治正确。孟源刚刚讲的一段就是政治正确,它会提供很多人一种莫名其妙的身份保护,因此我只要迎合政治正确之后,我就得到了我不应该拥有的身份保护,而使得我在某一些领域的竞争力会被高估。 o over estimation,那这个是很糟糕的情况,就是你所有的选拔的标准都变得很模糊,你可能选到一些很烂的人,出现一些反淘汰的现象。



王孟源 01:11:23 

而且是越烂的人,越会想尽办法钻漏洞。



唐湘龙 01:11:27 

没错,好,那现在呢?现在的美国正在遭遇到这样的情况,它可能才是你现在看到的美国的民主内战,或者是美国的意识形态的对抗的最核心的部分,它不是绝对的。我不是说白左白右,谁对谁错谁好谁不好,但是要注意这件事情就是wokeness本身,它隐含的政治正确会扭曲了这个社会的人才增补、人才的选拔以及资源的配置。那这个是你看不到的,一般人看不到的,因为反正只要跟你唱一样腔调的人都会让你拍拍手,但是他很可怕。



唐湘龙 01:12:05 

好,那其他的有关于通膨的问题可能我们下回聊,但是前面因为孟源,我们花了很多的时间请孟源跟大家去理解最近这将近一个月,大半个月以来国际媒体大轰炸的,乌克兰大反攻,那俄罗斯大溃败的这些讯息,可是它终究这场的战争。我们如果有三个面向,第一个站在战场,第二个金融,第三个从能源的角度来讲,你会发现不管在战场上面,不管在金融上面,不管在能源上面,最大的受害者都是欧洲,可是欧洲就困在里头,现在不知道怎么办那。因此欧洲现在明明自己是战争的最大受害者,可他却一筹莫展。



唐湘龙 01:12:52 

对俄罗斯来了来讲,小伤,对美国来讲基本上面是大赚,中国保持一个中性的态度,可是欧洲作为一个受害者,欧盟作为受害者,却一副完全不知道怎么办,这个对欧洲的伤害太大了。这个孟源上个月就提醒过了,好,来在我再把这个我们的相关的 donate 在面之前的时候,来,我呼一下我们所有的听众,还在线的 15000 位的观众朋友来刷一波,跟给孟源拍拍手。来,大家边刷一波的时候,我感谢我们的一些的超级留言,来,从这个黑战开始黑人,谢谢那中亚之行解决了北西南三个方向的挑战,习大大要用倾国之力解决了东方之患,好再来破论。谢,谢谢这我们的资深的听众胡少瑜,好听君一席话胜读万卷书。对,所以没错,就是我的。我觉得王梦云是一个很专注于思考的人,这不容易,他要有很多的主观客观的条件的配合才能够办得到。再来宇宙真理,感谢奥巴马说,你只需要用勒瑟迅奇搅乱一个国家的舆论场,提出足够多的问题,散布足够多的谣言,植入足够多的阴谋论,让这些国家的公民不知道该相信谁。一旦他们对自己的领导人失去了信任,不再相信主流媒体,不再相信政治机构,人民不再相互信任,我们就赢了。



唐湘龙 01:14:21 

好,再来, Grace 令好,感谢乌艾瑞斯,他说,为王老师打call。他说,我是您博客的忠实读者,王本刚预告了他有一篇的博客的博文准备要发表,你可以看。再来。他说,昨天欧洲议会正式投票,那认定的匈牙利是非民主国家和奥尔班了,奥尔班被认定为非民主国家。他说,这个。艾瑞斯说,我的老公是生活在中国的德国人,我们两个对欧洲的荒诞叹为观止。再来叶九成,感谢张家慧,谢谢 MQ 胡,谢谢 MMQ 胡的刘浩。简单念,纯地诗人,人地皆诗,纯人失地,人地皆得。他在讲的是俄罗斯现在的战术了,就是我不在土地上去跟你纠缠。你进来了我递给你,但是我先保存了自己的战力,减少了自己的伤亡。但是在所有战场当中,梦云这几个月在提醒俄罗斯的战术就是创造了最大的人命的伤亡的杀伤力,那有声战力的供给是俄罗斯的重点。



唐湘龙 01:15:37 

再来苏战 s 感谢 s 里感谢他说普丁是不是太押注在今年冬季欧洲能源上了?欧洲也明白,这样吧。对,这个我们下下回再谈,因为上个月,上个礼拜,我上个月我们已经分析了一些能源问题。Garfield,那谢谢香龙继续邀请王博士支持实在的声音和分析,不想再被假消息和大内宣蒙蔽,因为孟源在很多他专注的议题上面的功课是做得很全面很深入的,我觉得对大家来讲应该很有参考的价值。再来螃蟹说俄乌战争对大陆的警示就是要不然就不打,要出手就要下狠手,要不就打场工具与工具人的新型战争,震撼外部势力。那好,您请说,您有,你有你有,你有话要说吗?



王孟源 01:16:27 

其实这个说到一个很大的重点,就是 2015 年的时候,我第一次上视频访谈的时候,我那时候被邀请去谈香港占中事件,我说这个香港占中事件最大的意义在于给中共政权做了一个教训,就是他不会再搞这个很简单的一国两制,让你完整的保持你原有的制度,然后保持接受昂萨媒体跟 ngo 国务院 cia 这种影响力的体制。那现在这个也是一样,现在俄乌战争这一次我想中方也是会总结教训,所以如果台海打起来,我想他们不会像Putin这样打,这是一个可以预期的事情。



唐湘龙 01:17:21 

我跟孟源刚刚讲,他是台南小孩,待在乡下地方非常深绿的环境,我是台北小孩,但不管怎么讲我们都是台湾小孩,我们都都非常不愿意看到两岸之间用军事手段去解决彼此之间的终极问题。这个是,这是基本态度了。但是这过程当中的变化,这远远不是我们能够控制的,只能够尽可能把一些真实的声音反映给大家听。



唐湘龙 01:17:52 

好。那杰瑞丽,他说王先生,王孟源是华人的骄傲,感谢唐先生的广纳贤才,没有我,我就是这个是我的直播间,有相当比例的我们的固定的关注,那我尽可能的把很棒的人,我愿意把时间都给他。王梦云要给多少时间我都愿意给他,我能够做的就是这样。好,90K,谢谢谢。在 2035 去台湾,他说好久没有赶上王博士的直播了,每次节目都要反复听几次,而且反复听都受益匪浅,王博士是大神。对对,我也未同意。那肖 k one ring a power 好像魔界,我还没有看到他们说的黑人精灵时,我就看不下去了。开头那段的心灵鸡汤,如果有人跟我说是杨幂一边抠脚一边想出来的,我都觉得比较可信,有恶心。好,再来的, c c n n c n 他说感谢王博士下次能够聊聊中美博弈。好,我们反正每个月。



唐湘龙 01:18:54 

好,那配合这个孟源的时间再来凯瑞利,他说国家发展的和到所谓的公民社会或者公民政治,那人格跟智能彼此的相互的抵消,这是胡汉成的观点。好,那因为时间的关系,今天比预戒时间多了 20 分钟。那非常感谢人在美国的孟云山还有很多的议题还没有谈到,但是没关系,下个月的时间孟源还是会在龙行天下的栏目里面会固定出现。感谢孟源,感谢龙行天下的观众,拜。



王孟源 01:19:28 

谢谢你,谢谢大家的捧场,一定很高兴有这个机会。



唐湘龙 01:19:33 

好好梦,也谢谢。好,拜拜。来,我们下个月见,拜拜跟大家说周末快乐。



\twocolumn[\begin{@twocolumnfalse}
\section{英国即将崩溃!俄罗斯终将胜利!、}
\subsection{20221021}
\end{@twocolumnfalse}]有10處左右聽不清楚的字用xxxx(ctrl+f)(6月15日已修正)



Credit: 栗子



唐湘龙 00:31 

好,欢迎来到龙行天下,我是唐湘龙,我在台湾,我在台北。今天礼拜五的时间,龙行天下的来宾是我特邀的固定来宾王孟源博士,来在我们线上的王孟源博士,孟源,欢迎。



王孟源 00:52 

很高兴再和大家聊天。



唐湘龙 00:54 

好,因为我在上个礼拜的时候我没有经过王孟源的同意,我释放了一点小小的讯息,我说本周王孟源,王孟源的单元里面有彩蛋,那大家很好奇这彩蛋是什么呢?因为我本来以为王孟源会回到台湾,能够来到现场让我看到真人,但是很抱歉了,因为时间的关系,那孟源有事情,所以还是没有办法来到现场,不过我们仍然透过视讯的连线请孟源跟大家解读一下最近的一些国际形势的转变。



唐湘龙 01:33 

那本来你看到我们的标题,我想再外行的人也应该会感觉到乌克兰的战争,这场的俄乌战争已经到了关键的决战的时刻了,那所有的准备动作看起来大战一触即发,那会是在克尔松的这个区块。好,这个我们先放后面,因为在 12 个小时不到的时间出现了一个非常重要的变数,虽然大家都在预期英国的首相的特拉斯到底能够做多久,在这件事情上面来讲,那做文章的、开玩笑的、奚落他的...这新闻、这种的花絮太多了,但是无论如何也没有想到,前两天都还把自己定位为一个fighter,他不是一个quitter,他是一个fighter。但是讲完不到 48 小时他要请辞了。



唐湘龙 02:26 

那英国,英国在短短的 4 个月不到的时间里面,你看他的财相换了几个?他的首相换了几个?从脱欧之后英国已经换了第五个首相了,英国女王也换了。这个都是这四个月不到的时间里面英国政治的巨变,但是特拉斯毕竟在这种情况下面成为英国历史上最短命的首相,这个不是一件好玩的事情。



唐湘龙 02:54 

对英国这个国家来讲,过去他的首相的任期向来比其他的内阁制国家要来的长一点,他起码比像是南欧的这些好,比如说意大利的首相要长命,比日本首相要长命,英国的首相其实相对来讲任期都还比较长一点,因此特拉斯这时候 44 天就下台,对英国的冲击是非常大的。这件事情来孟源怎么看?我们从这里开始。



王孟源 03:23 

首先我认为英国之所以能够称霸,跟民主制度一点关系都没有,跟民选制度一点关系都没有。我想我以前已经谈过,在他们 1830 年选举改革法案之前,他们只有3\%,其实甚至可以说有几年是只有 2\% 的人口能够有投票权,那在 1830 年的时候,他已经很明显是世界霸主了。所以他这个成为世界霸主其实完全跟民选没有关系。



王孟源 03:58 

而且他们的这个民选虽然演化了几百年,就是从大宪章开始演化了几百年,但事实上你如果真正去读的话,就会发现大宪章的文化跟历史基础是来自北欧,就是维京人跟 anglo-saxon,其实也是北欧跟德国之间的一些种族,他们的那种部族的选举制,那个所谓后来扯上希腊罗马的那个古典传承、文化传承,那是到了 18、19 世纪往自己脸上贴金的时候,才自己发展出一些历史学家来事后的改写历史,他甚至到 1875 年还是 1872 年选举才从公开转化成秘密投票,你回去看他的历史,就是 1830 年他选举改革之后,投票的人数达到了我想好像是全国国民、不是公民,全国国民的 14\% ,从 2\% 点多增加到14\%。但是那个时候的投票依旧是公开投票,就是你到了投票所要把你的票拿出来给大家看。



唐湘龙 05:22 

要亮票的。



王孟源 05:24 

那当然是当地的土豪或者是男爵,可以派一个主管在那边盯着你看这个,有没有乖乖照我讲的投票?所以那个基本上跟我们现在所想象的民选真的是一点关系都没有。那你想想看,到 1872 年 1875 年,事实上英国的霸权已经开始盛极而衰了,已经开始面临德国的挑战。他们在 1870 年,普法战争之后德国统一,就成为比英国人口更多,陆军远远更强的一个强权。那时候英国已经开始担心被超越,所以你想想看,它真正开始有现代的民选制度有雏形,是已经开始盛极而衰采开始采行的。



王孟源 06:19 

那你怎么能够说这个对它的崛起有帮助呢?这基本上是他们在 20 世纪,尤其是因为要跟德国两次大战对抗,然后接下来在冷战要跟苏联对抗的时候,往自己脸上贴金的一种宣传手法,就是特别强调自己的这个民选制度的一个优越性。然后你比如说美国本身也是,党内初选也是到了 1968 年才开始有,在那之前党内完全是大佬搓圆子汤来选择候选人。所以你只要真正懂历史的就知道这个是骗人。那所以现在英国现在的乱,我为什么扯到这些呢?就是英国现在的乱,其实它的长期背景就是因为自己被自己的宣传忽悠了,所以真正相信这个民选制是万能药。



王孟源 07:19 

那在十年之前他们有一次政党轮替,就是 2010 年他们的保守党接替工党上台之后,虽然第一任的、当时的那个首相就是David Cameron,还算是有点常识的,就是不像现在我们看到这些小丑的首相,但是他的确也是相信这些自己的宣传,所以他做了很多改革。



王孟源 07:51 

就是现在的英国的选举制其实都是很新的,都是过去 10 年搞出来的,然后公投也是过去 10 年才拼命搞公投,所以你看看现在英国出了一点问题,接下来很明显的就是要四分五裂了,那这个所谓的苏格兰公投都是,第一次公投是 2014 年,对不对?这个都是很近的事情。所以正确的历史教训是你越相信这种所谓的民选,这个结果就越糟糕。



王孟源 08:30 

事实上,你去读古希腊历史也很明显的,苏格拉底在世前就可以告诉你这样子说,然后他因为这么讲所以被处死刑,然后柏拉图也这么讲,然后亚里斯多德不敢再讲了,但是你如果去看字里行间他所写的,他事实上也是很明显的这样看。为什么呢?因为老百姓的平均智商定义上就是顶多只有100,然后他们花时间的专业绝对不是政治,所以非常的好骗。



王孟源 09:11 

那政治这个东西可以说是人类社会最复杂、最巧妙的事情,你必须要有很多经验,而且花时间跟精力去了解,会有很多反直觉的效应。那这些事情我花了三四十年去研究,到现在都不敢说有完全的掌握。那你一般的选民能够掌握到几成?连一成都没有,那最后谁来决定?就是报纸。那英国的报纸也是最发达的,它光是小小一个岛国六千万人,有 12 个主流新闻系。



王孟源 09:56 

那在这 12 个主流新闻系里面,有 10 个是支持保守党的,那这 10 个支持保守党的主流新闻就是他们所谓的土豪派的最大政治筹码,因为它控制了 12 个主流新闻管道的 10 个,所以它完全掌管了保守党内部,还有对外的斗争。保守党内部其实派系虽然很多,但是远远最强大的就是这个土豪派,能够偶尔出来跟他斗争的还有两个派系,排名第二、但是比他弱很多的是伦敦的金融系。那么我们最近的这一次选举,就是在年中、7月的那次选举,7月、8月底、9月的那次选举,Sunak代表的就是当地的昂萨金融系,那你可以看到虽然他获得了较多的国会议员投票,但是一到了党员投票就一败涂地。为什么?因为报纸是掌握在土豪派,所以一般的党员都是照着土豪派的说法来投票,对不对?其实这里它还有第三个派系,但是这个派系在过去这三年比较销声匿迹,但是我觉得现在会崛起了。这个第三个派系就是白左派,白左就是美国发展出来做思想麻痹跟控制整个西方世界的一个新的宗教,过去 50 多年,五、六十年发展出来的,那在英国,因为它是昂萨新闻洗脑造假新闻的发源地,所以反而对美国的这个白左教有最大的抵抗力。在整个西方世界里面,英国对白左教的抵抗力是最大,所以因为这个主要的新闻只有一个是属于白左教,就是卫报,所以通过卫报,他们控制了工党,工党的最大派系是白左派、白左教,那白左教背后秘密结盟的当然就是国际金融系,也就是犹太金融系。所以现在你看那个工党的,工党的另外一个大派系是他们本土的工运,就是劳工、社会主义者。那他前任的工党魁是那派,所以他一上任就被抹黑的一塌糊涂,因为他是真正照顾底层劳工福利的,然后处处被撤走,然后在 2019 年选举选的很差,因为所有的新闻媒体不只是保守党的,连应该支持工党的卫报都在背后对他捅刀,所以他就被推翻了。然后现在上台的是Keir Starmer。



王孟源 13:40 

Keir Starmer 本身是犹太财团派来的,所以他现在掌握工党的就是白左犹太财团。那我为什么要详细的谈工党呢?因为现在很明显的就是保守党不可能再赢得下一次大选。所以真正要掌权的下一任首相,就是不一定是马上的下一任,但是下几任首相下去一定是Keir Starmer。所以到时候真正收拾残局的是美国和犹太财阀所掌控的白左。



唐湘龙 14:19 

好,我打断一下,因为就在这短短的这两个月里面,一个英国历史上面在位最久的女王跟一个在位最短的首相同时发生而都过去了,这种的最长在位的领袖跟最短在位的首相同时过去,这种看起来是一个极端值。不过我相信对英国政治一定有一个很深刻的意义。那在这一次特拉斯之后的英国政局会怎么样?两个问题,一个就是说特拉斯之后的英国的政局,你刚刚讲的那点我完全同意,就是说保守党下一次的选举只要解散国会之后再来执政,就不会是保守党,工党上来的机会是很大的,所以保守党无论如何这个时候看起来不会解散国会。所有的冲突内部处理,保守党一定还要拖,以拖带变。可是接下去的政局里面,第一个英国会怎么样?英国会进一步的崩溃吗?第二个就是说英国跟欧洲的关系会怎么样?因为最近光是英法的关系都变得很微妙。还有英国对中国、对中美之间的政治态度会有改变吗?



王孟源 15:31 

好,你一连问了其实是 4 个问题。我一步一步来谈。第一个是保守党内部的演化。 5 天之前我在博客做上一次评论的时候,我还觉得最大的可能是赶快换Sunak上来。金融系来取代,但是我现在已经因为过去 5 天的新发展,我已经觉得那不再是最可能的发展。



唐湘龙 16:00 

没错,Sunak看起来不大。



王孟源 16:02 

为什么呢?原因是这一次跟对Truss发难。就是Truss之所以出问题是并不是因为她自己一时高兴临时犯了什么错,而是他背后的土豪要求他做的事情,而且是过去四五年来就培养他所做的事。我先讲一下Sunak跟 Truss都是在四五年前就被培养了成接班人。你如果去看他的履历的话,他们都是平步青云,莫名其妙就是从一个新的国会议员,然后就很快就被拔擢,一直拔擢到接班人的地位。这个是土豪派跟金融系都内部做准备的,才可能有出现。就是这些人根本没有什么功劳,也没有特别的才能,没有什么特别的履历,但是他们之所以被拔擢是因为他们对背后的财阀效忠,而且被选拔出来。



王孟源 17:15 

当然幕后有很多政客拼命的要去找真正的权力阶级向他们效忠,但是呢,他们(Sunak跟 Truss)是挑出来的,那 因为 土豪派一直在过去,自从 Johnson之后就完全掌握了政权, Johnson 本身就是土豪派的代表,那 Truss 是原本就准备好的接班人。那土豪派他们最关心的是什么?就是减税。当初2016 年为什么要脱欧?就是因为在 2016 年1月,欧盟通过了反避税指令,把英国土豪最喜欢用的一些避税手段定为非法。那这个原本预定是在 2019 年实施,后来延期到 2020 年。所以你看是不是很巧?他 2016 年1月的时候出现反避税指令, 2016 年两个礼拜之后,英国国会就通过要公投脱欧,公投脱欧之后。Theresa May因为知道脱欧对英国非常不利,所以一直设法减低它的损害,但是拖到 2019 年不能够再拖了。为什么不能再拖了?因为那个时候反避税指令马上要执行了,所以她就被推翻了、就被土豪派推翻了,推翻以后换上土豪派的代言人Johnson,那 Johnson 马上就不管三七二十一,硬是先脱欧再说。



王孟源 18:55 

你可以回顾当时的细节,那根本就是硬闯,就是所有的谈判、细节、法规都没有搞好,但是先拖后再说。这原因很简单,因为那个反避税指令已经要生效,所以必须要出现。那当然你这样子硬闯的话,这个人只是一个傀儡,是用完就可以丢掉,所以他们也就马上就开始培养接班人,也就是 Truss 。那Truss其实在去年年底就可以确定是今年会升任首相。



王孟源 19:34 

那为什么?因为我说过的那 10 个新闻频道,全部都是在 12 月的时候就开始吹嘘 Truss 有多么多么多么的了不起,然后把 Truss 在国外出丑或国内说错话的事情都开始掩盖。这种待遇是只有自己派系的首相或者是首相的候选人才能够享受。我相信在台湾这个自由时报对民进党的领导或者肯定后任领导也是有同样的待遇。



唐湘龙 20:14 

当然,呵护有加,努力造神。



王孟源 20:19 

所以你可以从这些就可以看出这些真正的权力阶级,也就是掌握这些媒体喉舌的人,他们的想法、他们的棋局的布置。所以 Truss 一上台,在一个半月前上台,你事实上可以预期他一定会推出减税。这里的问题不在于他推出减税是不合理的,就是从他的观点来看,他的政治环境还有它的权力运作的模式来看,完全合理。



王孟源 20:57 

问题在于天时地利已经不在了,对不对?美国跟欧洲在 70 年代一直到 80 年代早期,通过了一段很长的通胀危机,这也是因为美元体系上一次在越战之后必须要推翻布雷顿森林的体系,所以开始拼命的印钞票,所以造成了一个 10 年期间的通胀危机,后来解决了之后,后来真正解决是把苏联给打垮了,然后吸收了苏联的所有的资源跟财富才真正彻底解决。



王孟源 21:46 

要不然在 80 年代的时候,美国的危机感也是很深的,虽然它通过非常高的利率来控制了通胀,但是美国 80 年代其实是危机四伏,觉得在军事上有苏联、在经济上有日本来威胁,他真正的觉得又凌驾在全世界之上是在 90 年代,就是冷战获胜,几万亿、几万亿的红利流入欧美的经济体系,那个时候他们才完全恢复。



王孟源 22:23 

那正因为如此,在 80 年代、 90 年代、 00 年代到10 年代这样 40 年,他们因为这个天时地利的关系可以搞这些减税,其实就是图利富豪,对不对?我从一开始就说,这其实是土地富豪的事情,那实质上看起来好像是利润极大的增加了,实际上是掏空自己的实体经济,把它让渡给南韩、台湾跟大陆。



王孟源 23:04 

那这些实体经济当然看起来利润不高,而且很累、做起来很累。但是你一旦只剩下金融的话,到最后这个自己的体质越来越虚。所以事实上我在三年前就已经预言,因为在过去这 14 年他们的钞票印的太多了,比 50 年前印的还要多。那么所以下一场经济危机一定是通胀推行的经济危机。这个通胀推行的经济危机,他们除非能够转嫁到外部,要转嫁到外部不能够凭简单的转嫁,就是 2008 年后的 2000 年那样的转嫁,而且必须是像冷战结束后那样子,把一个主要的世界经济集团彻底的解体,就是像30 年前解体苏联那样。所以你看到今年他们挑起俄乌冲突,美国并不是指望乌克兰能够在战场上打赢,他指望的是靠经济制裁来摧毁俄国的经济,然后造成俄国的民意沸腾,推翻Putin,然后推完Putin政权之后,才能够重复 30 年前肢解苏联的那些获利,这是他们的计划。



王孟源 24:49 

但是因为Putin已经准备了很久,他其实是从 2004 年就开始准备,到 2014 年上一次的乌克兰危机,他觉得自己还准备不够,所以又拖了 8 年,所以这一次他在乌克兰是有恃无恐,他在这个金融、货币、经济战线大获全胜。那这样一来,在这个通胀性危机的打击之下,这个美国所领导的这个所谓的 golden billion,就是美国体系下拥有高科技的,刚好台湾是这个 golden billion 的吊车尾,就是愿意听美国的话,所以美国容许他们有高科技工业的那 10 亿人就必须要内卷,尤其是在货币跟金融上的内卷。



王孟源 25:55 

那在这样的大背景下,英国还在做急剧的减税,这简直是自杀。因为他搞脱欧搞了 6 年其实体制已经很弱了。而且它的规模原本就是这这 golden billion 里面的 4 个集团最小的。这 4 个集团是哪一个?美国当然是主体,其次是欧盟、日本跟英国,对不对?这 4 个里面原本英国就是最小,那最小的再加上 6 年是最弱的,那你现在还在胡搞,那当然这个玩了 40 年的减税游戏就一下子忽然玩不下去了。那这个问题就是马上的出现了一个汇率跟债券利率的双重危机。



王孟源 26:43 

所以他在9月 23 号推出这个减税方案之后,当天就出现了危机,然后隔了几天他们的中央银行就必须要出手拯救。那出手拯救是什么?还不是印钞票,就是从原本在升息控制通胀,突然一下子要转过来继续做量化宽解。那我为什么讲了这么多?因为接下来发生的事情是,虽然他这个用量化宽松暂时的把这个危机往后拖下去了,但是很明显的通胀还是在那里,这就是为什么通,我说这一次通胀危机他们必须要内卷的原因。



王孟源 27:31 

就是因为他们过去 40 年是,“出了什么问题,就靠印钞票解决”。但是通胀的本质就是你原本就是靠印钞票去搞出来的,你没办法再靠印钞票来解决这个问题。所以,他这个暂时的拖下去之后,这个 Truss 背后的那个土豪派所掌控的那些主流媒体没办法,纸包不住火、压不住,全世界包括 imf,包括国外、美国人都在批评他。



王孟源 28:15 

这样子一来,金融系就觉得有机可乘,所以发生的什么事情就是刚好一个多礼拜前,中央银行行长代表金融系出手,他说到刚好 10 月 14 号是上一个礼拜五,他会停止救市。因为有很好的借口嘛,大家都批评我这样重新量化宽松会让通胀更为严重,那我就适可而止,救市两个多礼拜就结束,那这样一来就造成了危机。而且当时的那个 Truss 的财相Kwarteng刚好是在美国开会,所以他没办法应对,他急急忙忙的礼拜五紧急结束开会,然后飞回去,飞回去当天就被解职了。解职时候就已经是弃车保帅了,这个时候 Truss 要辞职已经是难免的。他们的问题就是在你刚讲过的,为什么 Truss 没有当场就被强迫要辞职?就是你刚刚讲的,因为现在这个选举的话,如果保守党去马上提前大选的话,马上工党就上去。所以保守党在上个周五之后,过去这一个礼拜还是在想方设法能拖就拖,而且问题在于他们这个金融系虽然是出手把它搞掉,我原本上个周末还觉得既然他们敢出手搞掉的话,应该是有把握能够获利才可以,就是至少能够当一年,但是后来去看不是这样子,因为接任财相的是谁?这很明显的是土豪派也知道,金融系给出手搞他们,所以存心要报复。就是我宁可放弃政权,但是也不会要交到Sunak所代表的金融系的手里。因为你看看他交给的财相是谁?Hunt是白左派,然后还有那个Penny Mordaunt,也就是几个月前选举的时候异军突起的,那个也是白左派。所以你看现在他们这个局势就变成很微妙,因为是三方竞逐,就是这个白左派认为有机可乘,可以在两大派系互相不满的之际来鹬蚌相争渔翁得利。那另外一些方面,这个土豪系还是有些人不愿意就直接释放权力,所以才会有让Johnson回锅的这个说法,因为土豪派没有其他的人派的出来了。他那个储君刚刚上任 45 天就下台。



唐湘龙 31:29 

就夭折了。



王孟源 31:33 

他只好找上一任的来,所以目前这个局面就很微妙。我觉得其实最可能的是白左派跟土豪派有妥协,但妥协的结果不一定是Johnson,也不一定是Penny Mordaunt,但有可能会这两个选一个,然后Hunt连任财相这样,就是金融系虽然是出手把 Truss 搞掉了,但是没有得到什么实力。对不对?保守党会尽可能的拖到 2024 年在大选,就是因为他们的法律规定是至少每 5 年要一次大选,他上一次大选是 2019 年,是吗?所以还有两年。那你反正一大选就一定会失去政权,那还不如自己派系里面妥协一下,那这 3 个派系里面虽然白左派是最弱,但是现在土豪派变成人人喊打的老鼠了,所以他们有可能胜出,就是 Penny Mordaunt有可能胜出成为首相,然后最可能的是无论 Penny Mordaunt胜出或者是Johnson胜出,我觉得Hunt都有可能要连任财相。



唐湘龙 32:52 

好的,我们回头来看俄乌战争。那从英国的角度谈起,英国是所有的欧洲国家里面,老欧洲里面对这场的战争的参与度最高的。那不管是摇旗呐喊,Boris Johnson三度到基辅,你就知道对基辅所表达的那种的赤诚。当然英国它有一个潜在的角色,北约虽然在带头的那是美国,可是美国毕竟不是欧洲国家,实际上北约的发起,北约的运作,英国是关键的角色,所以当欧洲有战争的时候,英国的重要性就是提高了。



唐湘龙 33:42 

从传统上面,刚刚的孟源分析的英国的历史,其中有一个历史就是英国长时间跟欧洲之间的战略,英国的态度只有一个,就是唯恐欧陆不乱,就是欧陆一定要保持混乱的状态的时候,英国的存活跟筹码才会达到最饱足的状况。好,那现在俄乌战争,我估计不管是谁来当首相,英国的传统的战略,美国在利用英国,用英国当做是在欧陆的纠察队继续打乌克兰战争这态度不会改变。可是最近俄乌战场上面的情势出现了变化,那对俄罗斯来讲也征兵了,那也发布戒严了,那我们看到不管是中国也好、印度也好,都开始有一些所谓的撤桥的呼吁或者是动作。在科尔松地区来讲,看起来大战一触即发,到底现在乌克兰战场上面,因为以基辅来讲,现在所采取的态度是一种所谓的战场静默,那泽伦斯基讲的就尽量不发布跟战场有关的讯息,那现在的这种的战场的静默,使得大家对于到底战争的现场现在的情况如何,大家就开始有一点点的犹豫,觉得讯息当中变得不是这么的确定。究竟现在在乌克兰战场上面的进展如何?接下去的发展会如何?大战要开始了吗?



王孟源 35:15 

我先谈英国在俄乌战争有昂萨集团对外战略上的角色,就是土豪派向来都是最坚持继续玩这个离岸平衡守则,就是英国自从有三四百年的离岸平衡政策。你可以看到英法百年战争,其实也就是那个离岸平衡守则的滥殇。那所以土豪派的 Johnson 跟 Truss 都是乌克兰最积极的支持者,但是白左派其实是由犹太跟美国来间接控制,所以他也不会改变这个政策。对俄乌政策非常的鹰派,自然也就是对台海形势跟东亚的形势也是鹰派,因为这个是同一个思路。



王孟源 36:16 

所以唯一有可能对外比较温和,然后对转而关心自己内政的反而是Sunak那个昂萨本土的金融系。但是我刚刚已经解释了,他们这次不太可能胜出了,看起来不太像是会胜出。所以基本上英国的这个对外政策,无论是土豪派胜出,还是白左派胜出,还是工党兼任,因为工党现在是白左派掌控,对不对?它的这个对外政策都不会温和化,就是你可以在预期未来的2到 3 年,他们不会有觉醒,不会有改变。那这里的一个很大的变数就是因为女王过世了,那女王过世了以后苏格兰公投就得到了另一个新的推力,这也是我已经讨论了 3 年的事情,那目前最新的发展是他们已经上诉到最高法院,就是苏格兰有他们的自治议会,这个自治议会是由苏格兰国家党所主控,是他们有绝对多数,那他们就想要再搞一次公投。



王孟源 37:38 

那目前的争议是:英国的真正的国会不容许他们公投,因为是保守党主控的。英国的保守党主控了他们在伦敦的国家议会,所以不让他们去再一次公投,那苏格兰就说那我们只做一个民意调查的公投,就是没有法律约束力的公投,那英国的保守党还是说不行,我说不行就是不行。



王孟源 38:13 

这就是所谓的民主跟地方分权的真相,就是我用来搞别人的时候可以,我自己出问题的时候就不行。所以这个苏格兰国家党就上诉到最高法院,说我现在只要搞一个公投,但是只是民意调查,没有任何法律约束力。我一个已经地方分权的地方议会怎么会没有权力搞民意调查呢?当然他们的所谓法律独立,这个也只是骗人的、骗小孩的,实际上都是幕后搓圆子汤搓出来的。所以这个他们会不会真的要拉下脸来,就说硬是说这个连民意调查也不行?这个也有可能,但是目前保守党的,我刚刚已经说过,已经是人人喊打的老鼠,他们目前的民调还不到 20\% 的支持率,所以他们这个对最高法庭还有多少的掌控能力很难说。那我们乐观其成,就是如果最高法庭容许的话,公投应该是在明年下半发生。



唐湘龙 39:36 

明年 10 月。



王孟源 39:38 

明年下半发生。那公投通过以后,如果公投通过以后可能还要再吵一阵子,然后大概 2024 年正式独立,这个 机率已经是超过50\%。所以我先讲一下。就是英国的对外的政策,就是掌控唐宁街 10 号的人,在未来三年仍然会是对外的鹰派,但是问题是他自己的这个国家,它的内部的结构有可能会分崩离析,就是超过一半的机会会在两年内分崩离析,这个也是我过去 3 年一直讨论的事情,大家可以到我的博客参考。那在俄乌战争,我想提一下这个俄乌战争我在过去几个月上你的节目,甚至更早,在俄乌战争一开始我就已经讨论过一些很奇怪的事情,就是Putin这一次出手,他的在战略上跟战术上都选择了很多明显不是最优解的事情,违反了孙子兵法的模式。



王孟源 40:54 

一个最明显的是:他动手之前只调用了 10 万的正规军,那我已经反复的说过了,你这个规模的战争需要 40 万才能够好好的打,那所以在2月的时候我就说,包括我在内的很多懂军事的人都说不可能打的起来,为什么?因为你需要 40 万人的战争,他只准备了 10 万正规军在外面做演习,这完全不够,结果他硬是去打,所以我们大家都眼镜掉了一地,对不对?那打了以后,你硬是去打那当然后果就非常的不利,头 6 个礼拜他们就损失了 1300 个阵亡,为什么阵亡?因为它的补给线太长了,他没办法真正的占领那些土地,所以他只占领了那一条公路,那在公路上运输的时候就被乌克兰的步兵小组很简单的伏击,所以你那时候看到有很多装甲车跟运输车被飞弹打烂的情形,对不对?然后到了3月底,他这个威逼乌克兰和解的企图被瓦解了,就是那时候是Johnson飞到基辅去跟他讲,“你不能够跟他和解”,你要是和解的话我们不做任何保证,那你等于是我完全被俄国人控制这样。



王孟源 42:36 

所以从4月开始,它又转化成用炮兵,慢慢的消耗乌克兰的军力。那有一段时间看起来他做的不错,但是到了8月之后战局又逆转,那事后我们可以发现它逆转的原因是什么呢?就是他派出的这 10 万正规军其实是俄国的募兵制,俄国并不是纯募兵制,他的募兵出来的这个职业军人其实只占部队的少数,它整体是一个防御的态势,跟美国这种进攻态势的体制完全不一样,美国是全募兵制, 100 多万人的部队,每一个都是募来的,就都是志愿兵。所以你既然是志愿兵,总统一指命令就可以派你到任何地方去做侵略、做打仗。但是俄国真正能够成建制派出国境之外打仗的,就是他的法律明文规定,义务兵不能够出国境去打仗,只能够保卫自己的国民。



王孟源 43:50 

他当初只派 10 万正规部队,原因就在于能够成建制出去的志愿军就只有那 10 万,而且这个合同里面说的很清楚,就是你在国境之外作战有一个限期就是 6 个月,所以你从2月底打到8月的时候这些合同都到期了,这些人你说你再怎么爱国,但是每天这样子辛辛苦苦的冒着枪林弹火,他们后来统计是死了大概将近 6000 人,包括头一个月的那 1300 人。你说 10 万人里面死了 6000 人,这是 6\% 的死亡率。你说一般的这些志愿兵会自愿的再签延期吗?我想大部分人是不会的。所以他在8月的时候就出现了兵力严重萎缩的问题。然后你知道乌克兰的作战除了财务跟后勤资源是靠北约之外,其实依靠北约最有效的一点是他的情报。就是北约的卫星还有电子侦察体系,一旦俄国的部队开始萎缩,然后你搞出空城计来的时候,北约看到一清二楚对不对?所以我们过去这一个多月所看到的这个进攻,事实上就是乌克兰看到通过北约的情报,所以看清楚他在俄国的俄军的布置非常的虚弱。



王孟源 45:41 

你说在Kharkiv那个大撤退,他明明就是只有一两千人,守1万平方公里的土地,那个可笑到极点了。为什么?因为一个次要方向。它为什么是一个次要方向?我上个月已经跟你讲过了,他那个时候被占领的最大的城镇是哪一个?Izyum。Izyum的人口在战前只有4万,后来一个月前我上你节目之后,我说他会退到后面的一条大河,然后事实上他又从那条大河退了大概 50 不到 100 公里,到另外一条比较小的河也是同样南北向。



王孟源 46:21 

那这个,就近一次的撤退呢,损失的最大城镇是哪一个?就是Lyman,Lyman比Izyum要小,他战前的人口只有 15000 人,他的意义纯粹是历史性的,就是Lyman是因为在十月革命的时候曾经有历史性的出现,就像我们这边讲武昌革命,所以武昌是有革命意义的,Lyman也是那个也是一样的。



王孟源 46:57 

事实上它是一个很小的镇, 15000 人的镇,你看这有多小?这放在台湾台南这边都根本是连镇可能都说不上。那你如果去看过去这五、六个礼拜俄军撤离了,虽然是 1 万多平方公里,但它损失了哪一个?哪一个人口中心一个都没有,最大的就是Izyum,就是Lyman,那也就是因为俄军的将领,他知道自己的兵员一下微缩了,微缩以后就变成乌克兰进攻的绝好时机,所以他必须要放弃一些守地,也就是以空间换取时间。



王孟源 47:48 

换取时间的意思是到哪里?就是要到 10 月雨季开始。那我这边雨季我必须要讨论一下。我谈雨季当然是指军事上的雨季,你如果是一个小学生的话,你要知道乌克兰的雨季,你可以到 维基百科上面看一看乌克兰的雨季,你会发现它雨量最高的是7月跟8月,但是那个是农业上的雨季,不是我谈的那个雨季。你如果对军事有任何一点的接触就会知道,在二战的乌克兰的战场从 1941 年打到 1944 年,打了 4 年,他们反复的谈的雨季就是秋季雨季跟春季雨季就是泥泞的时期,秋季雨季就是 10 月跟 11 月,春季雨季就是3月跟4月。为什么呢?因为乌克兰夏天下雨,跟我们台湾这里下雨一样,是暴雨,这个雨下来以后,大部分没有时间渗透到土壤里面,很快就流失,然后大太阳一晒,土地就干了。所以土地泥泞一两天的事情,你真正要看这个军事上意义的雨季是你土地的泥泞的程度是不是占天数的多数。



王孟源 49:17 

那你如果去看乌克兰气候的阴雨天的天数,它其实很明显的有两个季节,一个是 10 月中到4月中,这是他的雨季,他的阴雨的天数占多数,然后另外的那 6 个月他的阴雨天数占少数。然后因为从 11 月底,就是 12 月初到2月底是冬天,所以下的这个不是雨,而是雪,然后地面会结冰,所以冬天反而是可以机械化作战。那真正不适合机械化作战的就是 10 月中到 11 月底,还有3月初到4月中这两个时间。



王孟源 50:11 

我刚刚说过,今年年初俄军在做第一阶段作战的时候,他们补给线拉得很长,其中有一个原因就是那个时候也刚好是雨季,因为他从2月底打到3月底,3月也刚好是雨季,所以他的那些装甲车什么的不但是因为数量不够不能够全面占领,而且事实上气候的环境也不容许他越野,所以就必须要沿着公路对不对?沿着公路的话就非常容易被伏击。



王孟源 50:48 

所以你现在是半年后了,那反过来是俄军的将领一看,反正到 10 月中以后,那个土地就完全是泥泞的,那你这个乌克兰的攻势反正一定是会因而停滞,所以你放弃一些空地给他们不是什么了不起的,反而是对方要去在阴雨天之下守泥泞地里的战壕,这是对他们的士兵的士气的另一层的打击。



王孟源 51:23 

补给上也是非常的困难,对不对?那很明显的,我之所以会改变看法,就是原本你我刚刚说过,俄军所做的这些战略跟战术的运作就是遵循速战速决,你如果去看孙子兵法里面最重要的一点就是要速战速决,对不对?它里面讲说没有一个长期的战争是好的战争。里面是怎么讲的?但是这是孙子兵法里面的一个非常重要的结论。那你就必须要问你自己为什么会这样?他在第一阶段的时候没有派足够的兵力,还可以说是因为他不确定能够挡得住欧美的金融打击。这个我刚刚也讲过,欧美很明显的真正的主战线是在金融经贸上面,那在2月3月的时候还不确定能够挡得住,那个时候还可以说的过去,说他试图要提早和解,但是到了4月跟5月的时候,已经很明显的俄国中央银行大获成全胜。真正在经济上会出问题的是欧盟,尤其是德国。那你这个时候Putin还不征兵、还不动员就有点奇怪了。我也纳闷了几个月,然后一直到上个月我才想通,原因是俄国本身的民意被渗透,在过去这 30 年被西方的传媒渗透的太深了,所以他完全失去了对国家民族的认同。Putin在这一次的作战,它最重要的任务其实是重建俄国百姓俄国人民爱国的精神。



王孟源 53:43 

那你如果去看俄国的历史,俄国这个国家跟他的民族。素来这个国家民族的意识的核心就是卫国战争,对不对?第一次,第一次的卫国战争是在拿破仑, 200 年前拿破仑的战争,第二次的卫国战争就是二次大战。你如果真正看他们的社会风气、文化、历史传承,所谓的国家意识都是围绕着被侵略然后反击。



王孟源 54:19 

所以Putin如果真正要想重建一个健康的国家,民族意识的话,很明显的这个最佳的路线就是重复过去两次卫国战争的经验来唤醒民众,而且在这个过程中要提醒国民,侵略者是外来的,不是他自己。所以这才解能够很好的解释为什么他对乌克兰的民用的基础设施会这么的温和,对他们的 collateral damage 就是人民的附加的伤亡非常的小心,宁可军事上失利,事实上军事上早期的军事失利是他们这个卫国战争的叙事里面的一个重要部分,牺牲的那 6000 人是为了要唤醒国民。



王孟源 55:29 

而且我想特别讲一下这个,你可以说他是要争取民意,但是这跟西方社会里面所谓的民意是两回事,西方社会的政客争取民意是为了要争取临时的支持,以便自己为自己牟利。Putin的重点是要在教育民众,就是不是为了讨好民众,而是要把民众的意识扭转过来,正确的认知国际的情势。这个我在博客上有人问说Putin要争取民意,那是不是跟拿破仑三世在 100 多年前讨好法国国民有类似?我说那是在胡扯。拿破仑三世是靠着减税,然后刺激经济,来收买人心来保障自己的政权,Putin跟保障政权没有什么关系,这个是完全是为了建立新俄国。所以在社会风气上你必须要做第三次的卫国战争。那要做第三次的卫国战争,但是在表面上又是俄军进军到乌克兰境内,所以这才是让他撤手撤脚的真正的原因。



王孟源 57:02 

我一旦看到他一次动员 30 万,就很明显的了解,这必须是事实真相,为什么呢?你算一算,他现在这个既有的可以动员的正规军志愿军是 10 万,稍微缩减了一点,然后动员 30 万,然后再加上志愿军,现在各州都派出志愿军,加起来有七八万。这全部是不是刚好 40 多万,对不对?我从战争一开始就说这场战争要认真的打,你需要 40 万的兵员。



唐湘龙 57:44 

现在在国际上面在对这样俄乌战争来讲有两个气氛正在酝酿,一个就是说最后的决战的时刻是不是快要到了,这个决战对哪一方比较有利?第二个就是说俄罗斯有没有可能引发在乌克兰战场上面的就是核武器,就是说有关于,就是说核武器的使用在这场的战争当中可不可能出现,这两个问题,你怎么看?



王孟源 58:21 

我们先谈未来这几个月我的预期,就是一旦俄军能够派出四十几万的部队,你要了解目前的战线是 1000 多公里,将近 1500 公里,但是俄乌的边境还有另外的 1000 公里,是大家没有什么派兵就是沿着边境干瞪眼。白俄罗斯跟乌克兰还有另外的 1000 公里的边境,就是另外还有 2000 多公里的边境,而且基辅很靠近这些边境,对不对?他们2月底到3月的时候,俄国就凭着威胁基辅,长驱直入去威胁基辅来试图以战逼和,所以你一旦有了 40 万、 40 多万的部队,就可以完全利用这 2000 多公里的边境来进攻。所以我认为在2月底之前俄军会获得相当可观的胜利。



王孟源 59:29 

那这个胜利有三个层次,第一个层次是他会将乌克兰的部队完全逐出东乌四州。你如果去看现在的地图的话,乌克兰的东乌兵团,就是原本那8万的精兵,现在还是盘踞在Donetsk 中。他的这个大本营,后方的大本营就是司令部的所在,Kramatorsk, 也是在Donetsk里面。所以要收完全收复东乌四州其实不是很简单的事情,你必须要完全歼灭或者完全击溃他们的东乌精锐兵团。但是我认为这个第一层的胜利条件,俄军在2月底之前完成的可能性在 90\% 以上。为什么呢?因为你有 40 多万的部队,那个这时候它的兵力对比到达了 1: 1。



王孟源 01:00:34 

我们看 1941 年德国 Operation Barbarossa 图袭苏联的时候,它的兵力对比是 4: 3,但是在组织训练跟准备程度上还有空优上远高于苏联。现在你如果去看俄军,在组织训练跟准备跟空优的压倒性优势上比当年 1941 年的德军还要高,所以不需要 4: 3,1: 1也可以在局部造出 2:1 或者 3: 1 的优势,轻松的击溃重点部位。所以我认为在2月底之前,完全消灭或击溃乌克兰的东乌军团的可能性是在 90\% 以上。



王孟源 01:01:37 

第二层的胜利就是再进一步,这在完全收复东乌四州之后,再占领一州、 2 州或者甚至 3 州。这里面最重要当然就是Odessa,那这个可能性我认为在 60\% 左右,要看俄军的实际执行结果。然后第三层的胜利,这个也就是因为乌军的大幅溃败会导致乌克兰高层政权的更迭,而且可能会寻求若干的和解。这种完全战略上的胜利我觉得就比较困难了,比较难说,可能是在我,我估计可能性是在  百分之三十,或者略低一点。所以有这三层的考虑。那这个核战的这个问题,你如果回头去看,其实在1月战争开始之前,Zelenskyy 已经要求北约准备动用核武器,然后开战之后很快的波兰的实际执政人,也就是他们执政党的党徽也建议部署核武器,然后使用核武器来对抗俄国。然后来是英国的 Johnson 跟Truss,那时候还是外相,也建议要使用核武器来打击。



王孟源 01:03:08 

你如果去看 Putin的回应,他是忽然到了9月中,也就是决定要动员而且吞并东乌四州之后才主动说,主动高调的对媒体说,我们有很多核反击的手段,我们不怕你的核打击。这个我觉得很明显的是Putin认为自己到2月底之前很有可能会完全击溃乌克兰政权。我刚刚分析有 3 个层次的胜利,但是我认为无论是第一层,第二层还是第三层的胜利,美国在绝望之际都很有可能会动用战术核武,或至少会考虑动用战术核武,所以我觉得是Putin在未雨绸缪。我们现在讨论这个核武吗?真正准备要动手的不是Putin,而是普京觉得我在乌克兰未来几个月会很快的很戏剧性的扭转战局,到时候北约就会狗急跳墙想要以核武,所以我事先把这个话题丢出来让你讨论一下。



王孟源 01:04:36 

尤其你可以看到现在民主党政权因为其中选举,这个通胀始终居高不下,它的这个选情在今年夏天因为炒作堕胎的问题,有稍微回缓一段时间,但是现在又越来越不利了。共和党本身其实也有很大的内讧,就是他们的建制派不愿意看到Trump领导的那些民粹派的候选人当选,所以有意的去扯他们的后腿所以那也帮助民主党的选情。但是现在的经济态势实在是太不乐观了,你不要说美国了,这个欧盟更惨,然后英国又比欧盟更惨。在这种条件下,这个我觉得拜登已经是很绝望了。你可以看出过去几个月已经有很多绝望的手段,比如说他跑到沙乌地阿拉伯要求供油,这个是为了期中选举之前赶快解决能源问题,结果失败。



王孟源 01:05:58 

现在乌克兰已经到了雨季了,结果他在Kharkiv那边占领了一大片空地以后已经无力再推进。你如果去看细节的话,现在是俄军开始反过来收复一些Kharkiv的领土。那为什么呢?因为他把那个部队从Kharkiv那边全部又转移到Kherson那边这为什么?因为美国人跟他讲,你占领了这一大堆空地没有意义,我们的宣传了一个礼拜以后,老百姓就不管了,你必须要占领一个大城市。他最可能占领的那大城市就是Kherson,所以他把Kharkiv的部队又全部调到Kherson那边,现在拼命打,可是Kharkiv刚好雨季,因为我想是因为气候变化的原因。



王孟源 01:06:52 

今年的雨季来的比平常晚了两个礼拜,就是今天开始才开始下大雨。所以乌克兰这一次要赶在 3 个礼拜之后,就是美国大选是 3 个礼拜,要赶在 3 个礼拜之后就拿下Kherson基本上是不可,但是他还是尝试,因为美国人教他尝试嘛,对不对?那个布林肯打电话叫你说,我不管你们死伤多少,反正你们要试图占领Kherson,一旦如果能占领Kherson,那民主党的选情就能够起来,就可以大做宣传。因为Kherson是不但是一个主要城市,而且是已经独立的,是投票出来要并入俄国的一个主要城市。所以我目前看到的有关你刚刚问的这个核子战争的问题,还有这个战事的发展,我想到这里应该是很明显的,大家应该了解到一个转折性的改变即将到来,也就是我们到 11 月底、 12 月初俄军会发动总攻,然后在总攻之后,未来的那接下来的那三个月,我们会看到很快的、很明显很戏剧性的占据反转。



唐湘龙 01:08:20 

好,今天我们谈的两个大的段落,一个是英国的,英国现在的崩溃式的反应,这个对于欧洲政治来讲,不止对英国影响很大,对欧洲政治的影响会非常的深远,也会直接影响到在俄乌战争的后续的推进。第二个部分,我们谈到了这场的战事,因为大家高度的关注,毕竟已经耗了这么多的时间精力,同时现在对于强权国国家来讲,几乎已经把面子里子都已经赌上去了,到底在战场当中的推移如何,这大家是高度关注的。西方的媒体的叙事每天都是俄罗斯快崩溃了,普丁快崩溃了,这种的讯息。但是刚刚的孟源的分析告诉大家,接下去的这一两个月,大概你会看到的,俄罗斯比较建制性的、集体的有计划的反攻,大概在 2 月份的时候会出现了根本性的转折,就是比较完整的,那有具体成效的胜利大概会在明年 2 月份左右会出现,对吗?



王孟源 01:09:30 

我想这边加一句话,就是从事后来看Putin的这个战略计划,这个战略政略计划其实是很高明的,因为俄国真正的危险从来就不是军事,连美国自己在事先都没有预期乌克兰能够在军事上获胜,对不对?他们真正的打击是在金融跟经济层面,但是俄国先在经济跟金融层面战胜西方集团,然后接下来的过去这 7 个月其实是集中在美国人没有预期的外宣战线上,然后在第三世界的这个宣传外交上面大获全胜。



王孟源 01:10:22 

正因为在军事战线上俄方有恃无恐,你即使是拖个一年半载,表面上好像是战事僵着,世上俄军不可能失败,对不对?所以他这样子把在这个局势的演变造成欧美国内的压力。因为你看看我刚刚提到的有关美国大选的事情,造成这些政权在绝望之下不得不孤注,一直把希望完全投注在军事层面,而这个军事层面是俄军占领不败之地的。



王孟源 01:11:05 

这个在战略跟政略上就非常的高明,对不对?你看他这样引导下来,美国在一年之前做这个计划的时候,而且我说美国在一年多前就已经计划好了,为什么呢?因为他从阿富汗撤军了,阿富汗打了 20 年,一直都不撤军,明明就是不可能打赢,为什么不撤军?因为军工集团要钱,你不能够没有战争,那军工集团会同意让你撤军,为什么?你必须要



唐湘龙 01:11:37 

有另外一场战争



王孟源 01:11:38 

马上要开始打战。他在一年之前就已经准备,要作战的时候,是没有想要在军事战场上打赢的,但是Putin故意的示弱,这反而才是符合孙子兵法,“能则示以其不能”。这是孙子兵法里面这个欺敌的一个非常高明的手段。已经不是我们简单的说速战速决这种基本的原则够涵盖的。所以我觉得你现在回头去看,他表面上似乎是违反了孙子兵法的原则,实际上是运用的理念更高。



唐湘龙 01:12:27 

感谢部分略过



唐湘龙 01:14:00 

龙哥能不能每个月加开一档专门的王博士?我不是说了吗,不是我的问题,他是很内向很谨慎的人。



王孟源 01:14:13 

我的原则是我要阅读20个我才愿意发表1个小时。



王孟源 01:14:21 

我不是空口说白话的。我刚刚在节目里面批评说啊,小学生去查雨季,其实你看看目前台湾的政坛,愿意去把基本事实到维基百科去查一查的,已经算是好的了。





感谢部分略过\twocolumn[\begin{@twocolumnfalse}
\section{在乌克兰战场}
\subsection{20221125}
\end{@twocolumnfalse}]Credit: 栗子, Anonymous



唐湘龙 00:14 

好,欢迎来到龙行天下,我是唐湘龙,星期五的时间 9 点半的时间,今天龙行天下的来宾是我的固定的来宾。那当我的我有固定来宾上节目多半都是我很轻松的时候,其实我跟大家一样都在上课。好,今天在我们的在这个龙行天下的讲座的人在美国的,那大家非常喜欢的王孟源在线上。



王孟源 00:53 

大家好,很高兴再来跟大家聊聊。



唐湘龙 00:55 

唉,今天很棒,今天声音也很棒。好,那因为我为什么要王孟源?就如果之前没有听过特别说明的朋友再简单说明一下,因为我长时间我会去关注王孟源的谈论,那一些访谈,实事的谈论跟他的分析。那我觉得孟源对很多实事的分析是兼有新闻跟学术观点的,那这个是很特别的。好,那我觉得因此在听孟源在分析一些实事的时候,许多人会觉得干货很多,收获很多,是因为它里面的带着很强烈的这种的学术的观点跟学术的论证能力。



唐湘龙 01:37 

好,那今天我们的观众朋友们来准备呢?听孟源的,我开给孟源的菜单,我坦白讲是我开的,开了之后就这样的王孟源去做菜好了,今天我们从,还是从俄乌开始,那不只是俄乌,我每个月跟孟源连线,我都在关注俄乌,因为这件事情对世界局势影响太大。可是到了最近这个月,从上次的访谈到这次的访谈,俄乌战场上面发生了非常大的变化,一个是在南部的赫尔松,第二个是俄罗斯对于对乌克兰的,就是说战争的处理的方式,虽然像这两天的时间在乌东的这个这顿巴斯地区来讲,看起来俄罗斯颇有斩获,那他对一些的乌军部署了 8 年的防线,制造了几个破口?第三个就是波兰事件,好,我让孟源开始好了。在这个月的时间,为什么觉得俄乌战争当中整个气氛不一样了?包括美国似乎开始改变态度,希望俄乌能够进行和谈。为什么?



王孟源 02:42 

好,我们先从Kherson这个大撤退来,Kherson的撤退谈起,那如果是关心新闻报道的听众们应该都知道这个撤退是非常有序进行的。他在宣布之后,一天 24 小时之内就完成,这是非常惊人的效率,然后乌军隔了三四天以后才非常胆怯的,小心谨慎的,去占领它的市区,那其实这件事情在 3 个月前就有端倪,就是当时Kherson的一个市政府的,当地市政府的官员——就是原本就是市政府官员在俄军占领之后留任的一个官员——他说漏了嘴,他说是不是要把平民撤离,那当时就已经解释了撤离的原因,撤离的原因不是因为前线的军事吃紧,而是他怕那个Kherson大坝。Kherson刚好是在乌克兰最大那条河Dnipro的出海口,那它的上游就是全乌克兰最大的水坝,这个水坝不但蓄水量远远是整个乌克兰境内最高的,而且它还是Crimea的运河,也就是淡水来源的起始点。



唐湘龙 04:10 

克里米亚运河的供水。



王孟源 04:12 

对,在这个夏天,甚至在我们两三个月前谈那个他的,乌军的夏季攻势之前,他们已经在那个水坝那边试图用小艇,用一些无人艇,无人机之类的,还有火箭弹,还有火炮去攻击目标,一个是隔岸的Zaporizhzhia 的nuclear power plant, 就是他的核电站。



唐湘龙 04:45 

扎布罗热核电站。



王孟源 04:47 

另一个,很不幸的,就是大坝本身,所以大坝本身已经挨了好几炮火箭弹了,那这个乌克兰的民政官员,他提到,说漏嘴,说我们要我们应该积极安排让平民撤离的时候,他解释就是因为没有办法百分之百保证那个大坝不会被破坏,而大坝如果被破坏之后所造成的洪水后果太过严重。因为估他们可以事先估算,现在都有电脑,估算是整个市区都会被淹,而且最深的地方是好几公尺,就是。你可以想象如果台北市的上游有一个三峡大坝,然后它有可能被攻击的话,那这个感觉就是那个样子。那在而且这个撤离不只是要撤离当地的平民,而且还必须要撤离军队。这个原因是因为,1。如果有洪水下来以后,电脑的模拟也已经显示至少要一个礼拜才会消退到能够搭架浮桥的地步,也就是说在一个礼拜之内都没有向前线补给的能力。那这样子一来你的就有军事的风险,就不只是平民死伤的风险。所以其实在 3 个月前俄军,俄国就已经在评估是不是要从Kherson撤军。



王孟源 06:32 

这个撤军的原因就是你把整个市区,因为市区的主体是在北岸,或者说西岸,就是乌克兰的那一边,他你如果撤的话,撤道河的另一边,靠Crimea的那一边的话,乌克兰占领了市区以后就没有理由再去袭击大坝来放水,那这样一来你的那个Crimea的淡水供应也没有事,然后战争结束之后也没有必要去重建那个大坝,那一定是非常大的工程,要花很多时间很多金钱比他们这次去修补那个刻赤 bridge 就是克莱米亚的跨海大桥,要麻烦很多,大概要贵两个数量级以上。



王孟源 07:23 

所以,当时就已经有密切的讨论。但是上个月我上你的节目时候还没有谈,因为这个问题是俄国。其实是有两个选择,一个选择就是撤离。另外一个选择就是加派防卫部队来防护那个大坝就是,但是问题是现在乌克兰拿到英国的跟美国的一种一些先进的兵器,你连北溪都炸对不对?那你去炸北溪的那种无人艇,刚好也就适合炸大坝了。那事实上他们不是水下的无人艇,而是水上的无人艇已经开始参与作战了,就是在他们从Odessa放这些无人艇去袭击在Crimea的黑海舰队总部,那像这种事情,你看看就觉得很难做到万无一失。那为什么俄国不提早撤退?他的原因其实也有点好玩了,就是他刚好要等美国的期中选举完了,对不对?



唐湘龙 08:33 

他就中选举一结束,投完票了他就撤。



王孟源 08:39 

对了,他不愿意让拜登捞到便宜。他们也知道这个去炸大坝这件事情就是为了要拿下Kherson,而要拿下Kherson并没有什么战略意义,它的真正的意义在于政治方面,帮助民主党选Mid-Term,所以这个俄国人也是赌气,就一直拖到选举一完,马上就宣布这个。而且他还有意的炫耀了一下他们的效率。我觉得整个事情你作为一个观察者来看的话很有意思,就是他们这些事情都玩了一些小心眼就是。那他的这个这件事情发展,它的战略意义就是,当然既然你把北岸让给了乌军之后,他们要炸大坝更方便了,因为他们现在已经占领了北大坝的北边嘛,所以根本就不需要用什么先进的潜水艇,直接就用工兵去把炸药袋面连线连好就行。那这下一来这代表的是什么呢?就是俄军在冬季攻势,我上个月讲过了,我预期他们会在 12 月初发动一个强大的冬季攻势,然后将会对乌军造成很大的摧残,就是有三个月的时间是从俄国传统的反守为攻的一个时机,那在这个时间内他没有办法越重新越过 Dnipro 的下游来攻占Kherson,也就是这其实是要你要占领Odessa最短的距离,就是直线的路线,但是他们就等于是放弃了。



王孟源 10:32 

就是你不可能,一旦从Kherson撤离以后,你就是放弃了这条攻势的路线,那这样一来他们的这个攻势就只剩下两个选择,一个就是顿巴斯,就是原本争议的乌东的Donbas顿巴斯,那就像你刚刚 10 分钟前提到的,在过去这一周他们已经开始紧锣密鼓的加强了攻势。那另外一个选择就是从Kharkov,北线的Kharkov跟Kiev之间的那个乡村地区,乡下地区插下来也基本上就是,把战区局限在第聂伯河的东岸去了。这个是啊,这个撤军的实际的影响是这样子,是未来这 3 个多月的,我在上个月提过的预期的会战事的激烈程度会升级的时候的过程中,你可以预见这个战区会局限在Dnipro的东岸,因为他们俄军不再有很简单的渡河攻击的手段。



唐湘龙 11:57 

好,我们会感觉到就是说俄罗斯正在举准备一场,就是说冬季的大反攻,那这点跟美国不断的释放出就是说希望乌克兰开始跟俄罗斯谈判的这个气氛有关,不管是拖延战术了,或者实际上面认为这场的冬季攻势,那可能乌克兰是很不靠谱的。但是我们再来看,就是说俄罗斯最近对于乌克兰的攻击的形态,除了乌东战区看起来仍然是它的重点,仍然是他特别军事行动真正的重点之外。第二个就是说它的导弹的轰炸,其实前后已经六波,其中有三波是大规模的,大概都近百枚的导弹的供给,而供给的都是它的能源跟电力设施,我个人我觉得这波的攻击看起来是有效的。你怎么看?



王孟源 12:53 

我先讲个笑话。这个其实你如果回去看英美的新闻报道的话,从2月底开打不到两个礼拜,就是3月初他们已经有报道,就是像是英国的,像telegram,美国电信报 8 年,或者是美国的 New York times 或者 Washington post。他们基本上每个礼拜都会出新文章。估计俄国的导弹要用完了,因为你可以看得出这个俄军跟美军不一样,美军从来没有面对一个完整的先进的防空体系。就是在二战之后,虽然韩战跟越战算是打得很激烈,但是对方都没有真正像样的防空体系。那后来在中东打了几场大战,那更加不用提了,那个叫摧枯拉朽。但是乌克兰是有从苏联继承下来,因为正因为苏联的空军优势,空优,不像北约集团那么强大,所以他一直很注重发展它的防空武器。所以苏联还有俄系的防空武器一直是世界上最先进的,比美国的所谓的爱国者飞弹要先进多了。你可以光看爱国者飞弹没办法,打了就跑,你就知道这个是差了一大截。因为你一旦发射以后,无无论如何你的位置是暴露。那你看这个,苏联的 S300 系统是打了以后同一分钟之内就可以跑掉,那爱国者是二三十分钟撤收,这光这个就是你可以看出他们用不用心。因为乌克兰本身继承了很多 S300 导弹,还有其他比较短程的导弹,所以苏联俄国这一次打击的时候一直没有用美军惯用的制导炸弹,美军制导炸弹的好处是它便宜了,因为你这个战斗机或者轰炸机飞上去以后就是花点油钱。那不过正因为苏联的这个传统,所以它更看重长程的巡航导弹,所以我想我在三个月前,在你的节目上,还是四个月前,我就说他们,估计他们已经打了 1500 发到 2800 的导弹,从那到现在大概又打了一千五百八左右,这个数量是非常惊人的。然后所以我刚刚说这些盎萨系的媒体从3月初开始到每个礼拜都会说是俄国的导弹已经打完了。那我这个让我联想到的是什么呢?就是有一个老笑话,就是戒烟,非常非常的简单,我已经戒了十几年了,每天都在戒,



唐湘龙 16:04 

就是马克吐温的笑话、



王孟源 16:05 

对,马克吐温的笑话。那这个所以你要把俄国的导弹用完也非常的简单,你每个礼拜都把他的导弹用完了。事实上要摧毁恶俄国的经济,或者是要看中国的经济崩溃,也是非常的简单。你如果看盎萨的媒体的话,也是每个礼拜都摧毁了他们的经济,但过去,哈哈哈,不知道有多久了,哈哈哈,那照马克吐温的逻辑来说,那都是轻而易举的事,因为他们都已经干了几百个礼拜了。不过言归正传就是,这个导弹的升级刚好就是我们上个月讲过,我说我预计它会有一个冬季攻势,所以你准备这个冬季攻势当然就是要摧毁它的基础设施。那摧毁基础设施你可以做全面的摧毁,就是去摧毁水电油气,还有通信这些,还有道路桥梁。但目前你看到俄国摧毁的只是电力,那这其实是不是偷懒,而是电力是一切的基础。就是我在两年前,曾经我家曾经因为恶劣的天气曾经失去电力供应。 4 天,我可以告诉你,哈哈哈,这个非常的不愉快。



唐湘龙 17:44 

当然,当然。

’

王孟源 17:46 

对,就是两天之后。所有的啊,你的附近的超级市场,还有食品店的那个冷冻食品都已经解冻了,然后……



唐湘龙 17:57 

连手机都没电了。



王孟源 17:58 

对,然后手机也都没电了。换句话说,你如果去问一些真正的专家,他会告诉你,如果失去电力供应,整个区域失去供电力供应超过一周的话,你的自来水供应会有问题,因为你的那个Pump没办法运。没错,你的废水处理会有问题。,同样的也是因为Pump没有办法运作。你的通讯那更不用说了,因为通讯那些都是要吃电的,而且是吃电的大户,那也这些都会有发电机,但是那些发电机的那个柴油供应,最后你如果是没有电力的话,整个这个油气的供应都会停下来,然后最后的这些所有的这些其他的设施也都会停,那这个时间大约是一两个礼拜。所以俄国其实只要把电力全部抹杀,那他也倒没有一开始就急着百分之百的把整个电力系统摧毁了,而是打一波停下来三四天评估一下,看看他们这个重点的修补是在哪里,然后再打一波,然后再停下来评估一下。就像你讲的打了五六波了,那目前的估计就是我想应该是已经有 60\%- 70\% 被摧毁了,那而且因为乌克兰他们用的这个电器设施的标准是苏标的苏联标准,那西欧的北约的这些国家并不能很简单的送新的变电器,因为这一次他轰炸的这个重点是变电器。一直到礼拜三的这一波才,而且一直到礼拜三这一波才打发电站外面的那个变电器,就是原本都是打那个传输系统的变电器那,但是两天前他开始打发电站外面的变电机,那这样一来发电站就没办法把电力往外送。那就是乌克兰的剩余的那些核电站也都必须要关停。那核电站这种东西你也许有点印象,就是他如果你关停之后要好几天才能再打开。



唐湘龙 20:28 

他升载很慢,降载也很慢。



王孟源 20:30 

对,因为它这个里面会有poison……这个我想不要讲的那太远了,反正就是你如果硬是要急着弄的话,就会发生,当初Chernobyl就是因为急着要弄起来,所以出了问题。所以你看我预估在一两个礼拜之后,冬季攻势正式开始的时候,俄国应该可以做到基本整个整个乌克兰供电系统拉黑,那这样的它的后勤补给,包括它的,尤其是它的铁道系统,大部分是靠电力推进的,就没办法往前送,那这当然会对他的战事有很大的帮助。我们在2月三月的时候观察他第一波打击的时候,就是很多人,包括我自己在内,就是说你怎么不打这些基础设施?这根本不像是认真的在打,那你从现在这个基础设施,尤其是电力网,电网被摧毁的这件事情,你可以看出俄国是真的要认真打了,也就是我一个月之前讲的这个,可以看出他认为民气可用。而且现在你如果注意到过去这一个月俄国内的媒体报道他们的论的话,可以看到Putin慢慢的等他的全国人民,真的是有效的,现在全国人民已经跑在他前面了,那个基本上他们的舆论都是在骂他,为什么就是下手不够狠?



唐湘龙 22:15 

当普丁他自己成为俄罗斯最保守的一个,就其他人都已经比他更激进了,那这个时候它的动能就出来了。



王孟源 22:27 

对,就是事实上现在俄罗俄国有一个笑话,,就是这个北约如果要鼓励政变,把普丁搞掉的话,赶快跟普丁和谈。



唐湘龙 22:38 

哈哈哈哈哈,他是俄罗斯的温和派,他现在俄罗斯最温和的一个。你不找普丁谈,没有人可以谈了。



王孟源 22:48 

因为你如果跟Putin和谈之后,整个俄罗斯就会民意沸腾,就会把它搞成一个……那你所以我想其实这个战争的本身就是对俄国跟乌克兰的发展,我已经解释得很清楚,上个月就解释得很清楚,没有什么需要再做增补的。所以我们现在就谈谈你刚刚想问的,这个真正对世界其他地区的影响。



唐湘龙 23:22 

好。但其中有个附带的问题就是波兰事件,因为在波兰事件里面我们关注的就是说你看到现在为止乌克兰都没有承认,就说那两枚的导弹是乌克兰的,但是你从它的射向呢来看,就算你是防空导弹,应该是往东边射,往北边射,那为什么是往西边射?这个不合理,所以大家觉得里面是有鬼的。



唐湘龙 23:48 

那波兰算第一时间也是跟着乌克兰,同时也透过像这个我记得是美联社,美联社还是路透社,美联社把他的记者给 fire 掉了,就是第一时间发出来新闻是错的,他还跟大家去道歉,说他的内部的茶和不实。可是美国不是美国,那个时候拜登正在东南亚,七国集团也都在东南亚,可当第一时间拜登就已经是做了判断,就是说那两枚不会是俄罗斯射的,这跟过去不管是北溪二号也好,等等,或者是像是克里米亚大桥被炸也好这些的事件里面,高度的危机性的事件里面,美国要不然就是说就是否认,要不然就是不置可否。可是这一次这么攸关到北约体系的这个动员的,为什么美国的反应这么快?以及伴随着周围的美国的官员,包括Mark Milley,包括布林肯所释放出来的希望,乌克兰开始准备跟俄罗斯谈判,为什么?



王孟源 24:53 

其实这件事情美国异常的公开,就是当时拜登接到消息,然后跟Blinken跟Sullivan开紧急会议的。对,在巴厘岛上你可以那照片都发出来。对,没有错,非常的疲惫,这是过了半夜三经还在那边紧急讨论,(唐:他们都穿拖鞋啊)这件事情显然不是他们事先预定的。显然是突发的事件。那很多人为了这个方向大做文章,可是事实上它明显的就是 S300 的导弹,而且是乌克兰装备的那种第一代的 S300 的导弹,而不是俄军现在用的第二代、第三代、第四代的S300,S400 导弹,那这个导弹它最大射程只有 75 公里,那导弹的落点距离最近的俄军有六七百公里,所以你不可能是从俄军占领区发射的,那你连距离白俄罗斯也有 100 多公里,而且你如果是从白俄罗斯有导弹发射的话,那是在北约的前线,他们怎么可能雷达没有看到?,对不对?这件事情美国之所以可以马上确定他哪里来的,是因为那个地方原本就是北约面临俄国集团跟白俄集团的前线,所以原本就是有雷达网盯着看。所以它原本就可以确定这个飞弹发射的地点在哪里。就是在Lviv就是乌克兰西边那个城市附近,至于它为什么不是对着俄军巡航导弹的来的方向,这很简单,它的电子导航系统故障。我刚刚提到这些导弹已经有 40 年久了。好,OK,它一发射就跑出方向,然后跑到最大射程去,所以很不幸的掉下来,还居然打死人,这个几率其实并不大,但是偏偏发生了,这是很不幸的事情。那至于为什么美国不想炒作这件事情,而且乌克兰死鸭子嘴硬一直到现在还不肯承认,然后波兰一开始也是附和着。然后还有当然盎萨的媒体,像你所讲的,美联社一开始高调的炒作,后来必须要转过头来。



王孟源 27:32 

这个开除那个记者其实是所谓典型的 damage Control就是弃车保帅。那就是知道啊,自己太兴奋了,跑到太前面了。然后老板跟你讲说你在胡搞些什么的时候,他们赶快找个替罪羔羊推出去宰掉。那这个原因很简单,你如果是我到的博客去看的话,我不晓得讲过多少次了。美国立国 200 多年来,自从第二次开国战争,或者叫做The war of 1812,就是 1812 年被英军攻占白宫的那一场战争,之后他从来没有单独主动的去挑战同一级的强权。,你自己想想看,他对外用兵 400 多次,200 年来对外用兵 400 多次,哪一次是自己主动单独的去面对世界一流强权?没有。那这一次的这个挑动俄乌战争,我也是从一开始就讲,它的这个用意百分之百就是为了要在经济战线上扼杀俄国,然后在政治战线上拉拢北约,就是确定德国跟法国在美国在东亚——它真正的目标是在东亚打击中国,因为中国才是他的,挑战他霸权的第一挑战者,也就是他的第一目标,他的真正的目标,政治上的第一目标是为了要在东亚搞事,也就是在台湾搞事做准备。那这个做准备的意思就是到时候一样也不是美国人自己上,也是叫小弟们上。这个小弟们,可是在东亚呢,只有日本跟韩国跟澳洲,这个是不够分量的,你要打击中国的话,一定要有欧盟的全力支持。那你如果一开始就在东亚直接动手的话,以中国跟德国的贸易关系之密切,很有可能会拉不动他们,所以先在乌克兰搞事。那因为俄国跟乌克兰就在欧盟的门口,所以可以把整个北约拖下水,以后有了惯性,你才能够接着在东亚搞事,顺理成章,这是它的如意算盘。结果在那个经济上面被打垮了。现在俄国是全世界经济最好的地方。



唐湘龙 30:28 

没错。



王孟源 30:29 

那他基本上,俄国现在当时被制裁的时候,一个最大的影响是他的机械制造业,尤其是汽车业全部都停工了,因为他的那些汽车制造厂基本上都是跟欧洲厂商合作。没错,他的最大的汽车厂,比如说是Peugeot标致,标致Peugeot,那Peugeot在第在头两个礼拜就宣布我这个,我放弃这个工厂,然后由莫斯科市政府挖了 1 卢布把它买下来。那现在这个工厂已经完全复工了,现在已经完全恢复百分之百的生产,他没有讲它生产的车型是什么,但是人家看了一下说很像大陆的一型车,所以基本上是中国派了一个车商去把他接手。



唐湘龙 31:28 

中国在帮他。



王孟源 31:31 

然后又完全的把在媒体上面是掩盖着,一句话都没讲。那所以在经济战上是俄国已经大获全胜。在政治上,因为正因为经济上 EU 欧盟吃了大亏,他们在过去这两个月也开始理解被美国给坑了。就是因为美国的那个,石油供应是自给自足的,然后对天然气还可以外销,这一次的这个对俄国的油气禁运,美国是大赚了一票。欧盟的工业产业外流非常的严重,那这个是过去两个月慢慢的开始有人拿出来提这个,原本就已经是有点危险,但是我觉得最大的转折就是德国的总理Scholz,他从上台到现在一年是非常软弱无能的。



唐湘龙 32:47 

这次竟然硬起来了。



王孟源 32:48 

对,这一次是被,他们的工商业被基本上把他押到北京去,就是我不管你,反正你要带我们去,然后要去跟中国投诚,说我们还要继续做生意,我们绝对不会因为对俄国的种种不友善手段而。转过来把你也当做敌人看待。基本上就是为了这一点,基本上他是对这些工商业的资本拿着枪指着头说走。他去的那个整个旅程是非常狼狈的,就在北京待了 8 个小时,然后中方对他也没有什么好眼色看,但是他不得不去那,去了以后去的时候还拒绝带马克龙去,马克龙说你要去的话我也去。,那拒绝马克龙的意思,就是我这次是替我们德国的产业去求情的。我带一个法国那个不方便。



唐湘龙 34:00 

哈哈哈,对,就那个逻辑,大概就这样子。



王孟源 34:04 

唉,逻辑就是这样的。但问题是一旦这基本上就是德国对中国软服了,所以我们这个这次跟俄国人搞得灰头土脸,我没有底气再跟你敌对。那你可以看出这过去这个月接下来是哪些高官出来。荷兰的人,荷兰人站出来说我们不会再盲目追随美国。法国人也站出来说。然后最重要的是欧盟的外交部长,他没有外交部长,但是他等同外交部长叫外交专员Borrell,也站出来说我们珍惜对跟中国的关系。那这个就很可怕,因为Borrell本身在对俄发言的时候是极端的强硬派,也就是Neocon的态度。



王孟源 34:59 

一个土生土长的欧系的Neocon也出来说,我们不能够学美国一样来对中国强硬,那基本上就是等于欧盟正式接受这一次他们已经没有实力在参加北约,在东亚搞事。那这样一来的话美国的如意算盘也就垮了,这也才会有习拜会。讲了三个多钟头的原因就是Biden了解到他如果这在未来几个月在东亚也台湾也搞事的话,欧盟不会跟他,那欧盟不跟的话只有日本韩国,日本韩国恐怕也没有那个胆量跟,因为这种东西,他们自己左顾右盼都没有人上的话,那不是?美国大喊大家给我上,结果第一个欧盟不上了,后面大家都在那边左顾右盼,那不是反而很尴尬嘛。这是为什么?现在美中美的关系忽然缓和的原因就是Scholz被他们德国的产业界给强逼着去认输了。去完,那至于,但是美国现在连对俄国也开始讲要和谈,这是另外一回事。我跟你解释一下为什么:原因是这个国防部他们那些职业军人也已经看出这个冬季攻势要来,你想想看,连我这种,我的官阶最多就只有到少尉,对不对?



唐湘龙 36:46 

我懂了。



王孟源 36:49 

我都可以看得出这个冬季攻势马上要来了 40 万,甚至可能 50 万的大军,因为他们还有很多那个自愿的兵,还有那个地方民兵都可以动用这个。他们只要,他们这些将军们身经百战,而且又有卫星说一大堆雷达的消息,他们怎么可能看不出这个冬季攻势会有多么厉害?他们知道这个乌克兰的部队一定要吃大亏,乌军在未来 3 个月要吃大亏。那你想想看,他们或许想不到,我上个月跟您解释的这个Putin一直是故意的,要等到民气可用,才真正发动攻势。但是从纯军事的角度来上,也可以看得出这是山雨欲来风满楼的情势。那你想想看他们的职责,当然是立刻把这件事情解释给拜登跟他身边的那两个人,布林肯跟Sullivan来聊,他们在解释的时候,我设身处地去想一想,如果你是一个职业军人,从你的专业角度你知道俄军即将大幅反攻,而且会而且乌军基本没有可能抵挡得住。这时候我怎么解释给他们听?这些人都是政治性很强的,而且是Neocon,Neocon就是它的一个特点就是不听专业意见,然后,你越是面对困难越是要加码,那你如果要想要避免他们做出这种反动加码的事件的话,我想我如果是我的话,我会跟他们解释,就是我会特别强调从专业上可以确定未来乌军的前线只会一步一步后退,而且可能退得很快,这已经是他的目前的这个占领Kherson之后是乌军在整个战争里面的最高的水线了。就是你不可能再往上涨了,那从这边只有慢慢的往后退,慢慢的或者很快的后退的两种差别。你如果把这一点解释清楚了以后才可能说得动萨勒本跟布林肯这样的人。果然Sullivan本就很快的飞到欧洲去要求和解。



王孟源 39:44 

事实上在Sullivan飞过去之前,一天的,好像是一天了,那个美国国防部长,Austin他为了保证为了保证他的这个信息 message get through,就是这些政客们能听得下去。他还有意的泄露了这个消息给 New York Time。New York Time有一个重磅的独家新闻,就是国防部,根据国防部高官的评论怎样说这个应该要赶快和解,他也没有讲说这个乌克兰马上就要被打垮,因为他不敢讲。他们那个那篇文章讲的是俄军,非常非常的邪恶,但是和解有助于国家最大的利益,这样的就是东扯西扯的。但是实际上他们要讲的东西很明确,就是现在和谈的这个时机的一去就不再复回。因为眼看着乌军要溃败了,那所以这是,美国向乌克兰施压和谈的原因。那这个其实跟那个,这个其实跟导弹调到波兰那个事件原本是两个平行的事件,就是没有交集的,他们只是很巧。那个掉导弹掉到波兰的事件偶然的发生了,发生以后美国被逼,因为他们——我想我在两个月前上你节目都讲过——仅很用心的警告他们这个核战的可能,就是当时讲是那个脏弹,他们怕的是脏弹,然后第二个是他说他怕的是北约用战术核弹来打。他一直警告说你如果用战术核弹的话,我就会反击,而且可能是战略核弹来反击,所以原本美国就没有打算要自己下场,那一旦经过Putin在过去这两个月的警告之后,更加的坚定绝对不下场。



王孟源 42:11 

而这个你如果接受乌克兰泽兰斯基的那个栽赃,说这个导弹是俄国来的话,那在法理上北约必须要宣战。因为北约的第5条,条约的第5条规定你受到打击的时候必须要整个所有的签约国都一起一起参与战争。那他是因为避免升级,所以才急着撇清,才急着深夜开会,开会完以后马上打电话给波兰总统,说你这个不能够胡扯,因为你这样扯下去会是第三次世界的大战,然后美联社才必须要找戴罪羔羊,才必须要做damage Control。



王孟源 43:06 

那但是Zelensky 他,他的地位又很特别,因为他现在基本上是傀儡,就是那些Neo Nazi的一个傀儡,实际上掌权的是那些Neo Nazi,要叫他继续栽赃,他就只好继续栽赃,那这样很不识趣的去弄,就反而引发反感。因为美国很明显的不想打第三次世界大战。乌克兰却很明显的是希望能够挑起第三次世界大战。这是他们的一个基本矛盾,那美国原本是想要把它这个,就把它掩盖过去了,结果你乌克兰Zelensky不识相,想拼命的去反复的去揭这个疮疤,去挑这个矛盾,那这样一来实际的影响反而是我想我上个月提过这个Zelensky的地位在冬季攻势的结果呈现出来之后会有影响,那他这个其实是对他自己挖墙脚。就是现在这样子,西方世界的群众跟政客们都对他的印象很不好,就是觉得他不识相。你把这个的最基本的矛盾,应该要全力掩饰的东西拿出来,反复地加压来硬是要谈,真的是很不懂事。那至于他对他还有另外一个影响,就是Putin当然也看出,也就安心了,因为他不管对北约放话多少次,不管你的那个核反击的威胁讲提出多少字,你都不能够百分百确定对方能够听得下去,尤其对方是Neocon嘛,对不对?Neocon就是最疯狂的,但是,这下经过这次波兰导弹的事件,它就可以确定这个冬季攻势可以去放手的打,而不会继续升级,那所以基本上这一次这个导弹落在波兰这个事件的实际的影响是帮上个月预期的那些路线,它的几率又提高了一些,更加的稳定了。就是现在俄国能够更放心的更放手的去打,而泽兰斯基的地位则更加的危险,就是更可能被拿来当替罪羔羊,当你的战士失利的时候被推出来说是你的错。然后弄下台,然后怎样呢?反正美国对失败的时候总是找盟友的领导人来开刀。



唐湘龙 46:06 

没错,所以现在就Zelensky,现在是可能是全欧洲跟这场战争有关的人,现在风险最高的,他的每一步的风险都很高。那现在西方国家美国现在刚刚孟源的分析大概很清楚了,我也都同意。就是美国这一波希望乌克兰谈判看起来不是一个缓兵之计,不是一个假动作,而是真的认为继续这样子打下去,你乌克兰会受伤惨重,而且美国曾经帮助你的,北约帮助你的都讨不到便宜。那这个接下去就看泽伦斯基如何找台阶下,避免自己敬酒不吃吃罚酒,一旦北约或者美国真的跟你翻脸的时候,泽伦斯基最后被抛弃的机会还是非常大的。



唐湘龙 46:52 

好,我们转一个视角,刚刚提到就是说俄罗斯的这场的战争,因为它还没有到最后,要等到这场的冬季战事结束之后,我大概来做一个总结,可能比较适合。可是整体看下来,俄罗斯我认为他在三个地方大概都稳固。一个是在它的金融面,刚刚孟源提到他一开始的时候,它的金融面,它把卢布跟能源跟黄金绑在一起,这个是非常高明的,所以它的经济几乎不受影响,它的收入几乎是不受影响,比过去还要来得更好。第二个在军事上面来讲,它对乌克兰的这些所谓的能源设施的供给,看起来是一波非常成功的战术性的攻击。在这场的冬季战争里面,俄罗斯站立于不败之地,所以他在乌东的这个防线,乌克兰的军队准备了 8 年,所以说布置的乌东的防线可攻可守,但是最近陆陆续续被攻破了几个缺口,冬天在乌东地区,乌克兰可能会非常非常的惨。



唐湘龙 47:56 

好,那我们同样的倒过来看,刚孟源提到就是说像俄罗斯的经济情况可能是全世界现在最稳定的,就是说经济体之一。另外一个在今年经济成长率表现非常好的是沙特,沙特同样的是这次俄乌战争的受益国之一。那沙特当大家注意到从拜登上台前过去,他跟川普跟特朗普的关系维持的还不错,可是拜登没有上台前的时候就已经摆出一副作,我上来一定修理你,MBS,修理你,这就是说这个沙特态度已经很清楚,我要把你变成是国际社会的贱民国家。



唐湘龙 48:38 

好,那拜登的姿态一直很高,一直到这一次的其中选举。米特恩之前的时候,因为沙特所代表的欧派克跟欧派克 plus 联手,降低了石油产量 200 万桶,非常不给拜登面子,拜登跟你甚至听到包括布林肯都讲狠话,就是你,你等着,我一定要修理你,这件事情美国人一定会报复,可是选完了之后你会发现不是这样。你会发现现在呢,美国好像急于要去修补跟沙特的关系,一个大的背景当然是沙特现在跟中国的关系进展太快了,现在是 11 月底,当然全世界包括全中东都在关注的是卡达的世界杯,大家都在看看是世界杯,可是在世界杯除了你看到这一次是卡达的中国杯,除了球队没有来之外,整个卡达的从场馆,从运输、从能源、从它的相关的供水系统等等。从他的这些奖杯球,这个球场的这些电子仪器系统,全部都是中国制的。



唐湘龙 49:44 

卡塔还跟中国签了 27 年,大概是我看过的所有的能源国家跟个别国家签订的供需的双方的能源协议当中,最长的 27 年,而且可能是用人民币计价。 11 月底如果卡达世界杯结束之后,我个人认为接下去中东的中国月来了,大概习近平到做到沙特的访问大概是确定的,因为他在之前的G20他没有跟沙特见面,习近平在沙特在中,在G20 跟 APEC,他见了 20 个国家,没有。沙特表示习近平一定准备专访沙特,所以 12 月大概就会是中东中国月,那美国对于沙特的关系,对中东的关系,他慢慢的失去了控制了吗?为什么他急于对沙特做这种场面性的表态,把 MBS 排除在司法的追溯的这样的一个特权的范围之内,为什么?



王孟源 50:40 

这是一个最后的挽回。因为我们 3 个月前其实就谈过这个,习近平可能会访问沙特,那时候是 Reuter漏风。那它的用意当然是听到传言之后,由美国的情报的官员把他泄露给Reuter,然后叫Reuter去让他见光死,结果呢……



唐湘龙 51:03 

他在警告沙特,那个时候。



王孟源 51:06 

警告沙特,对,就是说我们知道了,你给我小心,但是沙特这个 MBS 不理他。,那所以这基本上是美国最后的软服,就是希望能够这个来硬的不行,来软的试试看。那事实上美国现在已经被逼到墙角,你刚好过去这礼拜这 3 天有一个另外一个事件,就是 Serbia 跟Kosovo。对。



唐湘龙 51:35 

赛尔维亚跟科索沃,本来很紧张。





王孟源 51:40 

因为Kosovo忽然要在那个汽车牌照上面宣誓主权,那这样子,那Serbia说你这样子当着我的面不给我,要用我们习惯的术语是什么?打破既成既有的格局。



唐湘龙 52:06 

默契啦,其实塞尔维亚跟科索沃,当北约或者说欧盟国家大部分都承认了科索沃,可是塞尔维亚到现在为止仍然强调科索沃是塞尔维亚的一部分,所以他绝对不会允许塞尔维亚有任何暗示独立的动作跟塞尔维亚切割。



王孟源 52:25 

对啊,其实跟这个台湾的局面是很相似,但是因为Serbia是,当然已经是 20 多年前被打烂过了。,所以是一是个软柿子。那如果不是因为俄乌战争实在让欧盟的经济给拖垮了,否则他们很简单的就会对Serbia施压,比如说你做一个经济禁运,制裁什么的,他根本被北约国家包围着。连那个货运都没办法做。



唐湘龙 53:05 

他没有出海口了。



王孟源 53:07 

对,但是你可以看出他们这个底气实在是不行了。就是这件事情最后是欧盟对Kosovo 施加压力,然后强迫双方和稀泥拖过去了。就是还,半天之前,十几个小时之前,还在新闻上。那从这一点来看出你就可以理解到为什么美国人也对沙特来示好,试图要讲好话,要来软的,那基本上是因为他没有那个底气了,基本上是因为欧盟不敢再另起冲突。那如果欧盟不起冲突的话,美国的这个这幅德性就是一向是送盟友上去送死的,不是,不会是自己抢在最前面。



王孟源 54:01 

那当然这并不代表我 3 个月前所讲的 MBS 被政变搞下来的危险消失的,事实上我想一定是在紧锣密鼓的筹备之中。问题是政变这种事情需要好几个月,你即使是总统下令给我加紧的去搞这个,也是要几个月才能够搞得起来的,所以我们不能够排除这个可能。那至于 MBS 本身在内部的布局是不是能够阻止这种事情的发生,或者还能够安然度过,我预期这个他们美国现在想要搞的政变会是类似 2016 年搞Erdogan土耳其Erdogan一场就是……



唐湘龙 54:51 

失败的政变。



王孟源 54:52 

由军方来搞。对,那这个,很难说,不过我觉得 MBS 看来是一个很Sharp。



唐湘龙 55:06 

比想象中聪明的人。



王孟源 55:08 

很聪明的人。就是你有几个,当前世界上有几个很年轻但是很惊人的老练成熟,老练的政治玩家。一个是金正恩。这个你他上台的时候才二十几岁,但是没有人想到他会手段那么的那么的厉害,那但另外一个是MBS,我觉得 MBS 真的难怪他的爸爸对他这么宠,他,这么看重他。那所以就正如你所说的,我也预期习近平很快的会去访问沙特,不是 12 月的话就是1月。现在可能是在确认那个签订合约的细节,还有确认这个安全保障,这些事情基本上就是未来这两个月了。然后这当然是最重要的墙角,因为我从半年多前刚开始上的节目,我就提过这一次的转折,如果要在棺材板上钉钉确认美国霸权的衰亡的话,是真正是必须要搞出一个替代美元的货币,那现在已经发展出来,我当时是希望能够发展一个合成货币,现在的确是要合成货币,不过那个名字我没有猜对名字。他们现在可能是要用金砖货币,然后这个金砖货币的最重要成员就是沙特,因为沙特加入以后就真正把石油美元给推翻了。那沙特跟这一次习近平访问沙特,除了要签很多双边协议之外,我想最重要的是要确定沙特加入金砖,而沙特加入金砖最重要的意义是沙特加入金砖货币,就是我预期他在自访问的时候,跟中方做能源贸易的时候会用人民币,然后等到明年沙特加入金砖,然后金砖发行,金砖集团发行货币之后全面改用金砖货币就是暂时的半年多的一年。



王孟源 57:41 

只有跟中国做生意的时候用人民币,跟其他国家做生意仍然继续采用美元,那中国的采购沙特的原油出口占的比率大概就是 20\% 不到。没错,那这个本身只有象征性的意义,还没有真正的触到它的市值,但是一旦全面改用金砖货币的时候,那就宣告美元的死亡,那时候刚好也是美国的消费者跟企业界的,他们的储蓄用的差不多的时候,就是 2024 年初的时候,他们那个时候如果经济衰退的那些毛病没有解除的话,然后美元又忽然被抛弃,被国际抛弃,然后他的会崩溃的话。这有可能在英国跟欧盟之后,美国承受同样级别的经济危机。不过在目前为止,目前就是在过去这半年多,然后再加上明年这一年,美国是不会有太大的问题,因为基本上英国跟欧盟的企业跟资金都是往美元转移的,对不对?所以他就靠着这个来控制它的通货膨胀,你可以看在半年前的时候,通货膨胀是美国先上去的。对,美国先到, 8 点几的,然后那时候德国还只有 6 点几,7点几,然后现在已经是很明显的英国跟欧盟的通货膨胀已经无法控制了。美国的通货膨胀反而是稍微缓和进来,这个原因就是这个汇率的运动。那也是,这些都是我事先就解释过的。半年多前解释过的,不过我想提醒大家再回去复习一下。OK,这样可以了解金融债这个现况。好。



唐湘龙 59:47 

我再跟王孟源借 10 分钟,我要关注一下美国的其中选举,虽然台湾的选举,美国的期中选举,我刚跟孟源在闲聊的时候,他说我这一次刚好都没投票。但这次的期中选举选完了之后,以现在选举的结果你看得出来,民主党觉得少输为赢,而且还蛮开心的,共和党反而觉得自己遭遇到了挫败,尤其是几个指标性的选区的指标性的候选人,那因此大家对于接下去的美国的国内政治会不会进一步的深度的撕裂?那当然是在国际社会当中关注的一点。好,你怎么看这个其中选举,这次其中选举之后对美国政治会有什么影响?以及 2024 年的总统大选什么影响?



王孟源 01:00:33 

首先这一次的期中选举是美国史上最贵的期中选举。



唐湘龙 01:00:42 

花了非常多的钱。



王孟源 01:00:43 

比 4 年前,要是 4 年前的将近 3 倍,就只有 4 年前哦,上一场期中选举。为了这个花费将近三倍,基本上是亿万富翁的选举。那这个在过最近这几年亿万富翁很明显的都集中到民主党去了。,原因是因为Trump搞民粹的话,这个他就威胁到建制派,也就是 deep state 这个既有的幕后的这些统治深层政府,那所以这一次期中选举其实是Trump试图要卷土重来,为自己下 2024 年选总统铺垫的做的一个准备。没错,而且在初选阶段他非常的顺利,他所支持的共和党初选的候选人超过 90\% 获胜了。



唐湘龙 01:01:41 

都被党内提名,都得到党内提名。



王孟源 01:01:44 

都被党内提名。对,所以一开始他这个风光无两。但是问题是你可以看得出,你自己设身处地去当那些亿万富翁,积极参与这个 deep state 阴谋的那些亿万富翁,比如说就是Soros,如果你设身处地,你当 soros 好了,你想想看,现在这个 Biden的执政水平怎么样?很糟糕对不对?,在搞这个俄乌战争,也看起来是已经在金融经济层面完全溃败,在内政上面乏善可陈,然后这个通货膨胀一副可能很可能控制不住,事实上是绝对控制不住的。这个我从我从 3 年前就一直这么讲。那么你说 2024 年总统大选如果让Trump出来选代表共和党,那么就很可能让Trump又选上。,Trump的第一个任期的时候他还不懂事。,他的手下的重要内阁阁员主要都还是建制派的。没错,你还可以管得住。那他在任期中所做的政策,事实上也都还是建自派 deep state 所想要的。,唯一没有的就是他没有去搞颜色革命。我以前已经解释过了,颜色革命是让这些政商权贵在国外搜刮最大的收入来源。他们,他们在本土去收那些企业界的贿赂,其实都是小钱,而且是辛苦钱,对不对?你这个到国外一次几万亿的那个资产可以贱买贵卖,那个才真正是大钱。那所以Trump的第一任真正挡到他们财路的只有不搞颜色革命这一点,但是他们怕的是Trum已经经过了第一次的任期,然后,然后过去这 4 年又在又在观众台这样子思考,然后也不但建立了自己的班底,而且可能了解要怎样动手进一步威胁建制派跟深层政府的利益。这是一个很大的危险。那一旦到了他被提名之后,你基本就很难阻止他,所以你要阻止他最好的时机是什么呢?就是这一次的期中选举。你要这个你必须要让他拿不到党内提名。那所以我从这一次选举,也其实是 4 个月前我的一篇博文我就解释过了,我一直这个这些深层政府的人会双管齐下。第一个是他们要挑出另外一个民粹派的领导人,来取代Trump的。



唐湘龙 01:05:03 

这个是像佛罗里达州的州长。



王孟源 01:05:05 

对,而且我那时候就讲得很明白,我认为他们会挑De Santis,OK,所以这个是 4 个月前我就已经在博客上讲明白了,他们的双管齐下。第一管是把De Santis推出来来争取民粹票,但是这个来的太晚,其中选举的时候的那些民粹派的候选人都是对Trump效忠的,所以第二步就是要把这些人都搞掉。那搞掉的这个,执行搞掉的任务的最重要的人就是Mitch McConnell。因为美国的选举虽然是有两个院,众议院跟参议院,实际上花大钱的是参议院候选人,你因为参议院的那个候选人,他基本上是整个州在选,所以你可以去买广告做电视广告,只可以你花钱比较有用力的地方有着力点。然后这些广告顺带着推你的全国性的政策,这样一来你就可以把那些众议员——那些众议员选区,基本上就是一个县那个大小——就可以把整个州里面那些县通通的一起带起来,母鸡带小鸡这样子,那这一次所以真正主管集中选举的人是谁?就是参议院领袖,那刚好就是McConnell 真正的建制派的大佬。所以他这一次去搞,你如果去看他怎么搞的,第一个是他克扣那个钱,越是对Trump效忠的民粹派的候选人越是拿不到钱,这还是小事。最重要的是他故意地没有打广告。我这个广告是什么意思?美国的选举现在基本上就已经搞成是跟商业企业搞广告公式完全一样,就是你要有一个响亮的口号,要有一首歌曲,还有一些 talking point,要反反复复的讲。他们这一次的talking point,有没有材料非常的多?我跟你讲那个 Biden这个执政非常失败,非常的多,你随便一找财政赤字,二万亿的财政赤字,这能不能讲?这个战争搞了七八个月,还要拼命的几百亿,几百亿的,每个月几百亿的往乌克兰送钱,这个可不可以讲?完全可以讲,顺理成章的讲。金融泡沫刚好,现在这个股市一路往下跌,房市也垮了,这个可不可以讲?完全是天赐的题材。你的食品价格,现在美国的鸡蛋是两年前价钱的 4 倍,OK,这个可不可以讲?当然可以讲,你的这个能源,现在的柴油是两年前的三倍,柴油价钱是两面的三倍,这可不可以讲?治安你有什么?你现在美国的那个黑人成群拉队的到商店里面去直接抢,大陆叫做零元购



唐湘龙 01:08:39 

0 元购就是。



王孟源 01:08:41 

对,已经成为一股风气了,你那个哪一天没有发生,反而是很奇怪的。这个可不可以讲?然后Hunter Biden的的贪腐,两年前他的那个apple的那个 MacBook 被人家揭穿出来的时候,他们全面抹黑是俄国人捏造的,现在已经被证实完全不是完全就是真的。这个可不可以讲?都可以讲,但是 Mitch McConnell把这些天赐的极佳的题材一个都不拿出来,他这一次全国性的,竞选活动中完全没有口号,完全没有政见,你说这有没有奇怪? 21 世纪的选举,有这么多好的题材,你没有一个全国性的政见拿出来来竞选。这为什么?就是为了要自己搞掉自己的候选人。而且是非常成功的,他把那个参议院给输掉了,为什么?因为美国的这些政客像 Mitch McConnell,他对国家的忠诚是最低的,他对他自己共和党的忠诚要远高于对美国忠诚。但是他对 deep state,他们自己那些政商权贵这个深层政府的才是真正的最高的忠诚。你可以牺牲共和党的利益来保护建制派的利益。那建制派的利益,是必须要搞掉Trump在 2024 年党内初选的形势,要搞掉的话就必须要让共和党在今年输掉其中选举。



唐湘龙 01:10:46 

好。我要问你,你认为 2024 年川普已经没有机会了吗?



王孟源 01:10:51 

我认为经过这一场操作,他的机会大幅下降,但是你还有 2 年,很难说对,对不对。还有这种事情有很多随机事件,不过你如果是算机率的话,这一次期中选举如果没有共和党,没有这样子被自己做自己人做掉的话,Trump的声势就很大。他在 2024 年出来,再从竞选的几率获得初选胜利的几率就在 90\% 左右,然后在大选获胜的几率更高,可能在 90\% 左右,那现在这个局面,他在初选获胜的几率就已经不到 40\% 了。OK,那即使初选能获胜也没有绝对的把握。所以我认为这一次的这个运作是非常成功的,一次建制派的大获全胜。



唐湘龙 01:11:54 

懂了,好,特别是最后面这个有关于其中选举的理解的部分,我不知道,就是我们的观众朋友们,因为有的时候对于国际政治,对大国政治,我会建议大家如果这样关心你就要稍微进到细节里面,因为那个细节非常的重要。如果只是跟着国际新闻的表面走,那你就只是被带风向的一部分而已,那个往往没有办法看到问题的本质。



唐湘龙 01:12:23 

好,今天不管我们在前面在谈到了俄乌战争,它到了一个新阶段,是一个,是个非常关键的 turning point。另外美国对中东,对沙特的态度的转变了,那个是对于对于沙特,对中东最后的一个温暖的回眸一笑,看有没有机会挽回中东。因为中东是美国非常重要的战略支点,失去了中东对美国的就是说战略部署来讲是非常沉重的打击。最后就是美国的其中选举,这个其中选举可看性很高,结果跟大家预期的不一样,对 2024 来讲也会有很深远的影响。好,感谢孟源来,我先感谢我们的观众和朋友来到。



王孟源 01:13:06 

有一件事情我忘了提。



唐湘龙 01:13:07 

你说,就是。



王孟源 01:13:09 

这次冬季攻势还有一个必然的结局。,是可以事先预期的,就是因为俄国必须要摧毁他们的基乌方的基础设施,然后再加上他的军队会推进几百公里。你可以预期会有一另一波的难民潮。



唐湘龙 01:13:30 

没错,那难民潮其实已经在蠢蠢欲动了。好,来我们感谢一下我们的观众朋友。



王孟源 01:13:38 

对不起,我,你说再讲 2 分钟,你说这个我去看了联合国的统计,他们是说有 300 万的乌克兰人现在逃到俄国境内,这当然是亲俄的。然后有另外 400 万在欧盟,另外有 300 万是到了欧盟以后又已经回到乌克兰,那在这 400 万到欧盟里面最多的是波兰,波兰有140 万。在德国有 100 万。那我想提醒的大家,这个 100 万这个数字刚好就是 2015 年叙利亚那名跑到德国最高峰的时候。 2015 年那一年门口收了 100 万的叙利亚,那可以想象当时的困难的程度跟反对的程度,那这一次我刚刚也提到有 300 万难民已经又回到乌克兰,但等到现在你没有水、没有电、没有、没有暖气,这个这些人绝对会又重新回到欧盟。



王孟源 01:14:47 

所以这次这一波新的难民潮可能会大于上一波,就是上一波难民潮一共送了 400 万人,那这一波难民朝可能是 400- 800 万人,,那也就是说德国可能会必须要再收至少 100 万,可能要再收 200 万,他如果再收二百万的话,就是达到 2015 年叙利亚难民潮的三倍。



王孟源 01:15:13 

大家可以想象这个对当地这些国家内政的影响,那所以基本上这可以确定欧洲的政治环境会进一步的恶化。而且像是荷兰这种国家,他们从夏天到现在三四个月了,其实他们的农民都不断的在抗议,每天都在抗议,不过是国际主流媒体不报道而已,我相信这也是一大看点,就是冬天天气冷,大家不方便大家上街游行,但是因为这个事情实在太离谱很难讲,说不定还是会有大但是到了春天绝对会有推翻政权,就是这次北约想要推翻不停的政权,看这个我们现在已经可以确定是反过来,会有大多数的欧洲政府,他们的政权会被推翻。



唐湘龙 01:16:08 

好,因为时间关系,跟孟源只能联系到这地方来。我们的观众朋友们来黑Zan,发现一个问题,香龙一上电视节目战斗力爆表,金句频出,在观点就可以让别人当拉拉队好了,谢谢两位的分享,谢谢 a 站 brother of K001 谢谢你。然后 w m 渺渺终于又等到王博士来了,先得为敬。我知道你们喜欢王孟源。



王孟源 01:16:34 

好。



唐湘龙 01:16:35 

带来的 potato had 王梦媛博士,为何联合报的博客好久没有更新了?莫非是换了地方?再问你博客为什么很久没有更新?你要解释一下吗?



王孟源 01:16:51 

主要是因为我对 Jack like 就是那个时差那个适应力很糟糕,这。



王孟源 01:16:58 

很差。



唐湘龙 01:16:58 

这个差很多。对,就是就在那日夜颠颠的颠倒了。这个我日夜颠倒。



王孟源 01:17:04 

对我睡眠一直是一个问题。所以 1 一旦有时差要适应的时候,基本上这个工作效率就直接往下掉。OK。



唐湘龙 01:17:14 

好,这个是很合理也能接受的。来了波纳的胡谢谢。然后的 z 红苏谢谢。再来破论,感谢。然后 c c n n c n,感谢,然后吴艾瑞斯,感谢。然后王博士,您好,现在大陆疫情持续恶化,政府要开不开的状态,想要走一个中间路线。我记得您的判断是冬天放开的概率是 5 成,那请问您现在的判断呢?你现在判断呢?



王孟源 01:17:48 

而我的判断一直都是明年上半年开放。OK,就是必须要有新的疫苗出来,对,才能才可能全面开放,就是。



唐湘龙 01:17:56 

其实他已经慢慢的在开放。你说你从最近习近平其实接待国外来这些的访客这么的频繁,其实那就是一个非常基本的开放的动作。



王孟源 01:18:07 

他们了解到这个欧米框是就是OmiCron的病毒,它的针对性从肺往上喉咙转移到……因为我们人的,他们靠的人是Ace2的那个receptor,但是这个Ace2在下呼吸道跟上呼吸道的版本不太一样。,那这个病毒为了要能够方便传播,病毒演化的第一优先是要传播力,所以要传播的话是上呼吸道比较容易传播。当然,那但是这样一来,你如果是针对上呼吸道那个 Ace2 的话,你就不容易产生肺炎,也就不容易产重症。那他们已经了解到这一点,所以开始开放。不过我认为要全面开放,必须要有一针对的性的疫苗比较保险。那这个针对性的疫苗在美国是一个月前出现的,我相信中国也是在积极的准备之中或中国政府的。这些事情通常是鸭子划水,在底下划得很高兴。对对,看不见。



唐湘龙 01:19:15 

对对对,他不能等到做好的时候,你看到的时候已经大概都到了很成熟的时候了。对,,这他的个性了。好,来,再来我们也就成。感谢,然后 j j 刘,谢谢在台湾,然后的无艾瑞斯。好,那再次澄清,我对严防严控没有意见,我理解认同严防严控的重要性,也同样知道放开后必然会面对到了社会冲击。但我觉得现在大陆的防疫政策有些不上不下,好像一块肉卡在喉咙里面。对,我就说其实它放宽是一个慢慢的过程,跟台湾一样,跟西方不太一样,西方经常放就全放了。但是你发现台湾也是一种渐进式的放松。



王孟源 01:19:58 

这个,我其实两个礼拜前在我的博客上讨论过了,就是我去看了台湾过去这一年放手,当然它不是有意开放,而是被迫的,但是实际上还是可以参照台湾、南韩跟香港的这个过程,你可以估计这个一旦全面开放的话,会造成什么样的人命后果。那很简单的计算就知道大约是 1- 2 百万的死亡,额外死亡。



唐湘龙 01:20:28 

对了,这当然对于对人来说,



王孟源 01:20:30 

当然你可以说都是七八十岁的老人了。但是我想中国的治理哲学跟西方的纯粹从经济观点来看,把人命,尤其是一般老百姓的人命当成价值为 0 的东西,这是不太一样的,因为儒家的思想是不太一样。



唐湘龙 01:20:50 

一样。没错,好,再来 n may 谢谢路易斯良,感谢龙行天下从来不会让你失望,嘉宾,非常非常赞。对,嘉宾王孟源好,再来旅游,感谢龙行天下的好节目皮特肖谢你在加拿大 90 Pay 在日本,感谢。好,今天非常感谢我们所有的观众朋友,因为听孟源上课大部分都是很专注,粘轴度非常高的,而且在对知识的极客症非非常明显的。那你也听到了孟源当他要连线,因为我们是时差是相反的,对我来讲是很方便,早上的时间对孟源来讲大概已经昏昏欲睡了,又要花很长时间准备,所以非常感谢那人在美东那透过我们的岳阳的视讯连线,提供了龙行天下的观众朋友好节目的王孟源。谢谢。



王孟源 01:21:45 

很高兴跟大家聊天,下个月见,拜拜。



\twocolumn[\begin{@twocolumnfalse}
\section{大陆解封会如何影响全球经济?}
\subsection{20221216}
\end{@twocolumnfalse}]Credit: Anonymous, 栗子



唐湘龙 00:02 

复制的,对,免推导协会方面的。来,我知道了, k 了吗?怎么样?嗯。



王孟源 00:24 

行,会掉耶,怎么样?



唐湘龙 00:38 

你说掉是什么意思?因为那画面只要。



王孟源 00:40 

放到影。



唐湘龙 00:41 

迭代后面的时候它影像。



王孟源 00:42 

会不接。对。



唐湘龙 00:44 

它不能在最上。



王孟源 00:45 

一层。



王孟源 00:46 

可是我们用视讯截取的。



王孟源 00:48 

时候会这样子。



王孟源 00:49 

好,来,先走。好。



王孟源 00:50 

来。可以了。



唐湘龙 00:55 

欢迎收看到龙行天下,我是唐湘龙,今天星期五的时间, 9 点半钟到 10 点半钟,王孟源的时间。那王孟源在美国,你待会看到一个穿着厚衣服的王孟源,我刚特别问了他一下,你今天干嘛穿那么多?好,因为当时他在美国,在美东了,在 New Jersey 那边是比较冷的。好,但是我相信我们的观众朋友每个月都高度的期待就这个小时的王孟源时间,大家敲碗,我知道很多朋友说,你可不可再加一点时间?可不可以再多增加一集?这个就不要问了,因为我要配合我们的来宾的时间。但是通常每个月的第三个星期五,那大概就王孟源上线的时间。今天我们大概谈几个主题。第一个就是大陆解封之后对全球经济的影响,包括了通膨,那包括了衰退这几个议题其实是纵贯整个 2022 年,全球的重要的议题,那到了年终的时候,这件事情同时发生,大陆解封了,那通膨似乎下降了,但是衰退的疑虑是不是就解除了? 2023 年怎么看?因为王孟源在金融圈子里面其实工作过蛮长的时间,这些的嗅觉跟专业知识,这是他的专长。



唐湘龙 02:21 

第二个部分我们再来关注一下中东,因为中东的地缘政治形势变化是很大的,那尤其习近平之前到了中东之后,可以看得出来就是整个阿拉伯世界鼓掌相迎。好,那这个对于对美国来讲,美国过去所有的这些保守派的战略学派、现实主义学派,总是强调美国必须要守住全球三个战略支点,一个是北约,另外的一个是日本的安保,还有一个就是中东,这三个战略支点中东的这块显然松动的很明显。



唐湘龙 02:56 

好,那这个王孟源又怎么看?因为这个全球未来的能源配置也有直接的关联。再来,最后我们有时间要来关注一下,当这个议题出现了之后,就是前两天美国用政府的等级,用国务院的等级,那敲锣打鼓地向全世界宣布,那美国在人造太阳这件事在核融合这件事情上面有了突破。那因为出现了能量能量增益争议,就是输入的能量小于,就是说输出的能量,换句话说它已经那具备某种将来可以作为能源产生的发电的形式,而且无碳无污染。



唐湘龙 03:35 

好,但是这个因为我外行,那从一个人造太阳在大陆在这方面已经走了非常非常久了,如果从商业运用的角度来讲,一般认为大陆中国可能走的比美国更前面,那这两者之间又有什么样的不同?好,这个待会请到了专家,那王孟源是高能物理方面的专家,来在我们线上的人,在美国的王孟源,欢迎孟源。



王孟源 03:58 

大家好,非常高兴跟大家再聊一聊。一开场先向大家致歉,因为今天,我的家里在过去这一个月还有未来这个月都会一直在整修,今天刚好有 3 波工人过来,尤其是那个中央空调。所以我没有时间坐下来好好的想,或者也是写作。那么准备也不是特别的充分,不过我想也好,因为刚好这个月没有什么真正的黑天鹅事件。所以不需要去详细的解释。那我们今天湘龙给我的这些题目其实都是我在博客跟节目以前都已经讨论过了。我一向不喜欢当事后诸葛亮。你如果是可以预见的事情,合理可以预见的事情就应该事先讨论规划,所以今天就基本上是以复盘的姿态去看看我以前的推论,然后跟现实的演进做一点比较,然后做一些比较深入的探讨、联想,比较抽象一点的讨论,所以希望大家能够适应,就是不是像过去这几个月详细的谈,比如说俄乌战争,我详细的谈那个战术、战略之间的比较,今天比较不会有那么完整的架构,而只是比较随性的谈一些事情。



唐湘龙 05:31 

好,我在跟孟源在交换,就是说节目内容的时候,孟源说俄乌我们每个月都谈,今天先跳过。好,那我们就先跳过俄乌的部分,因为我就想,我想孟源大概也都会高度关注大陆解封之后的情势的变化,因为之前上个月的时候,甚至于还在几个大城市里面出现了所谓的白纸运动或者白纸抗议,那其实在白纸抗议之前可以看得出来大陆的官方早在白纸出现以前,大陆官方已经开始一波又一波的这个就是说解封的动作。但是在白纸之后的那个速度变得更快了,那大白几乎都已经从市面当中几乎都已经快消失了。好,现在大家就是在做一些的医疗准备,不过这个它必然会经过这过程。第一个就孟源怎么看大陆的解封的动作,以及它会对于大陆以及全球会出现怎样的影响?



王孟源 06:30 

其实今年年初omicron。



唐湘龙 06:32 

莫米狂成为主营的。



王孟源 06:33 

成为主流的时候。



王孟源 06:35 

我们。



王孟源 06:37 

我就已经开始做了很多评论然后推测,当然这个主要集中在介绍科学的背景,因为这种政策反应是必须要,这种是完全应自然的,所以自然现象的那个特征跟影响稍微有一点变动,你的政策也必须要跟着变,而且甚至可能是起因稍微变了一点,你的最佳解决方案就会南辕北辙。所以这个必须要。



王孟源 07:09 

当然病毒的突变这种事情是随机的,它真正是量子力学的随机的,所以你作为一个治理最理想的方法是事先定好所有的预案。然后等确认这个病毒突变的速度跟性质之后,你就可以很有条不紊的去采纳以前已经计划好的步骤,然后从上到下大家协调一致来做。那我在几个月前的确有提到这个部分,就是执行的细节,我说大约应该是在明年上半年全面开放。这个原因是什么呢?因为你开放,不能够一下子就开放,因为即使我们现在今年台湾本身弃守,然后到大陆被迫解封,这个原因都很简单,就是它的起基本起因都是因为omicron的传染力又进一步增强,但是它的致命力降低了,而这两者是相关的。也就是说这个病毒的突变。



王孟源 08:23 

它要侵入人体的呼吸道细胞的时候,是要经过一个 ACE2 的 receptor 抗原,那这个 ACE2的 receptor 在我们肺部就是下呼吸道,跟上呼吸道喉咙这里的是有一点不一样的,这里是演化出来故意的,就是我们的身体故意演化出来稍微有一点不一样。所以病毒必须要选择,你如果对肺优化的话。



王孟源 08:51 

你就不会再对喉咙的感染就不是很好,你如果要对喉咙感染好的、强的话,对肺的感染力就降低。那这个分隔是在为什么呢?因为你如果是感染喉咙的话,上呼吸道的话,就容易咳嗽的时候,呼吸的时候,讲话的时候以飞沫传播出去。



唐湘龙 09:14 

病毒就排出去就是。



王孟源 09:15 

对,就是它传染力增强的最佳方式。所以基本上是我们人体演化出来强迫病毒要选择。你到底是要增强传染力还是要增强致病力?因为你真正要产生重症,基本上就是肺炎、呼吸道疾病,越往下的那个症状越严重,对不对?那但是越往上传播力就越容易,传播起来就越容易,所以基本上是演化中我们这种寄主对病毒的一种反应,那这个新冠经过 3 年多的演化,对,已经刚刚满 3 年了。



王孟源 10:04 

它已经高度优化。那优化的时候,在现代这种病毒优化它当然是优先选择传播。事实上病毒杀死寄主完全都是一种意外,对病毒本身没有什么好处。因为寄主其实是你的住的那个房子,对不对?这对病毒来说,你把这个房子都拆掉了、弄倒了没有什么好处。所以从演化的过程来说,长期来说,它应该是往传播能力越来越强。所以这个都是我 3 年前就是两年多前,新冠刚刚开始的时候我就都解释过的。当然这种是长期的趋势,你在这个演化的过程中几周或者几个月的周期里面,新的变种可能会是例外,会去反其道而行,就是突变,它是一个随机的事件。不过今年年初omicron出来以后,你可以看出他比起以往的所有的变种,真的是又往上呼吸道优化,又进了一步。那这种事情你只要几个月之后确认了,尤其是欧美,他们的很热情的为我们做人体试验,免费的人体试验,对不对?



王孟源 11:31 

这个你几千万的病例都是由他们自己冒险去帮你做,所以你只要观察两三个月就可以看出、确定它的致死率降低到什么程度。基本上致死率如果降低到跟流感是一个数量级的话,就应该要开放,这开放不是说就放手,尤其是,大陆可以说是全世界防疫做得最严的、最好的地方,这个没有之一。那这个附带的一个负面的影响是反而没有自然免疫。就是现在美国人平均每个人得新冠得了大概 2 次多。



唐湘龙 12:22 

你有得过吗?你阳过吗?



王孟源 12:24 

我没有,我没有,因为我是个宅男,我每天都在家里。



唐湘龙 12:27 

好,你继续。





王孟源 12:31 

例外中的例外,我儿子得了两次,我儿子今年8月得了一次,9月又得了一次。所以你可以想象那个有多厉害,而且都是奥密克戎。



王孟源 12:48 

因为你防疫做得越好,自然免疫就越低,所以你解禁的时候自然会造成很多人的死伤,即使是像奥密克戎这样,它的致死率已经降到跟流感同一个数量级。注意啊,我们科学家讲数量级就是相差不到 3 倍,在 3 倍之内。



唐湘龙 13:14 

3 倍以上。



王孟源 13:15 

所以并不是等同的意思,就是你取自然对数的时候,取以 10 为底的对数的时候,相差不到1,那个叫做数量级,自然对数的结果是数量级,所以你差一个数量级就是差 10 倍。那 10 的平方根是 3 点多,就开。



唐湘龙 13:37 

一开 log 的时候。



王孟源 13:38 

开平方根是 3 点多,所以这个数量级,半个数量级就是 3 倍多。也就是说我说同一个数量级的意思就是相差不到半个数量级,也就是相差 3 倍之内,那他如果还是比流感强 3 倍的话,我们现在流感都还要鼓励大家打疫苗,而且流感是每年都有,已经过去 100 多年每年都有,我们基本上大家都有天然的、自然的免疫能力。那你这个新冠现在忽然降到这个地步,然后因为它的传染力太强,然后致死率又不够高,你整体来看应该最优化的解是开放,但你不能够一口气就这样子放手让它传播起来,因为你可以简单的看、算出来,会因此额外死亡多少人。我再解释一下什么叫额外死亡。就是新冠刚开始的时候,在大陆他对那个新冠死亡的记录是其实是很算是非常严谨的,基本上因为新冠死亡的人 90\% 以上都会被官方记录。那接下来到欧美,你仍然是有,因为他们的医疗体系比较好,而且一开始的时候很紧张,所以都会有设法去测验,他们基本上还是能够做到 60\%、 70\%,然后接下来到了半年多之后,整个散开了,他们这个时候检测的结果就降到40\%、50\%,到现在基本上他们已经躺平了两年了,他们基本都懒得去检测了,就是你死了就死了嘛,谁管你有没有新冠,是新冠后遗症还是怎么样子。所以正确的测量新冠致死的方法,最好的方法不是他们现在官方报道,就是过去这一年多所报道的新冠致死的人数,基本上都是少报了一个数量级,就是基本上只有 10\% 、20\% ,在非洲的话根本更是不可能,因为他没有那个医疗基础设施,在欧美也是少报了一个数量级,所以正确的做法。



王孟源 16:08 

而且这不是我一个人说的,你如果去读经济学人,他们自己在过去这两年也是采用同样的方案,就是你用在新冠传播开来之前的那三四年, 2016 年到 2019 年例如,你把这 4 年的每个月的死亡人数做一个平均,这是你预期每个月这个国家会有死多少人?那现代工业化国家它对出生跟死亡的记录其实都是很精确的、完整的,然后你拿过去这三年每个月的死亡人数,去跟新冠开始前四年的平均每个月的死亡人数做对比。



王孟源 16:52 

如果它相差在 5\% 之内,基本上你可以说好,我们不能确定是怎么回事,但是绝大多数的国家都超过了5\%。在过去这三年欧美基本上是在 15\% 上下震荡,有一波来很强的时候可能达到 20\% 几30\%几,然后那一波过去以后可能降到 10\%、 5\%,但是基本上 15\% 震荡。



王孟源 17:25 

那omicron来了之后,它这个震荡的中心已经开始下降到10\%,就是还是有额外的死亡。就是新冠。omicron虽然是比较弱,但是他对你身体不好、年纪大的人杀伤力还是很大,就像流感一样,因为它的确是一个大号流感,只是大三倍的流感,对不对?那刚好这里又有一个天然的实验,就是像香港、南韩、新加坡,还有尤其是台湾,你如果去看台湾的额外死亡的话,这在我的博客有链接,有兴趣的人可以自己去看。就是有一个美国的科学网站专门介绍这个台湾的额外死亡在今年5月之前,完全基本就是零,就是基本上完全跟 2019 年之前一样,在6月开始跳,跳超过5\%,然后七月、八月到达高峰。



王孟源 18:32 

然后从九月又降下来。最新的数据 11 月是11\%。那你如果算过去这六个月这些额外死亡的人数加起来一共多死了多少人?就是死于非命。照理说如果没有新冠的话。



唐湘龙 18:49 

他不会死。



王孟源 18:50 

对,他不会死的,这个有多少人? 19000 多人,所以今年再加上 12 月一定会超过。



王孟源 19:01 

2万。



王孟源 19:03 

你可以说到那么精确,就是这种事情虽然是统计,但是没有用到任何假设,基本上就是官方的非常精确的死亡数字。所以你可以确定台湾因为防疫成功 2 年半之后解禁。因而在过接下来的六七个月死亡多死亡了,额外死亡了2万人。你把这个同样的比例弄到大陆上去,你猜会是多少人?



唐湘龙 19:40 

对,这是一个数,这是一个天文数字,不能说天文数字,是一个很大的。



王孟源 19:44 

120 万



唐湘龙 19:45 

120 万,就是今年,如果这个时候采取跟台湾相同的解封的,就躺平的方式的话,



王孟源 19:52 

已经开始了。



唐湘龙 19:53 

对,大概半年左右的时间会出现这样的一个数字。



王孟源 19:58 

2023 年的1月就是台湾的6月。



唐湘龙 20:00 

是没错,没错。



王孟源 20:04 

我预期。所以你可以预期大陆在 2023 年的额外死亡人数大约是 120 万。额外死亡人数,那这些大部分是什么人呢?农村里面 70 岁以上已经有糖尿病、高血压或者心脏病,或者是有其他长期疾病的人,尤其是没有很好的医疗保障的人。那么这些人也许是没有什么经济价值了,尤其在欧美,所以现在躺平了,一副天下太平的姿态,基本上就是因为这些人没有经济价值,尤其是他们连那个三代同堂的家庭都很少。所以你一个独居的老人,又是一个穷的老人,尤其又是乡下的穷的独居老人,死了就死了,没有人在乎。



王孟源 21:10 

所以半年前我在预估这件事情的时候,其实都已经看到这样子的局面。那正确的解决方法是什么呢?是你赶快把新一代的疫苗弄出来,就是你在 2020 年的时候花了一年做了疫苗,对不对?现在已经整套设备已经驾轻就熟了,你就赶快重新稍微调整那个疫苗,只不过是针对omicron的那个他的那个 ACE2 的突刺蛋白,来稍微修改一下,然后把它普及来,你只要,你宣布嘛,全国只要,全国哪一个省份只要达到 95\% 的疫苗接种率就开放。然后事先预定开放有什么步骤?然后囤积那些药、囤积血浆。你知道这个新冠对医疗体系的一个打击就是其他有医疗需求的人会出问题,会被新冠给阻碍住了。



王孟源 22:27 

现在大陆这两天出现了血浆短缺,为什么?因为你捐血的人。



唐湘龙 22:34 

大量的减少。



王孟源 22:34 

必须要检验核酸,对不对?那个如果,如果大家不愿意上街,也没有办法专门为了捐血而检验核酸,或者是光是检验核酸很麻烦这件事情,大家就不愿意捐血了,对不对?捐血的人少了以后你的你就没有血浆了。这些事情都是可以预见的。



王孟源 22:57 

而且我半年前我在开始谈这件事的时候,我只不过是很简单的说,我预期大概明年上半全面开放,而且是开放的步骤是先开发新疫苗,然后你新疫苗足够普及之后就全面开放。那这中间当然有很多步骤,我也没有详细的去谈。为什么没有详细去谈?因为这是细节,而且必须要因地施政,而且这个是他们政府官员自己的分内的事。那中共一直是世界上最理性的政府,也是效率最高的政府。那一直,我当时半年前我还觉得不会有出什么问题,这种事情我啰嗦是越俎代庖,这个杞人忧天,结果你看现在的结果出来,这个真的是乱了阵脚。就是他们因为抗疫,这个抗疫本身的规模其实不是太大,但是因为二十大刚好在交接的过程中,他们不愿意冒那个政治风险。那但是问题是、这里真正出的责任在什么呢?在没有预案,对不对?就是过去这半年多,连我这个山野草人。



王孟源 24:26 

都是半年多前就可以看得出应该做什么准备,他们连一套预案都没有,对不对?你这个事先中央应该准备好跟地方协同,然后怎么样教育民众?你不能光是事后跟民众说我要把你的小区封锁,你要有一个标准,对不对?这个标准要公布出来,然后每隔 3 个月换,这个标准开放一步,你要每个人都可以上网去查,对不对?这样子才是有效率的管理,所以这一次看来我是很失望。



王孟源 25:06 

那我在博客上已经跟读者在过去这两个礼拜讨论过了、互动过了,那就有一个读者说这可能是那个二十大这次卸任的那些现任官员故意捣蛋,对不对?因为权力才在他们手里,他们真正交接是要到明年三月。没错,但是我的回答是这样子的,在研究社会科学有一个很有名的Hanlon's razor,这个是原则,我讲出来给大家听听。



王孟源 25:45 

never attribute to Malice that which can be adequately explained by stupidity。



王孟源 25:53 

你如果用愚蠢可以解释的事情,就不要假设恶意,所以我觉得这些人怠工或者是因为担心自己职位不保,所以没有好好的去办事。我觉得既然这个解释你不能排除的话,你就必须要接受它吗?因为这是所谓stupidity、 incompetence 这个是必须要优先接受的。不过不论如何。



王孟源 26:29 

我今天开场的时候说我要讨论的深入,比较抽象一点,我要从这边顺便就讲下去,这个这一次的开放,如果能够循序渐进的话,那 120 万的额外死亡可能可以减少几十万,甚至可以减少一半。但是很不幸的我们现在追悔已经太晚了,悔之不及了。



王孟源 27:07 

我也说过这些额外死亡的人口基本上是对经济是没有影响的。我再说一次,这不是你放任这些额外死亡的借口。但是我现在不想再多谈,因为事情已经发生了,回去追悔没有什么用处。



唐湘龙 27:31 

好,我稍微打断一下,因为孟源提到就是说明年的上半年,当我这两天的时间,我看到 who 世界卫生组织的秘书长谭德赛也已经记者会也讲了,他说估计在明年,就是过完年之后,那 who 很可能就会宣布就是现在的就covid-19 已经不再是一个需要全球,就是说关注的一个所谓的重大事件。



唐湘龙 28:01 

意思就是说从 who 的对于这个疫情的追踪跟监管来讲,大概到明年的第一季, who 就是 case close,接下去就是常态化了,跟那孟源刚讲的那是一样的,但是当这个讲得很残忍,孟源刚提到就是说这可能 120 万左右,现在推估从台湾的 case 去推估大陆现在的解封的方式可能会增加 120 万左右的额外的死亡,里面有许多因为它不在全球经济分工体系里面会 touch 到的部分,所以它可能也不是这么被重视。但是它终究是人命。所以那这跟呢?解封的模式,跟你的准备,跟你的政策态度是有关的。好,那大陆这样现在的解封如果到明年上面看起来就是说解封的解封的方式跟解封的理由,逻辑,很大的一个理由,逻辑大概跟经济是有关的。那大陆现在的解封的速度跟它的方式对大陆的经济以及明年的全球经济会有什么影响?



王孟源 29:07 

现在就是有这段混乱的时间,而且我觉得真正让我担心的是要等到三月才换手。然后换上来的这些人还也需要一段时间来适应。搞清楚状况对不对?你再怎么聪明,再怎么有经验,你做一个新的职位,你总是要一段时间。几个礼拜是最最最基本的,一般人需要几月,所以我对中国短期的经济发展不是特别的乐观,就是短期就是 3- 6 个月以下,但是可是我希望没有什么太大的黑天鹅事件,没有什么运气不好的突发事件必须要处理。你如果有什么突发事件,那事实上,这个新冠解禁这个关已经过了,就是过得非常的不漂亮,但是已经过了。未来的这半年没有什么明显的难关嘛?所以你除非就是有什么黑天鹅事件的话,否则这个交接并没有什么太大的问题。就是他们虽然一开始会搞不清楚状况,但是如果没有什么大事情的话,那个现存的官僚体制就是自动的,以自动导航的态度继续运行 3 到 6 个月应该是没有什么问题。



王孟源 30:46 

那然后中国的经济经过承受了这些额外死亡的阵痛之后,大家开始上街,然后服务业开始恢复,这会是我预期在半年之后会是全球经济重新有点复苏迹势的火车头。当然,中国本...,这里还有要考虑欧洲跟美国,美国这一次这个我 3 年多前,这个真的是 3 年半前我就又讲这下一波经济危机会是一个通胀型的危机,而且是一举打破美国霸权的一个良机,因为这个通胀危机的基本动力会是因为货币超发,那美国的货币超发是世界第一的。但是今年到了三月的时候我就已经看出情形苗头不对了。那为什么呢?因为我假设的是欧盟跟中国都会做出最基本的自保动作。



王孟源 32:11 

不但是自保,而且我预期中国会进一步去做出比较委婉的落井下石的动作。什么叫做委婉的落井下石动作?就像俄国,俄国在战事开始之后,你看它的这个货币兑换率,这刚好它的卢布跟美元的兑换比是美国原本挑拨俄乌战争的焦点,他就是希望把卢布打垮。卢布,因为在 2014 年的时候卢布已经贬值了百分之百了,他们希望再贬值百分之百,然后有 2014 年的时候有 1500 亿美元的俄国资产外逃,这个他的当时卢布贬值百分之百,就是卢布打折了,百分之百我的意思是打折的意思,打五折的意思,打对折的意思。那他那个卢布当时会贬值的那么快,就是因为资产都外逃了吗?有外资,但是更多的是他们本身俄国本身富豪的资产。那美国认为上次那样子可以,那这次再来一次,然后再加上扣押他们的外汇就可以,你只要做得更狠,对不对?那效果应该更好,那就可以完全打垮俄国的经济,然后从俄国的经济崩盘之后来推翻普丁政权。这个普丁其实在俄国是温和派,但是因为



王孟源 34:00 

他不愿意让,他的真正的原罪是他不愿意把俄国开放给国际资本去搜刮,所以这是他的原罪。而这个原罪在 2003 年就已经明显化,所以刚好欧美对普京的态度也是头三年非常的客气、观望,其实不太在乎,但是从 2003 年普丁开始出手整治国际资本,以及他们国内的那些寡头,就是不是全部的寡头,就是那些作为国际资本的白手套的那些寡头。被他开手整治的时候,欧美的那个传媒对普丁的描述就做了 180 度的转向,他就是从那个时候变成恶魔的,这一次,你看这个Nabiullina,就是俄国的中央银行行长。她是经过去 8 年的准备,然后,但我刚刚刚讲过的,有了预案,这个预案一个实施之后,你猜现在卢布的那个汇率怎么样?不但没有下跌,卢布是过去这一年唯一对美元上涨的主流货币,对,而且至今,我刚刚去查了一下。



王孟源 35:34 

他比开战之前上涨了18\%。



唐湘龙 35:38 

哈哈哈哈,开涨了这么多。



王孟源 35:41 

上涨18\%。而且这个程度很稳定,就是他开战之后波动了一两个月,然后就稳定到这个地步,这基本上Nabiullina。挑了这个程度、这个汇率,然后就说我这很好,就定在这,他怎么挑?这个汇率是,明显影响俄国国内经济的最高值,为什么?这个汇率定在这里,因为这一次今年引发的这个通胀危机是与一般的经济衰退不一样的经济危机。一般的经济危机是通缩的,但是通胀型的经济危机,这个时候。在通缩的危机里面,你希望你的货币贬值,但是在通胀型的经济危机,你希望你的货币升值,因为你的货币升值以后,你进口的那些能源、原材料、成品。



唐湘龙 36:51 

它会便宜。



王孟源 36:52 

全都减价。所以等于是你把通胀的压力额外推到国外去,然后你自己的国内的通胀压力减低。那你看这一次,所以我说这次俄乌战争一来,欧盟自我牺牲,那些能源制裁、对俄的能源制裁,基本是使他们的国内的工业失去了基础。那工业产业就开始外逃,但是产业外逃需要时间,最先逃跑出去的是什么?是资金。游资对不对?游资一跑的话汇率就变。所以你看到现在。



王孟源 37:45 

欧盟的汇率,欧元的汇率是跟美国是怎么比的?它是刚好跟卢布相反,卢布上升了18\%,他还下降了18\%。那,你还可以说,欧盟是自愿去跟俄国对抗,然后把自己的能源供应商给制裁了,他降掉18\%。那么中国应该怎么做?你先猜猜中国实际上是怎么做的?你猜不猜得到怎么做?中国实际上是过去这 9 个月贬值了9\%。



唐湘龙 38:42 

没错,人民币是贬值的,到最近才开始升值。



王孟源 38:46 

就是升值之后还贬值了9\%,最多的时候贬值到15\%。你说这合理吗?中国有承受到产业外移的压力吗?有承受到资金外逃的压力吗?没有,没有。对不对?都没有。这很明显的就是他们的金融主管官员还是生活在美国一家独大。然后中国以寄生的方式活在美国主导的全球化雨伞之下的那种心态,所以美国人希望美元升值,他就配合。正确的做法应该学俄国的Nabiullina,选择不明显打击国内经济的最高值。所以最最起码,人民币不应该贬值,完全就没有贬值的理由。然后甚至应该升值 5\% 到10\%。但是这个就是人谋不臧,也是跟我刚刚讲的新冠解禁一样,人谋不臧。当然这些人民银行行长这些主管的官员,这一次二十大也是一样,都要退掉了。。



王孟源 40:20 

所以我对中国中期,也就是 6 个月到未来几年的这个观点的经济观点,我是很乐观的,为什么呢?因为习近平二十大,这一次二十大清扫的这些官员,他清扫的一个很大的标准就是那些迷信资本跟自由市场,崇拜英媒体制,自动的去拥护美国的霸权体系的那些人。那这些人换掉以后,他就不会遇到今年这种错失良机。因为你看看现在美国的经济怎么样?一开始的时候,年初的时候美国的通胀是先上去的,对不对?我不知道你记不记得。



王孟源 41:21 

今年春天跟夏天的时候,美国的通胀是比欧洲高的?现在美国的通胀下来了,欧洲还在 10 PERCENT,美国现在是 7 PERCENT。这个替换是什么?就是我刚刚讲的那个资金外逃的效应。资金跟产业外逃的效应。然后基本上就是美国在吸收欧盟、英国、日本、韩国跟台湾的产业跟资金。这些是美国的附庸国,他们没有反击的余地。他们没有说不的余地,中国为什么也要配合?去救美国?这真的是、真的是卖国贼的行为,那这些卖国贼当然退休了。那所以到明年年终之后。中国就不需要担心这些人来掣肘。至少在金融方面,金融国际战略方面不需要担心他来掣肘。然后与此同时这也是我在过去两年讨论过好几次的,最重要、世界头号的、产值规模最大的工业是什么?实体工业是什么?是汽车,对不对?那中国原本发展汽车是发展的跌跌撞撞,就是他们的那些合资企业。



王孟源 42:59 

这。很多人要批评那些合资企业不争气,这有他的道理,因为他们毕竟是地方政府设立的,而不是中央政府成立起来专门为了突破那些技术的。不过你看先天的背景也是,我们现在用的这个内燃机汽车的技术已经有 140 年的历史,对不对?这个在欧美他们是精益求精,欧美跟日本精益求精,你一下子要从后来追上,赶上这种非常成熟的技术是非常非常困难,而且它的规模这么大,所需要的资金这么多,所以这一次刚好有电动车。然后,我对中共政府执政能力的信心其实就是来自像他们发展电动车产业这样的案例来。非常的成功。对,非常非常成功,真的是可以写进教科书里面去。



王孟源 44:07 

而且你看像是引进特斯拉,用 特斯拉来倒逼国内的汽车厂,而且他把重点都抓对了,就是真正这个核心技术是什么?电动车的核心技术,第一号,远远第一号就是电池,电机是其次,电机是第二号。但是特斯拉是做车子的还是做电池的?做车子,电池在中国手里。对。



王孟源 44:45 

那你把特斯拉弄到国内,虽然好像在卖车的这个行业稍微有点牺牲。但是你是不是在支持电池的产业?对不对?就是你真正抓对重点了,就是重点是电池产业。那你把特斯拉弄来对电池产业好不好?好得不了,那所以这一次你可以看他们这个扶持电动车这个产业这件事不止十年了。是十几年前就开始的。这个你去看他们这个政策的调整,非常的灵活,非常的有前瞻性,所以就是因为有这样的案例,所以我以为你处理新冠解禁应该是一点问题都没有。这就好像拿过诺贝尔奖的人,连普通的维积分都做不出来。



唐湘龙 45:43 

哈哈哈哈哈,好,其实刚孟源刚刚讲的这一段了,虽然里面的牵涉到对于大陆的官僚体系的判断力的一些的批评,不过,这一段是非常有参考价值的一段了,大家可以反复的听。好,那我们再来呢?再回到一个最近我关注的议题,我讲孟源也会注意,就是习近平这次到中东去访问之后,阿拉伯世界当然摆出了很高规格的场面,这个西方的媒体等等大概也都报道了很多。那这个里面,因为毕竟你仔细看的看习近平的在中东的这几场的峰会,不管跟沙特的这种的战略性的这种的表达,或者跟阿拉伯世界的这种的博感情式的表达,或者是跟海湾国家的这些海合会这种是有关经济长远的这种的能源合作,能源供应的表达。他把所有的中东的最敏感的问题,其实你这到访让我意外的是把中东所有的敏感的地缘政治当中的矛盾,中国几乎都做了表态,所以之后才会有伊朗的不高兴。那你怎么看这一次的习近平的到访,他对于未来的大国政治的格局会有什么影响?



王孟源 46:59 

他的访问到目前看来只是一个结果,什么的结果?就是一带一路的结果,一带一路又是建筑在中国的工业崛起的,工业能力崛起的基础上,然后以这个工业能力提供一个替代。因为你说起来英美的这个霸权,在二战之后,其实他们还交手了,争夺了一段时间。那二战之后的那 10 年,其实是美国跟苏联合起来瓦解老殖民帝国的一段时间,就是为什么现在所有第三世界国家绝大多数都是二战后的那十年处立的,或者获得自主权的?



王孟源 47:49 

不是因为英法忽然大发慈心要决定把它放手,而是美国跟苏联在背后强力或明或暗去搞他们,强迫他们让这些殖民地独立。为什么?因为美国跟苏联是新霸权,对不对?他知道这个这些老殖民帝国的那个财务跟权力基础是这些殖民地,所以对他们来说搞垮这些殖民地是有利的。那搞垮之后,他们就建立了一个新的所谓的国际体制。那这个国际体制是建立在联合国宪章上面的。那联合国宪章你如果仔细去看,那个懂政治历史的人怎么可以看呢?你就可以知道。他这份文件是属于所谓的 Westphalia School。所谓 Westphalia 是17 世纪



王孟源 49:06 

哈哈哈,我没有准备。



唐湘龙 49:08 

没有,没问题,继续,我认真认真听,我认真听。



王孟源 49:13 

30 年战争,一般人不知道它有多么重要。 30 年战争是当时德国还没有统一。所以在德国打了一个内战,因为它有新教跟旧教的争议,后来旧教就是奥地利,现在奥地利还是旧教,就天主教。然后德国本身是新教。就是本来奥地给你跟德国都是德语区,这个大国德语区里面,新教跟旧教打起来,打起来以后就把周边的强权都扯进来了,这些周边的强权都用这个借口到德国去烧杀掳掠,整体的德国人口被杀死的占40\%,有些地区死亡人数超过50\%。你说这个这有多惨烈?



唐湘龙 50:09 

对,非常惨烈。



王孟源 50:10 

因为当时还是神圣罗马帝国,那个整个德语区分成几十个小的国家。对,都是什么侯国、公国,什么王子国、选侯国之类的。但打了 30 年,他为什么会打 30 年,其实是因为那个时候的犹太财团。哈哈哈,犹太财团两边借钱鼓励他们打,跟现在的那个美国的军工跟金融财团如出一辙,不过那个是另外的一番话,有兴趣的人自己去读那段历史。打完之后他们最后终于签了这个合约。正是因为死伤的太惨重了,所以大家觉得以往的这个中世纪的封建体系下的这种国际关系已经不能够适应近代的这个世界,所以他们签的这个合约叫Westphalia合约,在Westphalia就是德国的一个州签的。那这个合约第一次明确了近代国家有主权的这个观念。就是你当你说Westphalia的concept就是指主权国家的观念,这个主权国家观念就是从那里来的。



王孟源 51:40 

然后经过 18、19 世纪完整,但是,联合国宪章并不是简单的复述Westphalia,你要想想看,Westphalia这个合约基本上是一个主权国家,为什么主权国家?就是大国小国、强国弱国之间有相对平等独立的权利,这是它的核心的意思。但是你要注意这里的,这里给的都是欧洲的白人。联合国宪章的重要意义是把这个不分种族,不分洲,不分你的经济发展程度。把这个国家的主权,平等的赋予所有的国家,全球所有的国家。为什么当时能够签的出来?就是因为美国跟苏联正在准备要搞英法,把英国跟法国的殖民帝国整个,让他彻底的寿终正寝,所以他们愿意签签这样的协约。那为什么我们说冷战后全球化不到 30 年,美国人就后悔想要倒退,为什么?因为他们的核心还是一个殖民帝国,只不过是不再用军事直接强迫,而是军事辅助金融、金融辅助宣传。这三角来互相支持的一种新的殖民主义帝国。那这个殖民主义帝国,当然你只要本质是殖民主义帝国的话,就跟Westphalia体系合不下去,因为Westphalia的体系所谓的国家主权,它的核心观念并不是独立,而是平等。大国跟小国之间平等,先进国家跟落后国家之间平等。



王孟源 54:11 

那你主要是殖民,它的核心就是不平等。那美国现在他们实际建立的体系是什么?是一个所谓的golden billion,我想我跟你提过几次。



唐湘龙 54:24 

没错,黄金 10 亿人。



王孟源 54:28 

黄金的 10 亿人,也就是美国站在最前面,第二阶是其他的昂萨国家是,第三阶是其他的白人国家,就是欧洲,然后第四阶是东亚的,这三个先进经济体。这四个阶层能合成的黄金的 10 亿人,原因是什么?他们都是美国的附庸国。所以他们可以被容许开发先进工业。那其他的国家如果想要稳定下来、开发先进工业,第一步必须是普京所做的什么,排除那些寻租的、寄生的国际资本,要不然你永远都没办法真正的开发自己先进工业。台湾的台积电是用国际资本搞起来的吗?不是,是事后,搞起来之后,国际资本才来想办法进来……三星当初是国际资本搞起来的吗?不是,他们自己搞起来之后,然后在 1997 年被 imf 强迫卖给美国人的。



王孟源 55:52 

对,所以这个美国的新殖民主义,他不论如何还是会顶到这个Westphalia的,也就是联合国宪章的这种国家主权概念。这也是为什么过去这 30 年他们开始讲人权高于主权。这个所谓的主权代表的就是联合国宪章,就是 Westphalia的这个,各国国际平等的这种观念。人权是谁有国际宣传权,谁就可以定义人权,对不对?那所以谁有国际宣传权就是昂萨、英美。所以这其实是殖民主义的体现,是要回归当年殖民的时候,殖民白人老爷高高在上,不管他们说什么都是对的,那底下的人不管做什么都是错的。这样的重新恢复到这个姿态。



王孟源 57:00 

所以我为什么讲这么多。是因为中东已经觉醒了。全世界的第三主义国家都觉醒了。这一次习近平去中东并不是因为中国做对了什么。中国实际上做的只不过是自己发展起来,没有靠国际资本,靠自己的资本、内生的资本跟内生的努力。开发了一些工业,而且这个工业的产能到达能够对欧美做替代的程度。基本上除了半导体之外,你还有什么东西不能替代?对,那如果这样子的话,其他的第三世界国家就没有必要再依赖欧美,反正他们也不能够明目张胆的进去直接做侵略。就像 200 年前那样子,他们要侵略之前必须要先搞颜色革命。那现在搞颜色革命的这件事情又被俄国给搞翻了、翻脸,然后现在你只要,你只要摆明说我不支持你这些黄金的十亿人,我要跟中国跟俄国基于 Westphalia国家主权相对平等的做合作,那我就可以指望排除国际资本,然后为我自己国家的前途做独立的奋斗。



王孟源 58:50 

他们是因为这样子才会站到中俄这边,才会愿意跟美国翻脸,就是不是直接翻脸,但是至少是不再予许予求。习近平这次的访问是这个现象,这个大趋势,这个几百年来的一个大现象的一个结果,不是一个因素,要说有什么起因的话。它所造成的果还没有呈现,而这个这果我也是讲了 3 年多了,就是你必须要把美元霸权打下来。我 3 年多前开始说这个未来的这次通胀危机,或者台湾说是通膨危机,会是一个打垮美国霸权的基础的良机,就在于你只要能够把美国的通胀长久化、危机化,



王孟源 59:50 

美元的国际份额就会下滑,下滑之后,他们当初美元份额上升所占的那些便宜就必须要吐出来。那美国以现在如此虚弱的状态,如果他必须要吐出这些当初占的便宜的话,很可能整个霸权就会崩溃。那当然,因为中国在人民银行的那些内贼的影响,这次是不用了,除了他们之外,当然欧盟的那些内贼也是很帮忙。哈哈哈,所以我不敢说哪一边帮的比较多,但是这些都是牺牲自己国家利益来帮助美国的。



唐湘龙 01:00:42 

王孟源的这段的金融的分析是很精彩的。好,刚刚那你对中东的那些描述,其实我觉得对观众朋友,对我来讲都很有启发性。今天照例我要再延长 15 分钟,因为我还有个问题还没有问王孟源,反正你们家在整修,你就陪我聊天。好,我因为这个问题我一定要问你,因为我看到美国的,美国用国务院、国家的等级宣告它在高能物理,在人造太阳、在核融合上面有了重大的突破。可是核融合的技术是中国 2025 的 就是说中国的科技创新当中非常重要的一个项目,每隔一阵子也都会看到中国有这方面的成果的发表。那美国这个时候宣称它的突破,除了在新闻当中、叙述当中所说的能量增益的出现之外,还有什么重点?那也有很多人谈到,就是说尤其中方的一些专家在提醒大家,美国的这一次它有关于核融合的技术的突破,其实带有非常强烈的军事军事的宣告的味道的。为什么?



王孟源 01:01:56 

全都是骗人的。我可以明确的讲全都是骗人的。这一次他们这个 nif所做的这种所谓的 inertial confinement fusion,中国十几年前的时候还被忽悠了,也跟着模仿他,那叫做神光计划,做了神光1、神光2,结果他们发现那个镭射或者大陆叫做激光。他们可以做得出来。但是他们内部检讨之后发现完全没有前途,没有任何使用价值,所以他们就放弃了,而且这不是这两年才放弃的,是五六年前就放弃。那 NIF 设计,为什么一点实用价值都没有?我们没有时间解释,我只是跟你讲一样,它原本最最基本的要求就是所谓的ignition,也就是,它的那个燃料,燃料所接受的能量要它放出的能量,燃烧之后放出的能量要比燃料本身接受能量大,这个叫做ignation,这其实跟工业用途一点关系都没有,这只是一个随便定义的东西。因为实际上他们的那个激光从激光发射到这个燃料球能够接收,已经损失了两个数量级,然后那个激光本身产生又损失了一个数量级,就是你要有这么多电, 10 度的电进去才有 1 度电的能量变成激光出来,所以你已经损失了 3 的数量。然后你产生这些电进去。



王孟源 01:03:58 

外面还有一大堆要运行的东西,然后还有生产你这个燃料球所需要的能量,这个又高了一个数量比,所以你这个,你其实要花1 万份的能量,你的燃料球燃烧才放出一份能量,然后这一份能量放出来以后连热能都不是,更不用说是电能了,它是快中子。这个快中子非常不喜欢直接转化成热能,更不用提变成电能。快中子最喜欢做什么?



王孟源 01:04:39 

把你的周围的那个原子打的,鸡飞狗跳,也就是说你建一个机器,旁边的那些机,包括你的钢铁的护壁,那个钢铁的晶格都会被打得乱七八糟,那有任何一个专家,所谓的“专家”或者是哪一个评论员敢跟你说这个有什么价值、任何价值,你问他一句话就好了。每一个核融合所产生的快中子会扰动多少墙壁内晶格的原子?就是你这个快中子打出去以后,外面的这个钢铁墙壁里面的原子被他打中以后就会跑出去了,跑离晶格,那你这样你这个钢铁的晶体就完蛋了,对不对?你问他一个简单的问题,一个快中子会损坏多少个晶格?他如果答的出来的话才有资格,做评论。核融合的那个快中子,14 MeV ,一个快中子可以打坏 100 万个原子晶格, 100 万个晶格



王孟源 01:06:07 

世界上没有任何材料能够承受的了工业化的核融合。



唐湘龙 01:06:13 

你这样讲我就懂。



王孟源 01:06:18 

这是第一个物理上的问题,所以核融合本身就没有任何工业意义。第二个问题是核融合用的Tritium,就是氚,这个是一个氢的同位,是世界上最贵、最毒的元素之一。比煤比铀,就是我说的是那个铀燃料就是核裂变用的燃料毒上几亿倍,贵上几百万倍。要是几千吨几万吨的烧这种,世界上比黄金贵几万倍的东西,你要几万吨的去烧,你这些人是不是头脑有问题?



王孟源 01:07:14 

那基本上整个核融合就是一个骗局,然后在这个骗局里面还有更离谱的骗局,就是这种所谓的激光,是吧? inertial confinement fusion,因为它比普通的那种所谓 Tokamak的那种的离子体,或者我们台湾叫电浆了。电浆式的核融合设计还要更离谱,所以中国在五六年前就完全放弃了。



唐湘龙 01:07:48 

哈哈哈哈,好,我我打个岔,我记得我在跟王孟源聊的时候,因为王孟源我在简单跟大家说,王孟源在台湾他一直都是资优生,那求学的过程当中除资优之外也跳级,他念物理,那当然我们都知道,就顶尖的脑袋呢,念物理天经地义。可是我记得你跟我讲过说当你在研究高能物理,研究到一半,你之所以就离开了,你觉得高能物理是个骗局,跟现在你所描述的情况是一样的吗?



王孟源 01:08:23 

我的博客有详细的讨论,我认为高能物理的骗局,比核聚变还要糟糕。



唐湘龙 01:08:32 

哈哈哈哈哈哈,所以自己觉得你被骗了半个半辈子,所以才改行的。



王孟源 01:08:39 

我在哈佛念书念到一半,我就了解他是个骗局了。那你这时候已经快要拿到博士了,你能怎么办?



唐湘龙 01:08:54 

好,然后你继续。所以现在我们在谈的核融合,虽然大家谈的很开心,然后美国用它的国家等级的来宣告这个科技的突破,你认为那仍然只是个宣传,是个骗局而已。



王孟源 01:09:10 

完全是骗局。美国人自己也知道是骗局。连核融合的人,自己做核融合的人自己也知道是骗局,他也不好意思再要钱了。这个完全是十几年前骗到钱,结果原本应该在 10 年前就达到的目标,一个随便定的目标,一直到现在才达成。达成之后反正是跟政府公关报告,他们也不需要什么诚实。现在美国学术界也不讲诚实,所以自己人,只要是这个行业里面的人,只要学过物理的人都知道他们是在骗人,他们跟政府的报告还是写得漂亮,然后国务院拿到了,刚好拜登现在政治有需要吗?为什么有需要?因为他把那个搞新能源把,但是今年刚好遇到那个石油跟天然气涨价,我跟你讲过美国的柴油现在的价钱是两年前的 3 倍。



唐湘龙 01:10:17 

哈哈,这是今天王孟源他不肯去买柴油,然后烧他的发电机,他宁可穿厚衣服跟我连线。



王孟源 01:10:27 

不是,这是原则问题,我宁可把钱拿来把这个中央空调系统改成热泵,我也不要拿去送给石油公司,这是原则问题。所以他们就是今天来处理,因为这不是一天两天的事情。装完之后还要来检修这一。



王孟源 01:10:53 

Biden有政治需要。Biden的政治需要是什么?现在共和党人在说,你看我们现在柴油这么贵,都是因为你搞这些新能源的。所以它需要把这个“核融合”。



唐湘龙 01:11:14 

好。唉,怎么样了?信号断了,我为什么没有声音了?



王孟源 01:11:22 

(没有讯号)城市聚聚困的是无限多年。xxxx



唐湘龙 01:11:27 

好,所以基本上这也是一个就是说科技界的骗局,因为王孟源是我认识的所有的,尤其是学理工科的,第一个彻底的否定了自己学的东西是个骗局的,然后觉得自己被骗,念到博士念到哈佛,觉得自己被骗了。王孟源是第一个。所以这个是一个奇葩。就是。



王孟源 01:11:53 

事实上这样。我研究所毕业的时候是 90 年代初期。那个时候高能物理在 1974 年有最后一次突破之后,整体的理论突破是0,在那之后所有的理论论文都是骗人的。那我在念研究所的时候,那时候还只过了十几年。所以需要一段时间来确定。现在已经过了快要 50 年了,你还要继续被骗的话, 50 年都没有结果你还觉得他会搞出结果来,那这是很奇怪的事情。而且50年论文已经写了将近百万篇了,这里面你只要能够拿出一篇高能物理的理论论文跟我讲说,这有什么价值?我就认输。



唐湘龙 01:12:48 

好,这当然也大概说明了为什么这些年的诺贝尔的物理学奖,就是高能物理几乎都没有受到太多的青睐。好,因为时间的关系了。不过我跟孟源在沟通今天节目的时候,你说你要推荐一本书,你忘了。



王孟源 01:13:08 

差点忘记了。其实我的博客有讲。我自从 2016 年开始无意间引发了大对撞机的论战,因为我说这个高能物理是个忽悠,这个反应起来对国家的损害就是他们想要去建一个大对撞机,这个对撞机实际上应该是总花费会达到 1000 亿美元的程度,当然他们是低估了十几二十倍。然后 2016 年的时候就因为我的一篇很诚实的博文,我说这是骗人的东西,结果被香港的一家媒体,拿去转印的,转阴以后,有人因为——你知道丘成桐是香港人吗?丘成桐刚好是这个忽悠集团的一个成员。



王孟源 01:14:12 

那这个香港的媒体你也知道了,他最喜欢去搞事,所以他们就把我那篇文章拿去给丘成桐看,然后丘成桐就把我臭骂了一顿,说我这个人名不见经传,然后什么?没有,什么成就。我当然没有成就了,那个忽悠人这种事情我怎么会有什么成就?整个高能物理就是忽悠的,一介忽悠嘛。我怎么可能会在那边搞出什么成就来?那他骂我的时候骂我倒也罢了,把我顺便讲到杨振宁杨先生,说杨先生绝对不会讲这种没水准的话。结果又有人大陆也有人喜欢搞事,就把这些这篇文章拿去给杨振宁先生,他说啊,我不管你说我有没有水准,但是我同意这个,OK。



唐湘龙 01:15:13 

OK。好,噢,这样这背景我就懂了,噢,好,然后书呢?你说那本书?



王孟源 01:15:19 

然后那是我第一次对政策。大陆的政策有影响,然后来在 2019 年之后介绍了一本书,就是我发现自己写博文有影响,需要很大的运气,因为我没有什么名气。那,但是 2019 年的时候我把一本书介绍给华语世界,叫做美国陷阱,我不晓得你有没有听说。



唐湘龙 01:15:46 

有有有有有。



王孟源 01:15:48 

对,那一本书英国跟美国的出版商都不愿意出版,所以它没有英文版。但是我注意到它有那个有文章所提到有,这个人在法国出版了一本法文的书,所以我把它介绍给大陆的媒体,然后大陆的媒体就把它翻译转译出来,然后后来有了很大的影响,就对孟晚舟的那件事情是有切身相关的事情。所以这次我又看到了另外一本书,我觉得可能有类似的影响,所以我正在设法在复制 2019 年来自经验。那这本书的作者是一个德国的中国通,就是。



王孟源 01:16:34 

所谓中国研究的专家,他叫做 Thorsten Pattberg 。这个人我其实以前就知道了,但是我一直到上个礼拜才注意到他其实写了一本书,而且这本书是用英文发表的。但是我到 Amazon 去看了以后觉得很可怜,因为没有人买,它是去年出版的,但是根本没有人买,连一个评论都没。但是这本书的重要性在哪里?因为他自己就是德国中国研究这个学业出身的,所以他完全了解这个美国是怎么强迫德国的这些所谓中国专家。在新疆的这件事情上面撒谎。他把自己的经验写下来,他解释了德国的这些学术、文化、媒体,解释怎么样被美国人强迫着。撒谎。然后我刚刚不是说过 EU 自愿的,自愿的舍身喂鹰,其实佛教经典里面有的案例。



唐湘龙 01:17:47 

没错。刚好。



王孟源 01:17:48 

就是舍身喂鹰。是,这其实是因为他们的学术、媒体、文化界。被美国人完全渗透掌控。那你这里有一个实例就摆在那,我希望能够有人大家去传上传播。



唐湘龙 01:18:03 

出来。他有翻译本吗?他有中文翻译吗?



王孟源 01:18:07 

我现在正在鼓励大陆的媒体公司去翻译出版。



唐湘龙 01:18:13 

但是它的书名是什么。



王孟源 01:18:16 

不太文雅,叫做 Shove Your Democracy up your ass。好,你你的懂吗?



唐湘龙 01:18:20 

哈哈哈,这样的,就是闭上你民主的狗嘴。差不多。这意思是不是?



王孟源 01:18:34 

不是不是,就是Shove 什么东西up your ass 是鸡奸的意思。



唐湘龙 01:18:39 

好,这个这更口语的翻译了。



王孟源 01:18:43 

难翻译。这个很难翻译。对,不论如何了,我希望能够翻译出来。然后这个重点是中共的外宣其实是一直很薄弱,但是是过去这两三年进步比较大的一个方面,就是他的现在的内宣是问题很大。这个我刚刚不是提到说我认为这个新旧交接以后,对金融跟战略的主管会比较有自主性,但是我觉得他们要也必须要对内宣做出彻底的整顿,但这一点我就这一方面我就不太有信心。因为还看不出,习近平自己带上来这帮人,有这么深刻的见识。不过不管怎么样,他们在外宣上面虽然有进步,但是在战技战术上面还是有缺陷,尤其是新疆这个话题,基本上没有反驳,就是只是做到能够让第三世界和俄国这些国家不愿意,不愿意接口,就是不愿意随着随之起哄。是,但是他们心里还是有疑心的,但你必须要真正澄清。根本这些都是无中生有的东西的话,必须要直捣黄龙。直接指出那些在鼓噪的那些人,他们是从哪里出生的?为什么会这么积极的去鼓噪?



王孟源 01:20:24 

OK,那所以这本书我觉得是个关键。因为他是难得的行业内的良心人。你如果能够把他这个书翻译,然后在中国卖出去,一举两得。因为你中方的外宣得到了很好的材料。然后我们这一位良心学者也得到经济的回馈,而且这个回馈是来自卖书,他们没办法说你是被中国收买。对,我在中国卖书。如果你怎么说是被收买。



王孟源 01:20:57 

对。



王孟源 01:20:59 

所以我希望大家有空去 Amazon看看。就是这本书是Thorsten Pattberg所写的Shove Your Democracy up your ass。



唐湘龙 01:21:11 

好,今天时间的关系,因为我,我要,我跟,我要跟王孟源聊可以再聊个两三个小时都聊不完了。不过要不要今天先感谢孟源?来,我感谢我们几位的,就是说得超级留言的观众朋友们,来 brother of can 001 感谢再来炎黄子孙华子华子说,第一次看王博士的评论是在史东先生的八方论坛。对的,我也我也,我最早也在八方论坛里面看到孟源。好,那王博士是位了不起的学者,是值得华人尊重的学者,他讲话很直白,所以有的时候有时候听他讲话,人有时候会觉得被他伤到。哈哈,因为他是不跟你拐弯抹角的。好,值得华人做尊重的学者,期待晋级哈。



唐湘龙 01:21:58 

可以再看到王博士的精彩点评,感谢然后的如 t 刘,感谢然后焦林特,谢,谢谢你 Po 伦,谢谢你,这些都是我们资深的听众。然后乌艾瑞斯,他说我十分厌俄国内媒体跟文艺圈子里面的自由派,虚伪双标,那不食人间烟火但又常常自以为代表底层的利益,他们甚至都不知道真正的底层是什么样子。他说我去过云南边境最偏远的乡村,他们 5 年前才有抽水马桶,全村只有一个赤脚医生,他们才是真正的底层。我不很不认同政府向这些举白纸的愚民妥协。好恩森尼尔罗,感谢。



王孟源 01:22:36 

我自己也是到 11 岁。还有抽水马桶可以用。哈哈,在 10 岁之前我们家烧饭是用煤球。



唐湘龙 01:22:43 

没钱。我们的世代不一样好吗?你都还比我年轻一点。



王孟源 01:22:50 

那我真的是在猪圈旁边玩土长大的。这我对农场……



唐湘龙 01:22:55 

对对,农村是很有感觉很有感觉。



王孟源 01:22:58 

是有感情的。对对对,这种事情,你想我为什么会成为一个社会主义者?



唐湘龙 01:23:03 

哈哈哈哈哈,因为梦人是在最近我们在讨论很热烈的台湾台南的农村长大的。好再来的EVA,这个EVA,谢,感谢我梦月老师好,然后喜欢你们的节目,加油。谢谢,谢谢你。好的 Candy 888830 感谢再来的乌艾瑞斯,他说在北京取消常规化的核酸的前两周,微信和微博就开始大量传播关于核酸的负面新闻,从核酸造假到核酸赚黑钱等等,有的十分荒谬,甚至于连核酸能够测出水痘流感是短视频都有。好等等。好再来零零散感谢。他说,王博士的节目不能错过。对,我完全同意。



唐湘龙 01:23:46 

1979 年的生活来上周二在龙凤被提到科普,邀请温铁军温老师上节目,那聊天室也提到王孟源先生也对温铁军很推崇。好,我们再找时间安排,因为他们都在,都都不在台湾了,我其实要安排他们的时间,以及要请他们克服时差。这个其实对我开口其实是有点为难的,像莫言,他要配合我,时间其实我都非常的感谢那苏战s,感谢他,王博士一定要支持安东尼,他的乌克兰的土地动石了之后,俄罗斯会不会发动了全面的进攻。



唐湘龙 01:24:21 

好,这我们下个月再来聊这个大议题, RS 流,感谢,受益良多,背靠祖国向东行,感谢王博士的精彩分析,谢谢谢再来。原美国加州的华人手抖,从史东先生的节目认识王孟源博士,看了几集,那王先生参加了龙行天下,觉得这一起是最精彩的。一起回来你要跟让王孟源很自由去讲究就可以了。好了,挥全陈应该陈辉。好,感谢这个 WK 骂,谢谢 2035 去台湾,谢谢。太喜欢王博士的科普扫盲这个字眼用得很好。科普扫盲?对,他在旧金山的 a 台通他很多人都看你在八方论坛的这个节目,一直追王博士,希望在微博上面发表文章的开始解惑再来周碧莲,谢谢你再来菲雷尔的书,谢谢。然后德江燕,感谢林赖。对,好的,谢谢王博士的精彩的讲解。好,因为今天又多用了王孟源 20 分钟的时间,不过我还是那非常没关系,对,我非常感谢。默,你看穿着厚衣服随意,他是不开暖气,穿着厚衣服在跟我连线,所以特别的感动。好了今天,今天不管我们的观众朋友在什么地方听孟源讲话的时候我就静静的,听的时候都觉得受益很多,感谢孟源,也感谢今天收看龙行天下,下个月我孟源来的时候再跟大家相会,感谢孟源。周末快乐,拜拜。



王孟源 01:25:55 

新年快乐。



王孟源 01:25:56 

拜拜。



\twocolumn[\begin{@twocolumnfalse}
\section{俄乌战争会扩大吗?}
\subsection{20230203}
\end{@twocolumnfalse}]6月15日,勘误完成。



唐湘龙 00:00 

最后我们再来谈的川普,谈美国的总统大选,好吗?



王孟源 00:02 

yeah,OK



唐湘龙 00:11 

标题就是俄乌战争会扩大。Okay,来,准备来。芝麻开门。迟到了?没有迟到了。



唐湘龙 00:35 

对,好,迟到编,我不是迟到了。OK。



唐湘龙 00:55 

今天星期五早上的 9 点半到 10 点半钟,一个小时的时间。那再稍微解释一下,那因为上个月,上个月因为春节假期的关系,所以我没有听到王孟源的声音,那我真情告白就是我没听到王孟源的声音,我还挺别扭的,挺难过的。所以我就跟王孟源说那过年期间因为大家放假跳过了,那是不是过完年之后就是再多多增加一回?那我王孟源也很阿萨里(注:粤语:直爽,大大咧咧的意思)说好, 2 月份,那我们就安排了两次的王孟源时间。好,所以开春之后的第一场的龙行天下,那在弥补大家在 1 月份过年期间的听觉上面的跟脑袋的损失。好说,我请了王孟源上线。好,今天龙行天上的王孟源的时间来,我们透过越洋的连线那人在美国,但是看起来没有什么过年味道的王孟源了。孟源,新年快乐。



王孟源 02:01 

大家好,新年快乐,我在这里是没有过年。



唐湘龙 02:05 

对对对对,就是我感觉上面就是跟你讲新年快乐呢,其实也很别扭。好,所以你在过年你也不会说想要出去玩一玩哪,等等,不会。



王孟源 02:14 

就是送几个……我连圣诞卡都懒得寄,我都是直接 email 给亲朋好友说,哈哈哈,新年快乐。



唐湘龙 02:23 

好,他连一般的这种的社群软体,社交软体他都不用的好,所以这个人连我要跟他联络都只能够用最古老的方式用email。很多人的 email 都已经快消失了,可是对王孟源非常的重要。好,这些先放一边来,我们回到了这一个多月的时间,我之所以这一个多月,我也很诚实讲了,我很想找王孟源聊,原因是因为这一个多月的时间过去,我们长时间关注的俄乌战争,其实他焦灼了一段的时间,那看起来双方面也都各自在做动员准备。在我们上次连线以前,但是这一个多月的时间,俄乌战场,尤其在东部战区的这个主要的战区看起来是有很大的变化的。好,那现在俄乌战场的现况如何?以及第二个就是我们也注意到西方国家,美国为首的西方国家北约现在对乌克兰战争的投入本来最少,德国、法国这些老欧洲对于俄乌战场当中到底还要表态到怎样的地步,是很犹豫的,不想把整个欧洲给拖进去。可是现在以北约为主架构的看起来对于这些老欧洲的施压是有效的,那德国、法国这些国家,那更不要说,像波兰、意大利这些国家陆陆续续的都已经做了承诺了。好,那这个在战争看起来是会进一步的扩大。好,那待会如果有时间我们再来谈中美关系,包括包括习近平,为什么呢?有可能在这个时候访问俄罗斯,这个我们待会谈。最后我们再来谈一下美国的政局,因为美国的政局今年也是美国的大选年,这个大选年因为美国的政局不只是牵动美国,它也牵动国际事务,美国总统的权力之大,对全球事物都会有影响,尤其中美关系。好,那我们先来看看俄乌战的战争,好在这个月的俄乌战争呢?战场上面我们我所看到的新闻了,我个人认为我比较平衡的在看,那西方的媒体的视角我也看,但是我也去寻找一些曝光的王孟源所提到的一些一些比较可靠的,但是在西方媒体当中不会呈现的一些的战场的事实。看起来俄罗斯在一个多月的时间里面在乌东战区颇有斩获,为什么?



王孟源 04:41 

因为我们一月没有聊天,然后 12 月的时候没有谈俄乌战争,所以上一次谈已经是将近三个月,我先简单的回顾一下 11 月的时候我所谈的那个重要主旨,嗯,就是那个时候。如果大家回顾一下,就是乌克兰在 10 月的时候发动了两场大的攻击,在Kherson跟Kharkov战区都占领了几千平方公里的土地。不过我那个时候就已经 11 月的时候我已经解释过了,就是因为俄罗斯它的合同兵,事实上它全部的总兵力 100 万人之中只有 20 万是合同兵,而只有合同兵能够在境外作战。那这个合同兵的合同是这样写的,你如果在境外作战 6 个月之后就可以就合同到期,自动到期,你可以拒绝签续约,所以他们从去年2月开始打到8月的时候就有一大票人,嗯,退休了,退伍,哈哈哈。那退伍了之后他们原本全部总加起来,连后面的空军跟后勤加起来也就是十八九万人了,一退就变成十万出头,十万出头以后就很惨。为什么呢?因为这个时候乌克兰的战兵——就是不是在后面,不是后勤的,直接在前线作战的部队有大概 20 万人嗯,从开战一直到去年年底一直都是 20 万人,就是他人死伤以后就赶快再动员一次,他现在已经动员到 10 次还是第 11 次了——他现在已经是满大街抓人了。那个你如果有兴趣的话,有很多网站可以看到他们在,有大概几百个这种视频就是,但是真的是非常的可笑。那然后俄军的战俘里面也是老的老,小的小,甚至有成队的女兵,女兵上前线打步兵当步兵用,这真的很惨。那我那个时候就已经跟大家讲,俄国会容许自己落到 1: 2 的战力比。这个这么悬殊的比率,原因是因为Putin他的第一优先始终是国内的民意,那国内的民意也就是要激起过去两次所谓卫国战争, 1812 年的第一次卫国战争跟 1941 年的第二次卫国战争。事实上整个俄国的国家观念就是建筑在这两次卫国战争之上。而过去这 30 年在苏联分崩瓦解之后,俄国内部的向心力也被昂萨的宣传洗脑体系。没完全有很多的带路党卖国贼。



王孟源 07:55 

那所以你可以看出去年一整年普京就是很高兴了,你们批评我对不对?好,请你逃到欧洲去,你们不想服兵役对不对?我事先就先放你们出关,你们要逃就干嘛?哈哈哈,就是这些不爱国的人,他知道是没有办法挽回的,所以最好的办法就是把他们踢出国门。那然后经过八九个月的苦战,那并没有,他也没有急着要获得胜利,他其实真的要获得胜利的话,提早动员对不对?但是他一直拖到8月,兵员捉襟见肘,然后因为北约的情报一直都是在帮着乌克兰的,所以乌克兰很简单就可以看出哪边的兵力最薄弱。嗯,所以 10 月的时候就是打那两个兵力最薄弱的地方,所以很简单就占了地。



王孟源 09:04 

那当时我如果大家记得的话,我就说这是他们的高位水线,从 11 月开始这个乌克兰不可能再往前推了,就是从这开始之后就是俄军要反攻了。嗯,那为什么呢?因为那个时候普京已经宣布要动员了 30 万人,然后俄国现在的那个兵制非常的复杂繁琐,除了正规军之外还有动员兵,正规军里面又有合同兵,然后还有各地的民兵,以后还有志愿兵也是不一样的。然后还有私募兵,就是那个所谓瓦格那个,对,他们基本上是百花齐放的那个态势。你如算的话,除了这 30 万的,而且动员的是必须服过兵役的,服过至少两年兵役的才他们才要,你如果是只是个壮丁的话,他们不要,那还很挑的。然后动员了 30 万人,然后在这段期间又有志愿的至少8万人,嗯,所以光这些就是 38 万,然后你再加上原本就有的 10 万出头,大概 11 万, 12 万刚好是大概 50 万的比例。



王孟源 10:30 

那我刚刚提到乌克兰的前线的战兵,事实上就是一直只有 20 万,而且它的数值一直在往下掉,就是正规军已经被打残了,两次打残,然后在动员补充,已经两次了,两个轮回了。那Putin它的第一优先是要激起国内的爱国心,这个在去年 10 月就已经成功了。嗯,他敢大规模的动员就代表国内的民心可用。



王孟源 11:05 

那他的第二优先是什么?第二优先是国外的第三世界的意见,其次中国,印度,还有其他的BRICS,就是南非、巴西,还有拉丁美洲,非洲,尤其中东这些国家,他们基本上在过去这一年都明显的站队到俄国的后面,所以非常的成功。所以这他在去年年底是把他的第一优先跟第二优先都满足了之后。



王孟源 11:40 

所以我就觉得既然你当初因为必须要优先解决的问题已经解决了,那然后又已经动员了,只要这些动员的兵力经过训练就应该可以投入作战,所以如果真的要急的话,这些动员兵经过两三周的紧急训练也可以投入。但是你如果看过去这 3 个月虽然暂时逆转了,其实这些新动员的 38 万人都还没有上前线,嗯,他们顶多就是在战线的后面当预备队,真正在打的还是当初的那 11 万 12 万的老兵,嗯,然后他之所以战局能够反转的主要是因为他放弃了那些次要的方向,能够专注到他重要的地方,所以兵力集中,然后又有这些预备队在后面,所以他不怕被穿插。



王孟源 12:41 

因为过去这一年你看到俄军有很多死伤的那个视频,其实都是因为它的兵力太少,所以那个前线没有办法完全布防。所以乌克兰很简单就可以送一些轻步兵去穿插,到后面去做袭扰的时候,就偷袭他们的车队或者什么,然后你就会有那个视频,然后事实上乌克兰他们很重视这些宣传性好东西,因为只要他敢发,整个西方的宣传体系都会一个转发。所以实际上,我们在上个月有一个网上有外传,以色列情报部门的估算就是到上个月为止。



唐湘龙 13:37 

没错,那个差距非常大。



王孟源 13:40 

俄罗斯死伤了总死伤是 16000 人,而乌克兰的死亡。对不起,俄军阵亡了 16000 人,然后乌克兰阵亡有 157000 人。嗯,OK,就是将近 9: 1,大概 9: 1 的那个比例。那你如果看 BBC 的统计,但是BBC的统计是真正更详细的统计,它是真正去看那个俄国所有的丧事。他把俄国国境内的所有的丧事的通告在过去这一年都去一个列举出来,他们一共列举了 11000 多个,到上个月为止。那当然这不是完整的,有些丧礼不会在新闻上报道。然后东乌的民兵当然也不会在俄国的报纸上是刊登他们的丧礼,所以,但是你可以大概估计就是2万,可能稍微少一点,他都是那个俄军的阵亡,俄军这方面的阵亡的数字。然后乌克兰则我觉得 15 万多是很合理的。那所以你光看这个数字,然后再看看他们的那个重武器的战损比。



王孟源 14:59 

嗯,乌克兰从开战的时候有 1000 辆可以用的坦克, 3000 辆可以用的装甲车,然后有 3000 门大型火炮,这些武器全部都打光了以后,去年夏天北约已经紧急的送了一波,就是 600 辆坦克,然后大概 1000 多辆到 2000 辆的装甲车。嗯,然后 100 门的火炮,这些都投进去,现在又差不多要快要打光了,所以他又紧急的在到处要饭。就是,哈哈哈,沿街乞讨。但是很奇怪的是,就是这个冬天,俄军虽然是反守为攻,他并没有发动大幅的攻势。嗯,那我觉得这里的解释我先从战略上来解释到战术上,有理由,就是今年的暖冬,我们这里美国东北的New England也是一个暖冬,到现在都还没有积雪过。



唐湘龙 16:10 

是真的吗?可是我看到美国不是有很多地方都下大雪吗?像加州的北边。



王孟源 16:16 

那个是中西部。OK,我们这边整个冬天都还没有积雪。



唐湘龙 16:20 

真的。



王孟源 16:21 

从没有积雪超过几个小时,就是太阳一出来薄薄一层雪就融掉了。嗯,那欧洲也是这样,欧洲今年也是10年 20 年一见的暖冬对不对?所以一方面那个天然气的消耗比原先的预期要少很多。没错,那另外一方面乌克兰的那个土地也没有冻实,所以不是完全适合机械化作战。不过这些都是次要的因素,你如果真的要打的话是可以克服的。



王孟源 16:52 

那我觉得俄军之所以没有发动大家都预期的冬季攻势,有两个原因,第一个是他们真的很珍惜这些士兵的生命,就是这些动员兵他们现在当预备队就可以在后方前线稍微后方一点,继续的训练,继续的融合,继续的操练,所以就是等于不是两三个礼拜的训练,而是两三个月的训练,这是一个很重要的原因。



王孟源 17:31 

但是还有另外一个更重要的原因,我待会解释我的这样说的理由,他我就是我认为整个俄军有一个战略的转移,而这个战略转移是他在去年秋天秋冬之间观察到一个现象,就是他发现他在春天的时候发现美国跟欧洲的经济跟金融,还有外交的力量完全没有他们想象的那么强,但是到了秋冬之间他们发现连军事也没有那么强。



唐湘龙 18:11 

这跟大家想的不一样,因为我们平常在台湾好像,除非你真的很认真了,像王孟源一样会去刻意的去寻找一些被压抑的讯息,否则现在西方的传媒上面来讲,俄罗斯简直就像快要垮了一样。



王孟源 18:27 

对对,事实上俄国。现在去年它的 GDP 降了 2 点多。2 点多percent ,(唐:比预期中好很多)你想想,它是完全替换了它的工业,就是它原本它的工业都是欧系的,现在是整个要替换成中系的,是对不对?你这么大的一个替换,它居然 GDP 只掉了 2 点多。



王孟源 18:52 

嗯,你说这不是大获成功了,那我不相信。你有哪一个国家能够把整个体系在一年之内从一个国际体系换成另外一个国际体系,然后没有什么经济衰退,整个一年掉 2\% 点多,没有通货膨胀,然后所有的民生必需品的供应都非常充足,你随便到网络上可以找到他们超级市场的情景,是那个比美国还要那个货架还要满。



王孟源 19:29 

那但是去年我们刚刚不是提到那个在战争前半年了,乌克兰就把他们原有的那些重装备打光,打完打光了以后就开始拿美国的装备。事实上在战前,也就是两年前,美国已经运送了几万发。我这个我以前去年有提过,我不晓得大家记不记得,在战前美国已经运送了9万发那反坦克导弹,然后还有另外还有成千、几千发刺针飞弹,这是为什么……这些导弹这些武器先不先进?足够先进。



唐湘龙 20:16 

其实很先进。



王孟源 20:17 

足够先进。数量够不够?非常的够,9万发,比全世界现役的坦克加起来都还多。OK,那。但是能够足不足够左右这个战局?不足够,因为俄国真的是世界一流的军队。你这些武器顶多就是头两三个月有影响,然后他慢慢的就适应了,适应以后就克服了。就可以,这是可以用新武器或者新战法来,几个月内就克服,两三个月内就克服的事情,然后克服之后美国就必须要送新的所谓的 wonder weapon 奇迹武器。后来的奇迹武器是什么?像那个M777,火炮结果证明是非常的差劲。那个M777,美国的这个钛合金做的轻量火炮,嗯,事实上它很慢,它的撤收很慢,然后射程又很短。嗯,还不如那个,还不如乌克兰原本有的那种 60 年旧的老火炮。



王孟源 21:31 

他这个为什么会这样子?是因为M777 原本设计的头号要求并不是性能好,而是要能够轻到用直升机空投。就为了那一点,他就设计成非常贵的,成为全世界除了后来中国也仿制,也不能说仿制了,但是依照它的那个性能发展了自己的版本。但是除了这两个国家以外,没有国家需要这种东西。嗯,没有国家需要能够用直升机空运的大型火炮。那这个M777 去的时候也是炒作了大概两三个礼拜。那炒作时间很短,你可能不记得。为什么?因为一到就被揭穿是没用的。为什么?因为俄国就刊出来一大堆那个 M777 被摧毁的那个视频,然后下一个神奇武器是什么?是海马斯。对,这个海马斯就比较有意思,就是它的射程比俄国自己的火箭炮长,然后又是有制导的,所以它是很精准。那,但是它的有效期限就比M777 要久一点,但这也就是两三个月。



王孟源 22:55 

现在你还有没有听到什么海马斯致胜了?没有!为什么?因为你可以用战术跟武器来克服的,它的战术就是把你在——在那之前俄国人不在乎说在 30 公里,前线的 30 公里之外,他们就考虑是安全的,所以就比较随便。那你这个海马斯的射程有 80 公里,那就打这些目标嘛——那俄军只要好,把这个前线地带的,可能被打击的那个地带改成 80 公里不就很安全了吗?对不对?就是你改变你的行为就可以克服他。然后他们的短程防空、机动防空武器叫做Pantsir,这个 Pantsir 原本打打海马斯,拦截海马斯效率不是特别高,这是因为它原本是为了导弹设计的,单人飞机的设计,那你换个软件嘛,这个软件也是两三个月就来了,那后来到了第四个月他们居然还有新的专门的Pantsir用的新导弹用来专门来拦截海马斯。



王孟源 24:15 

就是这种东西都是两三个月,三四个月顶多就克服掉,然后就Wonder weapon 就没有用,所以我们才会说,你如果不是只有金鱼的记性——这个美国英文里面有那个有一句话说你的记性只有金鱼那么长,就是过目即忘的意思——但是问题是受这些西方传媒影响的读者,他们真的就是只有金鱼的记性。



王孟源 24:43 

就是每个月都有新的 wonder weapon,然后大家都说俄军要完蛋了,但是他没有想到上个月的Wonder weapon  到哪去了?上上个月的Wonder weapon 到哪里去了,对不对?上个月的Wonder weapon  到哪里去了?现在的Wonder weapon , 上个月的Wonder weapon 是什么?是豹二坦克嘛,豹二坦克跟M1 嘛?。



唐湘龙 25:05 

对,现在还没有正式的投入了。



王孟源 25:07 

还没有进场。对,但是已经炒过了。接下来要炒什么?接下来炒 F16。



唐湘龙 25:13 

但是拜登已经说不会给他 F 16 了。



王孟源 25:16 

现在讲不给F16。等到乌克兰被打在地上,鼻青脸肿的时候,你看他给不给?而且不只是这些东西,事实上他们在给豹二坦克之前已经给了几百辆北约的制式的步战车。嗯,对不对?那个Bradley之类的,还有the Marder。然后我觉得在他们给 F 16 之前,他们还有另外一样东西可以给,就是远程的无人机,就是 predator 捕食者那个。像这种东西其实在他们炒作的时候,乌克兰的士兵都已经去接受训练。嗯,这些宣传洗脑都是只是为没有知识的老百姓准备的。嗯,这个你现在看到他们讨论的时候,其实他们都已经开始送了。嗯,这个传送的这个过程都已经开始了,不过没有什么用。



王孟源 26:18 

为什么呢?我刚刚说过,嗯,俄国在去年秋冬之间普丁做了一个战略的改变,对不对?它虽然满足了战争开始前八九个月的两个优先,就是第一个国内的名义,第二个是国际意见。都获胜,但是他有一个新,他发现他对那个战略态势的认知有了一个全新的了解。这个了解是什么呢?就是北约是纸老虎。嗯,OK。为什么这么说?就是你看他们送这些武器有多么的困难,比如说豹二,你说豹二,现役的豹二,还有那个储存的豹二有几千辆,但是他们能拿马上拿得出来的,半年之内能拿得出来有多少辆? 88 辆,对不对?就是因为他们已经马放南山了,德国自己的豹二坦克也就是一两百辆。嗯,你如果给的话基本上自己就没有了。。



唐湘龙 27:33 

你说的是德国的豹二坦克,德国的。



王孟源 27:36 

嗯,英国给了挑战者坦克,但是他自己的挑战者的坦克还有多少?不到 100 辆。嗯,他这一次一下很慷慨的给了 14 辆,以后剩下自己剩下七十几辆。嗯,你说这像话吗?七十几辆烂……基本上就是一个旅,传统的军事强国剩下一个装甲旅的装备。那俄国没有想到北约的军工衰败到这个地步,没有想到他们的工业制造能力跟军事能力衰败到这个地步。所以他在去年 11 月的时候宣布动员 30 万,这个是只是解决短期的俄乌战争问题,更重要的是在 12 月的时候他有了另外一个扩军计划,这个是永久性的扩军计划,我不晓得大家做就是俄乌战争一开始,在去年春天的时候他就宣布要永久扩军 15 万,就是从总兵力 100 万增加到 115 万,但是去年 12 月的时候他宣布另外增加 35 万。



唐湘龙 28:54 

到 100 万。



王孟源 28:54 

就是整个总兵力会增加到 150 万。而且大家要注意,这里最大的重点是增加的这 50 万全部是合同兵。OK,我说过,刚刚说过,原本 100 万里面只有 20 万支合同兵,所以他才会在境外打的捉襟见肘。现在你再加 50 万,全都是合同兵,而且你去看他们的兵种的话,全都是机动作战,就是像空降兵、空军、海军,那还有炮兵这些东西,这都是攻势作战的尖峰,也就是它的兵力不是增加50\%,而是从 20 万增加到 70 万,三倍多。



王孟源 29:44 

那大家想一想,如果,普京仍然是想继续过去 30 年的守势作战,就是如果北约打进来的话,我们一样先退,就以空间换取时间,就打持久战的话。嗯,有必要花那么多钱雇那么多合同兵吗?嗯,对不对?他要能够雇这么多合同兵,背后必然是他的认为能够在国境附近甚至是自己主动跃出国境打败北约,否则没有理由花那么多钱雇 50 万的合同兵。嗯,那为什么会有这个想法?就是去年还发生了两件事。第一个是北约要扩张到芬兰跟瑞典。



唐湘龙 30:47 

是,但是现在土耳其在踩刹车。



王孟源 30:49 

对,在踩刹车。不过芬兰可能是没有问题的,那最大的问题是芬兰,因为芬兰基本上就是二国第二大城市圣彼得堡的革命。没错,对,你根本就是用普通的火炮就轰,火箭弹炮就可以轰的一塌糊涂。那另外一个是,另外一个是波兰也在,波兰也在去年 12 月,其实就是他俄国宣布扩军计划的前一个礼拜,宣布他也要扩军,也是扩军 35 万,波兰的部队原本的总兵力是 15 万,一下子要扩军 35 万到 50 万,你想想看这是不是两边都准备要打起来?而且你波罗地海那三个小国这一次跳得非常的高,而且摆明了就是如果只要美国愿意,他们可以把战术核武器放在那边,从那边发射到莫斯科, 5 分钟的飞行时间很难拦截。



王孟源 32:00 

嗯,所以你如果遇到这种问题怎么办?如果波兰打进Belarus,打到白俄罗斯军去推翻当地的政权,你怎么办?如果他们把战术导弹、战术核导弹部署到波罗地海三个国家怎么办?如果是部署到芬兰怎么办?原本是没有办法,原本普京是只能打落牙齿和血吞,但是一旦他了解到北约的不但是经济跟金融如此的衰败,连军工跟工业也不行了。嗯,他就忽然了解到可以采行攻势作战,或者至少威胁攻势作战。



王孟源 32:49 

嗯,这是我个人的看法。嗯,就是因为一下子征兵 50 万的,而且都是合同兵,非同小可,这背后的考虑绝对不是随手一勾的,因为合同兵很贵。这才是为什么他们原本只有 20 万合同兵的关系原因,对不对?那所以原本 20 万的合同兵是要到叙利亚这种治安战的战场上,偶尔打一打,轮替打一打,那在真正打大型战争跟北约打大型战,他是准备要我刚刚说的空间换取时间,嗯,对不对?但是那是不得已的事,你如果可以的话可以负担得起,而且有打赢的……你认为可以打的赢的话,那为什么不雇50 万人?反正,反正现在欧美颓势已现,再过 10 年,顶多再撑过 10 年,这些危险就过去了。嗯,就是等中国完全取代美国的霸权的之后,那所以他真的是只需要扩军 10 年,然后撑过这 10 年,不让北约步步禁逼,他的国家的安全就可以确保。



王孟源 34:10 

我觉得这是他的大战略观点,所以从这个大战略的角度来看,我们可以说是从 11 月十二月的时候,你记不记得他去年一开战, 11 个月前他开战的时候叫做第一阶段,然后打了 6 个礼拜,进入第二个阶段。那第一阶段是他的那个目的是为了要吓唬乌克兰政权,希望能够以战逼和,结果没有成功,自己损失了 1300 人。嗯,那很多人说这是很愚蠢的,不过事后看一看,既然他原本就是准备要打持久战,你看看他现在死了多少人? 1 万多人对不对?那前 6 个礼拜死了 1300 人,还不到10\%,那你如果,如果吓唬成功了,嗯,后面的这 1 万多人就不用死。



唐湘龙 35:12 

本益比就很高。



王孟源 35:14 

对,所以你只要事前认为有 10\% 的成功可能,嗯,你就应该去尝试这个,所以这个都是很理性的,他并不是愚蠢盲目地去尝试。



唐湘龙 35:24 

好,我问一些我看不太懂的事情,我请教孟源,但因为现在我注意到,因为孟源刚提到就是说1月我们没有连线嘛? 12 月的时候跟孟源在连线,孟源说乌东战事没有什么太大的这进展就先不聊,那个时候看起来是一个很清楚的准备期,之后其实这一个多月的时间里面,俄罗斯大概每个礼拜大概都会攻下一两个居民点,那大概都是以这个进度。



唐湘龙 35:57 

你总体来讲,从战场来讲,其实一两个居民点范围非常小,但是它都有缓步推进,而且是在主要的战线上面,那不管是上个礼拜在苏勒达尔,或者说接下去巴赫穆特,或者说叫昨天开始对红利曼再重新恢复了总攻,那乌克兰方面可以看得出来它的兵力的损耗。你刚提供的那两个推估的这个时候死亡的数字,那个那不只是精准的问题,而是非常的惊悚,那个比例差距太大了,它就真的是个绞肉机。那我们也确实看到了一些西方媒体的报道,就是现在乌克兰境内,以前我们开会家长说女生出门比较危险,不,现在乌克兰男生出门比较危险,哈哈哈,你男生女生,(王:而且老头子都不能幸免)。对,就是你,你其实中年人被拉的都很多,那个很可怕,就是硬要把你从你的家门口塞到那香型车里面,就把你直接就载走了,能不能回来都不知道。



唐湘龙 36:59 

乌克兰已经打到这个样子,可是我们看到泽伦斯基第一个在西方的,就是说体系里面,美国北约的体系里面看起来仍然是捧着他,就是对泽伦斯司机来讲仍然是捧在那里,所以泽伦斯基还在不断的要那一些重要的一些的国际会议的时候,还是努力的,比如说达沃斯论坛,也还是让泽伦斯基他老婆能够有一个有个表现的舞台。



唐湘龙 37:28 

表示在欧洲的这个舆论场里面,对乌克兰战争的想法还没有什么改变,我们会或许会听对,我们会或许会听到说像是 Kissinger, Kissinger的呼吁要停火了,觉得这个停火最早是暂时性的停火,而不是谈判性的停火,就是未来的局面。先不讲,先停火再说,即使季辛吉会发出这种的声音,可是我认为在欧洲的,在北约的体系里面,对于乌克兰的战争的想法要继续撑下去的那个想法没有改变。



唐湘龙 38:02 

昨,今天是2月 3 号,昨天2月 2 号,昨天是史达林格勒的这个攻防战的纪念日。对,那你看到这个普丁的讲话里面是很有煽动性的,他是觉得不可思议,他讲了两次。他不可思议什么?他说我们竟然在那史达林格勒攻防战的 80 年的纪念的时候,我们又准备在战场上面看到了德国的坦克,德国的豹二坦克又准备要跟我们要开战了。表示普丁对于接下去的战争的扩大,他显然也有他自己的焦虑。政治、国际技术的扩大吗?



王孟源 38:39 

没有没有,我觉得是正中下怀。嗯,你如果去看俄国他们的中小学的课程的话,嗯,就会发现过去这三个学期每个学期都添加了新的爱国教育项目,OK,越来越多。就是你一旦民意能够接受,不把它当成洗脑,没有反感的话,嗯,他就开始加爱国教育。我刚我一直说他的第一优先是要建立爱国意志,就是因为从这些细节你可以看得出来,他绝对不是说不是说光,光是打仗,他的第一个要务是要活下去嘛,一年之前是要活下去,所以活下去的话第一步是金融战。嗯,美国挑起这场战争是最重要的。其实打击层面金融战一旦金融战获胜,然后在外交战又获胜,获得了整个第三世界支持,他其实已经高枕无忧。



唐湘龙 39:36 

利于不败之地。



王孟源 39:38 

对,他已经,他慢慢的拖,这个战场上的前进几公里,退后几公里对他来说根本不在乎,事实上你那个合同兵退役,他也就更好的激起大家的爱国意志。那现在的问题是你要预测未来的战争走向的话,你最重要的一个核心观点就是要了解普京的有恃无恐,又上升了另一个级别。首先你要知道俄国人他的民族性就是很特别的,中国大陆有一句话说他们是战斗民族,其实他不是战斗民族,他们是所谓的心大,就是抓大放小。你如果去看他们二战的过程,那个有很多详细的战史,然后或者甚至当初拿破仑战争战史,他们是只要能够确定胜利,他们不太在乎那个优化的,就是时间、空间、人力这些,他们不太在乎优化的。那就是因为这样子,所以你如果只看西方的宣传的话,他们专门去挑这些出错的细节,他很多啊,成千上百地报导,有的真的,有的假的,那事实上俄国人是真的不在乎,因为它不影响胜败。



唐湘龙 41:08 

好,你刚解释的这一段了让我感觉上面更懂了一些,就是说在新闻当中看不太懂的事儿,就是因为这个新闻当中就像是孟源讲的,比如说他的黑海舰队的莫斯科号被炸了,他的克里米亚大桥被炸了,他的北溪二号被炸了,他的甚至于在乌东的那个后备的军营被海马斯炸了也……



王孟源 41:34 

你要送几百辆坦克,然后每年送 200 辆。能影响胜负吗?



唐湘龙 41:38 

对,就你感觉上面刚刚的我们讲的这些的事件是西方的传媒大肆的炒作跟包装,对,那种乌克兰即将胜利的,而且再加上泽伦斯基的那种演员式的喊话,觉得就是已经快要把俄罗斯给赶走了,要重新恢复克里米亚。所以你在西方的媒体上面所得到的那个印象,都觉得俄乌战争,现在俄罗斯已经到了山穷水尽的时候,哈哈哈,可是现在已经二月三号了,2月 24 号就一年了,就一周年了。好,那一周年之后会怎么样?我个人我们从一个观察的角度来讲,俄罗斯有可能跟北约有直接交手的可能吗?



王孟源 42:25 

有,因为现在的北约基本上是一个美国用来掌控欧洲的工具,然后美国的外交政策是由所谓的Neocon完全掌控他们的,Neocon的特点就是无限升级,他们非常的蠢,非常的邪恶,然后。他们遇到挫败只有一个反应,就是升级。那所以这个升级一定是无限的,无限的话,可不可怕?Putin现在,Putin他自己的评估是不但不可怕,而且是我们俄国兴起的一个最大的契机。



唐湘龙 43:09 

心真的这么大吗?



王孟源 43:12 

我不但已经联合了整个全球第三世界,而且造成了欧美本身金融经济的混乱。而然后我现在连在军工方面我都有自信能够不但打败北约,而且是在北约的境内打败北约。我跟你论证了一个小时,讲的就是这一点。他的这个自信从过去这一年,真的你可以可明显看见这个人是绝对理性、非常聪明的一个人,而且筹划了这些事情至少筹划了 8 年,我认为很可能筹划了 20 年,然后但是从一开始,去年一年前到现在有三个阶段的升级,第一阶段还希望能够赶快吓唬乌克兰和谈;到第二阶段就是不在乎了。



王孟源 44:09 

反正因为我金融跟经济后没有后顾之忧,你要打就打,反正我不可能输。然后到了去年年底已经是你要升级。很好,我还等不及要跟你升级,那就是这样的一个态势。那为什么你看这个美国,美国,你问我说为什么美国国防部现在那边哇哇叫?其实是这样子的,他们原本是把那些导弹送给乌克兰的时候,那是一把好生意,美国官方部的将领如果不是原本就是军工企业的人,也是准备退休之后,去军工集团当顾问。



唐湘龙 44:48 

没有错,



王孟源 44:49 

他们的第一优先是要讨好军工集团。那这些军工集团,你如果是送这些导弹,这当然是很好的生意。但是你如果是送坦克去的话,这些坦克能够承得起俄国的导弹的打击吗?不能的。你这些坦克送多了以后一定会有残骸遍地,这个反而是很糟糕的广告。负面广告。他们海马斯被他们吹了一票以后,现在也不敢吹了不是吗?对不对?能有多卖了多少?没有多少。那你如果被吹得上天的这种Abrams坦克或者是 F 16 在乌克兰被打得满地找牙的话,这反而是对他们的生意很不好。而且事实上光是现在有的订单,已经到达了他们、已经超过了他们的产能。他们必须现在他们说要扩产,而且美国现在已经去工业化这么严重,他们说扩产都是要三年,四年,然后才能够扩产 20\% 或者50\%,对不对?你现在还在再去捞继续升级的生意没有什么意义。



王孟源 46:09 

那就是为什么最近他们又在跟中国那边讲一些鹰派的话,为什么呢?因为陆军的产,陆军装备的产能已经满了。已经没有剩下的产能,但是海军装备还有产能,OK。



唐湘龙 46:27 

他们当然……这个在这个我们待着,待会有时间再聊。刚刚孟源的两个论证,因为这些论证之后都是可以随着事态的发展来检验的。那论云的孟源的一个论证就是说现在的俄罗斯基本上面是认知自己立于不败之地,因此对于未来跟北约的交手是充满期待的,所以甚至于期望冲突能够持续甚至于升级的,对俄罗斯来讲是证明自己的国力复兴,甚至于在北约掌控的势力范围之内打败北约,这个对欧洲的情势跟政治气氛的影响是很大的。



唐湘龙 47:11 

第二个就是西方国家对于乌克兰的军事的投入已经到了一个临界点,它的投入不管是它的能量也好,或者接下去就战争宣传来讲,都可能会处在一个高风险的位置上面。没有错了,如果战场上面出现的大量的豹 2 坦克,或者是美国现在最夸耀的 M1A2 的这个Abrams 的这样的一个坦克的残骸,或者是 F 16 的残骸,那个对美国的那种的耀武扬威的那种的说服力伤害是非常大的。好,接下去我问的这个问题,就是说中国的角色,因为中国的角色在俄乌战争当中一直很隐晦,一方面西方国家也不断的盯着看,也在警告中国,但是我认为那不是西方国家的警告,而是中国本身对这场的战争有他自己的思考。那最近传出来这个消息,虽然在去年底的时候,普丁跟习近平在视讯的时候,那是过年前的,视讯的时候普丁确实有提到要邀习近平到了俄罗斯,到莫斯科做正式的国事访问,但这个事情大家之后也就没有太注意。可是俄罗斯最近又放出这样消息,现在已经2月了,甚至于 2 月 24 号就是俄乌战争的一周年,尤其最近王毅即将到莫斯科访问,那王毅以他现在的身份到莫斯科访问,那铁定大家就会说那你就是在为接下去习近平正式的访问在做彩排的。所以习近平要访问莫斯科这件事情,我认为是八字有一撇的是,虽然大陆没有正式地表达,但是我认为在Schedule上面。



王孟源 48:51 

去年战争一打起来,虽然有很多大陆的不入流的所谓专家学者在那边说要借机讨好美国,事实上习近平是从一早很明显就知道这是这是俄国带头起来起义,揭竿而起为第三世界挣脱霸权的桎梏。所以习近平一直都是支持俄罗斯,全力支持俄国的,所以他去俄国访问绝对是理所当然。我不敢说是未来两三个月,但是绝对是今年会发生的。那事实上昨天我才看到一个视频,我不晓得你喜不喜欢看这些视频。是有一个澳大利亚的军事博主,他访问一个澳洲的去的志愿军,他刚好是在巴赫穆特前线,他是一个军官,所以在。



唐湘龙 49:51 

你说叫巴赫穆特,巴赫穆特。



王孟源 49:53 

在巴赫穆特,24旅里。就是巴赫穆特附近整个正面只有 25 公里。嗯,你可以数乌克兰有 27 个旅,这很可怕。 25 公里应该是一个旅的正面,但是它有 27 个旅,而且这些 27 个旅基本都打残了。



唐湘龙 50:11 

那现在就是绞肉机就在这里。



王孟源 50:13 

绞肉机,对。Soledar,一个 5 公里的正面,至少打残了 6 个旅。就是等到他们撤退的时候,他们败退的时候只剩 5 个旅了,因为其中一个旅已经近乎消失,早就已经撤撤掉了。那,但是无论如何现在巴赫穆特还在继续作战之中,其中一个旅是第 24 旅,然后有一个澳洲籍的军官志愿兵在那边。然后他接受岳阳访问的时候,他说了一件很有趣的事。嗯,他说我们认为这个Wagner group—— 就是因为他们的正面对手是Wagner那个雇佣兵——我们认为Wagner group 里面有大疆送去的技术人员实地整修。哈哈哈哈哈,我不知道这是不是真的,不,至少乌克兰方面认为这是真的,如果是真的话,那这代表中国在幕后其实是全力支持的,这也是我认为正确的做法。所以你提到这个中国的态度,我就顺便讲一下,在私底下中国一定是全力支持。如果看去年3月我写的文章,我也是建议,这是天赐良机,有人愿意带头起义的话,你如果还在背后插人家的,在背后捅刀子的话,你真的是愚不可及。因为事实上美国打俄国并不是因为把俄国当作终极的对手,而是利用俄国来整合北约,这样才能够挟持整个欧盟来专心对付。



唐湘龙 52:02 

他,也达到他的目的了。就是说现在整个北约呢?整个欧盟都不见了,好像只剩下北约了,欧盟已经都不见了。好,刚提到就是说如果习近平,因为习近平上次到莫斯科,上次到莫斯科还是参加了这个圣彼得堡经济论坛的时候,那已经是 2019 年了,那已经是四年前,中间有疫情了,当然也就没有去。但是在这个时候,当全整个西方国家,美国、整个北约、欧盟都在带头在围堵俄罗斯,在攻击俄罗斯的时候,习近平的到访,对于普丁,对于莫斯科、对俄罗斯来讲,那个是非常大的精神鼓舞啊。当然你也可以预期到西方国家一定继续大做文章,就像现在预告着拜登即将要提出的国情咨文里面,他已经预告了他要针对中俄的关系要好好的谈一谈。好,但是对美国来讲,对于中国的施压不是只有在乌克兰或者在中俄关系上面,而是在中国的当面。比如最近你光看到他的国防部长在这过两年后天布林肯要到北京,可是这几天的时间,他的国防部长奥斯汀到了韩国,到了菲律宾,他一样在进行他的战争,准备在进行第一岛链的巩固拖,尤其是它的军方。



唐湘龙 53:27 

你刚提到的美国的,美国的军方几乎保持的每个礼拜都有人讲话的频率,不管是现任的,不管是印太司令,或者是某一个军总的将领,或者是某一个智库的军事专家,几乎每个礼拜而且那个调子都一样,都是在预言战争,都在告诉你说我们的兵推如何。我们对对中美之间的发生战争这些事情态度悲观,顶多时间点上面会有些差异。这种手上抓着军权的军人,现役军人或者智库不断地释放的这种的,就是说战争的悲观的论调。为什么?难道单纯的只是为了要预算吗?



王孟源 54:13 

一方面是因为这些军方将领要替海军装备要预算,但是背后还是有那个Neocon,我刚刚已经讲过Neocon是非常愚蠢,而且只知道升级的,而他们最后终极的目标是中国,这次打俄国纯粹只是,为了整合北约。



唐湘龙 54:33 

没错,练练身体而已。



王孟源 54:34 

所以当然他们会不断的继续在东亚试图裹挟日本跟韩国尤其是。不过我个人是很不希望看到台海战争的。当然我跟你讲一件事情,我对小孩非常希望能够拿到中华民国国籍,但是他现在已经 21 岁了,所以如果他要入籍呢,比较麻烦。要先回去,然后服兵役,然后住上一年。OK,那我跟他讲说你,你暂时不要回去,因为。



唐湘龙 55:13 

你一回来就要当兵啊。



王孟源 55:14 

你如果去当一年兵的话,运气非常的不好。



唐湘龙 55:20 

哈哈哈哈哈哈,这是很现实的事,台湾很多的父母亲很担心这一点。



王孟源 55:26 

对对对,那台海战役会不会发生?其实这个俄乌战争已经把它尖锐化了,一方面是把它从 2022 年、 2023 年推迟到 2024 年,到 2025 年是有……就是原本发生在 2023 年、 2024 年、 2025 年、 2026 年的几率都差不多,现在在短期内发生的几率已经掉到很低,然后在 2025 年前后发生的几率提高了。



王孟源 56:06 

为什么会提高?这个要看民主党在下次大选有没有胜选,因为民主党如果胜选的话,Neocon现在Neocon的大本营是民主党。对,就是拜登本身是一个Neocon的政权。如果我说民主党胜选,我没有说拜登胜选,因为现在看来拜登,我其实已经讲了 2 年了,拜登很可能会被替换掉。那。



唐湘龙 56:34 

你现在仍然这样子觉得吗?他很可能 3 月份就会宣布竞选连任了。你认为一个现任总统来讲,他如果宣布他要竞选连任,在民主党里面有人挡得住吗?



王孟源 56:46 

有,其实你现在去看的话,那个第一个先替换掉的应该是副总统。因为这个贺锦丽这位小姐很不得人缘,而且大家知道她太笨了。他们当然也希望有一个傀儡,但是这个傀儡也不能够笨到那个地步了。嗯,所以那个如果要换 Biden 的话,必须要先换贺锦丽。嗯,那所以大家等着看好了。如果换了拜登的话,他反正有一个备用的人选。已经备用待机很久了,就是Pete Buttigieg。我在我的博客上面还写了,还专门为他写了一篇文章。大概两年前还专门写了一篇文章,有空的话大家可以去看看,为什么他会是下一个民主党的明星?就是台湾如果不想要在未来三四年发生被Neocon当作棋子而牺牲,然后引发台海战争的话,唯一的希望就是Trump当选总统。



唐湘龙 57:54 

好,这个是我们接下去呢好,我一定要让你把这部分谈完。因为美国的选举我们刚一开始提到过了,他不只是美国的选举,我说美国的公民每次到总统大选的时候,他是投,他是选出两个总两个领导人,一个是美国的总统,还有一个是这个世界的领袖。这是现实,就美国,毕竟他仍然是一个,是个 super power,所以每一个美国的领导人他就会有两个角色去操作,一个是作为美国总统,那我应该要怎么样?另外作为一个国际的领袖,那它会发生怎么样的影响?从拜登跟川普这两个人身上,我们看到了谁当美国总统对这个世界会发生什么影响是非常清楚的。尤其现在共和党跟民主党在路线撕裂上面,那个已经是清清楚楚,几乎没有重叠的界面。除了反中以外,那你如何看 2024 年的美国总统大选?你认为川普还有机会吗?但是你对拜登又非常的悲观,为什么?



王孟源 59:06 

因为拜登丑闻实在太多了。



唐湘龙 59:09 

你说最近的,包括他儿子了,包括他最近的这个xx事件。



王孟源 59:13 

他的丑闻太多了,所以,而且他年纪实在太大,那建制派要推傀儡的话,宁可这个傀儡能够……这个傀儡的任务是什么?是搞公关嘛。美国的学术界是资本的公关。那他们的政客则是 deep state,就是深层政府的公关。对,就是总统的第一任务,并不是你说的什么治理全球了或者治理国家,而是为幕后的那些建制派,包括Neocon来搞公关,就是假装有另外一层真正的政府,真正的人,其实他们只是木偶,这个给人家玩的傀儡。那你说建制派,建制派并不是只有民主党才有,建制派是民主党跟共和党都是主流的,但是现在民主党根本没有非建制派的总统候选人可能选。那共和党有Trump,那Trump的是唯一一个不相信Neocon的,就是他不相信用军事手段还有宣传手段去瓦解其他国家的,他相信的是在经贸上作战,所以他如果当选的话,就不会继续搞这些东亚的直接冲突,而是跟中国直接搞经贸的冲突。



王孟源 01:00:53 

那这个呢?我觉得是对中国是一件好事,因为中国现在已经认清在半导体上面必须要自力更生,已经没有再幻,没有幻想。对,没错,那去年底他们的半导体大基金贪污的那几十个人都被抓起来了。嗯,然后对台湾也是一件大大的好事。嗯,因为你没有军事的话,就没有战争的危险。嗯,因为你问中国它会不会主动的提早出手?没有这个道理。嗯,因为他们的第一优先不是祖国统一,而是兴起,而是和平崛起。嗯,所以这个可以继续拖吗?至少再拖个 5 年这样子。那所以如果大家不想在赖清德当总统之后上街当炮灰的话,就最好希望Trump能够当上总统。我也是这样祈祷的,因为我也不希望我在台湾的家人遭受炮火的灾难。



王孟源 01:02:03 

而且这个问题是呢,Trump的真正最大的问题是在共和党党内,就是De Santis,就是他 2016 年当选的时候是出人意料的,整个建制派完全没有想象到他能够能够获得提名,然后获得胜选,所以没有针对性的准备。那现在当然是不一样,他们已经针对性了,准备了一个替代品,就是Florida州的州长,那De Santis现在你去看他的当州长期内的言论就是一副民粹派,就是一个比较温和有教育的Trump,他讲的都是创的那一套,但是他是讲的比较温和,比较有教育水准。实际上你如果看他在Trump出头之前,他当过国会议员,他当国会议员,他的投票跟言行,at完全就是一个Neocon,跟  Buttigieg 一模一样,他基本上是建自派培养的新星。但是一旦Trump下台之后,建制派看到Trump有可能卷土重来,就把他安排让他去做卧底,去争取民粹派的选票,那最重要的目的倒不是要他当选,而是要让Trump败选。那如果。所以Trump真正最大的危险就是他面对 De Santis的时候,党内初选的时候,被打下来了。所以这个我们必须要继续观察了,对不对?当然对,为世界和平着想,我是希望Trump当选的,但是实际上这个几率并不是特别高。



唐湘龙 01:04:10 

所以你觉得川普即使卷土,他现在的姿态当然是要卷土重来的姿态了,他都已经开始开始他的初选的动作了。那但是你认为共和党在提名他的机会也不大。



王孟源 01:04:24 

不是特别大,因为这一次他们有备而来,已经准备了解药了。好,那就是De Santis。



唐湘龙 01:04:31 

那如果川普的机会也不大,拜登的机会也不大。所以你认为 2024 年的美国总统大选,民主、共和两党会是一个全新的组合了?



王孟源 01:04:41 

有这个可能,我没有说拜登绝对会下台了。我是说他会下台的几率在一半,或者甚至更高。就是你不能忽略Trump,我说他不一定能够胜选,这个意思也是说他败选的几率也大概在一半或者甚至稍微高一点,就是我是在谈 40\% 50\% 60\% 的机率,我不是谈 90\% 或者10\%,不过因为这种事情事关世界和平,而且Neocon在乌克兰这样不断升级,到最后总是有核战的可能。



王孟源 01:05:20 

倒不是普京会打核战,因为现在普京老神在在,我们用台湾话;他认为用传统的军事战争它可以打的赢北约,那,所以他当然没有理由用核弹。其实他担心,他一直担心的是北约会输了以后会耍赖,然后就用战术核弹,这才是他真正担心的危险。所以你如果去看去年他谈到核子武器大概有四五次,每一次都是说你们不要想着用核子武器,因为我们比你们还多,都是这样的讲的。然后当然你西方的媒体就是,Putin威胁要用核子武器,哈哈哈,这个实际上他讲的是你们不要想着用核子武器,因为我要用的话我比你还多。这样……



唐湘龙 01:06:14 

好,最后一个问题,好,您先说。



王孟源 01:06:20 

我想我们时间也不多了,所以我想把一些战术的预期就是这种预期。是啊,很不确定的,就是大家,我说一说,大家参考一下。就是,大家都认为会有冬季攻势,那乌克兰自己,乌克兰的高级将领在受访的时候,两个月前受访的时候认为会是1月底2月初,但现在还是2月初,对不对?是的,这原因是什么呢?你有 30 万的动员部队,这些动员的人平均是三十几岁了。



王孟源 01:06:59 

嗯,就是他们是以前当过兵退役。那他们在国内经济有家庭跟社会跟经济上的义务,嗯,你不太适合这样子,要打长期的持久战。嗯,所以大家都认为既然你有,你原本十一二万都可以打的不错,你现在有了 50 万的,四倍的兵力,你可以多面开花,对不对?然后做运动战的,二战那种闪电战,运动战的战法,所以这是大家都这样预期,不过我刚刚也解释过了,就是Putin真的是有恃无恐觉得,所以他现在就算在再多拖一点,也就是这些人多服务几个月的兵役。不过我们在过去这两个月所看到的是什么呢?就是在东线的时候,东线他们可以放心的往前攻,而不会担心乌克兰反攻的,是为什么?因为他们有足够的预备队,然后上个月他在南线也就是Zaporizhzhia那边也开动了攻势,这也是我在博客事先就事先就预测过了,而且我还特别说过这个他一旦动手,大家不要以为他就是从南线主攻,因为他现在的兵力非常的充足, 50 万的总兵力非常的充足,可以围着整个乌克兰的 20 万战兵随便怎么打。所以他即使你是助攻方向去牵制乌克兰的部队也是好的,对不对?省得你在那个在某个方向被乌克兰有局部的兵力优势。



王孟源 01:08:51 

所以接下来几个月我不敢说一定是这个月,但是接下来几个月一定会有北线开打,那北线的话,因为有 1500 公里,其实有四个战线,一个是最东边的,就是Kharkov,这个哈尔科夫;然后往西一点是那个Sumy走廊,这个我 3 个月前有提过,我认为是最可能的,待会我再解释一下为什么我还是这样认为;第三个是基辅的正面;然后在过去就是西乌克兰,那俄罗斯其实可能,我如果我是Putin的话,我并不在乎波兰去把Lviv拿回去。而且那个当地那些人并不是俄国人,他们认为自己是波兰人,所以没有理由去冒那种游击队的危险。所以我认为如果要打的话,基辅正面当然是一个很好的助攻方向,因为那个政治性的威胁很大,乌克兰不能不对应Kharkov也是这个样子。但是Sumy才是一个最合适的穿插方向,所以如果要把北线当成主攻的话,我还是认为从Sumy打下来是最合适,那当然也可以在苏米当做佯攻,然后继续的在东线这样子做绞肉机的战事,那这样子因为你的火炮火力跟人力的优势更加的明显了,你这个绞肉出来的战损比也就更悬殊。所以你基本上现在有两个可能,一个是在北线当主攻,另外一个是北线通通是佯攻,然后东线继续的绞肉这样。



唐湘龙 01:10:46 

好,当然这个这个这预预测了,战术性的预测可以让大家判断一下,就是反正接下去一两个月的时间,每个月跟孟源在连线的时候,我们都可以再重新的检视一下。但是我记得孟源之前有曾经讲过,就是说俄罗斯的战术,俄军的战术基本上就很简单,就是努力的去消耗乌克兰的有生战力,那消耗乌克兰的有生战力,来慢慢的把战事达到他原来要达到的,就是战争目标。以我们刚刚所说到的,现在在战场当中的实际的就是生命损耗来看,那个体量是非常悬殊的,乌克兰在这场战争当中要付出的生命代价,接下去恐怕还很多。



唐湘龙 01:11:35 

至于基辅,如果说北线战士如果开打的话,那第二次再包围基辅的可能性看起来就提高了很多。如果再包围基辅,我估计这次俄罗斯大概就不会像上次这么的这么的轻松以对,大概就会做好充分的准备,那基辅大概就有事了。好,这些都是接下去以乌克兰为中心,但是其实它是东西方的角力,一次历史性的角力,那在我们的所得到的讯息当中,西方的媒体大概都会夸耀自己的战果,北约好像非常的强大,美国有很大的主宰力,可是在战场上面所嗅到的味道又不一样。看起来俄罗斯在这个地方即使不是轻松以对,但是也没有节节败退,甚至于开始遂行他的战术,就是一个绞肉机,不管你投进来多少的兵力,在那地方都会被消耗掉,那个看到的一些讯息其实是触目惊心的。好,今天非常感谢,在过完年之后第一次,因为我要凹王孟源,我每次凹的非常的辛苦,跟他瞧瞧时间这一件,那他每个议题他都一定要做好非常非常好的准备的时候,他来说这个可以谈好。那当然我特别感谢孟源了,在我们所有的谈论当中来讲,孟源都做了非常的认真的准备,同时在事后来看的时候,孟源的预测不止有他的话题性,而且有他的观点,最重要是验证起来的时候的准确度其实还很高的。来感谢了我们的一些的观众朋友,最近大概都是王孟源的粉来黑战,谢谢。他说为什么湘龙可以在林清源说解放军 60\%- 70\% 的中俄边境时毫无波澜?对了,所以这个以后再说,因为我当时本来是因为他在讲另外一个节目,少康战情室,昨天的对话我在当时有想要发言的,不过因为来宾很多,并不是想讲话就讲话,步调就转,就变得有点混乱。



唐湘龙 01:13:39 

好王者风范。他说今天在战情室不敢面对民进党的肺腑之言,真的方向不太清楚,大陆的量子芯片已经研发成功了吗?科技的东西再看看,因为我不太懂。好,再来梁耀文,谢谢。在台湾他说只要是徐晓星,这是徐徐晓,这是徐晓星。来再来 sincere 黄还是前面破论感谢让沈斯娅黄,感谢在加拿大90T,感谢康燕令海外华人支持两位理性的发言,在乱世当中保持清醒的头脑非常的重要。对的,谢谢。然后派个王良,非常喜欢佩服王孟源博士,希望能够多上节目会了。我已经尽量的都不放过王孟源。他连过年都不过的人呐。你知道他平台在干嘛?好 WK 骂,谢谢谢。还有呢,詹德森,嘉德森,无他,说到战术核武器,其实用战术核武器让对手不知道怎么应对,最后吃哑巴亏其实是美国在 16 年左右提出来的理论。顺便上次王博士提到的那本书,我在亚马逊上面直接搜索书名还是搜不到,最后搜到作者真名后,在其他的著作里面翻了三四页才找到的。应该不是巧合,你看王孟源推荐的书,因为这个议题已经好几个人反映过了,王孟源讲了他们去找不到,给他拼命找,所以他们对王孟源是很服气的。好了,今天过完年之后,在元宵节之前, 2 月份第一次王孟源的上线,第一个感谢王孟源,再跟你说新年快乐,这个月还有一回,到了月底的时候我们还会再请孟源上线。感谢孟源。



王孟源 01:15:26 

大家节日快乐。



唐湘龙 01:15:28 

好,同样也感谢我们所有的观众朋友,听众朋友,那经过一个月之后再听到王孟源的声音,是不是有振聋起聩的感觉?好,感谢收看今天的龙行天下,那谢谢孟源,谢谢了所有的观众,下个礼拜见,拜,只有10。



\twocolumn[\begin{@twocolumnfalse}
\section{美国的货币危机与中国的三份文件}
\subsection{20230224}
\end{@twocolumnfalse}]Credit: Anonymous, 栗子

唐湘龙 01:00 



欢迎来龙行天下,没有错。我是唐湘龙。当然这两个礼拜的时间,可能很多人会觉得诶,唐湘龙不见了哈,但是对吧,反正就有些私人的事,那就就不特别报告了。好了今天,星期五的时间,同时,也是在228的连续假期前的最后一个上班日,那呃其实昨天下午开始,我陆陆续续的恢复了一些录音的工作,好,那今天早上的时间呃,九点半钟,龙行天下的单元,因为我承诺过,我答应过,而且我要凹他不容易哈,那个王孟源老师。在二月份的时候,特别安排了两次的王孟源,一方面最近事情真的很多,再来,因为每次王孟源上完节目之后,那个我可以感觉到观众的那个忠诚度,那个热情,那个敲碗的力度,我那个碗都敲破很多了,所以,我只好就说,在凹王孟源,那请王孟源吗,这个二月份的时候,那安排了两次就说,王孟源的讲座单元。



好了,今天的我们会谈的几个主题。在标题上面,我之前跟孟源,在交换一些的一些的时事,议题的看法的时候,有关于美国的这些的货币跟财政的问题,其实我们经常谈到的哈。但是呃,大概到这个时刻,是一个可以给做一个系统性的讲解跟理解的时候。



再来,你也注意到了,今天2月24号嘛那刚好是俄乌战一周年。一周年,那之前的针不管是慕尼黑安全会议也好啦,那拜登的讲话,普丁的讲话,以及中国大陆外交部的这些的讲话,大家都在讲话。但你有没有注意到,中国大陆的几分的重要的文件,这些的文件都不是,一般部会所能够拟定的,那我们称之为超部长级的文件,那这个到底代表了什么样的讯息?



最后,如果我们时间还够的时候,我们再来看一下,就是大不列颠会被变成小不列颠,这是过去,我曾经用开玩笑的口吻就是大不列颠,有没有可能变小不列颠,因为你看到北边的苏格兰,嗯,这个,那个味道现在越来越浓厚。南边的北爱尔兰那个味道越来越浓厚,都想要跟着脱欧之后的英国Say goodbye。但是,现在当苏格兰的首席的部长,其实你就可以把她当做苏格兰的总理啦,那苏格兰的这个担任了八年的首席部长请辞了,这个请辞,有点哀怨,有点壮志未酬身先死的味道,那到底接下去?大不列颠的,会是个怎么样的情况?苏格兰又会是个怎么样的情况?然后这个?有机会,也谈到。好在我们节目开始之之前的时候,因为王孟源已经在线上了哈,那我刚讲了,我希望我们的我们的观众朋友今天的在线上的在线上的来先给王孟源的刷一波,来在我们线上的他,透过越洋视讯连线,他人在美东说跟我的实训,刚好是相反的,所以当我早上轻轻松松的在做节目的,说对王孟源也是有点辛苦的,因为他已经是晚上的九点半钟的,晚上的十点多了哈。在这个时间的做节目,确实是有点辛苦,但在我们线上的王孟源,孟源欢迎。



王孟源 04:24 

很高兴再来跟大家聊聊天



唐湘龙 04:27 

好,来,我们今天从从美国通膨看。看起来是在缓慢地在下降美国的这一波的通膨,这一波的货币危机过了吗?那美国的财政也是货币的另外的一面了,那我们也注意到叶伦,叶伦大概是美国的现在的内阁里面,感觉上面对于急着要和中国谈事,最积极的一个,除了布林肯之外就她了,为什么?



王孟源 04:59 



我们先回顾一下,我在2019年的时候就特别写了一篇博文而且做了一个视频,我预测在未来的2-3年会有一个很严重的世纪级的危机,这个危机会是50年首见的通胀性危机。那这个后来虽然有新冠出现,但是,只是把发生的时间点稍微往后推,推到我所估计的上限,就是基本上是2年半后出现的。这个最危险的时候就是2021年年底到2022年年初,这段时间的时候,美国的通胀是领先全球的所有先进工业国,就是你如果跟他拿他来跟英国或者跟欧盟相比的话他比高出至少3个percent(3个百分点)。但是,到了2022年年终他们就交叉了,到现在是反过来美国的通货膨胀反而比人家低了三四个百分点,这原因很简单。就是我们刚刚一周年的这个俄乌战事。在发动之后,欧盟非常的热情的做了经济性的自杀,切断了跟俄国的能源合作,那么这么一来就基本上等于牺牲了自己的工业前途,那所以立刻就所有的资金还有产业都是外移,那这外移的第一选择就是美国。尤其美国又刚刚通过了一系列的贸易性保护(保护性的贸易法案)。那所以美国的这一轮的通胀危机最危险的时候,就是大概一年之前,然后在欧盟舍身救主之后就慢慢的缓解了,我在我的博客上用一个比喻,就是理工人有一个概念叫做,任何的承力结构——飞机啦或者是土木工程——他的那个所能够承受的最大载重,都有两种,一种叫做动态、一种叫做静态,你如果是静态——就是很慢的话——它可以承受更高的重量,但是如果是这样上下乱跳的话,它这个在更小的重量就可以把它压垮;经济上也是这样子的,你如果是一个非常不稳定非常急性的症状的话,急性的经济危机就会造成恐慌,然后就会有连锁性的、非线性的连锁反应,那这样一来他所造成的恶果就会远远更为严重,整个经济体系的崩溃对急性的问题他的承受力更低。



我现在刚刚讲了说美国熬过了这一轮的通胀,我并不是说这个大事已定,他没有什么问题了,而是他把急性的问题转化成慢性的问题,这个转化的过程靠的是英国跟欧洲日本还有中国的配合,就是他们在去年一整年,这四个主要的工业化地区都非常识相的让自己的货币贬值,那货币贬值的时候就帮助压抑了美国内的通胀势头、就是他们的通胀...当然我想我们有时间嘛,刚开始嘛那我讲的详细一点。

 

王孟源 09:09 



通胀其实要形成一个严重的经济危机的话,就是像50年前的那个危机的话,它有三个要素。最长期的背景因素是你的货币必须超供,就是印钞机印的太多。但这并不代表会产生危机,他只是让你的时机成熟有这个可能。然后,下一步是要有导火线,这个短期的导火线,这短期的导火线通常是我们所说的供给面的问题。像我们这一轮看到的有两个供给面的问题:一个是因为新冠,所以他们很多运输业、还有制造业的劳工的、还有农业的劳工都没有办法好好的生产,当然另一个常见的供给面问题就是能源,50年前也是一样的,50年前的那个能源危机,1973年的能源危机是触发当时滞涨的的导火线。第三个要素是你这个通胀一旦被引发然后又有肥沃的土壤,这个肥沃的土壤就是你印钞印的太多。你必须要从导火线所引发的那一部分经济的那一部分传递到其余的部分,而这个所谓的其余部分最重要的就是劳工的工资。就是必须要因为通胀有了起点,然后因为货币超发所以没有阻力能够让他自然的消停,那这个时候,你还必须要有一个机制让他传递到大部分劳工的工资,然后这个工资开始普遍上涨。那50年前的话这个机制,在美国的这个机制是工会。当时的工会是每年都会去谈判看要调薪上涨多少。最近这几年这几十年,其实美联储的主席一直有恃无恐,他一个原因就是因为美国的工会已经被完全的掏空了,已经没有那个上调工资的那个能力。



王孟源 11:48 



但是我们在这一轮看到了有一个替代性的机制,他这个不是完整的替代,就是他比较弱一点,但是他还是出现,这个出现的是什么?就是过去40年美国的经济从制造业往服务业转型之后,有很多是个体户、个体的服务性的,比如说木匠或者是水电工之类的,或者是一个简单的顾问或者是教师这样子,这些人,我说的教师是补习性的教师。那这些人他们在过去这40年基本上他们的工资是停滞的、没有往上涨,这一次因为新冠导致服务业特别危险,因为你必须要面对面的做服务业嘛,所以就退休了一批,我上我前几个月有讲过,就是目前美国的这个失业率还是很低,然后劳工还是有供不应求的趋势,其实就是因为在新冠的那两年,他的有超过5\%的、最有钱的那5\%劳工提早退休了,那这些人通常就是50岁左右,他们原本是可以再干10年15年的,但他不干了因为他们有储蓄然后觉得为了那么一点钱卖命不值得。那这样一来的话剩下的人就有理由、有空间可以提升他们的工资,那这个工资的提升,我自己估计是大概30\%到50\%,就是过去这三年提升了30\%到50\%,这远远的比美国自己的所谓的CPI统计要高很多。



这点,我想要特别指出的是一年之前我也做过一个视频,特别讨论了一下通货膨胀这个计算很有问题。就是美国在1983年的时候改换了CPI的算法,到了1996年的时候又换了一次算法,每次换都是让新的数值比原本的低,而这个基本原因不完全是要为了哄骗大众,而是因为他的法律,尤其是最重要的是社会安全基金social security,他的一个老年的福利,他的这个是依法律,从90年前开始就是依法规定跟着通货膨胀率自动上涨,所以你的通货膨胀率如果被低估的话,这个社会福利金也会跟着降低,他们既然没有办法去...,要立法修改这个社会福利金的金额是非常困难的,你基本上是政治自杀,但是去玩这种统计的伎俩,把这个数字往下调是很简单的事情,所以他1983年的时候下调了一次,1996年的时候下调了一次,我记得一年前我是估计在过去这3、40年一共压低了平均每年压低是0.4几\%。

 

但是大家要注意平均每年压低0.4几或0.5并不是真的每年就比真实的通货膨胀率低了0.5\%。我刚刚讲过去这40年有很长的一段时间,这些服务业的零工,他的收入没有真正的上涨,就是上涨勉勉强强跟通货膨胀差不多,那也就是一年0.5\%一年1\%这样子,那在这种低通货膨胀率的时候,就是过去这40年有通货膨胀越来越低,一直到了最近20年特别的低,为什么?那个基本的原因是美国在1970年代他的那个滞涨过去之后,他们为我刚刚讲的要了打击工会,就是那个他们了解工会的这个太强大了,是造成通货膨胀为基的三大原因之一,所以为了要釜底抽薪,这才是他们1980年代开始outsourcing、就是向海外送工作,先送到台湾然后来送到大陆去。当时你说那些学术精英或者是战略精英他们懂不懂这个:你把你的制造业往外送有危险。他们懂的,对不对?但是不得不做,因为当时的Volcker、他们的美联储主席为了控制通胀,他把短期利率都已经调到20\%了,真的是已经在绝望,真的是非常绝望。我博客的读者可以到我博客去看,我想两个两三个礼拜前我刚刚引用了Volcker自己在1980年代初所讲的一句话,他讲的就是,为了要挽救我们的国家、为了要压制通胀,必须要把整个经济体系拆掉重来。它的意思就是不得不做outsourcing,就是后来Reagan上台之后就大力的推行outsourcing。这不光是资本家的那边推动的,而是学术精英跟战略精英也同意的,知道不得不这样做,要不然工会的势力太大了他们的通涨永远都是一个达摩克里斯劍悬在头上。

 

那我刚刚讲了这么多的意思就是,因为这个outsourcing,所以,尤其到了最近这30年这个outsourcing的对象变成中国大陆之后,整个潜在的制造业的工资不但没有上涨而且下降

,就是他们不再...,因为这个制造业实际上制造的那过程被送到大陆去了,那大陆的工资比美国要低了好几倍。所以隐性的同样一个电子产品,比如说是一个电风扇什么的,以前的工资可能占整个成本的40\%现在他只占5\%,那这个就带来美国过去40年的通胀压力越来越低,反而过去20年其实最流行的就是deflation通缩,对不对,这个原因就是因为中国提供了在劳工工资上的不断压缩的的反压力。那当这种整个经济环境是通缩的时候,他的这个CPI就是通货膨胀率的统计就不会出太大的问题,因为你原本就只有1\%的话,你没办法把它压的太低嘛对不对,所以我说过去30几年的平均每年少报0.4几的,这只是平均,真正测量错误是当通货膨胀偶尔爆发的时候,就是通货膨胀超过4\%5\%的时候。你通货膨胀,真实的通货膨胀率是1\%的时候你很难说只报告0.5,但是如果真实的通胀膨胀率是10\%的时候,你很容易就说他是8\%。 你懂我的意思吗?就是实际上的误差是出现在通货膨胀率很高的时候,那通货膨胀率很高的时候是什么时候?就是过去这两年。



唐湘龙 20:41 



大部分人,对于对很高通膨的那个灵敏度其实并没有这么高,差个一两个百分点你不会有感觉



王孟源 20:48 



你的误差,其实是百分比了,10\%报成8\%的话,看起来好像是差2\%其实只是4/5吗,对不对。但是你的那个如果真实的通货膨胀率是1\%的话,你要报成0.5\%就不容易了,因为人家就觉得,哎你这样子少报了一半。我说了这么多,除了说我们这些住美国的人他的那个社会保险金被骗走了一些之外,其实有很严重的影响。我跟大家提醒一下,就是GDP这个东西,我们现在日常所说的GDP,比如说现在是年初嘛,所以去年的GDP成长率刚刚出来,他的报导是美国是2.1、日本是1.4、英国是4.0,但是我想提醒大家这些所谓的headline number就是头条新闻上的数字,其实都是经过通货膨胀率调整过的。就是你根据名义上的GDP生产从1万亿美元涨到11,000亿的话,这并看起来好像是10percent的成长,但是你要把通货成长率扣掉,如果通货成长率是5\%的话,扣掉之后就只剩下5\%的成长了。所以你想想看美国的通货膨胀率去年其实应该是9\%10\%,但他只报了7\%8\%,差了1\%点几,那这个很自动的就让他的GDP成长率提高了1\%点多



唐湘龙 22:47 

就GDP被灌水了



王孟源 22:49 



被灌水,



唐湘龙 22:51 



对,被刻意压低的通膨数字给灌水



王孟源 22:54 



如果不相信我这么说,我对美国的这个2.1\%这个数字这样的评论,大家想想去年的这个经济环境日本凭什么成长1.4\%。日本过去40年的的成长基本上是负值,30几年的,基本上负值,去年有那么好吗?他一下子成长了1.4\%。英国更惨,英国现在的那个通货膨胀率还是二位数字的他居然能够成长4\%,哪有这个道理。



这很明显的是因为他们把CPI就是那个通货膨胀率给误算了,少算了一大堆。 我想美国至少少算了1\%、日本可能也少算了1\%、英国这至少少算了2\%,可能是3\%。就是这些国家其实

日本没有成长1\%,那英国跟美国顶多就是成长1\%。所以我先跟大家讲一下,就是当前的这个经济态势其实一点都不乐观

 

完全你唯一能说的正面的一个大好处就是他原本一年之前有急性的问题,现在已经转化成为慢性的问题,就是不再有我刚刚说的那种dynamic load(编注:动态承重) 的把那个整个成立结构完全打垮的那个危险,就是没有立即的危险。



唐湘龙 24:35 



那如果你说这是变成是一个慢性的危机,而你认为现在的经济的情况没有乐观的理由,许多的数字是被美化的,那是被包装过的大家在饮鸩止渴,那这个危机接下去会用着怎么样的方式呈现?



王孟源 24:57 



但是你如果去看美国的股市的话,今年这将近两个月他的那个股市又上涨了16\%,为什么?基本的原因是美联储在2019年年底转化成Qe之后,从QT转化成Qe之后,就是从量化紧缩转化成量化宽松之后,一直到2022年初两年半的时间就印了5万亿美元。那这5万亿美元基本上就进入三个领域,除了当然也有到国外的,但是到国外的这些已经开始回收了,回收了一年了,暂时先不管那方面,美国内有三个领域,他们一般都是称为所谓的Household、business跟finance。Household就是一般的家庭的私人储蓄,business就是企业界的自己的现金储蓄,然后金融是金融机构的手上的现金。那这5万亿虽然有一部分外泄到国外,但是在过去这一年

因为美元上涨的关系,他又回流了,回流了所以你事实上是这5万亿美元还是在美国。那我想我在半年前,也许是三四个月前在博客有一个讨论就是我说这个5万亿不可能一年就把它消耗掉的,现金太多了。所以你现在股市这么坚挺,事实上就是因为金融体系还有还有那些投资人身上的现金还是过多



唐湘龙 26:48 



烂头寸很多了,现在到处都是烂头寸



王孟源 26:51 



对。那过去这一年他,我说过刚刚说到了他一年之前从Qe又转回成QT,就是从量化宽松又转化成量化紧缩,这一年来一共回收了不到0.5个万亿、就是5,000亿,这个5,000亿是不够的,你回收5,000亿是远远不够,因为你放了5万亿嘛,对不对。然后我刚刚说的Household、business跟finance他们浪费的地方浪费损耗也没有这么快。所以目前为止这三个管道都还是有很多剩余的现金,尤其是大银行,就是因为2008年是大银行率先倒闭,所以他们现在剩下来...这次非常的谨慎,基本上反而越大的银行他的那个财务状况就越健康那你如果去看美联储现在他有很多交易窗口,基本上大家不是跟美联储要现金,而是现金过多往美联储那边塞,这个其实过去一整年都是这样子。往美联储手里塞现金的管道有两个:一个是所谓的excess reserve就是额外的保证金,这是银行界必须要按法规留在中央银行手里的叫做保证金,但是因为现金过多所以几年前美联储说你们可以放额外的保证金,那这个额外保证金有什么好处?就是美联储付蛮好蛮高的利息,你完全没有风险但是还一样可以拿很好的利息,那大银行为了准备,所以就有3万亿的保证金在那里,其中有1万多亿是额外保证金,到目前为止仍然是这样子,那这个至少还要一年才能够慢慢的消耗掉;另外还有一个窗口也是过多现金往里面塞的,不过这个叫做reverse repo反向回购。我想着讲的是很专业,不过我之所以特别提这个,是怕有那种半专业的股民听我在这边讲,我如果不解释一下的话他们会说哎那你怎么不提这个。这个reverse repo也有额外的1万亿的现金,但是这1万亿其实并不是真正属于金融管道的,而是所谓的money market fund,这个最喜欢用reverse repo,这11,000亿基本上都是所谓的money market fund,那这个money market其实就是大家的短期存款,一般老百姓的短期存款集合起来,然后由那个专业的金融机构设法产生利息。那所以事实上这1万多亿他所体现的就是我刚刚说的所谓Household的额外现金、家庭的额外现金,那另外还有那个企业的。就是这三个大致都还有1万亿在里面,所以你至少还要需要一年,可能到一年半。



王孟源 30:36 





那我讲这么多的意思就是说,你这样子还需要1年左右这些塞满着现金储蓄的管道才会慢慢消耗掉,与此同时美联储本身还在继续做量化紧缩,这个量化紧缩大约是一年0.5个万亿就是5,000亿,会不会马上今年出危险了?大概是不会。就是你如果只有这些很自然的金融经济性跟货币政策的事情来考虑的话,似乎是要等到最早是今年年底,更可能是明年上半年了才会出问题。



王孟源 31:33 





但是刚好上个礼拜有一个读者在我的博客发问,问我有关就是,也是你有你问到的就是这个美国的财政部长想要来访问中国,然后他就问说跟美国的国债有没有什么关系,我说没有关系。美国的国债对我刚刚所讲的这些事情有一个修正性的影响,那我在这里解释一下。美国的国债本身永远都不会跳票,因为他是用美元定价的,大不了就是叫美联储再多印一些钞票。所以,美国国债唯一的影响就是影响美联储量化宽松跟量化紧缩的速度。他们现在刚好又到了国债上限了,大概是今本月初的时候几个礼拜之前,他们开始不能够再发行新国债了,然后你如果看既往的历史的话,一般是要六七个月来解决这种这种事情,就是因为他有其他的会计手段可以假装说,可以从石头里拧出水来,就是拧出现金来让联邦政府继续花钱,所以要等到最后一分钟他们的这个政治闹剧才会结束,然后反对党也就是共和党才会同意再提高债务。上面那大家算一算这是什么意思,这个意思就是可能要到今年8月或者9月他们才能够重发国债,那重发国债的时候一次发多少?美国一年的赤字是1万多亿,那他有半年没有发国债这个必须要马上弥补,然后又有接下来未来两三个月的花费必须要赶快的筹措,所以他一次就会印大约1万亿,所以到今年八九月美国这个国会的作秀做完之后,他会一次发行1万亿国债。



这个对我刚刚讲了半天的现金的问题、美国金融管道的现金的问题是什么?就是它的效应其实是等同量化紧缩的,因为它就像一个吸尘器一样把那个现金从金融管道里面吸走了,只不过是美联储的量化仅说是吸到他美联储的账上,那美联储基本上是一个美元的黑洞嘛,那个进去以后就不见了。但是联邦政府把这个吸过来以后,他也是把它浪费浪费花掉了。所以对债市的...我这里的债市是指广义的债市不是指国债,用英文讲是所谓的credit market的影响是跟量化紧缩相同的。所以他的效应就是到今年年底下半年,会一下的让原本今年量化紧缩0.5万亿变成紧缩1.5万亿,而且这个是急性的,我刚刚特别讲掉你这个经济的这种危机或者是不利的态势他是急性跟慢性有很大的差别到,他是一个很急性的,在一两个月之内就会消灭1万亿的现金,那这就有可能在今年下半年第四季的时候造成现部分的现金短缺。一个更大的危险是这个时间点对美国来说非常的糟糕,为什么?因为第15届金砖会议是今年8月要开,而且我预期是今年8月在南非开会的时候沙乌地阿拉伯、就是大陆说是沙特Saudi他会加入金砖,我自己个人预期或者说希望是与其同时会宣布由建立金砖货币,然后Saudi会转为用金砖货币来做石油定价。而且,这些能源国还有这些第三世界国家都会很快的大幅的把他们的储备、他们的外币储备从美元向金砖货币转换





唐湘龙 36:50 



好在我我过去看过的资料,美国的实体发行的货币就是我们可以抓在手上的绿色的那张Greenback,那Greenback大概是6兆多美金,但是在账面上面,美国的总体金融体系里面的财富加上着乘数效应大概是二十一二兆美金,跟他的一年的总体的GDP的产值差不多,但他债务现在大概在已经在三十一二兆美美金。在当他现在的货币现在的困境,你刚提到就是说叶伦急着要跟中国对话跟国债没有什么关系。但是中国或者日本美国的主要的国债的持有的这些的国家最近大概方向比较一致都在抛售美债。这个抛售短时间,第一个就是中国有政治企图吗?第二个就是他会对美国发生什么影响吗?



王孟源 37:51 



我是希望这个金砖货币还有所有的外汇储备的政策是一个协同性的战略性的去美元化。因为美国在一年之前成功度过那个通胀危机之后问题转为慢性,那你要再重新创立一个急性的问题,我认为在未来的三五年,这个是最好的机会。然后我其实特别建议说提那三个新的文件,用意也就在这里,你开场白的时候提到的那三个超部长级的文件是我建议的嘛,我之所以建议是我认为它代表着中国外交大战略的一个成熟、一个转变、一个坚定化,而这个刚好就是在20大之后新旧交替,现在就是刚好在新旧交替,对不对?



唐湘龙 39:08 



这是今天的重点。因为孟源在跟我提这件事情之后,我觉得我觉得孟源启发了我很多的概念了,就说这些的文件摆在那里的时候看起来都是一些白纸黑字、死的东西可是孟源在告诉我说这个不太寻常,他背后应该有一整套的思考我觉得有道理。那我们刚刚讲的其实孟源刚刚讲的就美国的美国的金融、美国的货币,美国的联总会现在仍然形同了全球央行的时候,其实他的债务问题是不会困扰他的,真正会影响到美国、会打击到美国的金融面的大概只有一个就是去美元化能落实到怎么样的程度。美元的全球的市占率、交易当中的市占率,这才是真正会对美国造成重伤害的,所以我相信美国在乎的这件事情。好那我们回头来看这三份文件,这三份文件你为什么会觉得它非比寻常?





王孟源 40:10 



好在大概一年前就是他们王毅被提拔之前,他的前任,杨洁篪在Anchorage跟Blinken会议的时候第一次说出事实,直话直说,他说:“我们把你们想的太好了”。那是然后在那之后,中国的外交部发言人,目前的发言人叫汪文斌吗,他慢慢的就变成他也日常性的直话直说,批评美国的那些骗人的傻话。就是美国过去这二三十年这个他们的政治精英、政坛的权贵的精英的水准下降的太快,真的是像我这种来美国30多年的人真的是很感慨。但是……举个例子,像昨天我看到的——因为我知道今天要来讲汪文斌的那个发言的典型,所以我就简单的去看昨天他说了什么——他说的是布林肯口出狂言颠倒黑白。你说这种非常直白的话,你如果去看我的博客的话,那就在五六年前,我想整个华语界也就只有我一个人,是这样来评论美国人的。但是,当时2017年2018年贸易战开始的时候,我说要对等反击,那个时候连我的老读者都吓了一跳,大部分都不以为然,都认为我太疯狂激进了。不是!这是完全理性的事情,人家是要准备谋财害命的时候,没有跟他谈判的余地,没有跟……。



王孟源 42:27 



Well,但是这种会议发言,还有发言人做反驳这个,他现在固然变成日常,而且变得非常尖锐了,但是他并没有我要提的那三份文件的那么重要。为什么那三份文件特别重要?因为他们是政策性的,而且显示了战略决定,很明显的是反映的是战略决定。其中最重要的是第一篇,这一篇是3天前发表的叫做,美国霸权霸道霸凌及其危害,是一个外交部的专门报告,这个报告是他们花了几个月,我相信是把我博客的老内容,还有后来被转述的别人写的文章,通通吸收了一遍,(唐:向全世界控诉美国)。对,但是,这是以外交部的名义花了半年左右——看起来是花了半年左右——集思广益写的一个东西。这是非常非常非常严重的,就是我已经不再认定美国是一个可以打交道的正常国家。你是一个霸道的霸凌者,你面对霸凌者跟面对正常国家,是有完全不同的战略选择的。



王孟源 42:27 



然后接下来就是王毅提出了一个全球安全倡议,这里面他的核心就是我,上上个月,上上上个月,十二月的时候,我跟你讲的那个联合国宪章的那个,所有国家有基本主权的那个概念,对不对?有些基本的权利,是所有的国家都平等的,那这些权利就是所谓的主权,那也是联合国宪章的意义。那他的这个,你如果去看他的王毅发表这篇全球安全倡议,他强调的就是你不能够再让美国这种,我说了算、我高兴了就可以随时改变规则这样,唯一的规则就是规则我来定。他的这个安全倡议,洋洋洒洒,他特别去谈联合国宪章好几次。他的意义就在于,你这个规则必须要平等,必须要与联合国宪章的基本精神,就是我那时候讲的威斯特伐利亚的精神。



然后,更可怕的是那个我说的第三份文件其实不是文件,而是中国的驻联合国代表在昨天特别

发言,说不能够让北溪的破坏者为所欲为,必须要追究。那你想想看这是这已经是摆明了,我虽然没有指名道姓,但是我就是摆明了不让你,英文的所谓get away with不能够,就不让你逃。不让你从这些事上面有脱身的机会。



对,那与此同时王毅现在人在莫斯科嘛,刚刚跟跟putin会见,面对面。大家都预期今年春天,习近平本人会到莫斯科去跟Putin见面。我想有一些亲俄的欧洲评论员,在过去两天开始猜测,是不是中国要参战,或者是正式的提供军工。



唐湘龙 46:40 



没有错,你觉得会吗?



王孟源 46:44 



我觉得他们是想的太多了。事实上,第一是不需要。俄国的军工,已经进入全面动员。



唐湘龙 46:50 



你在你你在半个月前特别讲过,俄国的军工实力,是远超过我们的评估的



王孟源 46:59 



对,所以完全没有必要,第二个,这也不是中国传统的的方式。 那第三个,是你真正有效的打击点,我在过去博客写了9年,我从第一年就开始讲到现在。而且我刚刚讲贸易战刚开始,四五年前贸易战,5年前贸易战刚开始的时候,我就讲要对等反击,对等反击不是你贸易禁运或者什么或者是收关税,我也对等地收关税。不是!对等是找对方的痛点,用对等强度去打击。这痛点是什么?就是美元嘛!从博客一开始9年前就解写到现在,你要真正消灭美国到处霸凌的能力,他的那个基础要釜底抽薪的话,就是打击美元,美元是他的力量的权源也是他的软肋。



所以,我真的认为习近平如果到莫斯科去谈的话,主要谈的就是这个金砖扩大,在今年8月扩大之后,调整新的国际架构的一些细节。就是基本原则,就是原则虽然大家都知道,但是你必须要正式的会面,然后把它用白纸黑字写下来,但这白纸黑字他可能不会公布了。但是他们会面,是必须要把原则用精确的文字把它敲定下来。这个,我觉得这样是很好的,因为虽然说中共二十大之后的新旧交替,现在正在进行,但是这里面最重要的,就是把以前的那些金融体系,人民银行的人都换掉了。但是,换掉以后新上来的人我还没有信心,因为还没有看过他们是怎么处事的,我只知道换掉的那些人是早就该走了。但是,俄国的中央银行的行长Nabiullina,我是有信心的因为我观察她观察了10年了,所以如果是中俄这样紧密合作的话,我觉得在金砖货币这件事上,不会搞砸,细节上不会搞差,不会搞错。就是因此,这是为什么,我今天又想要谈今年年底的美国的货币财政问题,然后又要谈中共的态度的改变,外交战略态度的改变。因为这两件事情我觉得都是共通的:那个他们的焦点就是今年8月的那个金砖会议。



唐湘龙 49:56 



好,如果……当然,就像刚孟源提到的,最近这一两个礼拜西方的媒体对于习近平——因为王毅已经到了到了莫斯科嘛,而且王毅到莫斯科看也看得出来,就是普丁也好,莫斯科对他呢是非常温暖的接待。那不管是坐的位置也好,然后普丁的亲自的接见,那些的肢体语言,那不像是表演,就是中俄之间的互动,他们有共同的就是说战略利益,好,那这个战略利益呢,使得大家唇齿相依也好,有规划共同的未来也好,总而言之中俄的关系,现在看起来真的是坚若磐石——那习近平呢,如果说接下去四五月的到访,那美国也好,西方的媒体不断的在下针,下药,觉得这个时候怎么怎么还有人敢去莫斯科访问呢?你到底想到莫斯科干什么呢?



但除了金砖之外,这次的访问,会在对俄乌战争产生怎么样的影响吗?因为今天是俄乌战争的一周年,那今天我们两个人在谈这件事情的时候,好像一周年,没有总结讲点话也不对。





王孟源 51:07 



好吧,那就总结两三句好了。我上个月已经总结过了,我上个月总结其实只不过是,重新的重复我在11月讲过的东西。我不是真正的普通的媒体名嘴,我不是事后诸葛亮然后蹭热门出来凑热闹这种人。你要谈一个议题,有意义的谈话的话呢,必须是预测嘛,那如果预测完全精确的话,你不需要未来的4个月5个月的也每次都重复。



大家有兴趣的话到我的博客去看,就是我4个月5个月前讲的,完全都还是有效。就是基本上,俄国原本就立在不败之地,就是他们原本真正害怕的是在金融跟经济贸易上被打垮,这也是原本美国的计划。但是呢开战一个半月之后,到3月底4月初呢,就已经很明显的是俄方在这里大获全胜,而且下一个战线,就是外交战线也已经大获全胜。所以呢,在军事上Putin始终是有恃无恐,因为顶多就是退个几百公里嘛,对不对?你因为以1打2,对方的你的军力是1:2的时候,如果是真的没办法,你就退个几百公里,然后再等我这些一切都准备好,内部国内国外等民心士气都准备好的时候,我再征兵。他现在征兵征得差不多完了,他随时都可以投入那些部队。现在投入的部队还不到1/2对不对,可以投入的部队是50万,现在投入的才只有20万,你随时还有30万可以。而且呢,现在的战线只有800公里,他另外还有2,000多公里的战线可以随时启动。这,他这是胜券在握,根本一点意义都没有,我现在都已经懒得跟大家谈谈这些。其实这完全是在猫玩老鼠的态势,就是完全就是等着,跟他玩一玩,好,你们北约要送什么武新武器了,你们再炒一炒,好,再送新武器来,然后我看看怎么应对,然后把他打掉,然后你们再送下一波。反正北约现在的军工这么实力这么弱。俄国真正的对手是北约嘛,尤其是在波兰还有波罗的海三小国,还有芬兰那个阵线,那在未来5-10年那是真正的潜在问题所在。那你现在先把他们的军工消耗掉,把他们的财政跟那经济给打好好地打击一顿,对未来5-10年都是有好处的。所以,那他们必须这样偷偷摸摸地转送到乌克兰,那从波兰跟乌克兰的边境送到前线1,000多公里,然后一路上被俄国这样子高兴打就就打,对他们来说非常好的态势,对不对?你高兴送武器你就送嘛,反正你能够送到前线的只有1/3,2/3我帮你打掉了。这比你收在你自己的库房里面,到时候真的北约跟俄国正式开战的时候,可以一起打过来,这样好的好很多嘛。所以,我跟大家说,这种事情没有什么太大的意义,就是胜负已定,基本上只是时间问题





唐湘龙 55:05 

唯一大家只要想一个画面,就是说如果今年习近平因为3月份开两会嘛,大概两会期间呢,就是说

中国领导人出访的机会大概比较小,所以可能3月底4月,就春天春暖花开的时候呢,习近平会不会呢到莫斯科跟普丁见面。重点刚刚我们提到的两件事情都是很重要,不过那个是时间点,以现在大家对俄乌战场的分析,今年的就俄乌的一个大决战,会不会呢在在春季的时候展开,很有可能。如果以时间点来看很有可能呢,就是呢习近平呢在访莫斯科的那个前后,正是俄乌在乌克兰战场上面的,俄乌的大决战的时候。好,那个画面大家自己先脑补一下,到时候呢我们再来讨论。



那王孟源准备的菜呢,我都不想浪费。来,我们回头来关注一下呢苏格兰这个议题,是今年一定要面对的议题,那本来今年10月的时候,苏格兰是准备要再办第二次的苏格兰独立公投的,但是现在发生的事情呢,就是英国的法院看起来是做了一个非常政治性的处理,就是不接受就是苏格兰在没有经过伦敦同意的情况之下,再办第二次的独立公投,哇,看了一下那个解释,我怎么看都觉得就是一个政治。那现在就说苏格兰的这个首席部长,就苏格兰的这个女总理做了8年了,看起来在苏格兰的声望还不错,可以做8年,她请辞了,请辞了之后那接下去的苏格兰,大不列颠会变成小不列颠,这件事情的危机是不是暂时解除了





王孟源 56:48 



我觉得,现在我看到欧洲的一些评论员,他们是把Sturgeon下台,把Sturgeon拿来跟纽西兰刚刚下台的那个女总理相比,就是Jacinda Ardern,那你表面上看来她们都是女强人,而且也都曾经就是短短两三年前,民意还支持很高的,然后现在一下子黯然下台。表面上看起来有类似,但实际上不是这样。你这Jacinda是百分之百的主流的,她是所谓的温和派,她是百分百主流的。但是Sturgeon呢,不是主流,她是所谓的那种假叛逆,我解释一下这是什么意思。

唐湘龙 57:46 



这个你解释一下



王孟源 57:51 



我解释一下,就是欧美在过去这20年呢,他们所谓的主流呢就是资本或者deepstate能够完全掌控的,你不管是所谓的温和派中间派左派右派,只要是主流的派系,都是资本掌控得牢牢的。凡是资本不能够掌控的,就叫做极右或者极左,你如果是偏向于小企业、小生意人的,那个就叫做极右;你如果偏向劳工的就叫做极左。那偶尔在某些地方呢,他们有搞民族独立——那你知道英国跟美国,最喜欢全世界搞颜色革命的手段之一,就是去搞所谓民族自决或独立,(唐:对没错)但是呢,如果是发生在欧洲自己,尤其是英国自己的话,大逆不道!

唐湘龙 58:51 



那就没有机会了,就霸凌他,抓他,通缉他



王孟源 58:58 



对,所以呢另一个会被视为叛逆的手段,就是你搞这些民族自决的。所以,Sturgeon,她本身是搞苏格兰独立出来的。但是,除了这一点之外,你如果去看她的政见,是典型的白左,就是主流,比主流还要主流。就是,她对比如说同性恋呐这些这些议题的支持呢,比英国伦敦那些人还要强还要强烈。

王孟源 59:43 



那有另外一个女性的政客——欧洲的政客——也是这样子的,这就是刚刚上任的意大利总理Meloni。Meloni除了对移民反对以外——就是北非来的移民,因为种族的关系非常反对之外——其他也都是完全遵循主流的,她刚刚上个礼拜才刚刚到基辅去跟Zelenski握手。基本上,凡是白左或者是deepstate建制派说什么,他们都同意,但只有一个议题,单一的议题,他们是是反对的。那所以我觉得Sturgeon跟Meloni是属于同一类,就是他们基本受到美国价值观,白左这个价值观的完全洗脑,只在单一议题上不同意。那他们就不是,虽然他们不是deepstate,这个建制派的成员之一呢,但是他们是建制派可以容忍的,对不对?就是你反正就只有一个单一议题,我只要让这个在单一议题上阻止你,很容易嘛。那其他的事情你都同意,跟我同意,那没有什么问题。这个跟美国现在的这个佛里达州长,DeSantis,他现在是民粹派的宠儿,这个就不一样,有本质上的不同。DeSantis是是建制派送去民粹派卧底的,就是为了要阻止Trump在2024年在共和党的初选胜出,所以预先埋的一个棋子。我为什么这么说呢?因为,他现在不是只有单一议题,而是所有的议题都是哗众取宠的民粹,右派民粹,就是去跟Trump竞争。但是呢,你只要回头看10年前他还在当国会议员的时候,他讲的话完全相反,就是他当了州长以后,就这过去这三四年才180度转弯——不是360度——180度转弯,所以这很明显的他是,他原本就是建制派派的成员而是然后被派来卧底,来挖Trump的民意基础。



王孟源 01:02:18 



那所以我刚刚讲Jacinda Ardern是主流派,是她就是建制派的真正成员;这个DeSantis也是建制派的真正成员,但是被派去卧底;然后,Sturgeon跟Maloney,是只在单一议题上跟建制派有对抗,但是呢可以被(抵抗)。那她这一次下台呢,其实也是苏格兰国家党的一个传统,因为你看看2014年的时候Sturgeon,之所以上台,就是因为她的前任Salmond因为上一次的苏格兰独立公投失败以后,他辞职负责。

 

但是我要特提到Salmond,我要特别提一下Salmond是真正的反建制派,就是所有的,基本上所有的议题他都跟白左的价值观格格不入,这是为什么在2018年一直闹到2021年呢,有三年的时间他被刑事追溯,说他有性骚扰。这个是建制派处理反对派政客的一个典型手段,法国也曾经有一个有一个政客在纽约开会的时候,就莫名其妙地被控性骚扰强奸嘛,然后他就不用选总统了。那这所以你可以看出这个Salmond在四五年前呢,他们还很……建制派还为了担心他复出,而特别的去用性骚扰这个罪名去追溯他。那就是这个意思,就是他们可以接受Sturgeon,但是他们不能够接受Salmond。那现在Sturgeon这个下台其实也很简单,就是你刚刚讲到的,英国的最高法庭在去年11月的时候呢,用一个非常政治性的判决呢,挡住了下一次公投的路。那Sturgeon她没有一个很好的解决方案,所以她就只好下台,因为整个所有的苏格兰国家党,他党成立的意义,基本意义就在于要搞苏格兰独立嘛。



王孟源 01:04:51 





那是不是是不是英国这样就天下太平了呢?只能说他们盎萨的人手段很辣,就是你这种直接让最高法院睁着眼睛说瞎话,这种事情他很简单就可以做出来。那做出来以后呢,的确是没有很好的方案,那现在的……不过并不是绝望,苏格兰独立并不是绝望,而是呢,我刚刚提到,当前

通胀压力仍然最高的,仍然是在两位数字的,就是英国。那所以苏格兰要独立并不是不可能

,在未来两三年要独立并不是不可能。但是呢,必须要英国的经济继续地崩溃,那这个我们就必须要看,八月金砖会议之后,对美国的打击有多大。因为呢,毕竟我们从去年的经济走势就可以看出,美国打喷嚏呢,重感冒的人是谁呢,就是英国。因为为什么?因为英国的产业空心化比美国还要严重。

 

王孟源 01:06:17 



我以后有机会,跟大家讲一讲这个历史:就是1956年,艾森豪用第二次苏伊士危机,苏伊士危机彻底打败了英国维持殖民地的一个企图,就是那时候英国跟法国联合起来出兵苏伊士运河嘛。那艾森豪命令那个,IMF国际货币基金会跟英国要债。那英国还不出钱来,所以只好乖乖的撤兵。那之后呢,英国才真正的心服口服,不再试图当世界第一/世界强权,而变成美国的走狗。但是呢,这样一来,当时你一旦放弃政治跟军事的权利呢,就对金融跟经济有很大的影响。所以在1956年的时候英国的富豪就有一个panic,有一个非常紧张,那紧张的结果就是他们建立了现代的——在那之后的两三年——他们建立了现代的所谓Tax heavan——就是逃税天堂,OK。我们为什么现在的逃税天堂,所谓说出来的,十个里面倒有八个是以前的英国殖民地?就是因为——包括香港跟新加坡在内,全世界最大的逃税天堂之就是香港——那就是五零年代后期,英国人建立了现代的这个国际逃税体制。



然后呢到了1960年代末期,十年之后呢,美国人因为在越战的时候花钱太多,他们注意到英国所建立的这个体系,这个体系里面包含的一个叫做所谓的Eurodollar,名义上是Eurodollar其实是英国人自己搞的。那他说这dollar是我们的钱,结果被你们英国人拿去炒金融,赚黑钱去了。所以呢他们那个时候在1960年代末起呢,美国的大银行像JP摩根这些呢,通通也都加入。所以我想四五年前我曾经在一个视频讲过,说美国本身也搞这些逃税的,但是只要不是美国纳税人,美国人就会帮你逃税:就是你逃别人国家的税,我不管。所以这是1960年代后期到1970年代早期那10年了,美国去跟英国竞争。所以现在这个逃税的这行生意呢,大家都想的是瑞士,其实不是!瑞士只占很小的一部分,是所谓的银行秘密账户这种这个行业,其实真正搞的厉害的是所谓的trust,信托基金。那这个盎萨系的,用的是所谓的信托基金这条路,但你都听不到。为什么呢?因为盎萨控制这个国际媒体,所以他们要讲这些逃税天堂,他们会故意去讲瑞士,其实瑞士只在很小很小的一部分。



王孟源 01:10:07 



那我讲这么多是干什么呢?这个就是,因为英国跟美国在这些事情上高度重叠,那这些事情

他的基础是什么?是你的主权嘛。就是你的能够罩得住,别人要来查税的时候,你可以说这是我的领地,你不能够来插手,对不对?那这是,如果美国出了问题的时候呢,美国会通过所有的通道吸收资金跟资源来挽救自己。那这里面跟美国连得最紧密的就是英国,尤其是这些黑钱的金融通道。所以第一个受损的,受损最大就是英国。所以我们要看今年下半年呢,这个美国的金融体系在受到一些震荡的时候呢,不一定能够造成大规模的金融危机,但是呢,反而是英国的危险更大。那如果英国出了真正的问题的时候呢,苏格兰就很可能咸鱼翻身了,我们要看下一任是谁。







唐湘龙 01:11:24 

好,因为时间的关系,好,所以受益良多。今天,今天跟孟源大概只能聊到这地方,我简单刚有关于苏格兰的问题。苏格兰的独立它牵动到一个欧陆的因素,就是说当欧盟越强大的时候,我认为苏格兰想要寻求独立,以一个国家实体加入欧盟的那个念想就会越强烈,那这个中间是联动,尤其在英国脱欧了之后。可是现在的欧盟是在一个空前虚弱的状态,因为俄乌战争的关系,俄乌战争把北越给打高了,美国再度成为欧洲的共主,欧盟被边缘化弱化了,欧盟的功能正在大幅的萎缩。这或许跟英国为什么强烈的去支持就说乌克兰,因为乌克兰打越久,欧盟就越虚弱的情况就是在下面它就断了苏格兰的想要独立的这个念想,因为欧盟不在这有那么大的吸引力,这在舆论的发动上面来讲伤害是非常大的。好,这个,这个是我的逻辑了,我们要敢跟。



王孟源 01:12:38 

我不完全同意。唐(好,你请说)。第一个,苏格兰想要独立并不是主要,并不是想要加入欧盟,而是想要脱离英国,因为他们认为他们被英国人骗了,就因为事实上英国从 12 世纪一直打到 17 世纪, 18 世纪才成功的吸收了苏格兰,对不对?所以他有很长几百年的时间是世仇。所以你如果去看他们的基础教育的话,苏格兰的小孩子都会学那段历史,都知道他们打了几百年的战争,那,嗯,所以所谓的脱离英联邦去加入欧盟是一个借口。OK,真正的动力是要跟英国一刀两断,但这段实际上另外一个要考虑的因素是你实际上它的经贸也是跟英国结合得,远远比欧盟更紧密,所以你如果真正要独立的话,一个合理的方案就是名义上独立,但是你仍然跟英国有关税同盟,有自由贸易,是完全的自由贸易。所以欧盟这个因素我觉得不要过考虑。



唐湘龙 01:14:09 

好。当然毕竟以今天的英国的大英国大不列颠的总体的国力来讲,其他的国力除了金融以外,大部分都表现在表现在苏格兰身上,包括他的资源,他的许多的身材,那个苏格兰的影响是很大的。好,这个是英国接下去即使暂时挡住了,就像是西班牙挡住了,就说加泰隆隆尼亚的独立一样,那个都是用很丑陋的方式去打压,那也都是千年的几百年以上的世仇的关系,可是碰到国家主角的问题,你就发现这些平常强调民族自觉、著名自觉的这些老民主国家的那个嘴脸就出来了。



唐湘龙 01:14:51 

好了,我们来感谢我们的观众朋友,对了,因为我,你知道我得我两个礼拜的时间没出现,所以许多的观众朋友我接收到你们的想念了。来雷欧,感谢他龙哥每周只要上一天班。那只要访问王梦媛就好了是吗?哈哈哈,OK,可以啊。好,这个莱克利来一个感谢。好,那谢谢你,沙龙终于回来了。对的,好,来丽萨张是吗?好,那感谢。另外叶竹辰,感谢好王,欢迎回来。希望希望龙香龙的一些安好。对,都很好,谢谢大家。细节的部分我就不多谈。 w k 吗?谢谢谢,他说大陆开放之后我去大陆两次了,比一期前进步了很多,经济向好,好来坡论感谢我们这是老听众带来地球,感谢,加油。那再来杰瑞利,感谢唐渊回来主持王博士的特别节目,赞好,谢谢再来破例。好,感谢您破费了哈。那路易斯良,谢谢。在老金融在微博下有看到预告,所以就过来支持一下。微博有预告真的好,还有好久不见的这个在项目效应阶段安好。对,好了, 90 Day。谢谢。



唐湘龙 01:16:10 

那唐杨龙终于对接他的欢迎我回来是吧?好,陆配里好,再来。龙哥变得好憔悴。真的很明显。好好好,对,我也,我也同意了。好,但是这个说来话长,以后再说吧。杰森王来,谢谢。然后 go style 香龙什么香龙,什么之冷起来好吗?好,再来这个的乌艾瑞斯给王博士打Call

。

唐湘龙 01:16:46 

再来 l g l j n g。好,这在香港感谢洋葱,谢谢,那欢迎回来,他说洋葱说后面此处省略1万字的肉马的话。好,我收到了。好,再来,依然雅雅去,谢谢你,陆佩礼仪再一次的谢谢谢,然后再次喊这个龙哥加油,还是这龙粉都很关心你看到如此憔悴都很心疼。好好好,我会很好很快的恢复了元气来红书,感谢然后的那 straight SG。好,来,欢迎再来皮特 down 邓吧。好,感谢了。



感谢湘龙和王博士的天花板级别的访谈。那是王梦远,你看我每次在访王梦远的时候,你们发现我都听得很专心。好,美国的货币危机的恶化程度跟俄乌战争的激烈程度,会成为那大陆决策层决定解决台湾问题的重要因素吗?孟源,你可以回答一下吗?这个问题。



王孟源 01:17:49 

我不方便讨论。



唐湘龙 01:17:52 

好好好好的,这这好,我懂你意思。好了,这个 potato head,然后昨天看到看到这个Gonzalo Lira。那发视频呢?谈这篇美国霸权的危害,就猜到王博士今天可能会谈这件事情。好,那这个想请问王博士怎么看这个Gonzalo Lira的说的,这是中国对美国的宣战,你同意吗?



王孟源 01:18:18 

我就,我不同意。嗯,我觉得这个……中国你可以说,这是因为中国已经在战略上知道。不得不采取敌对态势,但是这并不代表中国会去学美国去玩那些下流手段,OK,就是,基本上因为中国可以也应该从替代美元着手,根本不需要跟他们在这些这些小事情上面执着。嗯,对不对?



唐湘龙 01:18:52 

好,但我们现在理解大陆的一些的言语跟他的对外作为的时候,我们还是非常习惯用美系的思考跟美系的概念,所以我觉得那个在概念上面来讲可能都会出问题,这是未来在大国政治谈论的时候,大家可能都需要建立起来的另外的一套的语言跟概念的系统,否则。



王孟源 01:19:15 

因为那篇文章,那篇所谓的全球全安全倡议,其实就是我们中国要走王道。他的内容总结起来就是:我们知道美国是混蛋,我保证中国不是那样的混蛋,请大家信任我们。



唐湘龙 01:19:32 

对对对,好,这就是王霸之辩了。好,来,王荣林,谢谢,还有 y a Yar 令他说他是海外华人,每一次听王博士的这个讲演都觉得胜读 10 年书。对的,好,米其林,谢谢王荣林,再一次的谢谢。然后 potato head 他说逃避税、私人银行这种东西是老王跟史东在谈香港问题时候所提到的,我还记得提了美国的,在Julius Baer的是个比较好的选择。



唐湘龙 01:20:06 

好,那感谢我们今天的所有的观众,随着王孟源的课程,随着唐湘龙的回来,大家也都回到了这个平台上面。接下去台湾方面有 4 天的年假。好了,台湾的观众朋友们跟大家说年假快乐。大陆朋友们,在解封了之后,我陆续有很多的朋友也都回到了大陆,或者往返于两岸,都觉得大陆复原的状况非常快,这是今年 2023 年我觉得最好的消息。不过提醒大家, 2023 年它仍然会是非常动荡的一年,它不是哪一年的问题,而是整个大的,二战之后的秩序,正在快速地转变,那整个的典范正在转移,生活在这个时代里面的时候,对这些的气氛是需要有一些捕捉的好,感谢。那个,嗯,你说比这个。



王孟源 01:20:51 

比那个还严重,嗯,是自从欧洲开始,全球大航海、大殖民之后。 500 年来,白人,西欧白人,嗯,占据世界顶端地位的这个崩解,这个体系的崩解的开始。也就是 500 年一次的变革,不是几十年一次的变革。



唐湘龙 01:21:15 

变革。好,就这样,就是这个WASP, WASP 的这个这时代的终结的味道。好,那时间的关系,星期五的时间,感谢孟源在二月的时候被我绑架了两次。那提供了非常精彩的内容,感谢呢,人在美东的王孟源,感谢。好,那也结束了。嗯,好,那也祝呢所有的我们的观众朋友呢,周末快乐。那下个礼拜见。拜。







\twocolumn[\begin{@twocolumnfalse}
\section{北溪调查、昂撒宣传、硅谷银行、沙伊和解、葡萄与宗教、巴赫穆特}
\subsection{20230317}
\end{@twocolumnfalse}]Credit: 栗子



唐湘龙 00:00 

真的非常有意思,这个家伙我注意他非常久了,就是一个。



王孟源 00:05 

而且是电信诈骗的罪名,然后他被逮了以后,他住的那个地方了烧起来了



唐湘龙 00:11 

对,那个很诡异,那个是超豪宅,那个那第五大道的超豪宅,这个他被他 FBI 最后撤了,就是 FBI 进去没多久就失火了。到底是怎么失的火?还是有一些很特殊的故事跟定时装置?我觉得这个火灾本身就非常的有戏,不过美国逮他的理由跟大陆通缉他理由几乎一模一样,这个就表示骗子到哪里时候手法不会变,其他的理由会变,但是手法不会变。好了,我们准备进现场。好,欢迎回到龙行天下,我是唐湘龙来今天星期五的时间 9 点半钟到 10 点半钟,有时候会长一点,很多的议题讲不完。好,今天是 3 月份的第三个星期五,通常没有意外的话,表定的第三个星期五就会是王孟源博士,王博士的讲座时间,我之所以叫讲座,就是说当这个栏目开了之后,跟孟源呢这样子聊了也快一年的时间。我的想法就是说,那我就不讲话,我就少讲话。其实王孟源的自己的准备的内容就够精彩,而且王孟源自己的叙事的风格,我觉得你让他专注的讲,对于我们的观众来讲收获最大,所以我就把自己退到旁边,就是除了去控制时间,然后尽可能的让王孟源准备的内容都能够讲得到。这个大概是我的工作。好,那今天王孟源的时间,因为王孟源一口气开出了 4 个题目的菜单,连题目的顺序我都不动了。就是就是,按照王孟源的准备一个一个来。那我先介绍我们的来宾人在美国,刚开场之前我在跟他哈拉郭文贵的事儿,就是觉得反正华人圈子里面光怪陆离的事情很多,可是在华人圈子里面,当我发现有王孟源这一咖的时候,是我觉得这一年最大的收获。好了,在我们一些线上的,人在美国的王孟源。欢迎。



王孟源 02:38 

好,非常荣幸



唐湘龙 02:41 

今天我要控制时间,因为孟源开出来的这几个题目我都很想听,都很想聊。我们先从这个就北溪 2 号,当北溪 2 号这件被炸了之后,它的整个事件到现在为止最少没有一个官方版的东西出来。



唐湘龙 03:04 

即使瑞典丹麦做过调查,可是那个调查本身讲跟没有讲一样,完全没有去指设到底是谁做的,连怀疑推论都没有。那相关的主要的两造,俄罗斯当然是高度的质疑,这背后是不是美国?那德国到现在为止官方一直沉默,这个对德国这么重要的事情可是一直沉默,但是最少到最近有两个主要的叙事的版本,一个是Seymour的这个所提的提出来的,在他的官网上面所提出来的那个看起来是深度的调查采访,那这个调查采访里面直接的就是指设了美国跟丹麦,美国跟挪威是里面最主要的在背后的这个操盘手,它就是被美国跟挪威所操作的这些的相关的行动给炸掉的。



唐湘龙 04:04 

但是隔了不久之后,以纽约时报为主的媒体所曝光出来的这个版本,这个版本说是一个亲乌六人小组所执行的,讲的也是活灵活现的,但这个版本出来之后,有很明显的在为特定国家、特定政府开脱的味道,就是这些周围的国家都无辜。其实我们都不知道,就是有 6 个不知道从哪里冒出来,现在连名字,连他们的背后的人脉以及他们的装备来源等等,或者这 6 个人到底去了哪里都不知道,但是说这亲乌六人小组六人帮他们,不知道为什么,因为他们亲乌克兰他们就自己去把这个北溪一号二号给炸了,那好像故事就到这个样子。可是这个版本出来了之后,最少你会看到俄罗斯是嗤之以鼻,俄罗斯觉得胡说八道,连Putin都公开讲话。好,我们先从纽约时报的这个版本谈起好了。这个版本,为什么你会觉得这是一个标准的假新闻?



王孟源 05:09 

Seymour Hersh 的故事虽然现在还没有直接的证据可以证明,但是它的逻辑都是正确的。没错,在细节上没有办法挑错。这当然是对美国很尴尬。很有趣的是,本月初3月3号,德国的总理Scholz只身的跑到Washington去美国访问,去跟拜登密谈,而且去的时候没有带任其他任何的阁员或记者。没错,他就是一个人坐飞机,连助理都没有一个人跟Biden密谈,密谈完以后没有记者会,没有公开讲话,然后他又偷偷摸摸回到德国去。然后四天之后就是3月 7 号,我们就有纽约时报,还有德国的全球就是这两家媒体,德国的Die Zeit。这两家媒体同步的刊登了那一道新的理论,就是有 6 个乌克兰人从波兰租了一招小艇,然后用scuba,就是那个潜水的氧气罩下去埋了炸弹。这个稍微有点常识的人都知道是童话故事。我跟你讲一下,为什么?因为北溪被炸掉的那一段,它的水深超过 80 米、将近 100 米。那 scuba diving 的极限是什么?即使是 20 年的教练也不能够超过 40 米。所以你要到达 80 多米去作业的话,不是你的随便在那个海滩上租一个scuba,那个旅游游客的东西就能做的,而是全世界只有军方,而且是就是联合国五常的军方,再加上少数最大的那种石油勘探的专业服务公司,才有在 80 多米作业的能力,就是装置那个炸药。为什么我说这个作业不是你光是潜到 80 米就完了?因为北溪是在 80 多米那种深度的一个管道、天然气道要承受很大的,每 10 米就是一个大气压,所以 80 多米就是 8 倍多的大气压。所以他的那个管道用的是高强度的钢, 40 多millimeter 的厚度,然后再加上特种水泥。非常强的一个墙壁,它的这个抵抗力,抵抗炸药的能力基本跟二战之后的主战坦克的正面装甲差不多。这不是你随便放个手榴弹那边就可以把它炸掉的。你要炸开北溪的话,而且他一炸就是,8 个管道里面 6 个,就一炸就成功了,这代表他用的装药是很大,而且必须是定向装药。所谓的 shaped charge,那这种东西是只有军方才能够提供的,所以你可以把原油公司,石油公司,就是 Schlumberger 这种专门替钻油平台做潜水的那些大公司也可以排除,因为他们必须要用专业的高爆炸药,然后定向,然后还要有专门能够把它定位的一大堆装置,每个装置大概重几百公斤,你要装 8 个,这是要一对专业人员用一天或者两天的时间才能够装置,在事先的话可能要几周由大国的军火商专门设计器材才能够做到。这种事情你怎么可能是 6 个人租个游艇跑出去,然后几个小时做,这是骗小孩的事情。



王孟源 09:30 

所以你从这里来看,从纽约时报这个这篇假新闻你所能得到的正确结论其实是三点。第一个。他们是重施故技,就是昂萨体系里面他们并不禁止、一般并不禁止你说实话,他只是用1万种的谎话、傻话来把你的实话淹没。这是他们操控舆论来以假乱真的手段,我们中国人感觉上我们的直觉就是有谎话,这个你就不能够让他讲,或者官方有一个说法的话,他就不让其他人讲。但是昂萨的体系是刚好相反的,他鼓励你去说傻话、去说谎话。如果 99. 99\% 的话都是谎话、傻话的话,那能够听到实话的人就很少,而且听到的话他也会以为是阴谋论。就是你鼓励了这么多假的阴谋论之后,真的阴谋被揭穿以后也会被人家当成阴谋论,对不对?所以我第一个结论、第一个感言是这是印证了他们的这个文化里面的老套。



王孟源 10:57 

第二个是他们真的把当代的自己的人民,然后还有支持他们的欧洲人民,还有像是附庸国家日本、韩国跟台湾的人民当成白痴,你想想看,这么这么可笑的童话故事,他们敢这样堂而皇之的散布出来,然后自由时报绝对不会指出我刚刚提出的这些简单的事实,这种事情你随便问一个潜水教练,他就会跟你讲不可能,对不对?但是台湾有媒体去查这种事吗?没有。



王孟源 11:36 

第三个你可以做的结论是,德国的传媒界比美国跟英国自己受昂萨的控制还要严谨。你看这一次全世界只有两个媒体同步发行,结果是美国跟德国同步,而且德国的报道还更详细。就是纽约时报还比较有点顾忌,还要故意含糊其词一点,那个德国人毫无顾忌,他们真的就把自己的国民当成傻子一样来骗,对不对?所以这里有三个观察,然后从这三个观察我要引申讲一些比较抽象的东西。欧洲的崛起它的硬件是来自 500 年前大航海,然后殖民压榨所得到的经济利益,这个经济利益结合了他的软件,他的软件是什么呢?是十五、十六世纪的renaissance文艺复兴,然后接下去是 17、18 世纪的启蒙时期 enlightenment ,enlightenment他的一句话就是理性。为什么讲理性?因为他要打倒以前的宗教对整个社会跟政治体系的的钳制,要能够理性,以逻辑来追求更好的、更合理的社会经济结构。所以他们白人过去 500 年的霸权,除了他们的殖民帝国主义,然后造成了全世界历史上最大的几场种族灭绝,也就是在北美跟澳洲,之外。



王孟源 13:36 

另外一个原因是他们真的有比较先进的文化,而这个先进的文化就是追求理性,但是这个理性,这个追求理性的过程到了 20 世纪的时候被反转了。这个反转的结果就是从他们以前的共和转化成我们现在看到的这个民主。我为什么这么说?你如果去看,大家讲西式民主,最民主的是谁?当然就是英国跟美国,对不对?但是英国跟美国在 100 年前的时候,他们并不是普选的。他们在 19 世纪或 18 世纪,他们的选举是你必须要财产到一道线,然后必须是男人,必须是有某某的产。对,我是跟你说过,到 1830 年的时候,他们还只有那个位数字百分比的人有投票,对不对?



唐湘龙 14:33 

要白人,要男人,要有钱人。



王孟源 14:37 

对,这种制度不是我们现在现代定义的民主,而叫做共和。他转成我们现在的普选民主其实是 20 世纪的事。美国的总统大选初选是由党员投票普选出来是从 1968 年才开始,而且到现在为止还不是全部普选。还有所谓的 super delegate 就是超级票,就是你如果是干部的话,你的票比人家强 1000 倍,这样子。就是他们之所以搞这些普选是非常晚进的事。



王孟源 15:26 

为什么会这样搞?其实呢,他是为了要反转过去 300 多年的理性传统。为什么要反转理性传统?为什么说反转理性反转?因为老百姓是最笨的,用那个传媒,我刚刚提到这个新闻,传媒弄的假新闻,大多数的人就会轻松的被骗,对不对?所以之所以要搞普选,要搞这种公投,要推算推广现在的这个版本的民主其实是为了反转理性。



王孟源 16:07 

为什么要反转理性?我想几个月前我也讲过,理性的人本主义就必然会达到社会主义的结论。那 100 年前发生了什么事?就是社会主义成为对资本主义社会的一个威胁,就是苏联,你为了要提防苏联,你必须要抹黑共产主义跟社会主义。要抹黑社会主义就不能够容许理性的人本主义,但是人本主义这个就是以人为本,以人为先的话,你这个不能够公开的讲说我不想照顾我的公民,我不想照顾国民。所以唯一的漏洞就在理性这上面。所以过去这 100 年英美的演化反而是越来越不理性,越来越非理性,他们的这个制度,搞的现在腐化到这个地步,其实是 100 年以来为了抵制苏联,所以必须要强化他们的普选等等的。然后这些作为,这些所谓的改革,所谓的民族改革,其实是往非理性的方向走,而真正掌权的资本是乐观其成的,因为他们可以趁机掌控国家的经济命脉。到过去 40 年,美国的 GDP 成长基本上全都进了最富有的 1\% 的人的口袋,就是已经成功的利用这个非理性的趋势而掌控了所有的权力。



王孟源 17:51 

我们待会可能会谈到俄乌战争,现在俄乌战争已经变成美国自己国内也开始觉得不太对劲了,因为你自己的经济搞得乱七八糟,然后还有火车爆炸之类的事情,你还在几千亿、几千亿的往乌克兰那边送。那固然这是民主党很乐意的去做,在共和党那边你会注意到他们现在所有的大咖都已经闭嘴了,只有一个例外,就是他们的参议院院长McConnell还在继续的挺乌克兰,高调的挺乌克兰。这原因是什么呢?因为 McConnell 本来就是出来顶锅的。他本来就是共和党里面负责讨人厌,就是我的专业就是要违反选民的意见,(他符合所有的川粉所讨厌的那种的条件,都在那 McConnell 身上),他的锅全部都由他来背,反正他的那个职位,他的那个参议员的职位是没有危险的。所以这样子一来,其他的共和党的大咖就没有问题,包括DeSantis就是明年,明年台湾跟美国都有大选对不对?如果Trump又得到提名的话,拜登很可能选不过。所以建制派在Trump被搞掉,在 2020 年被搞掉的那个时候就已经安排了一个后手,就是安排了一个暗子,就是DeSantis,DeSantis本来也是建制派,他以前在当国会议员时候,他的言论都是典型的建制派,但是在就在四五年前他当了佛州州长以后,就忽然完全模仿Trump的政策,就是他变成去抢民粹派的投票率。所以Trump这次明年选举的话,最大的危险在于初选的时候会输给DeSantis。



唐湘龙 20:27 

我把刚刚这个话题说个尾巴,就是有关于北溪 2 号的这件事情,当然北溪被炸其实切断了,切断了欧洲跟俄罗斯之间最后的联系,使得俄罗斯作为欧洲一部分的这个最后的希望,在那条线被切断了之后,你基本上大概就是中断。可是对德国来讲,这两个叙事版本里面来讲, Seymour的那个叙事版本,除了 Seymour本身的权威感之外,那个叙事版本你看也知道那合逻辑,你看也知道那要合理的多。那后面的这个 6 人小组的那个故事,一看就知道是鬼扯,就是胡说八道,可是问题是德国的政府为什么这么xxxx?你刚刚提到了,就是说Scholz到美国去的时候,他采取了一个不像是大国互动的方式,如果他只是想要去跟拜登谈些话,那为什么不能够用电话谈?你电话谈就好了,何必要专程跑这一趟?你没有带人、没有声明、没有任何的联合声明,也没有从你们的对话当中xxxx出对当下大家最关键的事情的共同立场都没有,那德国为什么是这样?



王孟源 21:48 

每个欧洲国家的政府,还有他们的军方高级将领都知道事实真相,问题是没有人敢说这个,就像国王的新衣一样,(没有人敢讲出真实的心话)。所以Scholz被召唤到华盛顿 DC 去开这个会,其实就是,我认为就是拜登跟他讲我们再过 4 天的话,要公开这个借口,你必须要乖乖的支持。



唐湘龙 22:19 

就是塞了一个故事给他,你必须要相信这个故事。



王孟源 22:23 

你最起码不能够顶撞。



唐湘龙 22:27 

这个也合理。



王孟源 22:32 

而且我们会要求德国的媒体界不能够质疑这件事情。所以你要配合,到时候如果有媒体界的人不高兴去跟你们政府抗议,你们不要讲话,就是我觉得是这样的一回事。好我们赶快讲,第二个话题,然后我们的话题不赶快的话。



唐湘龙 22:49 

因为你在美国,当这个礼拜对拜登来讲,我前两天我稍微整理一下这 3 件事情对美国政府颜面或者它的里子都很有伤害。一个就是黑海的这个撞击事件,那俄罗斯用非常强悍的这个手法,把它的一架 MQ-9 把它给坠个坠毁了。另外就是中国成功的调停了沙特跟伊朗之间过去的恩怨,这两件事情都让美国颜面受伤很大,可是我真正关切的是美国现在的内部的金融危机,那这个金融的危机跟他过去三年里面,在疫情期间,他的财政跟他的联准会的恶性操作是有关的。等他的防疫失败以后,我说他是用经济手段去防疫,因此开始大印钞、大撒币,大撒币造成了大通膨,大通膨之后他开始大收银根,他连续性的操作都是很剧烈的



唐湘龙 23:57 

这种的操作操作完了之后就是我们现在看到的SVB。 SVB 严格讲他如果在正常情况下面, SVB 的那个操作是不是没有什么问题的?我个人认为啦。可是这样 SVB 的事情一烧了之后,开始一个个烧signature,然后到昨天接下去是瑞士信贷,瑞士信贷,那么因为瑞士政府出来了,所以在我们在现在在做直播前的 12 个小时的时候,本来第一共和银行一开盘的时候就杀到底了,就熔断了。可是这个时候美国呢?美国开始再度的进场,包括叶伦,包括 JP 摩根再度去进场,把整个的市场的那个感觉危机受管控的那个气氛再拉抬起来,所以今天早上的股市最后是收红的。好,那美国的这个金融危机第一个它过了吗,那接下去会是一个怎么样的状况?到底在美国当地的像你这样子的关注的,关注的人来讲,这次的危机代表的什么?



王孟源 24:58 

其实我从 2019 年开始就把这一次整个美国通胀危机、然后泡沫爆破的这个形势路线都已经讲得很清楚,反复的讲得很清楚。它的一个最大的问题就在于美国现在美联储除了印钞以外没有办法,可是印钞太多了,它就会通货膨胀,而通货膨胀已经没办法再压下去了,但是他如果升息或者是把流通的货币收回来的话,就会造成泡沫爆破。



王孟源 25:37 

你刚刚说的没有错,这一次这个 SVB 的确是史无前例的方式而倒闭的,为什么是以史无前例的方式为倒闭?其实华语世界已经讨论的很多,而且我觉得大部分的描述都还蛮精确的,我现在简单复述一下,然后指出他们错误的一点,一个细节。就是 SVB 在三年前,在 2020 年初,它的整个的资产还只有 550 亿美元,然后经过两年的放水,就是美联储放水了5万亿,两年放水5万亿,它的资产增加到 2000 多亿。这个原因就是因为很多的这个放水的那些现金进了所谓的 private equity 跟 venture capitalist 的口袋里面。那他们这些风投基金就拼命的在矽谷去找投资的对象。尤其曾经在那两年还有一个通道,就是可以走后门,不需要股市上市,可以先上市再选择哪一个公司,就是一个基金先上市,然后这个基金再去买一个公司,这样子这叫做SPAC。



王孟源 27:08 

所以那这些这样就有几万亿流入了矽谷的这些,因为它有借贷,有杠杆,那接进了矽谷的这些公司以后,他们拿到这些钱必须要往银行里面存,那这个矽谷银行刚好是专门针对他们提供服务的,其实做的不错,他们的生意做得不错,服务的很好,所以他们的是市场的份额蛮高的。那到了两年时间,它的资产就增加为 4 倍,就是 2000 多亿。但是美国的定义是 2500 亿以上才是大银行, 2500 亿下是小银行,所以它还是小于。



唐湘龙 27:58 

快要到大银行,快要到了。



王孟源 28:00 

对,然后现在问题是在一年前他开始升息的时候,开始从量化宽松转为量化紧缩,然后开始升息的时候。第一个就是这些私募基金还有风投业就断了,因为股票先跌下来,股票一跌就没有IPO, 没有IPO 的话,那你这个你这些风投是所谓的有五六级的,第零级是所谓的angel,然后第一轮、第二轮、第三轮、第四轮,然后第五轮,然后到最后兑现、换现是必须要 IPO 嘛, IPO 是最后一步,你 IPO 一旦那个关被堵住以后,整个这个通道就不顺畅,他们就没有再拿到钱了。所以在过去这一年,就是 2022 年, SVB 的资产下降到 1800 亿,从 2000 多亿向下降到 1800 亿。那这里的问题是他资产一下子暴增 4 倍的那两年也刚好是 0 利率的时候,所以他为了要投资到利率不是 0 的资产,就必须要买长期资产。那这里是我觉得目前华语界的解释中唯一一个有点错误的细节,就是很多人怪到国债上面,他们如果买国债的话的确是会有问题,但是你实际去看他的账目的话,他的国债很少。它是真正出问题的是所谓的 mortgage backed security,那 mortgage backed security 就是把房贷打包转手当做金融债券这样子再卖出去,它的这个特性跟国债其实是很像的。



王孟源 30:12 

那我自己因为在过去这一年,这个长期国债的利率升了大概 3 percent,这个 MBS 也升了大概 3 percent。那这个问题是你的 yield 就是等效利率提升 3 percent的时候,他们这些债券、美国这些债券都是这样的,就是你去拿利息,或者到齐了本金拿回来,这个他的数额都是固定的,浮动的是什么?浮动的是你现在去买的时候的价格,所以比如说你如果买一年期的东西的话,它固定是要给你 100 块,那现在的价格如果是九十块的话,你就得到 11\% 的利息。所以你如果去算这个长期的、十年级的利息,去算的话,可以算这个国债跟 MBS 现货的价钱会掉多少。但这里有一个问题,就是国债的这个利息支付,(国债)每年有利息支付所谓的coupon,它这个coupon是不一定的,不一定是多少的,所以你可以算两个极端,一个就是没有coupon没有利息支付,只有到十年之后本金拿回来。那如果你十年利息增加3 percent的话,年利率增加了3 percent 的话,你这个没有coupon,所谓的zero coupon bond,它的那个价值会掉25\%,会损失25\%。如果全部都是coupon,而在十年之后就没有本金的话,那这个它会掉14\%。





那我刚刚讲这个  SVB 呢,它大部分的资产都投到 MBS 上面去了。那 MBS 因为是房贷,房贷你还款的方式是本金连利息这样定期的付,所以他是没有说到最后一期一大票本金给你的,所以就是相当于这些十年期的MBS,它的价值在一年之内掉了14\%。美国的它的资产储备就是杠杆率,规定银行的杠杆率最多只能是10,也就是它的储备必须是你资产的10\%,你的现金储备。那我在过去 4 年一直讲,我不认为大银行会出问题,为什么?因为他的大银行的储备在 2008 年之后学乖了,所以他们这一次很保守。他们的储备很多,大部分都是在 16,不是10\%,而且就是政府规定 10\% 的储备,他们其实有 16\% 、17\% 的储备,那这个SVB的储备率是多少?也是16\%,就是他的这个准备金的充分到什么地步?他一年前的时候,他的准备金等同于大银行。那你这样这么一个保守的银行怎么会一年之内就垮掉?而且事实上是一个礼拜就垮掉了,对不对?他这里的问题就在于他的这个生意的特别。



就是因为大银行没有像他这样子是在 2020 年跟 2021 年一下子暴涨 4 倍,所以全部的资产都是在那两年利率是 0 的时候投入的,其他的大银行没有这个问题,只有他这个小银行有这个问题。然后因为它服务的都是这些创投业的新公司,所以他们一旦拿不到新的钱以后,他们就必须要靠存款来维系,所以 2000 多亿一下掉到 1800 亿,它损失了大约十几 percent 的资产,在一年之内失了十几percent,这个也是大银行没有的问题。那他损失了这十几 percent 的资产,问题是什么呢?就是你这个 MBS 的价值掉了 14 percent 之后,其实是浮亏嘛,因为你实际上要拿的利息,还有未来的 coupon 都还是一样的。就是如果同样的这个 MBS 现在到现货市场去买的话,它的价值掉了14 percent,但是你如果是抱着他一直抱到 10 年之后,那没有什么差别,所以这种亏叫做浮亏。但问题是他有 400 亿的资金,去年被抽走了,那这 400 亿哪里来?你必须要把这些 MBS 拿到现货市场去卖,换现金才能够还给人家嘛,对不对?才能够给出户提出来,那就是这 400 亿的MBS他必须要认赔14\%。那这样一来他的储备金就看起来就远远没有 16\% 了,对不对?你这些浮亏变成实亏之后,你的这个储备金就降下来。



然后问题是一个多礼拜前他们在周一公开了这件事情,那公开了这件事情以后,第三个特有的毛病就出来了,就是为什么?你如果你的储户都是小老百姓的话,他们都有是有联邦保险的,所谓的 FDIC 保险。



唐湘龙 36:20 

就不会出现一个挤兑的情况。



王孟源 36:23 

就不会出现挤兑的情况,因为反正我有保险嘛。而且一般的小老百姓对这些事情,对这些金融新闻也不会。



唐湘龙 36:33 

敏感度没有这么高。



王孟源 36:34 

对,但是如果你的客户是几千个小公司,每个公司都有几千万的资金的话,他们这个资金超过 25 万就。



唐湘龙 36:46 

不在存保的范围。



王孟源 36:49 

然后他们都是有专业的会计师跟金融人员,所有 CFO 在那里,每天的责任就是盯着这个看,他们如果不赶快把这些钱拿出来,他们就是失职。所以上个礼拜周一的他这个新闻一出来,所有的这个这些新公司、创业公司就全部去挤兑了,所以这是他出的问题,就是为什么 SVB 这次出事这么奇怪?是史无前例的出事。



王孟源 37:24 

第一个它是三年前, 2020 年到 2021 年,它的资产扩展了 4 倍,而且扩展了 4 倍,必须投资在 0 利率的环境里面找东西投资。第二个是去年的时候,这些一下子反转了,然后所有的他的客户的那个没有进账,所以他收缩了将近 20\% 的资产,那这 20\% 资产你必须要把浮亏转成实亏才能够付得出来,那这样一来就影响到你的筹备率。一旦这个坏消息在上周一出现。第三个问题就是你的客户全都是大客户,都是几千万或者甚至上亿的客户,他们都天天的看华尔街日报,他们一看到这个消息,马上就去挤兑,所以他就这样垮。



唐湘龙 38:15 

好,这几这个礼拜的时间,当然整个的国际的股市都受到了SVB,因为两天它就破产了,所以从上个周末开始,这几天的时间就一直烧,那前天是烧到了瑞士信贷,然后在今天早上的时间,本来大家都在看加州的另外一家重要的银行,就是这个第一共和银行,那第一共和银行今天本来开盘的时候很糟,但是因为很快采取了这个救市的动作,然后他拉起来了,市场上面看起来现在觉得好像这件事情是不是就稍微稳住了?这个事件接下去第一个对美国的影响就结束了吗?



唐湘龙 38:59 

第二个还会有后续的受害者吗?



王孟源 39:03 

是这样子的,这个其实在 SVB 出事之后,下一个出事的是所谓的Signature Bank这是一个纽约的。



唐湘龙 39:11 

纽约银行。



王孟源 39:12 

对。他出事的原因是因为他跟加密货币有关系。在 SVB 出问题之前他已经摇摇欲坠了,所以他倒掉没有问题。这个first Republic 就纯粹是因为受感染、流动性的问题,大家也跟着挤兑。那这时候因为它的资产还是正的,所以你不只是联邦可以出手,其他的大银行也愿意接手。所以这个是很简单就解决了。那事实上我过去 4 年已经一直讲过,美国的大银行非常的健康,比 15 年前健康的太多了,所以这一次的问题不会出在大银行,问题是出在他的货币,也就是国债上面,这个国债并不是说他会赖债或者是没办法付出,而是我刚刚讲过的他在走钢索,你升息的话就会泡沫爆破,就像这样的它降息的话就会通货膨胀爆发。



王孟源 40:22 

所以,我认为我们现在看到的美国的这些小银行出问题了,只不过是这个真正深层问题的一个征兆、一个病症,它这个它还没有到,但是因为这个深层的问题就是有美元的国际储备货币地位来托底,现在还没有到会出现连锁反应的地步。所以这些征兆顶多就是引发紧张,而这个紧张在美国本身是很容易的遏止的,为什么呢?因为我刚刚讲过美元还是国际储备货币,而且在去年刚刚收割了欧盟,英国,日本,所以美国其实跟这些国家相比是相对健康的,所以我们现在看到欧盟的这个瑞银出事反而是比较严重,而且接下来说不定有英国,说不定有日本的银行会出事。



唐湘龙 41:30 

英国已经开始了。



王孟源 41:31 

更危险的是去年一整年通货膨胀已经打得很惨的,就是那些财政的赤字跟贸易赤字向来都很高的国家,像土耳其、阿根廷这些反而是真正的危险。至于美国本身,我觉得因为这个美元的地位到现在已经有快 80 年了,它根深蒂固,这个国际储备货币这个东西是有所谓的网络效应,大家都用的话,你也会想要用,因为它就是方便便宜,而且所有的附属的支持的设备都在,所以我认为能够真正威胁到美元国际储备货币地位的黑天鹅或者灰犀牛事件是今年八、九月,我想我上个月好像有提过,就是今年八、九月的时候金砖会议会收一大堆新的成员,包括沙特,而且沙特之所以会必须要先跟伊朗和解。



王孟源 42:44 

就是因为这是他加入上合跟金砖的先决条件。所以其实是一年多前就可以预言的。就是光沙特没有加入西方制裁俄国的那个行列,就代表他必须要加入金砖跟上合。他如果要加入金砖跟上合,就必须要先跟伊朗和解,所以这个调停一定是早年,一定是不止一年。一年多前,大概很明显,其实很明显的是普丁在推动的,那现在俄国人是在西方人人喊打,那这个沙特跟伊朗和解的这件事情,如果要在欧洲得到正面的回馈的话,就只好由中国出面调停,由中国来代做代表。

所以我认为对美国的影响,这个SVB出事是局部性的问题,它是一个征兆,而不是一个病因,它的这个病因可以拖到至少今年下半年。今年下半年,我上个月已经讲过,除了金砖扩大,所以是一个推出金砖货币的一个天赐良机之外,还有美国的国债上限。你要知道美国现在因为已经顶到国债上限了,所以从2月开始一直接下去六、七个月不会发新国债,对不对?美国现在的债务是多少?一年的财政赤字是 18000 亿,过去三年就是 2020 年、2021 年、 2022 年全部的财政赤字是6万亿,6万亿里面有将近一半是美联储买下来了,然后另外有2万亿是美国国内的基金还有个人、私人买下来,银行买下来的只有 3000 亿,所以我刚刚提到说 SVB 这个问题跟国债没有太大的牵连,为什么?因为所有的银行加起来一共只买了 3000 亿,对不对?所以 SVB  是一个 2000 亿的银行,那你所有的银行加起来只有 3000 亿,它当然国债是没有太大的影响,真正影响它的MBS,那另外还有外国的中央银行买了 8000 亿。



唐湘龙 45:31 

没错。



王孟源 45:32 

中国的人民银行在过去三年他对美元的外汇储备所持有的美债减持了不到10\%,那我刚刚解释过这些 10 年期的美债,它的价值在过去这一年下降了25\%。我顺便提一下,是因为人民银行的那些主管真的是一言难尽所以,这个事情我几年前就讲了。



唐湘龙 46:04 

对,但是问题是人民好像易刚这些都还是继续留任的,所以你显示说现在的大陆,因为这个事情孟源已经讲过几回,对大陆的这些现在的金融体系实际上面在操作的操盘手。那孟源一直觉得那个操作的手法跟心法都很有问题。但是孟源在提醒就是说这件事情其实大家过度关注美国,认为会引发美国的金融崩溃,但是孟源是说没有这么的严重,这件事情基本上是局部的、是个别的,但是这些火往外面烧的时候,美国本身的问题可能还没有美国以外的这些这些国家来的大。那重点就在看今年的下半年,真正的美元的生存战,应该是今年金砖国家开会之后,那个时候才会看的比较清楚,今年有可能是美元要打生存战的一年,那这个下半年的时候我们再来看。我们再来看孟源今天的准备的题目里面,其中有一个是我在看的时候我看不太懂,干嘛?这个人是喝葡萄酒喝太多,他在讲这个葡萄被人类驯化的过程当中,看起来这个是,这个是演化过程当中很重要的一件事情,为什么?



王孟源 47:25 

是这样子,我一直对人类学很有兴趣。你如果到博客上看的话,我有一篇文章叫做欧系文明的起源。这个就是谈,因为过去 20 年我们人类的基因技术有大幅突破,所以对古人类学的研究一下子有很多新的实证。那我这次想要谈的是云南(农业)大学有一位叫陈伟教授,他所主导的一个全球性的计划,就是针对全世界所有现在的葡萄, 3500 多个不同的品种都做基因分析,然后这个基因分析做出来以后,用超级电脑超算去倒推他们的演化过程。你可以比较他们的基因组合,然后这个叫做 Genetic clock基因时钟,可以倒推他们这些是什么时候因为突变而分开来的。那最后他的结论是,这个所有的我们现在人类吃的葡萄 、3000 多种品种的,都来自两个独立的驯化过程。这个时间点是 11000 年前,发生在以色列跟格鲁吉亚,就是乔治亚,就是独立的两次驯化过程。那我看到以后有两个感想。第一个是一直到近年,我还是常常看到很多这种全球性的大科学计划,中国有加入、有参与,但是他们还是只负责做采样。真正的分析还有计算,还有做结论是在欧美做的。那这一次这个是反过来,因为葡萄这个东西其实主要是欧系的文明的一个饮料,对不对?所以大部分的这些品种都是在欧美的。那结果反过来是当地的那些人,当地的大学的研究人员帮他们采样,然后由中国负责做基因分析,然后负责做超算的计算。这我觉得这是一个很正面的阶段性事件,是一个很好的现象。



那另外一个感想就是11000 年前这个数字把我吓了一跳。为什么把我吓了一跳?人类的农业起始是驯化小麦开始的。小麦是什么时候被驯化的? 11500 年前,所以你这个葡萄只跟小麦差了 500 年,这个很惊人,它事实上是人类驯化的第二样农作物。因为你看大麦是什么时候驯化的?大麦是 10500 年前驯化的、小米1万年前、大米 9000 年前。这些都是同样的基因的那个研究可以做出来的。这个小米最早驯化是 11500 年前,在哪里?在叙利亚。然后做出来之后,其实它的扩展是很缓慢的,就是周边像是以色列,或者是像两河流域了,或者是像土耳其,那个都是又拖了几千年才真正的农业化。就是真正农完成农业化,两河流域还有近东,完成农业化其实是 8000 年前的事。所以我看了这个数字以后吓了一跳,我说这个葡萄怎么会是人类驯化的第二项农作物,然后刚好我这几十年关心这个人类学的基因研究,我一直有一个猜想,就是我注意到这个麻醉药或者说毒品,或者是类似酒精这种东西,就是影响心智的这些化学品,跟宗教好像老是有很明显的关系。



我举几个例子,那个2007 年的时候以色列出了一篇论文,你如果去看旧约圣经,它里面他讲摩西在Mount Sinai跟上帝对话,然后拿到了 10 戒。你如果看他的描述是他看到树丛一下子就着火了,然后有雷声什么的,然后就听到上帝的声音,以色列在 2007 年的时候有一个研究人员注意到,当地生产在山地上的有一种草本类植物,它含有一种生物碱,而这个生物碱你吃下去以后的症状就是会有幻觉,会看到起火,然后听到了雷声。



唐湘龙 53:04 

我听懂你的话,摩西喝醉了。



王孟源 53:10 

摩西吃了那个草药。



唐湘龙 53:13 

他出现了幻觉。王孟源讲的这一段,我担心前面的朋友们可能没头没脑的不清楚王孟源的叙事逻辑,我简单讲,孟源一开始讲到的就是说 11000 年前跟 105000 年前,为什么那个数字很重要?就是以我们现在的生活的角度来讲,我们会觉得葡萄又不是我们的主食,葡萄为什么这么早就被驯化?因为人会想要去驯化一种的作物跟动物,它一定有很现实生活上面的需要,多半都是为了满足生存的需要。所以如果你驯化小麦,这合理,因为你有很稳定的热量的来源。可是葡萄作为今天我们都要把它当作水果,当作饮料,不会当主食啊,你在餐桌它不是主,那为什么驯化的这么早?一定还有另外的一套的逻辑跟需要,那跟宗教就有关系了。



王孟源 54:13 

社会的需要一定是很强的,对不对?已经接近要生存的那个需要。



唐湘龙 54:18 

我们需要葡萄酒,不然活不下去。



王孟源 54:22 

然后另外一个例子,你们大家可能听过希腊有一个所谓的Delphi,Delphi,有一个叫做Oracle,我不晓得,我忘记中文怎么翻译了。这就是他们以前希腊的城邦要出征的时候,都要派人去那边请圣女,给他们判断会胜会赢还是会输,这个是所谓的Oracle一直到现在还代表着铁口直断的算命,就在西方。在也是十几年前,这个大概在 2011 年的时候,希腊有一个团队去做了研究,他发现那个 Oracle 、Delphi Oracle 的遗址刚好就在两个断层的的交汇点,那个几率非常非常的小,你那个建筑刚好就建在那个断层的交汇点上,而那个交汇点上到现在还有一堆地底下的碳氢化合物出来,那这些碳氢化合物你如果吸进去以后一个症状就是你会失去神智,然后像乩童一样那样跳来跳去,台湾的乩童也许是假装的,但是当初Delphi的那些圣女是真的吸了这些气体,然后就开始胡言乱语,这是希腊宗教的一个非常重要的成分,结果是有科学的解释,他解释的是一种麻醉性、迷幻性的气体,也是迷幻药。



王孟源 56:12 

我再举一个例子,我们现在所有的西方的所谓的一神教的都是来自犹太教。犹太教是什么时候开始的?其实是,犹太人原本是有两个古王国,一个叫做以色列王国,另外一个叫犹太王国。公元前 700 多年的时候,新亚速帝国把以色列王国给灭掉了,以色列王国其实是比较大比较强的那个,灭掉以后以色列王国的那些贵族、还有他们的的宗教人员就逃亡到犹太王国去。然后把他们的神话也都带到犹太王国,然后犹太王国就根据他们所带来的那些神话改编,改编出说在 500 多年前曾经有一个非常繁荣的所罗门王,然后有一个大卫王,然后有摩西,那些神话都是那个时候写下来的,就是公元前 600 多年。然后公元前 580 年,八十几年的时候,新巴比伦帝国把犹太王国也灭了,把他们的贵族,1万人的贵族跟教士全部抓到巴比伦去。



王孟源 57:43 

在作为囚徒的那几十年中,他们才写下了犹太的旧约圣经。根据以前的那些神话整理出来,然后添加、添补出我们现在看到的旧约圣经。我为什么说这么多呢?我想要讲的是,他们的神话所讲的就是公元前 900 多年、公元前 1000 年就是所罗门王跟大卫王的那个时代,犹太人其实只是所谓的Canaan(?)的一个分支,他们后来在就业圣经里面痛骂这些Canaan,即使他们指的是不信我们犹太教的Canaan,他们其实是Canaan的一个分支,他们那个时候信的神,还不叫做耶和华,而叫做xxxx耶尔。Canaan的那个主神叫耶尔,但是他不是一神,他是有老婆的,他的那个老婆叫做Asherah,所以就有点像那个希腊神话那样子,有大神、有一个大神的老婆,然后有一大堆小神。那也是同样的十几年前,他们以色列的人去研究 3000 年前的这些神坛的时候,在Asherah 的神庙的祭祀神坛上面发现了化学品的遗迹,就是他祭祀的时候有一个石头做的祭坛,在祭坛上面他们发现了化学品的遗迹,分析以后你猜猜这个化学品是什么?是什么?大麻,他们在拜拜的时候烧大麻的。



唐湘龙 59:33 

所以其实当然那个时候人对于这些的这些东西,它只有很直觉的经验感受,就是这个东西对我来讲会产生一种的可能轻轻松感、快乐感,或者是这种的某种愉悦的幻觉,或者是喝了酒之后的那种的迷幻的状态。



王孟源 59:55 

对,就是那种迷幻的感觉,上古的所谓宗教跟这种迷幻的感觉,是有非常强的干连的。我再给你一个例子,全世界在昂萨的人在北美跟澳洲做种族灭绝之前,人类历史上最大的种族灭绝是什么?是亚利安人征服了整个欧洲。



王孟源 01:00:21 

还有波斯还有、北印度。他们在波斯跟北印度因为当地人口比较密,他们没有办法做种族灭绝,但是他们进欧洲的时候就做了种族灭绝,消灭了当地人口的一半以上。亚利安人因为二战的时候德国纳粹的影响,后来这个名字不太光彩,就不用了。他们改成Yamnaya,现在讲亚利安人不叫做Aryan,而叫做Yamnaya,Yamnaya是俄文、从俄国来的,其实是同样的东西。Yamnaya的一个特征就是他们非常的迷信,他们到处都有祭坛,然后他们用的最多的一个从外地来的贸易品是什么?也是大麻。他们的宗教观念也是建基在大麻之上,也就是现在欧系文明的最早起源,本身就是非常依赖大麻的一个民族。



唐湘龙 01:01:32 

所以如果我们管制毒品跟酒的话,宗教就会消失吗?



王孟源 01:01:38 

没有,我是说...现在当然不一样,但我的意思是说在上古时期,人类刚刚开始形成社会的时候,宗教这些迷幻药对宗教是一个决定性的推力。然后我现在再回头讲这个葡萄的事情。人类最早的庙有多久?刚好跟驯服小麦是同一个时期, 11500 年前,在土耳其的南部有一个地方叫做Göbekli Tepe。这个当时既然还没有驯服小麦,你怎么会有人建庙呢?就是因为你在冰河时期过去,冰河时期是在 14000 年前结束的。14000 年结束之后,早期人类原本是所谓的hunter gatherer 游猎部族,就是每一个部族就是十几个、二十个人一个家族,然后追寻着食物不断的迁徙。那到了冰河时期结束之后,气候稍微稳定一点,在远离着北大西洋的那些地方,就是在中东那里,古文明才可以开始发生,但是他们的第一步不是定居,因为你要定居的话必须要有农业的话,必须要先巡抚小麦。所以从 14000 年到 11000 年之间这 3000 年是怎么样子的?是所谓的半定居,你用英文去写的话,有兴趣的人可以去 Google 这个词叫做 semi sedentary neolithic cultures。



王孟源 01:03:31 

他们这所谓半定居是什么?就是他们也吃谷,但是吃野谷就是野生的小麦,然后收集这些小麦,然后或者就在湖边钓鱼,或者就去猎一些小的猎物,像兔子跟鹿这样子。就是在这边定居几个月,东西吃的差不多以后换一个地方再住几个月,这种叫半定居的。这个在土耳其的这个神庙一共用了 2500 年。事实上进入农业文明以后,它反而被废弃的。它是一个给这些半定居的游猎家族共用的一个神庙。那我认为它很可能就是人类宗教的起点。为什么这样说?因为宗教是什么?宗教并不是mythology。你的那个老祖母在晚饭的过程跟你讲一些神话,这个叫做mythology,这不算宗教,对不对?真正的宗教是有组织、有纪律的、有规模的信仰活动对不对?有组织、有纪律、有规模的信仰活动的话,你最起码要半定居,如果是以前的 14000 年、或者更早的那些游猎家族不可能有规模性或组织性或纪律性,就是一个家族嘛对不对?讲讲神话而已。要半定居下来,就是周边几十公里内的那些游猎家族都有兴趣要拜同一个神,那你才会去盖一个庙。但是你这个,你要感觉到这个庙,会真正跟神接触。在当时可能还没有大麻,唯一的手段就是酒精,对不对?你要大家喝醉了舒舒舒坦坦的才会感觉到神性嘛,对不对?然后大家才会愿意出钱出力去盖这个庙,去维护这个庙。



王孟源 01:05:54 

我这么说不是完全的猜测,因为我刚刚说过人类转为半定居的状态最早也只能够是 14000 年,但是在以色列它已经找到一个 13000 年前的遗址,是一个,我刚刚说他们在这边住几个月,然后再换一个地方,那这边有一个洞,是他们每年居住半年的一个洞。在这个洞里面找到了酿啤酒的地方, 13000 年,比那个最早的庙早了1500年。所以我觉得很有意思的是,我的推测是,是这个啤酒是造成后来 1500 年之后那些半定居的游猎民族群愿意集合起来盖个庙的原因。就是他们去这个庙定期的集合,集合起来干什么?就是那些家族的家族长,大家用啤酒把自己灌醉,然后享受神性,跟神沟通一下。这是宗教的起头。然后这也能够解释为什么 500 年后他们就急着驯服葡萄,因为当时除了你的家族内部的事务,能够跨家族的事务是什么?唯一的就是这个早期的宗教,这个是当时社会最大的凝聚力,所以会能够强到推动他们去驯服一个不管吃饱的东西,葡萄这种东西,必然是当时社会最重要的事情,那当时社会最重要的事情就是宗教。



唐湘龙 01:07:47 

孟源刚刚讲的就是说今天我们所经历的宗教。当然我们谈到宗教的时候,大部分不管你的信仰是什么,大概都会抱着某种的虔诚的神圣或者仪式性的表达。那宗教也确实是人类开始进到文明了之后,人类的熟识社会跟神性社会之间互动的时候,大家建立了一套的规范跟应对的模式。但是孟源只是从一个人类学的角度里面告诉你,就是说今天我们看到的宗教的语言,为什么有时候我们听不懂?有很可能他的创造者,他的起源是一群的酒鬼,基本上面可能是我们今天来讲,可能是违禁品、毒品的成瘾者。好,那这个可能是创造的宗教的那个氛围非常重要的基础,血缘可以凝聚一个家庭,但是只有宗教能够凝聚一个社群,在那整个人类开始开始规模化的那种社群扩张的过程当中,宗教真正的意义在这个地方。现在 10 点 10: 38,我还有几分钟的时间,我要问一下孟源,因为虽然俄乌在战争,今天我们不能谈很多,可是俄乌战争,现在整个乌东地区的这个巴赫穆特的这场的战势,这场的战事现在的情况,它会改变乌东战争的形势吗?



王孟源 01:09:15 

很显然的俄军并不急着把它包围起来。他们围三打一消耗乌军的人员那个伤亡比率已经很满意了。事实上他们现在正在准备复制另外两个,就是如果乌军忽然决定要从Bakhmut撤退出去,俄军已经准备好另外一个围三打一的地方,叫Avdiivka,你如果去看现在的这个地图的话,基本上已经完成三面包围,就是还没有开始施加压力而已。然后一旦Bakhmut撤守的话,它的这个那些 Wagner 的部队可以简单的北上,然后另外还有 Sivers'k 也是看起来很简单就可以变成三面包围的态势,就是现在西方媒体还在那边做梦说什么乌军会有春季攻势,然后就是想要复制去年9月的那个秋季攻势。但是我提醒一下大家,去年9月的秋季攻势是乌克兰动用了 12 个旅的预备队去打俄军两个营的守军,因为他们当时俄军,当时乌克兰跟俄国的兵力对比是 2: 1,所以它能够得到那样的局部优势。但是这一次乌克兰集中的、能够动员新部队、所创造的新部队有 19 个旅,这看起来好像很多有 6 万多人,但是俄国刚刚在四个月前动员了 30 万人,目前绝大部分还没有投入,他们在后方当做预备队的有二十几万人,你说你用6万人的预备队去打二十几万人能够打出什么名堂来?这个我不相信。



至于为什么俄军不急着速战速决,打运动战速战速决,其实我想我在 11 月、12月的时候好像有谈过,就是连乌克兰自己的总参谋长都预期会在1月底2月初俄军会有一个冬季攻势,那结果没有,没有的原因是什么呢?第一个就是我一直讲的在Bakhmut那地方打消耗战的那个伤亡对比,俄军已经很满意了,没有什么必要冒险。但是我觉得另外一个原因就是我刚刚也讲的,他目前投入的只有十几万人,不到 20 万人,他如果打全线的运动战的话,整个兵力会增加 3 倍,投入的兵力会增加 3 倍,目前他们俄军跟乌军的那个火炮的炮弹使用量的比率至少是 6: 1,但是你如果一旦投入的兵力变成 3 倍的话,你这个密度就会被稀释掉 3 倍。你的这个火力优势,局部的火力优势就从 6: 1 变成 2: 1,这是一个很大的问题,因为事实上Bakhmut的Wagner已经在抱怨他们的炮弹补给不是尽如人意,也就是说目前俄国的炮弹的后勤供给的能力是勉强可以满足目前,前线十几万人的需要,但你如果一旦是 50 万人下去的话,这个火力密度就要下降 3 倍,那这可能就得不偿失。我认为这是一个考虑之一,那因为毕竟全世界能够生产、及时生产足够 50 万人投入的,然后还维持目前俄军火力密度的只有一个国家,而那个国家到现在还不愿意公开的运送。



唐湘龙 01:13:38 

好,这个刚好是我要问你的最后一个问题,因为现在全世界,因为普丁已经讲了习近平要来看看我了,所以习近平什么时候到俄罗斯只是时间的问题。可是他到莫斯科会引起很多的想象,他到底是为莫斯科是来是添加军火的,还是就像他调停沙特跟伊朗一样,他是来扮演一个斡旋者的角色?你认为中国到了莫斯科,习近平到了莫斯科之后,俄乌战争接下去会怎么走?会有影响吗?



王孟源 01:14:16 

中国提供军火的话对俄国会是一个战术性的大大的加成。但是过去这一年,事实上我在去年3月战事一起,3月 1 号就写了一篇文章,当时我的建议就是中国当和平的白脸,然后让俄国人去当战争的黑脸。他们这样子合作,这是战略上的一个很好的姿态,所以当战略跟战术考虑有冲突的时候,我认为应该取战略优先,所以我不预期这次习近平会改变政策,而提供军火给俄国。



唐湘龙 01:14:57 

现在当然中国还没有释放出任何想要直接介入到乌的斡旋的那个味道。不过现在看起来,乌克兰从今年 1 月份的davos论坛看了起来,乌克兰希望中国进场,可是因为这场的战争,美国卷在里头,很多北约卷在里头,这不是单纯中国能够把沙特跟伊朗请到北京去谈一个北京回合就能处理的那个难度会增加很多,但是大家就在等中国的表态。



好,因为时间的关系。来,先谢谢孟源,但是我要先感谢一些的观众朋友们的donate跟支持。来 proud of 看001,感谢。



再来冯一伟,冯一伟问了问题,显然他对我王孟源很有很关注。他说上个星期他得知王博士去年7月应北大智库的邀请,写了一篇文章,叫做社会主义国家应该如何管理资本。这个题目我听了就肃然起敬。那他说这个文章还获得了习近平总书记的批示,真的吗?



王孟源 01:16:07 

是与不是我都不能够讨论。不过那篇文章在我的博客上,大家有需求。



唐湘龙 01:16:12 

我私底下问你。再来回想王博士曾在 2019 年主动接触韩国瑜团队,希望给予韩国瑜建议,结果热脸贴了冷屁股,有这回事吗?



王孟源 01:16:29 

不是热脸贴了冷屁股,而是我有两个管道去跟他们接触。



王孟源 01:16:35 

OK。



王孟源 01:16:36 

其中一个管道是无疾而终,就是我写了建议给他们,他们根本不理,的确是热脸贴。另外一个管道是反过来跟我讲,说韩国瑜是扶不起的阿斗,你还是不要浪费时间。



唐湘龙 01:16:51 

因为那个时候我知道王孟源,但是我还不直接跟王孟源互动,所以我不知道这件事情。但是这冯一伟是说,一则为中国感到可喜,一则为台湾感到可悲。他这他要讲中我的,我滑主不动了,滑鼠不会动,OK?快快快快,我看一下我滑鼠不会动可以吗? OK 不 OK 吗?现在宕机了吗?宕机了,我都划走过,唉。可以可以。唉,刚刚有动,刚才动,但现在一幕呢?一幕一幕等我一下。好,OK,来,他说电缆线没有连接。OK,现在没有没有,没法进来。好,没有,因为时间关系的是我们现在的系统,刚刚我这样子给他划的时候出了些问题。好,不管怎么讲,我还是我,我就先把节目先结束掉,我们的观众朋友的这些流化以后我有机会再回补,先跟大家 say say sorry。好,今天时间的话关系非常感谢王孟源王博士,我们透过越洋的这样的一个视讯的方式来到了龙行天下的现场。感谢孟源。好,下回聊。拜拜。



\twocolumn[\begin{@twocolumnfalse}
\section{俄国战略、巴赫穆特、中法会晤、美国金融风险、金砖货币、}
\subsection{20230421}
\end{@twocolumnfalse}]5月9日,拜谢部分省略,其余部分已完成

【我是另一位读者,已再度修正,如把KN8修正为原话的 KM-8“Gran” (前者是朝鲜洲际导弹,后者是俄国制导迫击炮)。另,记得时刻留存备份,开放文档需防止恶人小人作乱】



唐湘龙:0:00



(准备阶段):好,来。就位就位,好好准备,好表现一下啊!



欢迎来到龙行天下,今天星期五的时间,九点半到十点半的一个小时的时间,那从我开这个节目的时候,能见到龙行天下嘛?龙,当时我顾名思义,天下事。其实当时除了是我自自己的本科学生之外呢?另外,当意识到就是说,我们幸也不幸,生活在一个剧烈变动中的时代。这种的剧烈变动,就算你对国际事物再不敏感的人,我估计大概在这两年的时间,你都会感觉到的那个变动。它不仅变快了,而且变剧烈了,那每个人除了系好安全带之外,同时也应该要有一些的基本的国际事物的理解、跟分析、跟掌握的能力,作为自己判读的基础。起码知道风向是往哪里?走完了自己会被带到什么地方?因此能在龙行天下的栏目里面我,除了我个人之外了,我尽可能地找了几位其实不是太好找的朋友。那这些朋友,当然,我大概自己在工作上面的也都有的,有的不认识,比如说现在在线上的standby的王孟源。我过去完全就不认识他,但是当我在第一次看到他的视频的时候,嗯!我该怎么形容,不能说惊为天人,但是我就是很认真的听,我觉得这个人这个段数比我高。任何人的段数比我高,我都想办法能够在我的在我的这个栏目里面,能够让他们有机会。我愿意把所有我觉得最棒的人,跟大家分享。



好,那今天是王孟源的时间,好,那我们大概准备,王孟源呢大概准备了,主要他准备两大主题,一个是俄乌战争。俄乌战争,大概已经打一年了,大家都已经疲了。但是最近因为已经春暖花开了嘛,那整个的东欧呢,你看他最近啊,不管是谷物的问题啊,等等的都显示了季节转换了。大家开始再为接下去的半年做准备,对战争的是有很大的影响的。所以你看到俄罗斯的总统普丁也好,或者是乌克兰泽伦斯基也好,他们纷纷选择到前线。那去做了一些的政治表达。那当两个国家,就是交战的领导人都到了前线的时候,我们当时觉得战争似乎到了一个新阶段,是不是一个决战要开始了?好,那第二个部分的时候,我在我在听孟源里面,针对最近的国际的金融形式,因为其实这也是他的专业之一,那国际金融圈子里面,在国际的资本市场里面,他涉猎也非常久。



好,那这两块,最后如果有时间的时候,我们再来看一下,美国的朋友圈里面纷纷那出现了这种的疑美、厌美跟反美的潮流,不管是那马克龙、卢拉或者是最近墨西哥等等整个的拉美区呢那种厌美成为了它的主流品味,那为什么?美国能怎么办?对这世界会有怎么的影响?先介绍我们来宾,在我们的线上的人在美国的王孟源,欢迎。



王孟源 (3:38)

大家好,很高兴再跟大家聊聊。



唐湘龙 (3:43)

好,那孟源刚说呢,他要先跟他的博客的读者们,要先say sorry,他两个礼拜呢没有更新,为什么?



王孟源 (3:53)

最近身体有一点小毛病,所以精神不太好啊。时间最优先是给我的小孩子辅导功课,其次是自己的阅读。所以如果时间不够的话呢,我就没有去照顾博客。不过,这个身体的问题现在已经解决了,那应该就是上完这次节目以后,周末还有另外一个座谈会是学术性的,嗯,上完,下个周一开始,我就应该可以好好地照顾我的博客。



唐湘龙 (4:28)

好。其实王孟源的身体才是这个礼拜能让我觉得最焦虑的事情,上个礼拜我就在担心了,但我希望希望孟源的身体健康。不过呢,因为他说身体微恙,所以我也不想勉强他,我说那如果可以呢就可以,不可以没关系,我们Delay都无妨。但是王孟源今天上线了。好,那我们从俄乌战争开启。



我们的大概已经上个月,其实我们并没有谈到这个部分。因为它在一个准备期。上个月给我感觉就是大家在为一场的打仗,虽然在东部地在整个顿涅茨克地区,这些的绞肉机还在那继续的绞肉。那感觉上面,似乎战争在乌东地区的战争已经到了一个新阶段,可是就在这一个礼拜的时间里面,我刚提到了两国的领导人都分别到了对他们比较有利的前线。而在乌克兰方面来讲,基辅泽伦斯基也已经好几次地对外讲话,就是乌军在酝酿反攻。大家各有各的表达,但是我们现在看战场的情势,现在是个怎么样的情势?



王孟源 (5:46)

我们先回顾一下啊,我们还是从最大的战略角度,然后慢慢地去了解。那个俄国其实这一次在乌克兰有军事行动了,他最大、最基本的危险还是本身的经济与金融。所以他们从一开始就以那个为最大的重点,因为你如果自己经济不稳的话,尤其是他的货币不稳的话,才会有内部推翻普京政权的可能,那是俄国最大最大的危险;也其实也是欧美,尤其是美国的那些Neocon原本计划的主要打击手段。我想经过这一年,大家应该已经看得很清楚吧,就是俄国在这方面大获全胜,而且在外交上面能够反攻。我在这里,最新的过去这个月,又有两个新的数据出来,我跟大家分享一下:第一个是2022年的俄国的National Debt,就是他的国债,在一年里面就减了20\%,



唐湘龙 (6:59)

这很厉害啊,是非常厉害,边打仗还还敢边减国债。边打仗边减国债是多不可思议的事情?打仗国债只会增加,他边打仗边减国债。



王孟源 (7:15):

第二件事是他的inflation,就是通胀率已经掉到3.5\%了。



唐湘龙 (7:22)

这比世界上大部分的先进国家都低很多。



王孟源:(7:25)

除了中国以外,没有比它更低的。



唐湘龙:(7:28)

这也是不可思议的事,边打仗然后通膨还能往下降。光这两个数字就非常的惊悚,跟一般经验值相反。



王孟源:(7:39)

现在美国的通胀率是6\%,欧盟是7\%,而且呢,我想这个我还可以继续地往下讲,这也是为什么他最近那个卢布的汇率呢,他放松了,因为他的卢布汇率在去年开战之后两三个月猛跌之后回升,回升到比原本开战前还要高。那过去这一年,这种高汇率的结果,强势的卢布的结果就是,国内的消费非常的强劲,要进口的一大堆消费品那。这对长久的收支平衡不是很有利的,所以我想纳比乌琳娜——就是中央银行行长——觉得取舍一下,觉得既然这个在金融战跟经济战上面战败的危险已经完全消失了,就可以转过头来看长久的贸易平衡的这种细节,然后放任卢布稍微的贬值了一下。现在好像是85卢布对1美元,就是比战前稍低一点的,不过他在过去这二三十年一直都是这种缓慢贬值的趋势,是因为他们本身的消费品的产业并不是很强,所以进口的需求一直是在那里。我理解他们基本上是现在回归了和平时期的货币环境,这也是很惊奇的一件事情。就是他们根本不在乎你,美国在打击卢布了,反正我们的卢布就是这样子,现在单是消费这种小事。



那刚刚提到在外交战略上面做反击呢?我们上个月其实那个时候习近平还没有访问莫斯科,只是谈了一下,但是呢,我觉得更重要的是,后来国防部长李尚福去了,分别地去了。那这个分别去的原因,我想是因为习近平去的时候是以和平的使者的态势出现的,所以不太方便有一个国防部长跟他。后面接下来之后呢,隔了两个礼拜就轮到李尚福去谈,谈双方的军事合作。其实,虽然中俄两国没有正式的同盟关系,其实他的这个合作关系,跟彼此间的信任啊,比欧美之间要高。就是你看这次马克龙到北京去,美国还忙不迭地派冯德莱恩就是——他在欧盟的那个代理人,她全家都是美国人嘛,就他一个德国人——巴巴的跟去做监军,跟在旁边看,看你马克龙讲些什么。所以从那件事就可以看出其实欧盟是美国踩在脚底下的一个附庸,其实信任程度是很低的,那中俄呢?彼此之间的信任就非常的高。不过我必须要强调一下,当然中国除了在军事方面由俄方主导之外,因为中国必须要占领这个世界和平的这个宣传的制高点,但是双方的国力差距还是很大的。



所以呢,如果你要看这个联盟的姿态呢,其实年前有一个很相似的案例,就是韩战。因为你想一下啊,那国共内战结束之后中华人民共和国建立在1949年,然后在1950年就打起韩战,这并不是毛泽东事先准备的,而是斯大林驱使北韩突然动手,他这个动手的主要原因呢?是因为反过来,其实是因为毛泽东那个时候没有成为俄国的附庸。所以斯大林一直念念不忘想要恢复沙俄在东北的权利。那他的如意算盘是如果美国人三战打到鸭绿江边的时候呢,中共就必须要对苏共屈服,然后让苏共在东北驻军,那驻军以后当然经济利益都可以恢复。那结果没有想到,当时的共军凭着轻步兵也能够跟美军打成平手。所以打五场战役之后,斯大林对中共共军的战力刮目相看,这时候呢,才有中俄双方相对的比较平等的一个合作,但是当然苏联还是真正的工业国,中国当时并不算工业国,所以援助的方向主要是单向的,那现在是反过来,就是这个中国的大部分的科技程度发展到超过俄国的程度,而且经济也比俄国的大。所以这次普京所做的,其实很类似当年毛泽东在在韩战的时候一出手让人刮目相看,这次来也是普京的一出手。作为第三世界反抗帝国主义殖民主义霸权的尖兵的时候呢,让人刮目相看,所以能够赢得全世界第三世界国家的支持。这里面也当然包括中国,所以他其实是用军人的生命打出了国际上的尊敬,不然俄国根本在过去的这30年被人看得很扁了,常常说是一个冒充成国家的加油站,这句话是美国的政客说了好几次,对不对?所以我觉得你看过去这一年多的这个国际战略型态,从俄国的观点来看了,其实他就像1950年中共在韩战中一鸣惊人而得到共产国际的尊重的那个经验,也是同样的俄国也是一鸣惊人成为第三世界的领头人之一。当然不是最大的那个,至少平起平坐。我想在韩战之后,中国引进了很多的俄国的工业技术,那也是收益是相当大,那这次中国当然也会有很多消费性的产业跟俄国合作,我想对俄国经济未来十几20年也是会有很大的帮助的。因为至少中国对外没有什么剥削、掠夺,是真正的相信美国口头上所说的公平竞争自由市场。



这是在战略方面,那除了在中国方面,俄国在OPEC+的影响也是很明显化,因为一个礼拜前。他们很意外的消减了它们的石油产量,然后把那个石油价格维持在80块美元以上,那这当然是对大家都好,唯一不高兴的就是美国。其实美国也只是嘴巴上嘟囔一下,因为他们自己也是石油出产国、出口国,所以呢,以上经济利益也是在这边,只是不给他面子,因为拜登那个时候想着是要降通膨的时候,就口头上是希望这些汽油的价格能够稍微再降低一点,不过实际上对美国来说当然那些财团其实还是很高兴。



那我们再回顾一下,过去这一两个月的进展,就是你刚刚也提到了在那个东乌前线,基本上巴赫姆特还是一样继续地做绞肉机的战争,然后慢慢地一步一步地前进,并没有停止,就是很慢。但是还是很有效地、继续地以大概1比6到1比10的伤亡比例,俄军慢慢地在前进。那他当然是靠——我们上一集也谈过的——这个炮兵弹药的消耗量,是6比1。过去几个礼拜有一些新的数据出来,尤其是三月底的那次,美国的国防部的机密文件被泄露一大堆,被看到一些比较可靠的资料。就是这些资料,有关他们对俄国的评估呢,当然不能尽信,因为这是美国的观点啊,可能自己内部也是要讲政治正确的,那对乌克兰的数据更是可疑,但是对美国自己的评估,还有对北约自己的评估倒是蛮可信的。那这里呢,他就评估俄国的炮弹的生产量就是152或者155这种炮弹这个级别的炮弹生产量跟整个北约的对比是大概3比1。对,你这个样子你能想象到?俄方之所以有恃无恐,正因为它的军工产业,它的重工业呢,其实是对着整个北约都有压倒性的优势,更不用提光美国一个。过去这一年里,除了一开始六个礼拜是他赌博,要赌那个把乌克兰的领导阶层吓的以战逼和。结果后来到三月底,去年三月底就失败了嘛,那是他们的第一阶段。从第二阶段以来呢,他们的战争的准备,在武器啦,战术啦,上面啦,都是井井有条的,而且呢,现在回顾去看,都是始终就是准备要打持久战了,而这个原因是,虽然二战的时候,纳粹德国的闪电战非常有名,但是大家不要忘了纳粹德国使用闪电战并不只是因为他是一个很好的战术,而且也是因为它在战略上原本就资源匮乏,纳粹德国在整个战争期间,它的石油存量从来没有超过三个月。他在1941年六月开始进攻苏联的时候,石油存量只剩下两个月。很可笑的苏联当时在开战1939年到1941年之间的那两年呢,事实上是纳粹德国石油来源的一半,是找俄国买的。好的,所以他一开战以后,当然自己的那个能源供应就先消减50\%,那是现在的俄国刚好相反,他原本在资源上就有恃无恐,所以这样慢慢的用炮弹去消耗对方的生命,当然是最有效的战略。



在过去这一个月的一些新消息,最重要的新消息就是,因为乌克兰的防空导弹已经用的差不多了,苏联的防空导弹是独步天下的话,就是主要原因是因为在冷战时期美国有空优嘛,所以美国美方对防空导弹的投资当然不如苏联那么重视。所以俄国跟乌克兰这两方都继承了苏联的这些防空导弹,那也就造成过去这一年一个很奇怪的现象:就是双方在空中的战斗,永远没有地面那么激烈,然后空优的力量也远远没有大家习惯的那些美国去打第三世界国家的时候那么样的压倒性的优势,原因就是乌克兰的防空系统虽然只有俄国的1/5不到,但是仍然足以吓阻俄国对空优的有效使用。但是我刚刚提到上个月底所泄露的那个Pantagon的机密文件里面,又提到他们乌克兰的库存的苏联旧有的防空导弹到三月底就会用完。那刚好,我觉得这不是巧合,我想俄国也是事先就知道。



那这一次在过去这一年,另外一个很奇怪的事情就是俄国用了四五千枚导弹,现在用了四五千枚巡航导弹,或者是长程的自杀性无人机啊,但是呢,一直没有用美军最喜欢用的滑翔炸弹,制导的滑翔炸弹。那这个巡航导弹的航程,当然要比制导炸弹要长得多,就是一千两千公里,那滑翔炸弹嘛,一百两百公里啊。但是问题是这个滑翔炸弹要便宜很多,而且他的装弹量装药量一般是比那个巡航导弹要大:巡航导弹就是最大的,也就是装药也就是半吨左右,这种装药有半吨的炸弹,就是他所谓的2000磅炸弹,全重是2000镑,但是里面装药有400多公斤他的威力是相同的,但是它的价值只有1/10还不到。这是为什么美军喜欢用这个的原因,第一个美军有空优,第二个呢,美军很喜欢狂轰滥炸。所以俄国在过去这一年的这些巡航导弹,主要是打腹地的那些基本设施,像是供电站什么的,但是呢,在前线就没有这种靶点的。美军是很有名的,你只要任何前线的一两个士兵,即使是遇到一个狙击手,也会把空军叫来扔一个2000磅的炸弹,把整个楼给炸轰掉,很有名的事情,那你这次过去这一年就没有看到俄军做这样的事情,那原因有两个,一个是有那个当然就是我刚刚讲的乌克兰的防空导弹还没有用尽,所以他的那个这种短距离的滑翔炸弹不太好用,那其次呢,也是因为历史的原因,就是刚刚我提的这个因为苏军还有俄军都不假设自己有空优。所以他都一直没有库存这一大堆这些滑翔炸弹。但是在在几个月前他们开始估算也就是这个节骨眼上会形势有变,所以他开始大量生产了。现在的传出来的消息是那个所谓的FAB500, FAB1000,FAB3000,FAB9000都开始量产,那这些是什么样的炸弹呢?500公斤级的,当然这还是全重啊,还有1500公斤级的,目前已经有好几个报道就是1500公斤级的这个fab1500已经被使用在前线。就是乌军很喜欢驻扎在住宅区,用公寓大楼打巷战。一两发下去就整个都垮了。而且俄国的这些炸弹的装药比率比美军的还要高,就是美军的2800炸弹重量里面只有一半是炸药。俄军的这些炸弹有2/3是炸药,所以它的那个杀伤力就很不一样,更强。光是那个FAP1500的就有一吨的,就是1000公斤装药比美军的重型炸弹装药要多了一倍哟,现在我还看到他们已经正在开始量产FAB3000跟FAB9000,你可以想象这些炸弹丢下去以后,你再深的工事都没有用,都再怎么坚固的大楼都没有用。所以我想这个也是,正在发生,还没有发生的一个转变。大家可以观察一下。另外一个他们很重点的批量生产的新武器,那是所谓的KM-8“Gran”制导型的一百二十毫米的迫击炮弹。对,也是制导然后滑翔的迫击炮弹。那这个是给连排级用的,有的基本上现在基本上是俄军在Bakhmut打起来呢,有这种120毫米的炮弹,他的装药比一百五十二一百五十五的炮弹稍微少一点,但其实是同一个级别的,这是因为迫击炮的炮弹了,他的那个蛋壳可以做的比较薄,所以装药可以装得比较满。那像这种级别的你如果是排级都可以直接控制的话,你可以想象他们的那个火力一下就又上去了,而且又是制导还可以打行动中的目标。所以最近这两个月呢,乌克兰的Bakhmut的守军哀嚎遍野,抱怨的项目就包含这个新的制导滑翔的炮弹。



那另外乌克兰虽然是看起来是弱小的那方,但是他其实在那个情报啊,跟通讯方面啊,反而跟比俄军有优势,因为她一直是有着美军跟北约的支援的。那其中一个很特别的一个强处就是他的starlink,就是马斯克的星链,嗯,马斯克的那个星链。那后来断断续续有消息传出,说他们这些starlink非常有用,因为有几千个这样丢出去,每一个连都可以有一套,连排都可以有一套,所以你俄军根本没有办法阻断他们的通讯,他们随时都可以联网用美军的最先进的网络设备。我想几个月前我们谈过,他们同一个无人机所拍摄到的画面可以同时有几十个连排级的单位看到,当然是非常的方便,但是后来断断续续的有这些starlink被干扰的消息出来,那也是刚刚最近这几周才有细节出来:他们干扰的细节倒不是直接去干扰starlink,就是starlink是卫星通讯是对上直接的短波通讯、微波通讯。对,微波并不是很容易拦截,而且又是全新的,这个就是过去几年的技术嘛,所以发的那个抗干扰的能力很强。但是呢后来俄军发现了这个starlink,你要通讯之前呢,必须要先有他的那个时钟,跟必须要跟卫星先对准。那这个对准的办法呢,并不是直接的跟starlink的卫星对准,而是经过GPS的讯号。那GPS除了给你定位空间之外,一个其实大家一般人不知道的很重要用途就是精确的定时。所以那个starlink也是靠着GPS来定时的。但GPS是30年的老技术,那俄军有30年的资源投进去,准备怎么干扰都很知道,即使伊朗上一次也是拦截了美国无人机也是靠着干扰他的GPS。所以俄军就发现你只要先干扰他的GPS那乌军的starlink就没办法稳定的联网,所以现在大家,当然你如果看西方主流媒体的话,不会看到这个报道,可事实上(俄军)一直是很有耐心的把问题一个一个解决,然后改善自身的办法。



唐湘龙 :(29:57)

好,孟源我打断一下,因为乌克兰战场上面的情势啊,我觉得从媒体上面的,即使是在台湾或者西方媒体,我在看,但因为我觉得你看的材料的角度跟来源跟我不太一样,我们看的大概都一般性的西方的媒体,西方媒体的主流叙事不只是主流叙事的惯性,还包括了就是我注意这些的相关的新闻点阅率都大幅度下降,表示了大家的那种对战争的疲劳感,就是看战场当中这些的讯息已经不太有感觉了。好,但是呢,回到战场上面,我刚刚提到,就是说因为现在春暖花开了嘛,让他战场的新形势呢?看起来呢,是开始为另外一个季节准备。那之前因为泽伦斯基跟普丁的都分别到了前线,第一个这个是一个大战前的讯号吗?这个就是最近中国当然在在整个外交场合呢,表现得非常的热络,非常的优异啊,那不管是马克龙到北京或者是卢拉到北京,它们共同的话题就是和平,就是乌克兰的战争如何去化解?那马克龙回家了之后呢,立场还是很坚持,要要走欧洲的战略自主,而且抛出来信息就是说,他的总统府的外交顾问大概已经开始了,跟北京跟王毅在对口,在对接,要寻求一个中法共同地去推动的,乌克兰的终极的和平方案。这个怎么看?就说这个战争除了那战场当中,巴赫穆特一非常的缓慢,真的非常的有效,我认为那个慢是故意的,就是慢慢地推,然后呢,让你慢慢慢慢来,慢慢的消耗。可打到现在,当双边的领导人都到了前线的时候。一场真正的决战要开始了吗?



王孟源:(31:55)

其实跟决战没有太大的关系啦,这个Putin只是临时要到前线去鼓舞士气,然后,Zelenski接下来两天之后就赶快也必须去嘛。他们乌克兰的这些领导人并不是在打一个军事战,他们是在打一场战争电影,这个像五月,等到这个雨季结束,五月初就准备要做春季攻势这种。事先提早半年就一直宣传到现在,这反而是像美国好莱坞的战争大片,事先做宣传的样子。哪有人,你看孙子兵法有没有说我要出手之前半年先宣传一下?



唐湘龙:(32:42)

这个是我们看过看不懂的地方,我说好吧,就是就算你真的要反攻,你也需要敲门打鼓吗 ?而且敲锣打鼓很久了。



王孟源:(32:51)

这其实是因为美国国内的政治需要,那美国国内的政治需要,所以这个乌克兰才不得不这样执行的。那执行出来也是你刚刚说的,也是基本上是美国的政治需要,欧洲的政治需要就是为了要让他们民众看到,乌克兰有捷报,那这样才高兴。



那我上个月跟你谈的时候提到乌克兰准备了19个旅,那这次泄露出来一个机密,有更详细的描述,就是里面有九个旅是在欧洲各国训练的,都是机械化旅,现在有三个旅的在乌克兰本地训练,也是由北约派人到那边去练。那这些其中这些最精锐的就是目前捐赠的西方的,北约的先进坦克,就是像豹式坦克了,或者是英国的坦克,这个全部加起来还不够两个旅装备的,所以这个也不能够指望这些都是用最先进装备。而且听说三十几辆豹式坦克里面已经有三辆报废了,因为他那个乌克兰的士兵的水准实在太差。



唐湘龙:(34:05)

没有办法操作,他不是被击毁的,是他自己的操作撞到,搞坏了。



王孟源:(34:11)我给你举个例子啊,这是一个礼拜前在波兰他的这个训练营呢,有一辆这个豹式坦克,因为两辆撞车,结果有一辆撞到,那个掀翻的那个炮塔都掉出来了。对,就是这样一撞之下,这个炮塔崩出来了。那据说已经折损了三辆坦克,那你说这1/10的坦克。所以,而且那个美军的泄密也提到,他们原本计划要做5到7个礼拜的训练,现在变成2到3个礼拜的训练。就是基本上这些都是临时征兵出来动员出来的,他们的素质非常的差。基本上是没有经验,没有能力。所以我上个月已经讲过这场春季攻势,就是五月的春季攻势,我非常的不看好。不过,无论如何你在前线,光是在Bakhmut周边五六十公里,就有20万的,不对,八万的那个乌克兰部队。



唐湘龙:(35:31)

而且是他的主力精锐军队部队。



王孟源:(35:33)

而且是他的主力精华,已经调到前线,准备要出手了。那所以这也是另外一个原因,他们为什么在Bakhmut打的步步为营。因为他们,如果是乌克兰机械化部队一下冲出来的话,那一定是钳形攻势,就是这样两翼的去把它这个突出部包围起来。所以呢,你必须要花点时间在两翼做好防御的准备。那这个局面呢,其实跟1943年打的Operation Citadel (编注,堡垒行动,也就是库尔斯克会战)非常的相近。大家有兴趣的话可以去看一下,Operation Citadel ,这是德军在当时最后的一次大型的反攻,在那之后呢,就是摧枯拉朽,没有办法抵挡苏军的逐步的进逼。我想俄国的大战略在军事方面的战略呢,也是这样,等着乌克兰在五月的这次春季攻势呢,把它最后的一点预备队消耗掉。再把他们去年动员的然后训练了六七个月的那些动员兵全部统统(消耗掉)。因为我提过,他们现在至少还有30万藏在后面,没有真正的出手,前面打的真的是大概20万左右。就是20万对20万。那现在这30万一出去就可以长驱直入,但他不愿意在对方还有预备队的情况下,让这些部队去消耗。所以事实上,我们下个月跟下下个月大概可能需要一两个月的时间,打出个名堂来嘛,很可能1943年是80年前的那一场Operation Citadel的重演。就是希特勒准备了一大堆虎式跟豹式坦克,结果落入了苏军安排好的陷阱,有各种各样的地雷啊,然后还反坦克炮。那先消耗了德军的锋锐之后呢,等到苏军反过来反攻的时候,从那之后两年就是德军完全没有招架之力,都是完全挨打的态势。我想很可能这一次春季攻势,在六月七月之后呢,也会转成这个形势。所以,现在有人说这个俄军准备一直打到2030年,那这些人,我觉得有点……



唐湘龙:(38:10)

这个听听就好。你觉得,因为要讲到讲到乌乌克兰,我刚刚讲就是说这次马克龙谈的,就是说战略自主,慢慢地,在欧欧盟体系里面,老欧洲里面得到了一些的正面的回应,看起来他并不孤单。那所释放出来的就是说这个战略自主,同时对乌克兰呢,欧盟,至少法国似乎要提出一个他自己能够代表法国的,想要去解决乌克兰问题,简单讲就是法国马克龙这次到北京,我觉得被欧洲释放的一个很重要的讯号,就是说,欧洲或者法国已经不希望乌克兰战争继续打下去了,那有调解的可能吗?你认为法国的这个战略自主能发生作用吗?



王孟源:(38:57)

Macron在六年前刚刚上任的时候,就已经讲欧盟要自主,要成为世界多极世界的一极。他讲到现在,这个并不是新闻,那Putin跟他会面,有好几次重要的会面,讲的也都是这一些。那过去这一年,其实他在当经济部长的时候就亲手地比过,当年美国GE于吃下Alstom的那件事情,所以他对美国人占他们的便宜经验之多,我想不需要别人来提醒,而且,那个AUKAS潜艇案,他也是又吃了一次亏,是吃了一次Anglo-Saxon的亏,所以这一次他来,北京本来中方是准备跟他再开导一次。就是以前已经开导了十几次了,就Putin跟习近平都开导过他好几次,那结果美国也是得到风声,就赶快地让冯德莱恩跟着来当监军,对不对?结果中方当然没有给她好脸色看。对,宣布这个冯德莱恩跟着来的时候,他是说啊!对Macron说是应习主席的邀请,法国总统要来访问,然后对冯德莱恩的内容只是说,根据中国跟欧盟沟通,双方沟通的结果。你看看一下这个用词,外交用词。



唐湘龙:(40:41)

没有错,就是我没有要邀请她来,但是她一定要来,那我也不好拒绝她。好吧,那就让她来、大概就这个意思吧。



王孟源:(40:52)

对,然后等到冯德莱恩要走的时候呢,还羞辱了她一场,我想你们这些看西方媒体的大概没看到。就是一般这种外交的人员呢,都是会有通关的便利(礼遇),更不用说这种欧盟的领导人之一,照理说是应该是直接就通关,然后先坐上First Class。但是呢,冯德莱恩在离开中国的时候呢,中方的机场管理人员,特别不让她走外交通关。她在那边排队,被人家拍下照片的,她是循着一般的商务客,是自己搭飞机去的,商务客的安检通关。这是一个很大的羞辱,因为通关的时候是要什么脱鞋,然后照X光这些事情的。所以然后,当然,马克龙就被安排送到广州去访问。然后来访的欧洲总统或者总理,到另外一个城市去访问,这是很正常没错,但是不正常的是,习近平也跟着去了,那么又会谈。



唐湘龙:(42:13)

对,这个是的,最特别的就是马克龙到广州,我不意外,我说写习近平也特别下来了,这个就很特别。你讲到重点了,这我才对最近的中法关系呀,我觉得马克龙这次并不是讲假的,虽然你刚刚就是说,战略自主是法国的传统,它作为一个法国中的当选中一定讲战略自主,他也讲过,北约已经脑死类似的话,讲过很多,可是他这次讲终究是在一个俄乌战争发生之后讲。而且是在美国在拉他的盟友呢,要围堵中国的时候讲。意义当然还是不一样,但讲完之后怎么做?所以你刚讲就是中方对马克龙的款待,当然是很积极。也很周到,那马克龙似乎也吃了称砣铁了心,确实想要走出一些欧盟的或者代表法国的新兴局面。好,那既然谈到了,我们就顺便讲讲讲究这段冯德莱恩这次回去了之后,在欧洲议会里面刚开始辩论中国政策的时候。她的那个开场白的讲话就是基本上就是以一个欧盟的负责人的角度来讲,到欧洲议会去发表施政报告,虽然在她在中国呢,没有受到礼遇,可是她回去了谈话的调子还是很特别的调子。所以欧盟跟中国的关系会有改变吗?



王孟源:(43:33)

我觉得不会,我觉得短期内不会。就是因为Macron这种转变啊,他被人提醒一次以后,他就会说一次这类的话,当然这一次的中方特别地,我想也更加深入地去开导了他。所以他的感触也特别深,尤其过去这一年多俄乌战争,欧盟吃亏吃得之大,我相信他也都是心知肚明的。在金融、军工跟能源这三方面,欧洲损失是上万亿美元的。更不用说欧元跟着美元,一样被弃用,这种事情,这种潜在的损失更是超过几万亿。但是我想这一次,中方跟他坐下来就是指明了这些事情,然后呢,指出只有跟中国合作才是真正双方互利共赢的,而不是单向的,让美国吸血。



但是问题是,欧洲的政坛,还有它的这个多党选举制呢,已经让美国的、亲美的势力彻底的掌控了。这次搞出这么大的娄子,但是德国在绿党的支持率有没有下去?没有下去。芬兰的那个总理,就是那个女总理,她虽然选举失利,但是她的投票率其实比上次选举还要高的,他所得票率,就是上次是18\%,这次20\%。不是,刚好她排名第三,是因为另外有两个党都也得了20几个\%。20比21比21,所以它多一点点而已啦,都比她多1.5\%。她并不是他们被欧洲人民抛弃了,而是欧洲的人民在民生受到打击之下,有不满,但并不是针对性的,因为他们的有媒体还是完全封锁起来。真正能懂这些幕后的细节,真正能懂他为什么会吃亏的原因的,还是只是少部分,就是百分之十几二十几的人,当然百分之十几二十几的人呢?如果机缘巧合是可以出手影响,像次我刚刚谈了好几次的那个美国泄密的那个年轻人,Jack Teixeira。他21岁,没有上过大学,一个高中毕业生,我相信也是因为他看了一些,这一类的非主流的报道,理解了这一次俄乌战争的真相,所以他心有不满,所以才会泄露出来。当然他是一个很年轻很幼稚的人,所以没有小心地遮掩自己的行迹,一下就被抓出来了。但是呢,他之所以有能力接触到这些最高级的资料,而且不只是国防部还有CIA,跟国务院的资料最高机密,是因为他刚好是在美国空军一个恶名昭彰,臭名远播的一个极机密的单位,叫做102情报大队。他本身有所谓的Top Secret Sensitive Compartmented  Information Clearance,这是比绝对机密还要再高一级的。



唐湘龙:(47:28)

不可思议啊。



王孟源:(47:28)

不可思议,就是因为它毕竟有很多这种杂事需要士兵去做,所以因为他是纯正的美国人,完全没有跟外国的接触,所以当然就在预先清查的时候没有问题。但是,他去当兵,应该是18岁高中毕业就去当兵,那时候还傻乎乎的,美国的那个教科书说什么他就相信你了。但是呢,过去这三年,我相信他也是慢慢的接触到一些非主流的新闻,然后知道帝国主义和殖民主义的这些真相以后呢,开始有了怀疑。他也不是故意要叛变,他这些消息拿出来不是要拿给WikiLeak,或者俄国的情报单位。而是他在网络上交给几个朋友,然后这些朋友再转发的。所以他只是心里有不满,然后又很幼稚,没有想到这些事情有多么严重,所以才露出来的。我为什么提这件事呢?因为这个事,你说美国的主流媒体或者是英美的盎萨的媒体,还有德国的媒体,他可以控制多数民众。在选票上面,尤其是你如果是政坛分裂的多党制现象,你即使控制百分之四十,百分之三十的选票就已经足够主宰他的政策。那你现在看到的觉醒的民众还是少数,即使它是多数也不能够有把握说会扭转政策,所以我觉得这个Macron本身,他的政基就很不稳。比如说他现在的的党已经是少数党,在国会是少数党,所以你要指望他一下子在外交上面转向,我觉得,我个人不是太乐观。因为,第一个德国就不跟他合作,法国即使本身觉醒,没有德国的合作的话呢,也是孤掌难鸣。对这件事情,我觉得中方做得很漂亮,但是呢,这个成果我觉得是会有限的,Macron大概不会真心真意的去跟英美合作,但是呢,你要指望他把整个欧盟的这个方向盘抢来,这个可能是做不到。



唐湘龙:(50:11)

要让欧盟挣脱美国的掌握,挣脱美国的这个魔掌是很困难的、尤其现在的冯德莱恩的政治态度,虽然她在欧洲议会的这一方的讲话,相对于过去的讲话的调子有非常大的不同。但是也先不要太乐观,因为那个结构本身,尤其欧洲议会本身的反中的力量还是非常非常强大的。那这个这么强大的力量,我一直强调当中欧的全面投资协定没有重启审查以前,对于那中欧的关系不要过度的乐观。



我们来准备回到了孟源谈的下一题,就是美国的经济。美国的经济当然是一个很复杂的问题,这个庞然大物。但是我们从两个视角来看,一个就是虽然算中美关系不好,不管是那气球事件也好,不管是台海关系也好,但是美国最少一直在做一件事情,不管是拜登不管是布林肯不管是耶伦,不管是商务部长的这个雷蒙多都一样,他们不断地敲门,就说呢,他们要要进中国,要跟中国谈。那看起来这个里面有很强烈部分呢,是因为美国意识到了自己的经济的危机跟金融的危机。不过这个月,大概就这两三个礼拜的时间,我觉得有件事情在我们的国际谈论的这个舆论的主流市场当中,它不见了,好像那件事情就消失了。就是上个月我们都还在担心美国的金融体系的连锁性的倒闭事件,我们都觉得美国的金融是不是出问题了,是不是另外一个金融风暴来了?那从这个SVB开始,大家就开始紧张诶,可是呢,这大概三个礼拜左右的时间,这件事情好像就是没有,大家就觉得诶那件事情就已经过了。美国的金融危机,美国的经济的危机过了吗?它现在接下去,他到底想干什么?



王孟源:(52:08)

我当时,我想如果是我的读者跟听众,应该会记得,我当时我就说这个没有连锁反应的危险。



唐湘龙:(52:18)

对上个月的时候孟源特别讲。



王孟源:(52:22)

对我这个当时我就讲了,所以没有什么意外的。这原因是SVB是一个特例,它本身是一个小银行。它本身的体量不足以威胁到整个美国的经济跟金融体系,而美国经济跟金融体系呢,当然问题是很深很大。但是呢,因为过去这几年,先有五万亿美元的放水。然后这些储蓄,现金储蓄还没有用尽。然后呢,在过去这一年,又有俄乌战争,使得他能够尽情地去对他的同盟国,就是欧盟跟英国、跟日本、跟韩国、做吸血,大量的大量的吸血,把通胀压力转嫁出去。那所以美国的经济态势其实是比这些其他的西方国家要好得多的,所以根本SVB这种事情不可能有连锁反应,他不可能有。这我一两个月前就已经讲得很清楚了。事先就已经讲清楚,没有必要事后再回顾。那我们刚刚提到,像是日本,那刚刚我看到日本的最新的数据。日本在他的1990年出了问题之后呢,其后的30年,他的工资的涨幅是多少?4\%。不是一年涨4\%啊,而是30年长4\%啊。



唐湘龙:(53:57)

日本现在年轻人上班族非常的辛苦。



王孟源:(54:01)

他的那个PPPGDP,已经比台湾还低了,就是他的平均生活水准比台湾还低。这是很惨的事情。然后,尤其现在他最后的一个全球性的主要工业,就是汽车呢,刚好现在被中国的电动车企业颠覆,这也是我讲了两年的事。所以它的经济的前景只有越来越差,我现在只是特别把它挑出来,谈他的这个金融货币上的问题。就是他们过去这一年的,薪资涨幅有3.7,就是过去这一年的涨幅,就跟之前30年的涨幅差不多。OK,当然这个原因是它的通货膨胀的,实际上是4.4点多,将近百分之5。那所以,过去,我想你也知道,在所有的西方工业化国家里面,日本的国债还是远远最高的,光他们中央政府的正式的国债就有GDP的将近250\%。这比希腊在最糟糕的时候还多出一半。OK,那这个原因当然是因为他们这个日元本身呢,也算是国际储备货币之一,当然是一个次要的国际储备货币。所以他借钱的,主要可以用日元来借,而不是用美元来借。所以啊,那希腊没有自己的货币,那时候是用欧元来借的,就很容易出问题。那但是你现在这个这样负债累累的,长期来看,只有用通货膨胀,把这些欠了一一屁股债呢?统统把它用膨胀这样子相对的缩小下来,所以呢?过去30年,他们是压榨中产阶级劳工的,除了时间跟经历,那未来的这十年呢,他的这个会进入下一个阶段,就是所谓的Stagflation(滞胀),同时有很大的通胀,然后经济又停滞不前。但是这样的结果呢?就是不但你过去30年,这样子自己投入的时间跟工作没有回报,你原本的储蓄也会很快的缩水。所以你看这个有多惨。那要等到日本,然后接下来是英国,然后接下来是欧盟,就是德国,也都被吸干了以后呢,美国才会真正面临类似的危险,所以短期内没有什么可以担心。这也是我讲了一年多的事情,这里再重复一下。当然最近有一些新的数据可以跟大家分享啊,就是比如说这个他的企业债券。他的企业债券因为美联储现在利率调到5\%了,在过去这两年,他的企业债券的利率也提升了四个Percent,那你说这个企业会不会有问题呢?没有!因为这我也是讲好多次了,他们在三年前跟两年前的时候借了一大堆的钱,所以手上现金多的很。最高的时候呢,有资产的7\% ~ 8\%。现在呢,只掉到4.5\%,这个要掉到2\%~3\%才会有点危险嘛,才会有那个现金流的危险。所以基本上他再继续地撑上一年半载,没有什么问题。



那另外一个危险的地方,比如说他们Commercial Real-estate。就是2008年的时候出问题的时候是房贷,那个是所谓的Residential Real-estate。那现在这一次出问题的是Commercial Real-estate,就是办公大楼。为什么?因为因为新冠的关系,所以他们一都不到办公室,对,所以空置的办公大楼一大堆。至少绝大多数的办公大楼的价值都腰斩,有些下降了80\%。所以你可以想象,这也是一个潜在的问题,可是会不会爆发呢?没有。没有要马上爆发的意思。这里的原因是有两个,第一个这个它相对的Residential Real-estate要小很多。就是它的整个的负债只有5万亿。这个只有好像只有Residential 的1/3左右,所以它基本上是可以处理的。然后另外一个是真正投资,Real-estate,也同时会投资到农场就是Farms。所谓的地产的第三类啊,其实两三年前Bill Gates开始把资产转移到农场的时候,我就已经评论过了,这是因为他看到未来的通货膨胀会很严重,所以先去购买实质的资产。所以你看这些,我提的这两个都是一个是企业的债券,另外一个Commercial Real-estate,商业的地产,这些都是目前美国的企业,美国的经济问题比较大的区域。但是,都没有能够把整个国家搞搞垮的那个体量。



另外一个问题的是,我们刚刚提到,就是SVB是在Silicon Valley嘛。那这个,当然因为风投业是加州最主要的金融产业。所以过去这一年,SVB只是因为风投业的收缩,所以才爆出来的问题,所以风投业本身也有问题。你如果去看头条的新闻的话,他会说,他的所谓的Big Tech,就是大型的科技企业呢?在过去这半年多将近一年嘛,裁员了将近30万人,看起来很惊人是吧?但是你如果在研究的深一点,你会发现在那之前的两年多,他们雇了60万。



唐湘龙:(1:01:13)

没错,就是过度聘雇吧!他怎么过度聘雇就把它吐回去而已。



王孟源:(1:01:18)

对,这只是吐出一半而已,所以你这个根本就没有造成体系危险的那个程度。然后呢,美联储现在还游刃有余嘛。因为他过去一年的Tightening,就是从去年一月开始的Tightening,他只Tighten了6000万,啊,6000亿美元。就是他全部的那个印钞,在过去这20年印了八万多亿,将近九万亿,那过去这三年就印了五万亿。但是呢,经过一年的回收只回收了6000。然后呢,在SVB出问题以后,马上就又出手3000亿,又回吐了一半去救SVB了,那这个救的细节,我想已经十点半了那就不要再提了。本来准备好这些细节,准备要讲一些专业的金融工具啊,不过算了啊!总之是他们还是有很多余地的,所以我说你们大家不要心急,至少要等到八月九月,那个金砖货币出台,然后他们的那个国债上限解决了,然后一次发行一万亿的国债,才会可能有稍微有点气喘的情形。但是我也不预期它整个就垮掉,就是基本上现在完全不紧张,届时那可能会稍微有点紧张,



唐湘龙:(1:02:49)

五分钟的时间。因为刚好是是我要问的这个问题,就是在过去一年多里面呢,第一个跟我谈到就是要特别注意呢,今年八九月的金砖峰会,因为金砖货币大概会出来。好,那我记得孟源在讲的时候呢,其实整个的国际市场当中来讲,对这件事情,普遍是缺乏讨论和思考的。那甚至于那个味道呢,也没有像孟源讲的这么的笃定。不过最近我们会看到,除了就是说呢,跟人民币之间的本币交易呢,越来越普遍。再来当然关键的就是巴西的卢拉,巴西的卢拉这次去不只是中国跟巴西的关系,而且你看卢拉是到北京跟上海,马克龙是到北京跟广州。卢拉从上海进来,当然有道理的,除了去看华为等等之外,他是带着罗塞夫到金砖的新银行去就任的。那卢拉有关于这个,就是说金砖体系,看起来他对金砖的情感的投入是非常非常深的。因为这个背景,所以我们大概可以预期呢,今年的金砖峰会大概真的是有戏。好,那您讲的,就是说这个金砖货币以及现在人民币的这个本币交易的普及化。你能够更具体地讲,这金砖货币到底是什么?它未来对金融体系会有什么影响?



王孟源:(1:04:23)

哦,这个我过去两年,我的博客上有详细的讨论,就是我建议了一个很特定的方案。当然,实际上是要多方妥协,即使是只有中俄主导的话呢,双方的妥协也不是一件容易的事情。毕竟,比如说他们在合作大飞机跟重型直升飞机的事情上,都是双方一直争吵的,然后没有真正很顺利的谈。那是因为双方都有所谓Hard Bargain的传统,就是这双方的这个谈判人员都是老奸巨猾啊,所以很难谈到一起去。目前能够确定的事情,是美国对外的搜刮呢,只限于他的所谓的盟友的,就是所谓的Golden Billion。那你这些比如说中国跟巴西的这些外汇互换,然后支付改用自己的货币。



其实这些都还是次要的,真正最重要的一个消息,比如我举一个例子:就是你如果去看过去这三年,这些海湾的这些石油出口国,就是当然以沙乌地阿拉伯国跟阿拉伯联合大公国是最主要的,这些所谓的Gulf State还包括科威特啦,卡塔尔这些啊!他们过去这一年,2022年,他们的进出口的贸易平衡的是他们有3500亿的盈余;然后今年预期会有3000亿。就是这6500亿,过去这两年6500亿美金,想想这很大的一笔钱。在2021年之前,大部分会变成外汇,然后储备美元。但是呢,你如果仔细去看过去这一年半,他们这几千亿美元是怎么处理的?基本他们的外汇储备没有变。这些钱都进了哪里?这些钱进了他们的所谓的Sovereign wealth fund。就是国家财富基金,而且这些主权基金投资的对象不是美国,他对欧盟甚至还抽了,他把它卖掉。你看,上个月我谈到那个Credit Swiss要倒闭的时候,原本也是向沙乌地阿拉伯求救,结果沙乌地一口回绝了。为什么?就是他们有一个政策性的转变,一年多之前就是俄乌战争开打之后,他有一个政策性的转变。他们的投资对象变成在亚洲跟非洲,尤其是中国,而且那收购的是实质资产,尤其是公司啦,还有矿这些东西。那这样一来就,他原本所谓石油美元是你赚到的这几千亿,每年赚到的这几千亿,照理说应该是回流成为外汇储备,然后去买美国国债才对,对不对?所以像这种事情,他最大的影响就是美国的用美元的这个潮汐的作用,这样出去一次,然后吸回来一次,来回地来吸血,对第三世界国家没有用。现在的问题是,他们现在所持有的外汇储备呢,不再扩张,就是放在那里,然后大家都拼命的去购买实体资产。连Bill Gates都事先知道要买农场,他买了好像是几千万亩的农场,对不对?



所以,我不知道这个金砖货币在八月会谈出是什么样的细节,因为这个是绝对会保密的。那理论上我有一个建议,但是这只也是好几个可能的组合的其中一个啊,那实际上你必须要考虑外交现实。那但是呢,你可以确定的是,在过去这一年俄乌战争的一个最重要的影响,就是第三世界国家不再愿意接受美元的搜刮。而且这个是有确实的证据嗯,可以显示出来,你只要知道去哪里找,都可以找得出来。像是我刚刚讲的,海湾国家的这个外汇跟那个收支。你去年三千五百亿,结果没有在外汇出现,这很奇怪的事情。你这个去看一下就知道是怎么回事。



唐湘龙:(1:09:30)

好,我想孟源的意思,就告诉大家说在二战从这个Bretton Wood位置之后呢,美元取代了英镑,成为了国际金融浪潮当中的主流浪潮。那他是一个个大的浪头,那现在这个浪头的力道呢,是有点减弱的,但是浪头的方向要改变了,短时间之内的是不会的。所以对于就说呢,这些的浪头的这些的强弱的改变,大家当然都有点警觉性。我觉得现在全世界对美元的这种的这种怀疑,或者说减量,或者说呢,在持有的方式上面的改变。他基本上都是各个国家的一个避险动作的一部分,但是呢,要现在呢,要找到一个取代美元,我也不认为人民币的想要去取代美元,而且想取代呢也还不可能,所以呢,大家在看待这个趋势的时候,我觉得避险的警觉性是要有的,但是呢,预期就是说这些中美之间大的形式结构,短期间之内呢,会有戏剧性的转变,他仍然是一个非常长期的课题。



好,今天的时间的关系啊,其实每次孟源的大概都会被我多熬个十分钟,我觉得要不然话讲不完。虽然他还有很多的内容准备了,我们还没有谈到的,我们就下回聊,你说



王孟源:(1:10:45)

因为这一次精神不好,我怕自己准备的不充分,这次准备的特别多,所以刚刚讲的其实只是准备内容的一半。



唐湘龙:(1:10:58)

那个其实这个是我最在乎。不过我还是非常非常的感谢,因为反正孟源知道,每次你在讲课的时候呢,我都听得很专心的。虽然孟源今天感觉上面,就是精神啊,体力啊,声音啊,确实比过去稍微弱一点点。好不过我还是要在这种情况下面来讲呢,仍然提供了一个小时非常的精彩的内容。



拜谢内容省略:



“



好我代孟源感谢我们一些的观众朋友的刘华那不让看看001,感谢那等被k吗?谢谢谢评论感谢,然后同意中中吧,是不是好在一个香港感谢然后呢nonono?那诺贝尔说呢的自制豉香楼还有呢?王博士还有呢本娜娜的胡感谢,因为这个屋,艾瑞说王王博士找湘隆找我有一台海战争不可避免,只是时间早晚的问题,而欧洲被白左思讲长期渗透,再加上媒体长期妖魔化中国,它们早就没有独立思考的能力。就算马克龙有所觉悟,但未来的台海战争呢?对欧洲保持中立呢?也不是很乐观,那欧洲呢?大概率呢,宁可牺牲自己的经济力,也要参与经济制裁。好再来呢汤孙感谢然后呢?军人感谢令令令张感谢他都非常高兴,王孟源先生具有前瞻性的预测跟建议,那得到了采纳应用,那也希望呢孟源呢?多多保重身体,错过了三起了直播赶快的先岛内呢,惟惟敬。打算催他说今天在家办公,总上了该总算赶上了一次的直播,那王博士宁的那客观呢链接呢,可不可以麻烦呢?小编呢?分了博博客的链接,麻烦小编分享一下,好我我带回来给我们的小兵讲一声那把链接呢提供给大家方便你去搜寻了孟源的博客。啊再来呢斯蒂夫王感谢他,真希望能孟源呢王孟源能够每周每周或者每两周呢,能够能上一次课,一个月一次呢太久了跟不上呢,十日14,不过你不要为我为难他了我我我我我为大家想反应的我都已经反应过了,不不过。二个人的生活的调的调性,大概也很难改变了,有的感觉好不能理解,谢谢谢谢,他说啊龙龙天下还是只有那蓝军教父能配得上难得公司一株没有啦没有没有没有向你讲讲价九零天里谢谢华夏东升感谢谢谢,都是用的特别来感谢孟源的宇宙真理感谢,他说那人类靠前人的智慧累积,慢慢成长,懂得善用呢,随随便便,别让僵化的。于是呢,捆绑了自己哈,说的很好,范志林感谢他说我关注王孟源不是很久了。他说给李小龙点个赞,知道呢,找好的佳嘉宾对啊,就是对了,我我完全同意你的你的你的讲法,所以访访问孟源的时候呢,我就尽量的少说话,让孟源多多说一点,然后他煮碗一感谢她说呢,像主持人和来宾的感谢呢,非常棒。再来的范志林,谢谢,,杰森斯密和感谢,然后宇宙真理,谢谢,然后财政为基来的时候呢,想知道台湾和日本,韩国相比,那美国呢会先牺牲谁?这个你猜吧,反正反正不会是日本本就就是了,那接下去,感谢美国通膨压力好大,那要不要买买黄金好?

”

唐湘龙:(1:14:25)



这个问题我问孟源好了,就是说通膨压力很大,你觉得需要买黄金吗?



王孟源:(1:14:29)

我觉得通……你如果问我这个通膨压力会不会很快消失,答案是绝对不会。那但是呢,我不会做投资上的建议。



拜谢内容省略:



你好聪明啊,聪明聪明,还用孟源...孟源过去在在在金融体系里面呢呆很久了,我猜想在投资界面对他来讲是很谨慎的,那gary唔感谢在新加坡好那感谢那所有的所有的观众朋友呢也我想今天听听孟源讲话呢,虽然孟源今天的。今天的体力稍微的差一点点好不过呢,你要狠狠努力的提供了精彩的一个小时的节目的内容,好感谢孟源也带带大家的所有的我们的观众朋友的祝福你,下个月的时候我相信了孟塬呢,应该呢,又又是生龙活虎的,那我们就下周一见,感谢孟源,感谢。谢谢大家了,感谢感谢收看今天的龙行天下,周末快乐,下回见,拜拜。大大。



拜谢内容省略:\twocolumn[\begin{@twocolumnfalse}
\section{春季攻势、瓦格纳、七年战争、国军}
\subsection{20230519}
\end{@twocolumnfalse}]唐湘龙 00:13 

欢迎来到龙行天下,我是唐湘龙。今天星期五的时间已经不是 9 点半,已经 9: 33 分了,但是迟到跟王孟源是没有关系,是我的问题,是我们的问题。不是迟到龙,是迟到编,哈哈哈,推给小编好了,来,不开玩笑了。周末的时间。



当然这个这个是在观点平台在星期五的时段,就是说固定的直播当中的最后一档,那这档的节目我已经持续性的、就是说那个操作模式,大家很清楚,我邀了几位非常棒的朋友,那他们更有他们自己的专业的背景。那这些的朋友,其实因为我长时间在跟他们互动的时候,我也关注他们非常非常久,我自己当粉当很久,然后才冒昧的请他们能够在这个龙行天下的单元里面,那能够把一个小时间就是割出来给他们,然后在我们的对话的过程当中,其实谈不上对话了,主要就我简单地我们设定一些议题的之后,能够让(他们)充分的发挥,能够让这个单元的我们的观众朋友们能够有最大的收获。



唐湘龙 01:24 

好,那今天一个月的第三个星期五的时间,通常在这一天表定是王孟源的时间,来我们透过越洋的连线,在画面上面你看到的是人在美东的,其实跟我们相差 12 个小时王孟源,王博士。



孟源,早安。



王孟源 01:43 

大家好,很荣幸再上你的节目。



唐湘龙 01:45 

好,当然在孟源那个地方是晚安的时间。因为当这个礼拜这个月我们设定的主要的议题几个,但是一开始我们一直先从俄乌战争开始,因为俄乌战争这个礼拜,从上个星期开始,看起来是有一些的变化的,特别在乌东,那这个变化主要在巴赫穆特。这一波的反攻跟过去不一样,乌克兰这个反攻赶在 G7开会前,G7开会的那个时间点似乎也颇为巧妙。同时G7开会前的时候,你会感觉到美国、法国、德国这些大国似乎都公开的催促乌克兰说反攻,反攻,打啊,你为什么都没有打啊?那乌克兰是不是就开始反攻了?上一次的反攻我们还记得他对哈尔科夫州动手了,但哈尔科夫州我后来刚刚听了孟源讲座才知道,其实那个地方本来就不是俄军的部署的主力,所以上一次的反攻看起来很顺利,好像收复了大片的失土,可是其实那不是俄军的主力,但是这一次的如果这叫做反攻的话,它却是在现在战况最焦灼的绞肉机,在巴赫穆特的这个区域所进行的反攻。



唐湘龙 02:56 

那这个针对一个被包围得几乎已经快快被困死的一个位置上所进行的反攻,意义是什么?但最重要重点是说,这波的反攻实质上面来讲会改变了俄乌的局势吗?为什么是在这个时刻?



王孟源 03:13 

好,我这次想要多层次的跟大家聊一聊过去这个月最新的发展,所以先从战术上的一些细节的新消息讲起,然后再慢慢的提高层面讲到战略上面的一些事情,甚至会到大战略跟历史上做比较,所以请大家稍安勿躁,会讲……



唐湘龙 03:37 

慢慢来,没关系。



王孟源 03:40 

那首先这个月的 9 号是欧战的胜利日,所以每年都有莫斯科的阅兵,今年的莫斯科阅兵是很低调的,只有一辆T34坦克,那是二战时期的功勋坦克,那当然德国的国家媒体,他们的国媒也就是德国之声 DW。,然后马上评论的时候就说,这很显然是俄军的坦克已经都打没了,所以只有一辆坦克能够出来阅兵。



王孟源 04:17 

这当然是很可笑的,因为我之所以提这个,是因为要提醒大家,西方的主流媒体不只是美国的,就是欧洲的那些主流媒体也仍然是生活在梦幻世界之中,就是,自说自话,他们是先定了结论,然后再去想办法找证据来填充自己的论点。



,那如果这次俄罗斯俄军是派了几百辆坦克去阅兵的话,我相信他一定也是振振有词的,说他们这些坦克不务正业,不去打仗,然后打肿脸充胖子来阅兵,那你不管怎么做,他们都是有说辞。



事实上,我们如果回顾一年,将近一年前,大概八九个月前,就是去年秋天的时候,就已经有俄军把 T62 坦克拿出来,,然后过去两三个月也有少量的 T54 坦克上了前线。那这些都是冷战初期的坦克,事实上的确是落伍了。



那我在八九个月前我曾经评论说这个,事实上你做后方的二线防御,根本用不上最新的坦克,事实上乌军根本没有几辆坦克剩下来了,也不需要打坦克大决战,所以你不需要最先进的坦克。后来我想一想,还有另外一个原因,会把这些老的坦克拉出来。这倒不是因为俄军把他的最新锐的 T90 跟 T72 都用完了,而是你看看俄军现在打仗的这个模式,完全就是火力覆盖的模式,就是以 6: 1 的优势来打。那在这么个形势下,即使它的产量比整个北约比起来有 3: 1 的优势,它的消耗还是比较很快的。



王孟源 06:20 

所以这个 T62 跟 T54 的好处在哪里?他用的是不同口径的弹药,就是他用的是那些囤积了几十年的不同口径,不同规格的弹药,所以这样补给上就没有问题,不会去抢弹药。就是说的详细一点, T72 跟 T90 用的是最新的 125 毫米的主炮,那 T62 用的是 115 的,然后 T 54 用的是 100 毫米的,这些弹药因为是库存就是专门为了这些老旧坦克还留下来的,刚好用这个机会把它打光。打完就算了



唐湘龙 07:02 

生产线就给关闭了。



王孟源 07:04 

对,生产线老早就关闭了,这些都是几十年前堆下来的东西,所以大家看到这种消息,这并不代表俄军的后勤有什么问题,它纯粹就是这个废物利用的考虑,大家不要过度的解读。那西方主流媒体所讲的,您反正知道他们是专业的撒谎者,看的时候提起警惕就好了。那在5月 9 号阅兵之后没有几天,俄军一连的开始攻击,用长程的火力,就是像巡航导弹还有无人机去打击乌军的军火库,刚刚讲到的军火供应。那其中有两起是非常的成功的,就是他引发了连锁爆炸。在那当地的那个城市有很多视频传出来,那个爆炸的规模非常的大。



唐湘龙

没错,



王孟源

那其中有一个传说是有英国所提供的 depleted Uranium,就是贫铀弹存在里面,这基本上就是从这从作业上的角度来看,就是俄军知道这个春季攻势要开始了,所以先釜底抽薪把你的军火打掉。



王孟源 08:31 

那打掉这些军火部刚好都是在乌克兰西部接收北约的武器跟弹药的仓库,所以打掉的就包含英国特供的那些贫铀弹。那然后就有各式各样的风闻传出来。有说什么当地乌克兰去清理的时候必须要动用机器人,而不是派人进去,然后也有怎么报道,说什么当地的伽马射线提升了,或者是测到了放射性残余元素,当然目前的这个消息还是很粗糙,然后众说纷纭,我们没办法确定事实的真相。不过我在这里大胆跟大家做了个假设,就是我认为最可能的事实真相,就是的确是打掉了贫铀的弹药。



唐湘龙 09:34 

因为英国已经证实它有提供乌克兰贫铀弹。



王孟源 09:39 

对对,然后乌方的谨慎也是因为有贫铀弹,但是事后的传出来这些小道消息,什么测量到放射性上升等等,这是其实是俄国的宣传站的那个烟幕。就是都是假造的。那为什么是假造?因为我是物理出身,所以我看过以后知道这个是不合理。



唐湘龙 10:01 

就算贫铀弹已经不会是这个。



王孟源 10:04 

贫铀弹出来不是这样子,它不会提升伽马射线。没错,不会提升,不会有放射性的PMI。那事实上贫铀弹的确是还是有放射性,但是这个因为它放的是阿尔法射线。你必须要成块的吃下去,才会有危险,它如果爆炸之后变成粉尘或者是渗入地下水,它的主要危险其实是化学性的,因为铀的化学毒性其实相当的强,它是跟铅差不多的一种化学物质,但是毒性比铅要高很多,所以你除非是一粒一粒的那个铀吃下去,否则的话你担心那些贫铀它的毒性是主要是化学毒性,而不是放射毒性。所以这一次其实是因为我想是因为俄方要宣传,制造当地人的恐慌,那事实上也是打掉了,事实上也是有污染的危险,但是你没办法跟老百姓讲这种细节对不对?所以,所以制造一点假造一点证据来制造恐慌了,来加强他们的这个成果。



所以总之来说是,大的来讲,的确是有贫铀外漏。但是细节来讲,目前流传在外的证据是假造的,OK。



然后到了5月16号,两天前,有一个很大的新闻就是俄军集中地用了三批,首先是无人机,然后用巡航导弹,然后再用高超音速的弹道导弹,短程弹道导弹去攻击基辅机场的爱国者飞弹的阵地。



我想我也跟你谈过好几次,就是爱国者飞弹其实不算先进,这原因是美国从来都不在乎投资到防空导弹上面去,他们一直都假设自己会有制空权,所以爱国者飞弹根本在技术上只相当于苏联 40 年前的防空飞弹。它的这个限制性很大,第一个它不能机动,OK?第二个它的那个雷达不是 360 度的,而是 120 度的射角,所以有...必然有死角。你如果是多方面全面攻击,然后你再想想看,它又不能够简单的撤收运动,所以很它本身是很脆弱的。就是你对美军来说没有关系,你美军反正会有绝对制空权,所以威胁顶多就是一个方向过来。那没有关系,但是乌克兰这样就不行,基辅那边,很简单就可以从让俄国从三个方面来做打击,那这一次也的确是这样,两天前也是这样的,那后来有一个,有一堆视频出来了这些然后这些视频出来的第二天就有 6 个乌克兰人被逮捕,他们即将被判 8 年的徒刑,就是泄露国家机密。



王孟源 13:24 

其实他们做的是什么呢?就是他们在自己的公寓的窗口对着那个机场做录影,然后上传到网络上去。那谢谢他们的牺牲,他们的惨重牺牲,我们可以看到很详细确实的视频,这个视频是在两分钟之内有 32 发爱国者飞弹发射出来。OK,那然后随即有两个非常快的,明显要快很多的飞弹从上打下来,打下来了以后造成两个爆炸,两个很大的爆炸,其中一个就是那个 32 发爱国者飞弹飞上去的那个地点。



OK,那这里面有很多技术细节,,为什么大家知道那是爱国者飞弹,不是其他的防空导弹?有很多技术细节,其中,比如说最重要的就是俄制的防空导弹都是垂直发射的,爱国者飞弹不是垂直发射,是倾斜发射的。



OK,然后这个 32 发导弹刚好是美国最新的爱国者 3 型的,这个叫 DCI 分型分类导弹,它从以前一车 4 弹变成一车 16 弹,所以就是两个车全部打光的结果。那事后乌克兰说他们成功拦截 6个... 6 发俄国的所谓的 Kinzhal 匕首,意思叫做…其实是弹道短程弹道导弹,但他们现在流行什么高超音速导弹,所以他把它改名叫做高超音速导弹。



那我认为这一次又是一个乌克兰宣传睁着眼睛说瞎话的案例,因为上个礼拜就有一次他说乌克兰宣称击落了拦截了Kinzhal,听到结果乌克兰空军马上出来否认,说我们没有拦截,我们没有成功拦截 Kinzhal。但是他们的宣传部马上又出来否否认,说的确是拦截,而且发了照片。那这个照片大家一看根本跟就不可能是  Kinzhal,它的那个形状大小都不对,OK,所以很明显的是他们睁着眼睛说瞎话。



那么这一次他说拦截 6 个 Kinzhal,又是睁着眼睛说瞎话。因为掉下来的残骸目前确认有 8 个残骸,全部都是爱国者导弹自己发疯...爱国者导弹有一个很有名的缺点就是它有很大的几率,像你这次 32 枚有 8 枚发疯,然后 90 度转弯往下砸下去。这个两年前在沙乌地打也门内战的时候,就已经一连发生了三四次,就是他们用爱国者飞弹去拦截也门叛军的来的导弹袭击的时候,结果爱国者飞弹发射出去三四公里以后,就 90 度转弯往地上砸下去了。OK,结果这一次又有 8 个残骸,证明也是爱国者飞弹,所以美国的这个防空导弹这个品质非常地可疑的。那你再想想看,这还是全新的最新型号的。



唐湘龙 17:06 

而且这也是台湾的台北的主要的防空系统。



王孟源 17:10 

比台北的那些型号还要新。



唐湘龙 17:12 

对,这个台北在这次看到这条新闻时候,台北的军方应该要好好的想一想这个 O…O 不OK,因为刚刚孟源讲的这一段,它很重要的原因是因为说乌克兰跟美方的一些的讯息,把它包装成了史上第一次防空导弹击落了高超音速导弹。高超音速导弹可以被防空飞来拦截这件事情,当然如果在军事宣传上面,如果真的是,那当然是一件大事情。但是孟源在告诉你说那是鬼扯。



王孟源 17:42 

这是鬼扯,事实上是拦截,但,是用导弹车跟雷达在地面上拦截的。



唐湘龙 17:51 

就是这个就把它挡了,就是被炸到了,对,并没有拦截到,被炸到了。



王孟源 17:58 

被炸到。对,就是Kinzhal,俄军的说法是他们的用三波,首先用的无人机去引诱他们的短程导弹跟那个防空炮,然后顺便做侦查确认,然后再用普通的巡航导弹去引诱爱国者飞弹发射。发射完以后,完全确认了以后就由 Kinzhal,发射两枚Kinzhal,所以我们看到的是,然后来又有无人机跟卫星确认,当地实际上是部署了三辆发射车,但是其中有一辆是还在装填状态,所以没有用。那当天看到的 32 枚,就是真正部署的那两车,全部发射光了。



王孟源 18:53 

发射光之后,俄军发射的两枚Kinzhal 导弹,一枚打在其中的一个发射车上面,另外一枚打到了雷达车上面。OK,所以基本上是相当成功的,是俄军精心策划的一次攻击。



那国军的那些爱国者飞弹,我想能有多大的用处?从这次战例也可以看得出来,没错,事实上 30 年,,这是那 30 年前的两那个第一次伊拉克战的海湾战争的时候,美军也是宣称有 97\% 的拦截率,那个时候拦截的还是比较原始的飞毛腿导弹。但是事后复盘以后,正式的报告,这是美军的正式报告,不是我在这边估计,,就是发现他的断言是这样子的:不超过 10\% 可能是0 拦截比率。哈哈。不超过10\%,实际上不超过百分之十,可能是0。所以你看后来两三年前的那个沙乌地所使用的经验也是差不多也是 10\% 左右,你能够拦到算是运气,那正常是拦不到,而且还要担心它砸到自己人头上。这一次乌克兰用也是这样的。那因为现在美国的整个政治跟经济文化非常的腐败。



王孟源 20:24 

他们这连这些军工厂商,他的重点都不是在生产能用的武器,而是要想尽办法来多捞点钱,那最好赚的钱就是外销。但是连自己美军自己用的,他常常也是故意的拖延。反正你拖得越久,他投资越多,因为他们的那些契约都是所谓的 cost plus,就是你报账以后再加利润,然后算给你的。所以你拖得越久,花费越多,越浪费,你赚的也就越多。你说像这样的契约,你怎么可能没有腐败,没有低效的问题?那你这样去买来的武器,纯粹靠他们的主流媒体的,这样的胡吹乱盖,实际上能有多大的用途...实在是预先就可以知道的。你只要真正去看美国自己的事后复盘的报告,就会知道这完全都是在胡扯,目前的主流媒所说的这些都是在胡扯。



王孟源 21:38 

另外一个乌克兰在最近这个这一两周出的新闻就是他一直想办法去要 F16。结果欧洲很热衷,有好几个小国说要把他们的 F16 送给乌克兰,但是美国反而很矜持。



我想跟大家解释一下,这背后有两个考虑。首先你这些武器,即使是有欧美的这些主流媒体的胡吹乱捧,然后遮掩它的缺点,事实上还是会有打砸招牌的效应。因为真正会相信欧美媒体的,除了像台湾这样的少数几个地区,大部分的潜在买家,像是沙乌地他们都已经心知肚明了。所以这实际的战果多么的可怜,都是负向的广告。,像是前一阵子好不容易法螺吹得天漫天响的那个海马斯,现在也都是被俄军很可靠的很反复的拦截掉了。



王孟源 22:57 

那你这样一来还有谁想去买?那 F16 这种东西我上个月已经讲过,即使是乌克兰那么点剩下的那么一丁点数字的防空导弹,都足以吓阻俄方(不)用空军去做进阶的打击,密接打击就是 close support。那你想想看,如果乌克兰想要用 F16 反过来去进攻,防空导弹数量多 10 倍,然后世代新两三代的那些新的防空系统,那会有多惨?,这到最后不是反宣传嘛,对不对?那你这个 F 16 怎么卖?另外一个考虑是,因为乌克兰现在防空导弹用的差不多了,那这个爱国者导弹又不堪重任,所以欧洲这些小国就想着你拿这些 F16 去做防空,但是 F16 其实现在空战的这个效应它早已落后了,它是 1970 年代发展出来的东西,这个你面对俄国的苏 35或苏 57,根本就不堪一击。那实际上乌克兰一定会把它拿来用来投掷那些远程的空射空对地的导弹,像是英国刚刚提供的 storm shadow 这种都有射程150~2 百公里的。那乌克兰现在最想要的就是这些长程的攻击武器 strike weapon。,那手头上已经有英国的这些了。那顺便,先拿到 F16,然后再想办法跟欧洲跟美国继续要。那这样一来,他就有反击莫斯科的底气了嘛。你如果有几百公里的射程就可以开始打莫斯科。



唐湘龙 25:04 

就攻击俄罗斯本土,尤其是俄罗斯一些重要城市。



王孟源 25:08 

你看到他两三个礼拜前,费尽心机送了 2 个长程无人机去炸克里姆林宫,对不对?你就可以知道他们这个仗已经打的完全就是在宣传口上面的仗,他们根本不在乎实际的死伤对比是什么,他们在乎的是宣传口上,能拿什么象征性的新闻报道。,所以你这个 F16 给他们,他一定会把它拿来突击莫斯科或者克里姆林宫。那美国这个上次的那个无人机突击克里姆林宫的事件之后跳出来马上否认,说我们跟这个一点关系都没有,这个全部都是乌克兰自己搞的。那你想想看,这个 F16 如果交给乌克兰之后,用英国或者是美国的长程飞弹或者是滑翔炸弹去打克里姆林宫,那效果会比那个小小的无人机那大多了。要大多了。真的把克里姆林宫,杀死人之后这个美国要怎么善后对不对?你这个俄国就会有一个Casus Belli (编注,即P社游戏中的CB,宣战理由) 就是升级的好借口。这是美国自身非常不愿意见到的。所以这是为什么当前为了这个 F16 的事会有美国跟欧洲在扯皮的事情。,而且很奇怪的是,美国一直是在幕后鼓动这个战事,到了 F16 反而是红线划的很稳,为什么这样?跟大家解释一下,就是这样子。



王孟源 26:55 

然后另外一条上了新闻的一个细节,就是Wagear Group 的老板 Prigojine 最近大嘴巴说放了好几次。



唐湘龙 27:08 

这也是我想问问孟源,就是我有两个考虑,一个其实我最好奇就是你现在要讲的就是瓦格纳的这个老板普里戈金,普里戈金他不断的放话,那这个放话从我们事后看到,所接收到的巴克穆特的战事的进展来讲,就有点雾里看花了。本来大家认为说那个是在唱双簧,那个只是一个刻意的,就是说混淆乌克兰跟美方的判断。可是在战场上面好像你看到乌克兰借着美方的援助还真的发动了一波的反攻,而这个瓦格纳似乎似乎跟俄军之间开始出现了这则就说在指挥上面的裂缝,是这样吗?



王孟源 27:57 

我跟你讲一下深层的道理,然后你再去看,因为他的这个大嘴巴已经有几百次采访,哈哈哈,他去看了,对他稍微讲,我想绝大多数人都是看的都是雾里看花,因为他自我矛盾嘛,对不对?哈哈,所以我跟你讲一下这个深层的道理,然后你再回去印证一下,看看对不对。 Prigojine 其实是普丁找出来它有两个任务,第一个任务是要刺激俄国的正规军,国防部的嫡系系统,要提供一个对比去刺激他们,让他们更加的积极,更加的反省,更加的没有借口。



王孟源 28:44 

你看这个Wagner能够在巴赫穆特打的这么好,你们怎么打其他的巷战打不下来,那这就没有借口了,对不对?



这是战略上的需要,在战术上的需要。是的确需要有这么一个比较灵活的部队。没有这个尊卑长幼,尊卑锁死的那个官僚体系,当他们的去灵活的实验,灵活的探讨。去年的时候他们发表了一个宣传片,叫做 road to hell。还是什么呢?就地狱尖兵,中文翻译。



一个Wagner自己拍了的电影,这个不是纪录片,而是拍了电影,哈哈哈,它里面是记录的他们在那个Popasna,就是去年夏天的时候,他们第一次成功的巷战,拿下了一个城市的过程。那这真的是现代巷战的经典示范,是非常值得军迷去看的。但是我想提醒就是这其实它会把它拍成一个影片,然后精确地呈现出来。其实不是要让我们这些外国人看,它主要的听众是俄军的那些军官。



那你也许会问说,普京怎么会突发奇想去刺激俄军呢?那第一个是俄军的确是还有很多弊病,你如果因为毕竟大家都已经几十年没有打过仗了。那没有打过硬仗,美军倒是常常去欺负那个第三世界国家,但是那也不能不算是硬仗,他们就是步兵,遇到什么阻碍就把空军叫来,扔几个一吨级的炸弹把它炸平就是了,对不对?这不算硬仗。



那俄军的确是有一些腐败,有一些低效的事情需要这种刺激,然后也需要一个比较灵活的组织去探讨,去应对现代的最新的战况,比如说像无人机的战法,这都需要实际上去实验的。那你说是不是只有俄国人这样做的吗?不是的,你如果看美国的话,你会注意到美国跟全世界的国家都不一样,它不是三军,而是四军。,它多出来的第四军是什么?是海军陆战队。



海军陆战队就跟陆军有很大的重复,对不对?,事实上从越战一直到最近的这几年,这五十几年,半个多世纪以来,海军陆战队一直是作为刺激陆军官僚的一个手段。就是你看你陆军表现不好了,你都没有借口,因为海军陆战队面临同样的问题,他没有这个问题。然后另一方面他也能够自由的去实验新的战法、新的组织、新的作战原则、新的武器。你看美国的海军陆战队到现在用的武器还是跟陆军不通用的,他自己选择步枪什么的。那这个同样的思路你用在俄国军队上面,他又要临时赶快组建,没有一个现成的几百年前留下来的一个组织。那这个 Wagner你就可以了解,为什么Putin会拜托Prigojine  去做。



那你想看,Prigojine  的责任原本就是要刺激俄军了,所以你他凡是在后勤上面,指挥上面都可以把他的不满无限放大。然后你还要考虑到一点,他的部队里面有 1/3 是以前的犯人,,剩下的 2/3……



唐湘龙 32:52 

特别。



王孟源 32:52 

哈哈哈,剩下的 2/3招来的也都是好勇斗狠的狠人,对不对?,你如果不是喜欢杀人的人,谁会想要自愿去当那个雇佣兵?



唐湘龙 33:07 

对,因为瓦格纳这个雇佣军跟这种像是前海豹部队,他们的这些雇佣兵还真的不太一样。他的那些背景我都以为只有在以前,我们在看古书的时候会看到现代军队应该不会出现这种情况,可他就是这样子组成的。



王孟源 33:25 

对,你去管理这些人有多困难?,所以Prigojine 当然也要说一些话,故意的栽赃给俄军的那个后勤还有指挥官员,然后来安这些心嘛,对不对?就是你们,我们这已经有了死伤了,我们这边有额外的困难了,这些都是俄国的那些将领的不对嘛,对不对?,这是很自然的。所以你看他讲话呢,其实他的第一个听众是俄国的将领跟军官。,第二个听众是他自己的部下,他自己的士兵。我们在旁边人看的当然是雾沙沙(注:指看不懂),想他是在跟乌克兰喊话,或者是跟Putin喊话,或者然后甚至有人会想象他是在跟国际喊话,他不是这样的,他这他的讲这些事情的目标听众是俄军,还有他自己的部队的士兵。



王孟源 34:23 

所以你这样子如果再回头去看他过去这一年多受采访的几百次所讲的话,你自然就会理顺那些逻辑,就是他讲的那些就都理所当然。所以我今天讲的这些战术上的细节就讲到这里。然后我来评论一下你刚刚开场的时候讲的这场所谓的春季攻势。当然这个春季攻势他们已经宣传了四五个月了,然后我上个月的时候还在讲说其实到5月初,5月上旬这个天气就可以了,就是地面就不是泥泞了,,然后没有那个装甲车会陷到泥巴里面的问题,那照理说是应该在5月初就开打,但是我那个时候,就说过俄军很明显的是把他们去年底募到的那些兵,还在...还抓在手里当后备队。



王孟源 35:23 

,然后小量的到前线去轮替来获得以轻度的战斗来获得实际上的训练效果,那也就是俄军的后备队比乌克兰的后备队多了 3 倍, 4 倍。所以这个攻势完全就不可能有成功,你反而是会形成我所说的 1943 年的Operation Citadel的那个形势(编注:堡垒计划,即库尔斯克会战)就是当时是德军自以为可以在做闪电战的攻势,结果被苏军先消耗他们的锐气以后发动了战略反攻,然后德军从此一蹶不振,一直退了两年。那这个局势,那个 Operation Citedel是苏联所有军事院校的教材里面的重点。所以你说俄军的将领固然是熟记在心,但是乌克兰的这些将领也都知道。他们也是同一个军事学校训练出来的,所以你就可以看出他们一副心不甘情不愿就是能拖就拖,他们就是不愿意下场打,因为他们知道打了以后会有多惨。



OK,所以现在我们看到的情形是什么呢?就是乌方,乌克兰军方说什么就是不愿意开打,但是后面的欧美的这些政客说你们去打,你们去打,我要有战报。你这些主流媒体要欺骗我们的百姓,还有第三世界也不能够凭空的想象出来,你至少要真动手才行。那所以目前就是一拖下去,一直拖到乌克兰顶不住压力才会出手,那如果他出手的话,目前他们唯一还算合理的一个战术选择就是模仿 1968 年的 Tet Offensive,就是当时越共的春节战事,也就是以渗透性的多点开花,然后牺牲这些部队的生命来换取宣传上的优势。



王孟源 37:52 

就是因为你回想一下当时 1968 年的春节攻势是怎样子,实际上越共派去的人全部牺牲,全部被歼灭,但是对美军的士气打击非常的大。就是他在战术上失败了,但是在战略上成功了。那乌克兰既然在战术上没有成功的可能,就必须要在战略上,至少士气上获得一些收获,那所以他要是打,固然主攻方向是他事先已经声明全世界都知道的,从Zaporizhzhia往,Melitopol那一线,这样打下去。他一定会同时有很多游击渗透的手段,包括比如说它的 47 机械化旅,特别装备了美国的 M2 bradly,美国海军陆战队用专用的 aav7,这个也是台湾军用的



唐湘龙 39:00 

这个也是台湾军用的 。



王孟源 39:01 

对,还有苏制的 BRDM2,这也是一个两栖的装甲车。



唐湘龙 39:10 

是在渡河攻击的时候用的。



王孟源 39:13 

渡河攻击的时候用。还有那个 M3 是美军用的工兵造桥车,就是全部集中在一个旅,然后还很用心的训练、收集,这很明显的是要渡过……



唐湘龙 39:32 

第聂伯河



王孟源 39:32 

第聂伯河,对,然后攻击那个核电站。以及Kherson城市这两条路,这是侧翼的,就是突击,他们没办法预测……但是事实上我都知道,我都看得出来的,俄军的将领会看不出来吗?这个我不相信……对不对?



王孟源 39:53 

所以,实际上他们一开打,我上个月就说这个,他一开打的话,两三个月之后就没事了,然后就是轮到俄军回守,然后就会变成一个很大的战略反转。现在的问题就是乌军的将领自己也知道,所以拼命的在拖。你这个春季攻势再拖一个月就变成夏季攻势了。那你刚刚提到的这巴赫穆特,这个其实不是真正的反攻,它有两个意义。第一个是现在在欧美,在后面督军在催的时候,他可以拿出来说,我们的确是有试图反攻,但是这其实规模很小,就是旅级的,就是两三个旅再打那。,第二个意义就是现在瓦格纳过去这个礼拜瓦格纳已经快要把整个把巴赫穆特全部清光了,是剩下的最后两一两条街,今天刚好是剩下最后一条街了。那你这个时候,两翼做一点反攻,造成一些压力,方便这些守军撤退,所以实际上完全是战术上的小反攻。



王孟源 41:14 

Counter offensive 不是作业意义上的。我们一般人、外行人常常会讲所谓的战术tactics,然后战略是strategy。真正专业军事上,我们外行人讲的那些所谓的 strategy, 就是大规模作战,其实在军事专业里面叫做operational,翻译成作业性的。就是这个也有也很严格的定义,就是你直射武器的范围内叫做战术。,超过直射武器范围的就是两三公里外的,就叫做作业。



王孟源 41:51 

OK operational,那它这个是一个作业上的小反攻。战略是全面性的,会影响整个战局的,那这,这不会影响到整个战局,所以只是作业。所以你在军事上其实是战术,然后作业,然后最高的才是战略,不是两级。那我这里可以跟大家讲一下。,那所以你一开头所问的,有关春季攻势的这件事情,总结来说就是我过去几个月一直讲的,没有希望,不打还好,打了会很惨,那目前的乌克兰基本上就是拼命的在拖。



唐湘龙 42:36 

我接下去要跟孟源要来关注一下欧美的经济这些情势,因为那个非常的诡异。好,不过在谈欧美经济形势之前的时候,俄乌债战争的一个变化就是中国的特使李辉,李辉这几天的时间五国的访问,那乌克兰这边已经来了。好,那几个主要的国家他要走一遍,法国、德国、波兰,然后俄罗斯。中国的介入,对俄乌的停火,今年可以期待吗?特别在G7开完会之后。



王孟源 43:11 

在我回答这个问题之前,我先再退一步,或者说再升一级,再谈这个俄乌战争的意义。这个俄乌战争西方有一句老话,他说历史并不是重复,但是他会押韵。,所以你如果要……



唐湘龙 43:33 

OK,这个说的很好。



王孟源 43:36 

对对,但是你如果去看这个俄乌战争压的是什么的韵,我个人认为在战略级别最相似的是 18 世纪的七年战争,这个七年战争绝大多数人都不熟,所以我跟你解释一下,是 18 世纪中叶普鲁士崛起的决定性战争,就是那个时候。我想之前我提到 30 年战争很惨。那是 17 世纪神圣罗马帝国内部的一个内战,结果引发了周边的所有强权都参与,结果就把原本最富庶,欧洲最富庶最繁荣的神圣罗马帝国给打烂了。之后,他们才会有那个威斯特伐利亚条约,建立了国家主权的观念。,国家主权这个东西就是它的定义,就是任何国家都应该有的基本权利。它的定义就是这样子。



王孟源 44:45 

那此后 100 年,这个被打散的神圣罗马帝国就分成了三块,中间那一块就是奥匈帝国,南边碎掉的是意大利一小块,那北边则是目前我们所知道的德国,但是当时是分成几十个小的城邦。那在 30 年战争之后的那 100 年,有这里面逐渐凝结出一个最强的小王国,就是普鲁士,OK,那到了 18 世纪中叶的时候,因为奥匈帝国第一次要换成一个女皇,就是 Maria Theresa,那这违反了日耳曼人传统的继承规则,就是他们跟盎格鲁萨克森不一样,盎格鲁萨克森一直都是可以有女皇的,但是日耳曼人是不能有女皇的。



王孟源 45:48 

那但是 Maria Theresa 的爸爸特别要修改这个规则,要让他的女儿继承他,所以事先跟大家都商量好了,然后所有的人都说好可以,但是一旦他死掉以后,大家都翻脸,那这翻脸的人之中就有普鲁士的。普鲁士的 Frederick the Great 。那Frederick the Great 就是 腓特烈大帝。



我这,那我这边花 2 分钟跟大家讲一件小事情,欧洲历史上有好几个大帝,那除了我们后人,后人闲聊的时候胡扯的,真正的大帝其实就是那么四五个。,最有名的就是亚历山大大帝, Alexander the Great。然后来俄国有两个大帝,一个是Peter the Great (彼得大帝),一个是Catherine the great(叶卡捷琳娜大帝),那还有一个就是普鲁士的Frederick the Great (腓特烈大帝)。



一般中国或者台湾念历史的人都以为这些大帝是历史学家给的名字,其实都不是,都是他们自己叫他们自己的手下,上表来封的,自封的。包括Frederick the Great在内。所以你如果去仔细读历史就会知道,这些都是他们自己设法绕着圈子自封的封号。那我们现在看这个所谓的普丁大帝了,或者是习大大这些封号都跟他们自己本人没有关系,是网络上的网民自己开玩笑编造来了,相比之下反而是更加的真诚,更加的有实质意义。那我说这个对比是啊,让大家知道,因为一般念历史的人不知道,他们以为这些人是真正的伟大,所以才会被后世的历史学家尊称为大帝,其实都是自己封的。



那不论如何 Frederick the Great能够自封大帝,它的功绩就是这场七年战争。这场七年战争是他利用 Maria Theresa 上台的时候,名不正言不顺的这个机会,去抢占奥匈帝国的一个省份Silesia (西里西亚),那  Maria Theresa当然看他不顺眼,那当然不会让他这样顺利拿下,然后周边的强权,也都觉得可以趁机揩油,因为他们有 30 年战争的记忆,所以觉得可以趁机介入,然后再去瓜分这些德国境内的小国的利益,所以法国也进去了,俄国也进去了,瑞典也进去了。所以就变成 4 个强权围攻一个普鲁士的的一个局面。



王孟源 49:01 

那唯一支持普鲁士的就是英国,那英国也没有直接的介入陆战,他只是趁机在海外去抢荷兰跟法国的殖民地,英国后来的世界霸权就是从这里奠立的,他把这些欧陆其他国家的殖民地都抢光了。,那你如果去看七年战争的过程里面有十几个主要的战役,那  Frederick the Great是不是全赢?倒也没有,都是一些战术性的胜利。那也有一些战术性的小败,但是打到最后打了 7 年,打到最后有没有什么决定性的事呢?他有没有打到维也纳去?没有。是怎么胜利的?他这个他胜利之后就奠定了普鲁士的霸权:他胜利的原因是另外那四个围攻他的那四个强权,因为打他打不下来,所以在经济金融上面垮掉了,就是快要破产了。



后来那个法国的大革命跟这件事有很大的关系,就是因为那个军费用得太多。没错,所以民不聊生。



王孟源 50:27 

OK,那事实上我们回头去看这个历史,世界上的霸权有两个层次,最高的是所谓 Super power,再下来第一阶是所谓的 Great power,你要成为一个 Great power,一个强权一流强权的一个条件就是像普鲁士的  Frederick the Great这样子。能够面对一个众多强国的联盟,然后还打赢了或至少不输的,对不对?,那你看看现在这个俄乌战争,Putin是不是这样子一个人对抗整个北约。



唐湘龙 51:10 

三十几个国家



王孟源 51:11 

三十几个先进国家,然后他只要不输,然后等对方的经济跟金融跨掉以后,他自然就把自己的国家地位提升上去了。,OK。那我觉得很有趣的是,当时普鲁士唯一的盟国是英国,他也没有直接参战,他只是在外面按自己的利益去占领第三世界的殖民地。那现在是不是有一个很类似的情况,有一个更大更富有的盟国,再去收集第三世界的支持,但是没有正面的加入战事?你说是不是很像?



唐湘龙 51:52 

没错,当他尤其这次在G7,当日本意识到了中国在许多第三世界,在所谓的南方世界,中国的力量已经大到日本、美国很难去跟中国竞争的时候,所以这次的G7,其实我认为它的亮点在于它邀了非洲联盟,也邀了太平洋岛国联盟的代表进来。虽然不见得能发生什么作用,但他意识到在南方世界里面,他们已经不是中国的对手,所以G7这一次他基本上是一个军事同盟,同时想要试图扩大平衡中国在南方世界的影响力。好,我们来,回头来,我先呼吁我们的观众朋友不要光仔细听了,入迷了之后,因为孟源的课程,王孟源教授的讲座它是会有些门槛的,就是因为孟源在谈军事谈经济问题的时候,我不是恭维王孟源,而是我作为一个听众,我觉得我觉得他的知识的含金量是非常高的。那相对来讲,你作为一个听众观众,你自己必须要有一点的知识准备。那个门槛是要有的好,要不然听的时候你就反复多听几遍。



王孟源 53:03 

我觉得今天时间不太够,干脆不要谈经济了,我就沿着这条线再走下去。



唐湘龙 53:08 

也好,没关系的,那你要再让我凹时间。好,来,那你就继续来。OK。



王孟源 53:15 

你刚刚谈到的这个第三世界的转向啊,刚好过去这两三个礼拜也有另外一个案例。,就是原本沙乌地阿拉伯主导要让叙利亚重新加入阿拉伯联盟。那原本阿拉伯联盟的年会是在5月 19 号召开,就是明天,而且那边已经到了。



唐湘龙 53:39 

对,跟日本这次开G7同一天。



王孟源 53:43 

本来同一天,对,本来同一天。但是那个这个消息传出来以后,布林肯就说他要在5月 13 号,就是提前一个礼拜,,去访问沙乌地,那你想想,用肚皮想也知道是他要去挡住这件事情。但是又比他再提前一个礼拜,在5月 7 号,,还没有开年会,阿联酋就提前地公布,正式接受叙利亚加入。就是布林肯想要提前一个礼拜去阻止,结果。



唐湘龙 54:24 

沙乌地比他更早一个礼拜。



王孟源 54:25 

让他不能阻止。然后这一下来,我现在最近看到的是,既然这样一发生了,5月 7 号发生之后布林肯就……也没有取消,因为取消的话去沙乌地的话很难看。没错,他就说顺延到6月再去,反正现在去也没有什么用。那你光从这一点就可以看出,大家现在连给美国的最后的一点点面子都不在乎了。,这个对不对?,这是很明显,你只要稍微注意到这些讯息,然后想一想它幕后代表的是什么意义。



王孟源 55:06 

然后现在的真正的转折点就在于现在最大的问号就在于欧洲,能不能在这一次学乖。那所以我个人其实是非常希望这一次的春季攻势能够正式的打起来,打起来以后俄国彻底歼灭乌克兰的野战军,然后开始把战线向后稳定的推动,这个时候或许欧洲的民众会觉醒。因为欧洲的目前的政局是非常的锁死了,没有自由了。我上一次提过芬兰的那个非常亲美的总理,她被替换下来,但是她有 20\% 支持率。德国的绿党把掌它的整个外交政策,它的支持率还不到20\%。是,就是你搞这些多党政府,只要有主流媒体跟美国在后面支持,即使是 15\% 20\% 的支持率都能够把你的政策,外交政策锁死。那所以你说要欧洲真正洗心革面,从这次的,从中美脱钩里面理解到必须要中立的这个战略事实,必须要把它真正的彻底打痛才行。那要打痛的话最好的方式就是俄乌战争有明显的反转。OK,而且我们剩下的时间不到一年了。为什么不到一年?我待会再讲。但是你现在欧洲有多惨?我刚好两天前看到最新这一期的经济学人,他的倒数第二页有那个有一个统计专栏,他们自己做了统计分析,就是大家都知道过去这个冬天,我们刚刚过去的这个冬天是欧洲罕见的暖冬,就是他们运气很好,是那个天然气涨了 3 倍,但是刚好这是一个暖冬,所以没有短缺的问题。



王孟源 57:34 

但是即使如此,因为新冠的关系,经济学人在过去的这三年一直都是详细追踪各国的所谓的超额死亡,这个我也讲过了。没错,你因为新冠死亡的时候,大部分的案例不是因为新冠的急性肺炎所死的,而是引间接引发的高血压、心脏病、中风等等,所以你死的时候这个死因调查也不是新冠,而是心脏病。OK,那这么一来,最好的办法,最好的统计办法就是直接去看超额死亡,就是你 5 年前, 6 年前这个时段。,死了少人。那你今年死了多少人?然后用其他一些间接的统计手段可以提炼出说这个新冠的影响有多大。那他经过 3 年的实践,已经有一个很详细很精确的模型。那最新 这一期的报道结果是什么?他说他们去测量,结果发现今年冬天的欧洲的超额死亡新冠只占了40\%。那剩下的是从哪里来的?是因为在家里冻死的。



王孟源 59:04 

OK,这一件事情我也讲英国在过去十几年,就是 2010 年,他们社会福利政策改掉之后,他每年冻死到7万人,那今年结果经济学人,这是英国的主流媒体之一。自己承认欧盟在今年冬天的超额死亡,因为在家里冻死,因为天然气太贵,结果在家里冻死的人超过了新冠。OK,你说这样的代价大不大?非常大。当然是很大,但是够不够扭转他们的政治的政策?,完全完全不够,因为看不到。



唐湘龙 59:54 

看不到。



王孟源 59:56 

,而且你平均起来还是大约 1000 人里面死一个。那你主流媒体如果不报,大家就觉得我怎么刚好亲戚朋友邻居里面多死了一个心脏病的?没有什么不了的,对不对?这是偶然的起伏,也会出现的,所以我先讲一下这个,当前今年到年底之间的这个战事转折值得大家注意。那它的影响不是俄国会不会胜,俄国是一定会胜,他只要拖,即使拖七年,也一定会跟七年战争那样子把对方拖到经济跟金融垮台。这也不是在于美国会不会参战,美国绝对不会参战。,美国要参战早就参战了。它真正的影响是,今年下半年如果出现非常严重明显的反转的话,能不能改变欧洲的政治结构跟格局?,那为什么这个欧洲的政治格局改变在今年年底改变这么重要呢?就是明年1月台湾的总统大选。



王孟源 01:01:15 

,我一直、过去这 3 4 年,我已经讲明了,我不评论台湾的内政。那这里我也不是在评论台湾的内政,我是在评论台海的情势对国际整个国际大事的影响。从 2014 年、 2015 年我开始做评论开始,就不断有人会说这个台海会兴起战事,比如说 2015 年的时候,那时候准备要选举,吕秀莲说,她看到共军要准备明年就渡海攻击,那时候在博客上说,哈哈哈,这个傻人傻话不值一评。后来,到了 2016 年, 2017 年,大陆上有一个著名的战略学者说可能 2021 年,或者 2022 年,或者 2020 年会武统,那个时候我也还专门写了一篇博文说不会,OK。问题是什么呢?其实基本的调子就是台湾没有那么重要,,也没有那么紧急。



唐湘龙 01:02:20 

这个是我常常讲的,哈哈哈。台湾除了台积电以外,其他真的都不重要。



王孟源 01:02:26 

但是,台湾人自己关在家里面自以为是,什么民主自由的前线堡垒。



唐湘龙 01:02:34 

然后战略地位多么的重要。西太平洋不沉的航空母舰,我说那个都真的都不重要,台湾真的太低估……



王孟源 01:02:41 

,但是我那个时候就说真正要观察的是, 2025 年前后,后来我还详细的说就是指的2024年到 2028 年之间那四年,,OK。为什么这样?因为我那个时候预估是在这四年里面是中美的实力,,以及在整个世界的地位会反转的时候,也是美国用这个……最后的孤注一掷,挑起军事冲突来扭转这个大局面的最后一个机会。



王孟源 01:03:22 

,我从来都不是假设中国会去主动地做武统,,我从来假设都是因为美国选择要拿台湾来做引子,引发欧洲加入全面的制裁,经济制裁,这才是真正的用意。那从去年开战之前我讲到现在这个俄乌战争,原本就是美国先预习一下,来凝结北约的向心力来准备在台海搞事的,所以这个台湾在未来的这一年,真正地站到了国际所谓的 focal point,就是国际局势演变的尖峰,就是有从所未有的重要性。



王孟源 01:04:18 

那这个台湾人自己做什么一点关系都没有,都是我 10 年前就讲过。这是因为你从中美的大势平衡就可以预测,大约在这个时段是美国会有最后一次拿台湾祭旗的一个机会。那很不幸的是,刚好 2024 年的这场总统大选,刚好特别合适美国来做这件事情,所以中国如果要避免这件事情摊牌,避免美国,要吓阻美国来做这件事情来挑起战事,最重要的就是先争取到欧洲,如果他能够确定欧洲不会参与会中立的话,美国就没有理由来干这件事情。



王孟源 01:05:12 

因为美国干这件事情不是为了要防御台湾,不是为了要军事胜利,而是要事后有全面制裁中国的借口。,那如果你这个欧洲已经摆明了不会参与制裁,那你挑起这个战事有什么用?你这个美军自己去送死,哈哈哈,还要威胁到关岛被反登陆,对不对?所以我认为未来这几个月俄乌战争的走向有着决定性的影响,就是对中国对全世界都有未来十年的整个国际态势都有决定压力。因为我再讲一次,这个俄乌战争的全面反转,是改变欧洲政治格局的最可能的隐因,而改变欧洲政治格局,是能够吓阻美国挑起台海战事的最佳手段。



OK,那我在这边顺便再讲一下,台湾国军的那个防守的指导原则,一直都是机动反击,甚至是在海岸拒止。那大陆有很多军事评论员都笑说这个,你们拿当初古宁头的那个战车在海滩上齐射成功,所以您一直到现在,七十几年后还在想梦想着做这同样的事情。他这个是不入流的低级的反应,事实上我认为是国军将领有心的去引导。因为你从这次俄乌战争就可以看出,弱方要守住,要多拖一点时间,最好的办法就是像乌克兰那样子打巷战。,但是你打巷战的话就会造成成千上万的民众死亡。会把整个城市打成稀烂。



唐湘龙 01:07:32 

没错,变废墟。



王孟源 01:07:34 

所以,国军的将领明明知道你机动的反击,不是最有效,不是能够把时间拖最长的一个手段,他仍然选择这个去忽悠民进党的那些总统;,去忽悠国民党那个白痴总统。他的原因是因为他们真正关心台湾老百姓的死活。,他们知道,如果他们打的作战计划是像乌克兰这样打巷战,一个公寓, 一个公寓,这样守下去。那就是整个城市被夷为平地,平民死伤 20\% 30\%,对不对?你宁可是一些义务兵,在滩头牺牲几千的义务兵,比牺牲几十万的老弱妇孺要好得多。没错,这才是国军将领制定这些作战计划的真正原因。,不是因为他们幼稚不懂。



国军从来就不是大陆的那些网民所想象的那么不堪。他们虽然在国共内战的时候打输了,但是至少还有输有赢。你同样的那些共军到了韩战的时候一样把美军打回去了,你想想看,国军的那个后勤跟火力能跟美军比吗? 1/ 10都不到,对不对?他们不行,是跟共军当时的全世界最精锐的轻步兵来相比,是不行,但实际上他们的专业水准是很不错的。



那到一直到现在还有一点遗留,现在留下来的国军将领还是有些有良心,有学识。所以跟大家讲你们,我不管你总统选谁的,就当然因为主要的选择还是在选择权还是在美国手里,而是那你选错了总统只是方便美国人一点,但是你们千万不要让他们改为巷战,作为防守的计划,那真的是几十万条人命。我先跟大家讲。



唐湘龙 01:09:53 

孟源讲的最后的这一段,是最近我看到有关于两岸军事谈论里面我觉得最精彩的一段,而且讲得非常的简单,一听就懂。我希望这一段对于不管是台湾的军方或者喜欢讨论,动不动把两岸的这些武统军事冲突挂在嘴巴上面,好像聊的是事不关己一样的。这些的朋友网民们可以呢认真的听一听,想一想,我们在思考的是人民,这个是最核心的,尤其是平民百姓。刚刚讲到好比说巴赫穆特大概只剩下几栋的建筑物,而且那几栋其实也都不能住了,其实就是还看得出建筑物的样子,整个巴赫穆特尔那十几万的城市现在近乎夷为平地,那就是一个代表。台湾在面对到两岸问题,政治能够处理就尽量政治处理,在军事上面来讲那已经是万不得已,一旦开始了之后情况都很难预料。



唐湘龙 01:10:56 

时间的关系,现在已经就都超过了 15 分钟了,不过坦白说意犹未尽,我跟孟源在之前所设定的一些题目,还有一些大题目都没有讲到,好不好,我再给你约,因为我实在不忍心的打断你,因为孟源有很多的内容是必须要很完整去表述的。来,我先呢,先在孟源在的时候来感谢,感谢我们几位的观众朋友的支持,所以我刚刚一直想要讲说,请你的,请你按赞,请你打开小铃铛,请你的记得分分享。



拜谢部分省略



同时最重要是给王孟源刷一波,你现在刷一波也还来得及那 proud of 看 001 感谢旅游,感谢来跟梦游问好再来英无,感谢他,王博士找那王博士,我有个问题,从泰国最新的大选结果来看,受到美国资助的候选人赢得选举,他在选举前曾发表过挺台湾的言论,这对台海跟南海会带来什么变数吗?好,这个我们下次谈,因为现在的情况还不确定。



唐湘龙 01:11:57 

再来 mark 猪感周,感谢再来英武,感谢老师们向两位推荐边情的谁在导演世界,他把西方世界如何利用谎言包装下的普世价值渗透到第三世界国家,来洗脑和利用第三世界国家的知识分子,让他们忘根忘本以及为其所用写的淋漓尽致,非常到位。这是一个大课题,就是我们受说西方的西方思想跟教育训练的,尤其长时间在西方的媒体的覆盖之下成长的人。其实要能够看出就是说现在的西方的媒体跟政治宣传的那个迷雾,对许多的问题能够回到本直面去谈事情的非常少。好,所以刚孟源的最后的那段的谈话,我非常的惊讶。



唐湘龙 01:12:46 

来看再来康复成,感谢王博士的解读,受益匪浅。王泽他说感谢龙哥解锁好我们的新成就,喜提鸽子龙的称号,看起来龙哥唯一不敢放鸽子,只有孟源大师兄。对了,因为王孟源,我不想让他,我,我王梦云上线的时候我就让他专心的讲话就好了。再来拍地书感谢。他说王振芳先生将于5月 19 日在巴斯顿的东部的时间晚上 8 点受邀演讲。好,那大家可以呢?线上看线上能不能够听得到。再来的吉米摇大概好,感谢,再来90T,谢谢。再在日本感谢王泽。王者说,请,这个可以,孟源及时回答一下。那王者问的时候请到孟源,





唐湘龙:

从战术上面乌克兰肯定是输了,但是战略上面俄国算赢了吗?这个孟源能回答一下吗?



王孟源 01:13:42 

早就处在不败之地。,他们准备了这么多年。就是在战术上他们知道,不一定是最优化的。,可能会多死几千人,但是在战略上面是准备充分了才动手的。



唐湘龙 01:14:00 

OK。



王孟源 01:14:00 

好,而且是等着美国先落入这个战略圈套。



唐湘龙 01:14:06 

好再来 potato head 他说王孟源老师,你的部落格为何登不上去?他为什么要进?他一直说他想看的部落格为什么看不到的意思,有特别原因吗?



王孟源 01:14:19 

我最近也感觉到那个联合报的部落格有的时候会宕机。OK,这个台湾的。



唐湘龙 01:14:28 

台湾网路维护水准,堪虑是吧?对了,因为现在像部落格的大家都已经变得很知性了,一般来讲,我觉得很多网站维护的人,不太愿意花时间在部落格的维护上面。





好,再来 p 券感谢。然后 may z 感谢他说哥支持王博士,睿智犀利的观点给人启发。对,我自己在旁边听的都一样,收获满满。然后我忘记带完郎 IM American 我,王博士感谢。再来神奇光,他说感谢,他说老唐应该叫王博士多聊聊台湾,见解非常深入。

\twocolumn[\begin{@twocolumnfalse}
\section{乌克兰反攻、AI的局限}
\subsection{20230616}
\end{@twocolumnfalse}]Credit: Anonymous, 栗子 (全文词句有轻微改动,目的是让文字更流畅。)



唐湘龙 00:08 

俄罗斯里面的领导人的支持度的民调第二名,他的仅次于普丁了。这个俄罗斯的政局来讲其实也是件非常有趣的事情。好,来,我们准备来。



唐湘龙 00:41 

那欢迎来到龙行天下,我是唐湘龙,今天星期五的时间,我知道大家都非常的期待王孟源,那所有关于要王孟源要加更的、要延长时间的,你不要再提了,算了。就是能够让孟源的一个月的时间能够好好的整理一期,我觉得他已经非常非常的,以他现在的时间的调配情况,他已经非常的努力了。好,那今天是王孟源的时间,尤其从前两个礼拜在乌克兰战场上面乌克兰的反攻,这个反攻已经不管是普丁也好,也说乌克兰已经开始反攻了。那泽伦斯基也说,我们已经开始反攻了,那这在反攻之前的时候还发生了,这个就是说扎布罗热的这个就是说大坝的攻击,然后导致的溃堤事件。



唐湘龙 01:54 

大家仍然是个一如过去,包括北溪 2 号等等,大家各说各话,互相说是对方干的,那到底是谁干的?因为在发生在反攻之前,显然跟这场的反攻的蓄积的能量以及战术应对是有关的。那到底是谁干的?对战场又有什么影响?那我们看到的,在所有的反攻当中,似乎乌克兰仍然是把反攻的重点仍然是摆在了顿巴斯地区,尤其他刚失去了巴赫穆特的附近,为什么?再来就是说当有一些新装备进来之后,我们看到布拉德利步兵战车, M1 的, M1A1 的,这个就是说步兵战车,然后包括风闻甚久了,就是说北约的主力战车豹2也进场了,除了制空权没有以外,其他的就是说陆战装备、火炮大概都到齐了,那乌克兰也大官宣说他们攻下了那 7 个居民点、村落了。



唐湘龙 02:54 

那我们就用他们说法,居民点,那到底实况如何?第二个跟反攻有关的,俄罗斯的内部是不是出现了窝里反?就是绍伊古这个国防部长跟实际上面在攻下巴赫穆特,甚至于他们有一个勋章就叫巴赫穆特绞肉机勋章,你知道吗?他们特别有勋章叫巴赫穆特绞肉机,这个是我订的这个瓦格纳的这个channel,他们在颁奖的时候,我觉得那个算是很血腥。好,那他认为巴赫穆特几乎跟瓦格纳是画等号的,可是俄罗斯是不是发现了?发生了就是说他的军事指挥体系里面的内讧。



唐湘龙 03:34 

好,那第二段的部分我们会来谈AI,AI你现在不能不谈,你光看美国的股市,基本上面就是从辉达你就知道 AI 概念是未来正夯的概念,它甚至于对于解决美国当下的经济的迟疑跟困境可能发生了一些的作用。最后如果有时间的时候,我们会再把关注在大陆的这个失业率,青年的失业率到20\%,这又代表了什么?好了,先介绍来宾,那透过我们的越洋的连线,人在的美东,大家非常喜欢的王孟源在线上。



王孟源 04:10 

非常荣幸再来跟大家聊天。



唐湘龙 04:12 

好,我是很期待了,因为等着这个反攻,从春季的反攻到夏季,下个礼拜就夏至了,就端午节了。好,所以从春天等到夏天。那我们看到的这个新闻,当然我尽可能两面都去看,但是大家的这个数字跟战场的战果是都不太拢的。实际的情况如何?



王孟源 04:38 

我基本上战事是很简单,但是主要是因为昂萨就是西方主流媒体在那边乱放烟幕,就是他们已经堕落到英文里面有一句话说是 habitual liar,形容习惯性撒谎者的时候说,你要怎么判断他们在撒谎,只要他们的嘴巴在动就知道了。那现在这个西方主流媒体已经堕落到这个地步了,我想我过去这半年除了俄乌战争的新闻还有我的一些预测推演之外,也着重于这些西方主流媒体的扭曲报道。



王孟源 05:26 

我基本上这些叙事都还是有效的,所以今天我希望观众能够去,如果不太记得我前几个月讲的东西,可以去复习一下。甚至于现在、过去这个月的战事发展也完全都是在预料之中的。那今天就主,因为我上个月讲的太深了,结果没有时间谈第二个话题,今天会讲的比较快一点。



王孟源 05:56 

好,那主要就是假设听众还对前几个月的讲稿记忆犹新,我可以直接讲一些增补的、补充的一点,比如说刚刚提到的这个西方主流媒体的撒谎。在过去这个月我可以举几个例子,比如说这个,他的这个大反攻是6月 4 号清晨开始的,到现在 12 天了,开打之后没多久普丁给了一个演讲他说,结果那个整个西方主流媒体,甚至然后自然是所有全世界的这个受西方影响的国家跟地区,像台湾都照样这样报。但是我看到中国大陆也有这样报的。就是很可惜的事情。就是他说,普丁说这个伤亡比是 3: 1。这本身就很奇怪,因为你过去这一年俄军在做巷战强攻的时候,他们都已经可以到达 3: 1 的或者更好的伤亡比,现在是乌军自己主动从他们最大的优势,也就是工事掩体里面跳出来,在没有任何空军掩护,也没有足够的防空武器。



王孟源 07:24 

来抵消俄方的空优的情况下,然后又是俄军在过去这一年,很认真的把电子战重新的升级了一遍,针对着北约的最新装备升级了一遍,所以在信息战上面也已经有了长足进步。在这种条件下他们跳出来进攻,你想怎么可能只有 3: 1?所以我就特意去找他的俄文的原文,结果发现普丁讲得很明显就不是那回事。普丁讲的是一般军事学家认为进攻跟防守的伤亡比应该是  3: 1,但是我们这一次达到了远高于那个的比率。所以你把人家,这个不是听错了,这个是很明显的就是故意去扭曲了,像这个这种就是非常恶劣的事情,所以只要是西方主流媒体登的东西呢,你第一个反应就是这有9成9是谎言,必须要仔细的去看。比如说第二个谎言是他们就硬说是这个Kakhovka。



王孟源 08:40 

Kakhovka水坝被炸,然后他们就说这个是俄军做的。我们从两个方面来看,第一个是哪一方会得利?Kakhovka水坝被炸以后,整个水坝的水流失,长期影响最大的,就是说所有的恶劣影响里面占 95\% 的,是他没有办法用运河去为克里米亚供水,克里米亚是一个半岛,然后他的那个最大的民生问题就是供水。所以一年多前俄军开打之后一个优先的事情就是要打到Kherson,然后占领这个水坝,他占领这个水坝的原因就是为了要保护那条运河,那你这个水坝炸掉了,那个水位大幅降低,没有办法提供那条运河,在经济跟民生的层面上最焦头烂额的就是俄方。那俄方怎么会笨到这样?



王孟源 09:45 

那但是这个西方主流媒体这样报道之后,有人质疑说,拿这点拿来质疑,你知道他们的回答、标准回答是什么?俄国人非常的不理性,就是他们喜欢自己往自己的肚子插刀,哈哈哈哈哈哈哈哈。如果你认为说这还是间接逻辑,没办法确定是西方主流媒体撒谎的话,我再给你一个最明确的直接证据。



王孟源 10:19 

那个水坝是第聂伯河上面的最大的水坝,但是上游的还有好几个水坝,上游的水坝都在乌克兰手里。在这个炸坝事件的前一周,上游的水坝忽然水门全开,把那个水全部都流到这个下游的水坝里面,然后下游这个水坝炸掉以后,没几天上游的那些水坝又全都关了,水门全关了。你想这个是很明显就是炸水坝的那一方为了加大它的这个破坏力,所以专门所做的。



王孟源 10:58 

那这个控制这些上游水坝是谁?就是乌克兰。对不对?你这不是很明显的就证明这是乌克兰有计划做的,对不对?然后再讲第三个案子,在过去这个月,从乌克兰往北跟俄国直接交界的、俄国的那个州叫做Belgorod。有好几次游击队突击的事件,然后他们都宣称是俄国的反对派去突击的,每次大概就是 100 人、一个连级的单位。但是他们说这个是俄国的反对派、游击队所干的。但是后来两三天之后就有影像出现说,因为这些边界俄国人虽然不是重兵防守,但是还是有地雷区,还是有拦网这些东西。结果他们说这个这些游击队在出击之前,这些游击队不但装备的都是机械化的装备。



唐湘龙 12:19 

都是北约的正规装备。



王孟源 12:25 

北约正规装备,对,而且出发前还有专门的排雷战车去帮他排雷,然后有两辆主战坦克在旁边警戒。那整个乌克兰就只有 6 辆排雷战车,我不知道国军有多少排雷战车,排雷战车是非常特殊、珍贵的军事器材。像台湾全部加起来我觉得没有 20 辆排雷战车。你如果这个真的是俄军的俄国的反对派、游击队派出来的话,这个他应该是全世界第一个有排雷战车的游击队了,对不对?哈哈哈,然后明明你看到的那些装甲车都是美制的,但是第一时间美国的国安会的发言人出来说,绝对不是美,绝对没有用到美国的武器。然后是隔了一个礼拜,这个证据确凿了之后,华盛顿邮报才扭捏的说,的确是用了美国的武器,而且是比利时捐赠的美国武器。



王孟源 13:47 

然后他们这个游击队还不止打了一次,好像是打了三次,然后打完了之后,乌克兰方还放了影片、宣传影片,说这是他们在那边作战的英勇事迹,后来网络上有人去把那些建筑物拿去对照一下,发现根本都是摆拍的。他实际上那些建筑物是在Chernobyl切尔诺贝利。那个切尔诺贝利当然是在乌克兰境内。这些全都是摆拍的。



王孟源 14:25 

那我想熟悉我节目的听众应该都记得,我已经强调过了,乌克兰的有一个专门负责假造影视证据的那个特工部队,他们的任务就是专门假造这些战场的视频。比如说一年前就有人发现他们几个月之内发的五六个视频,全都是围绕着同一个建筑而拍的,只不过是从不同角度,用不同的灯光,不同的滤光片拍的。



王孟源 15:07 

所以基本上,我提这个老新闻就是要跟大家提醒一下,只要是西方媒体所报道的,基本上你可以先假定他是谎言,然后再去求证。对,然后另外一个很有趣的是,两三个月前他们不是因为那个北溪被炸的事情很尴尬,因为被暴露了是 CIA 去干的,那结果他们就先说是乌克兰的特工干的。那后来就有人指出特工不可能有那种深水的潜水装备,我上次上节目、三四个月前也提过,那只有大国的海军才可能有那种装备。所以他们在本周又新出了一个理论说不是特工干的,而是乌克兰的军方干的,是他们的总司令。



王孟源 16:08 

Zaluzhnyi亲自下令去干了。我看到这个消息的第一个反应是说,你要从这里面,一个真正客观的、深入思考的、求真的研究者。你看到这个消息的确是可以获得一个信息,但是这个信息不是北溪本身,跟北溪没有关系,北溪事实上是 CIA 跟挪威和联合伙,可能由英国的参与一起去炸的,这个已经是板上钉钉了。这个报道被西方主流媒体放出来,你所读到的讯息应该是这Zaluzhnyi的伤势应该很重,应该是要退休了,就是他没有用了才会拿出来当这种替罪羔羊。



唐湘龙 17:00 

是乌克兰的总司令。



王孟源 17:04 

他两个月前受伤了嘛,在医院待了两个礼拜,然后出来,出来以后就没有再正式穿军装出来露面。



唐湘龙 17:13 

没错,他并没有正式全身的公开露脸,你看到都视频。



王孟源 17:19 

对,最近的这些军事行动,你看他们摆拍的那些照片也都只是Zelenskyy跟其他的一些将军,Zaluzhnyi已经不出现了,然后我认为他们在这个北溪进一步做造谣的活动,其实透露的最有用的消息是Zaluzhnyi可能是真的要退休了,可能是真的已经要去责了,要这样子才会被创造这个假新闻的人拿出来做替罪羔羊。



唐湘龙 17:52 

让他去背锅而已,就把这件事情想要送过去。



王孟源 17:56 

对,当背锅的。然后,这个反攻是我刚刚提到他是 12 天前开始的,那一个很大的特点就是,乌方一直没有正式的承认这是他们宣传了大半年的大面积春季攻势,对。你如果看Zelenskyy,他被访问的时候好像说这是春季攻势,你如果去看他的用字,事实上是这样子,那记者问说:这一波的攻势是不是...春季大攻势是不是开始了?然后Zelenskyy的回答是,“如果你去看现在漏出的视频,有些人可以认为这是一个攻势”,哈哈哈哈哈哈,就是基本上是废话。



唐湘龙 18:53 

就是他不敢去证实这件事,这事情因为事实上它并不是什么大反攻。



王孟源 18:59 

他不敢证实这是大反攻。所以那个一周前就有朋友私下来问我说到底怎么回事?到底是怎么回事?我跟他说这很明显的是,就是我上个月已经讲得很清楚的,乌方知道他们不打,还有一些预备队在手里,俄军还不会真正的大幅的进攻,他们还可以撑久一点,但是你们能听得到吗?



唐湘龙 19:32 

可以听得到,没有问题。



王孟源 19:35 

但是如果他们这一次把预备队都都用光的话,就以后就只有挨打的份了。所以他们是心不甘情不愿的,是背后的美国为了对欧洲跟美国国内的民众做宣传,还有 Biden 要选举,所以才逼着他们去打这场春季攻势。那这么一来就有两个结果,第一个是乌克兰的军方很小心的不想把最精锐的预备队投入、一次投入。第二个是他们不愿意公开承认这个是他们的所谓的大攻势,就是因为他们知道不打还好,打了以后反而会把这些预备队全部消耗,不可能有正面的结果。那最好就是能够打一点 (这里断线)就说,然后就私下跟美国人说真的打不下来,我们还是不打了。那但是他们的美国老板不可能接受这样子,对不对?所以那个你如果去看,事实上这个战场的态势是这样子的。乌克兰在过去大半年、过去的八个月,累积了一共 22 个旅的预备队,其中有 9 个是北约训练的,用北约装备弄出来的机械化旅。那他们的这个攻势都是由五角大厦帮他们计划好的、交给他们执行的。那计划很明显的是,第一个它的主轴是在从Zaporizhzhia往南向Melitopol突破,那尤其是它的主攻方向是一个叫Orikhiv的小镇,这个都是过去半年凡是认真的军迷都知道的,因为他们唯一的希望,我上个月也讲过,他们其实应该打游击战,就像当年越南那样子,



王孟源 22:13 

但是他们第一个本身没有那个纪律,没有那个必死的决心,第二个没有民心支持,这样子一来你就不可能打游击战,那你就必须要打正规战,打正规战的话战略上唯一能够对俄军有可能有威胁的,只有 1/ 1000 的机会,但是至少不是0。就是从向南切断那个走廊,那个地廊,然后把整个二战的这个,俄军所占领的这个红色地带切成两半。



唐湘龙 22:45 

没错,就重点就是马利波拉,就是马里乌波尔,如果马里乌波尔他如果能够再拿回来的话(主持人口误,是Melitopol不是mariupol),那俄罗斯当然就有大麻烦了,但是现在看起来是不可能的。



王孟源 22:58 

他们这次实际战争其实就是围绕着这个主攻方向,而做了很详细的设计。你在节目刚开始的时候说他们在Bakhmut、在Donetsk 附近都发动了全面攻势,这个都是障眼法,为了掩护主攻方向。你整个 1000 多公里的战线,即使是原本已经很疲累的杂牌步兵,也必须跳出攻势来做进攻。这是制造烟幕、迷惑敌人。当然实际上没有人会相信他们在其他的地方有进攻的可能,有真正进攻,你这张图很好,请你留着。



王孟源 23:51 

好,没问题,大家请看这上面这个是去年年底俄国换了总司令之后,他们所建立的防线,是要塞化的防线,由 1000 多个排阵地组成,在这里你可以看到他们连成几道防线,越往西边就是越靠他们的主攻方向呢,这个防线就越多,多达 5 道,越往东边靠 Donetsk 他们就越少,只有一道,实际上只有一道,就是 1000 多个排阵地,要塞化的排阵地。然后你可以看到靠西边的那个主攻方向,它的第一线到防线就是在双方的交界线附近,就是推得很急,但是到了东边那些次要地方向,就是不太重要的方向,它就只有一条防线,而且这防线距离前线很有很大的距离,就是有大约 15 到 20 公里的距离。那他们在6月 4 号开打之后的头两天,就是由全线在东线跟北线开始进攻,然后南线其实是在6月 6 号才认真打的,然后打的时候它的主力是我刚刚提到 9 个北约旅之中的一个,就是 第37 海军陆战队,那他这个攻的方向是这个南线的东部,就是次要的那个方向。同样的,这也是为了要迷惑敌人,先攻击次要方向,然后最后才在主要方向做突击。



王孟源 25:43 

那这里他们头四五天毫无进展,然后又隔了两天,就是到了6月 8 号、6月 9 号,他们的主攻方向才动手,而且动手的是他们那 9 个北约里面装备最好、最精良、最精锐的47,就是这个 47 旅才装备了豹 2 坦克,因为乌克兰到现在收到的豹 2 坦克还不到 30 辆,连装备一个旅都很勉强,然后结果他在三天之内就被打掉了超过一半的,超过了一半的豹 2 坦克。



王孟源 26:27 

那他的战果后来又继续打到现在,他们宣称的那四个、五个或者六个、七个的村庄都在哪里?都在东边的那个次要方向,就是那个地方刚好有 17 公里的前沿阵地。这原本就是俄军不太在乎的,就是根本还没有碰到真正的防线,那个地方的所谓的村庄都是已经无人的村庄。



王孟源 26:58 

然后这就是原本就是要让就让,反正你要碰到我们真正的防线,你必须要先越过这 17 公里,这 17 公里结果被到目前为止被乌军占领了大概四五公里左右, 1/ 3 左右,不到 1/ 3。那这不到 1/ 3 就是目前唯一的战果。他们那个在主攻方向连一公里都没有推进,最重要的指标是那 1000 多个要塞化的排阵地,你知道乌军打下来了几个? 0 个,一个都没有打下来。



唐湘龙 27:35 

这基本上他还没真的碰到俄罗斯的阵地。



王孟源 27:39 

连防线,你看那个主攻方向有 5 道防线,他连第一道防线都还没有碰到。像这样子我刚刚已经解释过那个战略考虑,就是乌军这一波其实只动用了 8 个预备队的旅,就是全部 22 个预备队里面动用了 8 个旅,然后打了这一个多礼拜,其中有四个被打残了,就包括我刚刚提到的 37 跟 47 旅,这个必须要撤回去,但是你如果考虑到这样子他们还剩下 18 个旅,



王孟源 28:20 

然后要做像样的攻势,一次至少要,至少要比这次要强一点,那大概就要 12 个旅,对不对?那一次进攻三四天可能有一半会被打残碎,就是他还可以再尝试两次。但是我觉得是,他们要是能够碰到第一道防线就已经很了不起了。到目前为止是根本连第一道防线都还没有碰到。那我刚刚提到一个礼拜前就有朋友来私下问我说,这目前的战局怎么样?我说我其实觉得很可惜,如果我来指挥的话,如果我来指挥俄军的话,我就会把前线前沿阵地的那些部队减半,为什么呢?因为我刚刚已经解释过,这个乌军投入的预备队越多,死伤就越多,消耗的就越多,而且目前的阵亡比大概可能超过 10: 1,所以你必须要引诱他们深入才好。你怕的就是这个乌克兰,我刚刚说的如意算盘说打了一波两波,然后就跟美国组织说,真的打不下,我停手了,对不对?然后啊,手底下还有还有十几个预备队旅,那这样子你打了跟没打差不多,就是乌军多了一些死伤而已。那所以我说如果我是俄军的指挥官,我会先诱敌深入,让他们第一周第二周就先有些战果,但现在目前这个战果是0。那这样一来乌克兰到目前还没有真正承认说他们已经开始,还没有官方承认他已经开始打春季攻势,就是他们还在指望他们的美国爸爸容许他们收手,所以这个里面还有一些不确定性。那但是我先指明一下,再提高另一个层次来解释这件事情。我的优先考虑是中国、台湾跟全世界的利益。



王孟源 30:39 

所以最好的事情,就是我上个月已经讲过,最好的发展方向是这一次乌克兰把他二十几个旅的预备队全部打残,然后俄军可以长驱直入,然后压迫欧洲来做政治反思。这样一来这个美国就没有在台海出手的,没有在台海挑起事端的底气。但是你如果是做决策的不是我、而是普丁,普丁的考虑,是俄国的利益、从俄国的利益来看,它最好的策略就是这样,温水煮青蛙。



王孟源 31:21 

慢慢的打。为什么呢?因为我们刚刚看到,你刚刚也提到就是过去这一年多,北约的武器越给越多、越给越先进、越给越高级。下一个要给的是什么?已经确定了,是 F16,是你战斗机都给了,以后要再升级就不可能了,就不可能只靠提供武器,那要再升级,现在如果俄国在未来一两月真的把乌克兰的预备队全部打光,然后在线一路的往基辅靠近。这样子平推直入的话,北约唯一的可能反应就是让波兰参战。波兰参战之后,下一步就是从波兰起飞的战斗机直接支援。那这样子就是,这会分两步,第一步是波兰派军,然后波兰军在乌克兰作战,然后再下一步就是波兰军从波兰本土作战,那一旦波兰军从波兰本土作战,就是北约下场了,那这个战争就升级了,就真的会有核战争的可能。所以普丁的优先考虑就是继续占据优势,但是仍然是微弱建立(?),这个仍然容许西方的这些主流媒体撒谎遮掩的优势,然后不要一次很快的把战事结束,否则的话你一下子占据了大的优势就基本上必须要升级。



王孟源 33:18 

所以我把这件事情讲清楚了,那大家未来几个礼拜继续的看。当然现在美国人是不可能让乌克兰收手的,就乌克兰的如意算盘是说让跟美国爸爸回去喊疼,然后说可不可以不打了,但是美国人是一定要,一定会让他们至少再尝试一波,很可能会尝试几波。但是这都不可能打出什么结果来,因为你 8 个旅连他们的边都没碰到,你投入 12 个旅,顶多不就是在那个第一线那边刮刮皮而已,反而是俄军只想要做杀伤,他们并不想要把乌军的后备力量全部打残,因为真的打残了以后你这个 F16 一样,绝对也是没有用的。你看现在这个上个月的那个爱国者飞弹,被证明一点用都没有之后,Raytheon这样的股票一下掉了,一天之内掉了百分之十几。你接下来 F16 很明显的也是准备要。那个生产商的股票,要往下跌,对不对?



唐湘龙 34:37 

我们刚讲的就是说爱3 是雷神的系统,那因此跌的是雷神的股票,但是 F16 或者未来的 F35 这都是这个洛克希德马丁的系统。那军火商借着俄乌战争在乌克兰战场上面的大小事情都被拿来当作军火商的宣传,就卖武器的宣传,这个大概是西方的宣传体系里面来讲,大概最明显的,不过这接下去这个问题我要请教请教孟源,就是说这终究这一波我们看到了一些新装备,新的主战装备进来,虽然 F16 还没有进场,可是像豹 2或者布莱德雷德这些战车或者是英国的所提供的这些步兵战车都已经在乌克兰看到了。



唐湘龙 35:27 

那这些事前他是北约的主战装的装备,那讲的也都是风风火火的,看起来俄罗斯应该很害怕,可是即使还没有真正的碰触到俄罗斯准备已久的前沿阵地,可是这些的装备进场了之后,因为我看了不少俄罗斯方面的视频,而且那些视频都是俄罗斯后来大概受不了,做了深入的解读告诉你说这就是豹2,那这就是布莱德利,而且都被打烂了。为什么这么的不堪一击?



王孟源 36:03 

因为像这一次俄乌战争这样子的先进机械化国家的全面正面对抗,因为你看这次其实是俄国跟北约的对抗。事实上从二战之后没有再发生,没有发生过,已经有了快 80 年没有发生过了,所以有一些特别的方面,他们没有很好的经验。那其中之一就是到目前为止没有很好的反地雷手段。那其实两三个月前,俄军曾经想突击Vuhledar一个小镇的时候,也是有一个排的战车被困在雷区,然后损失了五六辆装甲车,没有像这次乌克兰这么惨。但是那个时候就被人家笑说你俄军这个不行,你们的排雷装备不行。其实不是这样子,你排雷车排在开在前面,它所能开出来的就是一条车道。那你这样子一来,你的所有的整个一个排或者一个连就必须要都跟着这条车道走,就完全。



唐湘龙 37:35 

就是纵队行进。如果以纵队行进,其实是战车的兵团,风险最高的时候。



王孟源 37:43 

对,你只要被对方的无人机看到一阵炮兵覆盖,那你就完蛋了,就卡住了,因为你连绕过前一辆都不行。你如果试着,这一次他们两次, 3 个月前俄军是这样的,这一次乌军也是这样子,你第一辆那个排雷坦克被人家用反坦克弹导弹或者是用炮兵摧毁之后,你后面的那些车辆都没办法绕过他去。你因为一绕过那一条车道,你就是到了雷区了,所以就变成十几,这次最惨的一个就是十几辆,整个连,就不见了。整个连这样的就堵在一个地方,就是因为他们在雷区里面,前面的那辆排雷车被摧毁之后,后面的就想要绕过他去,但是你一绕就是全部都死在雷区,那这个发生了好几次。然后乌克兰其实就把手中 6 辆排雷坦克中的 3 辆,这个是芬兰所捐赠的 6 辆豹 2 式排雷坦克,其中的 3 辆一字排开。因为你想想看他们的,我想他们的思路就是说你只有一条车道的话没办法动,那你三辆并列的开着,至少还可以机动一下。



王孟源 39:03 

结果因为他没有空优,所以就被俄军的那个,俄军的直升机很简单的用导弹打掉了。(唐,卡28)那直升机的反坦克导弹射程一般是 12 公里。那为什么是 12 公里?因为你这个普通的机动式防空炮跟单兵的那个肩射导弹,肩射型的,都只有三四公里的射程,所以你只要超过三四公里,你根本就不怕那种。



王孟源 39:43 

所以他们第一次有那个很惨的 11 辆装甲车死在一起之后,他们隔了两天以后用三辆排列坦克并列着推进,结果留下的是三辆排雷坦克再加一辆排雷卡车,全部死在同一个地方。就是,哈哈哈,所以排雷这件事情问题是很大的,就是你必须要有全面空优,绝对空优,才有可能好好的排雷。那我想这一次这个批评俄军,他们装备了或者组织不行的人呢,都是所谓的armchair general,就是坐摇椅的将军,自说自话,当然大陆上有很多军事的论坛,或者是所谓的军事记者,他们也是胡说八道。一直到前几天我还看到观察者网上有一篇文章,说什么俄军的所用的炮兵弹药不够,真正打起这种全面机械化战争,历史上有哪一个部队说他们的弹药够的?这不可能,对不对?这是鸡蛋里面挑骨头。那事实上就是有 80 年没有打过这种,人类有 80 年没有打过这种战争了。所以你总要让人家尝试学习嘛,我觉得俄军其实已经打得不错了,就是以他们这种大而化之的性格,原本就一直都是在战争中学习的,我想时间也差不多了。



唐湘龙 41:27 

所以我补充点。所以您的意思就是说现在的俄罗斯在他的,因为他到现在为止他还是叫特别军事行动好,就是换句话说,他对他的战争目标的设定并没有改变。过去一个多月比较大的改变就是瓦格纳从攻下巴赫穆特之后,瓦格纳就退涨去休整了,那第一线基本上就是俄罗斯的这个正规军在经营。俄罗斯的正规,你的意思俄罗斯的正规军在他的战线上面的经营是没有问题的,就是说面对到乌克兰或者北约的这些军事攻势的时候,俄罗斯的军规在正规军基本上是可以打的。



王孟源 42:06 

对,就是Wagner其实是所谓的 storm trooper,就是很特别的、强攻敌人工事的那种重装步兵,这在一战的时候很重要。那二战的时候就全部都机械化了,对不对?那这一次之所以打得很奇怪,需要这种storm trooper,是因为信息化,我们到了信息化的时代,机械化装备在信息化环境下,那个 UAV 无人机满天飞。你想想看,乌克兰跟俄罗斯两边都是几万、几万的小无人机,每个月这样子用,上个月有一个统计说乌克兰每个月损失3万架无人机,这个是为什么?大家如果要买股票可以去考虑买大疆,因为大疆是两边都用。一个月几万架他很消耗对不对?



唐湘龙 43:04 

那就不用买买买什么军规的?他就是一般民用的去改装而已。对,对,就是。



王孟源 43:08 

大部分的那些无人机是民用的,就是有 80\% 是这种小的、这种民用的无人机。那你在这种信息化、全面信息化的环境下,你机械化部队真的是没办法用凝聚成一个铁拳击穿,因为你这个铁拳一旦凝聚出来,他马上就看到,马上就是炮兵覆盖,对不对?或者没有空优的话更惨,被人家直升机的那个导弹覆盖,然后再加上那个有很多智能芯片,所以你的那个单兵导弹现在的效能提高非常多,比二战提高太多了。所以你现在打的,所以才会现在打的像是一个一战跟二战的那种很奇怪的融合体,就是你有机械化装备,甚至有空军跟直升机,但是大部分时候反而是像那个一战的那种战壕战,所以才必须要像Wagner那这样的storm trooper 的就是强攻部队。



王孟源 44:17 

我们时间不太多了,我赶快提一提 AI 这个问题。 我们目前所在的这个 AI (世代)...其实 AI 从 50 年代就开始有人设想, 70 年代就有第一波热潮,其后大概每隔 20 年会有一波热潮,最新的这一波热潮才是第一次 AI 有实际应用的热潮,它的滥觞是 2012 年的时候。那个时候 NVidia 推出了新一代的平行(并行)计算卡。就是那时候——以前的那个所谓的显卡——他们看到有其他的用途,所以把它变成一种通用的...创造了一个通用的程式语言,可以把这个显卡拿来应用,AI 刚好是这里面的应用之一。因为这些显卡有很高的平行(并行)计算能力,所以在 2012 年开始,这一波的 AI 就忽然有了很多突破。



我说忽然有了很多突破,要再解释一点。这个可能是技术性比较高一点,就是这一波的 AI 所用的技术是所谓的 Deep learning 深度学习。它的用法就是把 AI 做成一个黑箱。你真的并不了解深度学习它的思考能力是怎样,(深度学习)其实是训练好几层的所谓的 Neuron——这个技术细节我不谈了,我想对大部分听众来说没有什么意义。但是( 深度学习)有几个特点,我要先讲一下:第一个特点,它需要海量的数据去训练。大家回想一下几年前,他们下西洋棋跟下围棋,AI 突破的原因是因为规则确定,所以他们可以让 AI 跟 AI 自己下——它不但可以看以前的棋谱,而且可以让 AI 跟 AI 自己下,一下就是几亿或者是几万亿场的棋。



过去这几年非常抢眼的这个所谓 ChatGPT,这是所谓 Generative AI,它所需要的数据更是离谱。GPT 这个缩写,它其实是 Generative Pre-trained Transformer 的缩写。Transformer 是处理文字的、处理语言的一个技巧,这是 2019 年 Google 发明的。 2019 年第一代的 Transformer 出来的时候,他们训练用了 10\^19 个训练数据——1亿是 10\^8——10\^19 就是 1000 亿亿个数据。到了 2021 年就是两年前,一个画画的 AI 程序(叫做 stable diffusion)出来的时候。Stable diffusion 用了 10\^20.5 个数据点去训练,也就是 3 万亿亿个数据点。去年 ChatGPT-3.5 第一次能够写东西出来的时候,它用了 10\^23 个训练点, 1千万亿亿个训练点。到今年升级到 ChatGPT-4,所需要的数据是 10\^25个,10 亿亿亿。大家想想看这有多可怕?我们人体里面有几个原子?差不多就是 10\^25 个,所以这个是非常惊人的,比天文数字还要惊人的。



我再解释他第二个特点,就是为什么需要这么大的的数据来做这些运算。因为他用这种所谓的 Brute force 强力去突破 AI 智能,用深度学习去强力突破产生智能的假象。它这并不是真的智能,而是智能的一个假象,它是一个涌现的现象。我在博客里面有讨论,但我在这里再简单地介绍一次。所谓的涌现现象就是你在不同尺度之下会有好像质变的感觉。一个常用的例子就是:你如果只看水分子的话,这完全是一个量子力学的现象。是一个水分子跟另外一个水分子,然后又有另外一个水分子,对不对?但是这个水分子的数量够多,比如说 10\^20 个水分子在一起的时候,它就变成一个流体——就是我们平常人想象的水的那个样子——这就是一个涌现的现象。这个水的性质你又觉得它会从高处往下流了,然后它会有波动,然后你的手指去搅它会这个样子。这些现象你如果只看水分子的时候都是想象不到的——1个水分子你想象不到,2个水分子作用你也想象不到,100 个水分子你也是想象不到,要有10\^20 个的水分子在一起,它一下子就变成这样子了。同样的,我们现在在用的这个 Generative AI 也是这样子。你只送一个数据进去的时候,他就是在胡说八道,你送 1 亿个数据进去的时候,他还是在胡说八道,你送 1 亿亿个数据进去之后,他还是在胡说八道。超过一个门槛之后,它忽然就好像有智能了,讲得像模像样,所以这是一个涌现的现象。



我特别提这一点,就是要跟大家讲一下:除非 20 年后有一个新的世代的 AI 出来。否则,因为目前这一个世代的 AI 它显现的智能是一个涌现现象,所以你没办法避免它对大数据的依赖。未来的 3-5 年可能会把这些数据的需求压缩一点。比如说 ChatGPT 可能不是 10\^25,而是 10\^20 就够了。但是你不可能避免,它就是需要这种很大的数据。



我提这一点是为什么呢?因为在博客有读者问我说:“那是不是所有的行业都可以请老师傅把他的经验浓缩成 AI,由 AI 来代替这些技工?” 我的回答是说不行。因为你凭那个老师傅来写经验的话,你这个数据点一定是很少的几千个、几万个。你甚至花 100 年写下几亿条经验法则都不够。所以 AI 能应用的就是有天量数据点的东西:像是文字、像是图片、或者是音乐这种东西,都有可能。因为过去这 30 年的互联网发展,真正有 10\^20 的数据点可以拿去训练。否则没有这种大数据点的话,这个 AI 是不可能。我今天要讲的就是 AI 的局限。大家现在都在讲 AI 的可能应用。 AI 的应用是非常有意义的,它就是未来这十年最重要的一个科技革命。但是因为它的这些突破,别人已经讲得很多了,所以我今天要讲它的局限。我不是说它没有用,我承认它非常有用。过去 10 年我一直在说,(国家在)科技上面的投资, AI 是应该是一个重点。但是因为今天时间有限,所以我会专门讲它的局限。



AI 的第一个局限是它需要大数据,第二个局限是它并不是真正的逻辑思考。它是一个几率的估算,就是它经过几万亿亿亿个数据以后,它统计这个几率是什么。比如说我现在讲一句话,我现在讲,然后它就把“我现在讲”这几个字。参考他以前训练过的 10\^25个句子,来猜下一个字最可能是什么。结果它发现有 80\% 的几率是“一”,然后它就把“一”填下去。然后它再继续地说,我现在讲“一”,下一个字最可能是什么?“句”,那就填“句”。然后“我现在讲一句”,下一个字应该是什么?是“话”。这个 ChatGPT 的作用原理就是这样的。



唐湘龙 55:05 

所以按照现在我们所谓的 AI 的人工智慧,孟源的意思说,就如果没有一个庞大的数据库在后面,所有摆在你面前的 AI 仍然是个白痴。



王孟源 55:19 

我举个例子好了。我们大家小时候都要考数学,对不对?我们台湾三四十年前的基础教育下,真的念完一个中学的时候,你已经考了几千次的考试,对不对?但你要准备这些数学考试的时候有两个办法:一个是你真的懂这个数学的原理,你可以看到一个题目以后,你可以用逻辑去思考要怎么解。



王孟源 55:47 

另外一个办法就是你把那个参考书拿来,同一类的题目做 1000 遍。然后你就要记得哪一个公式、有哪些公式,然后你在临考场的时候就想、就猜哪一个公式适用,然后把那个公式拿出来硬套。你并不懂得为什么要用这个公式。但是你就记得,这一类的问题好像就是都是用这种公式。现在的 AI 就是这第二类的学生,它并不是真的懂这些问题。我为什么会说这一点呢?因为去年 ChatGPT 刚开始流行的时候,我也是被吸引去试用一下。我试用的时候所问的第一个问题就是,请问你怎么质数分解 1007 这个数字,结果它就开始胡扯了。它就说这是一个质数,然后又说这是 7,乘以什么什么什么,但是你乘起来根本就不对,一千零几,然后我说不对。



王孟源 56:59 

再来,这样子试了十几次,它给我十几个错误的答案。就是说它根本连小学程度的算术都不会做,它唯一会做的就是假装说的头头是道。根据它在互联网上所看到的那些大数据来猜一般人会说什么。所以这里面你就必须要了解它有两个局限:第一个它并不是发掘真相,它只是人云亦云,对不对?它一定是只是人云亦云;第二个就是,它不在乎这个是对的还是错的,它完全不在乎这是对的还是错的。



唐湘龙 57:48 

因为他也没有能力辨别那是对的还是错的。



王孟源 57:50 

不能辨别这是对的还是错的。所以我再给你一个例子,上个月有一个纽约的律师——因为美国的律师他们工作都是很累的,那个一天到晚要读一堆很枯燥的东西,然后要写一堆很枯燥的东西——他叫 AI 帮他写,他就偷懒叫 AI 帮他写,结果那个 AI 写的头头是道、洋洋洒洒的几十页,然后引经据典说有这个有什么案例,那边有什么案例?结果提上去以后,那个法官一开始的时候也是被唬得一愣,后来觉得奇怪,怎么十几个案例我从来没听过。(哈哈哈哈)



唐湘龙 58:32 

结果是被 AI 骗。



王孟源 58:35 

全部都是 AI 编造的,没有一个真的案例,所以这是第三个局限。那根据这些局限我做几个预测了。第一个预测就是接下来两三年除了继续把这个量变成质变的这个涌现现象的门槛稍微往下压一点,就是 10\^25 是太离谱了,10\^20 还勉勉强强,这是第一个方向。第二个方向就是针对我刚刚提到的两个例子:一个是它没有逻辑思考能力,数学题它没办法。这个补救的办法就是现在网络上有能够你问它说怎么样分解1007,它帮你做出来的网站,它是专门做这个,所以你可以试图让 AI 跟这些用手工把逻辑写进去的程序,小的 module 组件联合起来,这样子 AI 就可以回答这个问题。然后第三个发展方向就是,AI 如果继续发展下去,像现在这样子对错不分、真假不变的话,会是一个很大的限制。有些应用是需要确定、要求真的。我觉得在未来 5 年能够发展出一个新的套件,说要求必须是真的。这还是有一点可能,就是不一定能做得到100\%,但是还是有一点可能。就是你将来去接 ChatGPT 的时候可以说,我要求的不是普通的ChatGPT,不是那个随便打屁的的ChatGPT,而是要精确的ChatGPT。这在 5 年内是有可能做出来的。



唐湘龙 01:00:42 

好,孟源刚提到的就是:其实这是在当 ChatGPT 现在被广泛地讨论、广泛地应用,以及在认真地思考它将来在各个领域的应用的可能性。它确实是一个有关于 AI 这个概念大爆发的时代,可是当过度讨论 AI 之后,过度的盲目以为 AI 即将取代了人类的所有的这些思考,跟对那一体的处理能力,那恐怕有一段很大的距离。



所以孟源刚刚讲的就是说:它的局限性是因为你必须要...它是有前提的,它必须要有一个超级大的数据库才能够满足它基本的,而且很多时候仍然是一本正经的胡说八道的内容,即使那个数据库够大,因为那个数据是五花八门海量数据,你们本来就很多是错的。当你给他的题目很刁钻的时候,他也只能够去乱抓东西,凑一些东西来敷衍你,骗你不懂。因此不管是医生在诊断、开处方,律师在写状子,都可能会遭遇到这样的问题,他无计可施,不能够满足你他就乱抓东西。总而言之,他以那种交差的态度,可是它能够满足基本的需求。



就像我这两天看到大陆的一个一条新闻,他们用 AI 去做今年的高考的卷子,做数学卷,算一算之后, AI 大概可以考得上像湖南大学,或者勉勉强强接近,浙江大学还有可能考不上,AI 还考不上浙江大学。就如果跟今年的高考生比的话,那你说像什么北大了,清大那个都还是沾不到边的。所以 AI 本身在所有的应用当中来讲,确实满足了很多的要求,也启发了大家很多的思考。但它这种局限性,它是 Deep learning 之后所呈现出来的,但是它仍然没有办法 Deep thinking 就是这种。



王孟源 01:02:36 

就是你刚刚讲的。它目前还没完全没办法满足医学院跟法学院的要求,所以它能够帮助的就是润饰一下那个字眼。但是 AI 非常适合商学院,因为商人他们写的报告或者什么的原本就是胡说八道。然后刚好这方面的应用领域,就业数字是最大的。一般公司运作里面,写的报告是数量比那个律师写的或者是医生写的东西要多。



唐湘龙 01:03:12 

好,您请说。



王孟源 01:03:17 

所以我认为是,它在商业上能够摆脱我们目前,就像当年我们有人类开始有畜牧,有了牛跟马可以代步,可以来用来犁地来取代人力的时候,或者来用驴来拉磨子的时候。我们现在是把最枯燥的、写稿子的那一部分可以交给 AI。像我儿子他现在在大学,刚好在大学他们现在写报告基本上都是写完之后让 AI 润饰一遍。这都已经算是最认真最努力的,普通学生就是直接叫 AI 帮他。



唐湘龙 01:04:15 

对了,就现在的孩子或者未来的年轻人,你可能需要具备的技能不是去研发AI,而是你要如何去 Queue  AI, 叫 AI 能够精准地回应你的问题,因此你问 AI 问题会成为一个很高深的学问,能够精准地问 AI 是重要。



王孟源 01:04:34 

你精准地问 AI 是一门学问,但这个是可以自己摸索训练。最大的问题是我刚刚提到的那三四个局限是,这你必须是要学理工的人才能很好地改进。



唐湘龙 01:04:47 

没错,就同时 AI 给你的答案,你要有能力再去 go through 的去把它去把 review 过,从你的专业去告诉,去知道里面可能有一些的错误。



王孟源 01:05:01 

我现在要顺带提到上个礼拜出来的一个新数据,就是中国大陆他们 4 月份的失业率。他们 4 月份的整体的失业率是 5.2,一年前的失业率是 6.1,所以整体的失业率明显地下降。



唐湘龙 01:05:24 

那为什么年轻人增加这么多?



王孟源 01:05:27 

年轻人的失业率是 20.4\%,这是历史上的最高点、远远最高点。四年前,在新冠开始之前,2019 年的4月,他的年轻人(16- 24 岁) 的失业率是10\%。四年就成长了一倍。而且我跟大家讲一下,这个失业率的计算全世界都是统一的,必须是那个人本身主动地在找工作才会进分母,你如果不找工作是完全不算。



唐湘龙 01:06:09 

就是你不想找工作的,虽然你是这个年纪,但是不会把你算到失业里面。



王孟源 01:06:16 

对,就是你如果是就学,或者在军队里面服役,或者就是在家里面吃爸妈,根本就没有再找工作,一天到晚在网吧里面上网的,这些人都不算的。所以这个失业率是真的就是 20\% 的人在找工作,然后找不到。那这个问题在哪里?过去这两个礼拜,我在我的博客上反复地、详细地从多方面论证:这个问题主要就在于中国的高等教育跟产业完全脱节。他们训练了太多的文科,你光看过去这4年中国的高等教育,4年前他每年的大学生毕业800 万人,到今年大学生要毕业 1160万人,提升了40\%。4年毕业生就提升了40\%,而且这里面大部分是白领,就是文科出来打算要找白领工作。所以过去这几年大陆还有一些新闻,就是说考公忽然非常的吃香。大家都要当公务员,就是因为你不可能容纳那么多没有用的文科生。那我为什么讲了 AI 再来谈这件事情?因为 AI 本身取代的就是这些写文字稿的基本工作。



唐湘龙 01:08:12 

就是取代文科生,哈哈哈哈。



王孟源 01:08:15 

那是现代文科生存在的唯一意义。所以你一方面需求继续的萎缩,但一方面供给还在以每四年 40\% 的这个速率提升。这如果没有促成失业率暴涨,这才是奇怪的事情。所以大家考虑一下,原本中国就有很严重的错配问题,车间还有工厂里面找不到工,但是大学毕业生却找不到适合的工作。然后再经过去这4年的胡搞,中国的国务院里面最烂的单位不是中宣部,而是教育部。



王孟源 01:09:06 

现在真正解决的办法就是大幅缩减这些文科,你一年大学的毕业生不应该超过 800 万。你每年 1100 多万的毕业生绝对是会有大批失业。那这些大批失业不但浪费了4年的时间和4年的学费,而且这些就是能够在网络上胡说八道、满腹怨气,然后接受外国人渗透的这些人。国家花了这么大力气、浪费了国家的资源跟私人的资源,培养出完全没有用,反而是对国家社会有负面影响的人,你说这种政策合理吗?



唐湘龙 01:10:03 

当然非常有问题。



王孟源 01:10:06 

而且他们原本就是在每一个年轻时代里面中位的存在。就是他们不是顶尖的人才,不是那种可以去念顶尖学校理工科的人才,但是他们又不愿意去做工人,那这些中位的人才是国家政策必须要针对、好好利用,让他们对经济能够做出有效贡献的人。他们也能够有合理的收入,然后有一个美满的人生。他们是被国家政策害的,他们不是因为传统文化重视教育。传统文化重视教育是合理的,是全世界最先进的观念。是大陆的教育部的错误政策害了他们,害了国家,害了这些年轻人。



唐湘龙 01:11:06 

孟源后面讲的这一段了,我觉得很有启发性。好。当然前面不管我们在谈俄乌战争,让大家了解战场的事实的情况,孟源刚刚讲的跟我自己去想一些的管道,去做功课,去兜一些西方媒体背后的东西的理解的情况差不多,其实我经常都需要借助于孟源的分析来让我比较有底气一点,印证了我看到的东西或者我想的事情是对的。



唐湘龙 01:11:37 

至于 AI 的部分,这个我觉得也非常有启发性。孟源刚才,因为现在大家都在吹AI,你知道这种吹有时候会吹到爆,吹到脱离现实,会总会觉得 AI 无所不能,仿佛人类即将这即将过时,即将束手就擒。网络上面充斥着许多有关于 AI 的造假或者是夸大的这些的讯息,这些是需要比较理性的辩论的。简单讲,孟源刚告诉你就是说他现在有这种在 deep learning 的情况之下,在一个巨大超巨量的数据库的情况之下, AI 是可以把自己表现的有模有样的,或者大部分的时候是堪用的。可是你以为他有那种就是说 deep thinking 的能力,那你恐怕还想太远了。



王孟源 01:12:32 

就是要等下个时代,下个世代真正有逻辑思维能力的 AI 的话,可能要 20 年。



唐湘龙 01:12:39 

是有可能的。



王孟源 01:12:40 

一个很简单的标准,你就是那个奥林匹克数学竞试、奥数的那个题目,他们都是绞尽脑汁想出来,你没办法在以前的参考书里面找到的是新题目,如果将来 AI 。



唐湘龙 01:12:55 

它可以做,那就厉害了。对对对。



王孟源 01:12:59 

那就是一个真的新世代,能够深入思考来挖掘真相的。



唐湘龙 01:13:04 

一样。就像王孟源刚丢的就是1007的这个数字,给 AI 解的时候, AI 就开始发现,它自己不会发现它胡说八道了,因为它总是要有点东西出来交给你,网络上面它也一定可以兜点什么,但是你自己有没有能力再去辨识 AI 丢给你的东西的可靠度?这又是另外一个层面的问题。



唐湘龙 01:13:25 

至于失业率,因为我们刚刚引用的失业率的数字是今年 4 月份的、大陆的青年失业率,这么总体失业率是 OK 的,表示过去已经在职场的基本上面,现在慢慢的回到自己的位置。可是青年失业率高,那一定是你本身的教学、就教跟用之间出现了落差。那是4月,那这样发生什么事? 6 月份是毕业季,毕业季之后又开始有大量的毕业生又涌向市场,所以你现在看到的数字不会是最糟的数字,更糟的数字还在后面。那过去往往经过大概一个暑假,三个月的时间,找工作的磨合期之后,有关于青年 10 月的部分大概都会在 10 月之后慢慢的沉淀下来,可是现在的数字是这么惊人,表示今年的暑假会是一个就业市场非常大的考验,目前的经济情况又不是非常好的状况下面,接下去的失业的问题,希望就是说根本的问题、教育问题、刚孟源的建议的这个要长时间的规划、重新的安排,把人才引导到的一个合理的位置上面,否则它不只是你培养的人才浪费时间而且浪费人才,最重要是它会有社会动荡的风险。



王孟源 01:14:41 

对,就是你 800 万是合理的大学毕业生, 1100 万,多了 300 多万。那这些人就是注定要失业。是那这 300 万你要怎么样引导到职业教育去?高职跟工专,你怎么引导过去?很简单,你把大学的招生名额压下去就行了,他们进不去自然就会想办法到,这是是唯一的。



唐湘龙 01:15:05 

对了,就是跟台湾的过去,就是大家都迷恋了当一个大学生,所以把过去很多的专科,其实过去有很多很好的专科学校或者是职业学校,现在在台湾几乎都消失了,那个情况,希望大陆不会走上这一步,因为台湾在过去错误的教改政策之下,今天你回头去看,付出了非常惨痛的代价,培养的大量的无用人才。



王孟源 01:15:32 

这是自己打自己的脚,而且都是盲目地去学美国。美国有这个本钱啊,因为他从 80 年代就开始经济虚拟化、空心化、金融化,你这个没办法占据,他们有美元,台湾有美元吗?台湾发行的货币叫做美元吗?大陆货发行的货币是国际储备货币吗?没有那个本钱、你不想要产业空心化,你就不要搞这些胡说八道的。



唐湘龙 01:15:58 

孟源讲的这一点,他当然还另外一个就大陆应该去思考的公共政策问题,就是说美国相对来讲是一个对于精英移民开放而且鼓励的环境,所以许多其他国家培养的精英,美国可以收纳来作为这个使用,所以它对高端人才的培养跟增补,它低端的人才那价格压得非常非常低,所以还是有很多很低端的移民愿意的涌进美国,但是在大陆是没有这一块的,大陆所有的人才绝大部分都要靠自己培养。这个在教育政策当中必须要好好的思索的。



唐湘龙 01:16:35 

好,今天,今天的时间因为已经都耽误了孟源的时间,但是我发现很多的听友们所提出来的问题还没有机会请教孟源。来先感谢我们的一些的一些的朋友,一些听众朋友。 proud of can 001 感谢江汉,谢谢谢你。然后所罗曼陈,感谢你们都超爱孟源了。哈卓琳特,感谢波尔纳胡,谢谢你。另外这个冯一伟说关于中国青年失业问题,温铁军教授说现有中国高教曾在亚洲金融风暴是为了解决当时失业问题,经历了一次大规模的扩大招生,让本来应该进入就业市场的青年回到了学校,而这些增设的学科都是服务于当时中国沿海的外向型经济,也就王博士所说的垃圾文科,这些学科本来就是欧美资本主义帝国培养打工人的传声筒,但在当下的工业资本及金融资本过剩的背景之下,美国主动对中国发起脱钩,自然这些专业的学生就没有出路了。因此中国才将国家战略转向了陆权战略、推出一带一路,希望能促进国内的大循环。不过那个转型会是非常的痛苦,而且非常长时间的,牵涉到的人是非常多的,尤其是年轻人。好,我们再看他的看复陈。



王孟源 01:18:05 

刚刚那位留言的,很明显的是我博客的长期读者。



唐湘龙 01:18:08 

真的吗?冯一伟,刚才都是我跟对这个讲的对,我就我看,唉,讲的头头是道,而且对孟源的刚刚的话显然是心领神会。好,那肯夫倩,他说先生多,就是孟源多次提到大陆的金融系统需要整顿,今天的新闻大陆央行的原来的副行长范一飞被双规了,他在问孟源说这是不是意味着金融体系的整顿已经开始了?



王孟源 01:18:37 

整顿有两层,第一个是纪律,就是他们拿钱的问题;第二个是眼界的问题,就是很你即使是那些不拿钱的,他们还是受美式教育、



唐湘龙 01:18:49 

没错。



王孟源 01:18:53 

接受图利美国的观念,这里面最大的一点就是我刚刚提过的,他们忽略了那些背景,就是美国作为世界现存的霸权,然后这霸权里面还有刚好就是国际储备货币的地位,这个不对称的现象会造成更大的扭曲,他们是完全忽略的,结果他们人民银行被管理的好像是美联储的北京分行一样,原本这都是卖国的行为。我一直都说愚蠢要比贪腐还要恶劣,这个就是个典型的例子。



唐湘龙 01:19:34 

好,当然这个概念大概就是说你如果要学武功,那你可以上少林寺,你可以把武功可能学得不错,但是如果你在少林寺学的武功,你要打赢少林寺,那就很困难了。所以你跟美国学的一切没有错,能够把你发展到某一个水平,如果你专注投入的话,不过如果你要打赢美国的话,你必须要学其他的招数,跟美国学的招数,要打赢美国是困难的。



唐湘龙 01:20:00 

路易斯梁这个所问到的就是说对香港发展前景的看法,不是问题很大。我在锦梦园,下回再请梦园,我们再好好聊聊这问题。范志林提到就是说他在,他从八方论坛开始关注孟源先朵内违禁,除了感谢香龙把好多嘉宾都邀请到,观点充实我们的视野跟知识。好,也就支持继续把节目做深做广。



唐湘龙 01:20:24 

好,这是志林兄,谢谢那BLDM,感谢螃蟹好,螃蟹,这是我们的资深听众。他说信息化战争的难点不在于无人机和AI,而是在于野战信息的汇总跟判断、整理之后再发出的瞬间流量跟集成计算的能力。好,同样是螃蟹,他说汉语的这个语言的境界,语境会把 AI 给逼疯了,所以英语可能比较容易处理,但是汉语很难处理好,那就各种的发音跟这种语义当中的混淆,它必须在一个文化的氛围里面才容易学的好。那 AI 在中文的语境能不克服这个问题还有待观察。 WK 骂谢谢好跟孟源问好。9月初阳,他说上个礼拜他的女儿在 ChatGPT 问了一个问题,他说明月几时有,是谁写的? ChatGPT 说是李白,然后把静夜思复述了一遍,然后我们就估计,他说他们两个人都很凌乱,他说显然 GPT 的语文是没有学好。明月几时有?怎么会是李白?明月几时有把酒问青天这苏东坡的,但是他告诉你说是李白,他就发现很正经的胡说八道。大部分中国人,如果你问他一首诗说是谁写的,他讲不出来,他说李白,因为他只知道李白, ChatGPT 可能这个问题好, Jessica 感谢PL,感谢他说推荐对深层式语言大模型的影响。



唐湘龙 01:21:58 

感兴趣的同学去看最近新出的一篇的文章,就叫做 arXiv 的这个论文,他说 The Curse of Recursion: Training on Generated Data Makes Models Forget。其中提到深层式与的这个大语言的模型的潜在的危害。他说过污染互联网的数据让后续的模型训练的效果退化。



王孟源 01:22:25 

这篇论文我知道,他这个是,因为我刚刚讲到这个,目前这个世代的 AI 有这么大的局限。已经有人开始研发怎么样去有效率的污染他,就是你,因为他用的都是互联网的资料。



唐湘龙 01:22:40 

没错,大量的给你错误。



王孟源 01:22:44 

只要在互联网上留下几千个污点就可以让他固定的你想要他讲的事情。



唐湘龙 01:22:55 

就把 AI 变笨,把 AI 犯的错都。



王孟源 01:22:59 

把 AI 骗了



唐湘龙 01:23:01

对,就是简单讲AI, AI 是不知道他自己犯错的,以现在的 AI 来讲,他是不知道他说错了。



王孟源 01:23:09 

他是假设互联网上所有的每一篇文章、每一个句子、每一个字都是正确,是他是这样子来。



唐湘龙 01:23:18 

所以 AI 未来的纠错也是一门大学问了。好的,再来。我们看神奇光在问说王梦云博士能不能谈谈台湾的未来,他说你上次谈了一段很精彩。好,但今天时间不够了,下回连香港的问题再来 Jack honey 赖感谢,应该在加拿大。好,再来 t 缘,他说手抖,他说,从高中就开始看你的节目,看到现在还是戒不掉。



唐湘龙 01:23:46 

他说请,请龙哥跟王孟源教授喝咖啡,感谢好,今天的时间的关系,今天多这个花了孟源不少时间,可是意犹未尽,但是我每个月我只好意思就是麻烦了孟源这一次。但是我有跟孟源讲过,如果他随时有任何的议题想要有想法,想要表达的时候,我的时间随时都可以排空等待孟源。感谢在美东的时间,晚上的时间透过连线在这龙行天下跟大家分享了他的海量的知识的王孟源,孟源感谢。



王孟源 01:24:23 

啊,很高兴跟大家聊天。



唐湘龙 01:24:25 

好孟源说晚安,跟大家说说晚上,跟大家说周末快乐,跟孟源说端午节快乐,下个月见,拜拜。



王孟源 01:24:31 

好,端午节快乐。



\twocolumn[\begin{@twocolumnfalse}
\section{美国经济}
\subsection{20230721}
\end{@twocolumnfalse}]唐湘龙 00:21 

好,欢迎来龙行天下,我是唐湘龙,来,我在台湾,我在台北来,今天的龙行天下在我们的直播室里面做客的来宾是人在美国的东岸,其实跟我的时差刚好日夜是颠倒的。但是大家非常期待了,我刚看到我们的听众朋友留话说,来了,来了,王孟源来了,他带着知识和智慧来了。对,这也是我每次跟孟源在交换意见,或者我在这个单元里面,对我来说是我主持的许多的节目里面我最有收获感的节目之一。好,那也可以说没有之一。好,我相信许多的朋友们在过去,不管你认不认识王孟源,当然认识很好,可能已经是追踪王孟源,是孟源的粉非常久,但是有更多的朋友们可能是借着龙行天下认识王孟源。好,那他的知识准备是非常宽广,非常丰富的,这不是我能够办得到的。好,那今天呢,请孟源来做客在龙行天下,来先跟大家打个招呼,孟源。



王孟源 01:34 

大家好,很高兴再跟大家聊聊天。好。



唐湘龙 01:37 

来,今天孟源,我们先准备。孟源准备比较多的是在我的标题上面,美国的经济战略,那这一大块,因为美国的经济,最近你可以看得到包括市场派的这些的分析预测,乃至于官方的说法,对于美国的经济会不会步入衰退的说法正在修正。那同时美国的经济的数字面来讲,其实有许多的矛盾的地方,那这些的矛盾的地方使得过去对于美国经济分析预测的所形成的一些经验工具,在这一次似乎都不灵,比如说殖利率倒挂的问题,好像在过去,现在大家都已经钝化了,都不把它当一回事了。



唐湘龙 02:22 

好,那美国经济现况到底如何?以及到底这个操作背后有什么样一个逻辑?第二个我们有时间的时候我们再来关注,上个月在跟孟源连线的时候,那时候瓦格纳事件还没有发生。好,那现在瓦格纳已经正式的离开了俄罗斯。好,那已经到白俄罗斯去了,我们看到这个普里戈金都已经跟他说欢迎来白俄罗斯,如果你要找瓦格纳的话。



唐湘龙 02:48 

好,但是在俄乌战场当中,乌克兰的反攻到底状况如何?以及美国现在甚至于要提供了所谓的子母弹,就是集束炸弹,看起来俄乌冲突有在升高的情况,彼此之间的这个叫板越来越激烈。好,这个是后面的部分,因为它牵动到欧洲的情绪的时候,当然是一件超级大的事件。最后大概今天可能不会有时间,否则我对于美国在亚太地区的操作,对印度以及下个月美日韩的三国的峰会这些,就其实对于亚洲的地缘政治的稳定性。亚洲终究东亚,这之所以在这几十年时间发展得非常好,是因为它经历了一个历史上的长和平,几十年的时间,在东亚基本上面没有战争,但是现在看起来在东亚的情势又开始陷入到了新一轮的紧张,这跟美国的地缘操作是有关的。好,我们先从美国经济谈起好了,来,我们先看一下美国的经济的现况,现在各方的说法非常的分歧,第一个它的通膨过了吗?第二个它摆脱了衰退了吗?第三个就是说美国的股市看起来比预期中好很多。为什么?



王孟源 04:03 

Well,2019 年的时候我曾经做了一个节目,然后写了一篇文章博文来解释。就是美国当时最大的经济金融问题就是它的 QE 量化宽松达到了 8.9 万亿,这个数量实在太庞大了,所以随时只要就等于是一个火药库,里面火药装满了,随时有一个火花就可以爆炸。那结果真的后来就不到半年就爆炸了。



第一个爆炸的引头是新冠,新冠阻止了供给链和运输的效率;然后第二个是 2022 年的俄乌战争。因为这个战争一打起来,也是因为...尤其是能源方面的供应问题,所以能源的价格一下就上去了。但是美国(现在)已经平安渡过。其实我过去这一年已经讲了很多次:美国已经平安渡过这一劫、这一个大劫——就是我在 2019 年的时候说,是 2008 年以来最严重的,可能比 2008 年更严重(的大劫),要视国际的一些玩家怎么反应来决定。结果呢,因为中国人民银行的易纲决定牺牲小我,不让人民币升值反而贬值来维持那个物价。其实在 2021 年的那个时候,即使从中国的进口价提升 30\% 或甚至50\%,美国人也是会咬着牙买。但是谢谢,因为易纲先生的“无私奉献”,那个时候人民币反而是贬值的。



然后到了 2022 年,出了这个俄乌冲突。(通胀)明显化之后,也是又有另一个人挺身而出。这个人就是Von Der Leyen,她硬是把欧盟应该坐山观虎斗的一个局面搞成他们抢到第一线,结果把欧盟的工业基础,也就是它的廉价的能源供应都给挖空了。那接下来的这一年,这18个月我们看到的基本上就是德国以及其他一些周边先进国家的去工业化。去工业化的同时,他们的资本就很明显的外逃。外逃的除了产业资本之外,金融资本也外逃。那金融资本比产业资本还要喜欢跑到美国去。所以就是因为易纲先生跟Von Der Leyen女士两位的无私奉献,所以美国安全地渡过了这一劫。



今年4月的时候,3月还是4月的时候,有那个 Silicon Valley Bank 出事。那时候我也讲说这种事不会伤筋动骨的,就是因为它其实是相对小的银行,美国的金融界很健康。我待会详细地解释为什么它的整体是很健康的。但是你知道在日本有那个传统,就是有所谓的义犬或者忠犬,它会建一个小小的祠堂。我觉得美国人很可惜没有这个传统,要不然可以在美联储的门口盖一个忠犬祠。然后首先让易纲跟Von Der Leyen先进去让大家祭祀一下。哈哈哈。



唐湘龙 08:09 

哈哈。好,我让大家听得懂孟源这段讲话,这段讲话是非常重要的梗。再来。



王孟源 08:21 

在2021年跟 2022年美国逃过那一劫之后...其实我已经反复讲了一年多,就是你这个要用汇率或者国债方面的考虑来压迫美的那个时机已经过去了。所以我今天想要讲的就是将这个大结论的一些细节证据的论证分析讲清楚。所以...然后可以从这里再做新的预测,就是美国下一个经济金融方面的难关会是什么样的形态。当然最难预测的是时间点,因为时间点是可以拖的。但是就像 2019 年的时候,我说在未来三年会出事,那我可以跟你说这个未来多少年会出事,但是这一次不会是像上一次那样子是一个受通胀压迫的、一个通胀为主导的一个危机。



那这一次的新的危机是什么呢?我认为是一个财政上的危机,转移到金融方面就是它的长期利率的问题。那我们先谈一谈当前美国的经济。其实它比欧洲要好很多。美国现在的通货膨胀率是5\%点多,欧洲比它高1\%,然后它的经济成长率是1\%多,大约将近2\%,欧洲比它低1\%多,德国是在衰退的。你可以从这些看出来...比如说美联储现在好像是两难,可是它这个两难的程度就比欧元银行要轻松很多。在 2021 年他们量化宽松,两边都搞量化宽松搞到最大的时候,美国量化宽松最高纪录是我刚刚提到的 8.9 万亿美元,但是欧元银行的最高纪录是 8.8 万亿欧元,因为欧元到现在还是比美元稍微值钱一点,其实那个最高峰是比美国要高一点。你如果看现在美国美联储的资产是多少,它现在资产已经削减到 8.27 万亿,也就是大约 8.3 万亿,降了大概 0.6 多万亿,那欧元银行则是从 8.8 一口气削到 7.2。



为什么它会做量化紧缩做的那么紧张呢?因为他们的通货膨胀率太高了。那你可以想象,因为von der Leyen的政策搞的欧洲的金融资本都往美国、美元方向去窜逃,然后你再看看他们中央银行的紧缩又比美国美联储快了将近 3 倍。你可以想象欧洲现在的资金短缺的问题有多严重。所以它的去工业化不只是丧失了能源,而且也是滚雪球一样,没有了廉价的能源之后,接下来就是产业资本外逃。产业资本外逃之上,金融资本跑得更快。金融资本一跑的话,你更没有办法贷款投资。所以,昨天有人在我的博客问我说对中国的经济当前有点逆风的看法是怎么样。(这个问题)其实是很可笑的,因为你原本半年前的时候预期中国 2023 年的经济成长率是 6.1\%,那到现在刚刚修正的预期是 5.5\%,你降了 0.6\%,这有很了不起吗?事实上这反映的完全就是欧美他们的消费品跟所谓的 durable goods 进口的衰退。



那你可以看到,这其实真正的受伤的是欧美,中国只不过是因全球化的供应链的原因受到附带的伤害。要谈附带伤害,中国也不是受伤最严重,受伤更严重的是真正出口导向的代工经济项目。



唐湘龙 13:15 

包括台湾。



王孟源 13:16 

韩国还有台湾,对。所以你从什么方面来看都不应该觉得这有什么太大的问题。美国已经摆明了就是要把你干死,那结果现在美国受伤,那在他们完全脱钩之前还有一点联系,而因为这个联系你受到一点附带的伤害,这需要紧张吗?哈哈哈,这需要紧急的金融操作吗?所以我觉得很好笑。现代的这个美国、美式经济学跟金融学,他们是对民众搞资本主义,就是不管你们的死活,你们自己去搞;对大资本则搞社会主义,就是大资本出了问题以后,国家赶快要把钱投进去来拯救他们的利润。哈哈,其实是非常可笑。



我上个月提到中国教育部模仿台湾,来把工专跟高职的这些职业教育的根刨掉了,然后凭空创造了三、四百万垃圾文科的大学毕业生,所以造成它的总体失业率在下降,但是年轻人的失业率大幅升高到20\%几,这种现象才是需要改革的。你至于说紧缩房地产政策,其实就是 2008 年你跟着美国人放水,跟着美国的经济歪论而放水过度。因为那时候他放了4万亿人民币,其实1万亿就够了。但因为他4万亿,结果就把那个房地产泡沫涨上去了,涨上去以后又不敢让它一次爆掉,所以一点一点的消,消到现在总算有点看到隧道的尽头的那个灯光了。那现在就因为经济成长率从 6.1\% 掉到 5.5\%,就有人喊着要解除这些改革,其实这是大资本所控制的那些所谓的财经文字打手,他们在拱那些小资本、中产阶级民众来替他们打前阵。你知道古代的时候要攻城的时候,都是不会把你的精锐部队放在前面,都是到附近的那个农村里面抓民夫,然后把他们驱在最前线当炮灰了。(现在也差不多)。对,这些中国的股民就是大资本攻城的炮灰,所以看的是很可笑的。不过我离题了。



今天要讲的还是美国的经济。美国的经济要比欧洲要健康很多,我待会再给你一些,我现在再给你一些那个数字。首先,它目前的利率是短期...美国的国债是所谓的 treasury security,它一共有13 种,就是 13个不同的到期的时间。从1个月到 30 年,短期的最短的是一个月,然后最长的是30 年,一共有13 种,所以这 13 种就有13 种不同的利率。就是你这个到期的时间越长,它的那个利率就是、基本上是..., 30 年债券它的利率就是,未来三...,你要借钱,一次借 30 年这个时候的平均每年利率。然后你如果只借一个月的话,那当然就是现在、现行的利率。



所以这就是所谓的一个yield curve,那yield curve正常的形状是你越往长期那一端,它的那个利率越高。原因是如果我要跟你借钱,如果我一次借一天然后明天再来借,明天再来借相比一次借 30 年,这个前者对你是比较有利的,因为你随时可以停,对不对?但是我如果一次借 30 年的话,就是你等 30 年之后再来找我,这期间都不要来烦我。但这样子你这个债主的那个权益,就是他的那个隐形的所谓的 option 就比较少,所以它的为了要弥补这个债主的这个隐形的损失,你就必须,一般要给他更高的利率。所以这个 yield curve ,所谓的 normal yield curve,普通的、正常的 yield curve ,就是长期利率比短期利率要高,但是这些利率里面,因为这个债券的利率不是像 GDP 那样子根据通胀率修正过的,而就是字面上的数字。所以真正影响这些利率最严重的一个变数就是通胀率,对不对?你如果说我现在利率是...,我现在这个借债的利率是3\%,但是通胀率是 4\% 的话,那你这个实际的利率是 -1\% 嘛。因为你借了 100 块,那到期了一年之后,你只需要付103,但是这个时候 103 块只值原本一年前的 99 块,所以你事实上借钱了还赚,所以有所谓的真实的利率。



那这个所谓yield curve会倒挂。目前就是一个倒挂的情形,倒挂就是短期的利率比长期的利率还要高,那这原因就是因为你目前正在一个高通胀的情况,目前美国国债的短期利率是 5.5\%,那美国的通胀率也是 5.5\%,所以你大约实际利率是0,对不对?但是它的国债长期利率目前是大约4\%,那这里就很难说你这个通胀的预期是多少。但是我认为,我个人的预期是未来 30 年美国的通胀率大概是 4\% 或者5\%。所以美国目前的真实利率其实在短线是0,在长线可能是 0 或者是-1\% 的。那我讲这个是为什么呢?



原本在 2008 年之前,就是2008年那个危机之前,美联储只能够控制短期利率。它这个有很多所谓的window窗口、拆现窗口。这个是它...,然后它可以规定所谓的Fed Funds Rate,这个Fed Funds Rate是比国债的最短期的一个月还要短的,就是从一天到两个礼拜,就是极短期的,它能够控制的就是这个极短期的利率。所以这个极短利率它控制之后就可以间接的去影响国债的短期短线的那一端,但是对国债的长期是没有办法的。那2008年之后他发明了一个新的窗口,就是所谓的 QE 量化宽松,这个是很有名的。但是量化宽松一方面是印钱,但是他印了钱以后要干什么?他买的,这些印的新的现钞拿去买的资产主要是长期的资产,就是国债还有Mortgage-backed Security,就是房贷衍生化产品,然后后来最后还直接下场去买企业债券。



所以事实上QE跟QT你在考虑它的经济跟金融效果的时候,其实是有两面的。第一面是他把现金塞到金融跟经济体系里面去,第二个是他塞的管道是yield curve的长期的那一段。那长期的这一端跟短期的那一段有什么不同?长期的那一端其实是对产业还有长期的投资,比如说房地产这种投资。你做房贷的时候通常是要一贷...,在美国是一贷最常见的就是15年或者是30年。那因为美元是国际储备货币,所以像是它的这个Mortgage-backed Security,还有它的国债,它的长期端全世界都在买,中国也买了1万亿,对不对?是日本买了1万2千亿等等的。那全世界都在买,所以它发行起来很方便,所以在美国的房贷大多数都是固定利率的房贷。



那固定利率的房贷是什么意思?就是过去这 18 个月,美联储是在 2022 年1月开始反转,就是从 QT 转成QE(应为口误,实际为QE转为QT),然后利率从0\%一直提到现在的 5.25\%(这个当然是极短期的)。我说美联储提升利率,就是美联储现在是 QT 跟 QE 控制长期利率,然后它的那个其他窗口直接控制极短期利率,这个极短极的控制是历史上向来如此。



王孟源 24:49 

那他过去这 18 个月短期利率就上升了5\%,从百零点几...,然后长期利率,国债是从几乎是 0 就是零点几升到 4.0,看样子好像只升了三点几,但是其实因为房贷还有那个企业贷,他们还要有跳票的危险,所以它有所谓的 risk Premium,再加上这个 risk Premium 之后这些实际上一般消费者跟企业界可以借到的利率也是提升了5\%。



王孟源 25:36 

就是美国政府的借贷成本提升了3.5\%,但是一般消费者跟企业的借贷成本提升5\%,那你提升了 5\% 之后,你如果去看他们的利息支出跟利息收入的话,就是美国国内的那个老百姓的利息支出跟利息收入刚好抵消,这很有趣啊,就是你这个利息上去之后,美国人其实他的家庭是负债、很喜欢负债。



王孟源 26:20 

美国的负债率是很高的,但是因为他们也投资了很多国债,还有定期存款这些东西,所以你那个投资债券、还有投资定期存款这样而增加的收入,刚好跟你贷款所需要还的花费互相抵消。当然这不是说它的影响是零,因为你这个收入增加的是上中产阶级的收入增加,会去投资的人是上中产阶级。那真正要负债的,就是美国人的家庭负债主要的有五、六项,就是像 credit card信用卡,然后还有汽车贷,然后还有学生贷,然后还有房贷,然后还可以用房子再做抵押、做第二次贷款,这叫HELOC。



王孟源 27:16 

这些通常是下中产阶级的问题,所以你可以看到这其实是一个财富转移,但是这个财富转移问题大不大?不大。因为你看他这些贷款、所有的这些负债,它的利息增加之后,这个负担就跟他们所谓的 disposable income,就是可用的花费,来做对比。它这个比率其实还是跟过去的这三、四年差不多。就是完全没有危机要到来的。所以我过去这一年一直说美国经济不会出问题,连那个Silicon Valley Bank出了问题,我也说经济整体不会有问题,就是你看这些数据。然后另外美国是全世界房贷里面固定利率用的最多的。那你想想看,两年前买房子,他的30年贷款是2\%的利润。那你现在是不是完全都赚到了?对不对?那你知道像欧洲或者英国,他们就很少有这种固定利率,那你一旦利率上去以后,整个房地产市场就垮了,但美国的市场就不会垮。



王孟源 28:45 

美国现在的这个,所谓的household real estate没有垮掉,原因就是,你如果是两年前、三年前刚买的房子,你有 30 年期的2\%的房贷,你会愿意卖房子,然后要还掉这个,然后再重新申请一个房贷吗?现在的房贷是7\%,对不对?所以你这个利率一下上去以后,在欧洲所造成的就是整个房地产的灾难。现在在韩国也有一个房地产的灾难。



唐湘龙 29:28 

没错,对。



王孟源 29:29 

但是在美国就不会。那我刚刚已经提过,就是你追根究底到最后,是因为美元是国际储备货币,所以它可以很简单的,连它的消费者都可以用固定利率来借 30 年的房贷,其他的国家就很难让每一个中产阶级都能够借到 30 年的固定利率的贷款,而且这个利率还因为美元的地位而特别低。



王孟源 30:06 

2\% / 30 年,那所以是,你这个利率我刚刚讲过,过去 18 个月在短线跟长线都上升了 5\% 之后,美国的这个房地产市场就...,它没有垮掉,它的价钱事实上还是在高位。它的那个平均房产的交易是 2021 年的时候涨了 35\% ,2022 年的时候到现在掉了5\%,只掉了5\%,你想想看这个这利息上去5\%,结果它的价钱掉了5\%,而且如果是比两年前的话反而还涨了30\%,但是它的成交量直接腰斩。



王孟源 30:59 

为什么腰斩?因为就是人家那些 30 年固定利率 2\% 的那些人占了便宜,当然他说什么也要保住这个房子,不换,他就是不换,所以你如果没有人想卖的话,你就没办法交易,对不对?那这样反应下去另外一个结果,在房地产上面还有另外两个因素。一个因素是,这个也是我过去几年讲过好几次的,因为这个通胀来势汹汹, 4年前就可以看到来势汹汹。所以三、四年前那个时候真正的超级有钱的超级富豪们就开始投资农业的地产,因为农业的地产是在通胀危机中保值最好的。那像 Bill Gates 就买了几十万亩的农地,但是问题是这些农地的供应很有限,真正房地产最大的市场还是这种



王孟源 32:05 

家庭住宅的市场,所以等到其他的人也学乖了,尤其是其他的一些基金,也注意到长期会有通胀的问题的时候,他们就开始投资住宅房地产。所以美国现在过去 30 年他们的医院都变成私营的,然后都变财团在驻扎xxx。现在这个财团开始进入房地产,开始买私宅,就是你这种独栋的住宅也变成是他们一个基金所有的,然后再租出去给你,就是从私有住宅变成出租住宅,那出租的话当然是所谓的 Multi-unit,就是公寓。出租起来更方便。所以你如果看美国的这个住宅的房地产市场的话,过去这一年也是这个独栋住宅,我刚刚讲过它的那个销量直接腰斩,但是他的那个公寓的建筑反而是热火朝天。那所以讲来讲去就是这些私宅的房地产市场根本就没有垮,就是维持在高位。



王孟源 33:18 

那你刚刚问我说为什么股票也在高位,很简单,也是同样的道理,对不对?因为人民银行跟欧盟高抬贵手,所以美国已经...,哈哈哈,已经过了这个危机了,那然后那个美联储所抽的资金又比欧盟银行、欧元银行少了 3 倍,所以当然就没有资金短缺的问题。



王孟源 33:49 

你我刚刚讲了半天,你看看,都是资金还是很充足的问题,对不对?那真正垮掉的就是我两个月前,两、三个月前提到一点的就是那个 office building,就是办公大楼的市场,但是那个市场很小,而且它真正垮掉的原因是因为大家都在家办公了,对不对?因为新冠大家都在家办。



王孟源 34:19 

另外一个垮掉的是所谓的 UNICORN market,那个就是所谓UNICORN,就是还没有上市新起的 的公司,但是它的价值已经定价在 10 亿美元以上的就叫做UNICORN。那这个很奇怪的是 UNICORN 垮掉了,但是Bitcoin没有垮掉,那同样的也是因为先牺牲其他的那些杂猫杂狗的加密货币,所以Bitcoin到现在大家还是在迷恋嘛,对不对?因为人性就是这样子,你自己曾经有过的东西就不愿意放弃,你曾经即使是纸面上的利润,自己曾经有了就不愿意认赔杀出,所以这些人就认为自己投资在加密货币曾经获利多少,他们就非要在超过那个最高峰以后才获利杀出。



唐湘龙 35:21 

那还远



王孟源 35:22 

这是人性的基本心理,你可以在心理学里面找到很多论文来验证的,很多实验验证,所以你刚刚问我说为什么股票市场很好?第一,他的经济里面的那个现金还很多,对不对?第二个是他跟欧洲或者是日本比起来要好多了。他们还有另外一个房地产,这个就很有趣,这是房地产市场里面最小的一块,就是工厂,所谓的manufacturing industrial real estate,这个比我讲过的那个农地涨得还要厉害。



王孟源 36:12 

 20 年前,就是在 2002 年的时候,这个是小布希上台没多久,那个时候美国的建工厂的投资达到谷底,这个时候是每个月 20 亿,两个billion。这是他们历史上的新低,就是他们去工业化嘛,他们从 80 年代开始去工业化,到 2002 年的时候是在谷底,然后从那开始就开始翻身了,它的这个新工厂的建设慢慢的升高,升到 2021 年已经升了三倍,到了 60 亿美元,一个月 60 亿美元的新工厂投资,你猜现在是多少?今年是 16 个billion,160亿美元。



唐湘龙 37:12 

160亿美金。



王孟源 37:14 

每一个月,就是两年之内成长的三倍。



唐湘龙 37:19 

所以它真的从这个数字来看的话,它的制造业的状况是好的。



王孟源 37:25 

它的这个制造业之所以会这样子,就是我刚刚讲的von der Leyen的功劳。因为现在德国已经在去工业化。所以他这个能够迁到中国的就迁到中国,需要留在这个欧美集团的内部的,因为现在都有很多贸易壁垒,而且要中美分割了,对不对?那必须要留在这个分割线这边的,他们就迁到美国去,所以这个,他两年之内这个工厂就涨了三倍,所以你想看在这种,像这种数字你只要看一看就知道美国目前的经济没有什么问题了。



唐湘龙 38:09 

好,在我们换题目之前,这两个问题就这件事情我请他们。那如果是第一个美国的金融面的危机可能已经过了,那耶伦还在紧张什么?第二个就是说大家知道经济,如果美国的总体经济,现在大家对于美国的经济会不会出现衰退的情况,现在看起来越来越乐观、认为那个几率是很低的,那对于美国明年的总统大选会有什么影响?



王孟源 38:39 

美国的总统大选其实拜登的丑闻太多了,然后这个他的经济即使现在是比半年前要稍微乐观一点,就是半年前的时候大家认为它的今年的成长率是低于 1\% 的,现在大概大家都认为会超过 1\% 的。但是事实上不会有太大的影响。美国现在明年的选举最大的因素除了它内部的社会性议题之外,我想这个俄乌战争也是一个很大的問題。Bliken还没有先知先觉,但是Sullivan,Sullivan他本身其实就是一个竞选助理出身,所以他是在这方面最早开始紧张的,大概三四个月前就开始到处放风说,这个我们是不是要准备,不要求俄国先归还领土。



唐湘龙 39:48 

归还领土。



王孟源 39:49 

yeah,然后我们就先要求他们有一个和谈。他一开始还不敢讲和谈,他一开始讲停火,但是你知道他们这个Zelensky在过去 18 个月不晓得重复多少次,就是说你必须要俄国先把所有的土地都还给他们,才能够坐下来谈,对不对?这个是从去年3月之后就是一直这样讲,那结果Sullivan在三、四个月前开始放风,那个他为什么会放风?很明显的就是因为他知道这个会变成明年竞选的一个很大的话题。



王孟源 40:28 

因为你已经有上次从阿富汗那样子很仓皇的撤离的一个前例,你如果再有一次这样子,那他明年就会,这真的是躲不掉这个话题了。不过我并不是说美国经济就从此太阳高照、万里无云了,哈哈哈,而是他的这个问题从一个短期的急性的问题变成重新回到一个慢性的长期的问题。那这个慢性的长期问题其实在过去的三、四年还又恶化了。因为你三、四年前,三、四年前美国的...,我说的这个问题,它的核心是美国的联邦赤字,也就是它的整体的负债。



王孟源 41:30 

三、四年前的时候,美国的结构性赤字还只有一年...不到一年1万亿,他今年的赤字是多少?是 1.5 万亿,这并没有什么特别的花费,就是说去年、前年,在大前年都有很特别的所谓的一次性的大开销,因为像新冠这种东西,他们要刺激经济或者是弥补民众的损失这些东西,但是今年基本上没有这些一次性的花费。但是它反映的其实就是所谓的结构性的赤字, 1.5 万亿,其中有1万亿, 1.0 万亿是利息,国债的利息,那你想想看它的国债总额是 32 个trillion。



唐湘龙 42:25 

32 兆美金。



王孟源 42:27 

 32 万亿美金。那它的有效利率就是大约是3\%,因为它的利息支出如果是一,那为什么会这样?因为它里面有很多是两、三年前利率很低的时候借的 30 年期或者是 15 年期的国债,那时候国债的利率1\%,所以你现在还没有撤换,但是你现在再这样换新,你这些到期的所谓的 rollover 轮换,然后还要再加上每年的赤字 1.5 万亿,至少 1.5 万亿。那这样一来你这个不到五年,这个有效利率就会提升到接近4\%。尤其是你如果要这个国债,那些短期的, 18 个月前还是零利率,现在已经是 5.5 了,这个你如果要这个 rollover 的话,那是更是非常的痛。所以现在美国的财政部在发新的国债的时候,主要都是要发长期的国债了,因为它的长期国债,因为就是我刚刚讲的yield curve倒挂了,就是短期预比长期率还要高。



唐湘龙 43:47 

好,那我刚问的那个问题,如果说美国现在在金融面的危机看起来最少短时间是过,没有那种的急性的危机,那经济的面相上面来看也比预计中要好。就步入衰退的风险大幅降低,那耶伦这么急着到中国访问干什么?



王孟源 44:10 

这里最大的变数是下个月的金砖货币,因为也是在过去这几个礼拜有确定的消息出来。这件事我也是从 7、8个月前讲到现在。



唐湘龙 44:24 

没错,你是第一个讲的。



王孟源 44:27 

因为俄乌战争一开始我第一个反应就是赶快搞金砖货币。不过我原本的建议是信用性的,就是这个金砖货币是建筑在参与国、会员国的本国货币之上。这原因是因为我想要排除印度,还有巴西,还有南非。但是你如果真正要包括其他这些国家,这些都是赤字国家,双赤字国家。这样,那你如果要让这些双赤字国家加入一个合成货币的话,就不可能接受它的本国信用货币。



王孟源 45:10 

来做你的基础,所以最新的消息是会是金本位的,就是根据金来做本位,那这个金本位的消息如果是准确的话,他告诉你的讯息就是,这会包括印度,会包括南非,会包括巴西,还有其他的,很可能包括任何一个想参加的国家都可以进来。因为我不管你的赤字,反正你是黄金拿来嘛,对不对。



唐湘龙 45:41 

就看你的黄金储备,看黄金讲话。



王孟源 45:44 

对,用金本位的好处。所以他说要用金本位,你就可以知道这是广撒网的一种货币,不是我原本所建议的那种比较狭义的,就是你必须要是没有赤字的,就是盈余的,那个财政跟外贸双盈余的。那我原本是建议基本上就是中国、俄国跟沙乌地,沙特,那我想因为外交上的考虑,因为目前外交大战略上是中美脱钩,其实是这个黄金的 10 亿人跟剩下的 70 亿人脱钩,那在这个大环境之下,也许这种广撒网的金本位货币是比较有利的。



王孟源 46:39 

那不论如何,这个下个月如果它正式的出台之后,美国就会增加一道很强的逆风。那这个就必须要讲到金融,因为我刚刚讲了半个多小时,讲的其实主要是经济。我们现在必须要讲金融了,那请大家有耐心,如果有问题可以到博客来问。那这里的问题是,美国在过去 18 个月刚好是他开始从宽松转化成紧缩的一个时间点,就是 2022 年1月开始的。那过去 18 个月他一共紧缩了,我刚刚讲过是 0.6 个万亿,就是 6000 多亿美元,但是你可以去看美国金融系统的现金的数目,这是我个人发明的一个技巧,所以你在别的地方不要说教科书了,你在别的别人的那个文章里面是不可能看到的。但是所以我直接跟你讲是怎么样?但是如果有小细节的话到博客来问。



王孟源 48:04 

就是在当过去的这几年,事实上是从 2008 年到现在,你这个美国的金融系统如果现金过多,也就是尤其是因为量化宽松,所以那个美联储把钱往这个金融系统塞,那塞进去以后他如果是承受不了,就会出现两种,就只有两个地方可以储存。就是好像你长江如果水位上涨的话,就只好到洞庭湖跟鄱阳湖去,那这两个储水的湖泊是什么呢?第一个是他的所谓的 bank reserve,就是你这个所有的银行都必须要在美联储开一个账户,然后有现金储备。这个现金储备也是 18 个月前到达顶峰,就是 4.3万亿,但是现在这个月已经降到 3.0 万亿,所以它降了 1.3。



王孟源 49:13 

对,另外一个是给那个其他的一些 shadow bank 影子银行,像是所谓的所谓的 money market fund 或者Fidelity,这种名义上是所谓的 wealth manager 的,那这个是他们没有钱的时候去借就叫做repo,因为跟大银行不一样,他们是必须要拿资产,实际的资产像是国债、债券来借钱的,这时候就叫repo。但是如果你现金过多的话,就可以反过来把现金存到美联储,然后换国债债券,然后你收他的利息,收国债债券的利息,这个就叫做 reverse Repo。



王孟源 50:01 

所以这个repo, reverse Repo 其实就是这些 shadow bank影子银行,它反映的就是他们的手里的现金的多少。那 18 个月前这个 reverse repo 的数量是 1.6 万亿,那这个月刚刚到了 1.8 万亿,就它反而还上涨了,就是他们手上的现金还是有这么多,反而涨了 0.2。所以你这个这两者一个是下降了 1.3,一个上涨了 0.2,它的净值就是下降了 1.1。



王孟源 50:41 

那我们再回溯。美联储从金融系统里面收掉的是多少?收掉的是 0. 6。但是这个美国金融系统损失了、就是减少了 1.1,所以事实上并没有太多,不到两倍。那这个为什么会这个样子?就是因为美元要比欧元要强势的太多,有太多欧元资产转到美元,这也是我刚刚讲了好几次的,就是谢谢von der Leyen的贡献。



王孟源 51:21 

那这里看起来好像没有什么问题,但是这里就是那个金砖货币会去刨这个根。如果金砖货币变成金本位的话,这些其他的货币像欧元、日元、英镑,他们就可以去买黄金,或者直接换成金砖货币。那这样一来他们就不会去弥补这个美联储抽水的影响。那这时候这个抽水的这个杠杆就会比 2 大,我认为是在一两年内就会增加到超过3的一个杠杆值,那再加上美联储现在这个 QT 的抽水是刚好跟 QE 相反,那我刚刚已经讲到 QE 跟 QT 的特点就在于它影响的是长期利率,yield curve里面那个长期利率,其他的一些那个措施全都是集中在短期利率,只有 QE 跟 QT 是长期利率。那一方面没有这些从国外来的弥补的力量,因为金砖货币的关系,那另一方面你这个 QT 又在不断的卖出长期国债。那偏偏这个长期国债我刚刚也讲了,它目前长期国债利率已经是 4. 0,已经比美国财政部整体国债的有效利率 3. 0 要高了。



王孟源 53:02 

你这个如果再提升的话,比如说如果提升到 5.0 的话,它的这个光是利率的支出在五年之内就会翻一倍,翻到两万亿,然后你这个时候美国的结构性的赤字就会到达3万亿,就是你可以看大约五年之内美国的每年的利息支出的会加倍,然后它的结构性赤字也会加倍。所以原本你现在这个通胀从零点几加到五点几的时候,其实你是把那个国债贬值的。因为你是欠了债,我刚刚讲的是是数面上的债,没有算计你这个通胀率,所以你一旦通胀以后就债多不愁。以前欠人家100 块,十年之后的 100 块根本不值钱的。但是问题是因为金砖货币的关系,还有这个 QT 的关系,美国的这个财政赤字会从现在开始进入一个高速增长期。那这里还有另外一个 10 年期的背景,就是其实也没有 10 年了,就是六、七年之后,美国财政最大的一个支出就是所谓的 Medicare 老年人的健康保险会从正面的变成负值,变成负债值,那所以这也是在不断的增加他们的这个所谓的的结构性赤字。



王孟源 54:57 

所以我个人认为在 3- 5 年内,美国的结构性..., 3 年内美国的结构性赤字就会超过2万亿, 5 年之后它的结构性赤字会超过3万亿,那这是一个很大的问题。所以我认为在...,如果美联储够机灵的话,在明年下半就必须要停止 QT 了,然后在 2025 年可能就必须要反过来在 QE了,因为你唯一能够把长期 利率压下去的手段就是QE,那当然这是一种饮鸩止渴的手段,你们要看这种饮鸩止渴的效果是什么样?就看日本,日本现在就是这样。



唐湘龙 55:43 

好,日本是一个大话题,我就再让孟源,我很希望有一天孟源可以准备一下日本,让大家来了解一下日本到底现在是一个怎么样的一些情况。好,不过我们因为今天孟源讲的这一段其实是非常专业的问题,那我希望大家可以反复的听一下,大概比较能够理解。否则平常我们在谈一些的国际的经济跟金融面的问题,大部分都是在一些极短期的新闻跟数字上面在打转了,没有一些结构面的分析,所以你很难看到全貌以及比较深远的影响。



唐湘龙 56:21 

好,简单讲了,孟源在告诉大家说短期间美国在前两年所呈现出来的那个摇摇欲坠的经济跟金融面的危机,暂时他是离开加护病房了,他回到了一个一个慢性病的一个长期的情况,那不见得多好,但是短期的危机是过的,它的操作的空间是变大的。好,我们再来看,还有点时间,我还是要请教一下孟源的俄乌的情况,因为现在看起来大家对乌克兰的战场上面的情绪,乌克兰的反攻打了半天,看起来大家最近这一个多月的时间不只是瓦格纳的这些事件,而是总体来讲对俄乌战争,对乌克兰战场上面的情势。大家表现出来的第一个在关注的热度上面有点意兴阑珊,第二个两个重要的事件,我觉得后续值得观察的一个就是美国看起来不管是 F 16 也好,或者是集束炸弹也好,美国仍然持续在乌克兰战场上面采取加码,即使他已经公开的讲,认为这个反攻没有什么实质的进展。第二个就是现在当就是黑海的所谓谷物协议被撕毁了之后,包括了俄罗斯最近对于奥德萨的这些导弹的攻击,以及扬言只要进黑海的船只,我都把它当做是军事用途。这两件事情会有什么影响?俄罗斯为什么这样做?



王孟源 57:48 

首先乌克兰之所以从八、九个月前就开始宣传所谓的春季攻势,它原因是因为北约在一个礼拜前有一个会议,所以他们的预算盘是如果这个春季攻势打的顺利的话,就可以要求北约收他们进去,当然美国人不管你打的怎么样都不会收你的,都是骗人的。就好像有一头驴子,你要让他一直往前走的话,你就吊一根红萝卜在前面,他就会一直往前冲,但是打的这么悲惨还是超乎所有人的意料,对不对?除了我们这种比较脚踏实地的少数之外。我上个月的时候提到俄罗斯在从去年 10 月之后, Surovikin 建立了好几道防线,然后由 1000 多个要塞化的排阵地构成。上个月我上节目的时候,那时候才打了两个礼拜,我说连一个都没有被打下来。那在经过再多一个月,你猜有多少被打下来了?还是零个。



唐湘龙 59:01 

对,就基本上连那个第一条的战线都还没摸到。



王孟源 59:06 

对,事实上你如果看四个礼拜前,就是我上次上节目的时候,从那到现在它的战线变动没有超过 3 公里的,战线的这个上下浮动没有超过 3 公里,唯一的例外是在北边,Kharkov东北那里。有一个地方它的变动大约有 10 公里的变动。那这里最大的问题是这个变动是俄军在进攻。



唐湘龙 59:34 

没错,就是俄军在反攻,把之前在哈尔科夫丢掉了土地,现在一寸一寸把它收回来了。



王孟源 59:40 

收回来了。对,所以你这个春季攻势真的是打得很窝囊,而且我上个月也谈过他们被炸得一塌糊涂,就是有很惨的那些豹二的残骸在战场上到处都是。那后来第二波乌克兰就学乖了,他就变成派步兵前进。但是问题是这个步兵前进不就是去年一整年你们这些



唐湘龙 01:00:05 

没错



王孟源 01:00:07 

一战的那种壕沟战吗?那你躲在自己的壕沟跟要塞里面,还扛不住二军的那个炮火优势,你自己跳出来。除非是,你这基本上就是要堵俄军没有夜视设备,但问题是夜视设备这种东西算是民用的,中国并没有对俄国禁运,对不对?他的那个夜视设备一个几百块,一次买几十万架,这都是有记录的,而且是开战之前就已经都已经送货送到门。那根本这些小部队进去就是找死。而且老实说我不晓得你对英文的一个单字叫做pornography熟不熟。它其实是色情图片或者是视频的意思,但是后来变成广义的,就是任何刺激你的情感的东西都可以叫做porn。那有一个东西就叫做 military porn,这个军事的什么,我不知道怎么翻译,但是他的意思就是会爆炸,很大的爆炸,一大团烟火,让那个军迷看得很爽的那种图片。很不幸的,我也是一个军迷,所以过去这个月看了大概有 400 多个这种视频。哈哈,那真的。



唐湘龙 01:01:39 

没错。



王孟源 01:01:42 

好惨好惨。而且我从这 400 多个视频里面得到一个结论,就是俄罗斯他的这个军工的供应真的是上去了。



唐湘龙 01:01:53 

比预期中强很多。



王孟源 01:01:56 

强太多了,他原本就是靠炮火覆盖对不对?但是那个是没有制导的,现在全部都是有制导的。而且恐怖到什么地步。比如说我有看到一个视频,就是乌克兰的一个小小的班阵地,有五六个人在一个壕沟里面,然后有一个小小的工事。那在过去这个月俄军有至少有三种轻型的自杀式 UAV 比较大一点叫做Lancet柳叶刀,它的那个装药量是 3 公斤,大概跟一个炮弹差不多。然后再小一点的就是像大疆那样子的四个螺旋桨的那个叫做 Judgement Day,它的装药量是一公斤左右,然后更小的一种叫做Vampire吸血鬼,还是大疆那种但是基本上就跟民用的差不多大小了,没有说特别大,那它的装药量就很少了,不到一公斤。但是你想想看,这至少也是几百块一两千块的东西。我看到一个视频就是这样一个小小的班阵地,五六个人在那边,结果俄军就一个人Vampire下去boom,觉得的没有从那个工事,就是那个。



唐湘龙 01:03:16 

没有直接炸这个壕沟里面。



王孟源 01:03:18 

只炸在那个土洞外面,不能确定有什么效果。所以就再送第二个Vampire,然后再送第三个、第四个、第五个、第六个、第七个,一直到第七个才钻到那个土洞里面去,然后他们才放心。



王孟源 01:03:33

你想想看他们这把它当做不要钱的来打,但事实上机械化战争就应该这样子。你就是制导弹药当做不要钱的来打,那俄军真正做到了这样的。不只是我刚刚讲到最小的那种Vampire,他那个Lancet还有他的Judgement Day都是不要钱一样。然后呢?过去这 18 个月基本从来没有看过俄军用中大型的查打一体无人机发射导弹,结果过去这个月也出现了。那更不要提所谓的那个制导炮弹,现在真的是到处都有。所以你想想看乌克兰这样怎么打嘛?他根本就是打不来,就完全就是因为俄军等着再继续杀伤。所以不急的出手,即使是在我刚刚讲到Kharkov那地方也是推进的其实也是很慢。俄罗斯最重要的就是不想要赶快升级的太快。那从美方的话呢,它这个升级的问题是怎样子?原本是说要给 F-16,其实你如果去看两三个月前美方的那个官方的反应就知道他给 F-16 给的心不甘情不愿。是被欧洲国家拱上去的,西欧的那些国家,尤其是Ben Wallace,就是目前英国的



唐湘龙 01:04:58 

英国的防长,要卸任的防长。



王孟源 01:05:00 

对,最高调的说,我们要赶快给 F-16。那为什么美国人心甘情不愿呢?就是因为他们的那些先进武器,尤其是Patriot爱国者飞弹。这其实我上个月也讲过了,去了以后他的那个股价就掉了,对不对?因为。



唐湘龙 01:05:18 

效果有限,实战效果不好。



王孟源 01:05:21 

对,实战效果不好,所以美国的军火商不愿意冒这个危险吗?那而且事实上 F-16 已经停产了。停产的东西,你这个都只有二手市场了,结果你去把他招牌砸了,那不是很难看啊,得不偿失。所以事实上原本北约的理事长Stoltenberg,是原定今年 10 月要退休。那退休的时候呼声最高的继任者有两个,一个就是 Ben Wallace,另外一个是von der Leyen。那后来两个人都被否决了,所以 Stoltenberg只好再延长。



王孟源 01:06:02 

那这里有很多内幕消息,其中最确定的一点就是美国人坚持要von der Leyen。那我刚刚已经解释了很多遍,为什么美国人很喜欢von der Leyen。但是美国人之所以不喜欢Ben Wallace,据传就是因为跟Ben Wallace没有事先获得批准,就把他们的 F-16 拱出来。所以Biden跟他们的手下的军火商就心有不甘。那无论如何,von der Leyen好像是被Macron否决的。



唐湘龙 01:06:35 

没错,他们俩看起来就是已经翻脸了。



王孟源 01:06:41 

对,那这其实是有点ironic,就是因为当年其实von der Leyen是Macron捧上来的。



唐湘龙 01:06:50 

没错



王孟源 01:06:51 

Macron那个时候还不晓得von der Leyen是这副德性,哈哈哈哈。



唐湘龙 01:06:57 

不过你说马克龙之前去中国的时候让冯德莱恩自己搭民航机去了,你就可以看得出来他基本上跟冯德莱恩已经拆伙。



王孟源 01:07:07 

你讲到这个很有趣,Yellen这次到中国后来好像有跟九仙女。



唐湘龙 01:07:13 

没错,就去吃饭。



王孟源 01:07:17 

但是真正有趣的是中国外交部没有请她吃饭。还必须要...,哈哈哈,这里真正有趣的是,照理说一个美国财政部长级的人物来的话,当天晚上应该是由至少外交部长请他吃饭,结果外交部根本就不理她,说你自己去吃吧。



唐湘龙 01:07:39 

她也安排了安排饭局。看起来她吃的是还蛮开心了。



王孟源 01:07:43 

吃的是蛮开心的,但是中国官方的态度是要比四、五年前要强硬太多了。这是一件好事。因为你事实上就是这样,人家已经准备要吃你了,你越软他们越是得寸进尺。



唐湘龙 01:07:59 

好,刚刚还剩下一个问题就是谷物协议,就是俄罗斯现在当他在轰炸黑海,乌克兰在黑海边上的两个主要的港口。这个轰炸第一个它的用意是什么?因为俄罗斯现在看起来是弹药打不稳完,我们都看到了,就北约不断的在叫说他们的弹药已经,弹药存量已经见底了。那过去认为其实战争发动之前的时候,西方国家对俄罗斯所做的两个预判都是错的,一个认为俄罗斯的金融很快会崩溃,俄罗斯经济会崩溃,没有,他们现在承认跟预判的完全不一样。第二个他们认为俄罗斯本身的军工产业撑不了非常非常久,经不起北约在这场战争当中对俄罗斯的消耗,结果事实也是相反的。好,那现在牵涉到谷物的问题,这个事情就很敏感了。那俄罗斯现在对于谷物的出口采取一种看起来是非常强硬的态度,为什么?会有什么影响?



王孟源 01:09:02 

它其实是俄罗斯的最优先考虑,是对第三世界的那种外交宣传价值。所以既然乌克兰硬要说这个是要廉价卖给第三世界,你就看看是不是真的这样?结果不是嘛。因为事实上这些谷物都是美国跟欧洲的那些谷物商收购的。



唐湘龙 01:09:26 

没错,他们大发利市还是。



王孟源 01:09:29 

最后他们还是运到了欧洲去嘛,就近运到欧洲去了,你这谷物商不可能赔钱卖给那些穷困的国家,对不对?那所以你过去的这几个月的这个谷物协议已经证明了这一点,已经不需要再对第三世界国家证明这件事了,那你就可以把它收起来,不过不是回到一年前的那个状况。一年前的时候Odessa封锁是乌克兰自己封锁,自己放水雷封锁,因为他怕俄国来做两栖登陆。现在的封锁是俄国主动去封锁,主动的去把港口的设备打掉。那你这个就是,俄国也其实也不太在乎那个所谓的两栖登陆的威胁了。就是,反正现在已经摆明了我就是要跟你长期抗战,慢慢的把你杀到你最后一个乌克兰的男人也死亡为止,所以就不在乎这一点。所以他现在比较担心的反而是那种破坏,就是无人机,无人艇的那种破坏。那你就干脆把他封锁起来。所以战略跟战术的考虑就是这样子。那你其实 15 分钟前你还问了有关cluster bomb的问题,这个。



唐湘龙 01:10:47 

集束炸弹。



王孟源 01:10:49 

对,集束炮弹这种问题,其实就是因为俄国的炮弹跟制导弹药的那个供给量比北约加起来还高了一个数量级。那美国的最短缺的,尤其是 155 的榴弹炮这个是真正已经见底了,从几个月前从南韩那边拉过来那一批早就打完了,对不对?你现在到哪里去拿?所以真正剩下来的就是只有集束弹药,那只好把这个集束弹药顶上去。那这里又有一件很有趣的事情我要提醒大家,就是这些主流媒体不会跟你回顾的,就是一年多前 ,18、十几个月前战争刚打起来的时候,欧美那个时候他们的主流媒体宣传就说。



唐湘龙 01:11:44 

俄罗斯使用集束弹药。



王孟源 01:11:46 

说俄国用了集束弹药。



唐湘龙 01:11:48 

而且谴责他。



王孟源 01:11:50 

那个时候事后根本就没有再提起。那这显然要么就是无中生有,要么就是很少的一个特例,用了一两次没有再继续用。不管怎么样,这因为我没有绝对的证据,我不敢讲说他真的没有发生,但是当时美国拜登总统的发言人那时候还是一个白种的女人,她出来很明确的讲说这是斩钉截铁的战争罪行。



唐湘龙 01:12:23 

没错违反国际法的,违反国际公约。



王孟源 01:12:25 

俄国打一两发集束炸弹是斩钉截铁的战争罪行,美国送 10 万发集束弹药去打就是正义之举,所以我想要顺便跟大家讲一下这次事好笑在这里。当然这集束炸弹本身对俄乌战争是非常,从军事层面是非常不合适的,因为你普通的 155 炸弹还对装甲车还有坦克都还有威胁,那打到工事上面如果是直接命中的话,也基本上就会一发销毁,但是你那集束弹药基本上是里面有几百发小的炸弹,那这个炸弹掉下来以后它的爆炸力跟手榴弹差不多,那他这个是杀伤没有防护的步兵。



唐湘龙 01:13:18 

杀伤人员,但是不杀伤装甲。



王孟源 01:13:22 

对,所以对机械化战争,对壕沟战,也就是我们目前看到的这个俄乌战场都没有用。所以它这个完全就不是为了军事考虑去用的,而是为了他实在拿不出普通的 155 弹药,所以只好拿最后剩下的这种来凑数。而偏偏美国官方的数据是说这些集束弹药里面的小炸弹有 6\% 的哑弹率。



唐湘龙 01:13:52 

没错,就变成未爆弹。



王孟源 01:13:55 

对,未爆弹。你平均掉下去以后就会有四五颗,因为你同样一个普通的他们的155口径的集束弹药里,每次的集束的炮弹里面有大概六七百个小炸弹,那你这样子如果是 6\% 的话就有大概四五颗,会未爆弹在里面,不对不对,大概就会有 40 颗未爆弹在那里。那这个就是未来杀伤平民最大威胁。



唐湘龙 01:14:35 

没错,基本上就像是一个不知道在哪里,什么时候会引爆的地雷一样,就是其实之后对平民的伤害,或者说对于未来的战场后续的清理比较有影响的是在这一部分,但是实际上面它对于现在正在进行的战争的影响是非常小的,反而暴露了就是说美国跟北约现在在弹药补给上面已经开始有点弹尽援绝的味道。好,因为时间的关系,今天孟源,特别是孟源今天的前面讲的那一段了,我在很专心听的时候,我觉得让我听到了非常多不一样的东西。在我们平常在理解现在的就是说国际经济跟金融问题,尤其在俄乌战争发生之后,那全球的金融货币市场是非常混乱的,各个国家都在不同的 level 上面,有的在面对通膨问题,有的在面对通缩问题,有的股市在高点,有的股市在高点,有的甚至是有破产危机。大家现在真的就是台湾的闽南话讲的,日头赤焰焰 随人顾性命,大家各自解决各自的问题,但是美国看起来从它最糟的情况之下,他从密室逃脱了。好,来,我们感谢孟源,同时也感谢一些一些观众朋友的留话。来,我们的几位的观众朋友来第一个。



王孟源 01:16:00 

先等一下,对不起,我做一个更正,就是那个几百颗集束小炸弹,那个是空投的炸弹,那个炮弹的集束弹药里面可能不到 100 个,所以应该是只有 70 个,看他的弹药的类型,所以那个未爆弹,因为我记得很清楚6\%,然后那个未爆弹每颗是4个,所以应该倒算的话应该是大约 70 个,但另外还有总结的就是,我并不是说美国就从此海阔天空。



王孟源 01:16:36 

没有问题。这里的最大问题就是它的这个长期利率跟国债交相作用,就是它的总负债,还有赤字,财政赤字交相作用之下会在 2- 3 年内开始发酵,有一个很不好的加速器。那要看美联储怎么反应,要看国际上的其他中央银行怎么落井下石。然后呢,一个最基本的建议,我是说暂时不要去买长期国债,就是现在美国那个长期国债的利率可能再降低一点,它的利率降低的时候就是价格会上升,就是会涨。但是你如果是看到三五年之后这种,因为你买长期国债就是基本上就是长期投资,所以你一年之后涨了,这个没有什么太大的意义。但是你如果买了这个国债,结果 10 年之后他这个贬的一塌糊涂,那这个就没有什么意义了。所以我是不建议大家,尤其是中国的或者是台湾的中央银讨去买美国的长期国债。



唐湘龙 01:17:58 

好,希望大陆的这些金融体系里面有听到。来感谢来冯一伟,感谢非常喜欢的龙哥凤姐的节目。但是我觉得两位在讨论中国经济问题仍然没有走出西方经济学的窠臼。我的想法是大量引用就是说王孟源博士的观点,所以在此留言。好,恳请批评指教。他说当西方媒体想要贬低中国的任何成就的时候,都会在标题上面加上一句, but at what cost 就是成本代价是什么?我认为我们在审视当下的经济问题的时候,也应该问这一句话。同样是冯一伟,他说疫情期间,美国医疗支出占 GDP 从 18\% 升到20\%,这些 GDP 的增长换来的不是人民的幸福,而是平均寿命的下降。因此,数据不该是衡量的唯一标准,在环境永续的前提之下,尽量提高人民的生活水平才是经济发展的目的。



不论股市或选举制度,都有两个基本的假设,一个就是人都是理性的,二是不存在资讯的不对称。可是现实社会当中,这两项假设都是不存在的好。同样的,冯一伟说,美国股市之所以能够运作,基于市场上的主要玩家都是 old money 跟这个养老基金,他们都是长期投资,而选择能够运作是因为台面上的候选人都是 deep state的筛选过的,所以美国的建国精神是共和,但制度设计的初衷是为了压制民粹,在经营政治之下才能够实现一定程度的理性。那所谓的自由只有在公权力不限制资本的情况下面的自由。



唐湘龙 01:19:36 

好,那当下中国股市的主要的参与者都是散户,只要有一个呢?这个贾耀婷的在这样子一个靠 PPT 骗散户前的案例,长期的结果就是劣币足良币,大家都是搞这个假大空的,这样的一个在骗钱的活动,好,那的美国可以承受金融市场的很高的这个浪费,还催生出许多的创新,因此他们的一项优势,他们每浪费 6 美元 5 美元的代价是由国外承担的。



唐湘龙 01:20:06 

日本通过 YCC 把股市推向历史新高,最终受益的其实是巴菲特这种能够在日本国内介入,塑造日元套利的国际金融大鳄。好,那的同行冯一伟说,现在中国目前面临的经济危机本值都是生产过剩,那些在城市当中找不到工作的青年,只要放弃跑车那名表大豪宅的幻想,选择回乡,马上就给了。快乐躺平,身体也可以更健康。好,那个冯一伟的流化很多,后面两个我先跳过去来在了布拉德夫看001,感谢林俊点,感谢他说来回来来回抖来还愿了。他说终于通过辅人,终于通过了辅人哲学博士的答辩,恭喜。他说虽然是王博士特别点名意义不大的文科博士,但他还是毕业。哈哈哈哈,好。不不不,恭喜恭喜恭喜来在光觉海感谢。



他说孟源老师好,他说感恩解惑。他说他请问两个问,那第一个就是美国军人援乌克兰的集束炸弹,乌克兰已经开始投入使用,美国不承认乌克兰反攻失败而硬要再给集束炸弹,请问对战局有什么影响?



(没有影响)



对,没有影响,刚刚孟源特别解释了这一块。第二个季辛吉访问中国,中国高规格礼遇,请问这一趟季辛吉是不是来挑拨中俄关系?你认为是吗?他会得逞吗?



王孟源 01:21:34 

我觉得只是做做样子,因为毕竟是所谓中国的老朋友,这是一个官方认证的标签,对不对?所以千金买马骨嘛,虽然它一点用都没有,但是你既然已经有那个标签了,而且它是这个标签里面最有名的人物,所以你还是客气一下。



唐湘龙 01:21:53 

没有错,凸显中国在对待老朋友的方式,当个示范了。它的示范跟带风向的效果比较大,实质上面意涵我认为不会很大。



好,再来破仑感谢。还有 some summer 是young,感谢,他说他问一下孟源,他说美国的高额学费助学贷款持续多年,但美国高等教育的科研仍然很有竞争力。为什么?



王孟源 01:22:22 

因为这种东西有两个原因。第一个,它是传承。你那个环境,还有这些教授,他们本身就是有一个 network effect 网络效应,对不对?你作为一个研究人员,你希望的是一个设备最好的实验室,还有最好的老师、最好的同事,这种东西都是要看历史传承的。当初美国之所以能够建立它,除了用三倍的薪水去欧洲挖人之外,主要还是因为第一次跟第二次大战把欧洲给打烂了,才能够挖到最好的,那这是第一个原因。第二个原因是因为中国整个社会的对诚信的观念没有建立,这个在一般的经济活动里面还不是太大的问题。因为广告本来就是夸张的,但是中国的学术界在市场化的过程中认为自己也应该完全市场化,他所做的公关也完全广告发化了,那结果,比如说昨天山东大学的校长才出面否认说他们有所谓的,这个叫做什么,配对制,就是所有的外籍学生都会有安排一个学生负责的一对一的照顾他。这种事情是多少年来大家都是公开的秘密,结果他就是出面的直接撒谎。一个学术的人这样子直接撒谎然后完全没有后果。



王孟源 01:24:04 

你说这种环境是真正的好的学术环境吗?但是中国的那个领导阶层就认为可以啊。你反正学术搞假大空是应该的。那这样子你假大空搞得太厉害的第一个结果就是真正想要做研究的人不愿意来,现在你只好出大价钱去买。



唐湘龙 01:24:28 

好,再来。这个 join 的 INA 的大花生,他说想问问王博士,台湾 30 年人口没有增长,人均工资扣掉通货膨胀也几乎没有增长,但是 GDP 总量翻了七八倍。他说这个,这个总量的增长是从哪里来的?



王孟源 01:24:46 

从台积电。



唐湘龙 01:24:49 

对,从技术创新来的。



王孟源 01:24:53 

从台积电。因为那个时候美国要,在 80 年代的时候美国担心日本会超越美国,所以要把日本整掉。整掉日本的时候除了在金融货币上出手之外,一个很重要的就是要把它的整个半导体系抢过来,抢过来以后,美国又不愿意自己出人,对不对?因为是你出人的话会搞出很多很昂贵的劳工,而且要投资到教育上面去,所以他就把它转移到南韩跟台湾去了。所以台湾基本上还是受益,但是这种事情你是只有在比较小的经济体才能够谈得上,你如果是一个洲级的经济体不可能说去靠人家、捡人家不要的东西。



唐湘龙 01:25:44 

没错没有错,这是一个非常非常好好的观点。好,再来 solo man 劝感谢。然后王伟瑞说大国博弈就是经济优先,美元不垮美元,美军不败,美羽不为闭嘴,那就是王孟元讲得很透支。好, X a 1394 感谢。然后 summer young 感谢,好,再来呢,这个 90T 感谢。然后林冠感谢,他说等了一个月终于等到王博士 Bruce 燕感谢,然后菲菲的书,他说关于黑海协议,他说我做了个梦,俄罗斯的副外长亚历山大潘金说俄国认为黑海协议是让乌克兰抢走俄罗斯原来的客户,而且赢得了宣传,俄罗斯却什么好处都没有得到,所以特别的不爽。基本上这逻辑大概是合理的。就联合国某个葡萄牙籍的高官说,如果西方不同意把俄罗斯接入Swift,或者同意俄罗斯把化肥、氨气从黑海港口出口,协议将很难延续。对,这个确实是一个谈判的问题。



王孟源 01:26:56 

这是西方主流媒体扭曲过的说辞,事实上他们俄国人抱怨的是你协议上面已经说好已经要给俄国的好处,都没有拿到,不是空泛的拿其他的好处来抱怨,而是事实上你协议是说双方都可以卖一些东西,结果变成只有乌克兰可以卖,OK?他没有履行协议的另外一半。然后等到你俄国人出来抱怨的时候,他的主流媒体再修改几个字,就变成他不是在抱怨你违反协议,而是抱怨你没有拿到其他的好处。那这样就是欧美现在吃人还不吐骨头,就是这个样子。



唐湘龙 01:27:47 

再来的丹丹若舒感谢在澳洲哈布鲁斯一样感谢。好,这个感谢我们的所有的听众朋友,因为因为孟源今天的准备,特别是美国。的经济跟金融的现况的部分是需要,可能大家要非常的专注,他有一点点的门槛,不过听完了之后你大概就会知道孟源对当下的当下,过去大家在很表面的觉得美国好像快垮的那个叛徒是很重要的。



唐湘龙 01:28:17 

好,后面俄乌的部分就孟源总是有他这自己很独特的视角跟他的资料的整理,这部分对大家理解,就是说俄乌的现在的现况的现况的发展。唐相龙讲的不是不算数,以王孟源讲的为准。好,今天星期五的时间,感谢孟源在一个地球的另外一端是时差 12 个小时的地方,那透过岳阳市血脑提供给大家的精彩丰富的内容,感谢孟源。



王孟源 01:28:45 

我知道你还有其他的事情,谢谢你破例给我一个半小时,这个讲不完。



唐湘龙 01:28:52 

根本讲不完。好,感谢。没关系,反正反正我们就保持了每个月,因为我也我说过我也不能够在熬孟源的时间,因为那准备是很费力的。感谢孟源,也感谢所有的观众朋友收看今天的龙行天下。周末快乐,下回见。拜拜。再见。\twocolumn[\begin{@twocolumnfalse}
\section{俄烏戰爭接近尾聲、國際戰略新階段}
\subsection{20230818}
\end{@twocolumnfalse}]xxxx标示听不清楚的地方

唐湘龙 00:05 

我们可以用俄乌战争接近尾声吗?



王孟源 00:14 

好。从大角度来看,从国际外交战略的角度来看俄乌战争的影响。



唐湘龙 00:19 

好,okay,俄乌战争接近尾声!好,然后国际战略新阶段!好,这是孟源的重点。我看到今天早上最新的、那个美国的那个自由媒体人、资深记者那个Hersh的那个消息有意思,他说美国 CIA 警告布林肯乌克兰的反攻不会成功,然后Hersh在自己的官网上面说,他说引用一位美国情报官员的说,布林肯从中情局那边知道乌克兰的进攻不会起作用。因此布林肯可能想要效法Kissinger,然后来促成像 1973 年的巴黎和平协定来结束越南战争。



王孟源 01:33 

这个议题我今天准备要讲,不过我的意见跟Hersh不一样。



唐湘龙 01:39 

好,没问题,这个你待会都可以说,我待会可以稍微提一下来。欢迎来到龙行天下,我是唐湘龙,我在台北来,但今天的龙行天下刚因为这个的我们的连线的软体在更新,所以会有点耽误。但是在本周一开始我就预告,因为我知道每次都会有人不停敲碗说王孟源什么时候,王孟源什么时候,好,王孟源就是现在。好,那原则上每个月的第三个星期五的早上的 9 点半钟。



唐湘龙 02:27 

我跟那王孟源王博士谈好了,那这个时间就是王孟源的时间,好了,今天我们仍然透过越洋的连线,那人在美东的王孟源来为大家导读几个最新的、最重要的就是说国际的议题,那大概三大块。第一个就是乌克兰的战争,看起来接近尾声,你要闻到那个周围的政治空气正在快速的转变,为什么那包括了吉达的这样一个由沙特好像带头在推动的就是俄乌的和谈,那也已经因为兜了非常多的国家。再来第三个,你会发现印度即将要召开的G20,乌克兰非常的不爽,因为你竟然没有邀乌克兰,因为这一年多的时间里面大部分重要的国际会议,这些西方的就是G7 挂的,跟美国同一挂的,只要是办了这些大型的国际会议几乎都会把泽伦斯基、把乌克兰拉进来,增加它的能见度,也表达对乌克兰的支持。可是今年的 G20 没有乌克兰,当然因为印度跟俄罗斯的关系是不错的,这可能也是背景之一。



唐湘龙 03:38 

好,那接下去我们再谈,最近我一直很想谈,但是找不到 qualified 的人可以谈。就是西非,非洲的这个大的动荡似乎开始了,美国似乎摩拳擦掌,准备借着这个机会伸手进入非洲。就是大陆翻译作尼日,台湾翻译作尼日尔,那尼日现在周围的就是说西共体,西共体现在已经组成了联合部队待机而动。好,那到底尼日尔背后的就是说这一次的政变。那政变有这么严重吗?在非洲国家来讲,政变的司空见惯了,为什么尼日感觉上面特别让法国、让美国,让这些这些西方国家觉得特别揪心的感觉?为什么?好,最后如果有时间,我们再来先来预览一下金砖,金砖今天已经是 8月 18 号了,下个礼拜就是金砖在南非召开的日子,这次的金砖孟源已经提示非常久了,这次金砖是非常值得关注的,包括就金砖币的形成有没有可能对这个世界会有什么影响?来,先介绍我们的来宾,人在线上的王孟源。



王孟源 04:54 

大家好,非常高兴跟大家再聊一聊。我先讲一下,上个月我不止讲到了一个小时,而且讲到了一个小时半,所以很抱歉,所以这一次我特别少准备了一点,所以讲的会比较随意。



唐湘龙 05:10 

不可以少准备。没关系,你要讲多久,只要我后面的工作都还来得及,我都尽量让孟源讲,都是有收获的事。来,我们先从俄乌谈起好了。第一个,我们先看战场上面的情况,我看到的一些的讯息就是这次的反攻仍然没有任何的进展。为什么?



王孟源 05:33 

事实上从去年 10 月乌克兰开始,他们预先广告说要打春季攻势、春季反攻开始呢,我就一直评论说这是完全自找死路的行为。然后两个月前开始之后,就是 10 个礼拜前开始之后,我已经评论过两次了。那细节基本上都是没有变,就是我过去这半年多来,对俄乌战争所做的细节评论基本上都是精确的,所以没有必要再重复,我只是补充一些过去这个月新的信息。首先我一年多前战争刚开始,我就指出西方的战略重点其实是打击俄国的金融贸易跟经济,那经过三个月之后,他们就被挫败了,那这个挫败的第一大功臣就是俄国的中央银行行长Nabiullina。



唐湘龙 06:39 

没错,她很稳,她非常稳。



王孟源 06:42 

她非常的...我认为是目前全球头号中央银行行长。



唐湘龙 06:47 

最好的行长,我觉得比Powell都好多了。



王孟源 06:55 

那所以过去这几个礼拜就有人会来问我说,为什么卢布的汇率又跌到 100: 1 了?所以我想要特别提一下,其实他一年多前在金融贸易战获胜之后,这个胜局越来越稳固,而俄国就可以顺势的去美元,这个去美元现在已经基本到底了,就是最新的、他们现在贸易所用的美元已经小于10\%。美元跟欧元加起来三分之一,美元,欧元、日元、英镑加起来 1/ 3,剩下的 1/ 3 是用卢布,另外 1/ 3 是用人民币。那所以第一个我们要先确定俄国的经济跟金融情况是非常好的。



王孟源 07:55 

我先跟大家讲一下,光是过去这个礼拜有两个新的数据。第一个是UBS,就是瑞士银行我以前的老东家,所做的新统计,他说去年 2022 年俄国国内的经济创造了 6000 亿美元的财富,这是近年来的最高峰,事实上绝大多数的欧美国家都比不上。那另外一个统计数字是经过苏联崩溃并且瓦解之后,俄国的经济又重新的登上世界前五、欧洲第一。这当然就...



唐湘龙 08:38 

用 PPP 算的。



王孟源 08:39 

用 PPP 算,就是那个 purchasing power parity,但是我认为那才是有实际意义的指标。



唐湘龙 08:47 

没错,购买力指数了。我们说购买力指数呈现。对。



王孟源 08:51 

就是它的PPP, GDP 已经超过了德国百分之十几了,就是成为欧洲第一。那你说这样的指标你看一下,你会说它的经济有问题吗?没有问题。而且正是因为他去美元化太成功,所以在一年半之后,在俄乌战争一年半之后会出现卢布再度下滑的问题。我跟大家解释为什么。有一个从过去...,有两个原因:一个是要看过去,一个看将来。过去是这样子,在过去这十几年,俄国所累积的外债,不是国家的外债而是他们公司的外债,大部分都是美元跟欧元的,那这些美元跟欧元的外债现在快要到期、要还了。但是他们借这些外债的时候,他们的收入都是美元跟欧元,所以会平衡,他们的外汇收入,但是因为去美元化,所以他现在没有美元跟欧元了,哈哈。所以就会有一些这些外币持续流失的问题,但是这是一个过渡性的短暂的现象,而且压力并不大。但是真正造成卢布现在下跌的最重要原因是一个看未来的那个角度,就是既然俄国在过去 18 个月成功的将他们面向欧美的经济跟贸易转向为面对中国,对不对?那么他们现在外贸平衡考虑的竞争对象是谁?是中国。他们这个外汇平衡针对的对象是谁?不是美元,是人民币。所以你虽然看卢布兑美元贬值,但是那其实是没有意义的,是你硬拿一个不合适的尺来测量这个事情,实际上卢布贬值的对象是人民币,原因就是因为它的工业不全面、有些地方不健全,所以对中国在工业产品上面有先天的逆差,那么为了弥补这件事情,而且再加上它的总体进出口是顺差的,那这个顺差是来自于石油油气的输出,但是这个油气输出有很大一部分是直接进政府的口袋,也是替代税收的。



王孟源 11:46 

那这个替代税收也同样是卢布贬值,它越有效,因为政府的开销是卢布的,但是这个卖出去是外币,所以就是以上的这几个原因造成Nabiullina决定让卢布下滑,但是他是对人民币下滑,不是对美元下滑,只是新闻记者报道的时候硬是拿美元的来当汇兑的对象,这样看起来……



唐湘龙 12:17 

因为大家都会觉得现在全世界的货币兑美元几乎都在下滑,所以就觉得卢布也是兑美元下滑,但是因为它已经到100,所以觉得不合理。所以孟源刚才跟大家讲说,其实重点不是卢布跟美元的关系,而是卢布跟人民币的关系。



王孟源 12:34 

事实上正是因为卢布跟美元没有关系了,所以它的竞争对象变成中国的工业跟货币,所以它必须要下滑。这个那我们先讲完这个经济、金融跟贸易的问题,我们现在来讲一下军事,就是事实上我过去两个月每次都出来说他们乌军的这个进攻连第一道防线都还没有碰到。经过又四个礼拜,他们终于碰到了第一道防线,就是现在已经在他们的重点突破对象打到了第一道防线的前沿,稍微碰到有一个地方就这样。



唐湘龙 13:22 

一个点,就一个点。



王孟源 13:23 

一个点。



唐湘龙 13:25 

对,那因为防线非常长的,那就一个点。



王孟源 13:28 

哈哈哈,然后我上个月讲的...,在过去春季大反攻变成夏季大反攻开始之后,它的前线变动最大的地方其实是在北线,Kharkov那一带俄军在前进。这个也是在过去这四个礼拜也是持续了。所以事实上没有什么太大的变动。那我在6月,就是 8 个礼拜前,当时这个反大反攻才开始的两礼拜,我说,他们的第一波在头一个礼拜就被挫败了,然后他们的预备队还可以再打两波,那现在是第三波刚刚开始。



王孟源 14:14 

就是第二波...它的第一波用机械化的那些坦克跟装甲车猛冲之后被大幅的歼灭,在他们最在意的宣传点上面非常不好看,不好看的结果就是他们上个月打第二波的时候改成用步兵突击。当然这个步兵突击并不是说比用机械化装备突击要更好,而只是他伤亡的打出来的视频跟照片没有那么惨、没有那么明显,因为那个士兵的、步兵的尸体比较没有像豹式坦克被打残那么那么吸睛。所以事实上也是死伤的速度反而是更快了。



唐湘龙 15:11 

加快了。



王孟源 15:12 

对,打到这个礼拜,就是一个礼拜前,他们已经只剩下,在大反攻、为大反攻准备了半年多的那 9 个北约旅,还有十几个本土旅里面, 9 个北约旅已经只剩下一个,就是 82 旅、 82 空降旅。那这个 82 空降旅是唯一装备了英国的挑战者2 坦克,但是只有 14 辆,那绝大部分的他们的机械化装备是美国的Stryker,还有德国的MARDER 就是他们这种轮式的,就是装甲比较弱,但是速度比较快。



唐湘龙 15:57 

装甲运兵车。



王孟源 15:58 

对,装甲运兵车。那所以它基本上不是像第一波所用的 47 旅那么强力,第一波 47 旅是有三十几辆豹2式坦克,结果全部都被打掉,那他们其实这是最后的所谓的精锐部队,而且也不是最重装的,但是现在已经所谓的(scrape) the bottom of the barrel就是已经到底了,不得不拿出来用。那他能不能打穿俄军的第一道防线呢?我从过去半年多已经讲了几百次,就是完全没有可能。现在大家就等着看那个挑战者坦克在战场上冒烟的视频就可以了。



唐湘龙 16:50 

哈哈哈,这实在是很残酷的事情。好,这地方我打岔一下,因为其实在过去一年里面,每次当北约或者美国要提供一些新武器的时候,国际的传媒当中敲锣打鼓的都告诉我们说这武器非常的厉害,俄罗斯又如何、又吃鳖,但我们不在战场上面,我们当然是用看的,但是一段时间之后都发现,对,这新武器刚上场的时候经过一段宣传期的时候好像都很漂亮,包括Starlink看起来也好像很有效果,在西方的这些宣传体系里面,可是一、两个月之后你就发现它就不见了,而且都不灵了,为什么?就是我好奇就是说俄罗斯真的这么厉害吗?就打了半天,西方的、北约的、美国的精锐武器大概也都已经出来了,除了空军、海军没有动之外,陆战装备的精锐我认为在战场上都看到了,但是为什么没有用?



王孟源 17:49 

这个答案其实我也给过很多次,就是事实上这种大规模先进机械化部队的对抗是二战以来的第一次。你想想看,过去的在二战之后的几次有名的战争之中,比如说韩战、越战,然后苏联出兵阿富汗,然后美国去打伊拉克打了两次,然后又打阿富汗,这个没有一个是工业化国家集团之间的对抗。



王孟源 18:22 

那这是第一次,双方都可以打立体,而且有足够的、足量的先进装备来对打,所以尤其现在有无人机还有卫星通讯,这些都是全新的装备,所以事实上一定会有一个适应期。而且俄国他们的态度,因为他们的国土太大、资源太丰富了,他们的治军的原则向来都是边打边学习的。所以你这些新的武器上场,俄军根本就不在乎,因为比如说举个例子,他们现在那个德国的金牛座巡航导弹要上场了。这个东西其实跟英国的那个 storm shadow是...(註:storm译作“突击”,【英國】脫歐鬥爭的細節)



唐湘龙 19:15 

风暴阴影。



王孟源 19:16 

同一个装置,完全一样的。那跟以前的那些所谓的 wonder weapon 奇迹武器一样,都是刚上场的一两月,非常的风光。但是一旦俄军摸着了,实验了一下,摸着了怎么反制,就完全失效。比如说这个 Storm Shadow,还有以前的HIMARS,都是这样子。因为俄军以前没有真正对抗过,所以他们的防空飞弹在硬件跟软件上面都没有针对的经验。有了经验以后,很快的硬件简单修一修、软件出一个新的版本,那就一两个月的事情,然后就完全的封锁,你就无效了。然后西方的宣传媒体就转过来讲下一个奇迹武器。



唐湘龙 20:11 

对,就是说在过去的这一段时间、这一年多,这场战争对我来讲,我觉得影响最大的就是,我发现过去被大肆宣传的西方的这些先进武器的保鲜期,在战场当中实际的保鲜期几乎都不超过两个月,两个月之后那个新鲜感就过了,连提都不提了。那它这种的保鲜期大大的影响了我们对于现代战争的思考的方式,因为平常我们都是活在宣传里面,但是只有这场战争是有机会让你看到这些武器真正在战场当中的表现,几乎没有一款主力的装备、新武器,它的保鲜期超过两个月的。



王孟源 20:50 

而且这也是另一个史无前例的特点,就是双方都有无数的自媒体在报道。



唐湘龙 21:00 

没错,实况转播的战争。



王孟源 21:02 

实况转播的,就是不是由官方过滤过的,所以我想我们对这个战争的观察是史无前例的实时性和细节。那所以真正的问题是这里面有意无意的有很多造假欺骗的行为,尤其是西方的媒体,所以大家在这种 tidal wave海啸式的假新闻之下要特别小心。我相信我的博客是提供了一个可信的,而且有系统的介绍,有兴趣的读者可以去看看。那讲到西方媒体在战争报道上的欺骗,我想提一提昨天的一个新闻,就是纽约时报、华盛顿邮报等等,昨天都有一个头条新闻,就是乌克兰有一艘货轮打破了俄军的封锁,成功的走了所谓的人道主义走廊。



唐湘龙 22:18 

就是运谷物进出黑海的那个走廊。



王孟源 22:23 

对。这一艘船叫 Joseph Schulte,他是Bernard Schulte group,就是Bernard Schulte集团手下的一艘船。那Bernard Schulte的是一个德国的船运集团,但是这一艘船不是一艘运粮船,你如果去看这些细节,你会发现它其实是一个集装箱船。



唐湘龙 22:54 

我们说的货柜船。



王孟源 22:55 

台湾叫货柜船,大陆叫集装箱船。货柜的话,乌克兰的货柜都是从中国进口,它出口的都是农产品跟矿产品,农产品跟矿产品都是散装船来运的,所以它并不代表乌克兰出口。但为什么乌克兰打通人道主义走廊,打破俄军封锁的时候会用一艘货柜轮呢?你如果开始问这个问题,然后开始挖下去,你就会发现这又是一次西方媒体的欺骗。



唐湘龙 23:33 

宣传战。



王孟源 23:35 

欺骗。实际上这艘船是2月的时候就进了奥德萨,半年前就进了奥戴萨,然后他卸货之后就被乌克兰人扣起来了,扣起来当..,然后船员就被驱逐出境遣散了。换句话说,他们是以海盗行为虏获了这艘船,因为那个时候他们的这个运粮协议还在有效中,所以 Odesa 民营的船运是可以合法、可以安全的进行的,所以没有任何理由不让这艘船回去。那这艘船虽然名义上是德国的,可是它实际上的所有人是中国的国有银行,所以它其实是中国的国有资产,那它也是在香港注册,他一旦被乌克兰扣押,那很显然的是要赎金,对不对?,那要赎金以后,虽然绝大多数人不知道,但是昨天被西方媒体拿出来闹假新闻之后,就有人挖出来说,事实上中方一直在试图营救这艘船,但是因为价钱谈不拢,所以 一直没有放。



唐湘龙 24:54 

好,这样子就突然间就豁然开朗了。



王孟源 24:57 

哈哈哈。然后由香港的外交官出面跟乌克兰方打交道,中国的外交部,中国的驻外大使馆就直接跟俄国交涉。这些都还是在当时那个运粮协议还有效的时候,那大家如果对实际的细节的证据有兴趣的话,可以到我博客去看,我刚刚写了评论,但是简单的来说就是乌方忽然的回心转意,没有拿到钱就把这艘船放了,很显然的是因为纽约时报跟他讲说你如果把这艘船放了,我们可以放一条头条新闻,然后说你们打破了俄军的封锁。



唐湘龙 25:45 

宣传。



王孟源 25:47 

为了宣传而放弃了那个赎金。而俄方的反应也是很有意思,就是他俄军根本就是有意的网开一面,OK?然后俄方明明看到了西方的虚假宣传,也没有出来反驳。你要知道俄方在过去 18 个月他们的宣传,外交部跟他们的那个 RT 的宣传是针锋相对的。你西方造一个假新闻我就出来把你喊破xxxx。但是这一次他们一句话都没有讲,外交部发言人不评论, RT 也不评论,后来是自媒体上有人去挖,挖到了一个军事网站,饿国的军事网站在4月的时候有登出这个中国驻莫斯科大使馆向俄国外交部发的公文,请他们协助把这艘船拿回来。



王孟源 26:56 

那我想请问大家,俄国人违反本性,对这件事情什么都不讲,这个风格是哪一国的风格?能够让俄国人买账的是哪一国?(当然只有中国),是不是刚好就是这一艘船的苦主?(没错)哈哈哈,我觉得很有意思的是,你如果去看纽约时报、看华盛顿邮报的话,你就说:“噢,这个俄国这一头大野狼要封锁黑海民船失败,不让乌克兰的民船出来,结果乌克兰把它成功打破了,大家鼓掌”,可是事实上是乌克兰海盗扣押了这艘船,扣了 6 个月,那正是为了这条头条新闻而把它放走,而俄国也心知肚明,所以就配合的把它放走,然后也因为中国的要求所以没有大做新闻。所以你看现在这个国际时事是不是很有趣? 21 世纪这个昂撒媒体假新闻...。



唐湘龙 28:10 

哈哈哈,对对对对,所以其实这一年多光是要看懂新闻了,那难度都很高。所以我为什么每次跟孟源聊聊的时候,我会对孟源很有期待?就是因为他可以,因为孟源做的功课很多方面是超过我准备的那个深度的,所以可以让我看懂很多我自己都看不太懂的事情。



王孟源 28:40 

还好有自媒体,但是问题是自媒体上也是 99\% 是谎言,谎言造假。自媒体的好处是实话存在,事实可以找得到,但是它的坏处是非常难找到,所以大家一起努力了。哈哈。



唐湘龙 29:03 

我们再来看战略上面,就说现在俄乌战争是不是已经接近尾声了?如果接近尾声的话,因为最近的一些的国际的氛围,不管是这个 Stoltenberg的办公室的主任的讲话,在北约的在内部会议的讲话,或者是我们看到Jeddah召开的俄乌的和谈的会议,或者是我们最近看到的,包括刚我在私下跟孟源提到的美国的这个资深的自媒体人所释放出来的讯息,似乎美方的情报机构也都已经认清了这一点,就是有关乌克兰战场上面的所谓的反攻是虚构的,是没有任何机会的,因此只有一个选择,就是在现状,然后寻求和解。是这样吗?那这样是不意味着乌克兰战争就要结束了?



王孟源 29:53 

正确的观点,其实我在过去几个月也已经谈过几次了,就是美国的...美国跟欧洲当然有明眼人,知道乌克兰打不赢,但是他们已经被排斥到第二梯队。就是政治正确要求你说什么,这些人必须要闭嘴,那只有在事实无法否认的时候,他们才会慢慢的一个一个跑出来。



王孟源 30:29 

在几个月前,美国的政坛高层第一个清醒的、愿意面对这些事实的,我已经讲过了,两个月前我讲过了,是Sullivan。为什么?因为Sullivan是Biden的竞选经理,他有两年前从阿富汗仓皇撤退的经验,他知道这在明年竞选会是一个大灾难,所以他第一个愿意正视这些事实。



王孟源 30:57 

那你现在看到的这些,其实只不过是因为证据越来越难否认,所以越来越多的明眼人敢站出来讲话。那我顺便提一下,美国这个,他们的这个社会的全面腐化,在过去这 30 年非常的迅速、非常的严重。我真的 30 年前,我在美国我不敢想象会变成这个样子。我举一个例子,两年前我儿子升大二的时候,他去想要参加、加入学校的社团,然后开始找工作打工。他去面谈的时候,他发现如果坐下来一开始就讲自己多聪明,有什么什么成就,得过什么什么奖,竞赛的时候得什么什么分,马上就会被人家骂是种族主义者或者是男性沙文主义,因为为什么?你这些成绩考得好的话就是对黑人不尊重,因为黑人考不好。



唐湘龙 32:08 

哈哈哈,对呀,这现在的美国的特性。



王孟源 32:18 

但是我儿子是男的,他不能去变性,他又不愿意假装是同性恋。他也不能够把皮肤染色成黑色的。所以你说他除了讲自己多聪明多能干之外,还能够讲什么,对不对?但是美国的社会现在就是这样子。你看他的那个总统发言人,白宫发言人是什么德性,对不对?哈哈,就知道。那你想想看,这就是他们的主流,这些主流嗓门最大,就是我刚刚讲的,为什么他们连在国家最高层政策的决策圈子里面都必须要讲政治正确。一旦是说我们要支持乌克兰,那乌克兰就是完美的。Zelensky就是圣人,他们俄军一定是不堪一击,你绝对不能够讲相反的话,不管是不是事实,所以这就是欧美的现状、欧美精英的现状。



王孟源 33:21 

那我顺便再讲的远一点,照理说现在这个golden billion就是昂萨的这个殖民帝国集团在跟其他的人类做撕裂,他们当然是想要直接跟中国做撕裂,然后全球来包围中国,但是问题是剩下的那些第三世界愿意站中国那边呀,所以你撕裂之后就变成反而是golden billion从人类的那几十亿人中自己撕裂出来了,(自我孤立),OK?,自我孤立。那这样看来好像他们自我孤立,然后他们自己又腐败的这么厉害,选一些最蠢、最烂、最不尊重事实的人来做高层的精英。那你是不是在大多数人类的这边就有必胜的把握?不是。你还是有一个办法可以保证会输。这个保证的办法就是你继续的崇拜欧美的制度跟政策,然后想尽办法去抄他们。照理说这是非常非常可笑的,他们已经腐烂到这个地步了,把 500 年的霸权都已经快要挥霍光了。



王孟源 34:48 

你还在那边崇拜他们,把他们的政策跟理论、学术理论还有思想什么的当做,放在神堂上一样膜拜。但是这在中国就是金融、教育、学术界的现况。他们什么事情都是,连国家中央政府的政策都是从美国、欧洲的理论框架来看的,他们的经验都是去看美国跟欧洲所采纳的政策,你看他们的教育政策是不是这样子?那个高等教育 4 年就增收了 40\% 的学生,这不是从美国,先美国然后台湾,所以换句话说这些政策侵扰了欧美、然后腐朽了日本、然后腐化了台湾,然后已经到了 2023 年了中国还在想要模仿,我真的是想不到世界上有这么蠢的人,但是这就是事实,我也没有办法,哈哈哈哈。明明之所以欧美拖了三四十年的腐化到现在还没有完全垮,靠的是那 500 年的累积。明明就是这样子。但是你就是不愿意把眼睛擦亮去看他们的这些恶劣的后果,我真的是,唉,我自己谈起这些事情来就只有不断的苦笑。(说得真好),这个事情非常荒谬。



唐湘龙 36:33 

那现在俄乌战争如果说是已经走到了要怎么去收场的阶段,对国际形势,我们来谈战略的部分,对国际形势会什么影响?



王孟源 36:44 

对。你这部分你刚刚问了,我还没有回答。对不起,我扯远了。



唐湘龙 36:48 

没问题,你说。



王孟源 36:50 

所以这个所谓的春季变成夏季大反攻现在已经完蛋了。那事实上美国跟乌克兰刚刚在上个礼拜又重新要谈春季大反攻,不过他们谈的是 2024 年春季的大反攻。



唐湘龙 37:09 

因为现在已经冬天都快要到了。



王孟源 37:12 

对,我可以跟你讲乌克兰的攻势到此为止了。但是俄乌战争会不会打完,这个我不能确定,因为主要原因是俄国从政略跟战略的角度选择了慢火煮青蛙的,这一点是我一直到去年 10 月、 11 月才彻底了解的,因为一般的军事常识是速战速决,但是我后来想一想,他是有理由这样的,并不代表说他一定是对的,但是他的这个理由是因为我们我刚刚讲过的,这是二战之后的第一次工业化集团之间的机械化对抗。不但如此,而且双方都是核子核武的大国。所以为了避免升级到核武的这个危险,尤其欧美的执政精英是已经蠢愚蠢腐化到这个地步,俄国多留一部空间...



王孟源 38:28 

…我觉得是可以解释的,我不一定确定这是最佳的方案,但是你可以解释的过来。那既然他已经采纳这个策略,就不一定会趁乌克兰虚弱的时候大幅的反攻。因为就是你刚刚也提过,这些奇迹武器现在已经快要到底了,下一个奇迹武器就是 F16。那现在他们连美国人都,美国人一直都是很犹豫的,这是因为它的生产商不愿意F16 在战场上被打得七零八落,对它的销路是非常不好的。



唐湘龙 39:12 

没错。



王孟源 39:12 

对,这个春季大反攻打得越惨,他反而拖得越久。现在上个礼拜华盛顿邮报有一篇报道说现在在训练的 F16,乌克兰的F16飞行员只有 10 个,而且要到明年夏天。



唐湘龙 39:31 

那怎么可能,到明年夏天?我不要说战争会不会结束,对,你这 10 个人到明年夏天。哇,那那那这真的是,就是不只是杯水车薪而已,真的远水救不了近火啊。



王孟源 39:46 

对呀,所以你现在就是俄军完全占领了战略主动,但是他们过去这一年半已经显示了他们并不一定会去抢这些战略主动,也不一定会去利用这个战略主动。所以我们现在只能说乌军已经打完了,但是俄军会不会真的动手?现在还不能够确定。那我想讨论一下这个外交层面的事情。我不知道你记不记得,记不记得五、六月的时候我有提到,那个原本叙利亚,因为当时沙特在跟很多阿拉伯国家做和解,包括伊朗,跟中东国家,包括伊朗跟叙利亚。



唐湘龙 40:49 

伊朗的外交部长昨天都已经到沙特了。



王孟源 40:52 

对,他们已经互换大使馆成功了。对,那在5月的时候叙利亚跟沙特谈判,就是要重新加入海湾国家联盟。那原本布林肯火急火燎的要去阻止,结果因为那个会议是在5月底,结果他们提前两个礼拜,在会议之前两个礼拜提前让叙利亚重新进入海湾国家。结果布林肯就把这个访问沙特照理说应该取消的,结果延期到6月中。我不晓得你记不记。



唐湘龙 41:35 

我当然记得。



王孟源 41:37 

我还特地说了这个是沙特削他的面子,然后他居然还巴巴的没有取消这个访问,只是把它延期。当时我只是觉得稍微有点奇怪,但是现在我可以解释了,为什么呢?你想想看,Saudi已经跟他们旧有的敌人和解,而且已经明确的站到第三世界的那一边,就是跟美国割裂。他为什么会在8月初办这个吉达的议和会议呢?这个吉达会议,它唯一的议题就是怎么样终结,讨论俄乌战争怎么收场,而且没有邀请俄国。



唐湘龙 42:27 

没错。



王孟源 42:29 

基本上是乌克兰单方到这个会议。来说我们要怎么样结束这场战争?你不觉得这非常奇怪吗?



唐湘龙 42:40 

没错没错。



王孟源 42:41 

对,而且第一个Saudi他是站在俄国那边,他为什么会做这个?,第二个仗没有打完。这个春季攻势其实是乌克兰很明显的打输了。怎么会他们自己自说自话到吉达去浪费时间,其实很简单,就是我刚刚讲的,他们的精英已经完全腐化了,所以你这些脚踏实地还跟现实世界还有接触的人都不敢讲话。所以他们这一次的这个春季大反攻在6月初开打之前,他们的计划,而且这已经证实,后来有泄密,就是在一个月前有泄密出来的文件说,他们认为一个礼拜就可以达到Melitopol,两个礼拜就可以打到黑海的海岸线。所以才会选在6月初之前必须要开打,因为他们7月有那个北约的峰会。



唐湘龙 43:50 

要交成绩单了,本来他是要交成绩单。



王孟源 43:53 

对,他们就是认为,他们那些傻瓜精英真的相信了自己的谎话,真的相信俄国人、俄军是一触即溃,所以他们真的相信两个礼拜就可以打到黑海,然后一个月就可以打到克里米亚,所以一个月之后刚好开那个北约会议,可以正式的将乌克兰收入北约,然后8月就有这个议和会议可以向国际公告:我们这是新的秩序,乌克兰的新秩序。我们已经把俄国打垮了,这个乌克兰正式加入北约。所以他其实这个吉达会议原本是准备要当胜利游行的,就是胜利公告用的。那我觉得很可能就是布林肯6月明明已经没有理由再去沙特还跑去沙特的理由。当然他5月的时候决定6月去,等到6月去的时候这个春季大反攻已经可以看出来不太对劲了,但是毕竟你那个议程已经安排好了,他还是必须要跟Saudi讲说我要你召开这个集大会议。你如果召开的话就算给我面子,我就原谅你的这些背叛,我认为背后就是这样子发生的。然后Saudi的那个的王子就说好, MBS 就说好,你们要办就办,反正我跟俄国老大哥也联系过了,他说没有关系,你去弄,中国会代替我发言,哈哈哈。



唐湘龙 45:36 

对,中国到了。所以你的意思说吉达的这个会议,其实它反而应该解读为,就是说美国在为俄乌战争的收场揭开序幕?



王孟源 45:49 

对,他们原本以为会已经到达大胜的局面,已经打到克里米亚了,所以要有一个正式的国际会议来收场。那因为俄国人知道根本不是这回事,中国人也知道不是这么回事,Saudi也知道不是这么回事,所以他们就,“好,你们要搞就搞,反正到时候丢脸的是你们”。事实上也是这样,他们在那边搞了一个会自说自话,有谁管,你的战场上打成那个鬼样子,有谁会理你,对不对?所以我们,就是我刚刚讲的,我们活在这个 2023 年,真的是一个诡异的世界,这个主流媒体所讲的、所描述的真的是那个好莱坞的产品,就是这主流媒体跟我们画了一个完全虚幻的世界,你如果不仔细去看的话,就完全脱离现实。



唐湘龙 46:49 

好,您请说。



王孟源 46:53 

哇,讲着讲着已经又到了。



唐湘龙 46:55 

对,没关系,我因为我本来想说,因为我很想要听你谈一下,就是说西非,因为台湾或者亚洲对于非洲的感觉是很远、是很淡的,但是你会感觉到,因为毕竟那个是所有的大殖民时代的欧洲殖民强权到目前为止势力还最稳固的地方。那我过去常开玩笑讲说非洲第一大国是法国,但是法国在这一次的就是说尼日尔政变之后,你会感觉到法国如丧考妣一般,那现在为什么?因为在过去不管是布吉纳法索也好,不管就是说这个马里也好,也都发生过政变,甚至于西非的比较没有恐怖活动的,像毛利塔尼亚也发生过好几次政变,政变在非洲不是什么大事情。但是为什么这一次的尼日的政变会引起,就是说法国跟美国非常强烈的反应,到底发生什么事?



王孟源 47:50 

我先跟大家介绍一下。因为西非这个地区一般人不会去注意的。非常贫穷,世界上最贫穷的地区之一。Niger是甚至在西非也是最贫穷的国家之一,为什么呢?因为他是在一个所谓的Sahel这个地区,这个所谓的Sahel就是撒哈拉大沙漠的南半。西非沿海还可以搞贸易的那些国家,经济还比内陆要好一点,因为内陆真的是什么都长不出来。



王孟源 48:27 

那法国,你刚刚讲到法国的殖民传统也的确是这样子。法国跟英国在殖民战争上面打了 300 多年,他是所谓的老殖民帝国,所谓的老是相对于新的美国式的殖民,美国式的殖民主要是通过国际组织,全球性的国际组织,在政治上通过国际组织,在经贸上通过金融手段,就是那个美元这样一收一放,然后在宣传上洗脑,让你相信我们的虚构的世界,接受美国的道价值观这样子的。



王孟源 49:17 

这个是一个新型的殖民帝国主义,但是法国是 19 世纪遗传下来的老式帝国主义,所以它吸收了一些美国殖民主义的手段,就是稍微修正,但是实际上它的本质跟思想还是比较,就是直接掌控,直接掌控跟指定,在细节上面比较所谓的 Micro-management 就是会去管控一大堆小细节。那他这些海外殖民地现在也损失的零零落落。现在剩下的最核心的就是西非的这 14 国,这 14 国里面用来管理这 14 国的一个代理人的机构,就是所谓的ECOWAS西非国家经济共同体系。



唐湘龙 50:11 

西共体。我们平常说西共体。



王孟源 50:12 

对,那当然你殖民的时候掌控的越详细、越仔细,反抗就越激烈,因为你的责任就越明显,就是因为他们的一些贫穷落后就是你殖民的结果,对不对?贫穷、落后、腐败是他们必然的结果。所以这一次的这个Niger 的政变,主流媒体想要报道成军事独裁者反抗民主政权,这其实是胡说八道,我给你两个数据你就知道为什么是胡说八道了。第一个是西非、非洲很明显的是都是部族社会,所以你选举的时候基本上都是一个部族选自己的。但是现任的这个所谓的民主总统是阿拉伯裔,他占Niger的人口只有0. 4\%,他们的这一次领导军事政变的两个军头加起来反而分别来自第一大跟第二大的部族,加起来占人口的73\%。



王孟源 51:34 

那你想想看,为什么所谓的民主选举会选出极少数的人?这其实是殖民地的日常嘛,因为你殖民帝国一定是扶植少数族裔上台来统治多数族裔,因为只有这样子,那个统治者才必须要依靠、才会对殖民帝国有依靠xxxx。要不然你已经代表多数人、多数部族了,你马上会起来反抗。所以这个本质上不是一个独裁对抗民主的斗争,而是被殖民的、被压迫的大多数民众对抗殖民代理人的一个斗争。



王孟源 52:24 

那至于他怎么选出来的,你自己想象。他们自然有办法,因为他对那个社会的政治什么的,政治、文化什么的,控制的非常严。而且你一个 GDP ,每人 GDP 只有 500 美金的国家,能够真的识字的有多少人,对不对?你只要这些识字的人之中能够选出来,那这是很容易的事。因为识字的人,城市中识字的人全部占总人口不到10\%,你这些用那些殖民的手段来影响他们的选票非常的容易。那另外一个统计数字是,这一次他们政变之后,西方还是有一些媒体去做民意调查。其中最有名的有两个,一个是说73\%,另外一个是我比较相信的就是经济学人,他说79\%。73\% 跟 79\% 的什么呢?人民支持这个政变。哈哈哈。



唐湘龙 53:36 

这我觉得很合理。对,因为孟源一开始提到的这个重点,就是这个被推翻的这个西方一直希望能够再把它赢回来的,就是要求,就是说这些政变的这个军头要放弃的、要束手就擒、要迎回来的这个号称遴选出来的尼日的领导人,你看它背景就很特别,你刚刚提到就是说它是阿拉伯裔,而且他是从我记得它是从阿尔及利亚来的,就是在非洲的撒哈拉的南部。



唐湘龙 54:08 

如果你是在地中海的沿岸,甚至在西非洲,这个背景就不特别,就是你的阿拉伯裔的背景、穆斯林的背景,或者你在当地有一些欧洲文化的沾染。可是在撒哈拉的以南的这个区块来讲,你在这个地方要当领导人,少数统治多数,背后的政治运作一定很明显。那所以我才会觉得法国,尤其美国这次虎视眈眈的,美国给我的感觉,颇有想要借着这个机会介入非洲的味道,有这个感觉吗?



王孟源 54:43 

这里的问题是,他们的这一次的政变其实是一个反殖民革命。他们做这些民意调查的时候也统计、问了一个问题,认为哪一个外国他们最信任?唯一一个超过 50\% 的,远远比美国要多出四、五倍的,是俄国。所以这其实是因为俄乌战争,我从俄乌战争一开始讲了 18 个月,说这是敲响第三世界反殖民的警钟,俄国是在最前线、抢在最前线的第一个人,所以他这其实是响应俄国。所以你去看他们流出来的视频跟照片的时候,这些支持政变的民众在他们的城市里面游行的时候,不是只用他们自己的国旗。



唐湘龙 55:48 

用俄罗斯,还有瓦格纳。



王孟源 55:52 

把那个红白蓝的布都已经用完了。这是真正的新闻,西方媒体报道说他们当地已经买不到红白蓝的布了,因为全部都用在..。而且我觉得很可笑的是他们的法国的宗祖国的国旗也是红白蓝。



唐湘龙 56:10 

但是他举的并不是法国,他举的是俄罗斯。他们的宣传都很有意思。对,他们说他们是普丁的朋友。



王孟源 56:18 

对,他们这一次的政变真的是受俄国启发而搞出来的反殖民革命,但是呢,是不是俄国人直接操纵的?不是。我先跟大家讲一下,俄国跟中国到目前为止都没有正式的宣称要支持这个政变的政权。



唐湘龙 56:38 

没错



王孟源 56:40 

为什么?因为你不管怎么样,他还是一个军事政变,还是推翻了所谓的合法选举出来的政体,所以这违反他们的政策。第二个是因为,你也许革命的时候理想是很好的,但是你的眼高手低,那这些军人、这些军头后来的治理能力都不怎么样,所以你也不想一两年后灰头土脸,如果他们搞出什么莫名其妙的东西来。但是另外一个你可以观察的就是Wagner,我在我的博客,我已经两个多月前他们搞出那个所谓的革命的时候,我就已经说这个,他们的结局应该是一个到非洲,一部分到非洲,一部分到白俄罗斯。你现在的确是这样,在白俄罗斯他们开了 6 个大型基地,但是在非洲,他们在撒哈拉边缘一共有四个驻扎xxxx国,但是其中并没有Niger,就是说这就再一次印证了我刚刚讲过的,这一次的革命是当地人受俄国启发所做的反殖民革命,但是并没有俄国的主动参与,他们只是受启发而已。



王孟源 58:13 

受被动启发而已,法国跟美国霸权集团之所以会反应这么激烈,其实也正因为这些政变以及他们背后的这些民众,喊的是反殖民口号,而不是简单的那个政权替换,是因为他们怕这样公开高调的搞反殖民会有连锁效应。所以你看现在 ECOWAS里面 14 国除了Niger自己之外有另外三国反对,坚决反对,这三国都同样在Sahel。



王孟源 59:00 

而且同样都是过去几年,两三年有军事政变的,所以他们那些政变的军头当然不能说我支持你外来的干预,对不对?但是为什么那三个国家可以搞军事政变没有人说要去入侵、然后军事干预?为什么Niger搞出来就事情这么大?,原因就是因为他们那时候没有喊反殖民口号,Niger喊了反殖民口号,Niger不只是喊了反殖民口号,而且是他们的头条,他们就是打着这个旗帜、打着这个招牌来搞这个政变,所以法国无法忍受,所以美国必须要支持。



王孟源 59:48 

那法国不能够自己派兵入侵吗?你现在乌克兰还没有打完,连乌克兰他们都不敢自己下场了,在西非他怎么敢。就必须要让 ECOWAS进去。那 ECOWAS 里面我刚刚说过有四个国家反对,但有四个国家非常的赞成在推动。这四个国家里面,最重要的,真正重要的就是Nigeria。



唐湘龙 01:00:16 

奈及利亚。



王孟源 01:00:17 

Nigeria刚好就是西非第一大国,而且刚好就在Niger的南边,就是靠海的位置。Nigeria因为靠海它的那个地域不是沙漠,而且又有丰富的油藏,所以它的 GDP 、人均 GDP 是Niger的 4 倍多,它的人口是Niger的 8 倍多,它的总经济力量是它的 35 倍。而且Niger(xxxx口误?应该是Nigeria)它的民族性简单来说就是非洲版的印度,哈哈哈。也就是说。



唐湘龙 01:01:02 

连诈骗都很像。



王孟源 01:01:04 

也就是说,他反殖民的原因是因为他是被殖民的,他反殖民的目标是因为他自己想要成为殖民者,他不是因为道德的原因。



唐湘龙 01:01:19 

说的好,印度就是这样,印度是被殖民,他想要倒过来去殖民这个地方。



王孟源 01:01:25 

而且是他们只是反对被殖民,所以当有机会去殖民的时候...。



唐湘龙 01:01:30 

对,没有错。



王孟源 01:01:34 

但是问题是那只是Nigeria的高层精英,就是被法国跟Golden Billion所收买的那些代理人、代理的精英,他们的民众大部分还是支持Niger的、还是支持这个反殖民革命。所以要出兵的话,其实这个 ECOWAS只是借口,真正出兵的是Nigeria,而Nigeria之所以会想要出兵有两个原因,一个是法国跟美国叫他出兵,另外一个是他们的高层有一些代理人,有殖民梦想,就是他想要把Niger变成他的附属国。那在这种环境之下,Nigeria有没有在俄乌战争中为西方战队?没有,他因为态势太明显,他们不敢这样站,所以他们的站队其实还是站在第三世界这边。



王孟源 01:02:40 

那么中俄是不是应该私下告诉Nigeria收敛一点?不要被殖民帝国为虎作猖。我认为是应该的。就是不要去干涉人家的内政。Niger这个政变是内政,但是外来的殖民干预、殖民侵入、侵略跟颠覆,你是有名正言顺的理由去设法阻止的,尤其Nigeria对中国非常的依赖,对俄国也有依赖,而且在大局势上面已经站到这边了。你是有私下的影响力的,这个私下的影响力应该要运用。我们在等未来这一两礼拜看看是不是有运用,运用是不是成功,看他有没有。



唐湘龙 01:03:30 

最后一个问题虽然还没有发生,不过因为刚刚谈到了沙特,也谈到了印度,那因为金砖下个礼拜就开会。那其实孟源从今年初以来一直在,大家在谈国际经济问题的时候,孟源都觉得今年的金砖应该是有戏、是值得关注的。虽然普丁没有办法亲自参加,那印度莫迪颠了半天之后,但是上个礼拜看起来受到了邀请之后,他又说他要去,那现在习近平是还没有正式说会不会去,但是我判断他是会去的,因为金砖毕竟这也是中国的主场。但是我关注的是沙特,第一个就是说今年的扩员沙特进来,沙特对金砖是怎么思考的?会有什么影响?第二个,印度到底想干什么?今年的金砖真的会在所谓的金砖币上面会有重大突破吗?



王孟源 01:04:28 

有没有突破我不知道。我只能告诉你应该发生什么,我不能告诉你实际发生什么。因为现实是有很多蠢人跟坏人的。所以你这个为人类利益最大化的策略不一定会实现。但是有关去美元化,或者是把美元从国际储备货币的地位打下来,是我的博客从 2014 年开始最重要的主轴,我从 2014 年讲到 2022 年,说你必须要把美元替代掉。



王孟源 01:05:09 

在 2022 年2月俄乌战争一开打,第一时间我就说时机到了。因为俄国已经自愿的作为第一个冲锋,第三世界反殖民革命的的领导,那这个时候你就会有人愿意配合中国来搞这件事情,所以你必须要搞。然后我说,但是要搞这件事情最大的问题是印度,所以最好是把印度排除出去xxxx。



唐湘龙 01:05:47 

哈哈哈哈哈,可是很难哪,他都已经加入了,他就变成是搅屎棍。



王孟源 01:05:54 

在博客写得很清楚,而且反复的。



唐湘龙 01:05:58 

对,真的有远见。就是我觉得这些证据就没有远见xxxx,王孟源的博客要看一下。



王孟源 01:06:04 

而且从金融的角度来看,当然你也可以用黄金、用金本位来搞,但是世界上的黄金不够多,而且这些黄金里面中国跟俄国手上所占有的百分比不够高,所以不划算、不实用也不划算,最好是用篮子货币,结果我把它写下来建议了之后,其实也呈上去了,但是当时的那个人民银行行长是易刚,所以忽略xxxx掉了。但是两三个月之后Nabiullina建议用篮子货币,那我就很高兴有她的背书,我是觉得很荣幸。至于易刚反对呀,那没有办法,负负得正,这种烂人所反对的就是正确的,所以我也没有什么,哈哈哈。



唐湘龙 01:07:06 

你自己竟然把背后的故事讲出来。好,当然对于俄罗斯来讲,俄罗斯在过去这一年多里面,我觉得他对这个世界最大的贡献,就是他已经完成了去美元化的金融实验,这个对第三世界国家来讲很有启发性,就是像他这样的一个经济体,不大不小、也算大的经济体,在这样子一个被围堵封杀的情况之下,自我完成了去美元化的金融实验,这个是了不起的事。好,那我刚刚提到在全球的第三世界国家金融里面来讲,沙特是很重要的,沙特这是很可能的会进入到金砖扩员的第一波,这个对金砖体系会有什么影响?



王孟源 01:07:52 

其实 18 个月前我开始讨论必须要把印度排除的时候,我也提到了这个,如果不能够排除的话,退而求其次就是金砖拼命的扩员,因为你金砖如果只有 4 个、5个国家的话,那大家都是平等。中国跟俄国就必须要听印度跟巴西的话。



唐湘龙 01:08:23 

对,被他牵制。



王孟源 01:08:24 

被他们牵制。但是如果是有十几二十个国家的话,那么就是谁的拳头大、谁的口袋满,谁的话有力量。所以你如果不能够避开印度的话,就只好扩张金砖,那扩张金砖的第一个人就是沙特,那当时我也讲,正因为这个冲突,所以中国跟印度一定会在扩员事情上有矛盾,那现在也是证实了。所以这种事情都应该是提前 12 个月、 18 个月就讲的。你这个市面上绝大多数的所谓的评论员都是事后诸葛亮,我觉得没有什么意思,那不论如何,印度很明显的是不想要扩员,巴西也不想要扩员。同样的理由,因为他们在这个组织里面的权利会被稀释,但是如果要扩员的话,Saudi是最合适的第一个人,而且他也有很强烈的意愿,所以我不能够保证会有什么结果,反正我们看他们的折冲嘛。我想因为易刚的关系,他们没有选择那个好的方案,而选择这个次优方案,所以自然会撞得头破血流,所以到最后能不能把这个墙撞倒,我们也不知道这个,这堵墙的名字就叫做印度,哈哈哈哈。



唐湘龙 01:09:55 

这个我最近常谈到。印度将来是会有一些麻烦的,因为中国在过去,因为中国对印度。其实我认为姿态一直都相对来讲是很温和的。那也一直想要去安抚印度,想把中印关系弄好,可是我觉得倒过来很。



王孟源 01:10:14 

原因很简单,因为西方要制衡中国最大的筹码就是印度。



唐湘龙 01:10:22 

就是印度。



王孟源 01:10:22 

所以他在经济上拼命的去捧印度,就是在媒体里面,财经媒体里面。那我刚刚,我半个多小时里面讲过中国的这些所谓金融、教育跟学术精英,他们基本上是昂萨媒体上面讲什么他们就信什么。所以他们就认为非要...印度已经不可战胜,所以必须要跟着昂萨体系到那边去帮助他们发展经济,这真的是愚不可及,其实是应该有国家政策来遏制他们的,印度人非常的邪恶,他们是天生的诈骗犯。



唐湘龙 01:11:03 

哈哈哈。好,这最后讲了这段话跟我想的一样,包括最近那个共和党的那个印度裔的那个总统候选人,那个叫啥名字的,我每次听他讲话的时候我都,噢,俨然就是美国的苏纳克的味道。好,这个时间又耽误了孟源,孟源是时间又 over 了。来,我先感谢一些我们的观众朋友来,从这个第的 Russia 猪感谢,然后 Polon 感谢,然后耶尼王谢谢,然后 broad of CAN 001 谢谢谢谢你,然后地球感谢他说每个月最期待的一天,我也是。好,再来黄旗伏,感谢我我我觉得你们都很有诚意,都没有流化,就是纯粹,就只是感谢我王孟源。再来。



王孟源 01:11:58 

今天讲的很随意,希望大家。



唐湘龙 01:12:01 

我觉得我这并不随意。我,我觉得你有好几个观点对我是很有启发性的,当然海尔森框框宽。好的,他说感谢王博士的真知灼见,对俄乌双方军事行动成竹在胸,对大陆金融的鞭策,虽然大陆很多方面都是后知后觉,但没关系,大陆还是结果,因为大陆比较保守,我觉得大陆虽然后知后觉还是存在的。孟源刚刚所说的对西方的知识体系跟宣传体系过度的依赖,甚至有时候会产生了性侵动摇的情况,但是因为它相对稳健保守,所以不会犯太大的错误了,我个人认为。好,再来林冠感谢他。虽然王老师不评论台湾,但是想请问 2024 年台湾总统总统候选人有符合你认为实用主义者的人选吗?



王孟源 01:12:50 

这个我有话讲,哈哈哈。



唐湘龙 01:12:51 

好,是你自己要讲的。好,你说你说我。



王孟源 01:12:55 

我不评论,因为我已经讲过,我不能评论,但是我讲一件私事,我觉得很好笑,就是我有一个大学同学现在是某前台北市长的助理人。那就是三个候选人里面。



唐湘龙 01:13:17 

其中一个。



王孟源 01:13:23 

去年我回台湾的时候跟他聊过。然后,他那时候刚刚加入那个团队,然后我说好,那很好。然后后来我的表弟也来找我,我已经 20 多年没见过我这个表弟,他是在台北大律师的,退休了,那算是相当成功的那个法律界人士。那他来我家找我,我也吓了一跳,因为很久没见了,但是就是聊起来,聊起来以后我吓了一跳,因为他跟我一样都是出生嘉南平原的,我认为我自己是出生深绿家庭,然后没有染上的台独的思想已经是很奇妙xxxx了。我一直以为他们家也是这样,但是他刚好跟我一样也是家里面的异类,就是他到了我客厅坐下来,然后就开始骂这个执政当局,刚好他是现任总统的以前政大的学生,他以前在政大,现任总统那时候的学生,哈哈哈哈,他骂的特别的尖刻,那时候我也就听听,反正这个事实怎么样我也没办法验证,我只是觉得很好玩。但是后来,后来回来以后想一想,跟我那个当竞选助理的那位同学又联络以后,他说我们现在正在找社会人士加入我们的团队,因为我们没有什么人。那就想到我的这个表弟不是很合适吗?



唐湘龙 01:14:59 

哈哈哈,然后你推荐他。



王孟源 01:15:03 

所以我就想要联络我的表弟,但是我忘记跟他要他的电邮信箱跟那个手机号码,所以我就打电话回去跟我妈讲,我要我的表弟的电话号码,她说你要干嘛?我就把这些事情解释一下,因为在家里我都是实话实说的,这个没有想到怎么样,结果我妈说不给,哈哈哈哈。



唐湘龙 01:15:29 

哈哈哈哈,你看你们家里面多分裂啊。(哈哈哈哈哈)对,这个在台湾的很多的家庭。



王孟源 01:15:39 

(为什么不给?)那个候选人是世界上最大的恶魔,支持他候选人的事情,我不能把脱发迷惑xxxx,哈哈哈哈哈哈。



唐湘龙 01:15:55 

好,这个是一个不错的story。来,我们再来看 00 再来看这个温迪利,感谢你那居民摇。摇哈,就居民摇,感谢你,范志林。谢谢就志林,感谢马克猪,谢,谢谢你,他说感谢孟源教授,如果台湾打城镇战,你预估能够撑几天?两岸如果政治谈判你评估呢能够争取到什么优惠?这个这题目很大,孟源你要回答吗?还是我们下个礼拜 。



王孟源 01:16:25 

这属于我博客不能回答的题目了。



唐湘龙 01:16:26 

好好,我们下个月好了。对,因为你可以谈的东西太多,而且我一直希望你能够再多开一集,但是这这看看你吧。再来感谢龙哥让我认识我王孟源。对,我告诉你,我,我认识王孟源之后,旁边就有一些在过去我不认识的人。那认识王孟源就很骄傲了,来跟我认认亲,就我是王孟源的谁谁谁。好,再来的马罗马克猪还这样感谢他说各国仍然持续抛售美债,你仍然坚持上次美国经济软着陆的观点吗?这个你可以回答一下吗?就是因为最近美国的经济的变数很多。那因为上个月你都认为美国经济的危机已经过了,它是软着陆,你现在仍然持相同的立场吗?



王孟源 01:17:13 

当然,这些预言都是我一旦做出来,都是持续有效的,无限期有效。要不然我的博客怎么会鼓励读者去读旧文了,对不对?你一般的那个新闻评论员都是像那个卫生纸一样用完就可以丢掉的。我写的东西可不是。



唐湘龙 01:17:32 

吧。好啊,来。好,那你就再去看看孟源的博客,再来索罗门陈。感谢,然后 journey 谢谢你,然后 sky 太感谢他说孟源老师开讲必听来自洛杉矶的问候张文玲,感谢,然后微愿书感谢,然后在前面跑掉,然后 m may z 感谢他支持龙行天下那支持王博士来自美西的问候。再来 n m 跨,感谢在菲律宾好,再来 then Andy 感谢,好,感谢所有的听众朋友,尤其一万六千位的观众,因为王孟源我很诚实。就想梦妍在说话的时候你必须要很专注的听,而且有一些是需要一些新闻准备跟知识门槛的。你刚开始进来听的时候,虽然孟源在线上我很深,就是说你刚开始进来听的时候你一定会觉得有点深或者有点困难,但是你需要一点时间,你会进步的非常快。这个事我找孟源的原因也是最大的收获。好,感谢今天星期五的时间,在美东的时间现在都已经是深夜了。那感谢呢?孟源透过连线跟大家分享你的知识跟你的分析,感谢孟源,下个再见。好,拜拜。周末快乐。拜拜。



\twocolumn[\begin{@twocolumnfalse}
\section{、國際政治與個人事業規劃}
\subsection{20230914}
\end{@twocolumnfalse}]Credit: 网友S——校对整理基本完毕。三处备注(听不清)待校。



【大学讲座】国际政治与个人规划

2023/09/14



王孟源00:00

发生在一战和二战之间的那二十年的去全球化,比目前在推动的去全球化还要彻底。而19世纪末期到20世纪的前十年,全球化的热情跟程度在100年后,20世纪末跟21世纪初那十年是完全可以相提并论的。所以我们现在面临这个去全球化是一个很自然的、很必然的既有霸权打压新挑战者的一个过程。那100年前的美国跟100年后的中国都一样试图韬光养晦,躲避既有霸权关爱的眼神。但是美国成功了,而中国却注定要失败。为什么?这里有一个很基本的差距是在于:百年前挑战英国的最激进的挑战者,德国跟日本,他们的目标都是想要复制英法,他们的目标是要分一杯羹,也就是说他们并不是站在解救被殖民者的立场,而是你的殖民地有这么多好处,我们是后来者,我们没有地盘,所以要分一点。

希特勒屠杀犹太人跟吉普赛人,跟美国屠杀印第安人其实上没有本质上的区别。希特勒去占领俄国攻击苏联,需要他的空间——所谓的生存空间——跟英国当初殖民澳洲、美洲甚至印度也没有什么区别,他的目标是一样的。事实上我们读历史的人都知道希特勒最崇拜的国家就是英国。为什么崇拜?因为他是世界上最成功的殖民帝国。但是我们目前所看到的这个由既有霸主所割裂的去全球化过程的,我们这一次看到的却是所有所谓先进国家跟第三世界的割裂,这个割裂不是先进跟先进国家之间的竞争,而是整个先进国家跟被压迫的第三世界的。这是一个很大的不同。

 

那要了解我们目前的这个国际政治秩序,你必须要知道所谓的Golden Billion。这是美国在成为霸权新霸权之后在二战之后七八十年努力建立的一个新的秩序,是一个新发明的秩序。跟英国不一样,英国他的秩序是有英国本土,然后有一堆旧有的原住民genocide(种族灭绝)之后取而代之的地区,然后还有一些是你没办法genocide,然后你把他占领起来纯粹榨取出。但是呢在这两者他们都是有直接的政治管控。

美国殖民的最新的、最高端的、现代化的殖民帝国并没有这么直接,这么粗暴。他实质上是一个很复杂的系统,但是简单来说有三个要素,就是军事压力、金融剥削跟宣传系统,这是他的霸权三要素,这是美国的发明,也是为什么美国能够成为如此富强国家的基本基础。那正因为这个差距,所以当年英国面临了许多新兴工业国的挑战,而现在美国却能够整合所有的新兴工业国来对付第三世界,因为他们至少在经济上面似乎是可以统一战线。这也是为什么我刚刚说,韬光养晦对100年前的美国来说是一个正确的战略,但是对中国来说是一个鸵鸟式的永远不可能成功的战略。

这个美军、美元跟美宣为三要素的终极版殖民帝国主义,在80年代的时候曾经同时一举击败两个强力的挑战者。一个是政治上的挑战者苏联,另外一个是经济上的挑战者日本。他们的强处各有不同,他们的弱点也各有不同。你去打败他们的方法似乎也都有不同。但是你仔细的去看有一个共同的规律,军事压力跟恐吓关系,金融的剥削跟打压,但更重要的是宣传洗脑。宣传洗脑的目标除了老百姓之外,更重要的是要让他们的高官自己相信,他们必须采纳美国制定的政策,这些政策是我所谓的割肉鹰的自杀性政策。

所以最终是苏联的领导人跟日本的领导人他们选择了政策将自己的国家爆掉。这是为什么我说这是一个终极进化形态的殖民帝国,因为他的代价近乎于零,他的收获却是以万亿来算。美国200多年的历史之中,对外的侵略跟颠覆一共有400多次。你们如果是去念纽约时报的话,你完全不能够想象这件事实。这是因为在纽约时报上面,他们讲的是世界的正义跟公理做什么事情,而不是美国做了什么事情。但是呢你如果去看看他400多次的对外侵略,从1812年的第二次独立战争——或者是The War of Eighteen Twelve,就直接叫做1812年战争——之后,美国从来没有独自主动的去挑战一个一级强权,即使在二战后他自己是超级强权之后,从来没有单独主动去挑战。如果你非要动手不可,你的美元跟美宣这种让他们自曝的方法没办法成功的话,他退而求其次是搞间接代理人战争,去年的乌克兰战争就是一个最典型的。

韬光养晦,就像我刚刚讲的,因为美国可已经整合了所谓的Golden Billion,也就是工业化的那十亿人,所以中国要试图跟他装孙子,永远不可能,因为美国远远的看到就只有你一个人去(挑战他的地位),他并没有像英国当年那样子面对好几个国家对他挑战。

但是中国的运气很好,在2001年的时候发生了911事件,所以有十年的时间,美国的军事力量用在中东;然后到了2008年又出现了一个很严重的金融危机,在那之后他们一直到2014年才勉强恢复危机。这里有很多经济是可以去看的,比如说量化宽松,从2008年到2014年才结束。换句话说,中国莫名其妙的自己什么都没干,就得到了14年的喘息机会。

在这期间,中国的工业产值在2010年超过美国。目前中国的工业产值是美国的220\% ,他的PPP(购买力平价)GDP在2014年超过美国,这是史无前例的。我刚刚提到,在80年代,美国成功的解决了苏联跟日本的时候,苏联的PPPGDP占美国的70\% ;日本的PPPGDP占美国的60\% ;现在中国的PPPGDP已经到达美国的120\% ,几乎等于当年苏联跟日本加起来。

我们说PPP是Purchasing Power Parity,这是因为国际上不同的国家同一个产品会有类似的定价,必须要这个产品是能够自由国际贸易才有意义。比如说你买一台电脑,你的这台电脑在美国的价格跟台湾的价格应该是类似的——不一定是完全一样,但是应该是类似的——就是因为这台电脑可以在台湾生产然后送到美国去。但是你如果是说一个汉堡,或者是你要请人帮你理发的话,你不可能请美国的理发师来这里帮你理发或请台湾的理发师。所以在美国理同样一个发,他的价钱可能是台湾的5倍,那你说美国理个发真的有台湾的5倍那么好吗?没有,可能他们更草率的,也更糟糕。所以你必须要根据这个来做修正,这个修正的标准就是所谓的Purchasing Power Parity,所以我说这个PPPGPT就是根据这个修正把你经济产出的价格全世界统一化,然后来做比较。对不对?我们所谓的GDP其实是一个会计手段,用来简单估计经济的产量,他原始的目的是同一个国家用来比较今年跟明年之间的经济波动,不是用来比较不同的国家。但是你如果非要用来比较不同国家的话,勉强可以用PPPGDP。OK?

在2014年美国终于喘息过来之后,他们才有余暇来考虑打压中国的事情。但是在那个时候中国的经济体量已经大于美国,所以他们内部也有争议,吵一吵呢选出一个总统叫Trump。(笑)他决定走比较直接、简单的手段,直接在经贸上动手。然后呢四年之后又换了一个总统,那这位总统就认为,除了在直接的经贸手段以外,我们还必须要搞我们的这些老手段,就是全面发展。所以就变成经贸再加往前的那种间接的政治、金融、军事还有文化、宣传,全面的发展。

 



到了2021年这个Biden上台的时候,其实他的国务院也进去了好几个所谓的NeoCon,NeoCon就是新保守主义,但是呢这些名字跟内容是天差地别,其实是一个广告标签,它跟新没有什么关系,跟保守也没有什么关系,它甚至不是一个组织,他们就是一群穷兵黩武的疯子。那他们的这个穷兵黩武的目标就是要为美国的宣传跟军事体系找一个打击的对象,如果能够成功的打倒以后呢,他的金融体系就可以继续全面的掠夺。这个他们在苏联跟日本被打垮之后吸髓知味。

当年大家应该回忆一下纽约时报上面常常讲的所谓的苏联垮台之后就有很多寡头,然后他们就把俄国的经济搞垮了。这些寡头呢就因为掠夺了国有资产而一下子变成Billionare(亿万富翁)。可是你有没有想过当时苏联垮台的时候谁有那个本钱,去用几亿的资本收购价值几百亿的国有资产?这些人其实都只是英国跟美国财团的白手套,他们是这些昂萨跟犹太财团的中介的买办。真正获利最大的是背后的这些财团,他们真正是用1\% 或甚至0.5\% 的价钱去买到矿产、工厂,还有其他的真正有长期价值的东西。





Golden Billion指的是先进工业国家加起来一共有10亿,全世界有80亿人,其中住在先进工业国家里面的有10亿人,叫做Golden Billion,那这个Golden Billion的核心就是美国;然后它的外层,第二层是所谓的Five Eyes(五眼联盟),就是Anglo-Saxon昂撒的另外四个国家:英国、加拿大、澳洲和纽西兰;然后再外一层是其他的白人国家,主要是西欧的国家;最外一层是东亚的三个人群,日韩和台湾。

美国治理世界是经过一连串的国际组织,金融方面就是IMF、World Bank;法律方面,国际法方面就有所谓的国际法庭;然后政治上呢更是五花八门,有G7、军事上有北约,基本上是许许多多的国际性组织。这些国际性组织就是把这四层壳联系起来,从外面吸血,我往里面补,不断的把利润向中心集中的一个过程。那在平常的时候呢,你可以靠外面的第三世界那70亿人来吸血,吸血像什么?像这个Niger(尼日尔)。

Niger刚刚有一个政变,这个政变让法国非常的不高兴。为什么法国非常不高兴?除了这个政变的主导人喊的口号就是要反法国,我要把法国踢出我们的国家之外,实际利益上,Niger是法国——法国有70\% 的发电是用核能发电,全世界比例最高的,他的那个铀矿基本都是从Niger来。他法国人跟Niger买铀矿的价钱呢是80欧分,0.8欧元一公斤,Niger跟他要价200欧元一公斤——这就是殖民跟独立的差别。80欧分跟200欧元就是你是不是被殖民的差别。那你以前跟着美国老大混黑社会,你可以压榨剩下的那70亿老百姓;但是你如果这些70亿老百姓开始反抗了,而且自己又做了一些愚蠢的政策,这时候怎么办?你只有回过来吃自己的小弟。

这一次的俄乌战争,我待会会详细的讲为什么它也是一个案例,但是呢在座的听众应该更熟悉另外一个案例,就是台积电——台湾的护国神山——莫名其妙的就被迫把最先进的制程到美国去建一个厂。这如果不是被殖民,哪一个国家会愿意这样做?

至于日本、韩国、英国跟欧洲其他国家是怎么样被美国在过去这两年吸血的,这个博客上有写文章大家可以仔细去看。



 

不过再回头来讲为什么美国会发动乌克兰战争。你没有听错,发动乌克兰战争的是美国,不是俄国。怎么发动的?从2014年到2022年之间8年,根据联合国的统计,乌克兰用美国北约提供的武器跟炮弹杀害了东乌独立军一万六千人——这不是我在俄罗斯的媒体上看到的,而是联合国的——16000人的平民死亡,被炮击的平民死亡。这个俄国还可以忍,即使这些人他们认为是俄国人。美国掌握了战略主动,他可以随时让各国采取军事行动。怎么采取行动?他只要说我现在让乌克兰加入北约——这条红线从苏联解体到现在俄国从来没有松口。哪一个一流军事强国会容许你的一个邻国,加入世界上的专业侵略组织?美国愿不愿意?你如果让墨西哥跟中国定一个军事同盟,看美国会不会愿意。所以表面上是俄国发起一个军事行动,实际上是美国逼他发动一个军事行动。

我为什么要详细解释这一点?因为台海也是这样子。台湾有些人说,噢中国也不愿意开战!中国当然不愿意开战。他的经济成长率是美国的三倍,长期来看只要再多等20年根本打都不用打。但是问题是,美国有一个办法可以逼中国马上开战。我想大家都知道那个是什么。

那美国的NeoCon为什么要发动乌克兰战争呢?他有两个原因:一个直接,就是复制当年30年前打垮苏联,然后白占了万亿美元级别产品的那个过程;但是另外一个更长远的战略目标:当时中国跟Merkel领导的欧洲在谈一带一路,甚至在谈自由投资条约——我不晓得大家记不记得,3年前的时候。这如果谈成了以后,中国跟欧洲的经济就会被整合起来。所以你必须要在这个条约谈成之前坚定欧洲的态度。要坚定欧洲的态度,就是要强化北约的价值。

大家回想一下,三四年前法国总统Macron说了一句话:北约已经脑死了。美国就是不能够让他脑死,必须要证明给欧洲的人说这个北约还非常非常的重要。



 

那你能不能直接在台海打起来?直接在台海打起来的话,欧洲人就说:诶,我们谴责,强烈谴责,你要我吐多少口水都可以,但是我们绝对不会动手。你要怎么样让欧洲对美国赞同?要对手必须是俄国,欧洲才会出手,因为这是在他们隔离的事情。而且这是他们的教育体系跟媒体吹嘘洗脑了80年的事情:俄国人就是欧洲的最大威胁,是德隔壁的大野狼。所以你必须挑起乌克兰战争才能够整合北约,这件事必须在台海战争之前的一到两年发生。



 

所以美国的态势就是他有战略主动权。他在乌克兰也有战术主动权。为什么我说他有战术主动权?乌克兰跟北约接壤,乌克兰被包围的态势是一半被北约包围,一半被俄国包围。你要把弹药跟补给送进乌克兰简单的不得了。但是台海是这样子吗?台海的战略态势是一样的,战术上是完全不同。台湾是海岛。你要补给的话只能够经过海上交通,这个海上交通线距离大陆海岸线不到300公里。大家想想看300公里是什么意思?100\% 的弹道导弹都可以打得到,即使是最小的弹道导弹都可以打得到。所以在东亚美国没有占据主动权,这是一个很大的差别。所以那些指望台海战争开启之后美国至少还会像对乌克兰那样子,虽然是让乌克兰战斗到最后一个人,但是至少武器是免费的或者近乎是免费的拼命送的,在台湾不可能复制,即使是美国愿意都无法复制。

而且你看即使在乌克兰,这个军事现在已经拖成一个长期的消耗战,这绝对不是美国原本的企图。美国原本的企图是,让乌克兰只要撑4到6个月,然后他们的金融打击跟制裁就能够让卢布完全垮掉,垮掉以后就连锁反应造成俄国的经济完全崩溃,完全崩溃以后内部的反对势力包括上街的民众就会推翻普丁政权,然后西欧的寡头就可以跟昂撒的财团合作,把普丁好不容易培养出来的一些国有资产再次拿到手里。

这个企图被俄国中央银行的行长Nabiullina粉碎了。当然Nabiullina本身是全世界最优秀的中央银行行长,但是她从2014年上一波乌克兰危机就得到普京的支持,全力的准备这种事情,准备了8年。为什么2014年俄国没有趁机采取军事行动?因为那个时候乌克兰也是欺负东欧,而且不但讲要加入北约,而且还讲说要获得核子武器。那个时候俄国不敢打,俄国想要拖延。欧洲,德国Merkel跟当时的Hollande(奥朗德,2012—2017 法国总统)也想要拖延。但是普丁为什么想要拖延?因为他知道当时俄国还没有准备好,承受不起金融打击。



 

真正的战线永远都是金融跟经济。所以准备了8年之后,到2022年普丁才准备好,才动手。他一旦在开战的头两个月度过了这次金融打击,事实上俄国已经立于不败之地。因为在军事上,北约真正是纸老虎。打仗本身,大家现在如果愿意去探讨真相,而不是只相信纽约时报的话,应该都已经看得很清楚。那在我们的东亚西太平洋战线呢,美国要靠军事斗争胜利的机会更加更加的难,所以更加更加的必须依赖金融跟经贸。那中国的经济体量远大于俄国,他的工业体系完整性远高于俄国。但是呢很不幸的,中国的金融主管远远弱于Nabiullina,哈哈哈,事实上他的前一任的人民银行行长在过去的三年,基本上他是作为美联储北京分行的行长(笑)任务圆满的完成,哈哈哈哈...



 

自2009年初開始,經過三輪的量化寬鬆,聯儲會一共新印了37000億美元,買下聯邦債券和MBS大約各半,再加上2008年既有的8000億,總共是45000億(參見下圖;這是當年美國GDP的27\% ,創下新的記錄,作爲對比,二戰最後一年聯儲會的賬目是GDP的20\% )。——《從回購利率暴漲談美國經濟周期》



熟悉我的博客人应该知道,在2019年我写了一篇文章(從回購利率暴漲談美國經濟周期),我说这个美元已经滥发了45000多亿,已经到了急救的阶段,很快的下一波的金融危机会是通胀性的,而且严厉的程度可以遏制美国的霸权。如果中国能把握这个机会呢能够一劳永逸的去美元化。结果这个危机是什么时候到达最高峰呢?就是两年前,2021年。那个时候因为新冠的关系,基本上美国只要能买到东西根本不在乎价格。换句话说,人民银行即使把人民币提高百分之五十,也不会减少出口的货柜的数量。他的出口额会增加50\% ,对美国的通胀的推力会加倍。因为当时的通胀的引发其实是因为美国国内的供给链,当地的铁路跟港口的运输跟不上,工厂也必须要因为防疫的措施而没办法运行。那保守的估计你可以升值50\% ,多赚50\% 的钱,不是把50\% 的利益哦,而是50\% 的价格都变成你的毛利。与此同时,你可以对一个战略流氓能够一劳永逸。那他有没有做?没有,他很乖的贬值了。哈哈哈哈...那现在的人民币对美元的汇率是7.3,比2019年时候还要低,你说这配合的好不好?乖不乖?



 

OK,这个对中国的外汇,因为中国的外汇有很大笔是用美国的国债——长期国债或长期的非国债;当你的通胀上去的时候,过去几年他的通胀从1\% 升到5\% 、6\% ,三年下来他的国债价值弱了40\% 。你如果有1万亿的美国国债外汇,你损失了实际的4000亿美元。这还是短期的,可以计算的损失;真正最大的损失是美国缓过气来了,现在继续要用半导体制裁,要做军事挑衅。

除了中国的积极配合之外呢,美国安然度过这一次通胀危机的手段主要是去吃英国和德国。因为俄乌战争的关系,我们要知道在战前俄国跟美国的贸易基本上是没有很大的比例,主要的贸易关系是俄国跟欧洲之间。俄国的经济体系是欧洲体系,他们其实是一个共生关系,俄国提供能源跟原材料或者是其他,然后西欧尤其是德国提供技术跟高度教育。

德国经济的基础有三个,一个是中国市场,因为中国最喜欢工厂的生产线,刚好是德国机械的最大的可以开放中的一条线;第二个就是廉价充分的俄国能源;第三个是来自东欧的劳动力。俄乌战争一下来把三个都打跑了,其中百分之百完全打掉的是来自俄国的一个价值体系。我想大家也许不晓得,在19世纪末20世纪初,这些西欧国家跟美国在做第二次工业革命竞争的时候,德国最强的一点是什么?是机械?他的机械很强,但是瑞士的机械也很强,对不对?法国意大利也有很强的机械公司。德国远远最强的是化工。化工最依赖的是什么?廉价的能源,什么东西?就是天然气。

所以,美国在80年代去工业化开始将产业外移。到2002年20年前他每个月的工厂建筑投资——就是一个房地产建筑有三大类:一个是办公大楼、一个是私人住宅、最小的一个是工厂。这个指标在20年前到达最低点,每个月20亿美金。经过20年的折腾,包括Trump四年的全力支持——他们所谓的Trump mercantile approach商业主义,重商主义的政策——在2021年这个数据涨了三倍,到每个月60亿美元。大家要不要猜一猜今年这个数字是多少?160亿美元。两年,从哪里来的?台积电、三星还有德国的化工。所以过去这两年美国是怎么从这一轮的通胀性金融危机爬出来的?是把...不是eat alliance's lunch,而是eat alliance like lunch(不是吃盟友的肉,而是把盟友当肉吃)。(笑)而是吃盟友像吃肉

那现在全世界的先进经济体,德国的成长率是负的,是1.3g(-1.3\% ),英国的成长率是0.5\% 还不到,但这是因为英国有欧洲最高的通胀率,而当通胀率高到一个程度的时候,就会有误差。你如果少算0.5\% ,这多出来的0.5\% 就会变成你的GDP成长率(笑)。日本也是一样,日本有35年来最高的通胀率,所以他的经济成长率是0.2\% ,实际上是,你的通胀率算的时候有0.5\% 误差的话就送到成长率去了。哈哈哈哈...

我们再看看我刚刚提到的Niger的革命,我们看看土耳其总统的发言、伊朗的发言、沙乌地的发言、南非的发言、巴西的发言、阿根廷的发言,连墨西哥总统的发言,他们在这次俄乌战争之后站队是站在哪一边?不是美方,而是俄国的那边。就是美国自信满满,觉得他前所未有地整合了所有的先进国,工业国;他没想到的是,他的对手俄国在中国的支持下,整合了所有的第三世界。为什么第三世界有这个胆量整合起来?不是光是因为俄国一个人起义带头,而是因为——过去500年被殖民帝国钳制控制的是什么?资金、技术跟人才。现在资金、技术跟人才可以到别的地方,因为这样子才可能第三世界整合起来抵抗殖民;因为这样子殖民帝国才绝对不容许中国继续存在,这不只是中国自己直接挑战美国霸权,而是中国的存在就让第三世界不愿意再做奴隶。

好,美国已经成功的度过了这一轮的通胀性金融危机,下一次会有金融危机应该是发生在长期国债和赤字上。这一点呢主要是,他的财政赤字还在持续的指数增长,进入了指数生长的状态。你如果去看今年的财政赤字,两万亿,GDP的10\% 。你再成长一下变成GDP的20\% 的时候——到10年之后,当你的财政赤字是GDP的20\% 之后,你的国债利率压不下去,压不下去,本身就是滚雪球。因为你的赤字是必须要再去借的,刚好就是要再多发行国债去借。

与此同时另一个很大的危险就是国际产业链被迫复制,美国体系不接受中国参与的产业链。所以你必须要复制全套。不只要复制全套,不只是在Golden Billion里面复制全套,甚至在美国国内都还要复制一套。这是为什么台积电跟三星必须要到美国去。

国际产业链因为政治原因而必须要损失,这里面的效率损失同样也是万亿。谁来负担?全世界的消费者。

 

好了!这些背景讲完了(笑)。60、70年代你能够留学留在欧美或者澳洲呢是一个美梦;80、90年代留下来呢还不算太糟糕;现在的话呢我真的建议(笑)可以做这样的长期规划。OK,首先在经济上面,欧洲跟英国因为他本身的工业跟财富被美国深入的掠夺,他的社会稳定性进一步的崩溃,很快的就会有连锁反应——全球的连锁反应。

更糟糕的是在过去的两年呢中国的电动车——中国原本是世界最大的汽车进口国——两年之内超过了德国跟日本,变成世界最大的汽车出口国。两年之内他的汽车出口数量增长了超过一个数量级。为什么光讲汽车?因为汽车是人类所有制造业中产值最大、雇佣人群最多的。而且他是德、日、法三个Golden Billion外围国家的经济支柱。然后日韩跟台湾的半导体再对美国釜底抽薪一下,所有的这些外围国家经济都不看好。

与此同时,那你说美国本身吸血吸的这么多,应该很好。这相对是真的比较好,但是问题是他有很多很多新的社会现象:治安、社会对立、更糟糕的是反向种族歧视已经到了如火如荼的地步。我的儿子如果早生30年,现在绝对是哈佛的学生。但是他因为晚生了30年,一点希望都没有。因为你只要是男性的亚裔基本上就讲不去,名额全都被女权运动者跟种族示威的拿去了。

唯一的两个手段,你在纽约时报上面登出你的名字是某某示威抗议的领导人,像Greta Thunberg那样子(笑);或者你说你自己是同情同性恋。我不是开玩笑,他真的就有认识的人是这样。就是他们完全已经不在乎Meritocracy,就是选贤与能。我儿子去参加社团活动,会去竞争什么奖学金名额,他进去开始讲自己多聪明,曾经考试多少分,然后得过什么什么奖,马上就被人家骂走。为什么?分数好的都是亚洲人,黑人的分数很差。所以你谈分数就是歧视。

   

 

为什么现在美国的这些政策倒行逆施到这个地步?因为你搞这种政治正确搞了二三十年,结果就是现在到台上的那些人物都是靠这样的活动出来的。中国文革搞了10年,活动家是不是只要红不要专?美国一样,而且他搞了不止10年,搞了30年。那你想想看这个结果是什么?

所以相对的,我建议你们出去留学的话呢,就计划学成以后回台湾或者是到大陆,甚至到俄国去——因为俄国在过去这两年很成功的从欧系的经济体系转化到中式的,其中最糟糕的一年GDP下降了2\% ,你说这是不是非常成功?你从一个完整的洲际转化到另外一个洲际,只掉了2.3\% 的GDP。回到台湾来有另外一个原因,就是这些所谓的民主政府其实我认为是所谓的票选政府,他们都有我刚刚提到的一个通病,就是财政赤字。反正未来的下一代都不能投票,让他们去死吧。(群笑)那台湾也有这个问题啊,那是不像日本,日本现在的国债已经是GDP的250\% ,欧洲,南欧最糟糕的国家希腊只有180\% ,意大利只有150\% ,美国120,日本是250。说大陆的少子化很严重,日本已经少子化40年了。

所以你说,这边矮子里面拔将军呢,台湾是最好的。因为你们退休之后的年金怎么算,我相信会有一个比你大50倍的经济体来替你出。(群笑)日本跟韩国都没办法讲这句话,你的保险业乱搞金融操作亏空了几万亿,到最后都有人兜底吗?美国自己都不敢这样搞。OK?

所以今天就讲到这里,可能跟一般大众媒体上谈的不太一样,但是是肺腑之言,希望大家能够参与。



听众(老先生):

(主持:接下来就是留10分钟的时间。)谢谢王博士如期归来,我们这个世代啊,生长在台湾这么一个畸形的时空环境,在我们成长的那个年代,王博士应该是属于旅美学员,应该现在不流行这么讲,但是实际上在我的心目中就是。当年的留美学员呢虽然今年不怎么流行,但是要谢谢您来,那我们来当面跟您交流的机会。我本来准备了许多问题,但是不到一个钟头其实你已经解答了大部分我的疑惑,因为您回来的时间点都非常的敏感啊,半年前、四年前、现在,所以我在想,您胸怀天下,但是心存台湾。(王:呃,我可以回答你的问题,谢谢~)但是,您愿不愿意说比较主动的来跟郭团队有一些接触?因为时间已经到了,您如果不介入的话,可能会有很糟糕很不幸的事件。



王孟源52:10

上个月我在龙行天下的时候也是有听众发问嘛,他问我说对台湾的选举有什么看法。我没有看法。因为4年前当他们开始捧出泥井的时候我就在博客讲过不再谈论这些事情。但是呢我谈到一个自己的私事,我说我表弟在6月的时候试图要联系我,原因是我有一个大学同学在柯文哲的阵营里面当助理,那跟他聊天的时候他说他正在找分区助理,我就想到我的表弟了,他是台北人,是名律师嘛,而且非常痛恨蔡英文——因为他是蔡英文在政大的学生,那时候他是大一的学生,从那个时候他就...他说在学生时代就知道她是一个脑袋空空只会唬人的笑话。哈哈哈哈...自从学生时代恨到现在。这我没办法证实啊他对个人的看法,哈哈哈说不定是蔡英文给他一个是C。哈哈哈哈...(群笑)当教授,是他的学生故意,有可能,哈哈哈——无论如何,我忘了跟我的表弟要电话,所以我就打电话跟我妈妈讲要表弟电话,然后她说,支持柯文哲这种恶魔的人,我绝对不能帮助你,所以不给!哈哈,哎呀我真的是没有想到,这种选举变成全民运动到亲子之间的关系都变成有隔阂,我是没有想到。我觉得这种事情应该纯粹求真,探讨真相,然后是否利益最大化。一个完全客观,事实上应该是很枯燥的。像我今天讲的这些,应该是非常非常枯燥的话,完全都没有感情的事,没有一句是我自己的感情来论述,全都是客观的谈这些问题。无论如何我回来之后,我又跟我妈提这件事,我妈说好我给你,但是你要跟我发誓保证跟柯文哲没有关系。哈哈哈哈(群笑)...好她就把电话给我然后我就打电话给我表弟,然后跟我表弟谈了以后啊,他就说这件事情不要找我,因为我已经在另外一边很高兴。而且我也奉劝你们,台湾这个烂泥摊,在美国好好的过着,不要碰。我说我当然知道啊,要不然我会找你?哈哈哈哈哈...

Well,事实上,我的博客里面有好几篇里谈到,这整个选举就是用来设计,制造一个你们有选择的假象,为财团掠夺财富、欺诈社会正义。所以我永远都对参与这个过程提不起兴趣,因为它是一个圈套。



听众 56:09

王博士您好,我之前有看到您讨论说那个液流电池跟AI非常值得继续投资,那想再问问看有没有对其他领域的东西你觉得值得投资的?比如我上次有看到你有在讨论说光子晶片是值得投资的吗?



王孟源56:35

很不幸的这个美国社会文化的全面腐化被推演到全球。我到美国的时候是1988年,我去的时候还满怀理想,认为这是世界上最接近理想的国度,事实上在某些领域里也是,事实上是这样子。到了其后一年,1989年,美国的学术界出了一件大事,就是他们有所谓的冷聚变,Cold fusion。这后来证明是两个完全不知道自己在胡说八道的人自欺欺人所编出来的幻想。但是呢他们有大学教授的名衔,而且在他们的论文被发表之前他们就先通过公关手段开了一场记者会。你也许觉得学者开记者会有什么了不起?

34年前,示范的大部分,你讨论学术,没有证明的议题,没有能够确定可以写进教科书的东西,你就在记者会上跟一些小老百姓讨论,这是完全违反职业道德的事情。34年前千夫所指,到现在习以为常。现在随便哪一家大型的公关,我拿过来第一个反应:他们这一次是怎么骗人的,骗人程度是100\% 撒谎还是99\% 撒谎。(群笑)这样的。很不幸的是,其实真正有前景的科技,反而被淹没在这些吹嘘的说谎者里。如果你有20个公司说他可以做到同一个点,然后会得到——我不是随便比的哦,最新的一个数据就是八万个——那真正能够对能源供应有帮助的液流电池就把它压下来了。



(编注:1989年3月23日,英国南安普敦大学的马丁·弗莱施曼(Martin Fleischmann)和美国犹他大学的斯坦利·庞斯(Stanley Pons)举行新闻发布会,宣称他们已经实现了核聚变:使用电流将大量的氘强行注入一根钯金属棒中。在一段时间后,钯金属棒开始产生比输入能量更多的热量。在他们的一些实验中,该过剩的热量可以持续了好几天,释放出的能量净量是任何已知化学反应所能解释的能量的数百倍。得出的结论是:其来源一定是核聚变反应——在这种特殊情况下就是成对的氘核聚变形成氦。)



为此我一直强调,讨论公共事务必须专注在求真。第一个问题是真假。我不是说你在网络上不可以抒发自己的情绪,...(听不清)你喜欢参加小吃店的活动...(听不清)或者你多么多么讨厌一个人,这些都可以讨论,这些是你私下的观点,你完全可以跟大家分享。但是呢,这种态度不能够应用到公共领域上,为什么这样?因为公共议题就是公共资源的配置。公共资源的配置就是公共经济产出的决定性的因素。而我们经济学一个很基本的要素就是,人命是有代价的。中产阶级的人命等同于人均GDP的150倍,穷人的命更不值钱,你的产出每减低150人份的GDP,你就等于间接的杀人了。我认为选民——我现在回答那个问题——选民高兴选谁都可以,但是你这个选出来如果真的会影响政策的话,就必须要是完全履行的。当然不可能。99.9\%的都是YY的(笑)。

那我回到这个问题,就是储能的问题,就是有感而发啊。那我在我的博客上讨论的我认为有前景的的确是许多新的电子技术。液流电池技术很适合用来储能。我们现在采纳新能源供电最大的问题就是日光能跟风能它的不稳定性。我自己家里有太阳电池,它每天即使是晴天事实上也只能够产出6个小时;如果遇到阴雨天甚至等于0。那你从有产出到没产出,跟我们消费的巅峰又有错开,那你这个东西唯一的解决方法就是储能。

那储能有很多我刚刚提到的那些骗人的,氢能源就是完全骗人的。这也是我已经讲了八九年了,有很多很多的技术其实都是骗人的。但是液流电池不是;然后呢锂电池所衍生的好几个技术也很合适。但是呢这些你必须要静下来...CATL(宁德时代)或者URV(东风本田)你才有机会,因为都是他们的那个大公司的研究部门在做,比如说大离子电池,比如说固态锂电池、半固态锂电池。我个人对半固态锂电池看好嘛,我认为2025年我们就会看到。但是问题是你现在毕业要去加入应该已经来不及了(笑)。那至于固态锂电池我就不那么看好,我技术问题博客上也讨论过,但是你如果要加入,完全合理,因为所谓不看好,我的意思是说他成功的几率不到10\% ,我个人认为成功几率只要在万分之一以上就值得。

我说氢能源跟核聚变完全没有成功的可能,那代表他们在十万分之一以下。我个人——虽然我博客上充斥着说这个科技是骗人的,那个科技吹嘘是假的;事实上我对未来科技的成功的评价是很慷慨的,你只要有万分之一成功的可能,我就认为你应该投资。

但是呢至于你问的问题,除了我想一般的学生不是专门做化工的,AI可能是最容易上手的。因为AI是软件,你自己学都可以。而且AI除了最基本的理论,现在基本搞出来的是一些基层的engine(引擎),他的应用还有好几个层次,这些应用都可以消耗大批爱好者的时间和精力,然后有价值的产品。

所以你不要觉得这只是AI,我不是指说你必须要加入Google或百度,去做他们的AI研究;但是人家已经有一个现成的engine,你把它应用出来,取代画家、取代照相师、取代小说家、取代公关文章的写手,这些都是很简单的应用,而且都是一般大学生就可以上手。

或者你还有其他的什么方向可以在我的博客留言。





听众 01:05:06

谢谢王博士来今天演讲,我也是从美国回来,我照顾我母亲的,但是在台湾我真的就是,您最后一个议题就是台湾本来有相对优势哈,但是我在很多地方看到台湾呢其实生于忧患死于安乐。我在我们那边有一个公园,有一个防空的设施,它上面有个标志,他说这是防空地区,但是在旁边写:你不用太紧张,这也可以是一个艺术的标志。(王:哈哈哈哈)因为这个我觉得是不断的在淡化这件事情,当美国到无可施的时候,您觉得他做的动作会不会让台湾跟大陆就两败俱伤?(王:当然。)而且你知道东南沿海有多少亿的人,包括大城市全部都在上海之下,但是有没有可能台湾的人说打起来他投降就好了?你看俄乌战争不是这样子的?你想想看,如果有一个飞弹就炸到长江三峡,然后他们说是大陆炸的,现在不是每天都在演这种事情吗?



王孟源01:06:27

我认为更可能的是,台北忽然有一个核弹爆炸了,或者东京有一个核弹爆炸了。



听众401:06:32

对!而且因为他已经派那个军事人员进驻,可能按的人是一个下士而已。但是虽然他是下士,他可能是上校(王:可以做成手动。)对,他可能是上校,但是他是下士按那个按钮的,我觉得这是我最担心的。



王孟源01:06:48

我也很担心,但是我已经看开了。因为你不管票投什么都没有用。那个体制就是这样子,所以...台湾人是不是对国际政治经济事务特别的愚蠢?是的。(听:因为你看新闻全是啊。)台湾的新闻大家都知道,对。(听:包括陈文茜过去两年都是乐观的,但是现在都...)

我为什么过去两年拼命的上唐湘龙的节目?不是要教育大家,而是要教育唐湘龙。哈哈哈(群笑鼓掌)。你看现在唐湘龙还谈不谈自由民主投票?他敢不敢?OK?唐湘龙讲一句话能够碰到的人比我讲一句话碰到的人多很多。

问题是我们目前盲目采纳的这个体系,是英美经过300年的沉淀,精练出来欺骗小老百姓的一个完整的系统。你即使要看穿其中的一半,都不是普通高等教育的十分之一能够做到的。这是为什么我写博客。我写博客的话,那些工作——那些收集事实,去做深度分析的工作我帮你做。即便如此我也知道,我所能够碰到的顶多只是1\% 的人。因为整个人口就没有探讨真相做逻辑思辨的能力和习惯。而且这个不是命中产生的,而是教育改革有意造成的。

台湾为什么20年前搞教改?越笨的老百姓越好骗。这个教改的idea是哪里来的?美国嘛!美国是先进国家嘛,先进经验。对不对。

所以你说要希望民众觉悟起来,做体制内的改革,不可能。我也希望能够一夜之间醒来变成世界第一土豪,但是这种事情是幻想。我们能够做到的是什么?要求他们美国拿台湾做第二个乌克兰?我们再回想一下俄乌战争中第一个目标是谁?俄国,背后的目标是中国,对不对?他如果在东亚再重搞一套,他的直接目标是中国,背后的目标是所有第三世界国家;但是他没有把俄国搞下来之后,他顺手去搞的是什么?德国、英国、日本、韩国跟台湾。

所以美国如果在台海搞出一个战争来,他并不一定要把中国搞下来,尤其是在西太平洋他没有办法用代理人。台湾的军力太可笑了,乌克兰4000万人,搞出200万的军人是可以的;台湾能搞出多少军人?斯拉夫人能够在街上当众拉夫,拉到现在连女人、护士全都必须要服兵役去注册,这是上个月发生的事。在台湾可能吗?乌克兰的武器第一波打光了之后,这些武器是过去8年——从2014年8年北约送进去的——打光了以后北约又送了一波,然后又打光了,现在第三波,打到差不多也快光了。台湾能够打第二波吗?军火运输船能进得来吗?所以如果美国在东亚复制的话,他基本不可能打垮中国。他真正能够拿到利益的目标就是日本跟韩国。

日本、韩国这次被坑了。如果是台湾打起来,经济上被坑的最厉害的一定是日本跟韩国;但是呢在前线被打烂的是台湾。那我们的小老百姓能不能决定不被打烂?不能。唯一能够决定要不要被打烂的是赖清德,但是你不信赖,我也不信赖,OK。所以呢就不要幻想一些事情,你唯一能做到的事情就是要求他们不要打巷战。这个事情你至少站出来喊,还会有人愿意。

 



乌克兰打仗是怎么打的?巷战。为什么?对方有优势火力跟优势机械化装备的时候,你怎样拖时间,怎样消耗对方的兵力?打巷战。拿老百姓的命,拿老百姓来当肉盾。乌克兰就是这样打的。台湾老百姓如果要自保,唯一能够做到的一点,就是站起来公开跟政客讲:你如果非要在台湾打保卫战,你到滩头去给我打野战去,不要到城市里面打巷战。因为乌克兰的建筑是苏联的建筑,钢筋混凝土多放一倍这样的建筑来的,地下室有三四层。台湾是连一层都想要把它偷工减料。(群笑)



听众 01:13:57

王博士你好,就是想跟你讲一下,像我在2018年的时候去上海念书,然后就是个人实力不足,所以...因为那边学生真的竞争非常激烈。那你说的宁德时代啊之类的,我念的学校是化工为主,实际上我有注意到,但是要进宁德时代的研发的话,都是硕博。我有好多同学他们每天都是一直K书,就为了保研这样子,然后甚至现在有人要直接五年念到博士。

那我想请教就是说,那现在年轻人内卷非常严重,我那个室友有一个江西的,他现在还在考公务员,然后有一个101博士跟我说,他打算去大专当老师,对,就是说我去跟他们交流,他就说年轻人真的很累,而且...就是那边提倡高考虽然是公平,但是还是有一点点不公平,就是大城市城乡之间的两极分化。所以我那个科系很多穷苦比较偏远地区的小孩,比方说江西来的东北来的,实际上他们都是怎么样?常常会跟我透露说他们以前的生活环境非常非常难受,甚至我在听他们讲说他们在高考过程中他们的心理精神,他们很容易造成心理压力这样子...(王:你说的都是事实。)对对,所以就是怎么讲,我还注意到你提到说大陆的学术管理问题,那像在我学校也差不多是这样的,就是有为了发好文章了然后博士生跳楼啦、学生念不下去的都有。(王:我知道你要说什么。)对对对但是你说假大...对对我知道,就是他们在那边念硕博压力非常大,如果是老师要晋升教授的话,可能就是要让他们学生发文章,然后可能一个大导加几个辅导,或者几个...加好几个学生这样子。(王:问题不是出在学生,问题是出在官员。)对对,你有提到假大空的问题,但是现在好像打假似乎...因为我写本科论文的时候其实我们室友一直在强调说不能乱写,要查重抓重,但是有一些...(王:本科就不应该写论文。)对但是没办法,我们那边就是这样硬性规定。那我的意思是说现在,比方说之前有那个裴钢的事情,好像是他们学术之前那个图片误用的事情(王:是PS造图。)对对,那之后好像都没有妥善的处理。(王:不是没有妥善的处理,拼命的处理,撒这一谎。)对我见你最新的谈论是说他们的文化还是有点太功利,那你之前也有说可能就是贫富差距大的时候会有这个影响,然后...不好意思不好意思,后面我就提一个重点,提到这个,就是陈平教授他有提到这个现象,他是说其实应该学美国的,就是说在某一个细节,他是觉得说美国还是比较好,就是把资助给好的学者而不是只给学校。虽然你提到说可能连打假都还没做起的话都没有用,对,但是我不知道你对陈明教授这个解决方法(有何看法)。



王孟源01:18:30

陈教授也是物理系出身,所以有些观点类似,但是有些观点,我觉得他比较naive,因为他没有实际社会经验,他一直都是在学术界。他对人性的了解就是(笑)怎么说呢。

你的所有这些东西我的博客都反复讨论过了,答案其实都有,我建议你可以去仔细看一看。我在这里只是针对这个话题做一些简单的回馈。首先,对个人好的事情不见得是对社会好的。我在这里做公共的政策讨论,是社会利益的最大化,而不是哪个人利益的最大化。

至于大陆现在学生内卷,我认为是件好事。当初台湾经济疯狂成长的80年代你认为那些学生不够内卷吗?我是我们全班最卷的学生。学生的任务是什么?不是去开party,不是去郊游,而是学习。OK?现在的美国教育体制最大的问题就是,完全放弃了Meritocracy,就是忘记了学生的根本任务——学习能力。



听众 01:20:24

(主持:最后一位。)嗯那个七公好,还是习惯叫你七公噢,因为还是比较...(王:啊!可以,你称呼!来)习惯,呵呵谢谢,本来想问你有关于武统的事情跟AI的问题,刚好你都回复了。好,那主要是想跟你问一下,因为我记得我们几年前有讨论过,就是您可能当时有提到说,如果有能力就离开台湾,可我今天听到你说,诶!台湾反而相对有优势。那这个转变可能在哪边?那另外一个就是说,像刚刚那位发问者也有提到,就是说大陆现在的这个不管是教育的问题——其实现在他们有非常多就是在讨论,因为有很多很多的科系,当高考完以后去分发科系的时候,有很多天坑科技念了以后发现,诶,无法学以致用,造成浪费。您在博客中也有提到,然后我注意到现在在大陆里非常普遍的在讨论这件事情,那我主要担心还是在于我们台湾包括自己的小孩,那可能未来两岸如果真的统合在一起的时候要面临这样的一个变化,甚至这样高度的一个内卷的竞争,那我们现在可以为我们的孩子做些什么?



王孟源01:21:43

我的建议在过去10年有转变呢,主要是因为我小孩上了大学以后了解到他们的大学腐败到什么程度。我原本是以我30年前的那个印象来评论美国的教育系统,后来我的小孩去了一下,他去的也是名校,但是呢完全不是我想象的那么回事,所以我觉得还是回到台湾还是比较最优的选择,尤其是美国现在的社会撕裂比台湾还要严重,而且美国人是有枪的,台湾只是放嘴炮,他们是真的放实炮的。OK?留学生在美国因为犯罪或者是其他的后天因素而死亡或受伤非常的常见。单从安全观点来看,我就不希望推荐一群年轻的学子去那边留学。

然后,有关大陆教育改革,那是我目前最痛心疾首的一件事情。因为他在过去四年,招生增长了40\% ,结果他们的今年失业率从8\% 提升到20\% 。你说这之间有没有因果关系呢?只有大陆教育部才会否认。那在这方面台湾当然也是先行者了,台湾先列步与进。



我认为这里面有一个基本的原则,就是你要做出任何一个社会公益的优化的决定呢,第一个前提是要诚实。你如果不诚实的话,永远永远不可能达到正确的答案。这里的所谓的诚实是什么?有些话很难听,但是它是实话,不是每一个人都适合接受到的。

不是每一个人都能够坐办公桌——是不是实话?百分之百的实话。但是在美式的选票政治之下你敢不敢讲?你不敢讲。你不敢讲,做出来的政策选择就是错误的。错误的结果就是成千上万的学子生涯被浪费掉,国家资源被浪费掉,国家的经济没办法继续专注在制造业上。

我希望...(最后一句听不清)



听众 01:24:57

王博士,你好,我跟刚刚那位忧心的家长很接近,就是说我们举一个实际例子,您刚刚特别有提到制造业就是做实事的产业,那我想到一个,我们曾经听过21世纪跟生物基因工程这个是很有关系的,那其中一个可能是一个比较小的环节,我直接举这个例子,这个科技领域叫做pharmaceutics science,简单讲应该是属于研究药物的科学,那这个应该是实际吧,做出药物然后研究对人体有帮助的。那我觉得这就是实际的产业,这不是假大空的东西,就是比方说新冠肺炎你真的能够抑制它。那在这样的一个前提下,如果说随便讲,我如果到美国或者到加拿大去,因为现在美加他们的制药——我不晓得加拿大,反正就是美加这种先进国家先去——就像你讲的出去留学,然后比方说念到他们的博士,那念完以后下一步比方说就是这个pharmaceutics science会不会在东亚,或者直接讲,像您说回台湾或者是到大陆,会不会有比较长足的发展?就是直接问这个pharmaceutics science。



王孟源01:26:07

生医方面的确是台湾慢慢的发展起来的。连印度都可以发展起来的东西,他都没有理由发展不出来。但是呢大学扩招的时候是扩招生医的吗?好像不是的,扩招的那些学生好像不学这方面。所以重点就是你资源放在错的地方,对的地方的资源自然就会受损。为不管你拿到多少选票,你的资源的总值永远都不会改变。



主持 01:26:49

今天谢谢大家的参与,那因为时间有限,待会我们要带王博士用餐,那明天三点半也是在这个地方,那如果说你们还有问题的话,其实我们可以把问题思考一下,然后明天再过来问。好,今天非常感谢大家的参与。









\twocolumn[\begin{@twocolumnfalse}
\section{、開箱王孟源:金磚峰會對全球金融的衝擊!}
\subsection{20230915}
\end{@twocolumnfalse}]2023.9.17 校正完成。



唐湘龙 00:01 

okay,okay,好。一个学物理的,学到后来都在金融圈子里面。不过因为你退休早了。我觉得你看你现在……



王孟源 00:10 

我四十几岁就退休。因为其实昨天饭局的时候我有解释,王强问我,我解释了,那你来的晚……就是我的专长,事实上我是全自动程序交易的发明人。



王孟源 00:29 

但是到了 2003 年的时候,因为我把那些传统的证券交易商的口袋都掏空了,Merrill Lynch跟 Goldman Sachs非常的不高兴。嗯,OK,所以他们就跑到 SEC 说要把我们禁掉,说我们这是违法,反正他们损失的一定是违法的东西。



唐湘龙 00:48 

那……如果,那你应该赚很多啊。



王孟源 00:53 

对,那时候十人的小组几亿几亿地赚了,这样子风光了几年。



唐湘龙 01:00 

这个我要改天再听你讲。这一点很好听啊。对,这个都可以写小说了。



欢迎来到龙行天下,我是唐湘龙,每个星期五的早上的 9 点半钟到 10 点半钟一个小时,龙行天下的单元,在这个单元里面会出现的来宾。



唐湘龙 01:30 

我想我们的观众朋友大概都知道有很多人说我花很多的一些心思去邀请。那原来可能也不见得认识哈,但是我觉得他们很棒,就是我主观上面了,我听到接触到的时候,我觉得,噢,那个会让我眼睛一亮,而且我自己都会觉得这个是这很强的。即使我在做节目的时候我都觉得一小时下来对我自己都是收获满满的,我就努力要他们来当来宾。好,那今天,今天的这个标题,你看到今天的标题是开箱,因为最近的关键字是开箱,到处都要开箱,拿到华为手机就开箱,大家都很兴奋。但是我们今天开箱旁边的这位王孟源跟大家打招呼。



王孟源 02:13 

大家好,很高兴能够来到现场跟大家直接聊天。



唐湘龙 02:16 

我跟你讲,你要能够看到,而且能摸到本尊,那很难,哈哈哈,我也是第一次,我昨天晚上才跟他吃饭。好,虽然之前我邀王孟源,我说了纯粹是因为我在看节目,无疑当中呢,我在看看这个八方论坛的时候,我在看八方论坛的时候,我就那集访问王孟源我就听了。之后我曾经我很入神,我很专注的听,我把那一整集看完了之后我就很好奇说怎么会有这样的人我不认识呢?因为我自己在台湾生活工作了这么这么久,在台湾只要是跟台湾有关的有点字号的人物,我就算没访过也知道王孟源,我真的不知道。好,但是因此我就很主动了,我再多看几集之后,我觉得我要认识王孟源,所以我就透过管道。那这个人很就是很难搞,我说他很难搞这个,反正他平常他强调他基本上也不用手机,什么都跟他联络,所以你要找他的时候还要用最古典的 Email 跟他慢慢发。所以我们每次两个人的三言两语都是用 Email 给他发来发去的。



唐湘龙 03:24 

好,那因为孟源是台湾小孩,台南人,当然之后受在台湾受很好的教育,他非常的聪明,我听他很多的跟他一起念过书的朋友,那在形容王孟源。都吹,都在跟我吹王孟源有多厉害。好,后面来慢慢,我了解他之后,我觉得这个人确实这个怪咖是非常厉害的。好,那以后有机会再慢慢的去开箱了。



唐湘龙 03:48 

王孟源其他的就是对我来讲都很惊讶的一些的,他的过去的人生经历。那因为他这次回到台湾,可能在台湾会待一段时间,然后也可能会到大陆去走一走,那这个对于对这两个两三个月做节目来讲有很大的方面。王孟源回来了之后,除了演讲之外,他就主动说终于可以拿龙行天下上现场的节目。那昨天晚上我们一桌人一起吃饭,包括陈凤馨,大家聊得很开心。那孟源这两天今天下午还有一场的演讲,在台北市立大学的天母校区。好,那昨天有些听众朋友听到了,那可能你有去现场的,今天下午的 3 点半钟,那在天母的这个台北市立大学的校区,仍然有王孟源的一场的演讲。好,那今天孟源来到节目的现场,上个礼拜我在跟孟源,孟源沟通的时候,因为我当知道他要回来了,回来孟源说,好。那既然今天要开箱让大家看到王孟源的本尊,不用看看两个框,那我们的设定又有,每次王孟源节目结束之后会有很多人,我想你有看留言就知道,很多人就留一堆的问题就在问我们,问王孟源那个,那都不是我能够回答,或者我也不想去转发,因为那个用又写的太麻烦。



唐湘龙 05:05 

所以王孟源说,好,那今天半小时的时间,王孟源把最近,尤其是金砖峰会之后的全球的金融货币体系,因为现在全球的经济都在一个有点不确定的状态里面,到底会发生什么影响?后半段的部分,半个小时的时候王孟源说就开放,那我们的听众朋友、观众朋友们,但不可能每个都都提问了,我们都能够面面俱到,但是反正又先后顺序,大家那个尽量提问,孟源有时间就尽量回答。



唐湘龙 05:36 

好,今天我们就从金砖开始好了,我看金砖的时候,其实我今天早上看看油价,那以美元现在这么强势的情况,下面以美元计价的国际的油价,北海布兰特原油今天早上来到每桶 94 美元,然后纽约轻原油已经来到九十美元,上个礼拜我说他一定会上90,他上来了。那过去 24 小时里面,比尔布兰特原油,纽约金元又大涨两美元,我说美元已经这么强势了,以美元计价的油价都还这么高,换来说,如果美元是弱势的,那你可以想见呢?这油价早破百了。好,那这么强势的油价,当然大家归咎原因就是全球经济好像起来起起落落也不怎么好,但是你看到了俄罗斯跟沙特联手减产,沙特跟俄罗斯联手减产对我来讲是一件大事情,这个大事情的含义就是说沙特已经进精砖了,俄罗斯也在进金砖里头。沙特跟俄罗斯的减产,它会慢慢的越来越形塑成一个金砖体系里面的南方国家,掌控资源跟人口优势的南方国家,开始透过他所掌握的优势,在对于没有人口跟资源优势的西方的经济强卷开始有发号司令,甚至有去影响,就是说他们的原来既定计划跟发展轨迹。所以俄罗斯跟沙特的联手已经成为一个非常成型的就是国际能源市场的操作模式。



唐湘龙 07:22 

但是因为现在都在金砖体系里面了,沙特是很欢欣鼓舞的进金砖的,看起来未来的国际体系,金砖未来十一国跟了 G7之间的这样的一个对抗性的一个态势,会不会变得越来越清楚。另外今年的金砖虽然扩容了,也谈到了就是说要有一个所谓的金砖体系的稳定性的货币概念的出现,这个因为习近平在金砖演讲的时候都提到了,虽然没有具体的成型,但是八字有一撇。好,我问孟源,就今年的金砖峰会你怎么看?



王孟源 08:03 

我认为是一个很大的失败,而且责任完全在中国的金融经济组官僚,专业官僚。先回到你刚刚谈的。的确现在这个油价的上涨代表着 1970 年代的OPEC的重演,不过这次是OPEC+,尤其是沙乌地跟俄国的联盟,事实上他们以这个联盟还有含有中国特色,就是低调。就是中国的传统就是明明是同盟,但是你不说它是同盟,那所以它的确是体现了金砖的特色。



王孟源 08:46 

那我先回溯一下历史,谈一些国际政治的惯例,然后可以跟现在的叫。这刚好是我昨天在北师大演讲的时候提到的议题,所以今天用时间比较有限,我用几个简单的图画,我从来没有绘画天才,所以这完全是释义的,没有艺术性,哈哈哈哈哈哈,OK,OK,我们现在这个霸权争夺其实是人类进入工业化时代之后的第二次霸权转移。那 100 年前也有一个霸权转移,那当时的霸主老霸主是英国,但是现在的霸主当然是美国。但是我们先看看 100 年前的霸权转移过程,基本上是 30 年,从一战一直到二战后。







王孟源 09:44 

OK,一战前到二战后,在这个过程中,现在很少人有人记得在 19 世纪末跟 20 世纪初也是一个全球化的高峰。OK,跟 100 年后我们刚刚过去的那个全球化高峰非常的类似,然后等到那个霸主理解到这一个全球化产业链的共享跟分散对它本身造成威胁,因为它帮助新兴国家来来挑战他的时候,两个霸主,当年的霸主跟现在的霸主同样的都选择去全球化来做隔离。OK,我们来看看一看这个。我现在画的这个是当时第一次霸权交替的时候,就霸主跟挑战者心目中的情况。上面这个是现实的状况,也是英国愿意保留的状况。最左边是美国,因为它有门罗主义,所以它控制了美洲,但是全世界大部分殖民地都在英国受理,事实上也有一部分在法国受理,那么他们基本上是同盟,我们就算是一块。



*



(博客后续补充了上半部分,这里列出,依然作为图1的一部分)

*



王孟源 11:00 

OK,我画成这三角形,代表它是一个金字塔,就是他是殖民者跟被殖民者之间的关系,(一个支配体系)那德国跟日本因为是新兴,所以没有他们自己的殖民地。嗯,那英国认为我们就应该维持这个样的这样的秩序。那德国跟日本的心目中的理想就是这样子(编注:指图下半部分),美国一块,英国一块,德国一块,日本一块。



王孟源 11:26 

后来那个希特勒一直都是非常崇拜英国,原因就在于英国是他理想中的殖民帝国,最成功的殖民帝国。所以纳粹德国其实是一个,没有成功建立的知名帝国。那他因为来不及去亚洲跟非洲分一杯羹,所以他就把眼光放到东欧去了,因为德国基本上是一个陆全国家,那他就要把斯拉夫人当作北美的印第安人,还有印度那些人那样来残杀跟压迫。



王孟源 12:06 

所以你可以说 100 多年前的, 100 年前的那一场斗争是殖民主义帝国跟殖民主义帝国Want-to-be,英文说 Want-to-be就是梦想着当做殖民帝国的人之间的一个斗争。那我们来看看我们现在面临的这个斗争。



*



(博客后续补充了图二的左上,这里列出,依然作为图2的一部分)



*



这个是现状(编注:指图二左上),就是美国已经成功的整合了整个世界,你必须要承认他们的这一套殖民帝国的新殖民帝国的手段比英国跟法国要成熟,而且老练了很多,用间接的手段。美国 200 多年打了 400 多场侵略战争,但是从来没有单独主动去挑战一个一流强权。他们永远都是站在后面让别人先去送死。然后试图渔翁得利,然后他军事事实上只是它相对较小的一部分,它的霸权的三大要素是美军、美元跟美宣。美宣是思想上控制,所以它这个是最高级的,然后利用美元也是也是兼控制兼搜刮。然后美军只是最后的吓阻的手段。那他很成功的整合了这个,而且事实上中国原本也是愿意的。你可以看到中国一直韬光养晦,他们的韬光养晦的战略,从 80 年代他们的国际战略就是我愿意,而且我愿意站在这里(编注:指图二左上,上半部分小三角的最底层)。你把我当做这个领导阶层的最底层,我们愿意。但是这个问题是中国的体量太大了,而且它能够提供所有第三世界(的工业需求)……你要知道这上面这个,这个小三角就是统治阶级的,这小三角也分四层,最上面是美国。嗯,第二个是五眼的其他的盎萨国家,然后第三个是欧盟,嗯,然后那其他的白种人,最下面才是黄种的三个东亚(日本、台湾,南韩)中国原本是希望说加入这个台湾的这个行列,但是因为中国的体量太大了,而且它在经济上。能够提供你这些殖民帝国(的工业)。一旦进入像美国的这种间接手段了,最怕的就是第三世界被殖民的那 70 亿人能够找到替代性的技术、资金跟人才。嗯,来为你做发展,那被殖民的国家是没有权力发展工业的。



王孟源 15:07 

嗯,OK,那所以中国一旦它的工业层次上升到接近韩国跟台湾的时候,匹夫无罪怀璧其罪,美国不可能容许他再进一步,OK,因为它体量太大了,而且它能够廉价的替代这些先进国家用来统治世界的经济跟技术根基,那所以就必须割裂。这个割裂不是目标,不是这个(编注:这里是图二的右上部分),这是我们目前的现实。这个 GB 是Golden Billion的意思就是黄金的十亿人,也就是这……



唐湘龙 15:50 

先进国家工业。



王孟源 15:52 

被美国整合的工业先进国家,这个割裂的意思是打击中俄,把他们打下去,让他们回归第三世界被殖民地的地位。然后这些归顺了以后,打服了以后,中俄已经不再能够站起来的时候,再连接起来,回归这个状态(编注:这里是图二的左上部分),这个是依旧是美国的理想状态,那中国现在的理想状态就是基本上是习近平上台以后才有的新战略,就是希望全世界都坐在一个圆桌上面,大家平等的讨论。那这当然是梦想,因为你对从金字塔下面的人来说这是一个梦想,但是对金字塔顶尖的国家就是一个噩梦,他们是不可能容许。



唐湘龙 16:44 

好,我承认你画画图的本事真的是不怎么样,不过,哈哈哈,但是这个图它充分的说明了王孟源的世界观,这个图可以很精准的去理解王孟源的世界观。好,那你刚刚,你看你刚一开始讲的就是说你觉得这次的金砖峰会是个失败的金砖峰会,而且中国要负很大的责任。为什么是这样的?



王孟源 17:07 

那个,金砖的成员原本就不是志同道合的朋友,他们唯一的共通点是他们没有被包括在先进国家。嗯,美国的核心。嗯,里面,我打个比喻好了。比喻当然不能够用来推理,但是适合用来教学,因为容易解释。你想象中国是一个数字,一个整数120,OK,俄国是72,沙特是48。



唐湘龙 17:46 

什么意思?这句子代表什么?



王孟源 17:48 

你们在讨论的,他们在讨论合作的时候就是要找最大公约数。OK,那中国的最大,所以它所含的质数最多,质数约数最多,所以你最可能找到公约数。所以你跟俄国,跟沙特在一起的时候,你可以找到 24 这个最大公约数,那你这个国际合作的时候就可以做到24。OK,但是印度看起来也很大,但是印度是103,它本身就是一个质数,你只要它有它在里面的话,你最大公约数就是1。嗯,OK,也就是能够开场会吃吃喝喝。



唐湘龙 18:36 

你这样比喻我竟然听懂了,哈哈哈,而且这不错。哈哈,就是我,我突然觉得这样的比喻是很容易懂的。对,虽然是很数学,但很容易懂。



王孟源 18:52 

所以这是……大家不要拿这个来推理。这就是结论,我只是为了要解释,我要这是方便很快的把一个道理解释给大家。那,即使巴西跟阿根廷也是一样的,这些都是财政赤字非常严重,而且内部对社会非常不稳定的国家。对,那而且他们也没有既没有资本,也没有意愿在国际上尽义务。OK,有资本,有意愿,尽一些义务的只有中俄,还有Saudi有一部分的程度。所以你像是Brics,原本就是。嗯,就是随便一个美国金融业的公关专家这样随手捏来的。就是基本上只是因为开发中国家最大的几个,他们并没有什么共通性。你要说这样的集团偶然形成的集团能够有什么凝聚力,这是痴人说梦。原本它就只是一个吃饭作秀的平台。



王孟源 20:17 

嗯,OK。那,真正有实质的政策必须是要有,最大公约数的国家在一起,把它扭出来,然后做领头的领头羊,然后其他的像 Brics 就适合用来做推广的平台,就是我们有一个核心的新的国际储备货币可以替代美元,你们爱用不用,但是我鼓励你们用。OK,哈哈。你们都有资格用,但是你们有没有资格来建立他,管理他?没有。嗯,因为你们没有最大公约数,因为你们都不想要尽国际义务,对不对?你们没办法,连最基本的财政赤字都没办法消减掉。嗯,所以,我刚刚说过美国的霸权三要素是美军、美元跟美宣,对不对?你跟美宣当然是只能够针锋相对的辩驳,跟美军,你整军备伍地准备要开战,对它作出吓阻的作用。但是对美元,其实因为是我的专业领域,我从写博客开始, 9 年前开始就写了一连串二十几篇文章详细的分析。那这里面有一些次要的,比如说它这个美元并不是一个单干的力量,而是有很多辅助的国际性组织。事实上美国的刚刚那个体系,那个金字塔体系,它的那个支柱有很多国际性的,比如说军事上就有北约,还有跟日本跟韩国的同盟,在政治上就有G7,还有其他一些类似的组织了。然后在法律上有国际法庭,什么,这其实都是他们的附属组织。那在经济金融上就有 IMF 跟World Bank,对不对?那在金融方面最重要的是IMF,因为事实上 IMF 的名字虽然是叫一个基金,它是实质上的国际破产法庭,OK,那国际上并没有真正的破产法,但是一个国家如果要在国际上发行国债的话,尤其你的那个货币,国际货币是美元,而不是这个国家。



王孟源 22:48 

像阿根廷当年的货币的话,那他一定有破产的危险,那破产的时候谁来仲裁?就是IMF,所以 IIMF 的意义就在于它是当你被美元压榨到破产的时候,谁来收拾善后?嗯,那刚好又是美国自己来收拾善后。那你可以想想看这种利益冲突有多么的厉害。所以我也是在 2014 年就开始写 IMF 必须要替代,结果那个中国的那些金融专家真的是一言难尽,他们好不容易建了一个亚洲开发银行,结果用它来替代World Bank,而不是IMF。我那个时候就开始摇头。我不会说凭空的说这些官员肉食者鄙,我通常都是先假设他们是称职的,但是他们如果在重点的政策上面选择明显错误的方向的话,那我就只能够根据证据来说话。那从那个,从那一次亚洲开发银行搞错方向,我就开始怀疑。



唐湘龙 23:57 

你说是亚投行



王孟源 23:59 

亚投银,亚州投资银行。



唐湘龙 24:01 

对,他不是亚洲开发银行。



王孟源 24:04 

亚洲开发银行的是日本。



唐湘龙 24:05 

对,亚洲开发。亚投行就是配合一带一路所推出来的亚投行。



王孟源 24:09 

他的任务。嗯,特指您如果去看的话是针对着世界银行,也就是扶贫的,那这是错误的,因为扶贫是假象,是宣传的一部分。



唐湘龙 24:22 

好,那他现在在金砖体系里面有金砖银行,那金砖银行可以扮演的角色吗?



王孟源 24:29 

正确的方向是这样,但是他们好像还没有专注,就是没有真正的专业的理解。嗯,你必须要有正确专业的理解,才能够集中力量在关键的点施加足够力量。因为你是一挑战者,既有既得利益者一定有(优势),整个国际体系都是为他们自己建立的,也就是你要拆一个房子的话,你必须要去找承力结构,你不能够只把那个装潢的东西,咱们你浪费力气把他的那个装潢都打烂了,那个房子还是站着的,没有什么意思。



王孟源 25:07 

那另外一个就是SWIFT,这个那个是他的结算,这个当然必须要尽早解决,因为它是美元霸权体现的一个重要工具。其实,从专业的观点来看,用对使用美元来做控制来做长臂管辖,并不需要用到SWIFT, SWIFT基本是一个 欧系的系统。OK,它的意义在于你用美元去制裁的时候,其他国家会自然的用第二国际储备货币,也就是欧元,来规避它,而使用欧元就避不开SWIFT。所以说的制裁最大的意义是在防止你用欧元来走漏洞。嗯,因为美元的制裁有他自己的体系。



王孟源 26:00 

嗯,OK。那我讲这个是 IMF 跟SWIFT的替代,都有意义,但是他们基本上还是治标而不是资本。治本的话你就必须要把美元替代掉。要替代美元的话,人民币非常的不合适。有好几个理由。第一个,你如果把你自己的国家的货币当做国际储备货币,你就必须要发行大额的国债。没错,要不然其他的国,其他的国家没办法投资,也没办法储备。也就是说他鼓励你财政赤字,大幅的财政赤字,美国这个大幅财政赤字就是有这个力量的推动。那中国要不要这样?不应该。



王孟源 26:54 

嗯,OK。第二个是它会让你的货币自然坚挺,你不管多发行多少,你的货币的价值永远都是相对的高的,因为大家都抢着要储存,对不对?嗯,你不管印了多少,其他的中央银行和其他的企业都还是要用这个货币。嗯,所以,表面上好像是减缓你的通胀压力,把通胀压力给全世界分享,但是它也造成你国内的工资特别昂贵,不合理的昂贵,那不合理的昂贵之后就会让你的企业想要外移。



王孟源 27:39 

OK,美国的,美国的基业就是从 80 年代开始外溢的,然后就去工业化,所以你如果拿人民币当国际储备户货币的话,也会面临去工业化的压力,这是非常不好的。你并不希望把所有的低端跟终端的企业都转移到印尼或者是越南,或者甚至是印度去。OK,这对中国是长期的饮鸩止渴,就好像 40 年前对美国是饮鸩止渴。



唐湘龙 28:12 

所以中国现在是制造一个大国,全世界最大的制造业国家。所以他如果要维持一个制造业大国,那他有很多的思考,他有把德国的发展模式当竞争导师,那如果要作为制造业大国,他就不可能像美国那样子做国际金融的操作,把人民国际化。



王孟源 28:32 

原因是,你同样的资本投在制造业,它的毛利率应该只有金融投在金融业的 1/ 10。那你想想看,资本家如果有个选择的话,他们会选择哪一个?对不对?绝对是金融业。你如果你的货币是国际储备货币的话,嗯,用金融手段去赚钱的方法太多了。OK,太多太多了,因为全世界都要用你的货币,那要用你的货币就必须走你的机构和管道,那你上下其手的机会太多,那这样一来你的……不只是资本,而且人才。像现在美国训练出来的最好的数学家跟物理学家,全都是进了银行去了吗?我自己也是,哈哈哈哈。



唐湘龙 29:20 

好,这下次我们再谈这段。那个,那次我对王孟源也很好奇的。好,不过你刚刚讲的这个呢?我,因为以前我在听你讲金砖的时候。嗯,我还不太能够很精确的去理解我王孟源在想什么,不过你刚才那样讲我大概听懂了。好,但是……



王孟源 29:40 

我还没有讲完。就人民币还有另外一个理由不能够做国际储备,因为你做国际储备,货币就必须要自由流通。但是美国已经有七十几年拿美元作为它的霸权基础,他用金融手段打赢对手的那个方法非常的熟练,你如果自由流通的话,根本就是引狼入室,鼓励他来打乱你自己国内的金融秩序。可是,尤其是中国目前的金融主管官员有很多是完全被西方教育所洗脑,那西方的经济学跟金融学其实都是依靠着美元霸权,他们所做的理论都是为了国际资本而好,刚好国际资本就是美国的,然后都是必须假设你有美元霸权,而且这个理论就是去持续的加强美元霸权,这也是为了美国好。



王孟源 30:52 

那但是你是中国人,你是俄国人,你是第三世界的人,你去学这套来维护美元,然后把自己的国家的利益输送给国际资本,这真的是好事吗?不是嘛,对不对?那很不幸的,中国的金融主管官员基本上都是这一套,也就是 80 年代苏联自己把自己搞垮。嗯,日本把自己搞垮,同样的就是因为被美国洗脑了,因为他们日本还有点比较可怜,因为是他没有选择美国在日本驻军,所以他们没有选择。苏联是完全就是自己傻乎乎的听信美国人的宣传,然后把自己的国家给拆了,以垃圾的价钱,乐色的价钱卖给美国。那现在的中国的金融官员也是这一套,他们牺牲国家的利益来维护美元的利益,那而且还想想方设法的要找漏洞把国际资本引进来,比如说国际的美国的JP摩根,或者是摩根斯丹利这些,他们想尽办法要让他们来中国内部搜刮,那最后的防火墙就是人民币的不可交易性、不可自由交易性。



王孟源 32:15 

OK,这个是基本上中国的金融秩序到现在还算稳定的最后一道防线。你如果被中国的这些卖国贼官员蒙蔽最高层而说为了要让假装说我要把人民币弄成国际储备货币,而把它做成自由开放的话,我觉得这是一个很大祸端的开始,就是引狼入室的的一个手段。



唐湘龙 32:49 

好,那你本来认为今年的金砖峰会应该做什么?应该完成什么?而你觉得中国竟然没有做。



王孟源 32:57 

OK,这个美元替代美元这件事情我从写博客写到现在,但是一直到一年半前就是俄乌战争开始,我才第一时间就立刻说时机到了。为什么说时机到了?因为以前只有中国急着要做这件事情,其他的国家都没有那个体量,没有那个默契来一起合作做这件事。那你一个中国单干,结果那一乎无应的话,这个很尴尬。而且对中国人中国的政治精英来说,他很看重面子,如果有这种危险的话,他们不可能去干。



王孟源 33:44 

但是,但是俄乌战争一开打,你很明显的就是俄国人已经领头的在反殖民革命了,嗯,对不对?所以俄国一定会加入,然后接下来Saudi又加入,因为我在他们那个开战的第一个礼拜我就写了一篇文章,说时间到了。嗯,你现在的最大的重点就是赶快找俄国跟Saudi搞一个新的国际储备户的货币,那因为是国际合作,然后你又不想落入欧元那种陷阱,你看到欧元搞的也是乌烟瘴气的,不同的经济体硬要和用一个货币,而且你这个融合一个货币的话,你并没有解决我刚刚所讲的那些问题。



王孟源 34:31 

你如果中俄沙合作新发一个国际人民货币的话,国际人民币的话,你一样有我刚刚讲的那些缺点,对不对?一样必须要自由流通,一样必须要发行大量的国债,一样要有去工业化的压力,对不对?一样会让你的经济金融化。所以最好的办法就是大家各自用自己的货币,中国继续搞他的人民币,继续搞它不能自由流通的人民币,然后在国际上你创立一个篮子性的货币,就是用人民币跟卢布跟Saudi的货币合成的一个篮子货币。



唐湘龙 35:17 

就金砖币的概念。



王孟源 35:19 

这个时候我当时因为我不想让印度插手,就是我刚刚讲的那个公约数的考虑,所以我并没有提到金砖组织,我说这个就直接叫亚元好了,因为三个国家都可以算是亚洲国家,OK?然后你也可以趁机刺一刺日本跟韩国,因为上一次提到亚元原本是中日韩。



唐湘龙 35:41 

没错,那已经是上个世纪的时候。



王孟源 35:43 

上个世纪的时候,那结果两个月还是三个月之后二国的 Russia today 就有一篇文章说Nabiullina在内部建议用,完全就我谈的那个方案完全一样,就是她没有说要叫亚元,但是除了名字之外,实质上就是中国跟二国,然后跟其他没有赤字的国家搞一个篮子货币。



王孟源 36:15 

Nabiullina我讲过很多次,是目前全世界最优秀的中央银行行长。所以她出来……我王孟源讲什么没有用,我不管我讲的是 100\% 的正道,但是那些在高位的人不会管你的是对错是非的,他们讲的是你的名气跟地位,Nabiullina的地位就很高了,对不对?哈哈哈,她也背书这个方案我就很高兴,有两方面,一方面是她的背书证明说我的看法是对的,第二个是她的背书就代表这个正确方案有了分量,因为有了一个有身份地位的人进来推。



王孟源 36:55 

然后我就想中国的那些金融官员,你就算是当美国狗吧,你如果有一个俄国人在旁边盯着,你也不敢偷吃自己厨房里面的东西,是不是?结果他们还真的就是。那就莫名其妙的这个搞下去以后就变成不是篮子货币了,而是要搞一个金砖货币了。那这个金砖货币有两个很大的问题,第一个它不是篮子货币,而是一个新货币。嗯,OK,那第二个是它有印度跟巴西在里面,那这个完全不可能搞成的,所以会从我原本讲的中二和为核心的篮子货币搞成这个金砖货币。这个大污烂唯一会去推动的就是中国的金融主管。



王孟源 37:55 

OK,因为俄国已经讲明了他们知道正确的方向在哪里,对不对?那这个中国这个责任最大的就是中央银行行长,也就是我推荐进入忠犬祠,美联储忠犬祠的那个易纲,他下台了,那他他对中国的损害,直接的损害是在外汇,因为投资长期美债损失了几千亿美元,但是真正最大的损害长期损害价值有两个。



王孟源 38:33 

第一个是它在美国通货膨胀最严重的,也就是两年前他它压着人民币不让人民币升值,从而缓解了美国的通胀危机,也就让美国很平顺的度过了这场危机,能够继续的打压中国这个战略损失至少是几万亿美元的价值。然后他又从中作梗,把Nabiullina已经计划好的新国际储备货币给搞黄了,这个就是间接的搞黄,就是我不说我反对,但是我不让你走对的路线,嗯,不是走中俄篮子货币,而是去搞金砖,而是去搞金融货币,就这样子把它搞黄了。那这个呢?对中国,其实不止中国,对整个第三世界甚至人类了——因为事实上你现在欧洲还有韩国跟台湾,不也是被美国炸的油尽灯枯,对不对?——你对人类所造的造成的损害是以几十万亿美元来计算,就是整个人类的一年的 GDP 都被(损失),要分散在未来的一百年。就是我们整个人类要记着易纲的贡献,就是他把这件事搞黄了。嗯,那搞黄了的话,美元就可以很安稳的至少在再做两三年。



王孟源 40:11 

我为什么说两三年?因为,他们从 2008 年量化宽松到现在印了 8 万多亿,目前还有一些残余在系统里面,所谓的系统就是他们的银行、银子、银行、企业跟家庭,所谓的Household。那这个呢?这些残余的现金,所谓的流动性现金,大概还要一年半到两年才会用完,那用完之后,并不是说美国就会有下一种危机。而是美联储不能够持续目前的政策,它目前的政策是稳定的、缓慢的做量化紧缩,那这个量化紧缩到两年后就做,一定做不下去。他到时候他是作为中和,就是停止紧缩,但是也不在宽松,或者必须要转过头来,像 2019 年那样子,转过头来马上做宽松了,要看它的财政赤字搞到什么地步。



王孟源 41:26 

那事实上美国的财政赤字现在刚刚开始进入指数升上级,就是两三年前他们的两三年前,三四年前他们的赤字达到1万亿美元,那个时候大家还吓了一跳,说这个是很可怕的,因为你看看他的一年的GDP,也就是二十几万亿嘛,对不对?那你的赤字,这是GDP,不是他们的税收,对,OK,哈哈哈,那你的赤字搞到1万亿不是很可怕,可是今年的他的那个赤字要到2万亿啊。嗯,OK。你说这是不是开始指数成长?没错,OK。你现在不但在上滚债,而且你的长期国债的利率从 2 年前的两三年前的2\%,本身现在已经到达 4. 5\%了。



唐湘龙 42:17 

好。那两个问题。一个,你刚刚讲就是说,嗯,虽然习近平在金砖的讲话它基本上是原则性的了。嗯,就是一个就鼓励扩容,然后,然后寻求一个一个比较稳定的一个金融商公司的商品。那既然没有走你所谓的亚元式的一篮子货币,嗯,那金砖币呢?现在隐隐约约,但是你这个金砖币是个很糟的选择。那如果单纯就是说推动本币交易,它能够绕过去吗?



王孟源 42:54 

绕不过去,我刚刚已经解释了。嗯,你就算做到了也是弊大于利。OK,而事实上是做不到的。因为你一个货币在本国内通行有很多的用处,但是在国际上有三个基本的用处,第一个是国际贸易,第二个是国际大宗货品定价,第三个是国际储备。嗯,OK,这里面最有价值的是储备,但是最难改变的是定价。嗯,OK,现在人工。



唐湘龙 43:26 

美元定价。对。



王孟源 43:28 

现在,人民币能够做到的就是替换交易,这个是最表层最没有意义的东西。那你说持续现在的这个政策,你要做到能够以定价还有储备通通替换美元了,这个是以 Decades,就是几十年来算的,你现在这个霸权斗争这么厉害,你源源不断的让美国从美元,用美元在全世界吸血来打击第三世界,同时集中力量来对付中国。嗯,你说这有多不智?这是为什么?我说这个易纲所造成的损失,或者是易纲和他那一伙的金融官员所造成的损失,是以几十万亿美元来记,嗯,就是这样算。



唐湘龙 44:19 

好,我再问一个问题,其他就开放我们的观众朋友的提问。嗯,就是你刚提到印度。西方国家包括美国,当今年我说是印度元年,他们在吹捧印度,热捧印度。但是印度现在又在金砖体系里面,那普遍大家都觉得印度在金砖里面看起来什么都搞不成。嗯,你怎么看?印度想干什么?



王孟源 44:46 

我想我上个月也讲过了印度,印度心中的理想也是这个样子(编注,指图2的左上角),只不过是不是美国,而是印度在这里。嗯,他的目标是这个样,那你说这样的人你怎么跟他合作?那所以这其实也是我刚刚讲过的一个很基本的原因,就是Brics 只能够当做一个打屁的平台。你核心政策做好了,核心机构定好了,用Brics 来推广,OK,但是它不是一个政策制定跟讨论的地方,而是一个政策推广。我事情都已经帮你安排好了,你们大家利用这个平台来分发一下文件,自己看一看,嗯,就这样子。



唐湘龙 45:35 

当然这接下去,这看起来,因为今年的金砖,虽然习近平到了普丁没有到,在表面上达成了扩容,但是在……



王孟源 45:46 

习近平去了吗?



唐湘龙 45:48 

金砖去了,他金砖去了,他去了南非,他只是第一场的会议的时候他请王文涛帮忙做发言。



王孟源 45:55 

对对对,他是没有去 G20,对对对。搞混了。



唐湘龙 45:58 

G20 他不去,那 G2G 就是看 11 月他会不会去APEC,嗯,就是 APEC如果去,虽然我个人认为习近平在今年底之前跟拜登还是会见面,但是不见得是在 APEC 这个场合。好,来,我们接下去,因为孟源难得回来,那王孟源说今天我不能说让他问到饱,因为时间还是有限的。然后那个把滑鼠给我一下好吗?让我把……我们先看Donate的留言的部分,先把超级留言交给我。对好对好,今天难得大家看到王孟源在现场,你。



王孟源 46:50 

帮我把问题过滤一下。



(编注:Donate 部分省略,保留问题部分)



唐湘龙 46:51 

好,没问题。好了,我我来我就跟你拿到现在柏林利好就是支持呢。开枪龙行天下开枪王孟源 brother CAN 001 感谢那马克州要请教孟元博士,台湾的政治终极方案是什么?渴望听到王博士的真知灼灼见。这题我先跳过去,因为跟今天的主题距离比较远,然后托意的中中。好的,再来这个的丹升张,他说 2023 年的重要的划时代大师是金砖集团扩远以及华为突破美国科技封锁,推出 7 纳米5G尖端手机,希望美国控制的话语权也有自我崩溃的时刻。如同 1968 年的越战的农历新春攻势,美国公众普遍体验到美国陷入越战,难以抽身,掀起反站高潮,不再信任官方到底杀死了多少越共的宣传。好。这里面谈到的就是华为的内制的手机。



王孟源 47:47 

我可以谈一谈。对,半导体发展是,其实中共高层一直都是有理解它的重要性的,主要问题是没有选择正确的方向,还好是他们的习惯。是啊,不只是两条腿走路,好几条腿走路。那所以他们的所谓的 plan a 就是 a 计划,首选的计划是所谓的半导体大基金,结果这个半导体大基金搞的乌烟瘴气,糟糕,有 100 多个人贪腐被抓起来。这个我去年写了一篇文章批评他,就是刚好那篇文章发表之后一个多月,这些人就被抓起来。那被抓起来就是代表最高层认识到那个方向是错误的失败。OK,那华为跟 SMIC这条路就是他们的 plan b,那你现在看到的就是这个 plan b 有初步的结果,但是这个初步的结果代表这是正确的方向,但是并不代表已经成功了。因为你不管怎么样,这个SMIC 的N+2 7 纳米技术,它还是一个 DUV 的技术,就是Deep Ultra Violet的技术。对,它基本上已经到尽头了,即使不是最终的尽头,也已经快到尽头了,你没有继续升级的潜力。



王孟源 49:20 

OK,那与此同时,中国在半导体设备上的自给率也仍然是相当的可怜。所以看成就,不是大成就,方向是完全正确的,有没有潜力?有。而且我觉得最好的,最正面的一个观点就是最高层已经完全理解以前那样子用自由市场撒钱的方法,是错误的,OK,你必须要集中力量办大事。



王孟源 49:52 

那如果目标已经正确的目标已经定了,而且这正确目标是 10 年前就已经定了,然后正确的执行方向、执行手段也在去年一年多前就已经被明了。嗯,那么我认为以中国的执行力就没有什么好担心的。这也是为什么我过去这一年基本没有在博客谈半导体的原因。因为我觉得没有什么好担心的,这完全是大家等着一个接一个的好消息。你现在华为之后,明年还会有其他的好消息,但是大家不要自满,因为这只是一个很浅的第一步,就是你真正重要的那些半导体制造仪器并没有突破。OK,所以这可能还要一两年,还会有一些比较初步的结果。



唐湘龙 50:43 

好,我压抑我心中的问题,来继续看我们的听众朋友的这个留话的问题。来,王博士,早。大陆近期舆论场开始有很浓厚的向外国博物馆要求归还被盗窃的中国文物的讨论,也引起外媒的注意。我的问题是,这可否是一个契机,让中国带领第三世界国家再一次的反帝反殖民,并且要求归还被盗窃的文物?



王孟源 51:09 

我上节目之前刚刚看到一个类似的问题到我的博客,所以我已经回复了。嗯,OK,那我就再简单的重复一下。嗯,我刚刚提到美国的霸权三要素里面其实最高段的也是最全面的是美宣。OK,那美宣的基础内,你看它的内容就是把当前的国际政治矛盾定义成自由民主集团跟专制集权、恶霸之间的斗争,你看他们怎么形容中国,怎么形容俄国的,对不对?那其实,实际上的现实是殖民帝国压榨第三世界和第三世界开始反抗的一个现象,所以这个问题所谈到的这个外国博物馆的产品归还的问题,其实是一个很好的契机,你可以用来宣传,然后把当前的政治冲突的Framing重归正轨。所以,Framing我不晓得中文怎么翻,就是言论框架、思想框架。你做这个辩论的时候,最高段的就是要让对手落入你的框架。嗯,有一些结论是很自然的,你只要选对了框架,那些逻辑结论就自然出来。比如说刚刚我举的例子,你如果把这个矛盾定义为殖民帝国跟反殖民者之间的斗争的话,那谁对谁错?嗯,估计马上就看出来了,对不对?你要你如果是继续走他们的框架,什么自由民主跟专制选,那你是自己找麻烦。



王孟源 52:59 

所以这个这件事其实不是一个文物或者外交的问题,而是一个宣传的、国际宣传的问题。嗯,对不对?战略的问题。那,我把这个 frame 这个框架跟大家解释之后,我想结论应该回答,应该怎么回答这个问题?这很简单,但是我对中国的新国务院没有信心,如果是前一阵的,就是去年之前那个国务院是绝对不会去做的,现在这个新的国务院上一个国务院基本上是正确的,是绝对不会去做。我们刚刚提到那个那个货币。



唐湘龙 53:41 

然后现在是李强时代。



王孟源 53:43 

李强的时代我们还要观察一下。



唐湘龙 53:47 

所以李克强跟李强还是不一样的。



王孟源 53:49 

对,少了一个克。



唐湘龙 53:51 

那耶律王感谢,然后 PEG 王亮感谢,然后 hero 的游戏感谢,然后 k 的黄好,那光好人真的比较帅。还在讲王孟源对不对?好的,乌艾瑞斯达,他说王博士,欢迎欢迎欢迎来大陆,如果来大陆有演讲,我一定会千里万里也要赶过去听。那华为!送我王博士一部手机。



王孟源 54:16 

这个我们昨天吃饭的时候我不晓得有没有听到,你昨天晚到,所以……



唐湘龙 54:24 

我是迟到龙,所以我迟到了



王孟源 54:25 

唉,对对,你是迟到唐,迟到龙。那我在讲到这个的时候你可能不在,就是当时饭局里面有人问我说当初那个华为美国陷阱那本书怎么回事,我就简单讲了一下。这个这本书是我介绍给华语听众了, 2019 年的时候,因为这一本书当然美国人不会去翻译,然后英国人也不会去翻译,所以他没有英国本,只有法文版。那我当然也不讲法语,我之所以看到,是因为刚好经济学人里面有一个段落提到这本书,然后我就去Google,我看到以后马上就知道它的重要性,因为那个时候华为孟晚舟已经被抓起来了。对,OK,我就开始用尽我一切的关系,就是包括八方论坛,还有跟观察者网,后来观察者网的总编辑也注意到这有多重要,所以他们甚至去把那个原作者从法国请到上海去,就是阿尔斯通的副总。



唐湘龙 55:33 

阿尔斯通的副总



王孟源 55:36 

那所以后来很快的任正非就有发表一张公关稿,然后他的照片就是他坐在他的书桌那边,然后书桌上摆的就是美国陷阱,哈哈哈,我不晓得他有没有看到那张照片,哈哈哈。然后接下来我就觉得我应该在家里面等,任正飞打电话给我,感谢我的贡献,结果等了等了四年,他好像还是说什么都不肯给我给打电话。哈任先生,我很佩服你,你有空的话跟我聊一聊,我想跟你认识一下。



唐湘龙 56:16 

哈哈哈,我觉得你这次去的话搞不好有机会。好,再来,我们看,我们看这个威力恰恰,OK。他说王先生四年博客的读者他都要支持,看了博客。他说在这边不用严格遵守了博客的规则。他想请问王先生,您个人认为人最重要的品质是什么?那他说——他可能谦虚了——他说作为智力不出众的普通人,经常都会感受到无力感的迷茫。



王孟源 56:48 

人最重要的品质是你要做一个人。你在做好的中国人跟好的美国人之前,嗯,你必须要做一个好人,这个好人的意思就是说你——人都有利己的本能。对,但是你在利己的时候,对于损人要有迟疑,要有抵抗力——因为喜欢损人利己的人就是坏人。对,损人利己有抵抗力的就是好人。嗯,OK,喜欢损人不利己的就是疯子,所以我希望你不要做疯子,也不要做坏人。嗯,它是最基本的,做人最基本的原则。



唐湘龙 57:33 

对,OK,好,再来,蔚蓝价的一根烟。好,请问王博士,你对金砖货币的设想作为一个使用储备货币与贸易结算之用,不进去金融市场,防止投机炒作。以您所设想的贸易结算,只有各国该如何实际操作?这我们刚才前面谈了一大段。



王孟源 57:50 

对,就是你这个篮子货币只用来做国际贸易。国际货品定价跟国际储备。各国内部市场交易或者其他一切活动依旧用各国自己的货币。所以你人民币如果继续要不自由结算,不自由流通,完全没有问题。



唐湘龙 58:12 

好,再来的杜拉巴柔不借来孟云博士好,破仑感谢,然后 solo 曼陈感谢布拉德夫 brother 夫 K 001,感谢。然后匡摇他说他说湘龙把你的Mate 60 Pro拿出来



王孟源 58:28 

是啊,拿出来,我也想看一看。华人的光荣,拿出来看看



唐湘龙 58:29 

我昨天我这个跟凤馨在吃饭,我放车上你没有先接说,不然我刚就拿上来了。好,这个我们昨天已经吊了王孟源的胃口了,已经跟他吹了老半天了。



王孟源 58:46 

对两个人,一个人唐相龙在我一边,然后在我另一边,两个人在边谈 Mate 60,然后就是不肯拿出来。



唐湘龙 58:54 

因为我们都没有带。那个场合,他是怎么好意思。好再来征信运的?征信,他说他请问王老师,中国经济到底出了什么问题?以及未来的展望?这问题听起来写得很简单,不过他确实大家关注的问题。



王孟源 59:10 

是,目前的确是有一个低潮,这个低潮的根本原因有外来的跟本土的。这个外来的主要是欧美的消费已经因为新冠疫情之间过度消费,而目前有一个反动,OK,所以这个是你是没有办法的,这种事情你只能承受。OK,但是内部有很大的问题,比如说你的教育,你的金融管理。嗯,我这个都是我博客讲了不晓得多久的。你因为四年大学的招生员额提升了40\%,多了 300 多万,那你自然就会有 300 多万额外的青年失业,OK,那这种经济问题是哪里来的?自己找的国务院的叛徒为你创造的?



唐湘龙 01:00:06 

哈哈,你一定要这么重地……



王孟源 01:00:09 

替习近平的这个改革派下绊子的。所以,大家看到那种为了几万块美金在当为 CIA 当间谍的,大家气得咬牙切齿。其实那些人对国家的伤害跟中国国务院教育部跟那个人民银行的官员所做的伤害比起来,还是小的多。九牛一毛都谈不上。



唐湘龙 01:00:38 

好,这段话很有启发性。来再来看。嗯,他说呢?林冠,他说王老师好,请问老师如何看待最近出台的支持福建两岸融合发展示范区?



王孟源 01:00:50 

这是唐湘龙很昨天谈了很多的话题嘛。对。



唐湘龙 01:00:54 

我昨天在饭桌上面有谈一些,



王孟源 01:00:56 

对,目前有很多的猜测,唐湘龙的猜测是认为他们的统战的路线从针对国民党变成针对民进党的选民,(的基本盘)这当然应该是正确的解读,但是我不认为是全部面的解读。目前可以确定的是在二三十年拿头撞墙——也就是这个国台办这种一味的讨好谄媚的策略的措施之下——它终于有反思,有一个全面的反思的过程。那所以我们可以,我需要一段时间来观察它这个新的方向到底是什么,但是我们可以确定是有一个新的方向。



王孟源 01:01:43 

那我之所以不确定他你的那个解读是 100\% 的故事的原因是,你可以更深层的考虑是不是这个转向,是不是因为从更高层的来看,他们已经不认为促成稳统是唯一的目的,OK,而是不管文统武统一之后的治理也需要未雨绸缪,OK,所以你必须要先跟未来的中国民众,台湾区民众来做好沟通,你也可以这样解读。所以,但是我们目前没有足够的证据来确定,就是到底要提升到多高的层次,他们当然没有公开了,但是我们还需要一点时间来继续的观察。



唐湘龙 01:02:40 

对它有很大的发酵空间。但是把一个过去的统战工作从就中央的统战部、国台办,然后现在以一个中央支持的表述的方式,嗯,然后把它下放到福建了。嗯,那你简单的想就是我认为其实闽台关系,隔着台湾海峡的闽台关系,在过去台湾民进党喜欢谈的一边一国,但大陆告诉你说不是一边一国,是一边一省。第二个就是说闽台的关系将来就是一国两制,一国之下两个制度的示范都会在闽台之间进行。



王孟源 01:03:17 

基本上就是你的战术解读是正确的,它绝对是故事的一部分,但我只是怀疑它是不是还有一个战略层面。



唐湘龙 01:03:26 

这个这个在进一步的观察。嗯,好,再来潜伏南洋。好,感谢然后的 so 菲亚,感谢他从王博士,如果美国持续保持大概 3\% - 4\% 的CPI,然后大概就是它的这个通膨率了,然后再来4.5\% - 5\% 的利率,那我们会进入怎么样的一个金融时代?比如对美国的房地产是什么影响?比如中国应该如何应对?如何保持低利率?会不会让很多人反向套利?



王孟源 01:04:01 

你所描述的这些美国数据其实是几率最大的数据,在未来两三年。



唐湘龙 01:04:05 

现在就几乎是这样。



王孟源 01:04:06 

但是因为美国仍然有美元霸权,所以它有主动权、战略主动权,它可以优先保证达到这些数据。但是从国际战略的观点,这是好事还是坏事?必须要看相对的,就是其他的国家跟他相对是怎么样?就是,好,美国是这样子不愠不火的,不是很好也不是很坏的局面。那你其他的国家,第三世界国家死了一片,或者是欣欣向荣,这个意义当然完全不同,对不对?那你真正关心的那些产业的流向,事实上也是取决于其他国家的经济走向,而这是最难确定的。



王孟源 01:05:01 

中国其实还比较好预测,就是如果中国的国务院愿意修正自己过去的错误的话,即使当然还来得及挽救这个局面,然后等国际贸易局势,经济贸易局势稍微缓和一点,中国的经济也会跟着再回去。但是如果他们坚持的拿着枪打自己的脚,其实是打自己的肺了。嗯,那下,而且还可能继续下一枪要打自己的心脏了。那我们英文说all bets are off,就没有什么好谈的,那个问题自己要找死有什么办法?



唐湘龙 01:05:43 

好,再来噢。在这个的微皇王,他说从几个网上的授课当中获益不少,谢谢王博士跟香龙的观点节目,感谢90K,感谢我们的老听众,感谢你。好的 Allen 陈,他说政治的选择性这么的清楚,连小学生都不如。好,他在说,我再来,我们再看这个苏战s,他说谢谢,一如既往的受教了,然后加油,大家加油。谢谢王博士地球哇本尊打赏致敬。



唐湘龙 01:06:19 

我说今天开箱瓦王梦,让大家看一下王梦云的本尊真的好,再来就要酱黄感谢,然后再来跌迪尔瑞夏朱,他说感谢王博士,受益良多,难得一见的在现场抖一下。那欢迎回家,在台湾好,再来。那斯丹利这应该是在泰国吧? THB 是不是在泰国?应该是的,我们查一下。嗯好的 g h 他说感谢呢,王博士跟唐先生之前,做生意在大陆,做学术在美国的环境更好。这个论述对年轻人是不是还成立?



王孟源 01:06:58 

这刚好是我昨天给演讲的那个终极话题,就是个人的生涯规划。我说你要留学还是去留学,不过 30 年前像我那一代去留学的话,毕业之后留下来在当地找工作是一个很好的选项,对大多数人来都值得考虑。但是目前我觉得在现在这个时代应该是反转过来了。我建议大多数人在毕业学成以后就回到台湾,或者是到中国或甚至到俄国去寻求工作的机会。



唐湘龙 01:07:36 

什么原因?你是因为那个环境不友善吗?



王孟源 01:07:39 

对,还是因为它的社会全面腐化了,他们已经完全放弃了选贤与能的概念。一切都以政治正确为主,然后社会的犯罪跟混乱都是有额外的危险,而且它的经济成长的问题,而且贫富不均的问题都已经到达一个层次。这个在欧洲特别明显。这个你去看英国跟德国的那个贫民的比例,年年增高。那你一个留学生要去跟人家抢工作,不是自己找麻烦。



唐湘龙 01:08:15 

对不对?这确实是个大问题。我每次看欧洲的时候,我都觉得从你看,从欧元出现到现在为止,欧元相对于美元的,它的价值最高的时候几乎就它成立的那个时候,之后欧元的一路都往下,整个欧洲的经济,你回头看有欧元出现的,美国跟欧洲的经济发展的对比,其实只有肥到美国而已。美国的以人均美元计算的人均来讲,美国还一路在往上走,可是欧洲近乎是停滞甚至是微缩的。



唐湘龙 01:08:49 

好,那再看青眼,难得一见那这个王博士的本尊,唉,跑到哪门去了?跑到前面。哇塞,前面还这么多好,然后看,好难得一见王博士。青年节目必须要抖一下。好了,乌艾瑞斯,他最近西方媒体疯狂炒作中国的经济崩溃论,我长期观察德国媒体也是跟风炒作。我很奇怪,明明德国经济更糟,那我想想,我说这是不是国际金融的大资本利用媒体在配合国内金融的卖国贼来迫使政府大水漫灌,然后再来收割一场?



王孟源 01:09:24 

当然,当然就是其实他们的最主要的忽悠对象是全世界的资本,尤其是工业资本,所以必须要吹印度,因为印度你如果不吹的话,它的现实实在是太可怜了,太可怜,太可怜又可怕。对,那昨天饭局里面也有一些台湾的企业家,他们也是,我们现在要从台湾撤离,我们现在要从中国撤离。对,其实这就是他们这些美宣的财经,美宣的用意,就是让你把——因为经济发展的要素之一就是资本,你这些资本如果错置的话,他的那个成本负担是谁?是这些被忽悠的人,而不是美国人,对不对?那一起损失的是被唱衰的那些国家,对不对?像中国、俄国对不对?那所以其实是这个财经战线其实就跟现在的俄乌战场的战线,还有潜在的台海战线一样,都是忽悠你,嗯,忽悠 B和 C 去打的头破血流,然后美国作为A 可以在旁边渔翁得利,对不对?你作为错误投资把钱投到印尼或者是投到印度,你说对美国有没有好处?当然有,因为这样子就削弱了中国,然后你这些企业没有美国政府的那个政治力量强力保护的话,会被当地的,尤其是印度这种国家吃光抹净对不对?你美国企业也许还有点保护,但是中国企业去那边你骨头都不会吐出来的。



唐湘龙 01:11:16 

哈哈,好,来再来。克伦伯逊网感谢然后的 joy 网感谢然后以前的龙井,然后看到王博士揭露中国金融高层的亲美和互美,感到愤怒,请问中国还能够修复失误吗?难道全中国再也没有人能看到叛徒?我不知道后面这句话什么意思,中国还有修复的机会吗?不过这个这是一个政治问题。



王孟源 01:11:43 

两年前那个机会已经错过了,所以下一场经济危机至少要 5 年,通常要 10 年,这应大家再看。



唐湘龙 01:11:51 

OK,好了,周感谢他。请问王博士金砖和 G7是不是一样?没有办法有什么作为了?



王孟源 01:11:59 

这就是我刚刚讲的重点。



唐湘龙 01:12:01 

对,好,再来Allen 陈,他说应该说谢谢老唐一个一个的反中人士浮出水面,这个也没什么。对,好,再来这个鲁分鲁都。



王孟源 01:12:14 

对不起,我这个人,你如果觉得我有立场,这是因为我已经做了科学的分析,根据事实来做判断,我选择好的。嗯,对的,是的正确的那个立场,而不是我先预设一个立场,然后他去找借口。OK,所以这请不要把我跟跟其他的评论员,或者是美国的那些所谓的intelligentsia(xxxx,作对比?)



唐湘龙 01:12:44 

没有错,这个是王孟源一直努力在避免,我觉得他一直在回避一般的这种的名,所谓的名嘴式的舆论场的那个调子。



王孟源 01:12:54 

你如果觉得我反美,并不是我决定要反美,我决定反对的是谎话跟坏人。结果我去分析以后发现全世界的 90\% 的谎言是美国人讲的,而他这个谎言是他全球搜刮,这也是全世界最大的恶。因为这个原因,所以我才会反对他,而我反对他,我所做的也只有一件事,就是揭露他的谎言。



唐湘龙 01:13:23 

好,来鲁芬奇,他说刚刚刚翻了王老师以前的文章,确实很多事情都预言到了。那感谢这王老师继续给到专业的意见。不同的声音是最有价值的。



王孟源 01:13:38 

错了!正确的分析,科学的态度是最有价值的,不同的声音只是噪音,只有负面的这一页。



唐湘龙 01:13:46 

好,再来 hi task。他说谢谢孟源老师,请问乌克兰离被抛弃了还多久?真的是西方陷太深,被乌克兰反绑出不来了吗?



王孟源 01:13:57 

这个决定权是在俄国手里。嗯,俄国决定什么时候摊牌,嗯,什么时候乌克兰就倒台。但是普丁觉得现在还言之过早,因为你继续拖下去就是继续消耗北约,对不对。



唐湘龙 01:14:11 

好,那美国现在又增加了 10 亿的军援,然后也要提供战术导弹……



王孟源 01:14:16 

是个无底洞,他一个月就可以赶紧烧掉 10 亿美元。



唐湘龙 01:14:21 

所以,乌克兰在这一波的所谓已经突破了第一道防线的说法。没有,其实他只有摸到而已。



王孟源 01:14:29 

但他打了三个月打下来的那个村庄 robotnik 内是在一连四个村庄,那一条主要公路,一连四个村庄,的第一个,而且是唯一一个在河谷里面,真正的第一道防线是透过第二个村庄这样子,所以他们打下这个第一个村庄花了三个月,打这个村庄以后其实是把自己陷入一个口袋,而这个口袋就是这个河谷这边有一个丘陵,然后这边有一个丘陵。嗯,你就只要去看一下地形图就知道,这明显是一个陷阱。而且所谓的第一道防线很容易就判别,就是所谓的龙牙线,嗯,龙牙是什么呢?就是那种水泥做成的,嗯,反坦克的那个障碍。他碰到了那个龙牙线了没有?没有,还没有。



唐湘龙 01:15:27 

好好再来罗琳,感谢他,谢谢了王先生的真心话,然后苦苦心化良药,苦口感谢。再来 Alan 徐,他说谢秦欢博士,那中国央行已经换人了,一篮子货币的亚元还有机会起死回增吗?



王孟源 01:15:45 

有,但是你至少还需要——即使这个人。这个新的人民银行行长现在就去全力推动——我想大概也还需要 18 个月到 24 个月。而且现在没有迹象显示它会全力推动,因为他是新任的。总之需要一点时间适应。而且没有什么信息。你看,即使是那种汇率被易纲搞到 7. 3,现在他等了半年了。嗯,换手半年了他才敢把它转过来。嗯,现在是稍微上升了一点。



唐湘龙 01:16:20 

是一点而已。对,嗯,好,还在挣扎当中,不过有反转讯号了。对,好,让他去元珠珠其元。好,那请问一下就有关于大陆EUV 的发展有机会吗?你刚才稍微提到一些了。EUV,嗯,就光刻机。



王孟源 01:16:39 

哦,EUV,对噢,没有任何迹象,目前连他们现在华为这个手机的那个芯片所用的 DuV的生产器材仍然是进口的,仍然是ASML,还有那个尼康的。中国的半导体生产器,才连 28 纳米的都还没有成熟,就是可以实验但是还没有成熟。



唐湘龙 01:17:09 

当然最近大家因为华为的这些新手机,然后对里面的这些这些晶片零组件的想象,大家兴奋了。所以我就开始有很多的——我没有办法认真假了——就是做中国的自主的光刻机到光刻厂、光刻村,然后几乎每天都会看到就是说中国的光刻机即将有大大突破。我不说可以或者是不可以,但是我认为本身第一个它不是一触可及的事情,还是要脚踏实地,刚刚讲的就是说那个大半导体基金最后非常的惨淡,除了亏了很多的钱,抓了一堆的贪污犯以外,最后什么都没有留下来。所以光,光是光用光光是用嘴巴讲,除了短时间让他有兴奋感知之外,他终究还是要见公婆的。



王孟源 01:17:55 

不过,就是这样的,你短期的话有一个明亮的闪光,但是它就是一个闪光而已。中期的话你必须要有耐心,必须要很保守。但是长期的话我很乐观,因为,最高层已经全国动员来办这件事情,而且选择的手段是对的。这个我觉得华为这个新手机最大的意义就在证明说您有一个可用的手段。那这在去年一年多前,半导体大基金被整个垮台之后,你没有一个明显的台面上的路线,这才是最大的正面意义。就是说你现在看到了一条路,并不是说你已经走到目的地了,而是你现在终于看到一条可以走的路。嗯,那这是我觉得这是让我长期乐观的一个最大的因素。



唐湘龙 01:18:52 

好的,文尼萨毛感谢,然后黄蜀锦感谢。那赶快送上束脩,从王博士学习到很多。对,我。



王孟源 01:19:02 

想提醒一下大家,如果你到我的博客,然后以为你可以打赏的话,我不收打赏了,我只收束脩,OK。



唐湘龙 01:19:10 

哈哈哈,只收束脩。好,那个王老师的说法我比较认同,去到 30\% 的绿色铁盘,剩下将近 30\% 的本省家庭,一部分的外省人在统一之后的治理民意上会获得优势。而如果本省年现在厦门置业就业,小孩上学,那么在支持上就会更倾向于大陆,这确实是有战略高度的安排,统一不难,统治比较难。好了,再来波南g,他说感谢王博社相龙,一句话来概括我的问题,请问美国商学院 MBA 教的东西是不是没有用了?



王孟源 01:19:42 

哦,很有用!它用来摧毁人类社会的那个生产力,非常的有用。



唐湘龙 01:19:50 

如果,那他是很认真在问你,他的意思说我如果要去美国念MBA,因为还是有很多的名校。对,你听到那个招牌还是很有吸引力,你听到华顿,听到斯坦福。如果你任何一个孩子,年轻人如果申请到,你还是会为他高兴,所以我认为他是很严肃的在问,他如果……



王孟源 01:20:09 

我认为阅兵的时候,你越重要的战略武器排在越后面,所以最后面一定是什么东风41。但是我觉得东风 41 之后还应该有美国 MBA 的那个院长列队,因为他们才真正是是全球毁灭性武器。



唐湘龙 01:20:28 

哈哈,你对这些家伙对这些人的那个批判已经到了最高级了。好,再来小楷王,他说很好奇王博士作为社会主义者到对唯物主义和形而上学的看法。传统马克思主义对形而上学是嗤之以鼻的,但是几十年的数据表现,人类不是能够以数据和模型作出理性判断的物种,很难成为真正的唯物主义者。



王孟源 01:21:00 

Well,这是因为你的看到的是比较低层的,也就是说原本的马克思的版本。马克思的理论有很多可取的地方,尤其是尤其是对资本主义自由市场的批评,非常的超越时代的。你在 19 世纪就能够写出那样理论是非常的了得。但是他有好几个很大的问题我在博客上也都讨论过,其比如说他没有怎么样用,完全没有一句话谈到怎么用国家机器,来解决这些问题,来解决这个贫富不均的问题。所以 20 世纪就有列宁跟毛泽东做了一大堆的实验,而既然是实验就必须会有失败,那这些失败就变成那个资本主义殖民帝国集团用来做反向宣传抹黑的一个很好的题材。那另外一个问题就在于它是一个基于西方哲学思想而推到极端的,所以他虽然是反神学,可是他的精神其实是西方一神教的一直传承下来的极端性的思考。



王孟源 01:22:18 

我认为以中国的理性人本主义的传统——儒家跟墨家都是理性的人本主义——完全可以包容这种逻辑辩证和社会主义的思维。那所以在实用上中国的最大的的优势就在于有中国有那种士人的传统,就是知识分子必须要对国家跟社会有义务,必须要做出牺牲贡献,这是在实用上;在思想上中国也有比较好的把人本主义跟理性主义扭合的一个既有的定论。



唐湘龙 01:23:10 

好来立王,感谢他。王孟源先生有没有定居大陆的计划?你在美国要注意人人身安全,你怕有人拿枪。其实。



王孟源 01:23:19 

我有问过我在大陆的朋友说我是不是应该长期到大陆去?嗯,那因为他人很好。嗯,而且我们之间也都很坦诚。他就直接跟我讲说你得罪的人太多了,你如果来这边会死的。他我也。



唐湘龙 01:23:37 

很担心,我也很担心你。不要,你不要太低估这件事情的风险。对,他在美国还不见得会莫名其妙的去找王宝元,这大陆就不一定了。好,再来。他说王博士下个星期还来吗?还听不够。不要了,我跟王博也讲好了,他的生活非常的steady,就是我,我觉得我维持他对他自己的作息形态的这个坚持是很重要的。以前龙井感谢他,很抱歉。他说他说我以前误会王博士是反中名嘴,我正式道歉,希望中国高层能够认真分析接纳王博士的意见。



王孟源 01:24:13 

没有办法,现在的这个网络跟舆论环境就是其实是有意地愚化大众。



唐湘龙 01:24:20 

没有错。好,再来这个的。小圆,他说今天在微博上有个新闻说高能光源加速器,不知道王博士知不知道这个东西?它可以是另外一条的生产途径吗?



王孟源 01:24:32 

噢,你是说用来做半导体啊?有这个可能,但是他的问题是能够做得到,但是难点是在于经济性。就是为什么我们现在的这个 ASML的EUV,他是用的是那个紫外光,锡粒子的紫外光光源,因为它你不管选择哪一条路都是非常的困难。最后觉得是只有这个,能够以经济性勉勉强强做到,嗯,足够的 Throughput 就是频率,生产频率是。那你这个用加速器来做最大的问题就在于它的那个重复性很差,你没办法一秒做几千次的这个的放射,所以当然这是一个工程问题,并不是说没有解决的可能,但是目前看来要解决的话它代价很大,在时间跟金钱上都是。



唐湘龙 01:25:31 

很大的。好,另外这个还有两位来弗洛瑞手,他说王博士你好,请问你对一下,你知不知道就是说这个 Michael Hudson,他说他一生都致力于半导美元霸权,我虽然喜欢他的学说,可是想要辩证的看待。你的这个问题,听听看其他人是怎么看待的他。



王孟源 01:25:49 

OK,我的小孩,我教育的时候我很注意端正他的三观,然后我原本送他去大学念医学预科,结果他去了一年以后回来跟我说他要转转主修,我说你要转主修什么?他说我要转主修经济。我说你想象中做经济是什么样子?他想象中的经济学就是像我这样子指点山河,批评国际的经济问题,还有政府。



唐湘龙 01:26:29 

不是,其实一个儿子会从爸爸身上,想象爸爸。我觉得很骄傲啊。



王孟源 01:26:35 

我跟他说,可是实际上的经济学家是在那边拼论文的,然后对想办法拜山头,然后要讨好那些大佬等等。对,事实上很多污烂很多没有意义也可笑的事情。但是他不听,所以他去了以后还是继续拿这些问题来问我。可是问题是他中文很烂,他不能看我的博客,然后我又不用英文写。



唐湘龙 01:27:00 

我不能这样说,因为你,毕竟你念物理,然后也出去,也有很好的一个学习经历,但之后你也在西方的金融体系里面给我工作了非常久了的时间



王孟源 01:27:13 

所以我的经历是比较特殊的。但是基本上到最后我终于想通了。嗯,他要问我这些问题其实都有解答,我就跟他说,你去读Stiglitz跟Michael Hudson的书。为什么呢?因为他们是用英文写的,第一个。第二个他们是pedagogical,就是教学性的教材。以他这个 20 岁出头的年纪,他所需要的是观察更多的案例。我是已经观察了几万个案例,我已经总结了一个完整的逻辑认知架构。但是你说我要把这个逻辑认知架构强加给他的时候,他有一个很健康的质疑。那正确的学习手段。是让他自己去先观察这些几万个案例。那我是不可能从头用英文把这几万个案例讲给他听的。所以我就跟他说,你去看Stiglitz跟Michael Hudson的书,因为他们是为大学生而写的。嗯,那么他们就会把这些案例一个一个举给你看。



唐湘龙 01:28:19 

好。但是因为我知道他们的书都有繁体中文版了,但是简体中文版有没有可能大家自己去搜搜。



王孟源 01:28:25 

所以如果有博客的读者觉得我的我的写作太精简的话了,也许你们应该先去看一看Michael Hudson跟Stiglitz。



唐湘龙 01:28:36 

嗯,好,大家的小九感谢。然后 g h。因为时间的关系。好, g h。感谢,然后与前龙井最后的一笔感谢。好,因为时间的关系的,我们都已经快到 11 点了。哈哈,那我已经尽量熬他了,这个刚好他在台湾了,所以我才可以这样的熬他。感谢呢,收看呢,今天的龙行天下也非常感谢。今天王孟源来到节目的现场,接受所有的观众朋友开箱,终于可以看到本尊,不用看那个框框。好,那一个孟源会在台湾,好会待可能两三个月的时间,有一段时间他可能会在大陆,下个月的节目的时间他在哪里,我再来跟他协调。感谢孟源。



王孟源 01:29:14 

很高兴跟大家聊天。



唐湘龙 01:29:16 

是感谢所有的观众朋友们来跟大家说周末快乐,下午再见。拜拜见。



\twocolumn[\begin{@twocolumnfalse}
\section{国际金融未来趋势}
\subsection{20230915 }
\end{@twocolumnfalse}]【大学讲座】国际金融未来趋势

2023/09/15—编校:上海网友S

PS:感谢视频主;原视频字幕遗漏和错误已基本修正,个别不明处以“...”标注。



*

王孟源 台湾北市大第二场演讲

*

王孟源:

...可以多印80倍,因为市场的价值就是这么多。我们只知道财富进了美国的口袋,那这些财富是哪里来的?它们是从全世界其他必须要用的一样东西,但是因为它是如此的间接,所以呢你们很难条目的指出它这个是怎么样运作的,这是巧取豪夺的最高境界,也是为什么我昨天讲我所创造的这个形式(△)是指最终形态。

那我们来看一下吧,这个布雷顿森林系统支持了它多久?只有不到30年。到了1971年——大家去看这边这个上去的——这个是越战,越战一开始美国就开始疯狂印钞;然后到了1971年,尼克松总统在那个时候他说,我们当初答应凯恩斯这个美元锚定黄金的系统,现在是打破的时候了。哈哈哈...所以那个时候他们就派了他的人通知——不是跟他们讨论,不是跟他们谈判,而是我去通知他们:你们的财富要消失了,我们要拼命的疯狂印。你们呢唯一的选择就是跟着我们印。(笑)要不然的话每年贬值下去对不对,你们的德国的马克还是只有那么多,每一个马克的价值都会上去,那升值上去以后就会对进出口贸易有非常恶劣的影响,你们的进出口马上就会越来越严重。当时的这个财政部长留下一句名言:Our currency is your problem(我们的货币是你们的麻烦)。哈哈哈,这真是豪气干云,全世界最豪的流氓。



 *

小约翰·包登·康纳利(John Bowden Connally, Jr),1971年上任尼克松的财政部长。留下名言:The American dollar is our currency, but your problem.(美元是我们的货币,但你们的麻烦。)



然后因为大家一撇呢这个通胀就很厉害了。然后到了1973年至1974年这里有一个高峰,高峰之后呢他们咬着牙忍受,他忍受的除了石油价格之外还有不断涌出的美元,他们当然希望不断的储备美元,美元储备永远是有价值的;到了1979年发生了第二次,这个时候通胀就完全失守了。



大家请看,这边超过6\% 呢...这边到这里五六年呢完全主旨让膨胀固化。固化的意思是什么?工会说,我们过去5年的通胀是6\% 甚至曾经高到12\%(表一),所以今年我的工资要上调。那一旦这个写进工会的协议里面,明年的通胀就至少是8\% ;那在下一年当中:哦!去年的通胀是8\% ,我的工资也是上涨到8\%,所以要工资上涨到10\% ——这个就是进入了冰雪前的支出计划。为什么大家谈所谓的deflation,滞涨?(Stagflation;deflation系通缩,或为口误)因为它是有这种非线性的滚雪球的危机。

那你看它最高曾经到达14\% ,那为什么又一路掉下来?因为在整个1970年代,这个通胀的危险一直在的时候,他们的美联储的行长Arthur Burns(阿瑟·伯恩斯),他的模式:我也不宽松也不紧缩,那通胀呢是6\% 我就是6\% ,是8\% 呢我就是9\% ——实际利率是9\% ;我祈求的是这个平均价格不要再多涨,然后呢这个经济很自然的就会遇到一些意外是通缩的一块,那这些通缩的一块也会发现它降下来,他基本上就是在赌运气掷色子,因为对他来说他这样子政治风险最小,他不需要制造经济效益。结果呢他赌输了,1979年的第二次能源危机证明他完全赌输了,然后整个通胀又失控;失控之后他就提早退休。(笑)然后换上来一位铁血的中央银行行长,他马上把利率提升到20\% ,那实际利率就是6,就是当你通胀率是1的时候——比如说在四年前通胀率还是只有1——你的实际利率6的话就是你的利率已经达到7\% 。这马上会有一个立竿见影的结果,就是整个国家会陷入一个经济衰退。



*

阿瑟·伯恩斯(Arthur F. Burns),1970年至1978年任美联储主席。



*

保罗·沃克(Paul Adolph Volcker, Jr),1979年至1987年任美联储主席。



*

杰罗姆·鲍威尔(Jerome Powell),2018年,鲍威尔正式接替耶伦出任美联储主席,2021年连任

-

因为没有人敢再去借钱,而且是拼命的还钱。大家都不敢借钱还拼命还钱的意义是什么?没有人敢开工厂,没有人敢消费;消费者不敢去买房子,他们不敢去买船,不敢买车——因为你借不到钱。所以1980——1981的衰退是除了大萧条、大衰退之外美国历史上最严重的一次;而且也造成了美国历史上少有的一任总统做完以后没办法连任,...。(注:卡特总统)

然后在那之后呢这个铁血的第一个政策,就把整个通胀压到5\% 。5\% 现在看来是蛮高的,但是呢曾经高到14\% 的话,5\% 算少的。然后这边又有一个节点(表一),就是90年的节点是两个——我昨天讲的——美国同时打爆了日本和苏联这两个宿敌,所以获得了许许多多的财富;那这些财富呢就替代了美元作为后盾,所以美元的虚发反而会下架,美元虚发下架之后呢通胀率就下滑——4\% 、5\% 降到2\% ;基本上维持到2019年。然后在过去两年达到6\% 。

大家有问题吗?我可以简单的回答一下。

那这一波的通胀有几个跟1979年1980年那次——一个现任的美联储的中央银行行长所采取的法令跟1970年代是一模一样的东西。你看现在美联储的利率是多少?5.5,通胀率是6\% ;基本上他是维持5跟6——也就是说并没有试图压制通胀;他是在等随机波动。他为什么敢这样做?上一次的教训还不够惨痛吗?他敢这样做一个很大的原因,是1870年代造成通胀固化和变成滞涨的一个主要因素——我刚刚已经解释了——是因为如果出来强力的话就能够为工人谋取比通胀还要高的加薪,那这样就往上自然多下去了。但是呢在过去40年,美国的工会被阉割了,所以呢他们不再这样。美国的工会为什么会被阉割呢?就是1980年那位中央银行行长他说短期内我可以制造一个经济衰退,但是长期的话必须要对公共负责,釜底抽薪不是靠数字,也就是产业外移。他们的产业外移就是美国的精英认为工会尾大不掉,必须要打断他们把任何一个通胀都变成滞涨的危险,所以呢必须要产业外移,要不然他们的那些——当时的精英还是有智慧的——他们怎么会看不出你这个产业的情况很危险?他们是权衡之后觉得两害相权取其轻。

我的博客上面有当时他们讨论的一些结论,可以看出这是一个很严肃的讨论出来的话题,他们决定允许企业去追求短期的利率,把产业转移到台湾那边,然后后来再转移到大陆去。后来的全球化都是这个主体的后果,所以请不要轻视这张图。(表一)

这张图可以解释过去五六十年的国际经济经贸形势。为什么?因为美国是全球霸主,一切的变动都是来自于此。

-

*

-

这是美国的美元占全球国际储备货币的比例。1971年打破布雷顿森林体系之后它反而上升了。因为它不是锚定黄金,所以大家都紧张——它不但不是锚定黄金,它本身就是锚,所以大家反而拼命的进美金。那结果就是它的占有率就从75\% 到85\% 。但是在1979年1980年那一场14\% 的通胀高峰的时候,这些欧洲国家被血洗,因为美国的印钞一下上了一个档次,所以呢他们就发文件然后取代痛苦;怎么来取代美元?用彼此之间的货币——德国马克、法郎之间的;这些储备都是从那时候开始的。他们讲的25\% 的——所以呢也就从85\% 到60\% ,然后到1984年1985年又往下掉,一度掉到1990年掉到50\% 、40\% 。为什么?因为里根上台,开始无限印钞,所谓“供给链的改革”。

-

*

-

这个供给链改革的意思就是说, 我们可以毫无羞耻的去印钞票,(笑)反正我们把产业外移之后呢,我们印的美钞越多,从国外买回来的东西也越多,基本上是免费的。因为这样的操作它就在1990年的时候掉到了47\% 、46\% ——这是历史上的最低,二战之后美国货币的最低。然后接下来收割的是联合国。所以通胀下去了,然后美元就又回来了——不到70\% ;然后到了这个转折点是欧元,欧元在2002年正式的对外发布,然后...60\% ;大家要不要猜一猜过去两年的这个储值,美元的所占份额是怎么波动的?它们上升很久了,所以直到剥削人民银行——所以当初我也讲过的:美联储北京分行(笑)。

好,我以前讲过,美元是新时代的终极霸权三要素之一,三者互相维护;但是呢军事是一个美国拿出来用来在像俄乌战争这种情况下来做回馈之用的。日常的搜刮是要靠美金的。但是呢用美元来搜刮从1970年到现在真的50多年了,我们还是会有一些有脑子的国外政客会想要反抗。那反抗的话就必须要高效的把他们消灭掉。那这个消灭他们效率最高代价最少的,就是美宣。那美宣之中一个最高档的就是他们的自由和民主。整个美国经济学跟美国金融学其实——我个人认为是真正的愚蠢,全是殖民帝国的国际资本垄断,整个都是;所以我昨天说大家要留学,还能去的这么多,不包括这些美分。我昨天早上在上唐湘龙的节目的时候有人问说我认为MBA有什么用,我说非常的无用,它是人类史上威力最大的破坏性武器。为什么?因为商学院教的就是全部都是这样。

所有的美式经济学跟金融学都建立在两个前提之上。第一个:欧美的优势就是从500年前开始大殖民、大航海的优势,是建立在他们的文化;他们的新教比旧教有优势;是他们的个人主义文化、家庭主义...这都是借口,无穷无尽。他们不提的,就是真正的掠夺,Genocide种族灭绝。然后呢他们的经济学跟金融学另外一个很重要的隐性的前提,就是他们优化的对象是资本投资报酬率要最大化,而不是整体社会利益的最大化。如果这个资本投资报酬最大化的过程中呢,穷人要死掉几百万——没问题;整个热带雨林要烧光——没有关系;整个地球的海洋要沸腾——没有问题。那投资报酬率指的是统治集团的投资报酬率,也就是国际资本。那这个国际资本基本上是很少数的,不到20个犹太跟昂撒家族主导。

把这个应用在过去这几年最有争议的一个问题上就是利率政策。这个利率政策是中央银行的储蓄会,中央银行决定利率,然后通过利率来影响汇率。美式的理论说,你根据自己经济的需要决定利率,然后这个利率呢会自然的决定一个汇率,这个自然的过程中它的这个背景的因素来自于进出口。实际上,因为美元是国际储备货币,所以你要...如果有一段——只要有一段经济稳定期的话呢,美元的流通量比其他的任何货币都要提高至少一个数量级。

*

-

我想请问一下,如果你要买一样东西,这个东西的销路比它的竞争力要高10倍20倍的话,你觉得这个东西——样品A——它的价格会不会比较领先?一定是如此。所以国际的企业在借贷的时候,一定是借美元的利率最低;那借美元的利率最低的时候大家都借。这就是我昨天讲到的,美国用金融搜刮的所谓潮汐式搜刮。这个潮汐呢先放水,在经济平稳期间先把这些美元借贷投资出去,然后呢等到有经济危机的时候,美国第一个先提升利率,然后大家就——这些热钱就必须要回来,一回来以后这个汇率就越来越偏美元,所以这些回归美国的资本,它所采用的汇率是越来越不划算,也就说对美国人越来赚的越多;借出去的时候是一个利率,但是回来的时候,全球都主动的把这几万亿美元往美国投的时候,刚好就是美国汇率最高的时候——这是他的典型潮汐式收割。如果有兴趣的读者想要看最经典的案例就是1997年的那一次亚洲货币危机。泰国、韩国跟印尼被剥削的太露骨,就是他们借了一屁股的美元债——而且不是政府借的而是企业借的——但是呢一旦风声不对,美联储决定升息;一升息,那些银行呢就拒绝rollover(延期付款、贷款展期、滚动)——就是你在平常好的时候这个借贷到期了,你可以跟投资银行说我们再重新借一个三年期的贷款,刚好就是还这个,这叫rollover——但是,他们就是到了时候翻脸不认人说,我不给你rollover,你现在就全部还钱。好,这下一来,只要那个国家他所欠的美元债超出你的外汇储备,那他基本就注定要破产。

*

因为大家要还美元的时候是到哪里找美元,拿我自己的货币开放去换美元。但是呢你所需要的那个数量超过中央银行自己的储备的时候怎么办?中央银行只好拼命的让开放贬值,所以你能够换的美元越来越少。那这样子你这个汇率越倾斜,这些被逼着还债的企业吃亏就越大,到最后压的破产,连中央银行自己都破产。等到中央银行破产之后——我昨天也提过——国际破产并没有破产法,实质上的破产法是哪一个?

IMF——IMF它的总部就在美国财政部半英里以外,基本上是——IMF就过来了,过来以后他就跟你推麻将,说我们需要这个文件需要那个文件...那一拖6个月。大家去看真的1997年的时候,南韩真的就是拖了6个月,拖到他们的中央银行的外汇连一美元都没有。真的是整个国家所有的企业连国家自己都要破产,都要宣布还债。然后他进来说,我们的条件是,你要把三星都卖给我——你要把所有的这些破产的产业都对美国财团开放。这不是只对韩国,韩国是这里面最忠实的盟友,是黄金的十亿人的盟友;对他们都是这样子,那第三世界的他泰国人也被抢掉了,俄国也被这样搞了一次——不过俄国刚刚在7年前才搞了一次所以主要是被(束缚)。

所以你即使不看我刚刚讲的那个实际案例而只看理论的话,我们最新的这个经验就是3年前因为新冠流行所以美国当时的问题很大,然后天天大家宅在家里的时候呢又需要各式各样的电脑跟软件系统,所以他们的消费反而加上去了;消费上去以后就造成一个严重的通胀。但是你这个通胀危机的背后呢是供给跟需求完全不匹配。照理说,他那个时候根本就不在乎你的价值,比如说他们的厨房要器材的时候你都可以十倍的价格来卖。但是美国的奸商十倍、百倍的价值在外面,中国呢就给他们发展。(笑)他们的汇率反而压下去了。你这个汇率至少要应该趁机加50\% 甚至100\% 都可以,完全不会影响你的出口。但是呢,中国那些被洗脑的金融官员他仍然在担心有竞争——谁跟你竞争啊?那个时候只要有货能够发到美国他们就拼命的进,有多少发多少。那实际上是这个汇率牺牲中国来帮助美国渡过通胀危机。

-

*

-

97年之后,这个潮汐式的收割呢被所有的第三世界国家都深刻了解;但是它可以持久,为什么?因为它很奇怪的效应——就是大家的结论是,我们没有办法去威胁美国,我们能做的是什么?多储备外汇。因为大家请注意上一次被宰的那么惨,外汇越少的国家被宰的越惨。所以呢大家都是储备外汇的,储备外汇的关键是什么?美元啊。(笑)所以你越是被收割,大家对美元的需求跟依赖反而更多。为什么?你不要觉得这是一个反逻辑的结论,它其实是一个很常见的现象。只要学过Game Theory(博弈论)就知道,如果你的一方是完全分裂的时候,单个玩家的最优解常常是——你把整个一边看做一个集团——的最劣解。类似所谓的Prisoner's dilemma 囚徒困境。

就是基本上泰国、台湾、韩国和后来的中国都成为囚徒,因为他们对美国设下的这个Game——这个游戏来彼此竞争;所以美国越占便宜他越依赖美国。

那么照理说这么一来呢,这个潮汐式的定期收割可以无限的持续,而且越收割大家的依赖性越来越重。问题是进入了21世纪以后,美国人得意忘形,他们完全不顾美元会被取代的危险,开始做了一些很愚昧的事情。第一个是,司法部官员跟他们有关系的律师开始把美元制裁当成摇钱树——这个典型的案例就是《美国陷阱》这本书里的Alstom;Alstom是法国的一个重工业集团,他们在印尼做生意的时候呢付的贿赂大概是一两百万美元——这其实不是Alstom特别腐败,而是印尼特别腐败,一旦印尼要达到发电站的工程一定要会付这些费用;但是呢因为2001年911之后,他们以反恐为借口加强了对美元交易的追踪——所以这些事情都是很自动的就出现在财政部那边;那财政部呢就把这些转移到司法部,因为司法部已经有人注意到,既然有这个法律又有这些资料,你就可以起诉Alstom。美国有一个反国外腐败法,但是呢它针对的不是美国自己的企业,而是他们可以宰的外国肥羊。所以他抓到Alstom以后呢,最后的后果是什么?Alstom认赔了几十亿美元,然后把他最大的那个收益(能源业务)被强迫卖给了GE(通用电气)。卖给GE之后司法部才愿意和解——就是司法部的官员跟他的律师进去呢就拿美元;每个人都拿了几亿美元之后呢,然后再去跟精英收了一些Commission(佣金),这样子划分。

但是这个过程为什么闹大了呢?因为一开始Alstom不愿意自刎,结果呢美国就到处搜捕Alstom的高官,逮捕以后呢就严刑拷打啊什么的把他关到重刑罪就是美国的杀人犯的监狱里面去。美国人要严刑拷打,他们是民主自由的社会,所以不需要由监狱官来对你严刑拷打,他们只要把你丢到重刑犯监狱里面,自然有一些重刑犯来严刑拷打强制愿意。然后这位老兄在Alstom软服把生意卖给GE以后被放出来——真的是前一天签约,第二天这位老兄就放出来。你说这是司法奴隶吗?这是司法尊严吗?很明显的是人治嘛——抢劫的人治。

然后他出来以后就写了一本书叫《美国陷阱》——当然,这种书没有美国出版商会愿意谈,没有哪个出版商会愿意买;那你没有英文的话呢,华语界本来就没有注意到,而且我是刚好看到《经济学人》愿意引用,简单的谈到这件事情,...——2019年2月的时候;我看到的时候我知道说,这事情是非常非常的重要,值得要赶快给大陆备忘录,为什么?因为孟晚舟已经就要(被关了);所以我第一个就先找了《八方论坛》做了一期特别节目;然后接下去联络上上海《观察者网》的编辑,跟他们解释清楚之后呢,那个编辑跟他们的总编辑解释,那总编辑也理解到了这件事情的后果——当时他们认为是Trump要对中国施压的政治体现;实际上是他们立场的——你光直数他们敲诈超过10亿,到那次,到华为也是一个数字。受害者包括大众、德银、瑞信、中兴、华为这样子。那这件事情那个观察者网的总编辑他把原作者请到上海去;然后任正非还特别发了一个公关稿,那个公关稿就是他坐在书桌旁边,然后书桌上明明的摆着《美国陷阱》的中译本(笑)。所以我相信后来孟晚舟在夸大祸事,加拿大的法官一定也拿到了《美国陷阱》。当然这件事没有报道,我只是个人意见。



好,这是长臂管辖。另外一个在用美元霸权的事情是SWIFT。SWIFT其实是欧系的一个支付系统。那之所以要用SWIFT制裁呢,是因为我刚刚提到了,2001年(美国)开始反恐之后,对美元的掌控已经到了无孔不入的地步。所以有需要的国家像北韩、伊朗、俄国,他们的支付就必须要改用欧元;但是欧元的支付系统就是SWIFT——我再说一次,SWIFT其实是欧洲的一个微信的系统,它并不是一个以美元为主体的。而美国人要用美元去敲诈也不需要SWIFT。他之所以要搞SWIFT的制裁,原因是大家很自然就会改用欧元来走后门,来换回去,所以你必须要把这个后门给堵住。到了俄乌战争开始之后呢他们一开始就是先SWIFT制裁,没有效,因为俄国从2014年之后就开始全力准备这件事情;没有效之后呢就直接扣他们的外汇储备。你也许奇怪,外汇储备不是俄国中央银行的,美国人和欧洲人怎么扣?是这样的,你这些外汇储备只要还是外国的货品有定价,你就必须要存在——你如果是美元的话——就必须要存入美国中央银行,也就是美联储认证的银行的账户,然后这些银行再把这些钱存在中央银行——美联储手里;所以呢他们是这样扣的。俄国的中央银行并没有权力也没有义务去储存,除非你是印钞票,但是印钞票的话当然不可能。三千亿,哈哈...把它送进银行了。那他查扣资产的除了这些主权资产之外,连一些所谓的亿万富豪的资产也都扣下来了。那这下就摆明了你的美元资产随时可以被偷窃;所谓的资本主义自由市场讲,你拥有资产的权利不可侵犯,证明只是国际资本的权利,不是让老百姓分担。

-

*

所以美元不但成为主要的外交手段,也成为主要的战争手段。我刚刚提到俄乌战争一开始马上就不断升级的金融制裁。这是因为二战后第一次有先进工业化集团做正面的历史对抗;它也是现代历史中首次金融经贸战线为主,军事战线其实是欧美的金融经贸攻击失效之后,才不得不惩罚的战线。但是因为Nabiullina有8年来的准备,所以对于卢布的打击呢轻松化解。大家如果回忆一下,在一年半前俄乌战争刚刚发生的那半年,卢布对美元的汇率不降反升,这个就是他们准备充分,而且这个Nabiullina确实不愧是全球第一的中央银行行长。

而且在过去一年半,俄国达到了几乎百分之百的去美元化——就是现在他们的对外贸易之和呢美元所占的额在5\% 以下;这是全世界第一个在支付上面达到如此高的完整的去美元化,但是呢没有用,为什么没有用?货币除了在国内日常做支付的交易之外,在国际上做外汇,讨论外汇的时候,国际上货币都有三个主要的功能:第一个是交易,就是我刚刚讲的,交易你把它卖了去换美元的时候;第二个是定价;第三个是储备——因为你总是要有滞涨,还有刚刚我讲的,被美国用潮汐式收割的时候总是要有防线。

但是呢这里真正重要的是储备,因为你的通胀是否能够转嫁到国外去,你是否能够对外收割,其实是看你的储备——就是我在一开始的时候给大家看的介绍。然后这里面储备是最重要的,定价是最难做的。因为定价是有惯性的,如果大家都习惯了石油煤油美元定价的话,你即使是用卢布来定价,大家也是先自动的用美元然后再换算成卢布——这个价值是最坚固的。所以俄国一国的汇率,只能做到把支付兑换,一方面去美元化,其他两方面他没办法。不过至少美联储因为过去三年的通胀危机呢转而收割欧元英镑日元台币,而且呢他们通通非常热情的配合。请注意啊这里面只有人民币不在Golden Billion里面——美国的盟友都是受到军事跟外交政治还有间谍的历次胁迫,中国没有这些还是更热情似火似的,唯一的一个主动的。

-

*

-

我们再重复一下,为什么我刚刚讲这个俄国有这么厉害的中央银行行长,有8年的准备,还有Putin这种强势总统的支持和决心,结果呢只能做到贸易上的去美元。第一个是——我刚刚也提到——销量比别的东西高10倍20倍的话,价格自然上涨,自然就比较方便。当初Microsoft Windows视窗出来的时候,它真的是最好的那个作业系统吗?不是;但是呢,我要跟我的同事跟朋友交换打麦的时候,大家都是用的微软的垃圾,那你也只好用微软的,即使你知道它不是最好的。那么既然是有这个网络效应,你要用一个新的国际储备货币来替代它呢,这是一个鸡和蛋的问题。大家必须要继续用美元,因为没有替代品;而大家不用替代品货币是因为美元现在是主流——这就是一个鸡与蛋的循环。

然后第三个因素是,真正出现经济危机的时候包括2008年,大家想想看,一般人把它叫做次贷危机,这个次贷是哪里的次贷?美国的次贷,损失最重的是谁?欧洲。最后暴的最厉害的是德国的地方银行,还有希腊的中央银行。大家回想一下是不是这样?因为他们能够转嫁,这个转嫁呢除了美元之外还有美军。对,这是货币效应。所以你出了问题以后呢反而在先进工业国——我说GB是Golden Billion——发生了十几年,所有先进工业国;美国的经济反而是一枝独秀的。美国的成长率是略高于2\% ;德国的成长率是0;英国的成长率是0;日本的成长率是0。

然后最后就是有崇美忠狗的内部破坏,这只忠狗的名字我已经讲过很多次了。(群笑)

所以过去这三年实际发生的事情是,你应该集中力量而且抓紧时间,首先落井下石,让美国的通胀居高不下;然后选择两个或三个——中、俄和沙乌地;这些选择呢是第一个,他们有坚决的独立——他们是坚决独立的,你的第一个先决条件,能搞这个的是主权独立的国家;台湾是不是一个主权独立的国家?不是。主权是什么?主权说来说去最基本的就是追求国家利益为优先级。当美国可以指着你的鼻子说我要你的台积电,就必须要双手奉上,还要自己脱掉moleskin(鼹鼠皮),这个不叫主权,这叫做殖民地。欧洲国家有没有主权?没有;日本有没有主权?没有;韩国有没有主权?没有。他们第一步就没有资格参与建设新的国际储备货币。第二个,是必须要有财政跟贸易上的顺差或者至少接近顺差。你如果是习惯于大的逆差的赖皮的话是不会让你定国际储备货币的——谁会让你定,你一定拼命印钱然后歪理那一套;那这就排除了印度、巴西跟南非。

明明有这么简单的考虑,他们还是选择走金砖货币的路线,而不是我一再讲的建议,由中俄的一个很小的集团建一个新的一篮子货币——这很明显的是以退为进的阻挠危害美国的结果。我认为,他们知道如果站出来说我们不能够取代美元的话,他们不容于最高层的;所以就特别是说,我们要替代它,但是呢用最理想的把全世界第三世界都加入金砖货币——但事实上是不可能的。那过去三年另外一个发生的事情是,没办法去美元化但是呢大家都去欧元化。因为欧洲连主权独立国家都不是,大家要去欧元都是有理由的。我举一个例子:海湾国家,你知道他们全国唯一建立的所谓的国家财富也是这样的,他们卖石油所赚的那几千亿或几万亿,集中起来——其实新加坡也是做的很好;他们这些投资呢以往是多数在欧洲,少数在美国。从俄乌战争开始也就是一年半前开始,他们不但没有继续投资欧元——欧洲国家的——而且是在欧洲撤资;对美国的投资保持一点,其他的都到哪里去了?亚、非和中国。我们说的是几千亿。

至于俄国呢他们的去美元化是不是很顺利呢?也不是。有两个问题,一个是他只能够去贸易的美元;他不能够替代美元定价跟美元储备;而且最近俄国的卢布在去年升值之后呢最近这几个月也不是那么厉害。这有好几个因素,其实我也讨论过,在龙行天下也讨论过:正因为他们去美元化,所以他现在的这个汇率的竞争是针对中国,...;不过还有一个原因就是,因为他太急的去美元化,所以他在卖石油的时候连印度的卢比这种垃圾货币也照收;那你收了几百亿几千亿这种垃圾货币呢自然你的利息就不见了。但是这是政治性的目的。...

所以呢我们可以简单的预测未来2-3年,美联储已经成功的度过危机,事实上最近的那次危机的高峰是在两年前;而且因为一个周期是以5-10年的周期,所以至少未来2-3年呢不会有的——正因为欧盟跟日本韩国跟台湾都可以严重吸血,所以相对来说美国的经济会是这样,一个成功范例。

但是呢我们的人民银行行长换了,他们的国务院总理也换了;所以还会不会继续配合美联储的政策呢不好说。...

*

-

事实上,连对台政策也刚刚改了,大家不晓得有没有注意到。过去二三十年基本上是他们所谓的“跪台政策”——这个跪呢就是实惠的惠在上面,跪台政策——这个政策他们也改了。首先是零零星星的对台湾的进口的一些启动;然后接下来王沪宁在演讲的时候公布了,以后不再是国与国之间的对话,台湾没有资格跟中国对话,台湾是一个省会,所以以后跟福建省对话。这是一个根本性的改变。今天早上我跟唐湘龙讨论到这个,他认为这是一个战术性的就是争取国民党的选民转过来变成争取民进党的选民,因为福建当然是有降的感觉,这是话语的问题。我认为这个战术考虑一定是有的,但是呢还可能有一个战略考虑,这个战略考虑是什么?就是我们的对台政策不再文统不文统,而在于理顺统一之后的治理。就是统一是另外一个问题,不管文统武统;反正先搞清楚怎么样争取将来治理所需要的问题。我认为这很深,很高层;不过目前还没有证据是不是这样,给大家参考。

第二个要考虑的是,美国在过去15年的量化宽松8万亿美元呢到目前为止还有一小部分留在银行、家庭跟企业之内;你如果去计算他们的消耗的速度,大约就是2-3年左右就会用完。也就是说,届时美联储——不是说会发生一个危机,而是说——美联储必须要改变政策,不能够继续当前的量化短缩政策。那他被迫要转化,他被迫转化的时候就要看外部的顾忌,外部环境会有什么变化。

第三个就是美国的赤字已经进入指数生长期。4年前美国的财政赤字到达一年一万亿的时候呢,大家吓一跳,为什么?因为他们的GDP只有20几万亿——GDP哦,不是税收。但是今年的财政赤字是一样的;而且并没有像新冠期间的那种大的赈灾的支出。莫名其妙的,你看俄乌战争这些,日常的利益输送而导致了2万亿的赤字。



*

-

从中国的观点来看最近的这些例子,很明显的是,有人故意把印度包括进来,让这件事被搞砸。然后其实人民币本身也不合适。如果去看中国的“爱国人士”还有一大堆小的发那些言论:我们哪需要什么新的一篮子货币呢我们人民币就可以。这根本就是不懂,所谓知之为知之,不知为不知;那很不幸的,公共论坛99.9\% 都是不知为知的噪音。我跟大家解释一下为什么人民币不适合取代美元。

第一个,你如果是国际储备货币,你就必须要发行大量的债务。因为国际储备货币是不含储备的,储备的时候必须要是以长期债券的形式来存。这就鼓励你财政赤字——大幅的财政赤字;第二个,你的国家货币如果变成国际储备货币,就有无限多的金融手段包括我刚刚描述的1997年潮汐式的收割。金融手段的投资报酬率,绝对比制造业要高一个数量级以上。你让大资本来选择,一个几个月就可以收入几十个percent或者甚至几百个percent;另一个投入要10年20年,才有可能有50\% 的回报,你想他会选择吗?所以一个国家的货币如果成为国际储备货币,一定会去工业化,因为你的资本全部都会进金融;第三个理由是——中国的金融主管官员已经这么办了——在最高层的监视之下,他们还敢拼命的自由化,以推广人民币为借口来引狼入室,竞相的让西方投资来进入,然后还搞什么金融自贸区...你说在这种情况下你如果,唯一保障中国国内金融秩序的一道防线——一道马奇诺防线——就是人民币不能自由流通。这个时候你如果为了把人民币搞成国际储备货币让它自由流通的话,马上就会让整个中国国内的资产市场还有工业,都面临美国资本的掠夺。换句话说,美国在97年靠组合拳把南韩搞得放弃三星;中国呢却有一大堆官员拼命的想要和平的促进、助长这个未来。而如果那些小粉红真的如愿把人民币做成国际储备货币,第一件事就是必须要自由流通,自由流通的结果就是这些人,马上就把国家通通卖了。为什么?卖国家的时候他们就会是白手套了,就会是90年的那些俄国的未来一样。

*

-

好!我们现在把最新的历史复习一下然后讨论一下未来两三年的经济。我们来讨论一下变数,因为这个现实世界里面总是有一些无法预测的地方。

第一个是美国的通胀率呢虽然没有工会,但是因为他的基层的服务业有四十年工资没有上涨,所以必须要补;这个不足的原因就是为什么他被新冠挑起了,因为新冠造成的生命危险最危险的是什么?就是基层服务业。那些超级市场的收银员就是必须要面对面的,他们冒着生命危险来挡新冠;或者是屠宰场的员工,他们都那样聚集在一起,只要一个人咳嗽整个厂都会感染。这些人冒着生命危险,你说他们会愿意看着通胀上去而自己的工资不上去吗?而且这些底层的工作人员之中,积蓄最多的那5\% ,新冠一开始的时候就退休了。他们退休是为什么?就是为了避免感染的事。所以你的劳工群体呢一下子小了5\% ,你的失业率一下子就降了5\% ;失业率一次降5\% 的结果是什么?他们的劳工在要求工资的时候姿态架高了,不得了。所以虽然没有工会,美国的通胀一样固化。

另外一个因素,是美国在过去30年他们的每一个行业基本上都变成要统一了。你说现在美国的大哥大基本上只剩下少量的;那也许说在台湾还没有三家(笑)。但是以美国那种规模的市场,有严密的经济学证明至少要有四个是大公司——四大公司一起竞争,才不会有垄断性经营。他们的媒体30年前有300多家一级媒体,现在只剩下6家媒体。他们的整合,在过去30年为什么会这样?就是因为产业外移之后,留在自己国内的只剩下品牌跟行销和IT,这些东西都是非常容易整合的,你不再有生产的,比如说你以前生产一台机器,必须要,每一个零件都有它特殊的知识,每一个小公司都可以专注在一个零件的小知识上;一旦把这些统统换了以后,这些小公司就全部到了中国到了哪国...台湾为什么这么多企业——无良企业?那么为什么这么多中小企业?因为是制造业,制造业就会有中小企业。德国也是一大堆中小企业,日本也是一大堆中小企业——制造业帝国就会有一大堆中小企业。



好,所以讲了这么多重点是:美国的通胀有一些惯性,所以他的长期利率呢还在继续上升之中。那上升的话就会造成一个严重的——就是我刚刚说的恶性循环,不过这一次是对美国不利的利率;就是国债利率上升了,你欠的一屁股的国债,必须要还的利息一步步的上升,你的赤字就升高,可是你赤字升高之后呢,你国债扩大的速度就在升高;然后这样一来呢市场一看,你的这个国债所占GDP的比例现高120\% 多再到150\% 了,我这个国债会被赖债或者会被严重贬值,抵消价值的危险就越高,所以我要求厂的利率也越高,那这是一个大的核心。我认为这个大的核心循环已经开始了。我刚刚讲的这个美国的财政赤字已经进入指数增长,这个指数成长的过程就是我刚刚讲的盈利。



好,第二个是在这一轮美国的吸血之中呢损失最惨的——台湾只是台积电,还不是最惨的——最惨的是德国跟英国;德国已经完全去工业化了,不但是所有新建的工厂都找不到一个地址,连既有的工厂都要完了。德国即将面临一个福利问题,就是现在申请失业,申请底层救助的人,会有一个海啸式的来临。如果今年冬天还是像去年一样的暖冬的话(还好),如果今年冬天是一个寒冷的冬天的话,德国跟英国会死,几万几万的死。一个温暖的冬天在英国会冻死7000人。为什么?英国是个富人的社会,其实这些老牌资本主义社会都是极端的不平等,极端的扭曲。我说的这些数据都是英国人自己数的,10年周期,每年平均冻死7000人。冻死,是确定死因是因为...没办法,突然放弃了。



第三世界集团——BRICS集团虽然因为中国内部的破坏也没办法搞出替代货币,至少能够战术性的取代SWIFT和IMF。那这个呢应该是在未来两三年会实现的,可以保持住。然后呢能否替代的新国际货币,这个必须要等新任的总理跟新任的人民银行行长来处理;这个过程至少要18个月到24个月,要看他们的背后有没有一个智库,人才;这个人才必须要有等同于我的勇气,智商要高于我——我可以随便得罪人(笑)。这个人必须要在体制内,官场里面敢站出来说,我们就是必须要做一个篮子货币而且只找俄国跟沙乌地。那为什么说这些要勇气呢?因为这是在打李克强的脸、这是在打易纲的脸、这是在打周小川的脸。\twocolumn[\begin{@twocolumnfalse}
\section{大陸調研、巴以战争、美国经济}
\subsection{20231027}
\end{@twocolumnfalse}]Credit:Anonymous、上海网友S

 

唐湘龙 00:00

来,今天是我听王孟源上课的时间,很棒,这个是我每个月最开心的时间之一。



欢迎来到龙行天下,我是唐湘龙。今天10月27日龙行天下的时间,王孟源的时间。好虽然过去呢从我邀请王孟源成为龙行天下的固定来宾,当然感谢所有的观众朋友们对于龙行天下尤其对于王孟源的支持跟喜爱,表示我很有眼光。第二个就是在这个月,因为通常是每个月的第三个星期五,因为我要方便我们的听众观众去锁定我们的来宾在线上的时间,所以王孟源通常是第三个星期五。不过这个月我们稍微调整了一下。两个原因:一个是美国的大事,就是以色列跟巴基斯坦的这场的冲突,其实它的所有的空气,整个的舆论方向以及理解方式都改变了。再加上王孟源他现在人是在台湾的,但是他之前有说明过,他这次回来台湾除了看家人之外呢他到大陆去了一趟,他去大陆待了一个多礼拜的时间——在一些科技产业、金融领域里面做了一些比较深度而且广泛的接触跟调研;他有一些——照他跟我讲的——让他觉得大开眼界的一些事物。那最后有时间,我们再回头来看最近的一些国际经济情势。

来现在在我们线上的,人在台南的王孟源,欢迎。

 

王孟源 02:32

很高兴再跟大家聊一聊。很抱歉打破了你的节目的本命。这是我第一次去大陆,所以...很重要。



唐湘龙 02:42

王孟源现在人在台湾了,然后这几天我在跟王孟源联系的时候发现其实王孟源也会有累的时候。就是他在告诉我说他去大陆的时候,这次的调研他觉得他被操得非常的惨,996 啊一天24小时就这样子分配;第二个就是因为他很累的,所以我说好,那就在台南家里休息,我们连线就好,虽然在台湾了。那上次因为孟源来到节目现场的时候,我觉得我们的观众非常的疯狂,盛况空前。好但是呢在台南尤其刚从大陆调研回来,尤其有关于这一次的中东情势所引爆的全球的一个——我认为内部有一些全新的脉络跟全新的...也就是国际的新的太阳表现出来,这是到目前为止如果只关注在以巴冲突的那个表面看不到的,这些都是今天邀请孟源为大家解读的。好我们今天先从孟源刚提到的,诶你刚刚说你是第一次进大陆?

 

王孟源 03:54

是。我以前已经在上海转机过一次,但是不是真正的进了海关,这次是第一次的到大陆。

 

唐湘龙04:02

这恐怕也让很多人很讶异,因为可能在大陆已经很多人都认识王孟源,你的微博在大陆的忠诚度是非常高的,所以孟源第一次到。好吧,我们先讲讲,就是说为什么会有这一次的第一次,然后你为什么一直在告诉我说你很累,但是你大开眼界,为什么?

 

王孟源 04:25

是这样的,我过去这几年跟深圳哈工大的一个研究院,也就是研究所阶级的...(唐:深圳,但是哈工大的,哈尔滨工大的。)深圳有本地的大学,深圳大学当然是他们最主要的学府,但是他们还有一个特别的外来学区,有三个外来的985学校,研究所;就是哈工大、北大跟京大。那哈工大里面有一个特别的经济导向的研究院,其实跟林毅夫也有点关系,是林毅夫所立的研究院同盟里面的一个成员。那他们三年前刚成立的时候就跟我联系,后来我才知道他们的首任院长——现在还是院长——是我博客的粉丝,所以他一旦出任以后,(笑)然后找我当他们的客座。那我跟他们在网上这样子交流了三年了。那这一次因为来回来台湾超过两个月,是我今年回台湾最长的一次。那刚好他们的季度会议——他们每一季开一个座谈会,我以前都是用远程的视频参加的——那双方讨论一下就说我是不是顺便过去一个礼拜,就安排了这个礼拜。我事先也没有想到说会是一个很正式的拜访,就是觉得到了那边(和)其他的研究员坐下来聊聊天,因为他说有一个办公室我可以用。那我还是循着美国学术界的那种习惯而已。那是我的经历...

 

唐湘龙 06:16

哇,他们真的是盛情款待啊,非常的重视你啊。

 

2020年11月22日,深圳高质量发展与新结构研究院在哈尔滨工业大学(深圳)挂牌成立。

右2起:首任院长 吴德林;林毅夫。

 

王孟源 06:20

我想的是上午跟下午到那边,然后坐在办公室,然后就会有人过来聊天,如果我无聊了我就过去找人聊天这样子啊。结果完全不是这么回事,他把那个行程安排得非常的满。我刚跟你私下解释过,他有安排了三种不同的节目,第一个就是到深圳当地的高科技产业去调研。那当然他们的调研是很常见的,除了学者以外,真正的主流是各式各样的领导。所以我到那边的时候接待员第一个反应就是:这位领导请。吓得我...哈哈哈哈。

 

唐湘龙 07:01

哈哈哈,就是把你当领导就是了。因为我光听你讲就很兴奋,我相信你一定看到了很多我看不到的东西。好,你继续继续。

  

王孟源 07:14

噢,这里我看到的比如说华大基因;(唐:噢OK,现在是大陆基因工程的第一名。)对,然后大亚湾的核电站也去;(唐:大亚湾我去过。对,那个核电站我去过。)东莞河的起源点就是当时因为要建大亚湾,所以后来才成立了中广核,所以很重要;因为我对核能发展也是很有兴趣,以前写了一篇博文讨论,所以很高兴的去看了历史介绍;然后看了比亚迪。不过有一个很遗憾的是没有办法去看华为,因为刚好金砖会议也是在上个礼拜举行,所以华为在那个礼拜就不接受地方性预约的要求,这是有点可惜的。那这是第一类;第二类就是我原本跟他建议的,就是我跟当地的研究员或者是来拜访的外地的研究员——基本上就是教授了,经济学的教授或者是其他社科类的教授——这样子坐下来聊聊天;有十几场我觉得收获很大,这其实是我这次去收获最大的一点。那另外一个节目就是它中餐跟午餐基本上都排满了,都是正式的宴会。(唐:哈哈哈哈,这大概对你很煎熬啦。)基本上就是这位院长,他的比较私交比较好的一些同僚跟朋友。(唐:噢~很热情的推荐王孟源,让他们认识一下这样。)其实里面有好几个是我博客的老读者,所以...(唐:OK,真的是卧窝卧虎藏龙,对,这是很棒的感觉。)我博客跟我现在这个演讲,其实目标受众是教授级的,至少要是相关专业的研究所研究生,才能更获得一些大部分我要讲的意思,所以这次去就是十几二十几个这样的教授级的人员也不断的聊天,我觉得是很好的。那...当然也胖回去了哈,这一块。

 

唐湘龙 09:37

哈哈哈,我有觉得脸变圆了一点。不过去大陆,包括我的其他的来宾像苑举正,他只要去大陆回来,一定胖一些。

 

王孟源 09:51

诶嘿,因为他们安排的很紧很忙,所以就没有离开深圳;然后我对深圳的印象是很深刻,觉得大部分的印象是很正面的,就是我不晓得你对深圳熟不熟,但是它的面积基本上跟台北加新北差不多,还稍微小一点;然后它的人口却是台北跟新北加起来的2.5倍;然后你再考虑到它的绿化跟水土保持做得很好,我在外面看到的那个植被率可以跟美国最好的地方相比,比台湾要好很多。那你可以想象这里的人口密度就很高,他们那里的那个楼通常都是几十层的,就是你想象——不是像台北市中心那样,但是整个台北加新北这样的地区有几十个类似台北市中心那样的高楼大厦林立的地方;然后即使是他们的比较平价的住宅区,就是他们所谓的城中村,里面也都是八九层以上的高楼,所以你可以想象是人口密度很高的。



我后来去看了一下他的GDP是大概4500亿美元,这个要比台北加新北也是高1.5倍。所以你可以说他们的人均GDP是跟台北加新北差不多的。那这个原因当然就是因为它集中了整个大陆的精华。我去那边不到两天我就彻底了解了,我去的不是一个典型的中国城市;其实刚好相反,是最不典型的中国城市,是一个最极端的。因为在那里见了两三百个人,只有两个人有粤语的口音,就是当地人。哈哈哈(唐:诶,这个观察也很有意思。)其他的是外省人,所以它是一个典型的移民城市,就像100年前的美国的底特律,还有洛杉矶这样子——都是外来的精英分子。



然后,他们的文化也是最不官僚的。因为30年以来它原本就是邓小平专门来做实验特区的,所以你可以看到他们的那些管理人员,都是最有冲劲的,最开明的,水准最高的。比如说有一些——我这次去的那个地主,那个院长,他本身是学法律的,然后后来曾经在市政府里面做了很多年的事;但是我看他...我们去调研的时候,那些都是高科技产业,他一个一个对里面的技术细节都非常的熟悉,都可以——我原本以为我是理工科出身,他是文科出身,然后我可以(笑)帮他解释一些基本的问题,这根本都不用!他对那个细节的有些地方知道的比我还多。所以你可以想象这些都是,别说是一个国家级的精英,就是对国际企业来说都是精英的人才。



那我在那边得到的一个印象就是,他们每一个人都跟我抱怨说最近这几年大陆的国家对尤其经济方面、产业发展方面呢管的特别的严,常常是一刀切的现象。他们每一个人都这样讲,后来我就把这两件事连在一起想了一想,觉得:可是你们这些深圳的官员会有这样的抱怨,是因为你们本来就是一个Special,本来就是一个特例;所以这些规矩对你们来说是阻碍,但是呢对其他的地方可能是(笑)完全适用的。因为那边的官僚,其他地方的官僚很可能水准比他们低一节两节,所以真的有必要避免他们这些官僚——不想做事只想鬼混的官僚——要有一些规矩。那原本深圳是所谓的特区,一直到最近这几年有什么大的国家政策呢,名义上深圳也还是可以作为例外;但是呢他们私下跟我的抱怨是,就是名义上是例外,但是实际上呢,国务院还是会越俎代庖的干涉他们的细节决定。所以我觉得这是唯一应该改进的一点,就是深圳因为有历史传承,有它的特殊的产业跟人口性,还有一批非常精英的地方官员;国务院应该真正的放权,不要只想着那种官僚保护自己的turf,就是自己的领土的那种官僚心态;这是一个小抱怨,其他的话我觉得都做得很好。

比如我再举一个实际的例子,就是我刚到的时候,刚下飞机,他们进海关的时候呢,在通道的边边有一排二十几台机器,说外籍人士采取指纹。那我跟另外一位女士在那边辅导了半天,结果不管换哪一个机器都没办法,它都是宕机。那我就觉得很奇怪,第一个我不知道采指纹是要干什么,为什么只有外籍人士要采指纹?然后第二个是我后来鼓捣了半年,到最后我注意到每一台机器的角落都有一个统计数字,就是今天成功处理多少人,然后处理失败有多少次。因为我看了一下,平均每一台机器的成功率,成功次数都是0,(笑)然后失败次数都在1000上下。所以你知道这二十几台机器就是摆在那边折磨我们这些非公民的(笑)。但我就觉得很奇怪哦,然后再进去以后他们也没有解释这个采指纹是干嘛的,我只有后来一个多礼拜后出境才知道,如果是入境的时候采了指纹,出境的时候有一个关卡可以走快速通道,就是已经采了指纹就不必再面谈,可以直接电子扫描的出境。但是他也不讲清楚,然后入境的时候失败了他也不解释,事后也没有再问你的。



我就觉得很奇怪,这是一个很丢脸的事情,我一开始第一印象其实是不好的,就是因为这样子。然后来接下去的第二印象呢是他们的海关就跟我说扫你的微信,扫你微信以后就可以快速的通第二关。那可是我没有微信啊?这不是说我不知道应该装微信,我在台湾的时候要装微信,但是因为他们最近要防治网络诈骗,所以一堆像银行卡啦或者是微信...(唐:没错没错,他们的限制非常多哦。)诶限制非常的严;那我在台湾要装微信的时候,在手机装微信的时候装到一半他就跟我说“需要一个既有的用户来作保证”对不对,我在台南家里,你到哪里去找保证啊哈哈哈...(唐拍胸:你可以找我啊?你可以找我啊?)哈哈...那天他讲不要装回去,然后在那边就没办法过。但是——我的习惯是你这种自动化可以,但你一定要有一个兜底的办法——所以我想着你就到了那边它有微信快速通关,另外的一个人工的处理兜底,结果没有,结果他们把我踢回来,他们说你没有微信,好,到前一关。我到前一关,那两个小姐呢就一副我在找她们麻烦的样子。说可能没有微信,基本上态度就是不是要帮忙,而是看热闹。所以我跟她们解释了半天,还好我会讲中文,我如果不会讲中文的话就惨了!哈哈哈哈。



这两个最早的印象都不太好,可是我事后发现其实不能够怪到深圳地方政府本身,因为他们这些可能是中央政府管,就是海关管。那那个指纹采集机器呢,我四年前曾经在上海虹桥机场过境的时候也遇到那些机器,那些机器在我过境的那一天也同样的宕机。所以你说是不是很好玩?我就怀疑这是不是一个全国性的贪腐案件,它那个工程范围让各个机场在那边收拾残局——不过我不知道细节的,反正就是很奇怪就是了。我出了机场以后,把这件事情跟我的地主解释过以后呢,他就发了一通微信。真的...在大陆没有微信活不下去。我还跟我的小孩说这个叫one APP for——就是用那个《指环王》嘛,你知道《指环王》那部电影里面有One Ring to rule all,一只魔戒来统治世界;我觉得这是一个应用程式来统治整个国家。(笑)

 

我跟那个地主解释这些以后他发了一通微信去给他认识的一个朋友,那个朋友呢认识机场的主管,结果第二天那个主管就回信说,很抱歉我们会用心的去处理这件事情——就是态度很脚踏实地,而且没有那种官员的傲慢的心态;所以从那个时候我就扭转了,我理解到深圳这个地方政府真的在各方各面都不是典型的中国城市,因为我没有看到一点官僚的气息——在我待的一个多礼拜没有看到官僚的气息;那真正看到官僚气息的就是刚入境的时候遇到要跟中央政府单位打交道的时候。



那另外一个我也想观察的是人民的文化水准,就是像我其实也是最近这几年回台湾才会到台北去,那时候我刚——前几年2015年第一次真正到台北去的时候,印象很深,就是台北的人民的文化跟台南就不太一样,怎么说呢,就是交通就很守规矩,大家自动排队,不会...

 

唐湘龙 21:49

哈哈哈哈哈哈...当我在听王孟源在讲这个话的时候,我就觉得特别好笑,因为他是个土包子进城一样,你对台北都有这种观察。

 

王孟源 21:59

我就是一个台南乡下的土包子。(唐:哈哈哈哈...好你继续。)2015年是我第一次真正在成年之后在台北逛(唐:哈哈,2015年诶!)啊是,然后就觉得很奇怪,因为我连在美国都没有看到这种现象,比如说:在台北的话,你如果行人要过马路的话呢,遇到一个红绿灯,绝大多数会停下来等红灯,其实没有,远远的左右一辆车都看不见,绝大多数的人会站在人行道上面等红灯——这个在美国是没有看到的,据我所知只有德国人会这样子。啊哈,我在美国、到英国,啊更不用提意大利了,都不用,都不是这样子。然后我后来就排了一个心目中的开车的那种态度,就是开车的人那种aggressiveness,就是侵略性——台北是最低的;然后台南其次,台南基本上跟美国的郊区差不多;然后纽约又高一级,纽约的那个交通很乱;那当然最乱的是——我去过的国家里是意大利,那个简直是疯狂,能用疯狂来解释。



我亲自到深圳呢觉得他们的开车驾驶的侵略性大约是比台南略高,也就是比美国的郊区要略高一点,但是呢远远没有到纽约的层次,所以...(唐:不错不错。)我觉得是可以接受。



然后他们那边排队基本上也没有什么人插队,最严重的两次插队是我要出境的时候,在排队的时候有,因为我前面的是一位——都是老太太,一连发生了两次——老太太的话呢就有的时候停下来不走了,然后就有人插到那个老太太的前面去。我的第一个反应是怎么会这样子?一个礼拜都没有看到这种插队的情形,然后回来的时候才一连遇到两次;后来我才注意到这两个人都是台湾人。(唐:哈哈哈哈哈哈哈...OK,好。)哈哈哈哈哈...我觉得不一定是台湾人习惯插队,而是他们觉得到了大陆就可以插队了,我觉得这心态是不太好的。这个橘逾淮为枳(唐:橘逾淮而为枳啊,嗯。)哈哈...不过,整体来说他们就是比较有侵略性。然后你刚刚提到的996这种事情在台湾大概是没有;我自己的经验是,我觉得你年轻,二三十岁的时候做个三五年的996可以,我自己也做过。我刚从物理转金融的那两三年我是自愿的做996——因为在美国没有人做996——我自愿做996是因为我刚刚转行,所以我什么都不懂,我对经济跟金融都不懂,所以我必须要花很多的时间在办公室学习,然后真正的实际的把工作按时做好;那后来慢慢的就变成我在家里面读论文,两三年之后就不再真正是坐在办公室。但是呢像我现在这把年纪哦快要60岁了,还要搞996的话两三天就很累,所以我这次回来以后,休养了两天,啊哈;然后再——因为我日常阅读的份额呢,大部分90\% 是英文的媒体,然后我自己每天做一到两个小时的数学题,挑战自己的逻辑分析能力;这些也都是外网的网站,所以我在大陆的那一个多礼拜呢就都没有办法做,所以出来以后又补了几天,所以这几天其实也很忙的,然后还要跟亲戚朋友见见面。那所以很抱歉,今天就没有办法讨论的太多。

 

唐湘龙 26:23

不会不会,好,因为王孟源是一个生活很有纪律感的人,我一直怀疑王孟源是我常常访问到的那种很聪明的,那种高功能的,那种有雅思症的症状的人。好,但是你这次去深圳,我们从一个大国的竞争的角度来讲,你看的是深圳,而且九天的时间排的这么的密集,你对于中国大陆未来的总体的——这个问题很笼统——就是从一个大国竞争的角度来讲,你对于中国当下的现状跟未来的竞争有什么看法,有信心吗?

 

王孟源 27:06

持平偏上,就是有若干信心,但是他们的问题很大。就是一个这么大的官僚体系,一定会有无数多的问题。然后里面一定有很多贪腐的例子。那这里的正面的观察是,他们不但有意愿处理——他们的体系有意愿而且有能力处理,而且事实上是很积极的在处理这些事情;我最担心的是他们在政策的大方向上面搞错了方向。对,那现在是有这个趋势。那当前严重的国际政策大方向的疾病有两个。那这刚好是我这次去的时候观察而且讲学最重要的重点。



我先讲一下,我大概去年6月的时候写了一篇文章,登在我的博客,叫做《社会主义国家应该如何管理资本》。那其实是...(唐击掌:好问题。这是我最近最常在想的问题。好,不过你继续说,因为我纯粹只是想而已。)那篇文章呢其实是我博客写了这么多年的一个总结性论文,层次很高,是一个把以前讨论过几百次甚至上千次的议题全部集中起来,成立一个系统。不过呢它还没有达到最高的层次。这里就是因为它只谈经济这一个方面的问题;就是它解答了所有现代社会里面应该怎么管理经济的方方面面的考虑;但是它没有考虑像政治、文化、军事、战略、金融、科技管理、外交之类的这些议题。我那篇文章写完之后,就想要再提高一个层次,来把所有的这些博客谈过的,当前的国际跟国内的治理问题全部统筹起来,做一个总结性的讨论。因为事实上我博客写了1000万字——如果包括留言栏的话有1000万字——讨论的议题非常多,非常全面;我一直希望能够做一个很好的总结。但是也许是目标定的太高了,我从去年年中这篇想要写的文章就一直憋在心里,然后每隔一段时间就拿出来想一想,可是总觉得欠缺了一些什么东西,没有办法真正的下笔。



然后经过这8天在深圳的调研跟聊天呢,我终于灵机一动,就是觉得那个阻碍松开了,我可以写了。事实上我回来以后也的确写了一半,所以大概下个礼拜能够成文。(唐:好!OK)



这里最主要的有一点,就是关于你刚刚所谈的这个问题,就是他们目前的这个治理的问题。我觉得他们现在的治理有两个很大的问题,一个问题是习近平已经了解正确方向,另外一个是习近平还不了解正确方向的。



这个第一个是外交、战略、军事跟国际斗争的问题。就是我在那边的时候注意到,凡是60岁以上的官员全都是无条件亲美。他们完全都不懂当前俄乌战争所带来的这个反殖民风潮,以及我们目前所面临的这个国际局面是什么样的斗争,它这个斗争的本质是反殖民起义。但是呢这些60岁以上的人都不懂,他们依旧是保留了二三十年前那种“美国就是老大,我们应该把握每一个机会讨好它”这样的态度。举个例子,俄乌战争他们就认为,这个是中国站队表明自己支持美国那个所谓“基于规则的国际秩序”的好机会,所以要趁机的站到美国那边去打击俄国,那现在这个以巴,那他们就认为这是用反恐来跟美国站队,表现自己忠心的一个好机会。其实过去几年他们也一直说管理碳排放这件事情也是跟美国合作,表达自己忠心的一个好机会,和解的一个机会;他们最喜欢讲的就是合作、友好,甚至用和解这样的字眼。他们完全不了解,这个其实是——目前的真正的冲突是美国殖民帝国控制了全球,美国是全世界第一个掌控全球所有国家的殖民帝国,就是统一的,相对于以前的战国时代;以前的早期的殖民帝国是一种战国的状态,那美国就是这个殖民时代的秦帝国,他第一次统一了天下。那统一天下以后,这一次的中国的崛起是第一次挑战了这个全球集权的殖民帝国中心。



而俄国呢在乌克兰打成这样子,其实就是陈胜吴广的起义。他们不了解这一点,而且我可以确定不只是我所见的那几个官员是这样子——这在中国是非常普遍的现象,就是有一个很高层的官员,他已经退休了,我们聊起这个以巴冲突的时候,他就说:我有十几个微信的群——他们也有那个朋友的群——他有十几个群,这十几个群里面除了两个支持巴勒斯坦之外,另外的全都支持以色列。...

 

唐湘龙 34:02

哈哈哈哈哈哈...原来我在这些群体里面是这么的孤单。好,对,因为孟源谈到这个地方呢,我们就开始进到下一个主题。顺着孟源刚刚的谈下来,其实我今天跟孟源谈的时候,除了他这一次在大陆的调研——因为第一次到深圳,同时这么深度的去深圳呆了这 9 天的时间,我可以感觉到王孟源他为什么跟我说他大开眼界,除了第一次的那种感受,许多东西都新鲜之外,我认为王孟源也以他所接触到的层面——他碰触到中国这个基本上非常不传统的新城市里面汇聚了中国的大量精英所反映出来的这个城市,以及从深圳去看中国的未来的总体的味道,那个感受是很强烈的。但是另外的一部分其实全世界现在都在高度关注,甚至于因为这个议题越来越分裂了,你看连安理会都分裂了,联合国都分裂了,国际舆论都分裂了,就是对这一次的以巴的冲突。好,那我们顺着你刚刚谈的这件事情讲,这个我倒是很意外的,因为我以为在大陆的舆论尤其大陆的主流官方,你看得出来有——包括在这两天在联合国,包括王毅的讲话——其实挺巴勒斯坦或者说同情巴勒斯坦,同情哈马斯的那个味道是很蛮浓厚的,甚至不惜跟以色列在联合国里面直接对上,这在过去中国是没有;中国跟以色列关系一直都保持很暧昧的。就基本上中国即使有所谓两国方案,但是并没有开罪于以色列,可是最近这几天就不是啊。因此孟源刚刚提到就是说,你在大陆跟朋友的这些互动的过程当中,他们的群族里面大部分都是挺以色列的,这让我很意外。

好,那我们如何去理解这件事情以及以巴的冲突?到底我们该如何去解读?

 

王孟源 36:11

我刚刚提到我认为中国目前面临两个很大的挑战,其中一个挑战他们已经知道正确的方向,这个就是国际战略跟外交方面,习近平已经理解到——其实他大概在三四年前就理解到——这个当前的潮流是反殖民的潮流,所以真正的问题是中央跟地方政府,尤其是中央政府,我接触的都是只是地方官员,但是我看到一个铁律就是 60 岁以上的人都是无限...(唐:无条件亲美派。)50岁以下的都是无条件支持中国自己独立;那 50 岁到 60 岁之间可能统计资料不太确定,但是一定有一个过程嘛。(唐笑:那就刚好是我们这年纪的人。哈哈哈)哈哈,所以原本我的博客在过去两年对人民银行有非常严厉的批评。



但是你想想看,人民银行的前一任行长跟前前一任行长就是易纲跟周小川,他们都是在60岁以上的人,所以你可以想象他们为什么会做出我所批评的——我用的那个词汇非常的尖锐,就是卖国。但是呢,问题是像金融这种事,它一方面是国际战略上的焦点,一方面也对国内的经济发展有很大的影响,所以他们在国际战略上所做的保护美国的经济,然后不让它通胀——我刚刚找到一张新的统计资料可以证明,的确我所说的事情,就是我过去两年所讨论的,是一个国家所有进口货物的平均价格,它在过去这五年的变动;你可以看到它比较的是美国跟欧洲的三个主要经济体,就是美国,英国跟另外一个次要的国家——这个我会登在我的博客——但是重点是在美国通胀最严重的也就是两年前跟一年半之前,美国的进口物品的价格提升了10\% ,但是德国的进口价格提升了40\% ——其实它超过45\% ;这就是欧盟中央银行跟中国的中央银行配合美联储货币策略,当时他们故意贬值让美元升值,他们保护的就是进口货物的平均价格。



所以美国的通胀完全就只来自他自己国内的能源、自己国内的工资还有自己国内的供给链的问题,而不需要顾及进口货品的通胀——他们的进口货品所贡献的通胀微不足道;相反的因为这个货币外汇汇率变动的关系,这些进出口货品的通胀就被全世界其他国家承担了。



欧盟按德国的进口价格平均上涨了百分之四十几,英国上涨了35\% 。那这些就是其他的主要经济体自愿替美国分担、承受了通胀压力——这在战略上是一个很大的失策。那我大概在十几分钟前讲到,我在这次去大陆心里有想的两个很重要的决策方向,一个习近平已经了解正确方向了,就是国际战略上,他已经了解这是一个反殖民的斗争;另外一个他们现在还没有理解的,也是我现在在写的那篇新博文的真正的核心、讨论的重点,就是国际新外交环境,也就是美国这个殖民帝国在进入瓦解的过程中,中国对国内宏观经济的管理必须要做出一个调整,这个调整就是——这里面的论证很复杂,请大家等我的博文出来以后再仔细的讨论;那我把结论直接丢出来,这个结论就是你不能够再期望有8\% 到10\% 的年GDP成长率,你必须要接受4\% 到5\% 的年成长率是长期的一个合理的、持续的一个层次;所以你的管理呢不能够再像以前那样的浪费跟激进。

 -

【选段,2023/11/20】

過去三年,世界處於多事之秋。2008 年以後美元的巨額超發,原本就讓新一場危機注定成為重複1970 年代的通脹模式,其嚴重性卻在新冠疫情突然發生之後被意外放大,從而在起自 2021 年的一年多期間内,賦予不願意也不可能接受美國殖民體系底層地位的兩大獨立工業國(亦即中俄)一個史無前例的短暫反抗窗口,可以通過對美國通脹危機名正言順、合法合規的自我保護手段而加速後者的衰敗。這個國際反殖民革命的天賜良機,來自美國在新冠疫情衝擊下對國内消費所施加的三萬多億美元緊急財政刺激,自然造成所有消費品都嚴重供不應求的混亂局面。原本人民銀行即使無視國家自身利益和友邦戰略需求,只盲目依循美式貨幣理論,也應該讓人民幣大幅升值,但美國政府出於扼制通脹的需要,原地扭轉前幾年對中方“人爲操弄壓低匯率”的指控和制裁,反過來私下要求人民幣和歐元緊急貶值,而這兩個中央銀行居然都乖乖聽命,從 2021 五月至 2022 九月,美元得以對人民幣升值 14.7\%、對歐元升值 23.0\%,導致進口物價成爲幫助美聯儲渡過這一輪通脹危機的主要助力(參見下圖所示的進口物價演化過程,對應美國的紅褐色曲綫顯然占了大便宜)。

——《金融史觀(一)歷史由來》

【後註五,2024/03/05】

李强總理在全國人大做政府工作報告時,定下“5\%左右”為2024年的GDP成長目標(參見《政府工作报告极简版来了!只有700字》);雖然不是博客所建議的4.8\%那般明確地以長遠深刻發展為重,但亦相差不遠,“左右”兩字尤其值得玩味。

——《金融史觀(三)政策建議》

-



我一直觉得美国的体系,他的管理——这包括他们的政府,也包括他们的企业——其实是非常混乱的,低效的。我在美国35年所看到的他们解决问题的办法就是砸钱。你事先没有预案,没有准备;事情出来以后手忙脚乱,然后呢解决的问题就是砸钱,砸到问题被解决为止。中国其实也学了美国的这一套,那因为现在国际情势的演变,我觉得已经到了一个历史的转折点,你不能够再——没有再那样丰富的资本,也就是货币超发来让你到处乱丢;不过这个提的太远了,我们回过来谈以色列跟巴勒斯坦之间你刚刚提的问题。



我最近这几年一直在提,这个世界在过去这 500 年是一个殖民帝国的时代。也就是500年前欧洲人开始大航海以后,随即进入大殖民,然后就统治了全球——欧洲原本是欧亚大陆的一个很不重要的小角落,一个相对资源富裕的半岛,但是人口跟科技还有组织力都不是特别的突出;但是在殖民帝国开始以后,他们就从全世界掠夺了无限的财富。最早一波是西班牙从中南美洲抢了一大堆金银,那时候他们一年掠夺的黄金最高曾经达到他们GDP的 10 倍——不是 10\% 哦,而是 10 倍——你想想看,你的 GDP 如果得到百分之千的刺激的话,那种发展会是什么样子?



那么你如果纵观过去这 500 年的殖民历史,其实可以分为三种形态。



第一个世代的形态就是杀人夺地,或者我们说种族灭绝——难听一点就是种族灭绝,他们欧美的学者自己谈到这一点的时候呢,当然,连杀人夺地这种中性的描述都不接受,更不要提种族灭绝;他们就说是人口替代。呵呵,这个人口替代最典型的是什么呢?就是北美洲被整个杀光,印第安人被整个杀光。(唐:印第安人。原住民几乎被杀光了。)诶,现在他们占到占北美人口的不到2\% 了。另外一个就是澳洲,他们现在原住民的血统占不到整个人口的5\% 。那这两个是最惨烈的,但是呢它其实是西班牙人在南美洲最早开始,但是西班牙人没有做的像昂萨那么绝,所以整体来说中南美洲的话,它原住民的基因还是远远超过50\% 的;就是他们并没有被种族灭绝——我觉得可以称为种族灭绝,而且事实上是人类历史上最大的三场种族灭绝的两场就是美国跟澳洲。

那到了19世纪以后它进入第二个世代,这个时候典型的就是像英国在印度,他们是用少数的官僚然后通过代理人建立机构,通过当地的代理人来管理既有的人口。这是第二个世代。



第三个世代就是在20世纪二战之后美国新开发的,是一个软性的、隐性的、间接的,主要利用宣传来控制,军事只是兜底,然后金融来负责收割的一个手段;就是它不像以前直接是把财政收的税收呢往本土去送,而是容许美国的企业跟对冲基金还有银行用金融的手段来收割当地。这是我所谓的殖民帝国的最高形态,也就是第三世代。



那以色列最特别的地方,就在于他是第三世代的殖民帝国在那一段时间成立的第一世代殖民帝国;他是人类最后一个第一世代殖民帝国。然后呢与此同时他有国际性的犹太裔的普遍支持,而犹太裔就是专门负责美国全球殖民帝国中三要素——我刚刚提到三要素就是宣传、金融跟军事——宣传跟金融都是犹太人。OK,所以你可以看到以色列的特点在于:第一个,他占了全球第三世代殖民帝国——也就是全球统一的这个殖民帝国中的三分之二;然后他还是一个区域性的,最后的,比别的第一世代殖民帝国晚了两三百年的第一世代殖民帝国,因为他真的就是杀人夺地。你看看现在以色列的正式的领土,比1946年他们刚开始的领土大了将近10倍,扩张了不知道多少次;然后在绝大多数的地方,他们就是积极的殖民,就是把原本在国外的犹太人引进来以后抢占了农场,然后慢慢的把巴勒斯坦人驱逐出城市关到一个集中营,就是加萨还有约旦和锡安两个大集中营。那因为这个原因,所以以色列是这一次反殖民起义中必然的一个目标,也是HAMAS这个组织为什么会趁这个时机出手。



因为我们看看这个,从去年年初俄国作为陈胜武广起义之后——就是反殖民起义之后——就全球响应;我在三四个月之前曾经谈到Niger尼日尔他们的那个政变(唐:对,那也是西非洲的反殖民主义。)对,也是高举反殖民大旗。像这个西非洲的军事政变平均每年一次,这一次有什么特别?就在于他们高调的谈反殖民,而且高调的以完全驱除法国的存在为目标。而且他们很简单就可以举出实例,Niger在国际上最重要的出口是它的铀矿——就是核能的那个铀矿;(唐:对,铀矿。)他们因为被法国用当做一个殖民地来管理了,他们卖给法国核能公司的那个铀矿价格呢是80欧分一公斤。国际上合理的价格——就是那种品位的油矿;合理的价格是200欧元一公斤。(唐:天哪,真的是...那根本就是掠夺啊!真的是...)



也就是说他们拿到的价格是0.4\% 。合不合理?(唐:掠夺,可怕。80欧分哪,一公斤的铀矿!)80欧分一公斤的铀矿。



所以你看到了,他们这次军事政变的那个领导人只要把这件事拿出来,就很简单的得到国内跟附近非洲国家的支持,因为大家都懂——大家都习惯,但是都懂;然后呢他刚好的利用这个分化了美国跟法国,就是现在法国已经吃瘪了,已经认输了,已经把他吓死跟军队都要撤出去了;那同时美军在那里有几百人,Niger呢就——那个军事政变的领导人就说,我同意让美军无限期的留下来。这就很好的分化了他们的殖民的老爷——就是法国,跟他这个背后最大的支持也就是美国,这也是他们成功的原因。



那巴勒斯坦这一次HAMAS出手完全是出于同样的考虑,就是因为这个国际大环境已经适合了。所以他们也是每隔三五年就会闹一次,但是这一次是过去十几年闹得最凶的一次,它的规模最大。OK,那规模最大并不是因为它背后有伊朗支持,那个都是主流媒体宣传的胡扯;因为你只要看伊朗真正关系最密切的是黎巴嫩的Hezbollah(真主党),对不对?这才是他们嫡系的力量;如果这后面是伊朗主导的话呢,Hezbollah一定会同时出手,而且出手力道会比HAMAS还要大。可是你看这一次的事变呢,Hezbollah是在这件事件已经基本——第一波已经打完了之后才匆匆忙忙的上前线,然后摆出姿态,也没有真正全力的出手,只是摆出声援的姿态,很明显就是一副事后反应的样子。从这一点就可以看出,真正幕后主导的就是HAMAS自己。这也是为什么他能够躲过美国跟以色列的情报网。

 

唐湘龙 51:53

没错!如果让真主党,让伊朗事先知道了,那大概就很难发动偷袭了。

 

王孟源 52:01

对。他们这次基本上因为HAMAS那些领导就是一些老的宗教性领导,他们根本不用手机不用电脑;他们如果纯粹就是见面开始私人策划,然后警告一下他们的助理不要用手机的话,这完全可以躲过美国跟以色列的电子监听。



所以你从这些线索就可以看出这一件事完全就是HAMAS的领导人看到在乌克兰发生的事情,看到在Niger发生的事情以后,为了响应这个反殖民的风潮而搞出来的反抗运动;而他也的确的很成功了,就是在军事上HAMAS完全没有对抗以色列的本钱,以色列要把Gaza每一个建筑都炸平,每一个Gaza的人都枪毙,他的军事力量完全做得到。(唐:完全办得到。)完全办得到。但是呢,因为现在国际有这个反殖民的风潮,所以HAMAS这样做马上就得到所有非美以核心的国家的支持,它这实际上有三个很大的效应。



第一个就是分裂美欧。因为美国对欧洲的媒体控制在过去这15年主要集中在把中俄画成大魔王;但是呢他并没有对以色列跟巴勒斯坦的这个事件做过多的描述,这主要的原因是欧洲一直都有同情巴勒斯坦的力量,他们有很多的穆斯林的移民,还有支持这种解放——他们真正的相信美国那一套所谓自由民主;其实美国在二战之后也搞过一次反殖民的风潮,不过它真正的用意是打击英法的殖民帝国,方便自己取而代之——用新世代的殖民帝国取而代之。但是这一套宣传体系在欧洲有很好的保存,而且他们真正相信了。所以这一次HAMAS挑起这个冲突跟事端呢,第一个就是分化美欧——就是跟Niger分化美法很类似,只不过是角色倒过来了,这一次是美国去维持以色列的殖民帝国,欧洲不支持他们。



第二个很大的效应就是让中东国家——尤其是在经过Trump的任期之后,中东有很多台湾的产油国已经表态愿意跟以色列和解,交换大使等等,连沙特沙乌地他们也在这个进程之中;经过这一次事情这样一闹呢,基本上就确立了他们不可能再进一步。从长远来看,从更大的视角来看呢,这些中东国家就必须要更加坚定的参与反殖民的斗争,就是在以金砖组织为核心的一个反殖民集团——哦当然现在还没有把旗子高高的举起,但是实际上是一个很松散的反殖民联盟。这是巴以冲突的第二个重要的后果。



第三个重要的后果就是,HAMAS在被关在集中营这二十几年呢,他的经济是完全破产的,那边不要说工业了,连农业都不够自产自足。那么我刚刚提到中东产油国在最近几年要跟以色列和解,所以他们对巴勒斯坦人的支援也在渐渐的降低。那经过这一次的冲突坚定了这些国家支持巴勒斯坦的态度之后呢,HAMAS也可以期望更多的财政跟物资的支援,这才是他们真正的目标之一。



就是从HAMAS的观点,他们有两个小目标,第一个是在国际上做宣传,针对欧洲和中东产油国做宣传;第二个小目标就是对他们自己的经济长期有帮助,因为他们在这个事件平息之后就更方便他们去获取援助;所以你可以看到这是一个理性的决定。



那么我顺便提一下目前这个冲突所看到的一些军事上的细节,就是现在观察这些细节除了军事还有宣传。在宣传上当然是因为我刚刚提到,美国的犹太人完全掌控了他们的宣传体系,这不只是国际,也包括他们国内。像他们国内有一个知名企业的总裁叫做Paddy Cosgrave(帕迪·科斯格雷夫)——他是一个爱尔兰裔的,应该是天主教的——他是Web Summit(网络峰会)这个公司的总裁,他出来指控以色列残杀巴勒斯坦人,结果第二天就被迫辞职了。



哈佛有一个社团站出来支持巴勒斯坦,就被人肉出来——然后这不是被网民人肉哦,而是被所有的金融大亨跟宣传媒体哦——站出来人肉说,这些人你别想找到一个高薪的工作。OK,然后事实上不只是杀害他们的职业生涯,也是在社交上对他们施压,所以基本上这个就是风声鹤唳。

其实我的博客上在几年前就讨论过了,在美国很多傻乎乎地接受美国宣传的人都说,你在美国可以批评他们的总统,你在中国不能够批评总统,所以一个是民主,一个是独裁。其实这里真正的差别在于美国的总统并不是真正权力的核心,他们的权力核心是那些亿万富豪的家族,其中有一半是犹太人。

 

 唐湘龙 58:51

美国的民主跟言论自由,只有经过对于犹太人的批判之后才能够真正的检验。

 

王孟源 59:00

对,就是全世界都不可能让小老百姓去指着权力核心的鼻子骂。在美国你看到的所谓民主的假象是因为他们的总统只不过是一个代理人;就好像一个企业,他不是董事长,他只是一个经理——所以你批评经理可以,但是你绝对不能批评董事长。



那我刚刚提到美国这个殖民帝国统一全球的一个体系,它的最顶点也不是我们现在经常在报纸跟杂志上看到的美国总统或其他政客,或者是他们的知名财团的CEO,那些都是雇来做代理人的一些职业经理。有一些是新进的亿万富豪,像Elon Musk或者是Amazon的老板Bezos。这些人虽然有了钱,但是他还没有历史沉淀,还没有足够的时间建立他们的政治权力。政治参与、政治权力的最高顶峰是需要时间的沉淀的。那你看像现在这些犹太的财团躲在后面,你平常在媒体上根本就是完全看不到的。我举一些名字,看你们大家知不知道这些人,这些真正的世界的统治者——世界的统治者就是100多家这些所谓的Old Money,老钱的亿万富豪。这个门槛呢是大约你的家族的财产要达到300亿以上,而且必须有二三十年的沉淀,你大概要二十年才能够控制州级的政府;大概要四、五十年才能参与国家级别的统治。最有名的犹太家族就是Rothschild(罗斯柴尔德)——但是呢他的根基是在欧洲,而且他有几百年的历史。



我现在谈一下美国的统治阶级。因为美国一个新兴国家他工业化也才只不过100多年出头,所以基本上是在工业化、金融化这100年呢犹太家族兴起,然后跟一些幕后的昂萨家族分庭抗礼,这样的结果——这些才是真正同一家的。我讲一下,第一个要提的就是——而且我再顺便提一下,这些家族往往都有地域性的统治地位,就是他们参与政治通常是从地方政府开始的,所以他都有一个基地——第一个我要谈的家族:Goldman Sachs(高盛)。Goldman (戈德曼)跟Sachs(沙切斯)原本是两个家族,但是他们后来联姻了。那这个家族你会听说过名字是因为他们的那个银行到现在还存在。他们的基地就是金融方面,他们是犹太人在金融势力的核心;然后他们也控制了纽约的地区政府。



另外一个家族叫Pritzker(普利兹克)。Pritzker这个家族主要在房地产,然后他主要控制的是芝加哥。所以芝加哥的市长什么的如果没有这个家族的点头是选不上的。



另外一个家族叫Fisher(费舍尔),这个家族控制了San Francisco就是旧金山。

再有一个家族是Hass(哈斯)。这个Hass是做石油起家的,他们在New Jersey(新泽西)的势力很大。



大家想一想,你除了像Goldman Sachs跟Hass这种是因为他们的公司就是以他们家族的名字来命名你可能有听过之外,你平常有没有看到他们家族领导人上新闻?没有!不可能。就是主流媒体连报都不敢报。



那我自己亲身经验的还有两个家族,这两个家族都是在过去这20年才兴起的。一个是我所住的Connecticut 第一富豪,这是所谓的 Cohen 家族。他们就是美国内幕交易,名义上是非法,可是事实上他们已经买通了整个法律系统,所以他们就是把整个内幕交易体系化、规范化、产业化的先驱。从 90 年代开始的,大概有二三十年的历史,成为康州的第一富豪——这个是 Cohen 家族;我知道康州的州长如果没有他们的同意是选不上的。我在康州住了二十几年,这一点我知道。



第二个你值得注意的是 Sackler 家族。这个 Sackler 家族呢——美国在过去这二十几年,药厂去买通医生乱开药是一个很大的问题;其中最恶劣的,就是让医生开麻醉剂,而在这个手段里面赚了几百亿的主推的那个财团就是Sackler家族。他同样也是买通了他们的法律系统,而且已经免责了。



那这两件事情,这两个家族我在我的博客都有讨论过,都是以犯罪起家的。但是呢美国的体系都没有办法处理他们。



所以为什么目前不只是国际上的反殖民声浪反对以色列,欧美的国内也有反对犹太人、反对以色列的声浪,就是因为虽然他们封锁了主流媒体,但是100多年下来还是有人知道真相。那这些部分知道真相的人就是反抗的人。为什么哈佛会有一个社团站出来?就是因为这些学生他们看到了这些事情。所以大家在以巴冲突这件事上面千万不要学中国的那些60岁以上的官员,反射性的就站到以色列那一边;事实上你所看到的新闻都不但是经过过滤,而且有很多是发明出来的——比如说以巴冲突刚开始的那三天就有两起新闻报道被高调的传播:一个说是有一个以色列的男孩被砍头,另外有一个以色列的婴儿被烧成焦炭;到后来都被证明是伪造的(唐:假新闻。),真正被炸成碎片的是巴勒斯坦的小孩子。



唐湘龙 01:06:21

好,在这个事件发生之后,因为刚刚孟源这段话刚好让我想到我这段时间的整理;我就觉得当下的全球的议题里面来讲,因为这次哈马斯攻击事件所引发的两个趋势它是非常重要的:一个就是国际的,从你刚刚谈到的就是从西非、北非的这个尼日尔开始——台湾翻译说尼日——尼日这次的政变想要把法国势力彻底的赶出去,那这是一个非常重要的反殖民的符号,(王:已经成功了。)对,让法国非常痛苦;第二个就是哈马斯的这一次的攻击行动,那也再度的让这个种族灭绝的议题有机会透过一个网络时代——因为前面的五次中东的冲突都是在没有网络的环境里面;可是在一个网络的自媒体的时代,战争开始直播了之后,那个巴勒斯坦人被灭绝的苦难每天都在现场直播,大众已经受不了了。

 

王孟源 01:07:32

说的很对。这是俄乌战争所引发的另一个效应。就是我看到的这些我所谓的“实话者联盟”——就是不是正规媒体,但是是自媒体上面传播国际事件真相的;我自己也是一个成员,我自己认为是实话者联盟里面最重要的华语成员——因为在过去这两年他们的收视范围都成长了两三倍,所以这一次巴以冲突就能够更好的、更高效的传播真相。对不起,打断了你。

 

唐湘龙 01:08:10

没问题,好,那第三个,就是我说未来的两个趋势呢,一个就是国际社会的这个反殖民浪潮会怎么走,以及美国如何在这个事件当中开始出现一个——在美国总统可以批评,上帝可以批评,可是犹太不能批评,这实在是太诡异了——因此在美国国内对犹太人的这种反省,跟国际社会对殖民的反省,是这次哈马斯攻击事件之后留给这个世界的两个功课。

好,我想我们下个月还会再谈到。还增加一个小问题,因为最近美国的经济我看不太懂了,比如说最近连续三个礼拜的时间都有美国的科技股带头在大跌;第二我们看到美国的债券——包括他的公债的发行,公债的殖利率不停的飙升,也就是他的公债的价格也一直在下降;这些都显示美国的主权的概念的这个经济体系里面似乎出现了某些的问题,可是我看不太懂,孟源你怎么解读这件事情?

 

王孟源 01:09:21

其实这个应该看看我现在在准备的那个博文,还有过去这个月给了四次的演讲——就是同一个题目,我准备的一套PowerPoint,我在台北给了一次,在台南成大给了一次,在中山大学给了一次,然后到深圳又给了一次这个演讲,就是专门讨论你问的这个问题,就是从金融的观点可以理一下美国的经济最近这一两年发生了什么事情。这里的重点是,我其实在今年年初——那个时候美国的长期利率还是三点几,3.2吧——我的小孩坐下来,他因为是学经济的,所以问我:当前美国经济的焦点应该是什么?我说应该是美国的长期国债。然后我看了一下长期国债的利率是3.2\% ,那我跟他说,眼看着这个新的焦点,就是上一波的焦点是通胀,这个通胀已经过了急性的阶段,已经进入慢性的阶段;所以下一个急性的问题就是长期利率会往上涨。



所以今年一整年其实我的博客有问题或者是我去给演讲都是这样谈的,核心都是这样:就是下一场美国的金融危机呢,会是——它的聚焦点就是长期国债的利率。这个长期国债的利率现在是多少?已经是5了——5.0,OK?那我提到的这个股市等等呢,基本上它的幕后都是因为这个长期国债的利率一下子上升了两个百分点,所造成的第一个冲击,然后反冲击。我们在华尔街有一句老话,就是有一个字叫做Dead Cat Bounce,这叫什么?死猫的跳;死猫跳是什么意思?它的意思就是说,你即使是一只死猫,你把它从高处掉下来,它打到地面的时候还会弹跳一下。它这个就是描述股市即使是已经死透了,不过有一阵大跌呢它还是会有反弹的过程。哈哈哈...华尔街说的粗口之一。



那它很形象的描述了我们现在看到的,就是大家终于理解这个通胀不是急性的现象,我虽然熬过了它的急性阶段,但是它会以慢性的态势长期存在。但如果会长期存在的话,你的长期利率就必须要上升,因为美联储所定的利率其实是隔夜的切换率,也就是最短期的利率;这个利率是国债市场上面决定的,而国债市场的买卖双方决定这个利率最重要的根据就是对长期通胀的预期。也就是经过这个通胀危机的真相大白,大家慢慢的知道它会是一个长期存在的慢性问题,所以你的长期的通胀的期望值要往上调,那你长期国债的利率就必须要往上走;长期国债往上走的话,就代表十年、二十年之后的100块美元,现在的价值低——就是你如果是一年 3\% 的利率的话,十年之后的100块美元现在可能值75块钱;但是如果它是 5\% 或者 6\% 的话,很快连 50 块钱都不值。



股票是什么意思?股票的内涵价值在于它在未来这几十年所付的股利。OK,你当然可以买卖进出,仍然有那个价格起伏,那不是沿着一个所谓的 Fair Value,合理价格而做浮动,那这个合理价格在学术界早就有一个公式,这个公式就是你预期未来这段时间,几十年时间他所付的这个股利,根据长期债券换算成现在的价值,加起来就是这个股票的合理价格。那你一旦这个利息上去了以后,这个合理价格就下降,因为你这个合理价格是以现在的美元来计价,而不是10年后、20年后的美元来计价。



所以这就是当前我们看到的,大家因为不确定这个通胀会持续多久会提升到多高,所以这个国债的长期利率经过一个很猛烈的运动,一年不到就从3percent提升到5点多percent,这运动以后一定会有一些不确定性的波动,那这个波动就传递到股市最敏感的因素之一就是长期利率。



好,那你所问的这些问题其实都是观察我们目前股市所看到的因为这个因素的波动,而且这些波动对所谓的 Growth Stock 就是高成长的股票影响最大,就是你刚刚提到的科技股票。这有两个原因,一个是在高通胀的环境下,这些所谓的 Growth Stock 就像高科技的股票,因为他们的进账是在很遥远的未来,所以高通胀影响整个经济的实际成长率以后,就会影响他们未来的成长率,那通胀之下所谓的 Utility Stock 就是 Value Stock 会表现得更好,因为它们是稳定的。这些Value Stock像什么?像电力公司,因为你经济通胀不管搞成什么样,大家还是要用电;或者是像烟草公司,因为他们的顾客真的就是有瘾的,不管你那个通胀...(唐:不管经济怎么样,我都还是照抽烟的。)和抽烟税。那另外一个因素就是这些 Growth Stock,High-Tech 公司他们的那个收入既然是长期的,他们这个回算成现在的,当前的美元,打折就更高。因为你像这个烟草公司或电力公司,他们是现在就可以收很多钱的,他们事实上反而是将来收的钱有可能会越来越少;而高科技公司它的价值就在于,我希望它的收入会越来越多,所以才叫Growth,Growth的意思就是成长,成长性。那股票就是因为我们目前这个环境而受到双重打击,打击之后它的价格下降,然后有反弹——我刚刚说的 Dead Cat Bounce,死猫跳以后再第二波,然后再死猫跳这样子做运动,我们现在看到的就是这个过程。

 

唐湘龙 01:17:29

好,因为时间的关系,因为王孟源从大陆回来之后我不能再这样子操他了,就是已经操过很长的时间了,不过呢下个月的时间我们大概就可以来让王孟源把他曾经上过的课程——只有专业人士可以听得到的;就是我很希望王孟源可以跟大家好好讲一讲,就是一个社会主义国家——因为这是人类历史上面绝无仅有的经验,一个已经非常成功的社会主义的经济体系,要如何去管理资本,跟国际资本形成非常丝滑的一个对接的关系,那个管理既能够维持社会主义本身的它的一个意识形态的主张跟精神,同时跟国际资本在互动的时候又不会产生格格不入的情况,这个其实是一个大学问了。好,这个就是下个月孟源能够为大家准备的功课,他现在写到一半而已。



来,今天我要感谢一些我们的观众朋友,刚刚听课听得很专心的但是忘了按赞了,现在可以补充按赞了,现在按赞数三千六百多,但是事后我看的时候几乎都是接近2万了,所以很多人都是光听,听着听着跟我一样,我就会听得入神了,所以呢好!可以趁这个机会按赞,然后就按小铃铛,然后可以分享今天我们的访谈的视频。



好,那一万六千多位的观众在我们的线上,(略)然后Wu Iris说,惊闻王博士去了深圳,她说大湾区是我和老公认为这个世界上最好的地方,那深圳是大湾区最好的地方,我们两个一直想要在深圳定居,然后把我老公的父母亲从德国接回来养老——她很羡慕你去深圳了。好再来。这个Tang Liu感谢,他说非常开心,听到王博士去我的母校哈工大,那明年我就要回到哈工大做副教授了——哇这么厉害啊Tang Liu——好希望王博士以后常常来,可以听到王博士的讲座。前两个礼拜有一个哈工大的毕业生跑来台湾找我,然后送了我两包花生,很开心的。他说他是我们节目的长时间的资深的听众,跟孟源分享一下,显得你在哈工大的粉是非常多的。好,那CH感谢,他说王博士第一次去大陆真是不可思议。我也觉得不可思议,我在长时间听王孟源在分析大陆的事物的时候,我都会觉得王孟源应该在大陆住过非常久的。所以当你听到王孟源第一次去大陆的时候,你的感觉跟我是一样的,你都会觉得不可思议,对不对?所以有很多的知识,它是有穿越性的,所以我们的很多听众朋友、观众朋友不要常常把一句话挂在嘴巴上面说你不住在大陆,你不了解大陆,这种讲法在任何场合好像都可以讲,但是其实它的意义并不大。

 

王孟源 01:20:41

我有一个标准的回答,就是:即使你把我踢出地球的重力圈到宇宙的无重力中,太空站里面;我对重力方程式的理解还是会比你强的。(笑)

 

唐湘龙 01:20:53

哈哈没有错,对这个上次孟源也讲过。好来再来,这个Huanng LuLu感谢,他想请问王孟源老师,有人说最近3年是大陆的逃命波,尽量要套现出去;但如果这个人真的爱国,不是应该有余力的情况之下鼓励大家消费,经济才能够向好吗?你同意这说法吗?



王孟源 01:21:24

我这次去深圳的时候有听说哦——这是谣言,我不能确定——比方像天津这样的城市,有一些底层的蓝领的工作人员外逃到欧美,因为他们过去几年真的对底层的蓝领工作人员尤其是服务业的人是很困难;可是这其实是全球性的,但是他们只看到自己...他们就想办法偷渡到外面,以为说到欧美会找到更好的工作,这是一个风潮。第二个是上中产阶级或者甚至是富有阶级——尤其是上海——经过了去年,一年多前在上海防疫的那个混乱的经验之后呢,他是对当地地方政府的治理能力出现了一些怀疑,所以他们...而且听说移民的主要对象是新加坡。不过这些都是我听到的二手、三手资料,我不能去做评论哦,只有一个source,只有一个来源。

 

唐湘龙 01:22:38

OK,参考了。就是说其实在大陆的蓝领所遭遇到的问题它可能是一个全球性的问题;再一个就是高科技发展的过程当中所导致的全球的财富的恶性的重分配,让大部分的底层在哪里都很辛苦,你越先进的地方其实越辛苦。



好再来(略)然后CRTC在新加坡感谢,他说最近美债价格暴跌,美国政府还在拼命发债,财政部的TGA也增加了许多。请问王孟源,美国财政部有没有可能趁机在公债市场买下这些已经腰斩的长期债券,然后把它注销?通过这些金融手段让美国的债务像魔术一般的变少。有没有可能?

 

王孟源 01:23:38

他们不需要,因为过去十几年美国的长期公债的利率是2\% 点几、3\% ,这些你买的时候已经吃亏了,就是这些长期债券的持有人在过去这半年多的长期利率上升的过程中,已经亏损了30\% ,那今年4月为什么会有硅谷银行的倒闭呢?就是因为它持有了这些长期债券。当时倒闭的4个中小银行呢,上上个月的时候他们终于有政府的正式的统计资料,说关键就在于他们持有这些长期债券,然后今年年初第一波开始长期债券利率上升的时候,它的隐含价值大幅下降,一下子崩了百分之二十几,然后现在已经崩了30\% 。那这时候全部4家倒闭的银行亏损了3000亿,足够把四家中型银行拉垮。那个美国的正规银行体系呢一共是亏损了9000亿;然后整个国际上的所有的中央银行呢,他们的总外汇储备是大概12万亿美元,假设一共有2/3是各式各样的长期债券的话,大概有8万亿美元的长期债券,就是由国外的中央银行所持有。



你想想看,损失30\% 就是他们损失了2.4万亿,OK?这其中可能包括中国损失的那3000亿美元。那因为中国的中央银行所占的外汇大约就是全世界所有外汇的百分之十几吧,所以这些是已经赔掉了,他们只不过是因为会计的原因不需要承认,这在会计学上叫做Unrealized Loss未实现的损失,这个都已经是Water under the bridge(覆水难收),已经过去了,美国财政部根本不需要做任何事情。



目前美国财政部面临这个长期债券利率上升的最大危险是,美国当前的国债是大约30万亿美元,它因为过去十几、二十年长期债券的利率低于3\% ,所以他们发行了很多长期债券。那到现在这30万亿的国债平均支付的利息还是3 percent。OK,也就是说他一年支付国债利息是大约1万亿。美国今年的财政赤字是1.7万亿,其中1万亿是利息;那长期利率上升到5\% ,或者甚至在5年之内很可能会上升到6\% 点几。你想想看,五年之后美国的国债应该达到40万亿。如果他的平均利率达到5\% ——因为你这些国债到期了必须要重新借,你借的反而比还的要多。(唐:没错,那利息一年就是两兆。)利息会很大的上升。五年之后,光是利息就超过2万亿,超过目前的总体赤字;与此同时,社会福利金跟Medicare(医疗保险)还在持续的恶化,每年大概各几百亿美元的这样恶化之中。所以你可以预期5年之后美国的赤字呢——结构性赤字,就是你不管谁当总统都没有用的——结构性赤字会达到3万亿,接近3万亿。



那我在今天讨论的时候稍早有提到,我认为这个大环境即将从量变到达一个质变,然后中国的国际金融跟经济管理必须要重新检讨这个态度,可是重点是不能够再浪费。这个的意思就是,我们这个国际社会其实在过去30年冷战后呢,因为美国拼命的印美钞,然后用很宽松的货币政策,它其实是有一个刺激作用的,这个刺激不只是刺激美国的经济,而是所有参与美元交易的经济。所以尤其是像中国这样的出口导向国,他事实上是获得了一个外来的刺激,只不过是这个外来的刺激持续了很久,持续了30年;(唐:已经习惯了。)所以你以为是常态,其实不是,它是美国霸权达到顶点,可以无限超发美元的这30年的一个例外,这个例外的现象即将要到达一个终点,就是当美国的赤字——五年之后美国的结构性赤字达到3万亿的时候,你觉得美元超发还能够让大家心甘情愿的接受吗?不可能的,到时候一定会有一个止点。(唐:这段说的非常好。)美元超发发生的一个后果就是,你不管这个质变是怎么发生的,美元被替代,还是大家是不是有一个新的国际金砖货币或者是亚元什么的来确认,这个都不能确定;你也不能确定美国的这个经济受的打击是多少外来的国家来帮助他承受,这些都不能确定;你可以确定的一点是过去30年的这个全球经济刺激效应已经要结束了,结束以后你就更没有理由硬是要追求每年8 percent的GDP成长率。你硬是要投资,靠举债投资去刺激经济,将来吃的亏就越大,而且这个亏会在5年之内就实现。

 

唐湘龙 01:30:13(答谢部分选略)

好,这段我希望大家可以反复的听,虽然是从一个听众的问题所引起的,不过孟源讲这段的时候我觉得讲得非常的清楚了。



因为今天时间的关系不耽误孟源了,所以各位的一些问题我们下回再想办法回复。还有这个林冠感谢:谢谢王老师,受教了。



好,感谢今天所有的观众朋友们今天对于龙行天下的捧场以及对于王孟源的追粉。好,那所有的刚还没有按赞的,现在赶快来补按赞。那今天星期五的时间,感谢孟源虽然人在台南仍然透过连线没有时差的提供了龙行天下的观众朋友非常丰富精彩的内容,感谢孟源。

好,也感谢所有的观众朋友们,来,下个月见,拜拜!

\twocolumn[\begin{@twocolumnfalse}
\section{務必看懂國際政治三大趨勢}
\subsection{20231208}
\end{@twocolumnfalse}]12月11日,校对完成。



唐湘龙 00:00 

好,欢迎来到龙行天下,我是唐湘龙,来,今天星期五的时间,龙行天下的单元。跟大家报告一个好消息,就是上个月不见了的王孟源回来了。来,在我们线上的,人在美国,我刚问了他一下,说你周围的温度几度,他说大概在零度左右,所以你看到王孟源今天虽然人在室内盛装打扮来的,上个月失去了王孟源,终于回来了,来孟源跟大家打招呼,同时我对,同时我要预告一下,就这个月,这个月有两次往王孟源好,补一次。除了今天这之外,下下周的星期五的时间仍然是王孟源的时间,一方面是因为要弥补上个月许多人的项羽之汗,因为上个月是失去王孟源之后,我接到了非常多的抱怨。



唐湘龙 01:25 

好,那但是这个月一方面到了 12 月,其实许多的国际政治的大事,它还是有一个有一个时间轴的,一个是循环。到了这个阶段的时候,我们会看得出来不只是 2023 年,而是这两三年的时间,许多的国际政治的大趋势正在一个剧烈的结构调整的阶段。好,今天我们就从三个主题,这三个主题都是在年终就之前你必须要关注的议题。一个俄乌冲突是不是接着接近尾声了?那对于,就是说以欧洲为主体的西方的政治权力的板块有什么影响?第二个就是以巴的冲突,第三个我们再回头来看中美的习拜会来,我们从尾巴的冲突看起,当然今天看起来之后,在过去有很长时间,我们从西方媒体当中所得到的都在吹捧北约,吹捧乌克兰的那种的乌吹现象,乌吹现象到这个月的时候几乎都不见了,为什么?



王孟源 02:27 

因为他们被打脸的太多次了,而且这个所谓的春季大反攻,后来变成夏季大反攻,他们打到现在。事实上乌克兰占领的土地比起今年年初的1月 1 号还是要少的,他们反攻的反而是越来越少。他们唯一能够说有前进的基本上就是在南线有两个地方,一个是原本俄军的突出部,很快的就让出来,然后另外是一个真正的重点方向,大家在半年之前就已经预言会突击的,就是南边的Zaporizhzhiam,一条很重要的战略公路沿线,他们攻克了一个村庄,花了两个月攻克了一个村庄,然后就停滞了。你如果去看他前进的那 5 公里,刚好深入的那个口袋就是一个山谷。换句话说,俄军基本上就是让乌军前进了 5 公里,然后进入这个山谷谷底,然后对他从高地上进行三面包围,以后就不再退让,然后就围着他打。



王孟源 03:38 

我记得过去这两年来,我被人家说是俄吹,不晓得说了多少次,事实上俄国人他是因为战略跟战略的原因,故意不速战速决,这个决定你原本是可以商榷的,就是尤其是头两个月是非常地冒险,但是我们现在战事已经将近两年,可以回顾,他的那个赌博是完全的成功了。



王孟源 04:15 

俄国明明是为了他自己的国家安全,侵略了一个主权国家,但是两年下来,他反而成为第三世界反殖民运动揭竿而起的陈胜吴广,获得了全世界除了美国殖民帝国的核心成员之外,所有的重要国家的支援,而且,不断的有其他的国家揭竿而起。半年多前是Niger在西非要反抗法国,四个月之后就成功了,法国已经完全撤出了Niger。那接下来更重要的就是巴勒斯坦。这个基本上俄国跟巴勒斯坦他们所做的,是他们愿意牺牲他们军人的生命,来换取国际态势的改善。那实际上,战场上完全不是他们眼中的重点,因为对俄军来说,他们必胜,对巴勒斯坦人来说,他们必败。但是这都不是重点,在战略上,他们都是想要挑起反殖民,全球反殖民运动的高潮,而且成功了。



王孟源 05:45 

那我们来看一下这个俄乌战争,过去这两年,因为这个特别的战略背景,所以它的战线其实相当焦灼的,其中最大的运动就是在两次,就是刚开始的那两个月,俄军是孤军突进占领了很多土地,然后到了去年秋天,就是一年多一点之前,去年9 月 10 月的时候,乌克兰做了反攻,在俄军的比较兵力薄弱的一些地方有了若干的突破,那在那之后基本上战线没有太大的变动。



王孟源 06:28 

那这么一来,相信俄欧美主流媒体的人就可以指着这个战线说是俄国打得很烂,他没有办法前进,然后他死伤惨重,但是我们这些比较能把整个战略跟战略态势也考虑进去的人,就是说其实是俄国不想打。那俄国在军事上它的目标其实也是就只是杀伤而已。没错,那么双方都可以各执一职。为什么?杀伤这个东西是没办法,你在地图上标示出来的,看不出来。



王孟源 07:11 

对,那但是你如果真正去研究新闻的细节的话,你就会知道真正可信的消息完全是支持俄军这方面的报道。这里最重要的一个证据是BBC,我如果查看我博客的应该知道 BBC是英国国家资助的官方媒体,当然它历史上跟 MI6有扯不清的关系。那在这一场战争刚开始就是去年2月的刚开始他马上就联络了俄国境内的一个NGO,这个是一个很有名的反政府NGO,他的创始人,两个创始人都被关过,然后他跟这个 NGO 合作,开始在俄国境内统计他们的伤亡人数。他们的用意很简单,就是他们预期俄军的伤亡人数会很高,所以他们就想要做出严谨的统计,然后想要用次来羞辱俄国。结果没想到俄国非常的开放,他们的军人死亡以后没有什么保密的。该出殡的就出殡,然后该在报纸上登讣闻的就登讣闻。嗯,然后他们两年来辛辛苦苦的统计,结果统计了,才第一年 BBC 就受不了了。为什么?因为统计出来的数字基本上就是俄国国防部公布的数字,你相差个 5\% 10\%,那基本上是统计上的一些很简单的误差,必定会有。那所以到今年年初 BBC 就不再提这个统计了,因为你这个统计提出来以后,反而是只有其他媒体所造假的数字的不到 1/ 10。到一个月前他们刚刚又有人去挖坟,发现这个 NGO 还是在继续做统计,就是当年拿了钱,拿了足够他们一直做下去。虽然 BBC 到了今年就假装没有这件事,但是他还是继续在做。那到 11 月初的时候,他的统计是2万出头, 23000 人,就是俄军阵亡了 23000 的。那这跟二国防部自己的报告是同一个数量级的。



王孟源 09:49 

然后你看乌克兰的阵亡人数,当然都是他们的国家机密,他说什么都不讲。在一年多前Von Der Leyen就曾经说漏了嘴,说他们已经阵亡 10 万人。好,你可以说Von Der Leyen是外行人。那当时也只是说口误,所以不能够确信。但是也同样是在一个月前,我看到一个消息,我觉得很有意思,就是他们庆祝那个退伍军人节的时候,他们的退伍军人部发表了一个数字,说过去两年一共有 61 万人因伤退伍,其中有 87000 人严重残疾,意思就是你必须要坐轮椅了,或者就是躺床上了。你如果大家可以想象你这个能够因伤退伍的至少也是中等伤势,那这个残疾到要坐轮椅。这个是典型的重伤的定义,而且是比一般的所谓的重伤——传统上有的时候所谓的重伤指的就是不能够再上战场的,那这个是已经比那还要严重很多——那一般的估计是重伤的数字跟阵亡的数字应该是差不多的。OK,那因为但是那个重伤的定义是不能够再上战场的,所以你可以把这两个数字参考一下,就可以估算出二菌的乌菌的阵亡人数应该是 20- 30 万人,就是你如果是轻重伤一起算的话,伤跟亡的比率一般是 3: 1,嗯,如果只算重伤的话,伤亡的比率应该是 1: 1,这是现代自动武器下的战争的一般的常态。



王孟源 11:55 

那根据这个比率来算的话,乌军已经阵亡了 20- 30 万人,也就是说它跟俄军的交换比至少是 8: 1,可能是 10: 1,甚至 12: 1。那在这个情况下你就可以知道为什么现在欧美的媒体都没办法再提了,这一方面是乌克兰人已经快要完了,人已能够动员的人都已经快完了。没错,现在乌克兰的前线已经反复的多次地出现了女兵的士兵。嗯,那唯一还没有动员的就是大学生,那这大学生呢?第一个人数不多,第二个他们基本上是地方土豪的子弟,避免服兵役的一个幌子。你如果去动的话,这个政治代价很高的,就是基本上地方上的那些有势力的人都会反抗,所以这基本上可以说它已经动员到无可动员,这是在人力上面。然后武器方面,现在一年之前我讲俄军可能已经用了 800 万发到 1000 万发炮弹,现在我看到最新的估计是过去这两年,不到两年已经打了 2000 多万发 到 3000 万发的之间就基本上是用同样的速度,那就是一开始的时候是靠库存,然后后来它的产能上去了以后,基本上就可以满足这个非常高的速度,还跟朝鲜进口的没错一批来补充,所以基本上是可以填平的。



王孟源 13:45 

那乌克兰目前所发射的炮弹数目基本上就是这个的 1/ 10,就是少了一个数量级。所以你跟那个战损比也是合理的,因为在现代战场主要的杀伤手段就是炮弹,对你的炮弹的使用量,如果是 10: 1 的话,那战损比很可能就是 10: 1,所以这都是一致的。那原本我刚刚讲俄国大概一年用掉 1000 多万发炮弹,原本欧盟在战士一开始承诺要给乌克兰送 100 万发炮弹,大家猜一猜这 100 万发炮弹后来出了什么事情?在和平状况,就是在两年前,欧盟德国定制一发 155 毫米的炮弹,是 2000 欧元,一旦战事一开始,马上就涨价到8000。这就是自由市场的魅力。



唐湘龙 14:45 

OK,155的炮弹而已,只是 155 的炮弹。



王孟源 14:50 

就是普通的炮弹,没有什么制导,就是那种OK,台湾有很多。



唐湘龙 14:55 

OK,那打每一发真的心都会痛。



王孟源 14:59 

对,8000 块欧元就是一辆小的基本的车了,微型车,对不对?那他们原本两年前说要支援 100 万发,结果到现在,过去两年总共生产了6万发,就是它的产能就是这么可怜,他们新生产的只有6万发,他们倒是从库存有 30 万发,全部都送过去了,然后又从亚洲,像是南韩那里买了 12 万发。嗯,所以总共送了不到一半,就是 48 万发。



王孟源 15:42 

OK,那你想想看,俄国自己一年就可以打掉 1000 多万发,欧盟两年才省吃俭用弄出 48 万发,其中还只有6万发是现生产的。那你可以想象这两边根本没办法打。那我为什么现在欧美的媒体开始承认这个残酷的事实?是因为他必须要和谈。那要和谈有四个原因,第一个是乌克兰的人快要打光了,第二个是欧美的兵器跟弹药填充不上去。刚刚谈的是弹药,你那些兵器,至于他们那些什么奇迹武器,其实过去几个月我记得你也提过。对,就是他每隔两个月就会弄出一个新的奇迹武器。



唐湘龙 16:36 

然后吹牛吹半天,但是两个月之后就不见了。



王孟源 16:41 

一旦应用以后,对,过了两个月就不见了,然后要换一个新的奇迹。我记得他们上一次提到的新的奇迹武器是F16,但是问问 F16 上不去。嗯,你知道为什么吗?因为也是在一个月前,俄国人在一夜之间打掉了十几架乌克兰的战斗机,就是把它当时在空中的所有战斗机全部打掉了。嗯,一次就打掉了。那这很显然的就是俄军有了足够的预警机,能够完全控制乌克兰上空。那在这之后乌克兰基本上没有办法再派战机升空了。那如果你的苏式的战机没办法升空, F16 就能升空吗?不行的嘛。因为他们那些制导导弹,像 S400 这些导弹,不管你F16还是苏27,嗯,你一样都是打下来的。所以这个 F16 送过去以后会比其他的那些奇迹武器像是豹式坦克还要丢脸,而且死的要更快,所以他们现在就在拖拖拉拉了,这个没办法再送上去了。这是第二个原因,第三个原因是财政上的,乌克兰现在基本上没有税收了。他们整个基国家的经济已经瘫痪了,连农业都没办法自给自足。那么这个时候他们的政府不要说军费了,连一般的政府的开支都必须要靠欧美的援助。



王孟源 18:27 

过去这两年一个简单的统计是美国在军火援助上面给了 400 亿,在纯粹的金融援助上只有 200 亿。这原因是因为军火都是用美国价格跟美国军工厂商买的,所以不用转一圈就直接进了自己人的口袋,对,比较慷慨;那欧洲人就比较傻,欧洲人的军火援助是 500 亿,但是它的金融援助是 1000 亿,所以乌克兰现在的这个财政基本上就是靠美国,但是尤其是欧盟来撑,那美国现在因为换了众议院院长,他这个新任的众议院院长是极右派的,不是建制派的,所以他不愿意再通过法案去资助乌克兰,这是一个问题;更大的问题是德国也是在上个月出了一个宪政危机,他的财政赤字被判定违宪,所以现在他们的财政被冻结,整个被冻结了。根本没办法再几百亿几百亿地往外丢,所以因为这两个原因,所以乌克兰现在已经眼看着要断炊了,这是第三个原因。第四个原因则是因为美国要大选了,那个,眼看着这一件这一个摊子越来越烂,越来越难收拾;然后第五个原因是以巴冲突,OK。美国人要整俄国人就说要让乌克兰加入北约,然后俄国人就不得不出兵去打,这是他的痛点。美国人要整中国的话也很简单,只要叫台湾独立,那么中国就不得不出兵去打。然后美国人就可以联合日韩跟欧洲来做制裁,因为你出兵嘛,你侵略嘛。



王孟源 20:47 

但是过去这两个月出现的就是一个现世报的报应的情形,而且中俄还没有去安排,不像美国用 CIA 跟国务院这样特意去安排。他是巴勒斯坦人看到了,哈马斯看到了俄乌战争的启发,自己揭竿而起来响应它。那美国如果有一个类似的痛点是必救的,就是以色列。没错,那这个以色列这件事情很快的就变成国际谴责的一个焦点。那在这个谴责之下,以色列就没有办法用他们,即使他们有一个部长说要用核子弹把他全部轰平。那你如果去看那个加萨走廊的话,其实是很小,人口密度又很高。对,如果是你无限制的使用重火力,尤其是空投的弹药去炸的话,基本上是没有防御的可能。嗯,那里面的那两三百万人,你把它抹掉 3/ 4 都是可以的。



王孟源 22:07 

但是问题是外交态势不容许,那不容许的话,美国就更急着要解决这件事情,尤其 Biden特别尴尬。我以前解释过美国的这个Neocon就是这个到处煽风点火,一切都以战争来解决的这个思想,是在 1996 年才确立的,那一开始的时候是在小布希的政权里面进入政府,进入他们的国防部跟国务院,但是在 2009 年的时候他们政权更替,这时候奥巴马上台,奥巴马本人并不能够接受这种穷兵黩武的思想,但是 Hillary 可以。所以当时在原本在小布希——这些很奇怪的一个现象——在美国政坛上很少看到政党轮替的时候,一大批的国防部跟国务院的高官留任,而且是转到了统统转到了国务院。



王孟源 23:17 

那 Hillary 当了四年的国务卿,后来退休了,退休的时候这些人就另投明主。这个人是谁?就是当时的副总统Biden,他们的这些 Neocon就从国务院有的留任在国务院有的人就进了 Biden 的团队。所以后来Biden在 2021 年掌权之后,他马上又接收了这些 Neocon的团队。也就是说这些Neocon其实只有在 Trump 那四年离开了政府,否则在过去的这二十几年一直都是主管着美国的外交国际战略政策。



王孟源 24:02 

但是你 democrat 这个民主党原本是在,比如说在越战的时候是反战的那一派,现在因为 Hillary 遗留下来的遗产,接收了这些Neocon这些战争贩子,那这就非常尴尬,因为你一旦你共和党如果掌权的话,比如说Trump对以色列还可以无限制的支持,反正就硬着头皮不管这个国际外交态势,至少它国内的共和党选民是会权利支持的。



王孟源 24:44 

但是民主党是一个所谓的 big tent 这个大帐篷,美国国内的回教徒都是属于民主党的。那你这个民主党明年大选的话,如果 Biden 真的是就是不顾一切的去为以色列背书的话,嗯,那些回教选民就会抛弃民主党,那么在原本胜算就不大的这个大选形势之下,基本上就必败无疑。



王孟源 25:22 

所以这个就是为什么现在这个民主党政权来搞Neocon,比共和党政权来搞Neocon,面对以巴冲突还要更为尴尬的原因,也就是为什么他们真正开始急了。其实在半年多前我就说当时的Sullivan,Sullivan是Biden团队里面负责他搞选举的,搞连任竞选的,所以他当时就已经开始有点担心这件事情,拖到 2024 年的话,这个俄乌战士拖到 2024 年的话会对大选有影响。他当时就开始半年多前就开始放一些试探的气球,那现在你就可以看到他急了。



王孟源 26:09 

一方面是没人,没钱、没武器、没弹药,嗯,然后国际局势跟大选又都不容许这样无限的打下去,那这时候就有一个和解的动力,那当然这个和解俄国人是不会接受的。你这个俄国人在前两个月要恐吓Zelensky 要赶快和谈的时候,当时他并没有任何领土的要求,他只要求乌克兰保持中立,不要加入北约。你后来到了3月底这个和谈失败,然后开始认真打之后,就不可能再有这么优惠的待遇了。那很快的,在一年多前,他把已经占领的东乌四州正式的吞入了俄国。但是这四州只不过是现在被俄军占领的,真正俄国人就是自认为俄国人的人,占人口多数的有八州,东乌有八州。所以我认为,Putin的和谈的底线是拿下东乌八州,而不是这四州。嗯,拿下这八州,然后的这八周就包括Odessa,OK,还有Zaporizhia,还有Kharkov。那拿下八州,也就是说乌克兰必须把目前还占领的土地再让出四州,然后再承诺中立,然后承诺不加入北约。这是应该是俄方的和谈的底线。那我们来看一看,乌克兰可不可能答应这样的条件,Zelensky 是不可能的。



王孟源 28:06 

OK,乌克兰现在的治理团队总统府系是文宣团队,它基本上是被欧美的宣传体系捧出来,作为一个所谓的现代的丘吉尔,OK,然后对抗邪恶的侵略者。但是实际打仗的军事有另外一个体系,这个体系是以他们的总司令Zaluzhny (扎鲁日内)为主的。那目前斗争的最激烈的也就是一方面是总统府的文宣体系,一方面是军部的Zaluzhny这个体系。那你说哪一边比较贪腐?两边都一样的贪腐,但是至少军方是必须要拼命的。而且Zaluzhny 在从一开始终就是激进派,他原本就跟那些新纳粹主义者的乌克兰国家主义者走得很近,所以他是真刀真枪的为了反俄出来的。那你看Zelensky 的历史就刚好相反,他竞选获胜,其实他的政纲是要跟俄国和解。



王孟源 29:28 

嗯,是因为前一任的那个总统跟俄国闹得很不愉快,那经济搞得太烂了,所以Zelensky 以跟俄国和解作为政策要领才当选的。所以它其实是被欧美拱出来的。但是正因为他拱出来以后做了这个急转弯以后,过去两年是谩骂俄方不遗余力,他已经没有那个余地,也没有那个信用,更没有对军方实权的掌握,来坐下来和谈。他的这个正因为他的地位是虚的,所以它只有强硬到底。才有正当性。你真正要能够以鸽派的姿态坐下来和谈,然后割让国土系来获得一个和解,唯一可能的就是战争英雄。嗯,你看那个历史上凡是你要认输的时候,都只好把以前战争的英雄推出来,像是二战的时候,德国闪电战把法国打垮以后,法国把谁推出来认输?把一战的英雄贝当元帅推出来,只有他才能够压得住国内的反对声浪。嗯,来签那个投降条约。所以目前乌克兰内部的政坛的两边斗争很有意思。你如果欧盟跟美国都认为可能要准备和谈,但是他们现在还没有认真。那问题是 2024 年就是乌克兰的大选年,那Zelensky 你知道他如果竞选的话是绝对没办法连任的。国家已经被搞成这个样子,所以他就把他把大选给取消了。



王孟源 31:36 

那这对欧美的那个宣传来说是非常不好的一件事情。那但是你看,从Zelensky 的角度来看,它已经是四面楚歌了,这个它已经被弄到角落里面,他原本上台能够获得欧美的支持,就是靠着绝对反俄,绝对的激进的一些言论,那现在背后的老板反过来要准备和谈的时候,他最容易被人家丢掉,而且这时候还有一个非常合适的替代者,就是Zaluzhny 。



王孟源 32:18 

那当然Zaluzhny 你本身愿不愿意出来当这个傀儡来顶下这个骂名?他本身就是一个乌克兰国家主义者,他愿不愿意为了这个总统的虚名来承担这个历史责任?嗯,我觉得是本身就是一个很大的问题,不过目前谈起来还太早。为什么太早?因为双方,欧美还没有下定决心,然后即使他们下定决心,等到他们发现俄方的要价是那么高的时候,又还会又有转折,中间又还要有好几个月。



王孟源 33:00 

嗯,那在这之前乌克兰的那两边已经开始内斗得很厉害了。那Zelensky在上个月刚刚把乌克兰军部的那个军医总管给撤职,这因原因是因为这个军医总管就是Zaluzhny的人,所以他开始一个一个的撤,那第一个就是先撤掉这个这个军医的总管。那我们在未来这个几个月可以看到他们双边的政治斗争激化,但是你如果从……他们这个激化之后,会不会从动口到动手?就是有政变?或者是暗杀之类的?这很难讲,那我没办法预测,但是有这个可能。但是如果没有弄到政变或者是暗杀的话,最终的决定必须要由他们背后的支持者,也就是欧美,尤其是英国跟美国来决定。



王孟源 34:07 

那,英美的这些外交战略决策者,我刚刚讲过美国现在的还是这些Neocon,那这些Neocon是很笨的,他们只知道遇到什么事情就只有出兵,如果出兵不顺就出兵更多,然后投资更多。那我刚刚提到这个很简单的考虑就是Zaluzhny比Zelensky更适合坐下来跟俄国和谈这件事情本身就需要一点头脑,而且和谈这件事情本身就触动了这些Neocon的逆鳞,所以这些美国的Neocon应该是会是 Zelensky 的支持者,因为他们知道泽如果是 Zelensky主政,就不可能有真正的和谈,所以他们应该会继续的支持 Zelensky。



唐湘龙 35:02 

孟源,我请问一件事情,然后我要继续再问你的以巴的中东的情势的未来。就是这场的战争,俄乌战争,当然就在我们今天现场的连线,昨天美国的参议院已经拒绝了拜登的那个 1000 亿的那个拨款的法案,那这个拨款的法案里面看得出来,其实 100 亿给以色列, 600 亿是给乌克兰的,但是没有了这个这笔的就是说援助法案之后,乌克兰大概无以为继,那个对乌克兰的信心是一个非常沉重的打击。可是我们回头去看这两年的时间了,我觉得他可能留下一个课题,就是在二战结束之后,最大的军事神话北约在这场的发生在欧洲的军事冲突里面的时候,一开始表现得非常的有自信,觉得是可以帮助乌克兰去对付在冷战之后其实一直没有办法有效重建的俄罗斯,所以整个的舆论的那种的吹捧,以及不断的去秀那些奇迹武器的过程当中所展现出来的那种的自信,已经满足这个神话的情感需求,其实过去两年非常的明显,可是当战争打到这个样子的时候,我们就得回头去检讨一下,北约的神话破灭了吗?



王孟源 36:26 

他们自己还不知道,但是在第三世界眼中是已经破灭了,而且整个欧盟的核心其实是德国,那德国被他们这些立党这些人乱搞之后,是在外交跟国防还有产业政策上面非常的损伤,德国是今年所有主要经济体中唯一一个负成长,另外一个是韩国,但是韩国还不确定,韩国还没有德国这么惨,德国是过去这三个季度,每一个季度都是促成的。



王孟源 37:03 

那而且我刚刚看到,就是昨天我才刚刚看到最新的数据,就是他们在 11 月,他们的工业产值居然又掉了0.4\%,而且这个不是同比,而是环比,就是 11 月比 10 月又掉了0.4\%,一个月就掉0.4\%,这个是一年掉 5\% 的节奏。OK,那这个去工业化之快,然后基本上眼看着就只好吃老本了。



王孟源 37:40 

德国的……你能够说它有什么好处呢?好处就在于它,它的赤字跟国债相对是比较低的,即使是以欧洲的水准来说都算是很低的,那所以他这个老本大概还可以吃好几年,但是你说要真刀真枪的上战场,或者是在经济战线,你跟中国比电动汽车,或者是在软件 AI 发展,跟美国去比,根本就不可能比的过,他们都没有前途。这就是一个破落户的败家子,现在还才家产还没有开始卖,但是眼看着就只能够一路卖到底了,这是很不幸的。问题是他们连最基本的现实都没办法认清。



王孟源 38:41 

刚刚讲到电动车,我刚好联想到一个一件事,就是我讲到电动车,又讲到认清现实。你知道我一个半月之前到深圳去调研过,那两个月前了,两个月前我到深圳调研,那时候我去了好几家深圳的大企业,基本上都见到了他们的真正的经理,甚至有研发主任,我最喜欢是跟研发主任谈,比如说我到华大基因,就由他的研发主任带着到他那个生产的厂房,就是里面一大堆那些基因分析机器, 100 多台的机器,这样子看了半天,然后他把他们迷藏的一个还没有上市的一个研究计划也跟我介绍,这我不能够谈,哈哈哈。



唐湘龙 39:33 

非常先进。



王孟源 39:34 

但是很厉害,印象很好,但是唯一一个例外就是只去看了公关的就是BYD。我想我两个月前有提到,我本来想去华为,但是华为因为当时有 Brics meeting,还是哪一个国际会议,结果华为就没有办法安排,然后 BYD 也只安排他们标准的公关,就是没有安排他们的研发主任,那跟他们的公关去参观了以后,流程之后,我就说我有两个小问题,那位公关的小姐就说没有问题,你把这个问题交给我,我带回去给我们的研发主任看,然后他再回答回复给你。好,我就把两个问题写下来,其中一个问题就是为什么 BYD 的外销数字这么少?OK,当然我自己事先就已经有答案了,我其实是想要看看他们怎么回答。结果他的答案对我来说是一个满分,你知道……



唐湘龙 40:43 

really?所以他后来有回答给你。



王孟源 40:46 

后来有回答给我。



唐湘龙 40:48 

OK,那还是很有诚意的啊,表示公关的人、要不然一般根本问了之后丢一边了,根本不理你。



王孟源 40:56 

对,然后两个问题都很详细回答,然后这个问题他回答我觉得是满分,刚好就是我事先自己内定的答案。他的回答是,因为我们历史上抄袭别人的车型,然后品质管制又非常的差,所以就没有办法打开外销市场,一直到近年才有改进。你想想看,这种话,美国企业绝对说不出来。



唐湘龙 41:26 

其实不容易。



王孟源 41:27 

容易,他绝对不会认真的检讨承认,因为其实必须要先对自己诚实,才有可能真正切实的改进。我想全世界只有,这是我第一次看到一个重要的一个大型企业。世界 500 强的企业。有这么诚实的对外承认自己的缺点。他就是,这就是为什么中国能够搞电动车。美国人必须要在中国生产。Tesla 必须要在中国生产。



唐湘龙 41:59 

这样你给他打 100 分是应该的了,打 100 分是应该的。



王孟源 42:03 

对,然后为什么德国人都生产不出来?因为德国人以前就是这种脚踏实地的态度,现在不行了。现在你看看他们那些执政者,现在他们那些企业者,愿意说实话的人都被解职了。都被推到后面,都必须要闭嘴。所以跟你说就是,中国发展实体经济能够成功,它的最大的动力就是这种实事求是,对自己能够真诚实的批评的这种态度。这在现代的 21 世纪的社会有多难能可贵?我自己在美国的商场上打滚了二十几年,我非常清楚。



唐湘龙 42:50 

那你刚问他第二个问题是什么?我说你问比亚迪的第二个问题是什么?



王孟源 42:57 

第二个问题是,我问他,他们基本上是你知道他们连轨道车也做,他们那种火车他们也做,就是捷运他们做,他们也做,当然 巴士他们也是很有名的。但我就说你们基本上这种凡是电动的运输工具你们都做,但是你们就是不做大货车。嗯,为什么不做大货车?然后他给我一个答案,然后,不过那个不是今天要讲东西没相关,所以不必再多谈。就是他的答案是说因为大货车的销售对象是个体,很多是司机,那他们觉得他们不擅长,巴士的话通常是跟那个市政府打交道。那这个一次卖就是几百辆,他们是比较擅长。就是换句话说就是他们觉得自己做营销是弱点,他们就不愿意去做这种零售的事情,我认为这是一个合理的取舍,当然我个人是认为有办法能够绕过这个人,就是你做供应商,你不要搞零售,你去设计一个体系,还有一个标准,然后来对那些其他的做货车的人提供主要部件,就是像联发科的,对那些你自己不做手机,但是你可以当联发科,不过这个是另外的话题了,我不深谈。不过你可以,我给你一个taste,你可以看到我这次去调研,基本上是怎么样的讨论的问题。自己学了很多,然后我相信我所做的评论,偶尔也对这些企业的经有些启发。



唐湘龙 44:52 

好,那除了就是说以巴(编注:联系下文,此处应指俄乌)所留下,因为以巴(编注:联系下文,此处应指俄乌)战争不管他怎么结束,其实受伤最重的是欧洲,欧盟的这支伤得非常重,他战场在他这里,而且事后不只是以不只是乌克兰的重建,我觉得整个欧洲体系的重建会是一个非常复杂的工程,因为这场的战争把北约的神话给打没了,把欧盟这种的团结感,原来对未来的这种乐观的感觉也伤到了。但是中东呢?中东其实这次的就是以巴的冲突发生之后,那美国很快的在周围做了军事部署,那果然是有效的,周围几乎没有任何一个国家敢介入到这场冲突去帮哈马斯就真正的提供援助。



王孟源 45:40 

本来就没有,本来就不可能,我跟你讲,这种事情,因为哈马斯是必输的。后来Reuter后来有报道说,其实哈马斯在开战之前有去秘密询问伊朗的意见,结果伊朗说,你要打,我给你精神支持,但是我不能够真正打,因为这种事要牺牲自己。我刚刚讲过,你在一个国际反殖民浪潮之下,站出来冲到最前线,是要以鲜血为代价的。那你既然已经有俄国人愿意冲在前面,你既然有哈马斯愿意冲在前面,那其他人就在第二线声援就可以了,然后代价最小,但是同样的是有帮助。嗯,那你说伊朗是不是真的有这个政策说不参与?你如果连伊朗都不参与,更其他的国家更没有理由参与嘛。



唐湘龙 46:38 

参与吗?那由。



王孟源 46:39 

你连那个Hezbollah(真主党),Hezbollah是他的嫡系的部队,他从头到尾就是一副没有准备的样子,而事后也只是声援一下。你就可以知道,的确,伊朗是有明确的政策。讲到这里,我再提到一个很有趣的新闻,这是上个礼拜在以色列的一个国内的一份报纸。嗯,揭露的就是在哈马斯动手之前,摩萨德 CIA 都说他们没有任何迹象,就是他们的电子监听都没有看到任何的迹象。但是他们事后去检讨,发现有另外一个很明显的迹象,当时没人注意而已。就是在动手之前那个礼拜有人做空以色列的股市,而且他们追踪到的这个做空的至少是8亿美元。以色列的股市的规模来说相当大了。那Hamas 的那些,那些Imam,他们连手机都不用的,他们会去炒股吗?他们有8亿美元来炒股吗?



唐湘龙 47:50 

就应该没有。



王孟源 47:52 

我觉得这反而有可能是以色列自己的,自己人知道以后闷声大发财,然后不跟国家,不跟公家报道,不过内幕如何我们不能确定,这很有意思就是了,真正是有人去做控。



唐湘龙 48:11 

在有关以巴的问题,有两件事情请问孟源,一个就是因为这个月刚好中国是联合国安理会的轮值主席国,那你看得出来中国不管是王毅去主持会议,或者是在之前的这些安理会的表决,中国都用了非常大的力,或者在突出中国在以巴问题当中中国的一个基本的立场,再加上中国在整个波斯湾,在沙特,在伊朗这之间的这种的平衡者的角色,那这些的这些中东的,包括阿拉伯国家的外长,他们出访的时候第一站就到了北京,要想要让中国能够扮演更重要的角色,但是中国在未来在整个的中东地区能扮演怎样角色?这第一个,第二个就是普丁在俄乌战争还没有结束,过去两年里面他几乎什么地方都去不了,除了到中国,然后除了到伊朗一次,然后到了白俄罗斯,他没办法到任何地方,他所有的会议、国际会议他都不敢去。可是在此刻当美军在中东做了最大的军事部署的时候,他人竟然到了中东。这两件事情你怎么看?它代表什么?



王孟源 49:21 

这代表的其实两个月前我上个节目我就提到那时候以巴 冲突刚开始,就是说最大的影响就是把整个中东跟以色列和解的态势完全扭转过来,就是原本中东的这些产油国,尤其是以沙特为主的,对这些海外国家还有点观望的态势,虽然他们是基本上对中俄很同情的,但是因为以巴 这个冲突,他们完全就是站在第三世界这条线上,这是一个彻底的战队。



王孟源 49:57 

但是我认为以巴冲突这个我刚刚提到,它基本上是捅美国之必救。那全在全球的战略上,尤其是现在是殖民帝国与反殖民阵营的一个对抗的态势越来越明显的前提之下,你可以说基本上是迫使美国在所有的战线都开始退缩。嗯,我们刚才已经花了半个小时来讨论为什么在乌克兰,美国必须要退缩,其中以巴冲突就是最后的原因,也是最重要的一点,因为他必须要他无暇他顾,他必须要专心的来保护以色列,那在中东国家的影响力基本上已经放弃了,你没办法再去争取沙特,因为沙特现在已经跟伊朗和解了,然后经过以巴冲突以后,根本又不可能再站回美国那边。但是我觉得更重要的就是,我在提醒大家,美国真正眼中最大的敌人是中国,因为能够有足够体量,能够推翻美国霸权的只有中国。他之所以挑起俄乌冲突,其实是要先整合欧盟,进入北约。这样子,你在台湾挑起中美冲突或者是军事冲突的时候,欧洲不得不跟美国站队一起来对中国做经济制裁。那这个以巴冲突他对美国的威胁之大,优先程度之高,甚至高于对中国出手,所以我认为它最大的影响就是干扰了美国在 2024 年的一个战略选项。



王孟源 52:03 

什么战略选项?就是即便在俄乌战争打的还在焦灼状态之下,仍然硬着头皮在台海挑起战事来帮助 Biden的大选形势。这是一种战略冒险,但是你不能够排除它的可能性。那因为以巴冲突这个选项他们必须要放进。我再说一次,台湾的政坛轰轰烈烈讲谈国际事项,讲议题,做节目的非常的多,每天都有几十个。然后每天都有几十个人说今天的国际新闻跟台湾有关,是台湾影响的,我跟你讲百分 99.9\% 都是自欺欺人。因为人家根本就是不理你。



唐湘龙 52:54 

对,当然是。



王孟源 52:56 

你这个无关紧要的一只小蚂蚁。但这一次是一个例外。嗯,这个以巴冲突最大的影响就在于美国暂时选择放弃明年大选年,也就是台湾大选结束,美国大选还没有开始之前那段时间挑起台海战事的这个战略选项。所以才有习拜会。



唐湘龙 53:23 

FineOK 最好,但是好的在谈习拜会以前。谈习拜会以前,你对普丁呢,到中东的看法是什么?看着好像是在谈谈谈跟沙特之间的原油的。就是 OPEC+架构下面的两个国家的合作的问题。这么简单吗?



王孟源 53:43 

不是,也是战略。就是这个反殖民的战略战线要怎么样来反抗?除了这个,OPEC+是它的经贸的最直接的一个合作项目。但是你还有国际货币储备,尤其是替代美元这件事情,俄国是非常希望沙特来支援的,俄国已经做到了将近 100\% 了。他希望中国跟沙特也能够跟进。这才是釜底抽薪,否则的话美国永远都能够缓得过气了。为什么?因为他只要再多印美元就行了。他的那个军事不管怎么样乱搞,他不管国内的经济怎么乱搞,反正多印钱嘛。撒钱永远都可以解决。那这个钱只要全世界愿意接受,愿意拿这些钞票来换他们的劳动跟他们的矿产跟农产,那美国就一定能够喘得过气来,那所以Putin谈短期的事情谈的一定是 OPEC+,现在油价刚好跌到 70 块美元,对吗?必须要再进一步支撑了。那怎么样削减产能?这是一个很头疼的问题,因为都是要牺牲小我完成大我的事情。但是我觉得长期上来说就是这种经贸战略上的问题,当然以巴直接的问题你怎么解决?以色列当然是希望占领之后把它交还给巴勒斯坦的那个中央政府,就是要把哈马斯完全消灭,但是这个对阿拉伯国国家来说很可能是不能接受的。



王孟源 55:34 

嗯,所以这个还必须要争吵。那以巴的战事,我认为它的真正的影响是在于对美国,强迫美国做战略收缩。嗯,至于它的立即的一些战术细节,就是以色列准备在还在思考要怎么打才能够消灭哈马斯,然后成功的把这个加萨走廊转移给巴勒斯坦的中央政权。那不过我觉得要考虑对以色列很重要,对美国也很重要,但对世界其他国家不重要。对世界——对不起,我要再说一下——中国之所以愿意聚弄习拜会,就是因为要稳定,要消除……明年有两个原因,第一个是台湾。OK,因为你知道赖清德基本上是一定会选上的,对不对?原本赖清德的话,他绝对美国人叫他宣布独立他就会独立,一点阻力都没有。那现在这个习拜会的意义从中方的收获就在于釜底抽薪,保证美国不会做这种叫唆。当然如果以巴战事结束的很快,然后美国在明年年初就缓得过气来,说不定他们就反悔了,然后你知道美国人是没有什么信用的,但是所以这件事情中国是赌的,是以巴的,一个冲突能够拖一段时间,拖几个月,能够拖到明年下半年最好,我觉得可能拖不了那么久。但是美国人会不会反悔?你至少现在先得到一个承诺再说,这是现在能够拿到承诺。



王孟源 57:33 

第二个就是其他在搞事的,尤其是欧洲。你有没有注意到习拜会一结束立陶宛就就对中国示好,嗯,要求重新建立关系是对不对?同样的就是美国都已经站出来说,我现在因为以巴冲突的关系,所以我暂时不想搞中国了。那立陶宛还冲在前面干嘛?现在欧盟唯一还会给中国难堪的就只有冯德雷恩这种。



王孟源 58:03 

不需要选举的人。你像立陶宛这种,你不管说他们的那个民意有多么的愚昧,或者是这个政客本身有多么自私,他就算自己可以风光的退休去欧美支持着 NGO 拿高薪,他还是要担心一下自己政党下一次选举的选票。冯德雷恩完全没有这些考虑,所以冯德雷恩可以坚持跟中俄强硬到底,因为它可以把欧盟的经济搞砸到无限的烂,然后没有任何她自己带关心的后果。她没有政党,她没有选举,她这个经济搞得再烂,民意再怎么沸腾都跟他无关。所以你如果去看的话,就是立陶宛都已经反过来愿意跟中国和解了,但是今天还有昨天跟今天中国跟欧洲的这个经贸谈判,冯德雷恩还是绝对的鹰派,就是这样子。所以,然后你再看美国对在习拜会所做的承诺,很明显的有仅限于军事战略外交方面的,对经贸上面,对中国没有做任何承诺,你这个从哪里可以看出来?就是他几天前那个电动车规则的细节,所谓的外国实体的那个定义的细节解释凸了出来了,是大家认为可能最严厉的一种解释,也就是说只要整个电动车的生产链有中国人手上的有一颗指纹印在上面,你就不能够拿美国的优惠。那这很明显的就是代表着习拜会的时候中国跟美国谈的完全就是军事外交的,完全没有谈到经贸上,美国没有做出任何经贸上退让的承诺。



唐湘龙 01:00:14 

那我们在这两件事情当中,因为上个月的时候是在APEC的习拜会,这个主要是其实它习拜会,中美之间的这场的会晤,掩盖了整个APEC会议,APEC变得就不重要了,就是大家都要关注习拜会。今天当我们在连线的时候,冯德莱恩跟米歇尔就是欧盟的总统,现在正在北京,那现在中欧的峰会,那毕竟过去我们在谈就是说全球的政治问题的时候,最主要的三个的 player 就是中美欧。那中美的这场的峰会,它当中我是换一个视角,就是从中国的视角来看的时候,我当这是我很主观的感觉,我认为习近平在这一次的习拜会当中所呈现出来的一个领导人风格,他要展现出来的那种的嗅的味道是跟过去不太一样的,他花了一些的功夫做一些比较柔化自己的线条,也柔化中国线条的那个感觉,那不管在镜头上面,或者是之后的跟这些企业界的会晤等等,或者说刻意去邀他在美国的老朋友,在这场当中来讲,习近平做了很多在过去中美关系当中,中国几乎不会做的动作。



唐湘龙 01:01:29 

那这将习近平回来了之后,其实你会发现他把他几乎把他所有的公开的露脸都摆在跟财经有关的议题上面。跟过去几年不一样,过去几年习近平经常会曝光的都是去视察军方,去对军方讲话,这个比重很高。可是最近并不是,习近平只要有公开露脸的机会,官媒在刻意放大的到上海去看企业,或者去看央行,这些是习近平刻意去释放的中国的经济,习近平是不是有意思透过这次的习拜会去对中国的经济去做某一些的调整或刺激动作?



王孟源 01:02:08 

我想两个礼拜之后下一个节目我们就谈一下经济的问题。



唐湘龙 01:02:11 

OK,OK.



王孟源 01:02:13 

刚刚发表的那篇博文是其实是博客这 9 年半来的很多讨论的一个总结,那这个其中的一个重点结论就是中美脱钩是美国金融殖民体系,要对中国这个难以消化的大个子把他踢出这个体系,因为美国的这个金融知名体系,它是先放牧十年 收割两年之类,刚刚我们刚刚经过了这个收割期,你要说这个被收割,台湾其实这一次运气很好,只不过是物价飞涨了一下,承担了一些通货膨胀。但你看我刚刚讲过韩国的话,它今年的那个经济成长就是被 OECD 预期是负 - 1.5 percent,那越南他们的他的工业产值下降了将近 10\%,那这个就是典型的收割。那欧盟的话刚刚也谈过了,德国的这些产能去工业化,这些产能全都跑到美国去了吗?对,美国,美国在 20 年前它的每个月的工厂建设是还只有 20 亿美元。先我今年夏天回台湾演讲的时候,当时的数据是每个月 160 亿,已经成长了 8 倍。刚刚上个月最新的数据是每个月 185 亿,又继续往上涨了。就是你想想看这些在工业化,这固然是政策要求,但是你必须要有外国的企业把它的产能转移过来,这个全球的经济并没有在高速成长期。嗯,你不能够光是靠这个增量来投资,在美国必须要有一些转移。这些转移从哪里来的?从韩国来的,从德国来的,从其他欧洲国家来的。那这种就是所谓的收割。那我把中国排除,踢出这个收割体系,固然是像俄国那样子,不再有被收割的那个阶段,但是你在放牧的那个阶段,也就那是景气的那十年,你就没有办法拿到美国人来的资金、技术,更重要的是市场。中国今年的经济成长率非常的不理想,但是你如果去看原因是什么,它相比韩国相比德国还是很好的。对,那同样的你可以追溯到它这个不景气的原因就是美国的市场转移掉了。嗯,美国不但想方设法要把中国的产能拿走,第一,最简单的就是对市场做障碍。那今年的数据还没有出来,所以我上个礼拜去自己把月份的那个数据加了一加。中国今年前四个月对美国的出口加起来刚好是 4200 亿美元,你如果平均每个月 420 亿的话,再加最后两个月,今年应该是刚好是 5000 亿左右。一年之前, 2022 年中国对美国出口是将近5900 亿美元。



唐湘龙 01:05:53 

这样一个大幅衰退的状态。



王孟源 01:05:56 

对,衰退了17\%。



唐湘龙 01:05:59 

快两。成。



王孟源 01:05:59 

你在考虑到这个都是用当年的美元,不是用通货膨胀算的。等值美元的话,过去这一年的美国的通胀率大概是5\%,美国的利益率也大概是5\%,你再把这个 5\% 算下去的话,其实中国对美国的出口在过去这一年下降了超过20\%,相当于大约 1000 多亿的美元的的份额。这 1000 多亿的市场消失就是中国现在经济不景气的最根本原因。这个能不能够扭转?没办法扭转,你就算中国跪着去跟美国求,美国也不可能再把这个市场给你。



王孟源 01:06:49 

所以,习近平跟他的团队,尤其是今年整个国务院都换新的团队,我觉得,我注意到它里面终于有一些比较像样的经济跟金融幕僚给出比较切实际的建议,他们认清了这个中美脱钩,至少美国单方面把中国踢出他们现有的美国主导的体系,这个是无可逆转的话,你一定会有逆风。现在你看到的只是市场,接下来资金还是一个更大的问题。



王孟源 01:07:25 

因为美国的体系用的是美元,因为是全球通行,所以它的利率一定是特别的低,他们只有在十年一次的收割期才会利率调高上去,那其他的景气的阶段你就可以享受很廉价的资本的价格就是利率,你利率低就是资本的价格低。所以中国第一个损失的是市场,这个是无可挽回的,今年 1000 多亿,未来这五年至少还是一两千亿的市场损失,就是额外的,还要额外的,你不可能不但不可能反弹,而且还会继续的损失。



王孟源 01:08:13 

第二个技术,这个还好,现在目前仅限于半导体,这是因为中国的技术发展已经到了,除了半导体以外,基本没有什么需要跟美国人求的。于是未来几年最严重的会是资金的问题,因为你如果继续留在美元体系的话,就必须被它收割。你要离开美元体系的话,你怎么离开?你的资本的成本,资本的价格都会上升,所以这是一个很强大的逆风,也就是说你投资的报酬率要跨过的门槛,就是能够你的利润,能够支付你的资本的利息的那个门槛要提高了,也就是说你的政府跟企业的效率都必须要提高,否则的话就会亏本。



王孟源 01:09:15 

那这我觉得是过去这一年新任国务院的那些幕僚跟总结出来的一个结论,所以他们认为中国目前的当务之急是要苦练内功,加强自己的行政效率跟企业的经营效率,所以他们的重点转移到这里。这也是为什么一旦以巴冲突一发生,而美国愿意在军事外交的冲突上面妥协的时候,中国立刻抓住这个机会。因为中国需要这个喘息,需要面对内部来做改革。



王孟源 01:09:55 

你看他过去这一年多两年先整顿的是什么?房地产,然后接下来是银行体系,为什么?因为这个金融、房地产这些是财政跟资金的血脉,你这些东西不通畅的话,继续像有恒大那样的浪费的话是不可持续的,对不对?那所以他先优先处理这些人,你既然是在优先处理这些,而且这目前这个只不过是第一步,接下来还有很多很多困难的事情要开始做,因为房地产牵扯到的幕后是什么?地方政府的债。没错,他们的地方政府的支出以前是靠房地产的,所以现在你房地产要整顿,中央跟地方的财政怎么分配?嗯,必须要彻底检讨,这个都是伤根动骨的概念。



王孟源 01:10:56 

嗯,那你可以理解为什么习近平认为必须要专心致志地来处理这些事情?为什么在对美的外交对抗,虽然本质上是一种反殖民的斗争,是一种你死我活的斗争,但是一旦有一个短暂的喘息的机会,还是必须要把握住,就是暂时的和解来给予自己增强内功的机会。就认为正确的解读是这样的。



唐湘龙 01:11:31 

好,孟源最后这段的分析我觉得很有启发性,对于大家,特别在许多的中国朋友、华人朋友,大家都很关注大陆的经济到底现在是一个怎么样的状况,他固然有他一些表象上的危机,房地产这些我们都是看得到的。但是当这些西方的这个信评机构开始在调降了中国的信评,然后把它转为负向了之后,在资本市场当中你也看得到的,就是就是大陆的股市是跌跌不休的,那之前力道非常非常弱,现在又跌破 3000 点,但是。



王孟源 01:12:10 

你说的是穆迪调降吗?这个不是战略的打击,他们所说的言之有物,是真正的问题。问题是他们认为说这是不可解决的。其实我认为,我很乐观,我审慎的乐观就是因为中国政府的态度,就跟我刚刚讲到的 BYD 一样。认识到这个短处了,他承认我们有这个问题,而且我愿意做出牺牲来解决,不像美国人,他们唯一的办法就是以邻为祸。



唐湘龙 01:12:42 

对,没错。



王孟源 01:12:43 

滥发美元。



唐湘龙 01:12:45 

好,当然有关于经济的大大分析。刚孟源已经讲了下个星期的时候,这个月有两次王孟源的时间,那在过去很多朋友说那要加更,没办法加更好,但这个月可以更两次,所以下个星期的时候孟源我们再针对,就是说一些总体经济的问题的时候,再请孟源为大家做准备,再请孟源大家上课。好,今天的这个时间到了,来感谢我们一些的听众,观光观众朋友的这个,这算是 14000 多位的观众在我们的线上,其实孟源的内容,孟源的议题本身是有门槛的,就是你我会,我高度的建议就是大家除了听之外,你要很专注的听,而且有一些的细节的部分来讲是可以拉出来在做功课的。这个是我在长时间看王孟源听王孟源讲话的时候的心得。



拜谢部分省略



唐湘龙 01:13:38 

来我的WKR,谢谢。人类社会永远需要无私为公益奋斗的人,那他也许是少数,但你可以期望每一个人都是这样,而且事实上他永远是少数。但是一个制度的优劣其实是取决于是否容许这些有理想以公寓至上的人能够发挥好。再来这个 KG 顾党,感谢。再来 broad CAN 001 谢谢,然后冯一伟,谢谢,然后一烟尼王,感谢,然后 go go 等三九,感谢他,王博士向龙哥,那请到王博士,那如果特朗普当选的话,对中欧及中欧关系会有什么影响?这个大家问,我觉得下这个情节的时候,让孟源来回答一下美国的国内这政治是未来一年当中观察就是说国际事物的一个非常重要的角度。再来雷欧。好,那上个月呢,有人把王博士直播忘了,我不说是谁了,只是忘了说我啦,哈哈,是我对,我承认我梦人有跟我讲。



王孟源 01:14:45 

但是脚我刚好那天。



唐湘龙 01:14:46 

上飞机,他那天呢就在飞机上面说我根本联络不到他,我紧张紧着紧张死了。我说了我紧张的是因为这王梦云去哪里了?王梦云会不会生病了,然后他一个人的孤家寡人数呢?我担心王梦月没有人照顾,我在担心这件事情。好了,那 CCRN NCN 来感谢王博士,再来 BYOEN 然后肖,好的,感谢王梦先生。您好,请问中国近期的经济金融的政策,人事调整与主席的视察是否有跟您的这个就是说博客中的论述跟预测相符,刚我已经有回答过了。那威尔森真感谢来支持王博士,希望多看到两位,感谢。然后汤姆感谢黄,应该是姓黄,感谢。然后柏林丽,感谢支持龙行天下90K,感谢你田野,谢谢。有人说内地经济不景气,那你应该关心一下内地的车市,而且今年的比亚迪已经杀疯了。



唐湘龙 01:15:48 

对,刚刚也有稍微提到过一下。那么美丽娜张,谢谢你,林冠,谢谢你,谢谢谢王孟源老师的开导。然后肖明小明同学想问一下王老师对当前的中国股市怎么看?那新华社接连发布了中央金融主管机关负责人的访谈,那可是股市仍然是惨兮兮的。孟元刚有讲过哈,它其实是审慎乐观的,不要是太受到了目的性频机构的影响,最少中国是有自觉,它现在的面对到了一些的困难,而且它因为有自觉,它也积极的面对,而不是那种就是具剑法的。我最怕就是政治人物做一些空洞的谈论,只有加油打气的那个很空虚的喊话,如果只是这个样子的话,那中国就值得担心了。但是中国官方的应对的策略,到目前为止我觉得最少态度是实事求是的。这梦圆刚刚讲的这很重要,好时间的关系,非常非常感谢重新出现在龙行天下的王梦圆。感谢。



王孟源 01:16:56 

很荣幸跟大家聊天。好,两个礼拜后再见。我会讨论一下我最新的博文,因为里面讨论谈的就是金融殖民这个体系,然后这个体系之下会有的变动,以及中国被踢出这个体系所必须要做的应对,都在我的博文里面讨论过了。所以如果有人愿意先预习的话,欢迎你到我的博客去看看。OK。



唐湘龙 01:17:25 

好,感谢孟源的连线,跟大家说周末快乐,下回见。拜拜。



\twocolumn[\begin{@twocolumnfalse}
\section{中國、美國與ECFA}
\subsection{20231222}
\end{@twocolumnfalse}]唐湘龙 00:11 

欢迎来到龙行天下,我是唐湘龙。在今天星期五的时间,在周末之前,同时这个周末今天是冬至,就是北半球的冬至吃汤圆的日子,今天大概在冬至的准确的时间点大概会落在今天的 11 点多,就接近中午的时间。第二个就是说因为冬季,那这个周末是Christmas Eve 下个星期一就是从亚洲时间来讲,下个星期一 12月25 日就是耶诞节,所以在西方的西方国家来讲,大概都已经过年了,过耶诞的气氛都已经非常浓厚了。再来就是这个时间点刚好非常强烈的冷气团正在影响大陆跟台湾地区,所以天寒地冻的。



唐湘龙 01:08 

好,那就跟大家说冬至快乐。好星期五的时间,这个月我觉得龙行天下的观众朋友特别期待,特别幸福,就是我熬王孟源两次,熬了上个月,错过了王孟源,所以就跟他不断的扒,可能为了熬他两次,好,那他刚好因为已经到 2023 年底了,所以我就请孟源说好。那我们在概念上面来讲,因为今年的事情很多,而且预判明年也仍然是会是多事之年,那不管从一个总体的国际政治的框架,那两个星期前请孟源帮大家做了一个总体的分析整理,做了一些预判。好,今天我们把重点摆在过去,许多的观众朋友敲碗,希望孟源发挥他的专长,尤其王孟源在西方的金融体系里面工作很长的时间了,其实这些是我一开始我认识王孟源的时候,我最心动的地带,地方就是我在听王孟源说话的时候,我觉得我周围很少有人,在一些的国际来讲,虽然是自己是受到了就是基本上理论物理的自然科学的严谨的训练,可是在社会科学的领域里面,在总体经济的领域里面来讲,又有很好的实务经验。



唐湘龙 02:35 

今天请孟源就我们的标题当中的三个大块的主题,本来我在犹豫说要叫孟源谈谈ECFA,会不会有点为难孟源,不过孟源的自告奋勇说要不要谈一下ECFA。好,那今天在我们线上的,现在在天寒地冻的美国的东部,而且时差 12 个小时,这边亚洲时间早上的 9 点半钟,在美东时间就是已经开始准备进入深夜了,来在我们线上的王孟源,欢迎。



王孟源 03:06 

大家好,非常高兴再跟大家聊一聊。



唐湘龙 03:09 

好,那当然。孟源刚跟我说你那边现在两三度左右,对不对?天气是蛮冷的。但是来我们回到今天的主题,就是在封关之前的时候,我们把整个的国际经济的今年的、大的、势头以及接下去明年会是一个怎么样状况,做一个大的研判。那 ECFA 是一个突发的议题,不能说突发了,酝酿非常久。这个我们放到最后面,然后我们先从美国谈起好了,美国的经济发展变数非常多。



王孟源 03:43 

我们必须要从美国谈起,因为我今天准备谈的是从国际经贸开始,最后专注到我对大陆的政策的一些建议。那你如果要谈国际经贸的话,你不能够避开一个很严重的事实,就是即使在 21 世纪的今天,美国还远远是全世界的主导者。那英文里面有一句话是:狗摇尾巴,不能够尾巴摇狗。那在这个国际经贸的议题上,美国就是那只大狗。中国也算是一只体量蛮大的狗,但是像是台湾、韩国甚至欧盟现在感觉上都像是尾巴了,都是被人家摇来摇去的。所以我们要谈的话,就必须要从那只狗,那只最大的狗身上开始谈及。



王孟源 04:44 

美国在过去这三年遭遇了一个非常严重的通胀危机,是 50 年来最严重的一次危机,但是它很成功地缓解了,比 50 年前缓解的还要成功得多。这原因是 50 年前他们没有经验。当时的主要竞争对手(也就是其他的先进工业国家)是欧洲还有日本,他们被美国切断了。美国自己单方面地放弃了Bretton Woods系统,就是原本金本位的那个系统,美元锁定黄金,然后其他的货币锁定美元。



唐湘龙 05:39 

就是战后的金融秩序的基础了。



王孟源 05:42 

(战后)二十几年的基础,然后主要是因为越战造成了财政赤字,那尼克森决定我们要无限印钞,所以要无限印钞就不能够再继续锁定黄金,对不对?放弃金本位,他们就放弃了这个金本位的规则,然后美元以后就实施无限印钞。以上是 70 年代他们通货膨胀不断的恶化的原因。一直到 1980 年到达高峰,当时的通胀最高曾经达到 14\%,这是美国本身的膨胀,这里面有三个因素:第一个是美国自己开始无限印钞;第二个是当欧洲跟日本去跟美国抱怨的时候,美国跟他们说,你们也跟着我印钞嘛,反正我印你们也印,对不对?这样还是公平的对不对?结果这样一来就变成全球都有通胀,而不是只有美国一国的通胀,所以事情就快地恶化;第三个就是他运气不好,遇到了两次的能源危机,所以又两次地把它的通胀给推高了。



王孟源 07:01 

这一次的通胀其实...美国在那边无限印钞,它的规模比 50 年前要大两个数量级,尤其是在 2008 年之后,它开始有一个金融创新,货币政策的创新,叫做量化宽松。他们一共宽松了 8.9 万亿美元。这个比 50 年前高了两个数量级。那照理说它造成的通胀压力应该也高两个数量级,但是事实上没有。没有的原因是因为刚好 21 世纪初的头十年,全世界尤其是亚洲这种出口走向国家就是中国、香港、台湾、南韩、日本,甚至泰国、马来西亚等等,新加坡等等。他们的外汇储备总额从 2000 年的1万亿增加到 2010 年的 11 万亿。



王孟源 08:13 

OK,那为什么这些国家这么傻?他们把他们自己辛苦劳动所得所获得的美元重新地堆回去买美国的国债,主要是美国的国债,然后导致美国的国债它的合理价位,合理利息应该是 5\%,6\% 到最后被推到低于2\%,三年前最低的时候到 1.5\%,每年 3.5\%的利息差。



王孟源 08:48 

30 万亿的国债,大家自己算一下,你省了 3.5 或者是 4 \%,30 万亿的国债一年省多少钱?一年就是省一万多亿,20 年就省多少,对不对?明面上就是 25 万亿。但是你再加上债上滚债的话,事实上就是省了 30 多万亿。刚好那十年增加的10 万亿外汇储备就被美国通通用来填充他的财政赤字,然后消化美国多印的那9万亿美元。



王孟源 09:33 

为什么亚洲国家会这么傻?他们并不是傻,而是因为上一轮美国的对外收割,美国这个金融殖民帝国对外收割是 1997 年跟 1998 年,他针对的对象刚好就是亚洲的这些出口国家,死得最惨的,像南韩、印尼、泰国、马来西亚,他们这个当时的手法用的就是Soros 在 1992 年先对英国尝试的那个版本,就是对你的美债(下手)。不是国家去跟美国借钱了,而是你的企业...因为美元是国际储备货币,所以它的利率特别低,然后流通性特别高,特别容易举债借大笔的钱,他们看上那些低利率的债,然后就去举债。刚好 90 年代是一个冷战后的红利时代,大家的全球化刚刚开始,然后这些国家都有很繁盛的经济,主要由美国的市场,美国的公司来投资,来给你技术,来给你产能,然后这个产能产生,生产出去以后再外销给美国这个市场,那这么一来他们就拼命地顺便跟美国的银行借贷。结果到了 1997 年,美国人来了一个双管齐下,一方面银行不再给你 roll over,就是滚债,在到期以后他不让你借新债还旧债了。



王孟源 11:13 

然后另外一方面,Soros 这些对冲基金来攻击你的货币,这个时候这些国内的企业拿自己的资产...比如说韩国的企业三星,它的资产,他手上的现金都是韩币...他要还这些美元债,必须要把韩币拿去跟他们的中央银行换成美元,来还美元债。所以你能够撑得下去的一个前提就是,你中央银行的外汇储备要超过你国内企业所欠的那些美元债的总和。



王孟源 11:58 

Soros之所以能够确定他会赢,就是盯上了...他就是他做了一个简单的算术,发现你们这些企业这两年要还的美元债加起来超过了你的外汇储备。所以我只要一直这样盯下去,你总会到还不出来的时候,后来还不出来的时候要怎么办? IMF来...南韩求 IMF来拯救,IMF说:好,我们坐下来谈,这么一谈就谈了 6 个多月。为什么谈 6 个多月?不是因为真的有事情要谈,而是因为他们开的条件韩国人不想接受。他们的开的条件是什么?你所有的优质资产要完全对美国的金融资本开放,三星要变成美国企业。



唐湘龙 12:51 

非常狠了



王孟源 12:55 

韩国人不愿意,拖了 6 个月。拖这 6 个月是什么呢?他们想尽办法到处去借,到处变卖黄金什么的。到了 6 个月之后终于再也挤不出1美元出来了,他们被迫定下城下之盟,向 IMF 完全投降。三星从那个时候开始就变成美国人的企业。台湾当时并没有经过这一段。为什么?因为台湾台积电早就是美国企业了,不需要被逼,台湾就直接投降了。香港那个时候差一点也走上这条道路。



唐湘龙 13:42 

没错,中国出手。



王孟源 13:44 

对。但是不论他这样的敲诈,这样的压榨有没有成功,事后大家都是学乖了。以后大家知道,你要能够抵抗这种供货币供给的时候,第一个条件就是你的外汇要够多,所以大家才会在十年之内把外汇增加了 10 倍,那增加的那十倍,也就是 10 万亿美元,刚好就被美国人立刻拿来做最大的利用,最大利用就是拿来做财政跟货币的支出,他的美联储就多印了9万亿,它的财政赤字,国债就增加了十几万亿这样子。



王孟源 14:37 

我之所以谈这个,是要先让大家知道,过去这三年的通胀危机其实已经是美国在 1970 年代放弃Bretton Woods后,(对外搜刮)的第五轮了。1997 年那次只不过是第三轮而已。而且每过一轮它的这个规模就扩大,然后它的效率就提升。而且它往往都是针对你上一次的经验教训。受害国作出对应措施,那些对应措施刚好就可以被他利用来,在下一次尽情地压榨你。



王孟源 15:32 

1997 年不是大家学乖了,就是要提升外汇嘛。那十几万亿的外汇怎么办?你都只好去买美元资产了,因为它是美元的外汇啊,对不对?你买美元的资产到最后是怎么样子? 2008 年的那一次收刮就刚好反过来,既然你累积了十几万亿美元的外汇,而且被迫买美国境内的美元资产。就可以从上一次 10 年前你去低价掠夺亚洲的优质资产,转成 2008 年是把美国内部的劣质资产以高价卖给你。想想看, 2008 年的那个危机叫做什么?叫做次贷危机。对,次贷是什么东西?就是美国国内的劣质金融资产



唐湘龙 16:24 

烂债权,哈哈。



王孟源 16:27 

对。那我们这一次终于...过去这三年其实是 50 年来美国经过五轮的金融压榨,又回到了第一轮,就是 1970 年代的那一次通胀危机,而且他们因为有了上次的经验,这一次他们对欧盟、对日本的政治跟经济主导权要强得多。



王孟源 17:01 

你在 1970 年代的时候,欧洲国家还有日本,他们还有他们的财政政策、货币政策,还有一些独立性。他们在 1971 年抗议之后,不得不跟着美国发债,不得不跟着美国印钞,结果全球通胀搞成滞胀之后,他们觉得这样不对了。他们(现在看起来是)完全是美国的殖民地,予取予求。但是你要知道, 1980 年美国为什么经过十年之后,它的滞胀危机才会到达最高峰,原因是在 1980 年,日本这些还有欧洲国家开始弃用美元。你如果看美元占当时国际外汇储备的比率,从 85\% 一下就掉到60\%,一年之内就掉了 20 个百分点。为什么?这些国家忽然明白被他们(美国)这样剥削,你即使自己跟着印钞票,只要美元仍然是国际外汇储备所通用的储备货币,你印钞票一定是你吃亏多,占便宜一定是占的少。所以他们到最后终于理解到根本是在于这个国际储备货币的地位。而他们从 1980 年开始互相持有对方的货币来作为储备货币。当时 1980 年代,德国马克跟日元跟英镑都是,那个时候一下子,他们在各国中央银行的储备占有率一下子上去了。你跟 1980 年代 40 多年前,他们的那个脊梁骨对比一下,你就会发现三年前完全没有这回事。



王孟源 19:20 

就是从 2021 年初到 2022 年中,美国的通胀危机最严重的时候,欧元一下子兑美元贬值了20\%,人民币兑美元贬值了14\%。当时是什么情况?当时是新冠,大家抢着...而且美国又多印了3万亿的财政紧急刺激。当时美国国内,大家蹲在家里面,每个人都在买电子产品。买电脑,买平板电脑,买新的手机,买运动脚踏车,这些东西都必须要从中国进口。你在洛杉矶的外海有 100G 的超大型货柜轮,每一个货柜轮上面都有 24000 个货柜箱,在那边排队没办法进来,在这种环境下...人家抢着要买你的东西,不管你的价钱多少...你反而还贬值14\%,你说这是多么的多么的乖,多大的自我牺牲精神。



唐湘龙 20:48 

好,这个地方我打断一下。孟源在过去有几次提到认为中国的央行在过去在面对到中美贸易结构问题的时候,采取了错误的金融政策,讲的就是这件事情,对不对?



王孟源 21:05 

这是最恶劣的,上一任的就是易纲,今年年初的时候被换掉了。但是他其实犯了 3 个很大的错误,最小的错误是他没有想到这个通货膨胀、美国印钞会导致美国长期国债的利率上升。从我刚刚讲 1.5\%增加到 5\%左右,这导致了我们今年年初4月的时候有谈过那个硅谷银行就是 SVB 破产。 SVB 就是因为这样损失了 1000 亿美元而破产。



王孟源 21:46 

那现在最新的统计是,美国所有的正规银行因为持有美国长期国债,在过去这一年总共损失了1万亿美元。我刚刚提到所有的亚洲国家的总外汇储备已经提升到 11,12 万亿左右,这个比美国正规银行的储备还要高两三倍。那你可以想象这些中央银行损失的数额是不是也应该比美国正规银行高两三倍?这些所有的亚洲央行的外汇储备里面,中国占 1/ 4,那中国赔的是不是就应该?嗯,最多对不对?那这几千亿,这里面几千亿美元的损失,这些是浮亏了,就是即使在会计上面账目账都不会看出来的,因为它实际上的损失是你这些 30 年的债券,你现在拿到 30 年之后,拿回本钱之后,那个本钱只值现在的 1/ 10,OK,是这样的损失,所以是完全是隐性的。那,他们那个前任的人民银行行长易纲,易纲造成这个几千亿美元的损失,我认为是他所犯三个错误里面最小的。



王孟源 23:18 

OK,第二个比较严重的错误是我刚刚在一开场讲的,明明可以让美国有一个很严重的通胀危机,这样一来他自顾不暇,就不会去挑起,不会去挑起俄乌战争,但是他乖乖的违反了他自己表面上相信的美国货币理论,就是美国经济系所讲的货品理论,你如果中国的货品,中国的出口货品在供不应求的时候,你的货币为什么要贬值?你的没有一个国家的经济系的经济理论会跟你讲应该贬值。



王孟源 23:59 

但是美国的美联储打电话刚好我这边有一个美联储的,哈哈哈哈哈哈,OK,对,美联储是不让人家去观光的,你必须是美联储的职员职工才可以买的。OK,OK,好,OK,我这是有人送的,哈哈哈,美联储跟他讲叫你贬值,他就乖乖的贬值,所以我现在都把人民银行叫成美联储北京分行。OK,这是他犯的第二个错误。第三个错误就是一年多前,去年年初的时候,俄乌战争一开始,很明显的俄方的中央银行想要跟中国的人民银行联手,是推出一个取代美元的新国际储备货币,嗯,结果被他搞黄了。这个我刚刚讲了,是釜底抽薪解决美国金融收瓜的。对。



唐湘龙 25:07 

这是我想问的问题。



王孟源 25:08 

对,这个所牵涉到的,我刚刚讲易刚所犯的第一个错误,代价是千亿美元,嗯,几千亿美元。他所犯的第二个错误,让美国这一次很轻松的摆脱金融危机,这个可能代价是几万亿美元,但是他把取代美元作为新国际储备货币这件事情搞黄了,这个代价我认为是十万亿、几十万亿美元的。



王孟源 25:41 

嗯,代价。为什么?美国我刚刚解释过去这 20 年的金融收刮就是二三十万亿,嗯,你让美国能够活到下一轮,再做下一轮的收刮,至少是 10 万亿的出入,而这个钱是谁出的?中国人出的?嗯,全世界的第三世界出的,当然欧盟也出。台湾现在的通货膨胀回事人就是也是因为美联储要把通货膨胀外销这种出口。对,OK,所以台湾只拿到通货膨胀,而没有到 GDP 倒退收缩的地步,像韩国或者是越南已经是很好了。谢天谢地,不过这个不是因为美国人仁慈,台湾之所以过去两年在这种中型的被放牧的经济体,因为美国的这个金融斗国帝国,他们是周期性的半个月每十年收挂一次。我的一个类比就是他把这些靠美国,原本靠美国技术,还有还有那个体制,还有它的市场所开发出来的新兴工业国,其实是一种放牧的现象,就是放牛,放羊。嗯,那个他拿这些市场,拿这些技术,拿这些产能当做饲料,给你吃饱了以后,美国每隔十年他来收割宰杀一次。嗯,那这个台湾这一次的宰杀结果过得相当不错。跟其他的那些欧盟,你看现在德国已经连续 4 个基金季衰退。没错。



王孟源 27:33 

英国报告今年的经济成长有将近一 percent 大家,但是大家不要忘了,这是因为他年初的时候改了 GDP 的算法,是,所以找出了两 percent 的GDP,这跟他实际成长一点关系。英国的 GDA 成长是它的国家统计局制造出来的,不是它的经济制造出来的。



王孟源 27:59 

那韩国的经济今年已经注定是要衰退了,越南的出口跟工业产值都要下降 10\% 到15\%,所以你说台湾为什么会一枝独秀的原因很简单,就是ECFA,嗯,阿克法克,嗯,一般的台湾人不晓得这个 ECFA 给台湾占的便宜有多大,嗯,台湾通过 ECFA 对大陆的出口占台湾GDP,不是占总出口,占台湾 GDP 总产能的24\%,OK,台湾从大陆的进口只占台湾 GDP 的11\%,所以台湾 GDP 凭空多出来13\%,这个是它的那个顺畅,OK,而且这里不只是大陆人民买单,台湾人民也跟着一起买单。



王孟源 29:04 

我这两年回去台湾了,一个很大的感慨就是凡是大陆很强的这些新兴工业,嗯,你像是光伏、电动车,嗯,这些东西台湾都是远远的落后,因为你都不让大陆进口,嗯,它不但比大陆贵 4- 5 倍。举个例子,我在我去深圳的时候有研究一下当地的那些电动车,还有电动自行车的价钱。我回到台湾来再看,完全一样的电动自行车,电动车真的就是大陆价钱的 4 倍,嗯,OK。然后台湾的那个,台湾的那个 go go 楼是一个,那个换电池的,对。



唐湘龙 29:57 

嗯,电动车、电动摩托车。



王孟源 30:00 

那个经营模式完全都是美国式的,它的定价也都是美国式的。我的,我说美国市就是比大陆要高 5- 10 倍的那个价钱。很多那个台湾的用户自己抱怨说加入 Google 肉之后比以前骑油车还要贵一倍。哪有这样子的,你这种插电的东西怎么会比烧油的还要贵一倍?这就是因为它定价过高。它为什么定价过高?因为没有竞争。



王孟源 30:30 

嗯,为什么没有竞争?因为你把大陆的供应商排除在外,大陆的电商,台湾的电商根本跟大陆没得比,你没有电商就没有竞争,台湾不只是比不上大陆,连美国消费者的那个便利跟权益都比不上。嗯,所以你这些 ECFA 这些不公平的现象,台湾这些对大陆的这些占便宜,吃亏的不只是大陆的厂商。嗯,吃亏的有很大一部分也是台湾本身的消费者。



王孟源 31:07 

没错,就像嗯,现在的中美脱钩,美国就是因为对大陆没办法轻松的宰杀。嗯,你一次别人像大陆这种体量,对,从别人那边应该是拿到5万亿、6万亿,结果只从人民银行那边拿到几千亿,这种零头他觉得不划算,所以他要把他的饲料是什么技术,就是中美为什么科技脱钩?为什么他学术界不让他你再交流?为什么他会把在美国的华裔科学家嗯数或者关起来,弄得风声鹤唳,大家不敢再跟中方合作,为什么?就是为了要把技术拿走。嗯,你的下一步就是市场拿走。现在中国的电动车,它的企业已经变成世界第一了,中国今年会变成全世界汽车出口最大国家。



王孟源 32:15 

他去年超过德国,今年会超过日本。那但是你在美国一辆中国制的电动车都看不到。嗯,对,所以美国的电动车就硬是比中国贵一倍。嗯,所以现在美国的福特在这个月刚刚宣布,因为电动车卖不动,所以他把投资电动车的投资要大幅的砍腰斩。嗯,为什么卖不动?因为太贵了。为什么太贵?因为你不让大陆的进来,抱着电动车进口,所以基本上是跟台湾一样。台湾的电动车是不是也比大陆贵一倍多?嗯,对不对?你比大陆贵一倍多,那接受率当然就不行了。大陆的电动车已经比油车还要便宜的,是,对不对?那现在在美国呢?你如果看到路障有人开特斯拉还是一种奢侈品,嗯,对不对?你还是那种年入年收入 20 万美元以上的家庭才能够玩得起的大玩具。嗯,对,那台湾的话就更离谱。在台湾的话,基本上泰斯拉跟法拉利一样的罕见,嗯,对不对?偶尔看到一辆,对,这当作奢侈品,你说这是正常的经济发展的那个的常态吗?不是,你这是在以剥削消费者为代价,让自己国家的整个工业水平落后、隐藏落后来为代价来独立自己的国内的官商勾结的利益集团。



王孟源 34:06 

OK,所以,美国为的资料我刚刚讲了技术市场最后一个是什么投资?嗯,OK,美国现在这个已经撕下他的面具了,你的根本就别想拿到美国的投资,而且还在撤资。嗯,从大陆大幅的撤资,那毕竟你市场已经都没办法进了,那美国厂商还跟你投资个什么?嗯,对不对?那为什么我上个月有一个大陆的朋友给我微信跟我抱怨说为什么大陆的股市又跌下去了?为什么?嗯,他想办法投资港股,连港股跌得一塌糊涂。是很简单,因为撤资,因为中美在脱钩,所以那些外国的资金在撤掉。香港现在正在看到大规模的服务业从香港向新加坡转移,这个是有益的。因为香港是中国的一部分,所以英美主控的金融业、服务业必须要。



唐湘龙 35:31 

离开香港。



王孟源 35:32 

从政治要求而撤离。所以你这些国际资本撤离以后,你自然就泡沫吹不起来。嗯,美国现在它的股市又上去了,你只要看,你要看美国,美国恢复的有多好,你只要看它这个泡沫是不是又重新起来。三年前它有一个 everything bubble,每一样都是泡沫。OK,当时最泡沫是什么东西?就是加密货币。嗯,Bit Coin那个时候到6万美金,嗯,现在Bit Coin 跌到1万美金之后,现在又回到哪里了?又回到4万多,没错,对不对?你光看这一点就知道美国现在的经济态势好的不得了。嗯,在所有的工业国家里面好的不得了。为什么?因为欧盟跟中国、跟日本跟韩国跟台湾跟越南这些国家都让他尽情的宰割。英国也是他拿到的,那这三年拿到的是又是十几万亿美元,然后最惨的我觉得是德国了。为什么?德国损失的不只是钱,不只是资产,不只是资金,他损失的是真正的工业产能。嗯,因为他自己把自己的廉价能源来源给割断了。没错,自己割了。所以大家一直以为德国最大的工业是汽车,其实汽车是他们一整套的工业体系里面一个很下游的,它真正的核心是化工,从一百多年前,第二次工业革命的时候,德国领先全球的就是化工。那化工最需要的是什么?是廉价的天然气,哈哈哈,那所以德国的化工工业基本上一系之间就完蛋了,全部都转移到美国去了。



王孟源 37:50 

美国在 2000 年的时候,他的每个月的工厂建设投资是 20 亿美元,每个月 20 亿美元。到 2020 年上一次美国大选的时候,提高到每个月 60 亿美元,这就是Trump搞了 4 年这个美国化的结果,一切供产业化结果。但是你知道今年夏天我到台湾给演讲的时候,这个数字已经提升到每个月 160 亿,这个月我刚刚又去看最新的数字是 190 亿。



唐湘龙 38:38 

嗯,从 20 亿到 190 亿。



王孟源 38:41 

从 20 亿到搞了 20 年才到 60 亿,嗯,包括创那四年,然后过去这三年一下子就提升到 190 亿。嗯,大家可以想象到中美脱钩,还有从欧盟吸血吸实体产业这个政策有多么的成功。嗯,对不对?因为你这个工厂是用来干什么的?用来实体产能的。为什么会大家拼命建工厂?因为这些美国的厂商你只要是要针对,只要心里还想着美国的市场,你就必须要从中国撤资,因为中国出口不能够再进美国了,所以你就必须要在美国自己产。



王孟源 39:30 

同样的,欧盟现在失去了廉价的能源,然后他又因为货币贬值的关系,在最不应该贬值的时候贬值了百分之二十几。导致欧盟的通胀比美国还要严重。乱印钞票是美国,结果两年前开始到现在,一开始的时候是美国的通胀高于欧盟,结果两年前开始因为或汇率变动的关系变成欧盟的通胀告于美国,一直到现在美国的通胀已经压到3.7,欧盟的通胀还是远远高于美国的。



王孟源 40:16 

那这就导致什么?导致美国的这些现在美国最大的产业是什么?金融产业就是那些,尤其是影子银行。所谓的影子银行是什么?保险公司、对冲基金,还有其他各式各样的私募基金什么的。嗯,他们即使在我刚刚讲到美国的这些资产的泡沫已经重新吹起之后还觉得不过瘾,还要再把它重新吹到三年前的最高点,所以他跑去跟美联储主席要求要降息。嗯,原本美联储的计划是还要再升一格,然后才考虑是不是要降临,结果一个礼拜之前他们最新的会议,嗯,Powell投降了。



唐湘龙 41:14 

准备要提早降息。



王孟源 41:15 

了。他已经承诺,嗯,明年会降息 3 次,而且不再加息,就是承诺明年降级 3 次。事实上现在市场已经预估会降级 4 次,但是他承诺只承诺了 3 次。这是什么?就是对金融业投降,重新的要把泡沫最大化。嗯,那偏偏他还可以这样干。为什么?因为现在欧盟它就是通胀比美国还要高,所以虽然欧盟的经济没完全没有复苏,德国的经济还在衰退之中,然后其他国家也都是成长率零点几,在这种情形下,它反而没办法降息,欧元银行反而没办法降息。所以这样一来美国在景气下来,你想想看又是怎么样子?又是更多的资金跟产能向美国转移。



王孟源 42:17 

嗯,那所以我讲了半天想要跟大家讲的是美国现在真的就是用英文讲,就是 sitting on the driver seat。嗯,亦凡是为什么?过去这三年其他的国家实在太乖了。嗯,除了恶国之外,没有一个国家的中央银行有脊梁骨的。嗯,全都是无脊椎动物。那在这种情况下,中国的对策是什么?第一个,中国一定已经要损失来自美国的技,直接技术转移的。嗯,管道你们可以要自己开,好处是除了半导体之外,其他都没有什么太大的问题。



王孟源 43:04 

嗯,第二个是市场。嗯,你电动车不用想要卖到美国去了,这是想都不用想,但是你可以跟欧洲车厂合作,然后在东盟抢日韩机车企的市场份,这个是可以的。事实上汽车一直是制造业里面营收最高的一个分支,制造业里面哪一个营收最高?汽车第二,没有之一。那中国现在吃下这个为什么还会现在经济一副很衰败的样子?为什么上海一大堆有钱人往新加坡去逃,他们现在不敢逃到美国去了,也不敢逃到欧洲去了,因为那些恶国的富豪被扣钱,哈哈哈,被关起来的那个资产被扣押的那个前景实在是太可怕了。



王孟源 44:10 

但是现在大家就很喜欢往新加坡去逃,这原因就是中美脱钩,市场跑掉了,而且国际资金也跑掉了,你不能够再利用美元的放牧,就是美元,以前是美国所谓的饲料里面,这些外销型的国家所吃的饲料里面,这些都是待宰的猪牛羊了,但是还没有被宰之前,你总是要有一段成长期,对不对?嗯,财富要累积,你的劳动要有环境去贡献,这些财富累积,所以你吃的就是技术市场,还有美元,美元就是资金,对不对?中国现在拿不到廉价资金了。这个廉价是什么意思?美国以前多印这些,它是以一方面是金融借贷,另一方面是以进出口的逆差来把这些资金转移给你,所以你有很多的廉价资金来做投资,那这样结果就是你可以粗放的生长。



王孟源 45:20 

然后每隔十年被人家宰一次,那这宰一次,一年、两年,然后再复苏。这里面八年、九年,你可以很粗暴,是吗?这些人这些被宰的最早是日本,后来是四小龙,然后是东南亚,然后接下去是中国,这里面来的最晚的是中国,所以他成长的最快的也是中国,他所习惯的成长环境也是最粗放的、最容易的。为什么?因为它的治理效率也最高,它的体量也最大,所以越容易造成规模效应、经济规模效应。



王孟源 46:06 

那我们现在中美脱钩,问题是美国还是全世界最大的那只狗?嗯,它真的要摇起来的时候,任何一个国家都是可以被它当尾巴一样来摇。那美国搞这个中美脱钩之后,不论如何中国一定会有逆风。你事实上看看中国目前所达到的GDP,它的那个发展水平,他现在竞争的已经成功成为世界第一的是什么样的企业?电动车这个短短十年之前还是高科技的象征,以前掌管汽车先进技术的是哪些国家?德国、日本,这都是世界一流的工业国家。那现在变成中国了。中国也许在名目上它的 GDP 人均 GDP 不是很高,但是它的工业层次其实已经是威胁着要进入第一梯队了。他只不过是来的晚,所以还来不及补齐全部的。有一些东西真的是他对资金还有技术累积要求的特别高,然后中国本身体制又有一些缺陷,这个我待会讲,就是半导体,所以到现在还没有补齐。



王孟源 47:34 

但然后你再看一看中国的这个社会制度,中国的社会福利,事实上 gdp 是可以跟很多隐性的生活水平标准来互换的,你可以牺牲别的东西,降低整体的生活标准来追求GDP。像什么环境保护,对不对?你如果愿意破坏环境,你的 GDP 就会成长的快一点。保护环境是要花钱的,摧毁环境是可以赚钱的,对不对?还有什么贫富差距?你这个如果这个社会能够贫富差距扩张的话,财团就有更多的资金做来做投资,没错,来追求更高的利润,利润率,回报率,对不对?没错,韩国就是这样子搞定的,对不对?美国也是这样搞定的,这个这里的前提是你贫富差距大了之后社会要稳定。



王孟源 48:42 

美国之所以能够贫富差距越搞越大,原因是它可以对外搜刮,在对外搜刮的过程中,自己的底层民众也可以收到一杯羹,所以他的贫富差距并没有他那种社会达尔文政策所应该造成那么大。OK,但是你要像中国这种很在乎贫富差距的,习近平甚至花了 10 年来扶贫,专门扶贫,针对性的扶贫。这个是要花钱还是会创造GDP?我大家诚实来讲,这是要花钱的。你要把很优秀的干部投到乡村去,盯着几个几户来,把他们从贫困这样扶起来,不止是要花钱,你还要投入一流的人力资源。这些人力资源你可以去管企业,同样一个人管十户的一个小村子,绝对是有资格去管几千万或者几亿的那些几百个企业的,对不对?但是你就是要这样的投入,还有什么东西?还有你的这些社会里面只要有很严重的扭曲,往往扭曲的原因就是因为它方便 GDP 成长,像是什么中国的房地产政策,房地产所有制不合理。



王孟源 50:18 

OK,它就是因为在粗放生长的时候方便提高,快速提高GDP。所以才会导致这样子。你如果大家没有快钱的话,不会把它往推到这么糟糕,对不对?当然你可以说 2008 年的时候他们多印了4万亿,事实上4万亿本身就是一个错误的数字,4万亿只是当时他们中央政府出的钱。嗯,当时还一起放宽审查,容许地方随意发,无限发债。



唐湘龙 50:55 

没错,地方债



王孟源 50:57 

地方在那两年, 2008 年跟 2009 年一共发了至少 55000 亿,所以中国再难一波,所投入的所浪费的资金至少是 95000 亿,不是4万亿。OK,房地产政策一发不可收拾,房地产这个烂摊子一发不可收拾就是从那时候开始的。嗯,中国的另外一个初放就是学习美国的这种虚位公关,美国的企业要资金,要骗钱。嗯,就是先随便发一个公安稿,说我这个,这个能够赚几万亿,大家你们只要拿几个亿来给我,我十年之内就会赚几万亿。像什么?像这核聚变,像是量子计算,他们的资本,他们的资本市场可以这样的随意浪费。嗯,为什么?因为他们主导了全球的金融体系。没错,这些钱最后还是进了美国的金融大亨的口袋,对不对?所以你中国去学人家这个干什么?最后到最后还是进了国际金融巨头的口袋,这可不是留在中国,你就算进的口袋是中国的,中国级的金融巨头,到最后他一样是把资产转移到外面去。



王孟源 52:44 

人跑到新加坡去,你刚刚在恒大,损失了那么大的那么大的浪费,而且是已经盯上了,有意地要帮帮那个泡沫泄气,结果还没有想到他会做资产转移。你说这个有多么的笨拙?真的是真的是,我如果是习近平的话大概会有心脏病,这手底下你都已经本身恒大这个问题就是因为政策有意义所创造出来的,就是因为它有三个要求,就是三年前开始说你不能够再随意的对这些房地产商做借贷,结果没有想到他们会把资产转移到美国去,你说这些去处理的那些官员,这,我真是不知道怎么说,哈哈哈,你到底是人还是猪?



唐湘龙 53:52 

因为孟源在过去一段时间,我在和孟源每次谈到这件事情的时候,我都会感觉到就孟源的那个很无奈,就觉得觉得大陆在明明你已经感觉到美国在恐惧中国崛起的过程当中,对中国进行了各种釜底抽薪的打压,然后要去围堵中国。围堵是表面,重要的就是压抑中国的成长,要在整个中美对抗的过程当中的全力的去封锁。明明知道美国是在这样做,可是中国在面对到美国的种种动作的时候的阴硬的手段上面来讲似乎都没有什么太亮眼的地方,以至于始终都有漏洞,在你所有的采取的对应措施当中的漏洞,都抵消了你原来政策的企图。



唐湘龙 54:44 

好,那以现在的情况来讲,刚刚因为孟源在今年初讲的时候,其实我还半信半疑,孟源告诉我说美国的经济没有什么大问题,美国的危机已经过了,但你今天来看的时候可能是这个样子,虽然很多国际经济的谈论还是认为美国有一些潜在的风险,不过这种风险谈很久了,你总体看下来,它现在比年初的时候好很多。那中国怎么办?



王孟源 55:09 

我记得你问的一个很重要的问题,嗯,就是,这是不是代表美国有金融危机?嗯,因为SVB 刚刚倒了。对一点。对对。



唐湘龙 55:18 

没错没错。



王孟源 55:19 

他们这四家一共亏空了 3000 多亿。嗯, 3000 多亿,但是我那个时候就斩钉截铁的跟你说没有问题,这个是非常局部的特例。OK,因为为什么倒霉?真正倒霉的是外国的中央银行。嗯,真正倒霉的是欧盟跟人民银行。欧元银行跟人民银行。这些美国国内的银行出的都是小钱, 到最后赚的是美国的财政部、美联储,还有美国的那些金融大头。



王孟源 55:55 

那所以说我上个月发表的那篇一连串三篇博文,我今天所讲的其实是很浮面的一些东西。大家有兴趣的可以仔细去看。然后我想现在时间也差不多了,我赶快做两个总结。嗯,中国应对当前的局面,现在有两个很大的危险,就像三年前明明知道要处理恒大,结果还搞得一地鸡毛。嗯,让他们把资产、主要资产都转移出去,未来这三年也面临着一个规模更大、更,但是有点类似,性质类似,但是规模影响更恶劣的危险有两个。



王孟源 56:49 

第一个是目前有很多中国国内的所谓的学者专家在鼓吹中国学美国一样无限印钞。嗯,发债来刺激经济,你这是绝对不能搞的。为什么?因为美国这个发在印钞,它这个是潮汐式收割,每十年收割一次的一个循环的一部分,这个循环是奠基于美元,是国际储备货币。嗯的这个地位,霸主地位,中国有没有印?中国没有。嗯,你中国这个印出去多印钞票所造成的通货膨胀都必须要自己销售。



王孟源 57:40 

美国在 70 年代那一波滞胀,他们有很多经验教训,其中一个教训就是人民在经济上所感受的痛苦,有一个所谓的痛苦指数,嗯,这个痛苦指数很简单,就是失业率加重通胀率。中国现在通胀率是在 1\% 左右,甚至有点通缩的迹象,是你适当的刺激确保它不通缩,这个是可以的,但是你不能够去追逐 6\% 的 GDP 成长率,嗯,这个 6\% 的 GDP率在我刚刚已经讲的中国的经济层次其实已经很高了,而且它的社会福利虽然没有像欧洲那样子是明面上的发钱,但是事实上它保持的,它投资了很多隐性的社会的资产,嗯,比如说环保,比如说社会的治安安全,嗯,这些都是要钱的,比如说去扶贫,嗯,OK,比如说现在在整治房地产这个东西就几万亿的事情,嗯,甚至是几十万亿的事情,你去做这些长期的对社会有益的事情是必须要牺牲 GDP 的。也就是说他的这个社会的安全、社会的秩序,社会的福利、环保等等,其实都已经到了世界先进国家的水准,甚至更高。



唐湘龙 59:20 

这个很重要。好,你继续。



王孟源 59:22 

你到了这个地步的时候就不可能再追求10\%,不可能追求 8\% 的经济成长。嗯,事实上也已经不可能再追求 6\% 的增经济成长。为什么?因为美国在跟你中美脱钩。嗯,美国现在还是国际体系的主导者。嗯,他正要把你从美国的学术科研体系切出去。嗯,他正要把你从美国主导的国际贸易市场切出去。嗯,他正要把你从美国主导的美元金融体系切出去。在这个时候你怎么可能还奢求 6\% 的经济成长?嗯,中共的政治局在未来这个月会发布明年的经济成长预测目标,预测他们现在有人在鼓吹5.5\%,我觉得这是一个很错误的。嗯,方向,你实际上可以继续追求5.5\%。我认为你应该站出来说,我们知道 6\% 是一去不复返。嗯,我们的对应策略不是想要延续这些初升初长的快速增长,而是要高质量的增长,汲取它的数字。低于 5\% 也在国际环境不容许的时候也可以接受。也就是说未来 10 年的平均经济成长率目标应该是5\%,而不是 5\% 以上。OK,所以他现在有一个很好的机会,就是站出来说,我明年的目标是4.8\%,OK,然后到明年年底,嘿,5.5\%,很好,我们超额。



唐湘龙 01:01:17 

完成。没错,就把目标定低一点了。



王孟源 01:01:20 

但是我们不要求5.5\%,我们只要求4.8\%。嗯,他有这个机会,但是我怕他不会把握住。嗯,这个是,这是一个很糟糕的事情。因为中国现在他国内治理最弱的一环是什么?学术很科研。嗯,又是他最弱的。他们这种,我们已经不能够再快升快涨,应该是他国内学术界的主流,但是不是他现在主流,还是要赶快刺激经济。他上个月中共国务院上个月刚刚新发了1万亿美元的而1万亿人民币的新债。嗯,用来做什么?用来帮助地当政府解决它的隐性债务。嗯,我觉得这是可以的,因为这些隐性债务在美国用金融或者货币手段来打击你的时候会是最大限,所以你如果是为了解决这些隐性债务的这个地雷而多发一点债,这是可以的,但是你绝不能够多发债来刺激经济,来刺激需求。



王孟源 01:02:48 

中国很多不懂经济的人,他就坐在那边空口说白话,说你自己内需,没有办法,没有那么多内需,中国的对外顺差是多少?已经从 15 年前的 10\% 的 GDP 降到现在 1.5\% 的GDP, 1.5\% 的GDP,那也是已经是,意思意思而已了,就是在还在正确的方向。嗯,这样子而已的,根本你没有办法再消再削弱。



王孟源 01:03:26 

嗯,所以你光说在空口说白话,说刺激内需要哪里来资金哪?你刚好现在跟美国在脱钩,你的廉价的美元都不见了,你还要刺激?刺激就只有发债,发债的结果就是你通胀率,你的失业率每下降一percent,你的通胀率要上去两个到 3 percent,嗯,对不对?但是我说过痛苦指数是失业率加通胀率。嗯,你失业率下降1\%,但是要以通胀率上升 3\% 的代价,这划算吗?嗯,不划算。你如果说明年会通缩,所以我们适量的刺激个几千亿保障它不通缩。嗯,这个我可以理解,但是你如果说要以通胀率每上升 3\% 来降低失业率1\%,我觉得这个是饮鸩止渴,就是为了未来两三年的容易赚钱,而让国家在 56 年七七八年之后的治理面临大恒大的风险,到时候你连 4\% 2\% 的成长率都可能保持不住。



王孟源 01:04:46 

嗯,美国制造下一次全球收割的时候,你没办法抵抗这个未来的风暴。嗯,到时候中国已经有多少年没有经 GDP 收缩的经验了?嗯,到时候来一次 GDP 收缩,你看看这个对整体的政局跟体制的影响会有多大。嗯,OK,好。另外一个,这是我第一个建议,嗯,就是你不要再追求初生初长的等快速。第二个,就是要提升你这些成长的效率。嗯,就是不能够靠资金跟产能的投入。嗯,你已经没有那么多内需了,你的外需也没因为失去美国系统的,美系的市场也没办法容纳你要求的 4\% 5\% 的成长率。



王孟源 01:05:45 

这里我依然看到很多你不懂的人为不懂就不懂,这个世界上你没有人真正所有事都懂,但是你懂的话就不要在公共论坛上面胡说八道,有人说你电动车,现在多卖给俄国,多卖给东盟,那么多,为什么不能够支撑起 6\% 的增长率?嗯,我跟你讲,你很简单,去看看电动车过去两年的增长,来跟房地产那个窟窿比一下。嗯,你发现他连 1/ 2 都不到。



唐湘龙 01:06:27 

小巫见大巫。



王孟源 01:06:29 

哈哈,小巫见大巫。为什么今年经济会这么差?因为即使是全世界制造业最大的一个项目,比起房地产都是小如尖刀,根本比不过。我这个 1/2,还是用很宽松地去低估房地产的那个窟窿.你如果去严谨地算,可能会挖出来是 4 倍、 5 倍。经济这个东西是几十亿人参与的,它本身就有几十亿的维度。你不能说我看到一个正面的维度,拿起来就以偏概全说这是正面的。



王孟源 01:07:26 

我说的我自己的切身的例子,几个礼拜前中国有一个造船厂出来说他们要造核动力的货柜船。



唐湘龙 01:07:37 

没错,我看到了。



王孟源 01:07:39 

而且不是用现在已经验证的那些核动力潜艇或者核动力航母所用的动力,它是要用一个所谓很新的叫做熔岩堆。对这东西,刚好五六年前我做了研究,还写了一篇博文。里面的问题,它的那个技术问题我都已经写得清清楚楚。所以就有博客读者在我的博文留言说,这是不是像你博文讲的完全不成熟的机会,我说的确是不成熟,至少 30 年,所以你现在一个造成它不能用,是纯粹的假大空。



王孟源 01:08:26 

结果后来就有人把那个公共论坛上骂我的文章发来给我看了以后,真的是,这些就是基本上是现代的红卫兵了。有人说是基红,有人说是无脑,但是我认为他们是当前中国公共论坛上最大的危害。 10 年前、 20 年前中国公共论坛有很大的危害是叫做公知,就是他们对采纳对美国完全开放,而且采纳美国的政策跟理论,那现在他们这些公知还有没有市场?没有。



唐湘龙 01:09:15 

公知没有市场。



王孟源 01:09:16 

他已经不造成危害。现在中国公共论坛上最大的危害就是这些低级红,这些老红我看了一下他的文章,洋洋洒洒,然后但是实际上只有两个论点,第一个,这个要是实现了以后有多好?第二个,我们完全有钱,有人去拼命把它完成,这是重点吗?你完成之后好不好是重点吗?长生不老药也是很好的。我跟你讲,你这些核动力的那个货柜船比起长生不老药来说,根本不算什么。但是为什么没有人叫你投资长生不老药?因为你做不到。核动力跟永动机比起来好不好?根本小巫见大巫,永动机是连燃料都不用,它永远就在那推动,更好。你为什么不去搞永动机?因为做不到。同样的道理,既然是做不到的东西,你投入人力财力就能够做到吗?当初投入永动机的人力财力不够吗?当初中世纪的时候投入要炼金术的财力人力不够吗?中国自古以来投入要追求长生不老药的人力财力不够吗?对不对?这些东西真正的问题是能不能做得到。



王孟源 01:10:56 

我可以跟你讲,他们这个熔岩堆在 1960 年代被美国放弃了,为什么?当时第一代的熔岩堆跟第一代的压水堆竞争,什么都竞争不过。熔岩堆的最大的技术问题是它里面的那个石墨障壁必须要更定期更换。压水堆最大的问题是它的燃料棒必须要定期更换,结果压水堆更换燃料棒的技术被突破了。熔岩堆更换石墨壁的技术没办法突破,所以就被放弃。



王孟源 01:11:42 

中国在五六年前,我之所以会写那篇博文,是因为中国在五六年前又有一个研发团队,是一个学术性的实验团队,建了一个实验堆,这个实验堆相对于 1960 年代 60 年前美国那个第一代的熔岩堆,它的技术甚至连那个熔岩堆都不如。他是第0代的。



唐湘龙 01:12:12 

哈哈,这么惨吗?第0。

	

王孟源 01:12:14 

他是用传统压水式的设计,硬是把它转换成用熔岩来做,就是第一代的熔岩堆,它还设计有风格的两种熔岩,所以这样子你才能够追求效率,才能够追求安全性。你只要能够解决他们那两个两种熔岩之间石墨壁的更换问题,你还有理由去追求它。但是中国的这个实验性的熔岩堆是第0代的,它连性能、安全性、稳定性能不能做,甚至能不能发电功率高于你提供的数据的电力供应都还不能够保证,它完全就是实验性的要来获取一些学术参数的第0代,但你知不知道现在压水堆发展到第几代了?发展到第3代了。



王孟源 01:13:17 

你第0代的熔岩堆,他那个造船厂说要用的熔岩堆就是那个实验堆,那个第0代的熔岩堆,你说这有多离谱?你只要稍微有一点技术概念就知道会有多离谱。但是他们公共论坛上夸夸而谈的就是那些低级红就是,技术智商是 0 的人,他们的声浪占据把握公共论坛上的99.9\%。



王孟源 01:13:54 

为什么?我为什么会忽然扯到这里?这个就是过去三四十年初,生长在美国放牧环境下有廉价美元,有美国市场,有美国技术转移所得到的心态,一种很浪费,一切,求快、求方便、求捷径的心态。它所反映出来最低层的一面。在这里实际上问题最大的是中国的学术科研界,所谓的专家,他们都是真正的以诈骗为主业,以科研为副业



唐湘龙 01:14:40 

这样讲太狠了。



王孟源 01:14:43 

我不是开玩笑的。真的,他们这一次十四五的排名最前面的几个科研项目都是纯粹的诈骗项目,都是像我刚刚讲的要把熔岩堆拿去当那个传用动力的那种项目,就是根本做不出来核聚变。我刚刚提到对量子计算,为什么现在那个团队现在做量子计算?因为他 10 年前它的核心专业其实是量子通讯,结果骗了一波钱以后,客户发现一点用都没有,没找不到客户了,所以他才会转行去做量子计算。



王孟源 01:15:29 

他的本行本来不是量子计算,但你说像这样的诈骗项目排名最前面的诈骗项目反而是排在后面的,有真正的贡献,你说这样好吗?你这样的那些管理的。事实上为什么科技部现在半年多前被重整?因为你这些反映出来就是过去这 20 年几万亿,真的是几万亿投到了半导体里面,结果什么都拿不出手来。



王孟源 01:16:04 

我有一个朋友过去这 20 年刚好是在北京做半导体投资的,他跟我讲,中国的之所以半导体前投的越多,摊子越烂,原因就是那些院士自己没有本事,而且是靠替那些骗子遮掩,从中分红,然后再去买自己的地位。 20 年前汉芯那个是纯粹诈骗,为汉芯遮掩的那些院士一直到现在还是院士,一直到今年还在主导半导体产业发展,你说这种人能搞出产业发展,能搞出结果来才是奇怪。



王孟源 01:16:59 

我们半年前我们谈到华为,我说这个是 plan b。嗯,OK,现在过去这两个月,一个月前 Bloom Berg 有一篇文章 expo j 证实了现在中国政府的确是把华为当作振兴半导体的,解封了。OK,华为本身本来不是做自己做半导体没错,但是在今年成立了一个偷偷的成立了,至少在文博格的文章里面讲的是他们为了避免被美国打击,所以 total 成立了一个所谓深圳 major industry investment group,这个 INS n group 里面掌控了 6 家新成立的公司,都是过去这一两年才成立的公司,其中有 4 家基本上是跟华为联动的。



王孟源 01:18:08 

这 4 这 6 家公司做的是什么?光科技,光科 e 各式各样的半导体的生产,就是基本上很明显的是国家在替换科技部的同时,知道他们的科研管理没有办法弥补,快速的将半导体带上来,所以去找了任正非,跟任正非说,我需要你们的科技管理。嗯地,然后偷偷的由深圳出面建立了这个集团来跟华为合作,由华为负责管理。这国家偷偷的出钱。这我半年前说 plan b,现在被完全证实了,这就是他们的 plan b。为什么他们必须要 plan b?因为科技部不管用。嗯,因为中国科学院里面,中国工程院里面诈骗集团太多。嗯,好。那我想想又想问大家,公共论坛上的那些低级红,嗯,那些无脑红你们是在帮助国家还是在残害国家?嗯,这些诈骗集团一出来,他们为什么会去搞公关?为什么敢搞这些虚假的宣传?因为他们知道会有这些脑红来捧这些无脑红制造舆论,刚好可以过中宣部的审查,然后就造成对政府官员,尤其是地方官员的压力。



王孟源 01:19:57 

我这次去深圳调研,就有一个深圳管科技的人的官员。我在骂那个量子计算。你知道量子计算的大本营在哪里,就是中科大在哪里?在安徽合肥,他就是,他就说我才刚开始骂,他就说你不用讲我都知道。为什么他知道他认识安徽合肥管科技的?他们私下聊天的时候,那个安徽合肥的管科技的人在外面叹气。嗯,他完全知道中科大就是在搞量子计算,就是在诈骗,但他不敢讲,因为就是因为这些无脑红已经占据了舆论了。



王孟源 01:20:47 

嗯,你即使知道他们是在诈骗,你自己是管科技的官员都不敢讲。嗯,因为你的权威性没办法跟中科大,没办法跟中科院,没办法跟中工程学院那些院士来比,你如果公开的说实话的话,第一件事就是你的职业生涯完蛋。嗯,那你想想看,政府搞成这个样子,我也没有办法,对不对?这个,这个,这是典型的利益集团,精英利益集团霸占国政,但是你至少平民百姓自己国家的前途你自己照顾,你不要再与公共论坛上面帮着他们胡说八道,你自己只会吃亏了。



唐湘龙 01:21:42 

孟源刚最后讲的这一段了,我坦白讲就是很尖锐,但是非常的善意。我过去想过这件事情,可是我没有办法向孟源表达的这么的清楚跟完整。那在没有在网,没有网络时代以前,攻击在引导的这个世界,有了网络之后,所谓的网红无知再引导的这个世界,这个其实是很可怕的,孟元刚讲到就是说中国的经济正在过深水区,因为外部的打压跟内部的结构的调整同时在发生,所以莫言刚刚提到了两点,就第一个就不要再盲目的追求不切实际的那种的快速成长的成长率,那个可能已经过了许多有质量的成真成长可能才是最重要的。



唐湘龙 01:22:35 

第二个就是避免了无谓的浪费,因为这种无谓的浪费不管是其实科技诈骗不只是浪费,而其实它是隐性的贪污,用各种很漂亮的名苗名目,但是把国家的资源做错误的配置一甚至于就掏空了,这个是需要很有智慧,很有洞察力才能够去抓到这些都东西,不能够盲目的用美国的那种,借着它本身的美元的霸权,可以砸大钱,不计成本的过度浪费的实验性的那种的投资,去维持它的国力跟它的剥削的能力。



唐湘龙 01:23:13 

中国既然不是这种的制度,那人民币也没有这样的地位,但是又面对到美国的打压,中国在眼前过深水区的时候,必须要有一套自己的哲学跟自己的,就是接下去过深水区的一个治国的态度,这个难度很高。其实我觉得孟源刚最后讲的这部分,我们甚至于下个月,我认为孟源都还有很大的延伸的空间。好,今天本来我想问孟源就是说 ECFA像现在像大陆方面看起来是准备要陆陆续续要准备停掉 ECFA那个,那。



王孟源 01:23:48 

现这个现在只不过是所谓的a shot across the Bow就是一个警告射击。



唐湘龙 01:23:54 

没错。



王孟源 01:23:55 

因为他所昨天所 12 项所牵扯到的。



唐湘龙 01:24:01 

12 个税目。



王孟源 01:24:03 

只占 ECFA里所有的项目里面的2\%,是 ECFA 全部的2\%,也就是台湾 GDP 的百分之零点四几,嗯,OK,你就算全部都废掉了,嗯,也是不痛不痒,嗯,就是它基本上是一个警告,嗯,口头警告,那,但是大家要想想看,这只不过是ECFA,这 12 项只占ECFA整体的2\%,OK,他如果继续下去的话,对台湾的经济有什么影响?嗯,大家实际想想看,这些都是白送的钱,而且这个白送的钱不止来自大陆,也来自台湾消费者的名字,没错。



唐湘龙 01:24:53 

没,没错,这个,这个是我,我很希望台湾能够能够循着孟源的刚刚讲的这一点去认真的而且务实的思考并且去处理两岸的经贸的问题,不要用意识形态以及用这种政治立场得了便宜还卖乖,只想把它变成是政治的可加工利用的筹码而已。好越时间的关系,其实尤其现在美东这么冷的是深夜,黄某月被我绑在网络上。好,我们感谢一些的观众朋友来 brother of CAN 001 感谢破论谢谢谢你那耶迷王感谢然后的 WK 骂谢谢谢那祝王孟源冬至快乐。唉你在那边会吃汤圆吗?



王孟源 01:25:36 

没有不会哈哈哈。我来美国三十几年了,今年才是第一次回台湾过中秋节,所以我第一次吃超过一个月饼。



唐湘龙 01:25:49 

哈哈哈哈哈可怜。真是好啊,山西辛苦来,然后祝王博士的新年快乐,雷欧,感谢,然后接着刘感谢斯丹利,看来是在泰国哈,感谢,然后老衲拿不赢了,冬至快乐。然后 YT 居然可以直接打赏了,感谢支持王孟源,刚孟源在线上做一万五千多人在星球的这个时间点,尤其以孟源讲的深度来讲,非常的不容易,表示我们的听众观众的水平非常高的。好,那这就one,他说每次听博听王孟源谈论的内容都觉得脑袋又被重新整顿了一次。对,这,这是我找王朋友原因。好,斯丹利,在感谢。然后俊林,感谢。醍醐灌顶,好,感谢。那我们所有的观众朋友,尤其在冬至的这一天,好,在天寒地冻的情况下面,然后分享了王孟源的智慧,感谢呢,在美东通过电话连连线在龙行天下里面的我的重量级的来宾王梦云,梦云,感谢。



王孟源 01:26:54 

好,谢谢大家。



唐湘龙 01:26:56 

OK,好,跟那么孟源说呢,除了耶诞节快乐,新年快乐,然后 2024 年见,拜拜拜拜。拜年快乐,拜拜。



\twocolumn[\begin{@twocolumnfalse}
\section{中国的新金融战略}
\subsection{20240119}
\end{@twocolumnfalse}]1月19日,已校正。



唐湘龙 00:00 

…今天礼拜五的时间,龙行天下的单元,那王孟源的时间,王孟源在线上。好,待会跟王孟源谈几件事儿,这些的议题他们让我也不是在新闻当中没有接触到,不过我觉得我没有办法谈出味道,谈出观点,所以,为什么需要很好的来宾?然后我在旁边扮演的就是一个串场跟一个学习者的角色。好,那今天我们大概会谈几个主题,你看到我标题上面用的是习近平的新金融政策。因为从资本市场,譬如说今天美股涨,那台股也大涨,那大陆股市就在我们今天要是做现场的直播连线的前一天,大陆的股市昨天的下午盘拉尾盘,这个当然因为王孟源在这金融市场里面其实工作过很长的时间了,为这个就是说瑞士银行公告工作,其实它对于国际的金融情绪的这个把握的那种脉动感的那个比一般人都要强很多。那因为毕竟大陆的金融是市场,即使不是雪崩式的下滑,可是一直都很低迷,气氛都很差。



唐湘龙 01:51 

那昨天的这个拉尾盘的动作,相对于就是说国际股市来讲,过去一整年, 2023 年全球的主要的股市几乎都是正成长的。全球经济并不好,可是全球股市基本上面都是不同程度的涨,唯独表现最差的就是陆港股。好,那在资本市场低迷的情况下,面对于整个的中国大陆的经济环境、经济气氛当然有影响,所以你也会看到官方不是没有态度的。官方一波又一波出台的,其实都是想要去拉抬这经济信心,可是前面的几波的拉抬,反而让从一个资本市场角度看到大陆的股市,上证指数从跌破 3,200, 3,100, 3,000, 3,000 跌破,大家觉得差不多了,(结果) 2 900, 2,800 都快跌破了。好,那习近平讲话讲了什么?这一波的讲话为什么呢?王孟源会觉得非常的重要,然后之后我们再来谈,嗯,我们会稍微小谈一下一个叫做呢叫做Gonzalo,Gonzalo这个人,因为他可能在乌克兰被乌克兰给虐死了,可是他却是一个美国的媒体人,专栏作家,智利裔的背景。Gonzalo的这个死,其实到目前为止美国还没有跟乌克兰好好的计较一下,那他到底是一个怎么个原因?怎么个死法?那王孟源觉得他是个烈士 。对了,我也看到媒体上面有些人认为,认为他牺牲了他是个烈士,因为他在,他独排众议,即使不是唯一的,但是他是所有的西方的媒体铺天盖地的在吹乌克兰的时候,他作为一个美国的专栏作家,媒体人他其实一直在质疑这件事情。好,之后我们把最后的重点摆在包括乌克兰、欧洲、红海那最近的情势的一个大的演变。如果有时间的时候我们会再来谈一下美国的总统大选,川普是不是要回来了。那介绍来宾在我们线上的那人,在美国穿着厚大衣的王孟源。欢迎。



王孟源 04:09 

嗨,大家好,现在刚好有寒流来,所以外面是零下 5 度。我也不想太浪费能源了。一个人住一个很大的房子。嗯,偶尔开暖气的话。开到底的话很浪费。



唐湘龙 04:26 

对,我每次听完王孟源;讲这一段的时候,我就可以想,哇在一个很大的房子,然后非常冰冷的空气里面,一个老男人孤孤零零的守着那个房子在跟我讲话。好,但是那个是王孟源做功课的地方来,我们就从,因为这个问题我坦白说,所以即使我认为习近平谈了一些很重要的东西,但是我很诚实说我没有能力谈,而且我发现我周围的人普遍对于习近平的讲话没有办法形成观点以及有效的去诠释习近平为什么要这样做?你怎么看起习近平这次讲话。



王孟源 05:04 

这件事有关他们的金融政策,或者是更说的更高一点,金融政略,这件事我是有亲身参与。而且是我博客开始写的博客, 2014 年开始写博客,十年前开始写博客,第一年还是针对台湾的读者,但是到了第二年开始。我发现。真正对我所谈的国际政治、经济这些议题有兴趣,然后肯发言的人反而大部分是大陆来的。所以到了2015 年,然后慢慢的到 2016 年之后,就变成我面对的观众、读者反而是以大陆的为主,所以在那之后我就深刻的去讨论了很多大陆当前的问题,当然我不会去谈那些人事的细节,或者是民生的小事情,这种事情轮不到我来谈。我谈的是最高层的,就是他们的政略,还有战略事情啊。



王孟源 06:21 

经过这八九年的讨论,其实熟悉我不熟悉我博客的人应该都知道,我已经总结了,大陆就是中国政府,它的阵列上有三个主要的问题。第一个是外交。就是原本他们有幻想要所谓的中美夫妻论。这个这个问题在习近平上台之后就开始慢慢的扭转。到 2017 2018 年 Trump 撕下脸皮,然后开始贸易战之后,基本上就彻底解决了,就是基本上最高层,至少最高层就是中上中层的外交部经济部,还有其他的那些官员或者尤其地方的官员,也许还会有幻想,但是最高层已经完全一致对外,知道美国的殖民帝国体系是怎么回事。然后你在外交上应该怎么处理,我待会大概半小时后,我们我谈国际事项的时候会举一个最新的例子,就是李强现在在达沃斯论坛的讲话。



王孟源 07:37 

那这是第一个问题,所以我觉得这个问题在几年前就已经扭转解决了。OK,第二个问题是他们的金融政略问题。就是因为邓小平改革开放之后,他们那时候落后欧美落后的太远,所以就在心态上过于崇拜,就是一切欧美的制度还有系统都是摆在神坛上面膜拜。



王孟源 08:15 

其实欧美的学术体系主要是人文的问题都很大。那其中问题最大的就是它的经济跟金融体系,因为它这个是资本,他们是一个资本主义的社会,那资本主义社会的定义就是以资本的利益为最优先,那资本要推行有利于他们的政策的话,第一步就是必须要先洗脑,要先立下杆子说,有利于我们的政策都是对的,都是好的。



王孟源 08:54 

所以我过去这八九年有很深入的讨论,为什么美国的经济学理论本身就是烂到核心,就是一套公关冒充成学术。那很不幸的是中国在过去三四十年很完全不过滤地去吸收引进,而且崇拜这些。所以他们,尤其是上一届的国务院有很多真正就是直接受这些美式经济学所荼毒的高级官员,所以他们的金融政策基本上不理解美国利用美国的金融殖民的手段。



王孟源 09:46 

那我最近最新的三篇博文,基本上把这个是一个大整理了,是吧?过去博客十年来有关金融经济方面的讨论彻底的整理出来,有兴趣的读者可以再去复习一下啊。那其实我在两年前就已经写了一篇文章,里面很简洁的讨论了这个,一个社会主义国家对经济金融的态度应该是怎么样。那里面就明确的讲金融是越简单越好。为什么呢?因为金融就是钱滚钱的东西。那你实体经济所产生的那一点利润跟资金,如果在一个复杂而且黑箱的金融体系里面滚一滚的话,嗯,那羊毛就很快就会薅光了。



唐湘龙 10:45 

没错。



王孟源 10:46 

我举一个实际的例子,过去这几年因为新冠的关系,民航业非常的波折很多。那但是在 2020 年新冠来之前,在,就在 2010 年代末期,美国的民航业其实有一个稳定态,但是在这个稳定态下,你如果去看他们那些几个大航空公司,像 American Airline美航、 United Airline联航这些大航空公司。他们的股票的股价的估值,市值都很低。当时出现一个很奇怪的现象就是,我不晓得你熟不熟所谓的 airline mileage 就是里程。



唐湘龙 11:46 

当然,累积里程。



王孟源 11:48 

累积里程的这个其实是在 40 年前左右,美国航空公司当时的 CEO 发明的,它原本的用意就是希望能够让顾客比较忠实一点,是因为你奖励忠实的顾客。但是 40 年发展下来,到 2010 年代后期,他们那个 Mileage program 的估价,股市估值已经高于他们的核心载人的民航事业。



王孟源 12:22 

这是很可笑,就是尾巴已经开始摇起狗来了,为什么会出现这种现象呢?就是因为这个mileage program 已经被金融化,它被包装以后卖给银行、卖给广告(公司),然后经过金融手段。结果那个你民航一个实体,好好的实体经济的一个产业,它赚的钱反而都被这个纯虚拟的金融手段的东西、附属的东西给吸光了。但你吸的时候如果你不仔细的花几个月的时间去仔细研究,你还不晓得他是怎么吸的。但是我现在举这个例就跟大家说,事实上无可避免的,你如果有一个复杂的金融附属在一个实体经济上面,它一定会开始对实体经济吸血,这个金融发展的越久,越复杂、越自由,吸血的就越多。那尤其是目前现在的这个国际金融体系其实是全球化的,那全球化的基础就是美元。所以一切的金融去国际上的金融,只要这个国家还是在用美元。



王孟源 13:51 

你就必须要让美国剥削。你如果不用美元的话,你就没有办法享受全球化的那个效率,所以这是一个两难的局面。俄国也是一直到了俄乌战争开始以后才下定心去美元化的。那我说我的博客写了 10 年,吸收了很多大陆读者,其中常常跟我讨论的能够跟我做了论证的,通常都是博士以上的学位的人。这里面就有一个大陆来的 AI 的博士,他的专职是 AI 的博士。当然在过去几年在我的博客上也蛮活跃的,我相信他很认真的把我的博客至少读了两三遍,那一共 1, 000万字。但是他一直到上个月才给我写了一个留言。说他经过反反复复读了我的几十篇文章,读了不晓得多少遍,一直到上个月他才彻底理解为什么弃用美元是唯一能够躲开金融剥削的手段。



王孟源 15:06 

OK。他做了一个比喻就是,你只要是还用美元,它的那个组织机构,还有机制、法律规则,还甚至大部分的玩家,都是美国控制的,这就好像你到美国开的赌场去赌,你如果是一个小玩家的话,或许赌场大概看不上你,可以让你随便的凭几率来赌,来玩耍,但是你如果是像中国这种体量的,他不宰你要宰谁呢,对不对?所以我之所以提起这件事情是因为你想看一个博士级的,而且我觉得是相当聪明的。我在博士之中也算是相当聪明的人,尚且要经过好多年反复的努力,我知道他一定是投入了很多很多,几百几千个小时来读我的博客。但这样之后他才有这个的觉悟,所以我可以在这边跟大家这样聊聊天,把结论丢给你。事实上我不可能在这边做这些论证,因为这些议题都是极度复杂的。绝大多数的教授跟博士都没办法,即使是金融或经济专业,他们都不是真的懂。



王孟源 16:36 

所以我的博客才是适合辩证的地方。那这里我只能够把结论丢出来,那有兴趣的读者可以到我的博客区,那你如果有了心得,以后也可以发问,我们可以来讨论,我刚好过去这个礼拜有不少很好的问题,我的我写了很长的,一连写了好几个很长的答复。那所以水准够的读者,我鼓励你去好好的看一看。那在这里我只想说我的,我一直是希望我的文章,尤其是有关金融的文章能够直接影响中国的政策。那我相信两年前我的那篇社会主义国家应该如何管理资本理念,它就其中的最重要的一点就是讨论金融。那我之所以认为现在这个礼拜中国所出的几个新政策值得讨论,就是因为它反映了大概里面有百分之八十九十反映了我当时两年前所写的那篇文章。很巧的是那篇文章里面也讨论了半导体大资金运作,里面有明显的贪腐结果,一个月之后——这可能是巧合了,也可能是有因果关系,但是我自己都不知道——那一个月之后半导体大基金就被办了。几十个人入狱。而且有传言说你的贪腐的金额如此之高,要达到 5,000 万人民币以上才会被抓。 5,000 万贪腐的如此普遍。



王孟源 18:39 

不过我觉得科技管理里面贪腐还算是小问题。诈骗股市才是真正最大的问题,所以这也是我在,这就是我刚刚讲我过去这十年来认为中国最大的三个问题,外交这个已经解决了,金融现在这个礼拜刚刚彻底解决。然后残余最后一个是科技管理,这个完全没有解决的迹象。所以有兴趣的读者可以去好好的看,那我今天关于这个科技管理,就讲到这里。



我们现在回头来看这个礼拜到底发生了什么事情。首先是 1 月 16 号,习近平特地到中央党校的金融研习班致辞。这致辞里面,他致辞的核心是八个坚持,我再利用这个机会把它念给大家听一听啊。坚持党中央对金融工作的集中统一领导,坚持以人民为中心的价值取向,坚持把金融服务实体经济作为根本宗旨。这句话刚好就是我两年前写的那篇文章里的。



王孟源 20:08 

坚持把防控风险作为金融工作的永恒主题,坚持在市场化、法制化推进金融创新发展,坚持深化金融供给侧结构性改革,坚持统筹金融开发和安全,坚持稳中求进工作总基调。这里面我刚刚提到那个坚持金融服务实体经济作为根本宗旨,这个是刚好就是我那篇文章里面反复强调的主旨。嗯,其他的那 7 个坚持大部分也都是符合我的意见跟看法,只有一个例外,这里面只有一个一条我认为是旧的金融主管他们塞进去的一条折衷条款,就是改革没有完全彻底了,还是有一些老式的思想散进去了,那这一条就是坚持在市场化、法治化轨道上推动金融创新发展。我个人认为是金融不要创新最好。



王孟源 21:34 

你越简单,越明显越基本越好。所以他们这些人是把金融创新这只猪攃上一些口红,这个口红就是市场化跟法制化,然后一样把他推了出去,不过八条里面只有这么一条,我其实觉得蛮好的啊。然后这是 1 月 16 号。那1月 18 号,也就是昨天这个李强人虽然不在北京,(去达沃斯)但是他签署的一个国务院令,这个其实就是你放在一起的话,就很明显是将习近平的讲话具体化、政策化。那他这个国务院令是叫做国务院关于修改部分行政法规和国务院决定的决定。有点绕口,决定的决定。



王孟源 22:42 

里面很多条,就是很多法规还有组织做调整,那基本上是根据刚刚我提到的那些基本原则来做调整。那我举两个例子好了,一个,第一个是修改中国人民银行货币政策委员会条例,也就是对中国人民银行他们的货币政策加强了监督。熟悉我博客的你应该知道我对……你或许也记得,上个月我猜猜刚刚批评了人民银行在过去这三年的表现,那所以这个做出新的监督是很合理的。另外一条是将金融控股公司的审批和监管主体由中国人民银行修改为国家金融监督管理总局,这也是一个很合理的改革,就是人民银行连他自己的本职也就是货币政策都做不好,你叫他去监管其他的金融机构,这是不合适的。所以应该是有一个总管的单位。他说了算,不要政出多门。那这个总管的单位就是国家金融监督管理总局,所以我认为这些都是很正面的发展,基本上到这里我想中国的金融政策已经上了正轨。不过我并不是说股民现在可以赶快去炒股,然后上海的股市会很快的冲破 3000 点、 4000 点、 5000 点啊。因为你要知道美国的股市,它反映的是它的美元霸权。是全世界的人都必须要为美元买单。那尤其是它一个放牧系统,就是它把工业生产跟矿产、农产这些东西散出去给其他的国家,然后把用美元去反过来购买这个,然后利用这个美元的涨跌的波动来剥削其他的这些国家。那我上个月也讲过了,这里面一个很重要的手段是你当其他的国家累积了十几万亿美元的外汇储备的时候,这些外汇储备就不得不送回美国来炒作美国的低级资产,嗯,所以美国的公司你不管它本身是多么的烂、多么的虚弱一样有人愿意买单。没错,一个很简单的例子就是Elon Musk,他的,他开了好几家公司,有一个叫做superloop(编注:此处应指Hyperloop,超回路列车,其炒作的概念大致可以理解为真空+磁悬浮地铁),这个从一开始,几年前所有懂一点点机械跟工程的人都知道这是100\% 的骗局,但是他还是拿了5亿美元。OK,拿了5亿美元。这5亿美元是哪里来的?Dubai的国家资产,对不对?这种事情他们的股市的估值是全世界买单的对不对?



王孟源 26:26 

中国如果要那不受……而且中国在过去这十年基本上是美国全力打击的头号对象。在金融上是美国最强的一点,你怎么能够指望自己的股市还会有外资来帮忙?对不对?在过去这两三年刚好又是美国系的、欧洲系的外资撤离的最快的时候,虽然现在中东系的资金有进入中国的趋势,但是还远不足以取代以往欧美的总财富,所以不要太注意股市的股市指数的高低,因为决定股市指数高低的刚好就是你的货币进出国际的金融的手段,而这个是一个大战略的问题,那不是你内政管理或者经济开发展,或者甚至产业升级做得好,你的股市就一定能涨上涨。决定股市水准的头一条,头号动力是金融货币,而不是产业的竞争力,这是很讽刺的,但是事实上就是这样子。我做的所有的分析从来都不在乎股市,因为这种东西是虚的,是殖民体系底下损人利己的一个手段。我不想看到全世界被美国剥削,我也不希望看到中国去剥削全世界。最好的是好好的稳扎稳打,建立好你的实体经济。那这个,你的实体经济做得好不好,另外有评论;跟股市的高低一点关系都没有。



唐湘龙 28:42 

我打的岔,因为习近平这一次的讲话很特别,你看他的这八点的指示,八b点的指示,当然外行人看会觉得那就是很多的专有名词,很多的口号性的堆砌。可是它在于第一个就是说对于大陆的金融体系上面设一个监管单位,类似台湾也有像金管会这样子的,这样的体系。那这个是确定的。第二个它总概括来讲,大陆媒体谈到就是说习近平在强调一个有中国特色的金融政策跟金融体系。这个概念化跟政治化的描述,我也不能够说他错,可是他对于他对人民币的国际化的,人民币的总体的表现会有什么影响?就是当这些的政策逐步落实了之后,对于大陆的金融体系跟国际的接轨跟互动会有什么影响?



王孟源 29:43 

well,这件事情我也已经讨论了好几年了,就是真正的长期的最重要的步骤是建立一个新的国际储备货币,替代美元的国际储备货币,原本欧元是最合适的,但是欧洲既然已经,欧盟已经被美掣肘了,这个是一个很讽刺的。原本他们当初建立欧洲共同体,或者然后最后升级为欧盟,那些早一辈的欧洲政治家,他们的用意是要跟美国鼎立,跟美苏鼎立,然后后来变成跟美中鼎立,结果 Merkel 一退休,整个欧盟就被美国整吃下去了。所以。前几代的六七十年的努力反而是被美国完全操控,这是一个很反讽的事情,也很不幸。所以现在剩下的能够做出为对美元釜底抽薪的就是中俄这个联盟。那我对俄国中央银行行长Nabiullina所做的发言一直是很留心的,我知道她很急着要推进。那而且她推进的方向,那个细节、技术细节都是对的。那么中国跟俄国扯皮了,过去这两三年,结果什么东西都没扯出来。那你可以想象是谁在阻拦,不是俄方,还有另外一方是谁,嗯,对不对?现在中央银行长换人了,我们不知道他的眼光怎么样啊。国务院总理也换人了,我们也他的政策取向、对金融的了解是否足够深刻,我们也还不知道。不过很明显的,你短期有很多小事情可以做,比如说去年,他们还,去年一年之前他们还在搞什么?金融自贸区。那个时候一个座谈会邀请我去谈金融自贸区,我就拒绝了。因为这种东西,我去我也只能说这是祸国殃民的傻逼东西。



唐湘龙 32:19 

你要讲这么直,好,那找你去干嘛呢?



王孟源 32:24 

我去干嘛对不对?扫大家的兴。我想这种事情以后至少习近平立了基调之后,这种事情,这种傻事、祸国殃民的傻事会比较少一些。但是能不能专心致志的去跟俄国联合搞一个新的国际储备货币,还很难说。人民币本身是不适合直接替代美元。这个时候博客已经反复论证过了,那原因如果用一句话来概括就是美元当前的地位,就是用来搜刮全世界的。人民币替代美元的话,如果成功就是人民中国搜刮全世界;如果不成功。就会被美国倒打一耙,然后把中国的金融体系全部打垮啊。所以不适合。我想时间也过去一半了。



唐湘龙 33:26 

这段讲得非常好,但没有就我们要因为这个是一个,因为习近平的这段的讲话在台湾的内部我觉得引起的讨论跟思考是非常少的,所以孟源刚刚的那个解读其实很有启发性,当然这个。



王孟源 33:41 

中央党校的内部谈话,一般人不会注意。



唐湘龙 33:45 

对,但所以习近平的这个金融政策的讲话,我觉得他有长效性。就是你如果关注就是说国际金融体系的未来的变化的,那这一次的中央党校的讲话你最好仔细的再解读一下,把孟源刚的话再听一下。好,我们回到呢,回到第二个,这个话题其实有点伤感,这个叫做Gonzalo 的,那从Gonzalo 我们来看一下现在,现在整个俄乌跟整个中东情势的变化。Gonzalo 他就死了,虽然他也,他在还活着的时候,他指控他是被乌克兰给虐待的,那乌克兰认为他所发布的讯息都是黑乌的。那甚至在整个西方的媒体的报道的这方向上面来讲,很明显的它不是主流的,可是它黑乌就黑乌嘛,那他有这么大的影响力吗?为什么要把他整到死呢?



王孟源 34:43 

因为他人在乌克兰。嗯,对,我先讲一下,我是想要为他致哀,不过在为他致哀之前,我先为你致哀。嗯,就是上个礼拜台湾的大选。我知道你很难过,哈哈哈哈哈。



唐湘龙 34:59 

没有那么严重,因为我已经知道那个结果了,那个在选前就知道的。



王孟源 35:04 

十年前你就几乎可以有 80\% 的确定性,一年当然今大选的结果,当然一年前就可以100\%。



王孟源 35:14 

所以你没有什么好难过的。



唐湘龙 35:17 

我们就只是知其不可为而为之,就孔老夫子讲的就消极的悲观了,就知其不可为而为之。



王孟源 35:29 

事实上,绿跟蓝的基本盘有 10- 15\%的人口的差距是,嗯,所以你不管怎么选都不可能选上,而且你如果有第三方来搅局的话,这第三方一定是以反贪腐的清新角度来批评绿营嘛。那你讲到贪腐的话题的话,你说国民党对民进党有什么优势吗?嗯,一点优势都没,对不对?所以这个第三方绝对不可能跟国民党联合起来打掉民进党。为什么?因为你第三方先天就必须是反贪腐。反贪腐又不止会反到民进党,也会反到国民党。



唐湘龙 36:16 

包袱都很重,蓝绿的包袱都很重。



王孟源 36:19 

嗯,这个其实真的就是过去这一年,我跟我的台湾的朋友,包括我夏天到回台湾聊天的时候都是这样子分析的,就是明眼人都可以很简单的看出来,这个所谓的白蓝联合根本就不可能成功。不管怎么样,我们谈一谈Gonzalo Lira本身是智利人,但是他移民到美国。



唐湘龙 36:43 

他父母亲都是智利人。



王孟源 36:46 

对,但是在加州住了一段时间以后,他是像你一样是一个媒体人,然后后来就退休了,然后退休到以后他就到乌克兰的Kharkov。



唐湘龙 37:00 

哈尔科夫



王孟源 37:01 

在Kharkov结婚生子,然后所以这一次战争来的时候他并不是为了去报导这次战争跑到乌克兰的,对,而是他原本就在乌克兰。



唐湘龙 37:11 

他本来就在乌克兰。对,而且就在哈尔科夫是乌克兰的第二大城。



王孟源 37:17 

对,那因为他有社会主义倾向,所以他一直对欧美的主流媒体,事实上也这些欧美主流媒体也原本就是 CIA 跟 MI6 的外围组织了,工具。那所以他对这些批判是早年就已经开始,那这次乌克兰作为美国的马前卒去想要挑起 俄乌之间的冲突,他也是一早就看得很清楚。我过去这两年一直说俄乌战争。所一个 any intended consequence 就是无意的结果,就是有这些有良心的媒体人跟自媒体人,尤其是自媒,他们结合起来要揭穿欧美主流媒体的那些谎言。那我在我的博客开玩笑的把他们叫做实话者联盟。事实上他们之间虽然有串联,但是一直没有一个真正的组织,就是我邀请你上我的座谈,然后你邀请我去跟聊聊天这样子,然后互相引用。那Gonzalo,因为他人在Kharkov,所以他对乌克兰内部、国内内部的谣言特别的敏感。没错,他听的特别多,所以他的报导也很多。



王孟源 39:00 

我记得。因为一开战以后乌克兰国防部特别的成立一个造假小队,就是他的那个任务就是拍摄假的战争视频,(就打认知战),就基本上打认知战,这个,这什么样的假视频啊?就是比如说制造俄军什么暴行这些,然后呢?然后说什么俄军的飞弹摧毁了什么住宅区,那后来也是Gonzalo最早揭发这个组织,还有然后他提醒之后大家去发掘,才发现。比如说有一个例子,就是大家发现有六个不同的视频,就是有两个月的过程之中,有 六 个不同的乌克兰国防部所发的视频,说是俄军摧毁住宅区造成伤亡,他们其实是同一栋住宅。用不同角度在不同天候下拍的这个就。他们的造假专队去两个月之内去同一栋住宅取景取了六次。所以这种事情,我觉得他对们现在第三世界揭竿而起,就是在俄国揭竿而起来做反殖民帝国体系的一个斗争的情况下,不断的传播散开这一个过程。Gonzalo这些人,这个实话者联盟有很大的贡献,就是他们大部分都是以英语来传播的,然后各个不同的语文的体系会有人去翻译,然后采纳或然后再进一步传播。我自己也认为我是实话者联盟里面的中文分舵的舵主。然后所以Gonzalo Lira在一年前就已经被捕一次了,是在Kharkov那被捕了两三个月,之后被释放是因为当时的有外交压力,所以他就被释。但这一次他被也是大概被捕了四五个月就,外交压力就没效了,很明显的,乌克兰这一次是铁了心要把它整死,然后他就真的整死了。



王孟源 42:00 

我认为他是为了求真,为了传播真相而死亡的一个牺牲自己硬的一个烈士。而他的思想也这的确是社会主义思想,他是一个在美国的一个社会主义者。那所以我值得在这里哀悼一下。



那谈到乌克兰的造假,我谈一下现在这个达沃斯。现在美国的布林肯在那里,对,李强在那里,然后泽兰斯基刚好也在那里



唐湘龙 42:38 

没错。



王孟源 42:38 

我不晓得你有没有注意到他们居然在Davos,要搞一个和谈会议。



唐湘龙 42:47 

没错,而且是泽伦斯基他主动提的。



王孟源 42:53 

对,俄国人不在。嗯,然后你明明只是一个互相吹风的一个论坛,然后你说要一下就变成一个和谈会议,这个很明显的就只是一个外交造势,你要拍一张照片,然后就可以鼓吹说所有的国家参与的国家都支持乌克兰的和谈,即使是乌克兰单方面的和谈,俄国人根本不在那里。你和谈什么东西?就是支持。然后他们就可以再扭转一下这个公关,就变成这些国家,全世界的主要国家都支持乌克兰。



王孟源 43:39 

对,然后几天前就有泽兰斯基就有发表了一篇,发表了一句喊话,这也是欧美主流媒体广泛报道的。他说他希望中国参与这个和谈的工作。你如果只看字面的话,就会又被他骗了,就以为他的意思是希望中国在未来的这一两年当中战争打够的时候参与和谈。不是,它这个实际的意义是因为中国的外交人员看穿了Blinken 跟 Zelensky 的意图,所以李强不愿意去跟他们拍照。不愿意拍照,所以  Zelensky 就讲这句话,说我们这里在和谈,你中国的不支持和谈,是不是战争犯这样子?给他压力,他们是要给李强施加压力。结果我半个小时前不是说中国的现在的外交政策总算是觉醒了吗?李强真的就坚持的不去,所以他们这个拍照的时候就缺一个中国。



王孟源 44:53 

然后接下来欧美的主流媒体报道,我不晓得中文媒体有没有报道,大概是没有。但是我觉得这很有趣,就  Zelensky,人家问他说这个中国抵制你的这一次的拍照,这个外交的会。然后有什么看法?  Zelensky说我是总统,他是总理,他本来就没有资格跟我同时拍照。你说像这种傲慢、这种虚伪,对,这是不是可恶到极点?



唐湘龙 45:35 

自圆其说到这样的地步。



王孟源 45:37 

自圆其说对不对?他要骗人,我们不被他骗,他还反过来说你没有资格被我骗。



唐湘龙 45:44 

差不多就这个意思。



王孟源 45:46 

可是欧洲的现在当代欧洲的这些政治领导真的是真的是从垃圾桶里面捡出来的。最主要的两个国家,一个是德国,现在的经济已经被整完了。就是你如果我上个礼拜才刚刚在我的博客上面刊出一张最新的统计数据。



王孟源 46:17 

就是德国现在的平均发电量,在过去这一年,就掉 20\% 发电量,这。



唐湘龙 46:29 

太可怕,因为发电量是衡量一个国家经济现况的最准确的数字。



王孟源 46:35 

对,现在他们的发电量已经掉到低于 1978 年的水平, 1978 年。他的去工业化一下子倒退了 45 年。它的工业水平一下倒退了 45 年,你说这是不是可怕?一年之内倒退了 45 年,那结果现在你也知道柏林现在有农民的大规模的示威抗议。



王孟源 47:06 

可是农民的示威抗议它的起点是什么?它的长期的原因是它的经济不行了。嗯,就是你去工业化以后,自然就没办法支持政府的开销,支持农产品的高价位,对不对啊?因为比如说你台湾的,台湾一个台积电名义上它的营收只有台湾 GDP 的5\%,但是它支撑起来的台湾的经济至少有15\%,可能有25\%,因为因为台积电的那些工程师,他们买房子,他们去理发,他们去上馆子。



王孟源 47:47 

这些都支持,其他的更不用提半导体上下游的产业,什么日月光之类的,对不对?你全部加,全打全算。如果没有台积电的话,台湾现在的那个消费水平,跟个人收入都会大幅的下降。我们同样德国的化工产业,已经被因为不用俄国的天然气而完全消灭,然后它的发电量就一下直接掉下去,影响了其他的冶金跟机械。那在更不用提他们的这个汽车出口业也面临了中国来的电动车的竞争。所以雪上加霜。那你自然就没办法支撑农产品的高价位,政府的税收也不够去补贴农民。所以这是原因,近因则是因为他们真的是财政,因为经济出了大问题,财政跟着出问题,尤其是用几百亿的往乌克兰送,所以他准备调整农民的农产品的补贴,还有税收。最重要的比如说像那个农机跟柴油的那个补贴政策要调整,那农民是因为这个导火线而去抗议的,德国的经济被搞垮,他们的政坛里面的始作庸者是谁?是绿党,但是现在倒霉的是谁?倒霉的是自由民主党。为什么?他们是一个三党的联盟,总理Scholz是社会民主党,然后下有两个小党,一个是绿党,一个是自由民主党,只控制财政部。就是你绿党这样乱搞,把经济搞砸了,把税收搞没了。



王孟源 50:01 

嗯,到最后花销没办法,入不敷出了管理财政部的自由民主党要出来做坏人。你,然后。Scholz当然也必须承担责任。但是。在美国的安排下,最有希望替代Scholz是谁呢?是他们的国防部长,这个国防部长比Scholz还要亲美,你说是不是很好笑?你这就是。现代的美国殖民体系下的所谓的民主选举的现况,你搞得再烂也是好人被替换掉,然后当初把事情把国家搞砸的元凶,反而越升越高,越升越多。这就是一个很可笑的现象,所以思想跟制度是非常重要的,要认清真相。不是一般老百姓,不是一般选民就能做到的。我说过。我的博客已经我花了 10 年写了 1, 000 多万字才把这些事情解释清楚。博士级的,聪明的博士也要花 10 年几千个小时的时间才能够理解整个逻辑结构,所以你说这种事情你能交给一般的选民去投票理解吗?嗯,你投票的结果一定是主流媒体说什么就他们就投什么。那世界上控制主流媒体的是谁?是欧美,对不对?你就算把他抹干净,有一个新的竞争。媒体最需要的是什么?资本。国际上控制资本的是谁,谁控制美元?对不对?这种事情很明显,所以他们吹嘘这种所谓的自由。所谓的民主,原因是因为他们会赢,会占便宜,’不是因为这是对的,或者是好的,或者高效的。刚刚讲完了德国,我们现在谈谈法国,Macron今天才刚刚又讲欧洲必须要独立,你猜猜他为什么忽然哪一根筋不对了?忽然又讲到要跟美国要独立了,(为什么)因为Trump大概会选上。对,哈哈哈。



唐湘龙 52:47 

对对创,哈哈哈,对了,就是,当然他这些盟友都很紧张,Trump甚至昨天甚至于讲就是说大陆股市前天之所以大跌,就是因为我在Ohio打赢,把中国的股市都给吓趴了,(编注:这里应指Iowa,Ohio的初选在3月)他可以用他的一切去解释这个世界的一切,当时也很自恋的人。不过确实就是说你看到,你看到就是说的加拿大的总理的杜鲁多,或者是马克龙的讲话,其实都反映出相同的问题,他们对于川普即将回来其实颇有戒心。



王孟源 53:21 

今天大概没有什么时间谈美国大选。不过建制派已经层层布防,但是看来是没有用的,他们有一个De Santos 是他们的内奸。他原本在Trump任期内,他是国会议员,那时候他还是标准的建制派,然后在 2018 年他变成佛州州长以后,就摇身一变变成民粹派。这个就是派内奸进去要分最派的选票,失败了。另外一个Haley。



王孟源 53:59 

他这个是当初Trump第一次当总统的时候,他是南卡的州长。南卡因为是美国大选过程中的第三个州,所以很重要(编注:此处指的是共和党党内初选的第三个州,和大选不同,初选时间跨度长达5个月,先后顺序很重要)。没错,所以那个时候Trump明明很不喜欢她,还是把她拉拢到自己的团队里面来。那现在她就变成那种共和党里面建制派的头号旗手,但是这两个人在共和党里面已经,大家都看清楚他们的真相了,不可能。然后美国在过去这三四年虽然成功的软着陆了,但是你毕竟还是有 40 年来最严重的通胀危机。所以老百姓还是很不高兴的。我以前解释过老百姓对经济的感受,最重要的两点,就是失业跟通胀,两个加起来叫做痛苦指数。所以基本上今年年底Trump是可以躺着赢,这也是为什么现在民主党拼命的在各州搞官司,要把他拦下,要把他困住。这些都是两三年前,我在我博客就已经精确预言了,就是民主党一定是这样搞,然后建制派一定会派内奸,然后去层层阻截,然后 Trump 为了反击反而更是要坚持的选总统,而且他的言辞会更加的急。



王孟源 55:38 

其实我这样子就简单把美国大选讲一下,算是讲完了,我们回头再讲刚刚国际的问题,讲到德国跟法国的发展,那欧洲基本上已经完蛋了,没有希望了,尤其是德国,这个美国原本该怎么死,现在呢?是德国去替他替死,就是去工业化,然后经济入不敷出,通胀这些东西啊。那么我们回头来看看中东,这个中东这很有意思,就是红海。那因为俄国的揭竿而起,后来在非洲西非有一个起义,这个起义的对象是Nigier起义去要驱除法国的势力,结果成功了。真的居然这半年之后,搞了半年,把法国人的企业连他们的大使全部都赶出去了。然后巴勒斯坦就是Hamas 决定揭竿而起,利用这个机会来打击以色列。打击以后,果然在这个大环境下外交压力非常的强大,以致以色列到目前为止还不敢真正的狂轰滥炸。



王孟源 57:10 

OK,他们是炸了几个医院,造了几栋大楼,但是他们如果真的要搞的话,他可以跟美国要足够的 2, 000 磅炸弹把每一栋大楼全部都炸平了,但目前没有。没有的话你就必须要派步兵跟坦克,真正到市区里面,那步兵跟坦克进了市区,那就可以有游击战的余地,城镇战,那在这个过程中伊朗就躲到第二线,中国顺利的躲到第三线。我在一开始就讲过这个躲在第二线、第三线是合理的。



王孟源 57:58 

我上个月其实也在讲过,这个都是合理的,你没有必要。对有自愿要去第一线作战的人,我们配合他,但是其他人在第二线、第三线支援就好了,那美国本身的历史的战略一直也都是躲在第二线、第三线,美国从 1812 年主动挑起跟英国的第二次独立战争之后,到现在 两百一十二年对外侵略颠覆 400 多次,没有一次是主动挑衅一个同级的一流强权。



王孟源 58:35 

嗯,它的如意算盘就是让你挑起俄乌战事,然后让欧洲去制裁俄国,然后两败俱伤。你同样的,如果他要在台海搞事,也是要让日韩跟欧洲去制裁中国,然后两败俱伤,然后美国在后面收渔利,对不对?这一次的欧洲制裁俄国,美国是赚了,因为德国的那些去工业化,那些化工厂全部都到美国去了。那结果原本看来他在这个以巴冲突里面他也不会下场,结果有另外一个楞头青,就是这个胡赛组织,也门的Houthis,他真的就拿着飞弹跟无人机去打商船,逼着美国军舰出手,这个也是同样的一个多月前,两个月前我在我博客就讲得很清楚,一定是空袭,一定会导致美国人的空袭。后来上次 Blinken 去Davos之前先到中东转了一圈。转这一圈的目的就是要跟沙特还有阿联酋这些国家,还有土耳其这些国家,讲好说我要空袭了,结果他拿到的……因为你知道沙特跟那阿联酋其实跟Houthis打了好几年了,而且是巷战之类的,那种热战打的很激烈。



唐湘龙 01:00:27 

而且完全没有讨到便宜。



王孟源 01:00:30 

没有讨到便宜,对,所以这一次美国去的话,他们是顺水推舟,你们要打就打,要炸就炸。但是在巴以冲突的这个背景下,在俄国现在已经挑起反美殖民帝国的这个国际大背景之下,他们不可能真正的容许美国继续升级,美国自己也不想升级。我说过美国绝对没有跟自己同级的一流强权挑起战事的意愿,而且在过去的这一个最近的这个世纪中,你看看韩战,他以为北韩跟中共是第三世界,越战他以为越共是第三世界的军队,然后阿富汗战争、伊拉克战争,他们以为这些都是第三世界的军队,结果是怎么样子?陷入游击战的泥淖之后,通通是惨不堪言,所以他们现在也学乖了,连这种去打第三世界,他也是尽可能只空袭,而不作巷战。那你看看现在过去这个礼拜的新出来这个消息是什么?伊朗到处的去用飞弹去袭击了叙利亚,袭击了伊拉克,又甚至袭击了巴基斯坦,那这是不是很好玩?



唐湘龙 01:02:08 

他就攻击那些的独立运动。



王孟源 01:02:11 

对,他攻击的都是独立运动。以巴基斯坦的这个来谈,这个是所谓的俾路支,俾路支省。俾路支这个民族刚好是居住在巴基斯坦跟伊朗中间的那个沙漠里。这个沙漠叫做 Gedrosia沙漠。其实喜欢读历史的人应该对它很熟,为什么呢啊?2,500年前,2, 400 年前,亚历山大横扫中东、西亚,然后从西亚进入印度。OK,在印度打败了好几个当地的土豪国家,然后他的手下就哗变了。手下哗变说我们想家了,我们要回去了,而且占了这么多的地,抢了这么多的黄金。你不回家花的话也心里越来越不舒服,所以他被迫的撤军,从印度撤军往巴格达回去,当时叫做。当时还不叫巴格达。(编注:这里的古称应指巴比伦)当时他的帝国实在是太大了,包含现在整个中东跟西亚、阿富汗那,但是他在撤军的路程上故意的选了一个沙漠,就是这个沙漠,就是现在这个俾路支所在这个沙漠。然后这个行军让他自己的部队损失折损的 1/ 3。当时所向全世界所向无敌的这个。这一支部队在打片西亚无敌手,西亚跟南亚无敌手,但是在行军过这个沙漠的时候死了 1/ 3,你说这是不是很惨?他是故意的整这些部队。因为他们然后要叛变,那这是历史上这个沙漠出现的一个最重要的案例,那在最近的这几年其实关心国际事务的中文读者应该也知道。为什么?因为俾路支刚好就是说瓜达尔港,中国在巴基斯坦所建的那个瓜达尔港所在的地方。所以中国工程师被炸、被抢、被杀,很多都是俾路支组织做的。



唐湘龙 01:05:03 

都跟这个都是这个有关。



王孟源 01:05:04 

那因为他们追求的是独立,那他们的那个地盘刚好有一部分在巴基斯坦内,有一部分在伊朗境内,所以打伊朗的那些组织就以巴基斯坦为基地,打巴基斯坦就以伊朗为基地,所以,很好玩,所以伊朗这次就直接用导弹去炸这个独立组织在巴基斯坦境内的基地。然后巴基斯坦想说,哎,这个事可忍孰不可忍?你侵犯了我的国土了,而且你又不叫做美国,对不对?美国以前一年炸巴基斯坦,炸几千次的,巴基斯坦还合作。那你既然不是美国,你还敢炸我就必须要回敬,回敬的话,就是他也要也去炸俾路支在伊朗境内的基地,所以皆大欢喜,大家都很高兴。然后,但是我们退一步想,伊朗是要参战吗?嗯,没有没有,他这一次打的都是不会回手,保证不会回手的,他没有跟着去打美军,为什么?正因为美军现在陷入Houthis 这个泥淖之中,所以反而是让伊朗有这个余语去解决这些,去做一些很有针对性的军事冒险。但是这个冒险是完全跟美国无关的,所以它可以安全的去做,那所以我们可以看到由俄国所引发的这一场反殖民潮流还在波澜起伏,还在有不同的浪头此起彼伏,很值得大家的关注。



唐湘龙 01:07:07 

好,我问一个很笼统的问题,因为今天和孟源的连线会是在农历年前我们最后一次连现,下次就是龙年了。好,那因为 2023 年,今年我们回头去看的时候,其实每次跟孟源连线的时候,我们都觉得这个世界就是摇摇晃得非常厉害,就是就是很许多的事情都在一个结构重组的过程。 2024 年你怎么看?



王孟源 01:07:38 

2024 年我怎么看?俄国并没有意图要迅速的解决这个,嗯,战争他们的大。



唐湘龙 01:07:47 

他现在不急。



王孟源 01:07:50 

对他已经很明显的。拖得越久,对它越有利,欧美越没有元气,就是消耗起来。俄国基本上他的那些军需已经自给自足了,而且在经济上,这场战争对俄国经济的贡献就好像第二次大战对美国经济的贡献一样。反而是促进了他的复苏。



王孟源 01:08:19 

你这个仗要打多久?随便,反正到最后民穷财尽的是乌克兰,是欧洲。OK,那所以我不预期到今年年底这场战事就会结束,那所以在这个背景之下,会有新的反殖民的此起彼落,就是这些被定义成恐怖组织,或者是那其实是被压迫民族的一些彼此的冲突会不断的冒出来,你在那个西非的,你觉得政变之前你没办法预测是没错,会发生这种政变,对不对?但是你事后的话就看起来很合理。就是合乎情理,但是无法确实的预测。那觉得目前的大趋势就是会继续。那当然, 2024 年最大的变数就是美国大选。没错,那尤其因为 Trump 的在国内民意的强势,所以你可以看出 Biden 会很着急的解决这个国际上的这些战争跟冲突的问题,因为这些都是他大选上面的拖累嘛?你说那个前两年他们一直说这个俄国马上,再差一把气就要死了。甚至连Putin本人阵亡……而不是阵亡,就是病死的消息都出现了。



唐湘龙 01:10:05 

没错没错。



王孟源 01:10:07 

现在还有人会相信这种事情吗?没有对不对?现在急着要和谈的是乌克兰方,为什么?因为知道打不赢呢。而且到年底就会对大选有恶劣的影响。很好玩的是,上个月纽约时报还特别刊了一篇文章,说是根据匿名的情报官员,普京现在很着急的要跟美国和谈。我这很奇怪,你普丁刚刚在俄国做了两场公开讲话,都是史无前例的强硬。OK,你明明是,你如果看他所讲的话的字面,是非常非常的强硬,而且你如果是从客观的战局来看,我上个月也才刚刚讲过俄国如果没有拿下乌东八州,



王孟源 01:11:13 

不是乌东四州,而是乌东八州的话,还有整个黑海沿岸的话,不可能结束非常战斗。那这种在这种条环境化时间下Putin怎么会急着急?他根据纽约时报这篇匿名报道,是Putin急得不得了,如这个锅上的蚂蚁,要跟美国和谈、那你用肚脐想一想也知道,事实上是美国急得不得了,要跟俄国和谈,要为了颜面。所以先事先铺垫,说是Putin要和谈的,然后你这样和谈以后才可以自圆其说。为什么我们每天都打赢了,两年打 700 天不停地打赢,打到最后还必须要跟他和谈哦?是因为Putin跪着求我们,对不对?你说他们就是这么无耻。那另外我刚刚谈的是这个,各地的势力。都觉得在这个大环境下可以有自由,可以为所欲为。我刚刚举的例子是伊朗跟巴基斯坦,可是伊朗这样轰炸了一些境外的目标之后,连约旦都去轰炸境外的毒贩毒枭,对不对?约旦你算是哪门子的地方强权,地域强权啊。连约旦都觉得可以出手,所以这个世界的外交局势,战略局势。嗯,在 2024 年会是很混乱的,就是小猫小狗都觉得可以出手。



唐湘龙 01:13:00 

对,就是和平不在是种信仰,用军事手段去解决问题变得越来越轻易。这个是 2023 年所留下来的后遗症。好,孟源刚提的一点,如果你原来预判的就是说的,俄罗斯如果没有拿下乌东八州跟黑海沿岸的话,这场的战争都没有停止的条件。要拿下黑海沿岸,那表示他得要打到敖德萨。



王孟源 01:13:28 

对,如果需要,还需要两年,他们就会再打两年,如果需要三年,那就会再打三年。



唐湘龙 01:13:34 

那这个就这要碰到敖德萨的话,那就是很大的事了,这个比哈尔科夫更敏感了。好,这个呢大家就可以再观察了,反正这是孟源的预言了,就是让大家知道,就是说 2024 年仍然非常动荡的一年。但是要如何避免这种情况获取川普当选的是一个解决方案,因为川普讲过,只要他当选了之后,第二天俄乌战争就结束了,所以他就要等那当选的第二天,俄乌战争是不是就戛然而止?好,今天非常感谢,因为是时间的时间到了,那透过呢?我们的越洋的连线,每个月跟孟源的连线其实是很有获得感的连线了,那我很多的我们的平台上面的粉丝们对孟源的也都非常肯定的评价非常非常高。



唐湘龙 01:14:27 

好好来,我们来感谢了几位的观众朋友,来 brought K001,谢谢逍遥游,感谢,然后破论,感谢,然后这什么陈Sophie,感谢,然后么? Mala 诚感谢,然后赖建成,谢谢您,您破费了,然后呢?这个叶珠晨,感谢,然后螃蟹感谢螃蟹说大陆在美国制定的游戏规则下不可能战胜美国,十年前的大陆别说帮助别人脱离美国规则,连自己都必须要为欧债买单。



唐湘龙 01:15:01 

现在中国才有足够的国力去建立自己的游戏规则,大陆这一次的物价通缩正是对抗欧美体系的表现。他说, 08 年、 08 年、 2015 年,美国跟欧洲印钞都把通膨转嫁给中国,疫情所超发的货币却只有欧美物价飙升,而大陆物价平稳还有通缩反而是一件好事。我们的这个螃蟹的这个观点您同意吗?



王孟源 01:15:30 

大陆现在失业率是偏高。对,但是通胀率很低。所以你整个痛苦指数并不高。大家要体谅执政者在美国霸权体系仍然存在的背景下,他们的这个政策最优化,其实他们的这个操作挪动的余地是限的。是那当前的这种环境经济环境其实是已经相当不错了,甚至你有 5 \% 上下的……上下啊,包括四点……然后通胀基本上不存在。事实上我听说现在大陆的最近有很多销售渠道,都还在拼命的降价。大家都在比性价比,这是很好的事情,对不对?你就当然这个失业率需要再压下来,但是大家不要忘了,你如果为了压低失业率而把通胀放松的话,很可能得不偿失,尤其是大陆,因为过去这十几年、二十年的房地产,错误的房地产政策,包括 2009 年之后的注入的,我上一次讲过,9万亿、 9 万五千亿的资金把泡沫吹起来,这个房地产泡沫到现在还没有消化。



王孟源 01:17:02 

如果在重蹈覆辙再拼命的印钱,去重启这个泡沫的话,这个问题永远不会解决,只会越来越大,对不对?你这个现在他已经忍受了三年的痛苦,想要把这个泡沫刺破,你如果半途而废的话,是浪费了过去这三年的,失业率高的这个痛苦,而且反而使这个问题延续下去,而且反而比不当初三年前不处理还要更严重。那这样子得不偿失对不对?



唐湘龙 01:17:41 

唉,好,感谢。来,再来,我们看到90K,谢谢。然后蔡明华、通华我不会念好,这个都该感谢了,感谢。好,再来呢。这个玲玲,陈镇,感谢再来斯丹利在泰国,感谢,好,我们感谢所有的听众跟观众朋友都专心的你的聆听,因为我咱们说听孟源讲话是需要非常的专注的,你才能够顺着他的思路跟他所提出来的观点才能够慢慢的消化好,当然特别感谢孟源,在他所在的美国的东部,冰天雪地的情况之下,外面是大雪覆盖,他一个人躲在一个连暖气都舍不得开的一个房子里面。然后就这样的连线。好,这就非常的辛苦。好了好了,我不管怎么讲,谢谢孟源,谢谢,谢谢孟源。好,那跟孟源先说新年快乐,然后我们下次见面的时候就已经是农历新年之后了。新年快乐,下回见啊。再见。快乐,明年见。OK,好,梦远。



王孟源 01:18:49 

拜拜,大家拜拜。



\twocolumn[\begin{@twocolumnfalse}
\section{全資開放美國投銀,中國歷史性錯誤!}
\subsection{20240216}
\end{@twocolumnfalse}]2月16日,已校对完成



我是另一位读者,已再度修正。另,记得时刻留存备份,开放文档防恶人小人作乱。



唐湘龙 00:21

好了,欢迎来到龙行天下,我是唐湘龙,星期五的时间,同时也是开春之后龙历年的第一档的龙行天下,但在过年期间,大家讲一些的吉祥话,就是我们龙年那种的,都要祝福大家。龙行天下,龙年行大运呐。所以当大家在祝福的时候,我都觉得与有荣焉,哈哈哈,自己的名字里面就有个龙字,今年是我的本命年了,好。大家在这地方第一次来到了龙行天下的栏目的那跟大家说新年快乐,因为现在还在农历的新春期间,那今天大年初七,那在台湾各地都已经正常的上班课了。那龙行天下的第一档,那我非常期待的就是过年期间的时候,我就想说过完年之后,因为这段的时间里面其实全球的政经形势变化非常的剧烈,在经济面来讲,你会看到股会是美国的股市大涨,频频创新高,延续 2023 年全球股市的那个热度。台湾也是一样,开盘了之后资本市场显得非常的兴奋,非常的有活力,就开盘红的那个味道在全球各地弥漫,俨然 2024 年就是风光大好的一年。



唐湘龙 01:50 

好不过有关于全球的政经情势的整理跟预测,对我个人来讲,从我第一次在网络上面看到王孟源讲话的时候,我就觉得王孟源是一个我非常愿意去倾听的对象,因为每次听孟源讲话的时候,我总是觉得有非常大的一个收获。好,所以开春了第一档的龙行天下标定每个月的第三个星期五的时间,我就会请了王孟源上线来,今天在我们线上,人在美国,天寒地冻的情况下面勉强供暖呢。大家聊了重大的国际趋势的王孟源,欢迎。



王孟源 02:29 

大家好,祝大家新年快乐。



唐湘龙 02:31 

新年快乐,好,你像你这样一个人过年的时候你在干嘛?就中国人过农历年,你一个孤家寡人在美国在扮隐居的生活状态里面的时候,你会有过年的准备吗?



王孟源 02:42 

我不过任何的节日,包括我自己的生日我都不过,这个又是很正常的阅读思考的日子。



唐湘龙 02:51 

这个跟我也非常非常像,没有,我跟王孟源也非常非常像。我常跟大家说我不过节,他很难想象说不过节连生日都不过嘛。对,我的生日是哪一天,其实我从来都不提醒,周围的朋友也几乎都不知道。好,这不是重点,来,今天龙行天下,我们谈两个主题,来,我们先从俄乌战争来谈起。俄乌战争到了 2024 年的时候变得非常的吊诡,一方面是美国总统大选,看得出来就是两个走主要政党的两位的总统候选人。拜登现在跟欧盟、跟北约国家、跟欧洲国家来讲,看起来现在仍然是航线,一汽就是要努力的撑着这乌克兰,撑着泽伦斯进来继续的打。那民主党看起来态度也很坚定,所以不久之前在参议院里面发动,就是说这种的突袭式的,那通过了包裹的以乌克兰为主的以色列跟台湾陪绑的这样一个对外的援助法案。当这个援助法案共和党一定很快有反应,共和党控制了众议院,是管钱的,众议院大概不会通过这个案子。



唐湘龙 04:00 

可是你会发现有关于乌克兰的问题,拜登的讲话跟川普的讲话是在两个极端上面。再加上最近就是说川普可以说是这个他的粉,就是这个塔克去访问了普丁这件事情,因为在网络上面点阅率已经好几个亿了,它的影响力是不可以小看的,以至于我们不太理解,就是到底乌克兰战场当中发生了什么事?那乌克兰的本身我们看到的是是是这个乌克兰总统泽伦斯基把这个指挥官给换了,那这个指挥官换,大家都知道这个指挥官本来西方的媒体的新闻一直在暗示,就是说这个指挥官在暗中的,他在第一线,他在负责处理跟俄罗斯之间是不是达成停火协议的可能性。可是从不管是普丁的讲话或者泽伦斯基的讲话,或者拜登的讲话,停火看起来不在现在乌克兰或者西方阵营的主要的思考的范围之内。好,我们就从俄乌战争谈起,俄乌战争的最新的进展。



王孟源 05:09 

我想延续一下两个月前我讨论的俄乌战争的分析,当时我已经特别预先强调Zelansky跟Zaluzhny两个人之间有一个很强的政治矛盾即将要爆发了。那结果就真正在这个月爆发,一开始的时候是泽兰斯基将Zaluzhny叫到基辅,然后跟他说我要任命你做个什么闲差?对,有说要到英国当大使。是的,说要当什么啊?国防顾问的,但是Zaluzhny就把他完全的拒绝了。那这样拖了一个多礼拜,最后还是硬是把他换掉了,这我两个月前其实就已经讲过,这必然会发生的。那我也提个到最后的仲裁很可能是由美国的国务院来决定。那国务院的主流主导者是Neocon,就是新保守主义者,那这些是最疯狂的强硬派,他们的负责对恶的那个人就是他们的复国务卿,叫Nulan, Victoria Nuland在Zelansky跟着Zaluzhny开始争吵的时候,就飞到基辅。那这个时候既然是她出名面了,那基本当时就已经可以确定Zelansky会胜出了啊。



王孟源 06:48 

那两个月前我已经解释过,泽兰斯基他基本上没有任何实际国内的势力或班底,他就是一个演员出身,他的任务就是对外做表演。所以他唯一背后幕后支持的就是美国的这些Neocon,OK。但是Zelansky是实权派出身的,他是一直跟那些NeoNazi就是新纳粹主义的民兵关系很密切,而且他手下有很多嫡系的最精锐的,比如说那个亚速营营,后来变成正规军之后就叫做第三旅,这就是他的亲兵啊。那在政策上面,当然他们两个人也有很强的冲突,那也是同样的,出于他们的立场的不同,Zelansky最关心的就是对外制造新闻表演项目,所以你他最不希望你的领土被俄军占领。而Zaluzhny,因为他本身是专业人士,而且手下的精兵又是自己的亲兵,所以他当然是比较惜命,不希望这些人白白的牺牲,所以这是他们两个最大的冲突,那所以每次俄军把乌军半包围,然后在里面消耗的时候,我想提醒一下,这两年来俄军打的包围都没有要试图包围全歼,而是每次都只是半包围,为三缺一,然后用火力制造杀伤啊。这个是因为俄国俄军的战略思想原本就跟西方或者德国系的不同。你光看他们的那个名字就可以看出一点端倪来,美国人在你把敌军包围起来的时候,它叫做一个 pocket /口袋;但是俄军俄国人所用的词汇不是口袋,而是翻译成英文的话叫做cauldron一个火锅。就是你并不是要把它整个包起来,然后丢到热桶里面去,而是把它放在一个很困难的状态下去煮它。嗯,制造杀伤,所以你可以看出这个其实是符合俄军的传统的战略思想。



王孟源 09:37 

那同样的,一年前他们所制造的那个cauldron是在Bakhmut。我想大概有讲,当时一年前我上节目的时候,我就说过这个Bakhmut即使到最后乌军被全部打出去,那二军还有接下来有两个可以打的,一个是 Avdeyevka,另外一个是Sivers'k。那后来的确是俄军选择了Avdeyevka。嗯,那我们现在看到的这个战况非常对,从乌方的观点来看,战况非常吃紧的就是Avdeyevka,阿夫迪夫卡,Avdeyevka一样,基本上是重复了一年前Bakhmut的那个惨状,目前据称有 4, 000 名乌军被包围在城区里面。



王孟源 10:29 

那遇到这种情况,Zaluzhny通常就会避免投入新的兵员,尤其是那些精兵,他通常是把国土防卫部队就是民兵丢进去填战壕,然后拼命的跟子兰斯基要求尽早撤退,这其实是在军事上是合理的,因为你要避免杀伤,但是Zelensky因为他关心的完全是美国国防部的那些Neocon给他的任务就是要制造好新闻,所以他一向是不惜人命的要往前填。



王孟源 11:14 

那这一次的冲突其实是来自这两方势力原本就看不顺眼,然后一直都有策略的偏重,也是互相抵触,所以一直谈不拢,还不是真正主要原因。真正的导火线真正会爆发出必须要你死我活的这个导火线,是因为他们原本照他们的宪法是在3月中必须要选总统,也就是 Zelensky 这个这一任的总统快要做完了。那但是他现在靠的是每隔三个月延期一次的戒严,这个戒严刚好在两个礼拜前通过了,最新的一次三个月的戒严可以拖到5月。那他们照他们的宪法应该是3月中选举,然后5月换总统。那  Zelensky  因为眼看着到5月之后就必须要靠着戒严才能够维持总统的位置,到时候就名不正言不顺,所以就很担心这个跟他一下子谈不拢的总司令会变成他的政敌的一个主要的力量,会产生政变的危险,所以才会急着要把Zaluzhny换下去。那果然不出所料,今天传来新闻,就是已经被换下去 4 天的Zaluzhny,正式的加入了Poroshenko(波罗申科)的政党。那Poroshenko是谁呢?就是乌克兰的前任总统,也是波Zelensky现在最大的政敌。



唐湘龙 13:09 

好,我请小编把几张照片找出来,就是说除了这个Zaluzhny之外,另外他们的新任的总司令,待会我也请了孟源帮大家分析一下。那波罗申科是前任的总统,虽然有点灰头土脸的,不过现在以美国以西方国家来讲,把他摆在了乌克兰境内,当作是随时可以替代泽伦斯基的力,就是说这政治力量之一。那另外刚提到的阿芙迪夫卡,这个在这个地点在战略上面来讲有什么意义?俄军是不是在它的传统已经固守了防线,已经离开了防线,再往前推进那个地点的战略位置,我也请小编准备一下。好,孟源请继续。



王孟源 13:49 

好,那 Zelensky换掉Zaluzhny之后换上来的新任总司令叫做Syrskyi,他是纯种的俄国人,他的爸爸还有他的兄弟姐妹都还住在俄国,很特色。当初苏联解体的时候,他刚好那时候在苏军服役,然后刚好是在乌克兰,然后就留下来了,分分家产的时候分在乌克兰这块。哈哈哈,那你想这么尴尬,这么尴尬的一个身份,然后又没有跟乌克兰的那些新纳粹,有极端主义者有关联。那你想他要脱颖而出,要往上升的那个管道,就只好去讨好Zelensky。所以你如果看他现在有很多报道说他的他就是凭着,不顾自己士兵的死伤而拼命的往前填战壕。所以他有一个外号叫做屠夫,可是你想想看,不一定是他本身的个性如此,而是他的出身就注定他只能够靠这样子往上爬,但爬上来以后,他当然第一个反应是先把Zaluzhny的那些亲兵,比如说我刚刚提到的第三旅作出处置。什么样的处置呢?就是第一个先把他的那个旅长还有其他的中高层干部,能换的尽量赶快换。第二个处置就是把他送到前线去当炮灰,要让他们赶快的去打掉啊。这两者都在过去这 4 天发生了。如果你去注意乌克兰出来的新闻细节的话,Avdeyevka最近战况非常的吃紧,被紧急送上前线,填的就是第三旅。三天下来就死伤了 20\%,然后被迫的往后撤。那这个 Avdeyevka其实是俄军已经打了半年的,一个我说过的Cauldron。Y那它比Bakhmut要小很多,他在战前只有3万人口。他是一个长条形的城市,从北到南,那乌军在里面构建了四个要塞,北边一个要塞,中间一个要塞,南边一个要塞,东边一个要塞,俄军从东边跟南边来包围它。



王孟源 16:37 

那在两个礼拜前发生的两件事,第一个是二军的 Wagner group 突破了他们南边的要塞,然后他的正规军,俄国的正规军从东北边插入了中间要塞跟北边要塞中间的城区,把它把北边的那个要塞分割出去,然后已经威胁到中间那个要塞的后勤路线。就是现在已经切断了他们主要的后勤补给线,现在乌军勉强靠着乡村的泥土路再继续的补给,所以四天前就是第三旅被紧急的送上前线去挡这一波的攻势,结果没有挡住。今天的新闻传出来是俄军在第三旅的正面又前进了一公里,眼看着守不住了,那这时候 Avdeyevka的城区,还有它的东方的那个要塞,一共还有 4, 000 的 4, 000 名乌克兰部队,大约是两个旅残余部队。那这个现在要逃出去的话一定会被狂轰滥炸。不逃的话就会变成俘虏,所以是非常困难的。那我认为是,虽然我们可以预期在未来几周Avdeyevka的战事就会结束,如同Bakhmut一样,俄方胜利而结束,但是我们很值得观察的是接下来俄军的下一步。在一年前我说Bakhmut吃下来之后,二军会重复这个 Cauldron的战术在不是 Avdeyevka就是Siversk,那现在Avdeyevka下来以后是不是一定会去打Siversk?其实不是,它有三个可能的选择,第一个是Siversk,嗯,就是重复我们刚刚提过巴木跟这个Avdeyevka的 Cauldron战术。第二个是做突破,然后突破又有两种,一个是在北线的Kharkov在附近作突破,另外一个是在南线的,就是 Donesk跟Zaporozhia之间的南线战线做突破。



王孟源 19:18 

为什么我说可以做突破呢?是因为又经过一年乌军的杀伤非常的严重。两个月前我提到乌军目前面临了五重的问题,第一个是快要没人了,第二个是快要没钱了,第三个是快要没有武器弹药了,嗯,第四个是美国要大选,它的支持有问题。第五个是欧洲现在经济出了问题,所以内部的示威抗议层出不穷,基本上所有的主要国家都有农民首都占领,你主流媒体低调报道,可是事实上人数非常的多。而且非常普遍。那在这种情况下,欧盟还是硬着头皮强迫匈牙利通过了 500 多亿欧元的资源,这对乌克兰来说当然是救命钱,但是还不够他半年花的,这也是为什么美国现在急着要把那 600 亿美元的援助赶快通过国会。但是问题是这个现在他众议院的议长是同情Trump的,所以他不愿意去跟敌对的……也就是Trump,代表的是民粹,民粹就是 America first,美国优先。那他的敌对的就是主流派,也就是建制派,代表的是大财团,这些大财团他们当然包括军工,还有金融,那这些军工金融跟能源在俄乌战争里面是有很大的好处的。比如说你光看能源,我一年前就讲过它光在 2022 年美国的能源产业就多赚了 5, 000 亿,军工规模比较小。但是你光看外销不算,美国自己的采购其实才是大头。我刚刚看到他们外销的数字是美国要从 2021 年的 400 亿增加到 2022 年的 500 多亿,然后增加到 2023 年的 800 多亿,所以你也可以看出他们军工业也是赚了多赚了几千亿美。对,那这些钱都可以拿出 10\%, 15\% percent 出来游说。



王孟源 22:02 

所以,当前因为美国众议院的议长是比较偏民粹派的,所以才会抵制这个建制派要资助俄乌战争的事情。那我两个月前有说过拜登的政府有在考虑是否停火,这个是因为未来要选举的关系,也是Sullivan这个那一派它相对的比较温和,但是你光看Nuland就是两个礼拜前是 Nuland到基辅去解决这些当前的问题的时候,你就可以知道是强硬派胜出了。



王孟源 22:48 

OK,然后你看美国国防部长(Austin),他过去这几个月因为生病,他生病这个这件事就很诡异,他明明是生了重病,他偏偏要保密,连内阁还有总统他都不报告,对不对?这种事情只有他自己职位有危险的时候或者他现在正在一个主要政策争议,正在紧要关头的时候,才会怕人家知道他生了重病,对不对?那Biden已经到任期的最后一年,还不到一年,不到 12 个月了。这个不可能会去莫名其妙的换国防部长,所以可以,我们可以排除第一个可能性,所以一定是美国Biden政权内部在两三个月前有重要的政策分歧,有内部的争吵。那这个Austin他不愿意因为自己的身体健康的关系影响到争吵的结果,因为它代表的是军工,对不对啊?Austin是应该是Ratheon的人,我记得好像是Ratheon的。那所以算一算的话,很可能就是当时这个强硬派跟温和派。在还在争执,就是我刚刚讨论的,那结果现在很明显是强硬派胜出了,就是Nuland过去把Zelansky 定调,就是由他来主持这个乌克兰,然后把这Zaluzhny换下去。那所以我们先可以看出,对,这位就是Syrskyi,他是纯种的俄国人啊。。



唐湘龙 24:59 

长得也非常俄国人的样子,而且我看乌克兰对他宣传就是前一年的哈尔科夫的大大反攻,那就是这个新任的就是说武装部队的总司令的战功,那就他的特别的描述。刚孟源谈到的现在的美国的副国务卿,就是说Nuland,我建议大家可以可以稍微做点她的功课,因为其实俄乌战争会打到今天这个惨况,Nuland跟她老公是非常关键性的角色,Nuland基本上面是就是Neocon 在操作乌克兰的关键性的人物,而她老公其实就是Neocon  的理论大师啊。这个对于整个的整个 Neocon 的思想跟实际的行动来讲,这个这对夫妻,尤其Nuland的代表性,不要低估他,虽然他现在是副国务卿,可是在乌克兰议题当中来讲,他举足轻重。



王孟源 25:55 

他们不只是Neocon,他们还是Neocon里面专门针对俄国的专业户。然后所以美国的国务院里面会有一个专门针对俄国的尼欧卡做副国务卿或助理国务卿,然后会有另外一个专门针对中国的业务官,那这个是前几个月刚刚退休换人,那这是值得观察的,那所以我刚刚提到这个Avdeyevka的结局其实已经确定了,就是俄军会大胜,那大胜之后接下来的这未来这半年到一年俄军会怎么打呢?嗯,他如果选择继续去打Siversk,嗯,那就代表俄军准备慢慢的打。



王孟源 26:46 

就是继续的去杀伤乌克兰的军队的数量,因为他人已经基本只剩下 25 岁以下的年轻人还没有动员。现在基本上是乌克兰已经全部动员了,就只差 20 出头的那些年轻人还没有动员。但你想想看,乌克兰是在苏联分裂之后啊。同样的比俄国的经济还要差的, 1990 年代他们经济比二国还要差,所以那十年的出生率一下子打腰斩。他原本二十几岁的这个人口就比三十几岁、四十几岁的要少一半。那你即使去动员这些年轻人也没有什么太大的用处。那所以俄军的一个做法就是继续的延续我们过去这两年,我现在大家已经有一个标准的描词汇来描述,叫做 aggressive attrition,积极消耗,那他如果是要继续这个aggressive attrition,就会去打Siversk,但是他现在有可能认为已经消耗的差不多了,可以把战线往前推的更快一点。这有什么好处?可以影响拜登的选举。



王孟源 28:17 

我刚刚说到前两三个月拜登的那个内部会有争议,有议和派跟强硬派的争议,主要就是因为怕这个战事的反转会真正的对Biden的选情雪上加霜。那相对的你就可以看出,从俄国的观点来看,可以加速的推进来影响美国的大选的选举。那尤其是基本上你如果从战略的观点来看,乌克兰的东部防线,这个Bakhmut的到往南,Avdeyevka往北,到 service 是他的倒数第二道防线,然后接下来就可以碰到他的倒数第一道防线。



王孟源 29:11 

那你已经打下,你如果已经把这个倒数第二道防线打下 2/ 3 的时候,你没有理由非要把它全部打下来不可,因为你已经可以直接去打最后那道防线了,下一道防线。所以如果俄军决定要在领土的变更上占领土地的变更上面有明显的表现的话,它可以直接的或者选择从北线打,或者选择在南线打,因为这两线是比较薄弱的。那我个人会认为南线比较更有道理,因为瓦解了南线以后就可以席卷中线啊。北线跟中线之间有一条河隔着,所以它的战略意义比较低所以比较合适从南线来打。







王孟源 30:13 

我顺便讲一讲,这一次的这场战争有一个新的兵器,有一个新的 miracle weapon,就是,但是不是美国人在边吹的,换一个什么海马斯之类的导弹。

这一次的战争,军事从来都是攻与守之间寻求的平衡。每次有一个新的进攻兵器或者一个新的防守的兵器,就会影响完全的扭转既有的战术甚至战略。比如说 1870 年的时候,普法战争当时就有两个新的因素,第一个是火车,德国用火车来做后勤,结果他们的能够比以往集结快三倍。然后第二个就是他们的后膛枪,所以普法战争的时候有后膛枪,到了一战的时候当然就是机关枪,然后后来有坦克,然后到二战的时候已有了战略轰炸,还有真正高速的坦克集群。那我们这一次真正看到这种革命性的新的武器,使这一次的战争它的形式超出所有人的预料,这个真正的 miracle weapon 奇迹武器是无人机,对不对?那我在半年多前在你的节目上的时候说乌克兰每个月要买3万台的无人机,就是大部分是大疆那种小型的无人机。那当时听起来很可怕,因为3万台一天 1, 000 台,对不对?这些无人机,每个月一个月的都要买新的,就代表他在战场上消耗了这么多。一个每天 1, 000 台很可怕,可是刚刚在前两天证实,就是乌克兰有一个高官说漏了嘴,他说每个月3万台,但是能运到前线的只有 1 万多,那所以的确是这个很可怕的数字,那你可以想象俄军当然用的更多。



王孟源 32:51 

那这个无人机为什么能出乎所有人预料的?尤其是这些商用的小型无人机为什么会有这么重要的表现呢?扭转战争态势的那个重要性,主要是因为它很难干扰。大家原本以为说一个无人机的话,你必须是要遥控它,又没有AI,所以你用电磁去干扰就可以让他们失效。结果不是,这就是我们就要回溯到无人机为什么会在这个年代忽然出现?它其实受益的就是过去这 20 多年手机的技术。第一个是因为手机,所以我们才会有现在这种高性能的电池。第二个是因为手机,所以才会有这种微型,而且 v 小,而且解析度这么高的摄影头,对不对?第三个就是数位通信,这个抗干扰能力极强的数位通信能够做到一个小小的晶片里面去,这也是靠手机做出来的,所以我们要了解这是一个革命性的背景。大家不要把乌军或者俄军的一些表现就说他们是菜或者是怎么样。并不是这样的,你事实上是换谁来都一样。就是大家原本都没有看过这样的战争,这种一边每个月3万架无人机投下去,另外一边可能一个月 10 万架投下去的,无人机投下去的战争,那这是一个全新的战争形式,所以以他们的在头两年双方的准备优化都不够好,这是很合理的。不是他的那个智商特别低或者怎样,或者是国力特别弱的原因,因为你看乌克兰其实是整个北约在负责后勤跟情报,对不对?你不能够说他这个战力不够高啊。我今天有关俄乌战争就讲到这里。



唐湘龙 35:13 

好,接下来我简单讲,就是说刚刚孟源提到的这一段,它显示在俄乌战场现在跟国际政治、跟欧洲政治,跟美国的国内政治是紧紧的绑定在一起的。那未来的俄乌战场的形势的推移,现在俄罗斯是有主动性的,包括这最近Tucker访问了普丁所释放出来的讯息,因为在这西方的传媒体系里面的讨论度非常高,点阅率很高,它是对于西方舆论是有影响的哈。从那塔克的访问当中的许多的回应来看,大家似乎是认为塔克尔平衡了西方媒体过去孟源,不断的在质疑的西方媒体对俄乌战场当中报喜不报忧,然后隐蔽俄罗斯本身在这一方的有力的新闻所导致的国际舆论的误判。那无人机的部分其实不只是在乌克兰战场,我们看到最近美国在红海当中为以色列为势的时候,无人机对于美国的威胁仍然是非常大的,美国的这些在红海附近维系的这些的舰队也受到无人机很大的威胁,这个对未来战场的影响都很深远。好,这个问题会留下来的,是刚刚的孟源提到的乌克兰现在遭遇到的困境,美国要援助乌克兰,预算现在是卡住的,看起来没有到总统选举之后,它没有任何改变的可能性。欧洲现在情况乱七八糟的,就是欧洲现在的农民的抗议只是其中的一个起火点而已,而是欧洲现在的总体的状况,大家自顾不暇,嘴巴上面讲是一回事情,有多少力量能够去援助很有限。



唐湘龙 36:54 

第三个是乌克兰现在的师老兵疲,许多的年轻人躲在家里面不敢上街就可能被抓夫。好,这种都使得乌克兰现在的士气已经不是换指挥官的问题,而它总体士气低落,对于这场的战争,乌克兰已经没有人在谈能不能够战胜的问题。现在只是谁能够把这场的战争做一个让乌克兰,让西方、让拜登能够接受的收场,这个是接下去在战场以及在政治层面当中我们会持续的观察的。



唐湘龙 37:26 

好,我们再回到另外的一个主题,这个议题是我陌生的,但是这个是孟源,可能非常娴熟、非常有想法的,过去一段时间我也看到我们的留言板,上面会有一些呢熟悉大陆的朋友们的流化,那对于孟源不断的在批评大陆的,就是说金融单位跟金融操作,那有很多关键的时间点了,可能操作错误方向是错误的,它不止给大陆带来了困扰,同时让美国呢?美国这个国家遭遇到空前经济困境的时候,竟然让美国给跑掉了,这件事情孟源耿耿于怀,他提过几回。



唐湘龙 38:07 

好,但是今天我们谈的这个角度,孟源特别关注上个月大陆方面片面的那开放美国的就是说投资银行可以全资,那在大陆投资这件事情在大陆的金融管控的角度来讲,这是一件大事情,基本上面就是等于是把金融市场相对的开放,让这些以美国为首的,全球的首屈指的投资银行可以在大陆里面可以自由的进进出出。孟源为什么觉得这件事情可能是中国大陆所犯下来的关键性的、历史性的错误?为什么。



王孟源 38:45 

我认为这一件事情之严重性远超于我过去这一两年所讨论的啊,在美国通胀危机来临的时候去,不但没有落井下石,而且是反过来去帮助他度过危机。这一件事那样子,那件事只不过是延缓历史潮流, 5- 10 年就是美国可以撑到下一场危机。但是这一次的这个开放投资银行就是金融对外开放的,我认为是在挖社会主义的根。从这个月开始,中国共产党不再是一个社会主义政党。OK,我来解释一下为什么,而且这件事情发生的之诡异就是我上个月才刚刚提过,习近平刚刚发表了八项坚持。这个八项坚持的精神就是你要遏制金融,要不能够模仿美式的那种虚拟化经济,结果在实际操作上就反其道而行。这唯一的可能就是整个金融业,包括它的金融专业主管,全部合起来,欺蒙最高层。跟他说好你的八项坚持是这样的,那我这个开放是。不违反你的原则,可是实际上它是最最颠覆性的一个错误决定。他们的借口,我想顶多只有三种可能的借口,就是我看不到他们内部讨论。



王孟源 40:48 

第一个是资金。美国从五六年前开始搞中美脱钩之后,他的这个态势就是要把中国踢出它的国际金融殖民体系。美国的这个国际金融殖民体系我在最近的这个博文里面详细的解释过了。他是一种养猪的体系,就是在美国景气的这个环节,比如说他这个周期性有十年的经济周期的时候在景气的那大部分时间,他向外放出科技,这个科技可能是学术交流,或者甚至是技术转移,我们台湾本身就是被养得很肥的一只小猪。所以你可以想想看,台积电就是这样子养出来的。当然当年的台积电的创始人花了很大的功夫去引进这些技术吸收,然后好好的重新再创造。但是当初如果美国不允许,美国如果用打压日本的同样手段来打压台湾的话,台湾绝对不会有机会。



王孟源 42:11 

这个,所以第一个养猪的饲料有三种,第一个就是技术。还有你的那个科技交流,你看过去这五六年,他们的第一步就是先去起诉了,逮捕,起诉了好多华裔的美国学员。为什么就是要切断这个交流?这个交流是什么?是他们养猪的那个技术的一个管道,所以要截断。第二个就是美国市场,我过去两个月也提过,我在美国的话就注意到美国的这个电车,电动车比中国要贵,所以它的渗透率就比中国低很多。



王孟源 43:00 

台湾更加的严重,所以这个原因就是因为你现在只要牵到牵扯到一点中国的制造,只要他认为是重要的高科技产品,他就这个市场连 1\% 都不能给你。这个都是确定的,没有办法挽回的,OK?对,过去六七年美国换了两个党,还轮换过,而且变本加厉的搞出来的。嗯,最后一个就是资金,这个资金当然你既然市场都已经没有了,然后科技又不能够转移,那当然个投资就要逆转,从以前的外资流入变成撤资逃离,那在这种大环境下,美国要负担的代价是通胀,对不对?因为他们就没有了来自中国的廉价产品。他们必须要重复投资,然后又在自己昂贵的环境、国内环境下重新的投资产能。那所以然后又刚好它有美元霸权,所以它可以很方便的印钱。那这样的环境下美国所面临的问题就是通胀。



王孟源 44:19 

OK,但是中国刚好相反,中国的产业还是实体的,所以你失去了市场之后,问题是什么?你的产能过剩,你的员工就会开始失业。那所以我过去这几个月反复讲的痛苦指数是失业率加通胀率,是你中美脱钩是一种摧毁效率的摧毁,全球化是一种追求最高效率的一个手段,你要把它逆转过来去全球化就会牺牲效经济效率。那提升经济效率以后就有痛苦指数的代价,那因为美国是占领了金融的高地,而中国是有实体产业,所以美国的代价来自通胀,而中国来自失业。



王孟源 45:21 

OK,那在这种环境下,他们,我觉得这些心怀不轨的中国金融主管就以要挽回这些外资为理由对投资银行所来做开放,那事实上能够挽回的很少,因为真正的投资是在你这边开工厂的,对不对?你那种热钱,这种炒股炒地皮的,这个钱是它的负面效应要比正面大。那中国才刚刚在房地产业整顿传统的产业搞得灰头土脸,这个恒大的收尾非常的糟糕,你要引进这金融性热钱干什么?所以他这个是纯粹的欺骗,完全经不起推销,但是他们就是欺负这个最高领导人不是这个专业。所以我在我的博客反复的论证说,中西文化最根本的差别就在于中国有 2, 200 年的郡县制以来的中央集权的传统。欧美的这个西方文明只有 300 年的中央集权传统,而且还搞得不伦不类。这个传统你有 2, 000 多年传统,就会自然的累积一大堆配套的文化。我们现在所看到的中西文化的基本差异都是这样来的。为什么郡县制之后 100 年就会独尊儒术,罢黜百家?因为儒家思想就是最配一套,最适合配一套中央集权体制的。



王孟源 47:20 

然后我接下来你必须要选贤与能,经过了 1, 000 年的演进才变成科举。科举的意义在什么?在武则天时代才确定是科举,科举的意义是什么?你从以往封建,周朝的这个封建是贵族主导,变成郡县之后一千年,头一千年是士族主导,你看魏晋南北朝,这不是都是士族,对不对?



王孟源 47:54 

到科举之后才变成庶族,也就是它不断地democratize就是普及化,你的权力分散普及,你的社会变得越来越平等,能够参与的精英越来越多。这个是中国经过 2, 000 年的演化才有的世界最先进的一个体制,还有配套的文化跟传统。



王孟源 48:23 

也是在这种传统文化才会有中国的世人愿意以公益为上。这也是儒家演化了 2, 000 多年的一部分,我自己也是深受儒家教育的影响,才会以国家、民族还有全人类的利益嗯为最高的指引,对不对?你如果是资本主义市场那种想法,就是个人都争取自己的利益,市场会自然的解决、会优化。这种是一种功利思想,市场经济所引发的就是功利思想。中国在改革开放这 40 多年以来,功利思想就非常的重,比台湾还要重。OK,但是还是有少数人愿意遵循儒家的天下为公的那种理想,像是杨振宁先生啊。 8 年前我在讨论大对撞机的时候,就是我,还有杨振宁先生站在公益的这一边,然后另外有高能所那些人为了他们的集团,小集团的私利,那个这种功利的态度。但是你可以说这种中国的传统是,中央集权高效,然后又有贤能的世人愿意为国服务这个传统,但是他在这里也不是完美的,它有一个很大的点,就是他对整个专业,整个行业的腐化没有抵抗力。嗯,完全没有抵抗。OK,就是为什么呢?因为我刚刚提到像高能所那样子,整个行业有一个共同利益。没错,整个行业变成一个利益集团的时候,以这种中央集权的制度就容许这些行业的领头人利用行政权利来打压专业质疑,然后用专业权威来通过政治审查。



唐湘龙 50:48 

没错,就没有吹哨的人。



王孟源 50:51 

对,就没有吹哨的人。嗯,我们八年前我作为那个吹哨人非常幸运的有杨先生。



唐湘龙 50:59 

哈哈哈,所以你才能活到现在。



王孟源 51:03 

中国历史上第一次也是唯一一次,可能是最后一次,那这一次也是同样的,整个金融行业勾结起来欺瞒最高层,因为最高层不可能去管这些细节。他也不可能懂所有的细节。那你如果整个金融行业合起来去欺瞒他的话,可以在他定调说我们要专心实业,不能够让经济虚拟化、金融化的时候反过来说,我们就开放投行,让美国的这些金融巨鳄近来我们国内玩这些花样,OK,明明是反其道而行,但是他们就可以硬是说要这样做。这是非常的不幸啊。



王孟源 52:02 

他们可能用的另外一个借口是鲶鱼效应,就是模仿特斯拉。就是因为它可以说,我们中国目前国内的这些投行或者是银行效率很低。



唐湘龙 52:20 

这可能是最好的理由了。



王孟源 52:22 

OK,但是问题是金融跟电车完全是两回事。特斯拉做的是实体产业。这个实体产业越先进越复杂越好;金融这刚好相反,美国金融业在过去 50 年对他们自己国家搞的是什么?是把他们国家的工业挖空了,他们的那些先进的知识跟技术就是挖刨自己的根,来图利这些金融巨鳄的伎俩。你积极的引进这些来逼你国内的这些所谓的低效的银行业、低效的金融业来做鲶鱼是什么意思?所谓在金融业所谓的高效,就是吸血的高效,你越高效吸血越快。那你这个鲶鱼效应根本也是一样的,经不起推敲。



王孟源 52:30

最后一个他们可能用的借口是你进来我们也不怕,因为我们在国内,我们的政治我们说了算,反正也是要我们来打你。这个也同样的经不起推销。你对高盛,你对黑石这些机构能够拿捏的比对恒大拿捏的好吗,对不对?结果你恒大拿捏的怎么样?我一直觉得你做这种论证不能够只给结论。你要你如果说一件事情不好,一个人不好,你要给出例证,因为如果他真的不好的话,应该很容易的又举出实例,他们为什么不好,对不对啊?



王孟源 54:25 

但是市场化的媒体还有舆论,他们的讨论就常常是很空洞的,就好像美国的小学生一样,没有写作业的时候,你这个作业要写报告,比如说读书报告没有读的时候怎么办?就上台说我很喜欢这本书,我给他 10 分,那这些是空口白话,主观的意见。就是基本上是因为前一天玩Nintendo玩的太晚,所以根本没有做写报告,没有读书报告,但是你看,现在不止台湾这样子,美国也是这样子,大陆也是这样子,他们做评论的时候基本上就是空口白话。就跟美国这些没有做作业的小学生一样。那我现在既然说这个放这些美国投行进来会是一个灾难,我跟你解释一下事实上怎么样的?事实上有成千上万的漏洞。他们会吸血,然后把你的财富转移进去,我只选我在博客过去十年讨论过的三种。



王孟源 55:47 

第一个就是恒大,恒大是怎么把财产转移出去的?恒大转移财产的手段就是,恒大的操作模式,就是它有官商勾结,所以它可以跟官方的银行借钱,借很多钱。那他把这个转移出去,就是先在美国设立一个公司,然后由这个公司来跟国内借钱,国内的他们官商勾结的那些官方银行借钱。当三年前他们开始追查恒大的时候,追到最后这一笔钱就追不回来了,为什么?因为他在美国就宣布破产,以后他们的资产就在美国破产法庭的保护之下,完全不在中国的政治掌控之下,明明是中国官方的公款,借出去以后拿不回来了,因为它是美国企业。然后他宣布破产,你还跟真的就拿他一点办法都没有。然后这个,他为什么会破产呢?这些资产是到了人家公司以后,再转到许家印的老婆跟儿子的名下,这时候那一家公司已经变成一个空壳了,在美国的空壳。你说你能够拿捏的住,你连这个都拿捏不住嘛。



王孟源 57:23 

整个他们这些处理,专案处理恒大处理了三年半,结果就是他们利用这种破产的手法金蝉脱壳。那你如果是让美国的投行进来,岂不是这一条龙服务了?他们直接在美国帮你开空头公司什么的,对不对?然后再安排你跟中国的银行借钱。这是最低级的手段,你都拿不住,你还要对美国开放,这只能是方便。



王孟源 57:53 

第二个案例就是希腊,希腊在 2004 年的时候为了要进入欧盟,请高盛来帮他做账,高盛就用了很多很复杂的 Derivative 衍生金融品,把他的账目做到符合进入欧盟的程度,就是它的赤字。原本是 10\%GDP 的赤字做成少于 3\%,然后这样就进了欧盟,然后就爆发掉了。



王孟源 58:29 

那。这就是对应的我刚刚讲过的美国的这些先进的金融技术鲶鱼效应,他们先进在哪里?先进在做账,先进在转移资金,先进在逃税。你把它,你担心自己国内的金融机构不够贼,还要把美国的这些专家引进来当鲶鱼。这不是?唉,我真的是不知道该怎么讲。那你明明有希腊的这个前科,你引进来的就是高盛这些人。你还要引进他们这些。



王孟源 59:16 

然后最后一个,我也在你的节目谈过,就是英国脱欧。我反复的谈了好几次。英国脱欧为什么会在 2016 年急着公投脱欧?因为在 2016 年的年初,欧盟通过了一个反避税法案。这个反避税法案就是针对着美国投行很喜欢搞的一个手段,他们专门替什么样的公司,就是 Google 跟 apple 这种美国公司来搞啊。我简单的再解释一下,其实过去这八九年我不知道解释几次。他就是比如说 apple 现在在欧洲每年卖几亿的iPhone,嗯,每年赚几千亿的利润,但是它在欧洲根本不缴税,嗯,欧洲的税务局还要倒贴,为什么?因为他报账的时候说我们是赔钱的。没错,你说那么明显那么大的利润怎么会赔钱呢?因为他 apple 在完全没有企业税的地方成立了一个空壳公司,这个空壳公司然后把他的所谓的知识产权转移到这个空壳公司去。所以他在欧盟所卖的那些iPhone,一台卖 1, 500 块美元,利润是 1, 000 块美元,成本是 500 块,零售跟在中国制造。但是他那 1, 000 块的利润一下就不见,为什么他还要付 1, 000 块?他的知识产权授权费是给在加勒比海的一家空壳公司。



唐湘龙 01:01:21 

那都是他的,都他自己。



王孟源 01:01:23 

这样一来,他就在当地完全没有利润,也就不需要交税了。OK,欧盟就是因为要立法来追查这东西行动,所以英国的财团因为英国本身投行就参与了这个操作,所以他们才操弄公投要急着脱欧。那你想想看这个如果高盛进入中国之后会不会搞?会不会提供这种一条龙服务?一,中国的金融监管单位能够拦得住吗?你,中国的财团或者是巨富要把几亿、几百亿、几千亿,往外转移的时候你能够拦得住吗?他在加勒比海弄个空壳公司,然后就说怎么知识产权的关系,所以这个你每年要付几百亿、几千亿的知识产权费,你能够说他人不对吗?你连恒大那些人都搞不过你,你要对抗这些手段,我真的是,这还不是最糟糕的。



王孟源 01:02:46 

这还不是最糟糕,这只不过是失血。就是你相比几年前中美还没有脱钩之前,那时候美国还为你那些饲料,然后隔十年来宰割来杀猪一次。你现在技术跟市场绝对拿不回来。资金只能够拿一点热钱,为了这么一点点的收入,你要让他全面的宰割,那真的是愚不可及。但这还只是战术层面的错误,你在战略层面的话,你的整个经济结构就扭转了。因为你只要赚了一点钱,那些巨富就会把钱往外移,对不对?你这些投行他们最擅长的就是逃税,就是资产转移。那你这样子一来,谁还会想要继续在国内投资,嗯,对不对?第二个就是扭转人才配置,你这些利润,这些资产转移都是必须要付出百分之二十、三十四十给这些投行。那这样一来,没这样,是这样子,所以美国的投行才能够随便一个中下级的人员都可以百万年薪。



王孟源 01:04:25 

OK,我的小孩还算蛮有出息的,他现在是毕业,刚刚经济系毕业,拿到美国经济系毕业生最想要的最高级的职务。一年只有三个人,一年只有三个人,你可以说美国顶尖的经济系出身的顶尖的前三名了。OK。但是他一天到晚都跟我讲着他在想要转金融。为什么?因为他做金融的那些同学明明程度比他差很多,但是进了这些投行或者是对冲基金,第一年的薪水就比他高三倍,五年之后的薪水会比他高 10 倍。



唐湘龙 01:05:15 

好,这个这段如果有听到人大概就会非常震撼,就会觉得,哇,这个如果进来的话,那不得了,它基本上会导致人才的流动的方向的改变了。那使得就是说你作为实体经济的中国。



王孟源 01:05:29 

他跟我说,美国在做工程,有数学、应用数学,还有计算机工程,尤其是计算机工程的,基本上最顶尖的学生都进了对冲基金。OK,那你说中国是不是准备也要这样子?你这个这种薪水所得的差异,以数量级的差异。你一定会扭曲人才配置,一定会让实体经济越来越走不下去。你第一个我刚刚讲的那些资金的资产的转移会,你原本口头上说是可以得到一些热钱,可实际上是会方便资金外流。然后你还有这个人才的扭曲,然后这是长期的,未来 20 年到 30 年的一个逐渐腐化的影响,而且它是一种润物细无声,就因为它是一种像是那个 entroby (熵)增加,就是因为你一旦放手之后,就你如果手里捧着一点水,你一旦放了以后就捡在地上捡不起来,那些水就捡不起来了,因为这是不可逆的。



王孟源 01:07:01 

然后更糟糕的是目前俄国所挑起的反殖民斗争,俄国已经成功去美元化了。结果中国却把美国的投银放出来,这里的一个附带作用就是中国会对美国绑定,因为这些投银带进来的就是廉价的热钱。当美国不需要那些热钱的时候,这些热钱很廉价;当美国在紧缩银根在宰猪的时候,你受的你被宰的更严重。也就是你留在美元体系里面的长期恶果。



唐湘龙 01:07:49 

好,因为这个,因为我,我不是这方面的专家,我纯粹在新闻工作上面的一点的,作为皮毛的观察,可是因为王孟源是在西方的金融体系里面工作过非常久的,由于尤其在瑞士银行里面,所以你娴熟于西方的金融体系的思考跟操作的方式,以及它这种就是说润物细无声的。对于这种就是说第三是世界国家的所有的这种的渗透以及渗透之后的那种无声无息当中的收割。我们在过去对于所谓的西方的投资银行,高盛、 JP摩根这些非常有历史背景的投资银行,它通常它大概会有三个的三个,三个主要的影响一个,一个是就对产业资金的引导,这个是在我们念的基础经济学术常提到的就是把资金引导到最有希望获利最高的,就是说配置最好的位置上面。



唐湘龙 01:08:47 

产业引导第二个就是它的一个主要的获利的来源,就是努力的去创造企业购并的可能性。第三个是衍生性商品,这个就很麻烦了。那你看的时候你刚提到习近平在年前的讲话,对于西方的金融体系里面,尤其是衍生性商品本身所藏的那些的非常的邪恶腐败的,以及对一般人无从分辨的投资风险。中国大陆你从习近平讲话里面,你也不能说他没有警觉性。可是当你把投资银行全资开放进来的时候,这些投资银行当中主要赚钱的在一个去吸老百姓血的这些所谓的衍生性商品,它一定会进来,会泛滥成尘灾。我不知道大陆怎么想,他也许会觉得说这个事希望能够加速中国的金融的现代化跟国际化,因为这方面确实是大陆金融体系里面作为高端服务业当中来讲看起来表现比较弱势的一块。可是要如何去避免了孟源刚提到的这种的西方的投资银行进来了之后,一定会同时伴随的一个是帮你做假账,那转移了资产。



唐湘龙 01:10:04 

第二个就是除了做假账之外,逃税这些都是中国大陆其实他强调自己社会主义的这种的经济模式当中非常重视的部分。好,那我问一个问题就是说这种的开放跟最近大陆股市的大涨有关联吗?



王孟源 01:10:29 

热钱当然可以炒。你股市其实最受影响的有两点,第一个是你的货币宽松,是不是有热钱?嗯,这个热钱可能来自你的中央银行,也可能来自外资。嗯这是最主要影响,最次要的影响是你是不是有好的监管?那些诈骗的集团,是不是能够把他抓起来?这些害虫,但是问题是监管利用加强监管来增加投资的报酬率,是事倍功半。但是它是有益于实体经济的,那这个钱其实只是收割韭菜,那是对实体经济有害的。所以,但是偏偏这个热情的影响是主导性的,你监管的那个影响是很小的,所以我从来不会用股市的涨跌来判断一个经济的走向,一个好的国家管理的好。股市好,其实是你监管的好,但是它对股市的涨跌影响是很次要的。



王孟源 01:11:50

股市涨的时候,这里有一个案例,就是 1990 年代,大家都知道日本现在已经是不只是失去了十年了,已经失去了三十几年了。对,他真正出问题一开始的时候当然是因为他被美国多方面的打爆。就是在 80 年代中后期的时候,它的货币政策被美国强制的变成。不明智的是在一年之内对美元升级升值了100\%,与此同时这造成了他们国内的一个非常严重的资产泡沫,与此同时美国又系统性的打击它的外销产业,包括半导体跟汽车,那汽车它的生产后来就转移到美国去了,然后它的半导体往转移到韩国跟台湾去了。那然后他的那个资产泡沫在 1990 年爆掉以后就有非常严重性,严重的就是百年一次的那种的灾难。但是它之所以为什么这个灾难会持续了三十几年还没有恢复过来呢?不是因为那个灾难真的那么严重,而是灾难持续了三五年之后,美国人哄着日本人说你们这个灾难之(所以)来临,而且很难恢复,原因是因为你们不够开放,所以在 1995 年、 96 年那个时候,日本采纳了所谓的 big bang,就是大爆炸式的金融改革。那这个big bang 是怎么样呢?就是对美国投银完全撤防。就是刚好中国所做的这件事情。他们被美国人忽悠了,因为美国人,美国投银在 90 年代后期,我为什么知道?因为那个时候我已经在投银工作了,我还被派到东京去,去了几趟。就是去那边掠夺他们的资金呐。我自己参与过的事情,所以中国是莫名其妙,现在美国对中国的 敌意 比当年对日本还要高。你居然一点代价都没有就主动的开放,噢,这真的是愚不可及。那我过去这两个礼拜其实没有怎么照顾博客,因为我真的心情很不好。我写博客十年了,只有这是第二次心情这么不好,失望到极点。第一次是 9 年前, 2015 年的时候马英九快要卸任了。眼看着民进党要上来了,那个时候我说你这个国民党跟民进党这样换来换去,其实都不是重点,真正的重点是从李登辉开始的去理性化、愚民化。嗯,那个是公共论坛上的讨论的智商点数一路下降。这个才是对台湾长久的最长久的伤害。



王孟源 01:15:32 

所以我说你,马英九,你至少要赶快在卸任之前把教改,就是陈水扁所推行的那些教改扭转过来。你说扭转过来,下一任的民进党总统不是可以再重来吗?其实这个美国也有经验,美国这边的这些白左的这些教育,美国的教育界大部分人是白左的,所以为了抵抗白左,他们就用我刚刚讲的那种entropy,就是我刚刚说你有些手段是不可逆的。就是放权往往都是不可逆的。你这个水掉到地板上就捡不起来的。那美国人,美国的这些保守派的家长,他们对抗这些白左的手段就是要求把教育放权,那这个,你这个教育一旦是由家长来审查的时候,你就没办法。那个白左的影响力就受到抵消。所以当初马英九其实是可以把这个教材改过来,改成比较理性、真实可靠的描述,然后把它交给每一个学校的家长会来决定。那这个一放权之后,下一任总统想要再扭转你都来不及了。你明白,话说因为你不可能说再从家长会的手里把这个选择教材的权利拿回来。所以你至少东北部,台湾的东北部不会有那种深绿的那种很蠢的人,很蠢的仇中的说法。所以我这一次看到这个我的失望,是跟 9 年前对台湾的失望一样。已经不能用失望来描述,就是绝望。我在 9 年前就已经论断台湾已经没有希望了,同样的,我这一次也是可以断定中国没有希望了。我把这个讲的再清楚一点,过去这两三年中国很流行一句话说百年未有之大变局。百年,过去这 100 年的历史主流是什么?是美国霸权。百年未有的大变局其实就是一个间接隐约的说法,说霸权要更替了。



王孟源 01:18:11 

那你高兴看的这个百年变局的就是国家主义的national。如果是爱国的,你就会很高兴看到你不再成为美国的附庸。但是我即使在过去的十年提过好多次,而且现在华语的论坛也都有,我看到有其他的人学到,就是这其实是一个 五百年来未有的大变局。过去 五百年是什么?白种人主导的殖民帝国?没错,所以这一次的我们所募面临的关头也是推翻白人殖民帝国,种族主义。一个关键战役最佳机会。那你任何一个相信人本主义、反对种族主义的人都应该很高兴看到这个变局,但是其实还有第三重,以我以前没有公开的在博客上讨论过,这第三重意义就是过去 200 年未有的一个变局,过去 200 年是什么东西主导世界历史潮流?金融资本主义?现代的以金融为主导的资本主义市场万能思想。又是从 19 世纪开始的。那个时候开始有真正的国际银行 Rothschild,然后一直到现在,我们 19 世纪针对这个金融资本主义的反动,就是现代社会主义,所以真正相信社会主义的人会很高兴看到这个变局。中国已非常不智的,非常愚蠢的。对美国金融资本巨头的投降开放,对美国霸权来说还是一个次要的助力,因为中国已经有了足够上升的动量,我认为美国自己本身被他们这些金融巨头腐蚀的更严重,所以这只有次要的阻碍;对五百年的这个殖民帝国的崩溃影响比较大,因为它这代表了中国摆脱美元体系的可能性降低了。而你要摆脱打垮这个殖民帝国,必须要摆脱美元体系。真正最糟糕的是我刚刚提到的,这会腐蚀嗯,中国国内的经济,一方面让他们的资金有一个方便的外逃通路,另外一个会产生人才的错配,会让虚拟化经济吸收更多的资源,从而阻挠了社会主义的下一步的普及。



王孟源 01:21:48 

我认为这才是这一次这个措施最大最大的失误,而且是最坏最快的影响。我已经快要60 岁了。我也贡献了 10 年给世界人类的公益。我失败了啊。我一直梦想着过去两百年无数的社会主义者所梦想的社会主义的下一步能够在我的帮助下,帮助推动下完成,但是我现在确认自己失败了,这个要等未来三五十年下一代或者下下一代的社会主义理想主义者来推动,很不幸的中国没有把握这次一个天大的良机,来为社会主义、为人类、为国际社会来做一次革命性的贡献。



唐湘龙 01:22:57 

好。孟源讲这段话很沉重,我跟孟源在沟通题目的时候,因为这个题目是孟源的专业,而且孟源是有很长时间实物经验的,他也一直在关注大陆的,就是说金融体系、金融服务业这些虚拟经济在面对到西方的这些的挑衅或者勾引的时候所采取的态度,但是门打开了之后,是不是一定就是清兵入关式的这种无所不能的破坏。但它后续在监管操作上面来讲比较被动,消极的效果或许有一些。

但是以这种西方的金融巨鳄,对于第三世界国家的这种破坏或者是掠夺,这个是孟源在过去节目当中一再的提醒的,所以他寄希望于中国的这种体系,中国过去的金融防火墙设的非常的高,非常的延时。表面上看起来它似乎妨碍了中国的金融业的发展,不过这种对于投资银行这个时候的开放,我比较好奇的是为什么就是这个时间开放?去年的旧金山的习拜会有没有关系?以及在过去一年里面我看到中美之间的沟通的交流,从去年6月 18 号布林肯访问北京之后,重新的再回到正轨之后,其实美国方面最积极的和中国对接的是财政部长叶伦。耶伦在金融面的着力跟这些的开放有没有关联,以及在这个时候的开放是不是意味着中国对于年底的美国总统大选的某种的表态?这个是政治面当中我们会去持续观察的部分。



王孟源 01:24:51 

不是不是,这件事情习近平已经定调了,要以实体经济发展为主,这是正确的方向。他定了正确的大方向、大原则。结果手底下的人来一个一手遮天。



唐湘龙 01:25:07 

五鬼搬运,这个这可能是大陆内部可能要特别留意的部分,这也是孟源的最近的几集,我们只要谈到国际金融的时候,孟源都非常忧心忡忡的,在提醒的好,因为时间的关系,今天跟孟源只能够谈到这儿,其他的我建议大家反复听之后去咀嚼孟源在第二段当中,到最后这将近一个小时的谈话的内容。



唐湘龙 01:25:40 

那我感谢几位的观众朋友们第一个来 brought K001,感谢再来。因无因,大陆长期对台工作就是不折不扣的全面失败,持续会台几十年了,那会出一代又一代的,会出一群又一群的高国民一等的把大陆人称为制纳制台的台湾人,那搞得现在伟大不掉,如何如何?对了,这个我也不能说反对,但是它是一个在政治气氛当中非常非常复杂,同时也是非常负面的议题。



唐湘龙 01:26:17 

好,校长有感谢,然后烟米王感谢,然后这是茉莉茉茉茉花茉花茶浴。作为一个学习、生活、生化工程出身的 50 岁的博士生,下个月我就要回中国工作和生活了。恭喜你,他说感谢龙哥把王博士这样的大财介绍给了我们观众。那也非常的感叹为何同为博士水平有如此的云泥之别。他肯定王孟源,这个我也会,我也有一样的感受。好,那王博士的博客,他说他在反复的学习当中如同一个宝矿,蕴藏着王博士和其他博友们的这个闪闪光智慧。最后那因为他常年潜水,第一次公开的发言跟抖内那欢迎龙哥来聊天,好感,感谢您再来范志玲,感谢他说渴望聆听王博士的精辟分析,感谢香龙的邀请,并祝两位龙年大吉。感谢你好再来这次 Lang green,感谢,然后开心让我们的资深的听众,他说向荣老师,孟云老师好,那他说我是 100\% 同意孟云老师讲的。那他想请教孟源老师,其实中国大陆好好的利用香港来经营金融业,而中国大陆应该专注在社会主义的实体经济,会不会是更好的方法?孟源,你能够回答这个问题。



王孟源 01:27:58 

刚好两三天前有一个新闻,就是国际上最大的一家国际律师事务所,总部在伦敦的,他把香港的分布降级了,事前把它的联网断掉,就是割断了。那这代表的是什么呢?就是昂萨集团,昂萨殖民帝国在扼杀中国的过程中无所不用其极,连香港他们都愿意舍弃。当然那这个是在这种环境之下,你中国自己主动开放,而且还不是香港,而是自己的内地,已经上海,我真的是只有一句愚不可及。嗯这真的是...在最高领导人已经有正确认知跟指导的背景下,他们还在偷天换月。真的是国家跟民族跟全世界人类的罪人。



唐湘龙 01:29:03 

好,孟源刚后面讲的这一段,虽然过去几次的我们在每个月我在访问孟源的时候,孟源只要谈到了,谈到国际金融,几乎都对大陆当下的金融政策颇有维持,也侃切的提出建议。固然会有一些一些户中的网友们对于对孟源的这些质疑表现出政治上的不以为然,不过我还是呼吁大家,当我们这个公共平台当中发言跟讨论的时候,我觉得认真的聆听呢?聆听发言者本身的关切的重点,这是至关重要的。好时间的关系,今天我也好好的上了一课,那感谢人在美东,尤其现在,其实在美东的时间,现在已经是深夜了,天寒地冻的时候,孟源打开了供暖,提供了一一个多小时非常饱足的知识能量的课程。感谢孟源。好,下个月见,新年快乐。OK,新的愉快快乐,下个月见,拜拜。谢谢,谢谢孟爷,也谢谢所有的观众,下个星期的龙行天下见拜。



\twocolumn[\begin{@twocolumnfalse}
\section{美國大選年:歐洲往哪裡去?}
\subsection{20240322}
\end{@twocolumnfalse}]Credit:Anonymous、网友 S



王孟源 00:01 

…但是除此之外,身心健康都还很好。其实我自从五年前开始减肥之后,身体健康其实是往上升的。



唐湘龙 00:16 

我最近也这样。我只要当天晚上事情想多了,睡眠品质多少就会受影响。好,我们准备。



欢迎到龙行天下,我是唐湘龙。今天星期五的时间, 9 点半的时间,一个小时原汁原味的王孟源。人在美东,跟我这边时差 13 个小时刚好相反。王孟源在克服日夜颠倒的情况下,上线跟大家分享他的知识经济。在我们线上的王孟源,孟源欢迎。



王孟源 01:02 

大家好,很高兴再跟大家聊聊天。



唐湘龙 01:04 

好,那今天我们原则上的两大块的主题,第一大块就在我们标题上面。今年当然全世界都在关注美国的总统大选,那王孟源是怎么看的?那待会请教王孟源。因为全世界都在准备大国政治的调整,以及对于美国选举的结果,都开始要买保险,所以也都在做一些相应的准备。因为这两个人——拜登跟川普,民主党跟共和党现在都在极端化的情况,就是美国的政策差异性,落差非常大。



第二个部分是很专业的问题,但是过去每一次孟源只要谈到的时候,我认为都会引起非常大的回响——就是有关于金融和实体经济之间的关系。大部分人对于金融业、金融面,都会把它当做是实体经济的一部分,但并不是。有很多的经济学家认为金融的所有的产值不应该计算在经济里头,它基本上面都是过路财神,在中间不创造什么东西。



这两大块的主题其实都很重要。来,今天我们从哪里开始?



王孟源 02:18 

我先借你的平台做几个有关我个人的宣布。



在我的博客里面,我会说这是一个 Metalevel 的问题。 Meta 这个字最近很流行,就是因为有一个公司改名叫做 Meta。一般人可能不知道 Meta 它真正的意思是什么。Meta 它原本的意思是自我参照 Self-referential——就是我谈我自己,后来它变成提升一个观点层次的意思,其实这也很自然。比如说我们现在在讨论欧洲的政坛——我们可以谈得很高兴,但是我们如果退一步,然后把观点提高一下,这个事实就不再是...讨论的内容就不再是欧洲的政坛,而是两个人、两个男人在讨论欧洲的政坛,对不对?那这样就叫做提升一个 Metalevel。所以原本是把讨论者自己也加进来变成讨论的对象,这个叫做 Meta。



因为近代科学的起源集中在物理上面,当时就有人搞起 Metaphysics ——就是提升一个 level。可问题是 Metaphysics 这个东西——我们一直到现在都还没有把物理真正理解清楚,所以你可以想象一两百年之前,那个所谓的要超越物理的那个讨论基本上就变成玄学,所以 Metaphysics 现在就翻成玄学。当然 100 年前日本人把它翻成形而上学。



顺便跟大家谈一下...因为我在博客里面写文章,常常会讲说,提升一个 Metalevel 来看。可能有些读者不太习惯这是什么意思。今天我们原本是要讨论欧洲跟金融,但是在讨论那个之前,我们先提升一个 Metalevel 来讨论我自己,还有我作为一个评论员的一些新闻。



第一个, 我原本一直到去年还觉得说,我对中国政策的谏言,经过这几年对外交跟大政略方面的讨论,基本我的推荐的方向已经都被采纳了。这当然并不是我能够确定说,有人看到我的文章,然后说这很好,我们要采纳;而是不管他们的来源是什么,同样的 idea,同样的那个理念被采纳了。



另外一个方向就是我一直批评的,他们的科技管理非常腐烂,这个体现出来就是 20 年的半导体工业的投资,结果统统打了水漂,基本上过去两三年是重启。尤其是最近十年有一个半导体大基金,这个现在已经没人谈了,原因是两年前那里面的高管统统被逮捕了几十个...搞成这样子灰头土脸的事情。所以它就变成我批评的一个重心。



那我反省的结果是……这里要先谈一个经验,就是在 2017 年的时候,那时候我已经注意到中国科研界跟学术界的假大空现象非常的严重。首先 2016 年,我批评了他们要建大对撞机的事,那刚好是我的本行——我原本是做高能物理出身的。后来到 2017 年的时候,有一个所谓的悟空卫星——这个是天文学跟高能物理合作的一个实验。他们拿到初步资料以后就说,他们发现了一个信号。这个信号是诺贝尔奖级的信号——就是突破了既有的理论,必须要推出新的理论才能够解释的。当时我就跟他们讲说,这其实在高能物理——至少我在念的哈佛的高能物理——一看就知道是统计噪音。然后他们就说这个怎么可能是统计噪音呢?我们用统计常用的方程式来估算它的误差的话,这个不可能是噪音。我说你们用的那个方程式有几个基本的假设,在高能物理里面是不成立的。



我记得我在学(统计误差)这件事的时候,哈佛的那个教授跟我讲说,你们如果是要念大学本科的话,根本不用来学校,自己找几本教科书到图书馆念就可以了。你们来念我们的研究所,就是要懂得超越教科书的东西——就是教科书上面写的公式是这样子,但是你的教授会私下跟你讲这个公式的局限在哪里。这就是一个案例。



当时我一开始还觉得很奇怪:怎么会...中国最顶尖的、投资很大,然后一两个团队——一个天文团队、一个高能团队,合起来牵扯了几百个人,然后这几百个人又有几千个学生,他们统统都觉得自己准备要得诺贝尔奖。那王孟源忽然跳出来说,你们这个是庸人自扰,这纯粹一看就知道是统计噪音。所以那个时候我在知乎上就面对着一千个行内的人——就是大约真的就是一千个人,我希望那个对话现在还在——就是那时候 1000 个人骂我王孟源胡说八道。我说我跟你们讲理你们不听,反正这个实验再过一年就会总结,到时候就会有新的数据出来。如果是统计噪音的话,这个新的数据就会让这个信号消失,结果我们就等了半年多真的就消失了,然后我就跟这些人结仇了。一直到现在我的博客还有很多很忠实的读者,你可以看到他们把 1000 多万字从头读到尾,而且读了好几遍。



唐湘龙 09:41 

但也不错啊。你要我读我还真的读都读不下来啊。



王孟源 09:46 

对,一般人读的很累。你如果是真的觉得我讲的是胡说八道的话,怎么可能花几个月的时间把 1000 多万字读透。然后我为什么知道这些人会这个样子呢?是他们在骂我的时候引用的是非常隐晦、非常小的一些我几年前写的一点东西——我自己都快忘记了,哈哈...很显然是这些人读得很用心。可是他一定要绞尽脑汁来骂我,这很显然是有仇——这不是普通的看不爽,这是源头。



很不幸的是,我这几年说实话说的太多了,然后把中国的一些利益集团得罪了,把台湾的一些利益集团得罪了,把美国的一些人也得罪了,所以我现在必须要很低调。我其实前几个礼拜才跟我的一个朋友私下讲,他跟我讲到一些...我最近太过高调的问题。我跟他讲其实不是,而是因为我 2017 年的那一次经验是——经过我高调的批评过悟空卫星之后,凡是提到那个计划的事情都没有人敢反驳,因为我是在事先就斩钉截铁地告诉大家。



然后与此做对比的是 2016 年、 2017 年的时候,那个时候量子通讯也是搞得如火如荼,然后在中文的媒体上大吹大擂,当时我就觉得因为还有另外一位良心人也在那边骂,我想说让他骂就好了,反正我已经为了这个大对撞击跟悟空卫星弄得满头包,就没有挺身而出。结果几年之后这个量子通讯这个潘建伟的骗局的确是泡沫爆了。他卖给中国军方的那一套量子通讯系统基本很快地就被弃置,因为没有用不能用。我如果那个时候高调地骂过,我想我现在再批评潘建伟,我的底气就很足。我可以说 7 年前已经对峙过了,结果事后证明是这样的。但是因为我那个时候把它让给另外一个人,所以我也没有想到潘建伟会这么无耻,他那个公司在股票股市上割韭菜割了大概几亿,然后只做了一票生意卖给中国军方。然后就没有后续生意了,他就把那个公司转去做量子计算了,它现在变成量子计算的大头,然后这一次在股票上、股市上面去割几十亿。其实像这种做实验的,隔行如隔山。你如果一开始是做量子通讯的话——量子通讯跟量子计算虽然看起来好像都是量子,它其实是不一样的东西——你一个一辈子做量子通讯的人,一下子就转去变成量子计算的大头,这很明显地就是在骗人。



可是因为我上一次没有抓住他的尾巴,我没有高调地批评他,所以我过去这几年的痛苦,就是因为你要解释这些专业的事情给外行人的时候,无论你讲的如何地逻辑严谨、证据确凿,大家还是会迷信权威的。你真正能够打破权威的就是事先跟他们对赌,然后一翻两瞪眼,事实上就证明我比他们讲的对,那么就可以有更好的公信力。所以经过这次的经验以后,我了解我必须要高调。那高调的话就会得罪人。那要得罪人的话,我就想今天利用你这个平台跟大家讲一下:我身体健康,绝对没有任何自杀倾向。我再说一次:我没有任何自杀倾向。



唐湘龙 14:40 

不会,没有。有我跟王孟源吃饭的时候他开心的样子,他不会的。



王孟源 14:48 

这是因为波音当初第一个吹哨的那个工程师最近自杀了,他听说是因为……我们根据 Occam's razor ——就是用最简单的解释来解释的话,目前没有证据显示是被人暗杀的。但是有很多证据证明波音收买了很多公关、文字写手来骂他,用律师来起诉他,骚扰他。所以他的确是被骚扰,很有可能是被骚扰到情绪不好而自杀的。



唐湘龙 15:28 

对,抗压性不够了。



王孟源 15:32 

我自己也已经是好几次的吹哨人了,所以利用这个机会跟大家讲一下,我绝对没有自杀倾向。所以我...



唐湘龙 15:40 

打个岔,你现在有感觉到这种就是说,系统性的操作在针对王孟源进行攻击,甚至有生命威胁吗?



王孟源 15:55 

我不敢说会有人想要杀我,但是绝对有人想要打我。所以你看我连我在台湾的老家是在哪一个镇上我都不敢公开讲,因为我也骂过绿营,绿卫兵的脾气也不是特别好的。



唐湘龙 16:17 

哈哈哈,真的,不过……红卫兵跟绿卫兵都得罪。



王孟源 16:26 

我真正挡人家财路挡的最多的是在中国,所以我不太敢到北京去,因为挡财路档得最多的就是北京的一票学术跟管理的人。



这两件事谈完以后,我还要再提一下,我过去这两年多经常谈在你这个节目上谈的一个话题是俄乌战争。对,那我感觉上是一年半前,就是 2022 年的 10 月,我曾经在你的节目上说这个,那时候刚刚乌克兰大反攻。抢回了 Kharkov。那时候我在你的节目上做了一个论断,我不晓得你记不记得。我说这是乌克兰的高水位线,英文是 High watermark。在当时我想至少在中文论坛界,只有我一个人敢这样说的,能这样说的。但是现在一年半之后,理解到战争真相的人很多,而且有很多是有军事背景的,有很多其他的博主自愿地去收集第一手的消息。所以我再来重复这些战术细节没有太大的意义。大家如果对军事有兴趣的话,可以去看这些军事博主,比如说台湾就有邱世卿,你们去看。然后我在 YouTube 上也看到一个不太有名的博主,叫做什么鹏城杰森——他也是总结俄文的第一手消息,总结的不错。



大家有兴趣可以去看,但是我觉得我没有必要再花时间讲这件事情。为什么没有必要呢?因为我现在刚刚发现一个很大的失望,就是我 10 分钟前提到我原本在外交跟政略上的讨论,我以为我认为大功告成,然后在科技管理上面目前还没有什么成果,但是也开始能够说服一些人。我原本没有想到的是在金融管理上面,也会出大问题。为什么?



这三个议题其实是连在一起的。为什么呢?因为中国目前面临的问题其实是霸权交替的问题,就是大约从 2009 年开始——就是上一次金融危机之后——美国开始衰落。那么这之后有一个大约 30 年的霸权交替过程,我们现在刚好过了一半,就是过了 15 年。在这个过程中,美国他的惯例就是,他过去这 200 多年对外用兵颠覆 400 多次,但是从来没有一次是主动单独对同一级别的强权挑起战争。你看他对俄国的手段,也是拱拱乌克兰去当一线的炮灰,然后拱欧盟去当第二线的经贸炮灰。所以对中国的打压它也是从同样的从金融经济的层面来打。从金融经济的层面来打,他就是扼杀你经济发展、工业化产业升级所需要的三要素。这个也是在我博客讨论了好几年的,经济发展的三要素是什么?市场、技术跟资金,对不对?当年日本复苏的时候是从美国那边拿了市场、技术跟资金发展出来的。然后来美国看日本不爽——在 80 年代看日本不爽了,把这个转移到台湾跟南韩,对不对?80、90 年代就变成台湾跟南韩崛起。然后到了 2000 年之后,中国变成一个更廉价、更高效的生产基地,刚好美国这个外包风潮风起云涌,所以就有一个全球化。所谓的全球化是什么?就是美国把市场、技术跟资金拿来为这些东亚的新进工业化国家,然后再定期(大约每 10 年)用金融手段来收割一次。



其实我的博客常常被大陆来的读者骂说我太过激进,没有所谓的中国的传统智慧。可是事实上我只是在战术上面要求比较主动积极——这是因为你用逻辑理性去推论,很自然地就会说你这个战术反应必须要是灵活主动的,才会最优化你的利益。但是在战略上我一直所坚持的其实是非常保守、非常内敛的一个战略。就是我认为中国应该是练自己的内功,专注在自己的产业升级上面,专注在自己的经济发展上面,专注在自己的社会文化上面。



你如果要专注在自己的经济发展,而同时美国从外面来扼杀你的经济,拿走他原本给你的市场、技术跟资金的话,你的对应方案是什么?第一个寻找替代市场。这就是为什么我的博客一直在讲这个政略上面要怎么跟美国在外交层面去搏斗。我也刚刚讲过我所建议的那些策略基本上都已经实现了,或多或少的都已经实现,第二方面就是技术。你如果不能跟美国拿——而且事实上中国的技术也已经发展到没有多少可以被美国卡脖子的方面了——这时候你的关键就在于你自己的研发、科技研发的效率。那科技研发的效率,它的关键就在于科技管理。所以这是我博客过去几年第二个常谈的重点。然后我刚刚也提过,我认为中国做的非常差劲,事实上可以打零分啊。然后第三个就是资金的问题。美国因为切断了市场,所以美国跟欧洲的厂家自然就不可能在中国继续大幅投资,反而是要撤资。那这个时候中国要解决融资的问题,就必须要有自己健全的金融体系。



我不知道你还有没有注意到,我上个月把中国的金融管理阶层臭骂了一顿。因为他们迷信美国商学院所教的那一套,讲说高等金融如何的有利。高等金融所优化的是什么?是资本的利润回报率。但是你如果不是看它的平均利润回报率,而是看 20 年、30 年的整体经济跟社会的利益的话,它绝对是一个负值。



比如说从 1995 年他们开始把 Glass-Steagall legislation——这个是当年1933 年跟 1934 年的时候在大衰退、大萧条之后所立的金融管理限制法案——他们从 1995 年克林顿的时代就开始放松了,一直放松到 2008 年。这 13 年来美国的金融业,它的利润上涨了超过 5 倍,你加起来将近1万亿。看起来这1万亿就是凭空来的,但是因为追求这将近 1 万亿的金融利润,他们在 2008 年产生了一个大的金融危机。这个金融危机所造成的社会财富损失,显性的跟隐性的加起来是在 10 万美元级别的。就是你搞这些高等金融是每一年都有赚——有 95\% 的几率可以赚钱,但是有 5\% 的几率会赔 100 倍。而且你别忘了在 2008 年的时候,因为美元是国际储备货币,真正大部分的损失还是转移到了欧洲跟东亚的国家头上去。最明显的是德国的中小银行,在那一个过程中次贷的最终的买者大部分就是欧洲的那些银行,尤其是德国的中小银行。因为这个次贷危机,德国的中小银行他们的资金链断绝,所以不再能够在欧洲放贷,这个间接的促成了两年后的希腊金融危机,然后连带的把欧元拖下来。



像这种东西你只要把眼光放得远一点,看得深一点,就可以知道高等金融这个东西纯粹是让国家社会损失 10 倍,还有世界损失 10 倍,让金融资本家赚那一份钱的利润的把戏。我是很失望,因为在两个月前我才注意到,中国居然把美国的投资银行所有的限制都开放了。这件事本身不是最大的问题,它的问题在于它是一个深层现象的表征。这个深层现象——为什么我说它是一个表征——这个深层现象是什么?这个深层现象是:中国的金融管理阶层迷信高等金融、金融衍生品。因为美国的投行他们过去二三十年的专业就在于搞这些高等金融,搞这些金融衍生品,他们把自己还有全世界的经济搞砸了两三次。



你中国还对他们不做任何限制的开放,这个代表的就是中国对高等金融没有任何戒心,认为它是一个正面的事情,所谓的金融创新他们要向先进国家接轨,这个才是我真正最失望、最担心的事情。我原本以为过去这十年的博客把美国在国际上搜刮,用美元做国际储备货币来搜刮全世界的这个事实解释清楚,也就是他的这个金融殖民的巨观现象解释清楚,那么最高层就会说我们不可以被搜刮,然后底下的那些金融管理人员就会注意到这个搜刮的过程、它的微观的机制是高等金融。这些金融专业管理人员就会说,我们不可以再沉迷下去,我们要用理性来检查每一个制度,来看这个制度是不是有隐性的损害。结果一个人都没有,非常的失望。我原本是因为中国在经济学方面是有自己的主见的,就是说虽然有一些被美国的资本公关洗脑过的人,但是头脑清楚的还是有的。所以我假设金融方面也是这样,结果不是,他们真的是一个人都没有,那这样一来我就必须要高调地来批评。



真抱歉,我原本是要讲欧洲,先讲欧洲再讲金融。



唐湘龙 31:18 

没关系,我让你已经开始之后,你就先把这个继续谈,没问题。



王孟源 31:23 

那我举一个例子。这个刚刚讲到希腊的那个之所以会做假账,而他因为周转不灵而必须要爆雷,但是原本的问题是起源于美国的投银投行高盛帮他做了假账。而做假账的手段就是用金融衍生品,用 Swap。那后来这个次贷危机它本身就是也是金融衍生品引爆的,这个是 Mortgage-backed security 房贷衍生品。为什么高等金融里面最毒的成分就是衍生品?我跟大家解释一下。衍生品先天就很复杂。很多连创造这些衍生品,金融创新的那些发明人自己都不知道里面有什么风险,有多少风险,对不对?这是它的复杂性。有了复杂性以后就有隐蔽性,因为他自己都没办法 100\% 的清楚,他只是 90\% 的清楚,他的客户当然就只有 10\% 的清楚了。有了这两点就可以创造信息不对称。我问你,你是好好的诚实公平交易会赚钱,还是利用信息不对称去敲诈赚钱快?



唐湘龙 33:19 

当然是用信息不对称啊,那就是像是内线交易一样。



王孟源 33:26 

对啊 ,你有这么大的信息不对称,有这么大笔的钱可以赚,你怎么可能说这些搞金融、高等金融的人会不占这个信息不对称的便宜?衍生品先天就是信息不对称,然后你再要考虑的是时间跨度。我刚刚提的一个例子就是, 13 年都赚大钱,然后到了最后一年赔了 100 倍,对不对?这个就是时间跨度,你时间跨度不够的时候,你还没有等到他爆雷的时候;时间跨度大了,它的爆雷的程度就几率就大,然后那个程度也大。



最后两点是衍生品的话特别容易搞出杠杆。你的这些复杂性跟隐蔽性,要遮掩的除了是信息不对称之外,另外一个常常要遮掩的就是你的隐藏的杠杆。你即使隐藏的杠杆是 100 倍,但是你可以表面上说只是杠杆 2 倍。这个是我亲自在美国最尖端的投行看到的现象。因为这些事情是没办法 100\% 确定的。我在这些最顶尖的投行里面,我看到一些其他的 Managing director——我那时候是 Managing director,就是队长。你有一行特别的生意,管这个生意的队长就叫做 Managing director —— 我那时候就看到另外的一些 Managing director,他们就在跟公司里面的风险管控人员扯淡,说我这个没有风险,然后....我说你这个完全就是在利用风险评估本身的不确定性来把它低估。明明你的杠杆是 100 倍,你硬要说只有 2 倍,对不对?你在欺些管控风险的那些人,因为最聪明的精英是会在当交易员还是当风险管控?当然是当交易员,对不对?你清华北大毕业的当交易员,然后那个三流大学的当风险管控,有的话哪一个人会扯得赢?当然是交易员会扯得赢,结果真的两年之后就爆雷了,那个公司损失了几十亿。这是杠杆度...然后最后一个是抵押品,你一旦隐藏了杠杆度以后,你的抵押的那个所谓的 Margin,那个储备金就不够了。所以我在我的博客上,特别说所有的衍生品都不能碰,只有一个例外,就是交易所交易的标准衍生品。



为什么有这个例外?因为交易所的标准衍生品不能复杂,它是必须要很简单的才能够标准化,而且它必须是公开的、常用的,所以就没有什么掩蔽性。大家都知道它的性质是什么,大家也知道它的风险是什么。而且你去看现在在交易所交易的那些期货跟期权,基本上没有人会去搞半年以上的。都是只有半年以内的才会有交易量。为什么?因为你上交易所做这种交易本身就不知道时间跨度,一长了就会有很大的风险,就会有非线性的风险跑出来。所以用交易所交易的这两方大家都是专家,大家都不会去碰这种时间跨度长的。然后也没有所谓的信息不对称性对不对?因为什么事都是公开的,都是标准化的。然后杠杆跟抵押品也不是问题,因为由交易所来帮你算,你的准备金需要有多少,风险有多少。这个算的也相当的准,因为它都是半年内的,所以风险可以评估的很精确,而且一旦价钱有很大的运动的时候,可以马上跟你要求多交保证金,对不对?一旦保证金不够了,通常就是一天之内要拿出去额外的保证金,要不然就没收了,然后连你的交易资格执照都弄没,对不对?因为这样子,所以交易所的标准衍生品才是能够有一些正面的价值。很多人都是拿着那个美国的金融教科书说这个衍生品多么多么的有正面的价。那些正面价值,其实你如果用 20 年、 30 年的眼光来看,还不到它的危害的 1/10000,那你必须要先把这个危害的 99.99\% 通通拿掉以后,才值得去追求那 1/10000 的正面价值,对不对?你要把这些 99.99\% 的危害拿掉的手段就是交易所的标准化。



所以我在我的博客上讲得很清楚,你要搞高等金融,不要搞;你要搞金融创新,不要搞。你要让美国的那些金融财阀进来,不要。你一定要交易衍生品的话,一定要限制只有在交易所的用的标准衍生品,才能够容许。今天我讲几种,就讲到这里,我觉得这个事情专业性很高——在这种演讲上面所能提的就只能够到这里好——我准备再写一篇博文,把它仔细地解释。



唐湘龙 39:55 

我打个岔,在我们进行下一个主题,以前因为孟源刚刚讲了,孟源刚前面讲的这一段其实专业性很高。因为孟源过去长时间在国外的这些顶流的金融机构里面工作了非常长时间,他自己除了是高能物理方面的优等生,最后他自己也公开讲过,他对高能物理他觉得是个大的骗局。这句话我最近跟台湾的政策面有关的时候,我常常在引用,台湾还是有很多人在谈什么核融合的、发电等等在谈的,反正谈的好像煞有介事的样子。不过刚那段我建议大家可以再仔细地听一遍了,之后如果你有一些市场上的交易的实務的话,你大概就比较容易懂孟源在讲什么。即使经过了 2008 年的华尔街金融风暴,那个是完全是衍生性金融商品的所搞出来的一个大黑洞。大家花了这么长的时间,而且是用矫枉过正的方式去补那个洞。今天其实都还在那个黑洞的补洞的过程当中所衍生出来的一些后续的问题。你不要以为 2008 年金融风暴的那些所有的问题都已经过,现在其实都还在一个处理的过程。



王孟源 41:11 

我再举一个例子。中国在过去这三四年它的这个出口成长数据非常的健康。即使是美国在扼杀这个市场,但是尤其是中国的汽车产业在过去这两三年异军突起,占领了第三世界,像俄国、墨西哥、巴西、中东、以色列、泰国、非洲这些市场。还有东南亚,更不要提东盟。但是你算一算,汽车这个工业是所有工业中最大的工业,它的销售额是最高的。全球第一号的工业,中国后来居上。在过去两年先超过韩国,然后再超过德国,然后再超过日本——我说他的出口量……但是你如果去算一算,为什么现在中国的经济还是低迷不振?很简单,因为你恒大爆雷了。你去算一算,恒大暴雷对经济造成的损失,至少是你这个汽车工业崛起所造成的正面贡献的两倍。这个对照,这还没有,你还没有去算碧桂园,还有万科这些连锁反应、骨牌效应的事情。所以昨天在博客上看到有人说他不觉得这个搞高等金融在中国会有什么严重的危害,我跟他说,你知不知道为什么恒大这个会爆雷的这么轻松?它能够这么轻松地诈骗到几千亿,然后让国家兜个几万亿的窟窿,因为中国有一个特有的制度叫做预售房制度对不对?这个预售房制度其实就是一个隐性的衍生品。



唐湘龙 43:11 

没错,就期货是房地产期货。



王孟源 43:14 

就是所谓的 Forward。中文翻成“期货”的其实是英文的 Futures。Futures 其实就是 Forward 预售在交易所标准化以后的版本,叫做Futures 期货,这个没有标准化的就叫做 Forward。恒大的爆雷会爆到让中国吃不了兜着走,有几万亿美元的损失,就是因为 20 年前容许搞这个预售制度,让一个小小的衍生品插在你正常的房地产交易里面。房地产本身就很容易搞成金融,因为房地产本身的价值在于除了你的那些建材跟工人的工资之外,就是地皮的方便,(编注:这里地皮的方便建议对照英文的convenience来理解)对不对?这个地皮的方便本身就有不确定性。那偏偏它又是一般中产阶级最大的资产。你一旦容许差之价格上涨明显的超过你的那个通货膨胀的话,它就变成中产阶级投资的最佳对象,那这个必然会产生一个泡沫。



王孟源 44:46 

你这在这种前提下,你还容许中国的房地产市场里面插入一个衍生品,虽然是一个小小的很不起眼的衍生品,你到最后就一定会爆雷,对不对?爆雷以后还逼的政府投鼠忌器,最后让许家印把几千亿美元转移到他的前妻跟儿子的名下,不就是因为他有这个预售制度,所以这个消费者的权益牵扯的太多了,所以处理的人员里面只要有一个,只要有少数是偏心的,就可以找到借口来轻松处理,所以恒大才会拖了三年,没有雷厉风行地一开始就把他通通抓起来。给了他三年的时间来转移资产,我跟他说:你这个小小的一个 Forward,就让中国吃了几万亿美元的亏,三年的工业产业升级,全球第一的工业发生了产业升级成为世界第一都没办法弥补这个窟窿。那你想想看,如果是全面地开放衍生品,全面地引进金融创新,这里的危险有多大?



唐湘龙 46:22 

这段很重要。来我们开始进入到了下一题,就是有关于就是这一次的美国,中美。我又先问一个简单的问题,现在从你的观察,美国的总统大选现在的情势如何,你也一面倒的看好川普吗?



王孟源 46:37 

对啊,我想你也记得我一直都是很看好Trump,其实两三年前我就说过这个,就是两年半前美国爆发通胀危机的时候,我就说了,单凭这个通胀危机,不管美国能不能软着陆,那个拜登要连任,不靠盘外招是不可能。OK,所以我一直都很看好Trump,唯一一个不能够事先预定说Trump一定当选的就是民主党的一些盘外招,就是像台湾 20 年前的那种,两颗子弹。



唐湘龙 47:21 

奥步了,奥步。 (编注:闽南语,不好的招数)



王孟源 47:24 

对。那他们过去这一这两年尝试的第一波盘外招就是打官司。对,真的打了三十几个官司。现在因为一个多月前美国的高最高法院明确的判决说不能够剥夺他的候选资格,所以那基本上这一条路已经死了,那这条路死了是不是盘外招就没了?不是,这个还有两颗子弹这一类的,这个我没办法预测。那毕竟美国有总统莫名其妙的死在枪弹底下的传统,所以我今天要提的话题其实也会再提到这方面。



王孟源 48:15 

那我先跟大家讲一下我们,我们最近这两三个月看到很多欧洲的新闻,欧洲跟美国新闻其实都是美国大选的余波荡漾,就是大家开始面对——我虽然是两年多前就可以讲说Trump 的民调绝对会超过Biden,但是现在尤其是最高法院判决说不能剥夺他的候选权益之后,大家知道这个危险非常的大,那就有一连串的反应。



王孟源 48:53 

什么样的反应呢?首先我们看乌克兰,我刚刚才讲过,说我不谈乌克兰的战术,但这个是政略上议题,专门提一下。他一个多月前换了总司令以后,不但是积极的把预备队投到前线,而且是把很珍贵的器材像是防空导弹,通通防空系统这些都还有那个远程炮兵通通往前线推,结果那当然就是方便俄军把他打得一塌糊涂。所以你如果去看那个俄军的战报的话,过去这一个多月打掉的乌军的防空系统跟远程炮兵,一下子变成过去这两年的三倍。



王孟源 49:39 

那然后你再考虑到一个礼拜前他忽然把珍藏了两年的那些俄奸,就是所谓的自由俄罗斯军团去跨越边境去打俄国,没错,那个当然是连一个村庄都没有打下来的,而且他们伪造了很多视频,这个我就不详谈了,大家有兴趣自己去看,但问题是这些都是他们藏了两年的那个战略资产,为什么忽然一下子这么慷慨的拿出来?很简单,因为 Trump 现在的势头非常的勇猛,那它的一个立即的效应就是在国会众议院的共和党人不愿意妥协,不愿意通过那 610 亿的援乌法案,那这样子没有钱了,你把这些战略资产藏在手里面,等到打到第二年,打到明年,根本就没有意义。因为你第一个今年就已经钱会打完了,第二个你到明年1月Trump就上。你把这些资产留在手上又就没有意义了。同样的道理我记得你也知道 Victoria Nuland就是他们头号对客的那个Neocon,新保守主义者,她刚刚宣布要退休。为什么退休?



唐湘龙 51:25 

可是她的刚刚接副国务卿啊。



王孟源 51:29 

他刚刚一个多月前跑到基辅去调停那个泽兰斯基跟Zaluzhny的事情,然后安排让那个总司令换人。然后安排完了她回来就退休了,OK?退休以后,她是大概两个礼拜前宣布要退休,然后两三天前宣布说会到哥伦比亚去当教授啊。哥伦比亚是什么地方呢?就是 Hillary 现在也是在那边当教授。是 Hillary 在 16 年前,15 年前, 2008 年 09 年的时候呢。原本 Victoria Nuland是共和党的,就是在那个小布希是手下的那些Neocon里面做中级官员。在美国这个党籍是很严重的,你一旦,尤其你一旦到中级跟高级的关联的话,基本上是不能够跨党的。但是Hillary 不管,硬是把她带到那个民主党,就是奥巴马的政权里面去了,所以他们原本是一个团队的,后来 Hillary 在 2013 年退休的时候,把他的这个Neocon团队送给Biden,所以他们后来就变成 Biden 的人。



王孟源 52:55 

那不论如何,第二,我在博客上也有人问说这个 Victoria Nuland 轮退休是怎么回事?很简单,因为你再过 9 个月, 10 个月整个行政团队就要换人了,因为 Trump 要上来了,对不对?那你在这 9 个月 10 个月里面如果连钱都要不到的话,你留着,那多留任 9 个月有什么意思?还不如赶快比同僚先离开,然后可以先挑选更好的养老院的位置。因为你知道这些政阀,他们退休以后都是有专门的养老院的。



王孟源 53:34 

当然,那如果你已经准备要退休,真正退休的话就是去教书。如果你只是半退休,准备 4 年或 8 年以后正党又轮替的时候再上任的话,你就会去智库。对,那当然是先到先得,所以你先退休的人。先退休几个月就有的挑啊。所以比Victoria Nuland这个退休这件事也是 Trump 要上台的一个间接影响。所以为什么我今天说是在美国大选年的欧洲,还有其他的一些间接影响,就是举这些例子跟大家讲一下,这些其实都是同一个因素的不同的效应。



唐湘龙 54:23 

你的意思就是说川普要当选了,所以美国的现在既有的就是说既得利益集团了,已经开始出现了跳船潮的意思。



王孟源 54:34 

对,然后更严重的主要是发生在欧盟,我刚刚已经提过乌克兰了,对不对?然后另外一个很重要的观察也是同样是这个 Trump要上台的一个效应,就是Von Der Leyen,原本Von Der Leyen本人是希望转到北约秘书长,因为对北约秘书长风光无限,也是州级或甚至世界级的、球级的、地球级的那个领导人。还不需要,还基本上还只有一个老板。就是美国,是不像欧盟有二三十个国家,大家吵来吵去。那所以他原本去年Stoltenberg就在 10 月就要退休。对,结果Von Der Leyen要接,遇到欧盟内部的反对,一开始是说Macron反对,后来上个月说是德国反对,Scholz反对。我认为Macron反对是比较合理的。Scholz 这个东西他有没有反对即使是媒体说了算,媒体要栽赃的话他也不敢出来澄清对不对?而且Scholz 这个人一点主见都没有,他怎么可能会为这种事情反对?我认为是事实上是Macron反对,然后但是因为Macron的威势比Scholz 要强,所以英美的那个媒体把它栽赃到Scholz 头上。不论如何了,Von Der Leyen去年她的转职被挡下来以后,一下子没有办法找到替代的候选人,所以 Stoltenberg就决定临时延任一年,能做到今年的 11 月,那目前是荷兰的卸任总理Rutte。



唐湘龙 56:34 

对。



王孟源 56:35 

Rutte他的呼声最高,当然另外还有丹麦跟 Estonia的两个白左女总理也要抢,OK。她甚至 Estonia那个女总理不要脸到说我们小国也有人权。



唐湘龙 56:52 

哈哈哈,不过吕特的声望。



王孟源 56:57 

好康的事情要留给我。



唐湘龙 56:58 

对了,吕特的声望看起来是不错的哈,他毕竟在荷兰的总理的评价是不错的,而且在位也非常久。



王孟源 57:07 

而且他也是白左跟那个美国霸权的一个忠实的走狗。没错,他要继任其实是没有什么问题,但是 Von Der Leyen去不了了之后,那她就只好留任 EU 的理事长,对不对?那现在我一个月前看到欧洲的一份通报,就是欧洲几个主要报纸,都转载就是吹捧她的一篇报道。那吹到什么地步呢?吹到她说是欧洲有史以来最伟大的领袖。



唐湘龙 57:54 

天哪,真的是把欧洲搞成这个样子



王孟源 57:57 

我那时候看到的时候都快吐了。



王孟源 58:00 

欧洲上一次有一个泛欧的领袖是罗马帝国。



唐湘龙 58:06 

哈哈哈,难怪你这句睡都睡不好。



王孟源 58:10 

8 世纪的查理曼大帝事实上也只占了德国、法国跟意大利北边的那个地方,不算是泛欧的领袖。嗯,对,比较你拿破仑跟希特勒都没有做到统一欧洲,这个你把Von Der Leyen捧到那个地步。噢,真的是。对为什么这样子?为什么他们必须要把Von Der Leyen捧上去?嗯,因为正是因为Trump要上台了。这个你不能够再靠着像Victoria Nuland这样子遥控了,所以你必须要有能信任的走狗,非常非常忠实的走狗。来在这个关键的欧盟的那个主席上面的职位上面来当,所以才非得要把Von Der Leyen连任搞定。



王孟源 59:09 

那你这些欧美的这些宣传机构,他们自称是媒体,但我认为他们纯粹就是宣传机构。他们这些宣传机构越重要的事就只有越激进,所以你可以从这个很激进的说法就可以看出这件事情对他们有多重要。就是这,而其重要性就来自于Trump。所以我说这件事情跟Trump大选年也是有关系。然后最后我们就谈Macron,在战争开始的俄欧战争开始的头两年其实是最理性的,欧洲的所有的国重要国家领导里面最理性的,最没有跟着起哄的一个,他是最常说和谈的,然后每次到北京转一圈以后回来就会说几句人话。那在欧洲里面是独一份的,欧洲国家领袖里面除了匈牙利的Orban以外没有其他的欧盟的国家领导人有这种能够说出几句像是人讲的实话的。但为什么在过去这一个月忽然就转了性子,从最稳健、最理性变成最激进、最疯狂?他不但要派兵进入乌克兰,帮乌克兰打仗和解,还说要把法国的核武核子武器贡献给欧盟。OK。你想想看整个欧盟里面哪一个,有几个国家有核武?一个。



唐湘龙 01:01:02 

哈哈,对,就法国



王孟源 01:01:03 

法国的核武就是欧盟的核武。



唐湘龙 01:01:05 

因为英国已经退出欧盟了,英国已经退出欧盟。



王孟源 01:01:11 

对,所以那么你考虑一下他这有几个可能的动机。第一个可能动机是他真的想要打垮俄国,可是你要解释为什么他头两年没有想要打垮俄国,而在过去这一个月忽然转性?事实上他这样宣布以后,法国有一个杂志叫做 Marianne,他就登了一篇那个报的新闻。这个新闻内容是法国军方替Macron准备的机密报告,这个机密报告是说乌克兰绝对打不赢俄国,法国进去也打不赢,对不对?你如果有这样的报告的话,他怎么可能会以打垮俄国为目的而搞这些事情?不可能,对不对?然后你可以说他去年在 Niger 西非那里吃了亏,但是他在Niger吃亏,俄国并没有出手,是Niger一厢情愿的,被俄国启发了。俄国并没有真正派兵,也没有安排,也没有去鼓动他们搞这个政变,对不对?事实上要说有仇的话是美国啊。因为美国他卖了那个法国,那时候要找西非共同体出兵的时候,是美国把他阻止下来,那所以也不是为了报复这件Niger这件事情。他是要把美国赶出去吗?最近这一个多月前发生的事情,是最高法院确定Trump可以候选,可以竞选,对不对?是美国主动脱离这个战线的可能性增大了,而不是减小,对不对?你美国已经要主动脱离战线了,你这个还要想着为着美国踢出去而搞这些事情,这个讲不通,对不对?然后第四个可能就是他是,我们上个月也讲过,欧洲现在到处都有大规模的示威,尤其是农民示威。那法国本身是这个很热衷这件的,这个示威是他们的国民运动。所以一直也都很严重。那Macron,但是问题是你出兵这件事情本身在法国的支持率、民调支持率非常差。Macron又是已经第二任了。他搞这些,他刚刚才竞选连任不到两年,还有三年的任期,对他搞这个有意义吗?嗯,所以你把这些通通排除了以后,你就只能够达成唯一的一个可能性,这个可能性就是Macron 从 7 年半前上台之后,又一直说要建立独立的欧洲。



唐湘龙 01:04:31 

没错。



王孟源 01:04:35 

那他基本上唯一的可能就是经过一个多月前美国最高法院判定 Trump 可以上台,判定 Trump 不会在他搞这些事情的时候扯后腿,因为Trump 本身也是说这个北约已经过期了,跟Macron当年说的北约已经脑死是同一个道理。他认为我可以先不管美国的参与,因为美国现在连钱都投不进来了,然后在 9 个月以后,根本就会换一个孤立主义的总统,所以我可以放手的去搞欧洲的进一步整合。他的目的是要利用这个派兵这件事情,激起欧洲向下一步整合,走入走上下一步整合的再进一步的这个动机契机。至于整合之后主管欧盟的还是Von Der Leyen,那现在也没办法,他也只能够希望说先整合之后,至少他这个美国不是欧盟的成员。或许再过四年,Von Der Leyen总要退休,到时候有一个比较强势、更整合的总是会有利的。所以它这个其实是以参战为由来推进欧盟在军事跟战略方面的整合。



王孟源 01:06:17 

嗯,这也就能够解释得通他为什么会说要把法国的核武当成欧盟的核武,嗯,对不对?而且最有意思的是他在上个礼拜的一篇演讲里面有人问他说。你参战,但是美国很显然的不想参战,尤其是Trump要上台的话更不会参战。有人问他这样问题,你知道他说了什么吗?他说有人跟我保证Trump当不上总统,这个这一句话非常值得玩味。



王孟源 01:07:09 

第一个他去问说Trump能不能当上总统,问的对象是谁?一定是美国政府的高层,对不对?这个人大概不会是拜登,因为拜登已经是完全老年痴呆了。是布林肯,很可能布林肯或Sullivan。第二个布林肯跟Sullivan,尤其是Sullivan,他本身就是竞选助理,他当然不会自己拆自己的台,所以他一定是说我们一定会胜选,但是我们不能够排除这里面有两颗子弹那类的盘外招的计划。这也是为什么我半个小时前提到这件事情。但是无论如何我觉得这件事最重要的观点在于Macron会主动去查询,主动跟布林肯或者是Sullivan查询说Trump上台的政策影响,不就正式验证了我刚刚讲的他的思路就是美国会不会管我搞这种欧洲整合的事情,对不对?所以Trump能不能当选对这个欧洲整合的影响当然是最大最立即的,所以他必须要先咨询收集意见。



王孟源 01:08:33 

那我今天举了四项,就是从乌克兰到 Victoria Nuland 到Von Der Leyen,然后再到Macron,四件很大的新闻全其实都是 Trump 被最高法院判决可以竞选之后,一个多月前,之后发生的余波荡漾。我一直是喜欢通过现象来看本质的。那我跟大家讲一下这个本质上的这些因果关系。这很有意思。然后我们就可以从这边来看未来的这一两年的欧盟,你可以说他们包括Macron在内都没有真正觉醒他们被美国坑的有多严重。欧盟要复苏第一件事情必须要摆脱美国的控制。那即使只要 Von Der Leyen还在,不管Macron的整合,尤其他的整合指的是军事上的整合,能不能成功,他在乌克兰的参战是否发生,当然大几率是不会发生,但是有小几率可以发生,这美国是不会参与的。但是不管他的参不参与,不管他成不成功,欧盟的经济是已经完蛋,因为欧盟的经济核心是德国,对德国的去工业化已经无可挽救。



唐湘龙 01:10:13 

没错。



王孟源 01:10:14 

而且目前他们甚至连讨论都没有讨论,就是民众在抱怨,但是他们的高层从总理一直到欧盟级别的主席、理事长,都没有人谈他们的那个经济被掏空的事情,所以也就不可能解决。这是我今天要讲的一个总结。



唐湘龙 01:10:38 

好,刚好就因为孟源的时间把握是很好的,那刚好也这个一个多小时时间到了,但是孟源刚最后讲的这一段了,那我没有机会去问了,那中欧关系会有怎样的影响?这我们下回聊,因为预期,因为今年法国会有两件事,我之所以认为马克龙出兵,嘴巴上面讲,但实际上不太可能,因为今年夏天法国要办奥运了,你法国要办奥运的时候,你说你要出兵,那咋想的?第二个就是说现在呢?预期今年可能就在奥运前,习近平是有可能去法国访问的,这个是中法关系的一个回或回访,所以中法之间所代表的就是说中欧关系。



唐湘龙 01:11:27 

今年这也是一个观察的重点,那因为要跟俄罗斯讲得上话的在全世界那就做中国,在欧洲老实讲,法国过去一直扮演的是代表欧盟去跟俄罗斯沟通的角色,所以其实马克龙是跑莫斯科跑得最勤快的人,不要忘了在俄乌战争开打之前,最后到莫斯科坐在那个大长桌的两个角落的是马克龙那张的照片,不过就是在战争发生前一个礼拜,所以你就知道其实马克龙的角色你听他怎么讲,不要过于简单的去解读。



唐湘龙 01:12:05 

好,来,我要感谢一些我们的观众朋友,因为一开始的时候王孟源很担心,就是说他自己有关于他的一个自我表述的部分,会不会让我们的听众朋友因为王孟源他们他的粉很多,他的黑粉也很多,但是没有关系,这都是一个有想法的人一定会遭遇到的情况。来,我们感谢一些观众朋友来对孟源的支持了。巴拉德 K 001,谢谢。然后令张感谢,他说王先生的忧国忧民令人敬佩。好了,明显圆感谢。然后这个是马自聪,感谢王孟源先生如此勇敢的觉悟和对世人的奉献精神,感谢。然后斯丹利,感谢斯丹利应该在泰国,然后 g h 感谢。他说可能和国内的金融学的晋升渠道有关系,比较迷信美国的那一些的金融期刊,这个都有可能,其实我们不只是在金融经济,包括我们念这政治学。



唐湘龙 01:13:13 

老实讲美国在二战之后,因为对全世界的包括知识学术领域各个方面,它是全面性的、压倒性的优势,在我们这个时代受教育,不管你学什么,你很难摆脱了美国价值观的影响。所以到了一定阶段的时候的自我的反省、沉淀,然后把那些会干扰我们思考的那那那些的,那些已经深入大脑的,影响你判断的那些因素要如何过滤?其实我们这个时代很辛苦的事情。



王孟源 01:13:45 

我提醒大家一下。嗯,美国的财阀开始反攻学术界,然后霸占他们的话语权,扭曲他们的教材。跟共识的时候是从 1959 年,是福特跟卡内基开始投资商学院资助商学院开始的。到 1970 年代初他们大笔的投资了好几个智库,然后收买,开始收买媒体。嗯,然后你看看衍生品是什么时候开始的?嗯,衍生品其实它的理论 Black Scholes也是在 1970 年代才发明的。



王孟源 01:14:24 

OK,所以它其实是在 1980 年代才兴业,所以衍生品出来的时候,高等金融进化到现在进步的时候,美国的财阀已经对学术跟舆论有了 100\% 的掌控,所以对衍生品跟高等金融的描述是都是完全一面倒的,正面就是忽略了那个 100 倍的损失,而只看 1\% 的利润。



唐湘龙 01:14:49 

是刚,这是刚孟源说的,如果是交易所本身,它直接发行的就是一级的,就是说这种衍生性的商品,譬如说台湾的这个证券交易所它有,台H7,就是说 0000550 就台湾550,那这都是等于是交易所,等于是官方机构它所发行的,它的衍生的透明度是很高的,那杠杆大概基本上也都是透明的,这个是你可以去琢磨风险的,相对的会比较小还是相对小?并不是绝对没有。



唐湘龙 01:15:25 

好了,下一个,那冯一伟要安慰一下王孟源,他说大陆有一个也是一个很有名的教授了,网红的教授陈平,最近访谈中他称赞了台湾的几位的评论员,他说介文汲的外交、帅化民将军的军事,还有雷倩委员的经济。但他说看问题最全面、最清楚的是王孟源,所以德不孤,必有邻了,你不要,不要太有压力。



唐湘龙 01:15:59 

好,那另外还有位叫卢麒元,卢麒元最近具名转载了王孟源博士的一系列批评中国金融政策的视频和文章,他把国内的这些就是大陆境内的卖国贼,称为金殖,就是金融殖民主义的金殖。最近和温铁军教授在 BB 站那 b 站开了一系列的对谈的课程,从中国过去的改革经验来批评金融腐化。那他特别提到。



王孟源 01:16:30 

你说我的印象中卢先生是香港人,他不是体制内的。然后其他的那些支持真相,也理解事实真相的人也都是经济学的,而不是金融学的。所以这个问题很大,你是金融学里面完全空白。



唐湘龙 01:16:50 

好,所以他只是要安慰你了,他说你没有你,你不要把自己搞得太有压力。他说德不孤,必有邻,王孟源博士的言论在大陆越来越受关注,也会有越来越多人急着跳出来要打压你。好,再来。邮件中说,王老师的独到见解总让人脑洞大开,感谢好再来清的代卡。他说一直有很多意见,想想反馈给王先生,今天总算找到了机会。我想说,王先生在金融方面的视角很纯粹,但是在政治和社会上面的衡量却考虑不足。我想提两点,第一就是说疫情时代那易刚降息被王孟源批评是通美,但当时民营经济受到严重的这个就是天天在要求政府降息,我们在做小生意的,很知道当时生意有多么困难。这个您说今年的经济目标应该趁势提出了,不及5\%,然后信心重于黄金啊。不到5\%,那民心就崩溃了。我想跟您说,大国的政府必须优先考虑到民众的需要,我优先于的中美对抗。



王孟源 01:18:01 

这个在我博客我都已经回答过了。当时的三年前的出口呢。嗯,它的Bottle Neck瓶颈是在于那个运输供应链。你的那个你降息什么降汇率根本没有影响,纯粹的就是影响你的利润,你降息以后你的利润少了,你卖的不会多。那至于说他刚刚讲的第二个问题是什么?



唐湘龙 01:18:30 

第二个问题他就信心了,就是惊人的经济指标的,就是说。



王孟源 01:18:34 

5\% ,真正打击信心的是你统计造假,而不是你的目标,对不对?你这个大家都知道它的统计有问题,那这个你还敢跟我说 5\% 跟 4.8\% 有很大的在定性上面的差异,那这个是非常的不诚实。



唐湘龙 01:18:55 

OK。



王孟源 01:18:55 

不可能是对的,再来。



唐湘龙 01:18:57 

这个的,在下一位网友JT,它是奉献出他在 YT 上面的第一次的 Donate跟超级的留言,那支持了王孟源那群戴卡,他说我们对中美金融战的观点是尽量不要和美国彻底翻脸,争取在第三世界开拓人民币的市场,真的把美国拖下马。现有的国际和金融秩序崩溃,对中国经济和人民的利益只有损害。那我们不破坏现有的体系,耐心建设新体系,我们不失国际的反美情绪的借刀杀人的刀,不会为了把美国脱下马让自己输,让自己……



王孟源 01:19:38 

这个同样是老生常谈,从六七年前我的博客就有这些什么中国传统智慧什么的,我都已经批驳了不晓得几千次了。这我刚刚,我昨天才在我的,博客上面讨论了一个很好笑的议题,就是你,你论证已经论证的严谨的不得了。同样的错误的言论已经反驳过几百次,但是很多人就喜欢重复,因为他们连基本的中文阅读理解能力都没有,或者是没有逻辑思考能力,我可以讲一句话,他们就是没有听进去,而且真的是确实的例子,两天前一个确实的例子。我在讲,我在讲说为什么这个中国最近在炒作的那个熔盐堆?不可能马上商用,是需要还要几十年,我说他刚刚实验堆才建成两年,而我们看,可以更简单的高温气冷堆,在德国开发了 21 年,然后在中国被清华接手接棒研究了 35 年,才成功商用。你一个更复杂的熔岩堆,它比高温气冷堆开始的起点还要原始。



王孟源 01:20:58 

你说要两年就商用,那不是明显的诈骗?结果你知道有一个人马上就跳出来说,你说德国人做不到,我们就做不到吗?那我已经明明写的中国的清华大学接手了 35 年,对不对?已经就写在同一个段落里面,他连同一个段落的那个描述都懒得看。事实上我们阐述这些真相,要打破利益集团,不管是美国资本集团或者是中国的诈骗集团或者是学閥集团,他们的欺骗手段的时候最大的问题就在于很多外行人他们没有足够的知识,甚至连基本的阅读理解能力、逻辑理解能力都没有。他们一定要来制造噪音,来帮这些坏人传播错误的学习。你不管用理性逻辑的论述解释过几百遍。我跟他讲 1 + 1 是等于2,1 + 1 是等于2, 1 + 1 是等于2讲了 200 遍,然后他又会回来说,你怎么知道 1 + 1 是等于2?OK,那这个,就因为这是人性,而且心理学真的有这么样一个研究,而且是最近 30 年非常普遍、非常受到重视的,叫做Dunning-Kruger曲线。



王孟源 01:22:37 

那这个我也在我的博客有专门的博文描述,就是大多数的人都是外行而且愚蠢的,可是偏偏是越愚蠢的人自信心越高。那所以他们就集中在一个峰顶上面,这个峰顶不是我发明的,它叫做笨蛋峰,在英文里面就叫做Mount Stupid



唐湘龙 01:23:06 

大多是。



王孟源 01:23:08 

笨蛋峰的居民,我没有办法,这种事情我除了忽视之外没有办法,因为要辩驳的那个内容呢。嗯,我博客上面已经讨论了几十次、几百次了。嗯,你们这看了以后视而不见我也没有办法呀。你还要我再写第一百次、第二百次吗?对不对?



唐湘龙 01:23:27 

好,那也现在已经是往王孟源那边半夜,唉,你现在那边是 11 点还是12 点?



王孟源 01:23:32 

11 点,所以脾气也不太好。



唐湘龙 01:23:34 

就快 11 点。就是已经是半夜不太好。对,就半夜 11 点。因为我在考虑他的睡眠,他最近比较常跟我谈他的睡眠的品质,所以因为跟台比,当我当跟孟源这样子聊,他每次都要做很多的准备。但是聊的过程当中不只是我们的听众,观众也有许多孟源在博客或者是其他地方过去认识王孟源的人会发拢到这个地方,这可能是孟源现在在网络平台当中来讲,他比较能够定期的畅所欲言,谈的比较久的一个平台上面。那当然对我来讲也是一个非常好的成长机会。好,孟源你就该去睡觉了,该休息就要休息,感谢呢。今天星期五的时间,龙行天下王孟源在美东的连线所提供的精彩内容,感谢王孟源好所有的观众朋友跟大家说了,周末快乐,下个月见,拜拜。



王孟源 01:24:33 

欢迎有理性逻辑质疑的读者在我的博客上言,可以去看我那里面将近1万条的回复。只要是能够真正的指出我的博文或者是辩证的错误的,我都是认了然后说谢谢,谢谢你指正。这个的发生的不是太常见,但是至少也有十几二十次。



唐湘龙 01:25:02 

很好,OK,好,我们下个月聊,感谢。



王孟源 01:25:05 

拜拜。再见。拜拜。



\twocolumn[\begin{@twocolumnfalse}
\section{國際金融危機長期化}
\subsection{20240419}
\end{@twocolumnfalse}]唐湘龙 00:17 

来,欢迎到龙行天下来,今天星期五的时间 9 点半到 10 点半的龙行天下的单元,那人在美国,在美东,其实跟我的几乎时间是相反的。现在早上在中原,标准时间早上的九点半,在王孟源的时间就是晚上的大概九点半的时间,所以时间相反,但是必须要在提供大家非常的精彩跟精致的思考的内容。好,今天我们定的主题国际金融危机的长期化,尤其在关注到中美关系的互动的时候呢。耶伦在短短的十个月之内两次到中国的访问,虽然是个大妈型的人物了,可是我认为他所处理的中美关系问题是当下国际的政治大国政治里面的最核心的部分。好,那我们就顺着谈,那今天那大家大概有机会在一个小时之内可以听到王孟源对于一些比较长期问题的思考跟整理,来在我们线上的王孟源。



王孟源 01:23 

欢迎孟源。



王孟源 01:25 

大家好,很高兴再来跟大家聊天,正如湘龙刚刚提的,我本人并不是很善于言辞的人啊。是啊,迷思而卓语言,而且我也没有什么虚荣,我也不求名不求利,所以我不是为了自己的虚荣跑来这里跟大家讲这些东西。其实我要勉强自己上节目的原因是因为我觉得我对这个世界的真相了解的比绝大多数人都要深刻很多啊。现在的这个社会充满了利益集团,忽悠群众,对,是带来的谎言跟错误认知,甚至整个学术界的科目都有可能是被彻底扭曲的。所以我花了 10 年来写博客来澄清这些真相。



但是博客这是一个很小众的媒体,所以我上这个节目其实需要来接触更广大的群众,然后希望能够给大家一个机会,因为我的受众毕竟是要有理性思考能力,而且越有好奇心,而且有那个虚心,有那个才智能够理解这些问题的人,这是本身就是小众。那如果我的平台又是小众,这个小众再乘以小众的就是变小众的平方,那我的影响力就很小了。



那所以我借湘龙的平台来吸引更多的观众,(湘龙:是,非常乐意),但是脱离了我的博客的平台之后,就有很多的不合适的地方,就是因为我所讨论的东西是完整的认知架构,它包含的往往是需要几百个事实证据,然后复杂的逻辑分、逻辑辩证,我花了 10 年才构建了 1, 000 多万字的博客。



那你在这种事一个小时的视频,事实上是不可能遵循那种原则的,所以就会流于浮面。而且如果要谈一些新闻性的问题的话,必须要当场回答,那就真的就是必须要很肤浅,显得事实上,也不切题跟事实它的那个紧急性也许很高,但是并没有什么重要性,所以我今天想要试着就是谈一些比较深入,所以比较散漫的问题。



那唯一刚刚湘龙介绍的那个金融的问题,刚好是我现在在博客主打的政策,因为我出来传播真相,而且希望示范这种求真的态度跟求真的方法跟原则。这是学术性的,那当然还有教育性的,因为有很多我刚刚提到那种有理性的知识分子,但是因为这个世界被利益集团所掌控,所以他们一直没有机会建构自己正确的认知架构来接触够足够的真相。那刚好我有这个兴趣,有这个历史。嗯,有这个知识能够你已经先理解这些真相,所以我可以跟他们来交换的心得。那这是第二个意义,就是教育的意义。那当然最重要的我还是希望能够真正的改变政策,因为政策才是对公益最大化有最直接、最明显的贡献,这也是中国传统士人的志向,你不管是在殿堂还是在乡野,你总是要关心民生社稷的议题。



那在现代我觉得比一个现代社会主义者,我的追求的是全人类公益的最大化,而不是某一个特定族群的利益的特大最大化。那我认为,我一直强调,我之所以对中国有偏爱,除了我自己本身文化的传承之外,另外有一个很坚强,最大的原因是因为中国的确是当前人类未来公益所系最重要的一股力量,远远最重要的一股力量,那所以这股力量之中被污染、被忽悠、被扭曲的一些文化跟学术认知是很严重的问题,我们必须要不顾情面的把它摊出来谈。那博客写了十年,就是有这三重意义。那我刚刚提过以政策建议是最重要是第一优先啊。那我刚开始的写的时候,当然还不是面对整个中国,而是针对台湾,当时台湾的政策层面最大的问题是对的迷失,就是整个台独化、反智化的问题啊。但是我思考了一下以后,我觉得你不能够直接去批判,因为像当时批判最流行的是什么方式呢?就是像李敖那样子,李敖他就跳出来骂,但是他是谁都骂,但是他没有一个逻辑结构,没有一个客观的认知,就是他的,他指出对方的非理性,但是他并没有提供自己一个理性的逻辑架构。为什么要反对?我认为这种辩证必须是对与错。你骂人可以,你争吵可以,OK,但是这个必须是对与错之间的争吵,是好人对坏人的责骂,而不是两个同样非理性的。嗯,两个极端,互相的指责,互相的主观的谩骂啊。所以我当时思考了一下,觉得的基本问题在什么?台湾这个走入这个歧途的基本问题在什么?在于对美式民主的迷思崇拜,这个迷思崇拜是从蒋经国就开始了。蒋经国时代在实际的经济建设执政上都还是脚踏实地,一步一步向前。但是从他那个时候已经对所谓的民选制有宗教性的崇拜。所有的美国的文化不管好坏,一律都把它放在神坛之上。反正台湾只要美国化就一定是对的。那正因为在这个文化思想上的大原则大方向走错了,所以才会容许。因为你在实际上是美国地缘政治用来戳中国的一根钉子。而美国的这些所谓的白左文化,什么民选什么,什么多元化,什么包容,什么自由媒体,这些其实都是为美国地缘政治服务的。美国地缘政治又是为一个为他的金融利益而服务的。



所以我觉得要解决,要彻底地刨台独的根,你必须要这一点解释清楚。所以我的博客一开始。如果你去看头两年, 2014 年到 2016 年,头两年我讲的东西都是什么?美国社会的真相。还有美国的地缘战略的策略,比如说有一篇博文叫做美国的东亚战略史。OK,十年前刚出来的时候我跟你讲绝对是独一无二的。就是也许有其他的人有类似的看法,但是把它这么完整的。比如说张文木教授,他的一些说法我在 10 年前有参考过,但是我觉得我写的那篇文章逻辑更完整,取证更确实。那但是我想我在两三个月前也提过,我到 2016 年就已经对台湾完全绝望了,绝望就是因为马英九已经是他任期的最后一年呢,是能够对教改作出拨乱反正的最后一个机会,然后他放弃。对,到 2016 年你基本上就可以放弃台湾,所以从那之后我这个就不再针对台湾来讲这些事情。当然你这种移风易俗的事情,两年事实上也是不可能,而且尤其是一个小众的平台,一个博客。嗯,那时候我也一点名气都没有。所以你看现在你去看台湾的论坛,凡是比较反独派的,他们用的论证方式都已经采纳我十年前所用的,就是从美国着眼,就是强调美国这一套本身就不合理,它本身就是要拿你当炮灰的一套,起码是口吻。那但是晚了十年,对不起,已经来不及了。马英九到现在,懂了没有?我也不知道,但是反正已经来不及了,他必须要在 2016 年就懂才行,但是他没有。所以台湾就已经万劫不复。



那 2016 年的时候,虽然我对台湾的关怀失败了,但是有另外一件事情很意外的成功了,就是当时中国高能所要推动大对撞机这件事情,他们名义上是大概 1, 000 亿人民币的预算,实际上是至少一万多亿。就是他们少报了一个数量级啊。这个我的博文有详细的论证,那当时刚好是因为我没有什么名气,所以那个他们高能所的所长王贻芳,还有他们在美国配套的丘成桐。



他们不知道跟我作辩论的时候会遇到什么样的对手,所以他们就高高兴兴的下场了,以后就让他们输得很难看啊。然后又刚好杨振宁先生他也愿意出来声援,所以我们反方是大获全胜,就是把他们的骗局完全的揭露。所以那是我博客第一次真正对政策有贡献。那我事先是完全没有预期的,因为我会提到大对撞机一这件事其实是因为我是清华物理,清华大学物理系本科毕业啊。那开始写博客,头两年讲的都是美国的事情,还有台湾的一些军事的议题,所以他们就我的那些老同学已经有几十年没有联络他们就早上我说,没想到你还对这些事情有兴趣,那可是我们都是物理的本行,你为什么不谈物理呢?



所以,才写了两三篇有关高能物理的事情,然后里面提到大对撞机也是完全没有针对性的,就是顺口一提说这个是他们整个,现在高能已经就是实质上是一个国际诈骗集团。那我就提到说他们最新的诈骗项目就是在中国搞。对,这就是一句话、



唐湘龙 15:07 

不,我要,我,我会注意到我王孟源的,其实就是从这个议题该开始的。我不懂物理,但物理我们就基本的 ABC 而已了。可是高能物理我当然是不懂,可是我注意到我们过去被这些高能物理的许多的实验大强子对撞机,欧洲做的如何,中国也要如何。基本上面你就是顺着那个风向走,被惯的就是就是,迷迷茫茫的。可是当我听到王孟源讲话,唉,这个是有意思的,有学问。所以我会注意到王孟源就从那时候开始的。



王孟源 15:39 

然后就有一个香港的记者拿着我的文章,一个博文,不连报纸上都没有,那个博文拿去问丘成桐,然后丘成桐开始人身攻击,然后攻击的时候他也是不小心一句话提到杨先生,说杨先生一定不会像王孟源这样胡说八道。然后结果杨振宁先生,就出来说我也反对建大对撞机。那这下子接下来就双方就都被赶鸭子上架,然后他们也必须要回应,大家来来回回了几趟,我想谁是谁非已经很清楚了啊。但是这个,这件事情就变成一个很大的问题,我现在后来要批评那个,要接下来另外一个大骗局,就是整个他们的那个合肥中科大把整个物理系都搞成所谓的先是量子通讯,然后来量子通讯被证明是没有用的时候,他们很高明地把它转移到量子计算去,变成一个很大的利益集团,而且它这个利益集团搜刮的方式又跟大对撞机不太一样。大对撞机,那个大概需要1万亿人民币的预算,基本上都是要公家的钱。但是这个量子计算所需要的钱其实并不多,他们的那个几千万的经费跟公费要呢,其实我觉得合理的,我并不反对他们拿这几千万的公费。他们诈骗的方式是拿了这些奖项以后还要收买科技部,还要收买他们的那些官媒,然后帮他们大吹大擂,然后他们到股市上去收割。他们在量子通讯的时候就收割了几百亿,然后这次量子计算,我想应该是又高了一个数量级,那这些钱最后还是中国的股民。



唐湘龙 17:54 

要付出代价。



王孟源 17:55 

对,所以这个中科大的这个诈骗其实是金融诈骗,只是用科技做名头来挂羊头卖狗肉而已。那我在上个月还是上上个月?我说我那时候没有及早在七八年前的时候,七八年前就是 2017 年、 2018 年的时候就站出来批评量子通讯,那时候他们还没有搞量子计算啊。那时候我还没有,我没有站出来批评量子通讯,有好几个原因,一两个月前我提到其中一个原因是我认为这量子通讯已经快要被证明是没有用了,所以我不需要做事后诸葛亮,然后再落井下石,反正他们该骗的钱也都骗了,那也就算了啊。



但是还有另外一个原因,我跟大家讲一下,这是我个人私下的一个原因,那我可能博客的多年读者可能都不知道这个原因,就是 2017 年我又因为大对撞机的一个关系,我又参与了另外一个辩证,就是那个悟空卫星。这个其实我在你的节目上也提过。悟空卫星是什么呢?它其实,也不是花很大的钱,我刚刚提到大对撞机,它最大的问题是花的钱太多了,你1万亿人民币去买一篇论文,即使是顶尖学术机构、顶尖其学术期刊的头条论文也是不值得的,而且一定会挤占国家的科研经费。当然。但是像你看那个悟空卫星的话。



他们那个团队第一个,他们只花了几千万,那个卫星本身只花了几千万,他们即使是完全在预算之内把它做出来,而且花的钱非常的合理。而且他们做的这个物理是真正的物理,就是已经有理论基础的,就是我们所谓的标准模型。而大对撞击那个是没有理论基础的,基本上是要希望瞎猫能够碰上死老鼠,碰上什么新的物理,所以我提到大对撞机或那个高能所的时候,我就说他们是诈骗,他们是纯粹诈骗团伙。我提到量子团队之后,我说他们诈骗,这是金融诈骗,在股市场诈骗。但是悟空卫星这个团队虽然被我批评,但我从来没有说过他们是诈骗。OK,事实上我觉得他们既不蠢也不坏。他们犯的一个错误就是他们没不没有理解到统计学教科书上面的一些公式有隐性假设。但是这是没有办法的,因为中国,在高能物理最发达的 50 年代、 60 年代,还有 70 年代并没有跟上美国的那个科学界,有什么交流,所以他们被排除在外,所以有一些那种不传之秘,就是我。我也提过,就是基本上只是口口相传,你教科书上面写没有写清楚的东西了,他们自己没有学到,这并不是他们蠢,他们就是因为历史遗留的问题,所以他们那个整个行业里面没有那个知识的储备,所以他们就犯了一个错误,把一个统计噪音当成一个结果出来宣传。



他们既没有要骗公费,也没有要股市诈骗,他们纯粹就是误解了。一个误解,但是他们还是犯了两个小的错误。那这两个小的错误为什么说是小呢?因为都是现在中国科研界、知识界的常态,并不是他们有了什么逾矩的行为,他们只是遵循常态,但是我认为这些常态都是很糟糕的。很不幸,都是很糟糕的。这两个是什么常态呢?第一个常态是你那个公关稿尽量往夸大的去写。就是,你明明是一个不能够确定的信号,还需要在 10 个月收集资料才能够确定他是不是统计噪音,但是他的公安稿是往好的写,就是说我们可能拿诺贝尔奖啊 Bla Bla Bla,这个是不太好的,至少在我,因为我进物理界的时候是 70 年代末期, 80 年代早期啊。在那个时候你,一个新的研究结果发了论文以后,你总是要还要等几个月,或者等几年,让整个相关行业里面的专家都有机会仔细的审查过,确定你做的东西正确,然后深入的甚至深入的理解他这个,你做的这个东西有没有有什么重要性?在整个行业中有什么重要性?等到这些盖棺论定了以后,大家才回来有共识写一个回忆,然后才发新闻稿,现在我们看到了这种随便写一篇论文,甚至还不是论文,就是一个所谓PrePrint,就是预印版。



根本就没有在期刊发表的,就先召开记者会,乱发公关,说我们有了世纪性的突破,要为国家争取一万亿、十万亿的财富,然后什么什么的,然后一举超越美国啊。什么什么。嗯,这种漫天随口就来的胡说八道,这是什么时候开始的呢?其实很短,才只有 35 年,就是 1989 年,我不晓得你记不记得冷聚变那时候开始。冷聚变他们也是两个化学家做了一个实验,然后就一个物理实验,然后就自以为有了可以聚变发电的新技术,然后就开了一张记者会,然后大家吵了两三年之后,大家才确定真的是就是胡搞,就是他们程度太差,自欺欺人,那但是他一个很恶劣的影响,就是从那时候大家开始发现,诶?你这样乱搞事实上并没有任何的恶果,对不对?反而是如果顺便的话可以到股市去。骗个几十亿、几百亿、几千亿的美元。



唐湘龙 24:56 

对,胡说八道是可以赚钱的。



王孟源 24:58 

胡说八道是可以赚几百亿、几千亿的。那从那时候开始就泛滥成灾了,甚至现在几乎每个大学都有专门的公关人员来写这些,中科大甚至还另外成立了一个机构,叫做风云之声。有一个我所谓的学术伥鬼叫做袁岚峰,他没有什么知识,也没有什么成就,他唯一的任务就是负责替中科大写公关宣传稿。那这个最糟糕的是他还假装是科普,那因为中国它的经济发展太快,而且事实上中国自从鸦片战争以来 100 多年了,追寻西方一直都是很羡慕这种先进的科技,很积极的一些科技,所以民众的心里有这种不平衡的需求。嗯,你想想看,像国术就是中国武术的这种迷信,或者对中医的这种迷信,这其实都是鸦片战争以后那半个世纪心理不平衡。到了 19 世纪的最后十年,还有 20 世纪的开头 40 年,这些国术对中医的那种迷信才被扩张出来,这其实是一种心理补偿作用。然后大陆在 80 年代改革开放之后,引进西方的科技,发现美国的科技这么先进,他们的消费者商业这么繁荣,他们的财富这么高,财富的程度这么高。所以他们就相信科技万能,你不要说是这种拿科普来假装是科普,拿公关假装是科普,就是摆明了是科幻,大家都把它当科普。我刚刚提到大对撞机,他们中国最有名的科幻作者是谁?刘慈欣。他为什么大队装机一开始的时候高能所会发公关,然后来发动这些网民来吹捧自己,原因就是刘慈欣的三体里面那个智子被封锁,让地球受外星人侵略,他们那个制止封锁的是什么?就是大对撞机。所以那最糟糕的是刘慈欣是个三流科幻作家,这个是见仁见智。(不确定)



唐湘龙 27:44 

可他现在很红。



王孟源 27:46 

他现在是很红,但是我对他有一个批评,我相信大家是没有办法反驳的。就是他一个科幻作家,假装成未来科技专家到处去给座谈,这个是很糟糕的事情,很明显是一种诈骗行为。我在我的博客上面有提过,中国人看到Zelensky,Zelensky是怎么发家的?他在乌克兰的电视台上面演一个电视剧,演一个很好的总统。演一演到最后就变成真的总统。



唐湘龙 28:25 

OK,大家看看戏,看久了太入戏了。



王孟源 28:28 

演总统的人选成总统了。那你跟读科幻读久了,把写科幻的人当成科技专家。这有没有什么不同?没有不同,完全一样的思路,就是我觉得这种反智的愚昧现象,你在看别人的时候看得很清楚,但是很好笑的是你自己犯的完全一样的错误却看不出来。那不管怎么样,我在 2017 年这个悟空卫星这件事情上,我看我认为他们犯了两个错误,第一个错误就是他们也去搞这个公关。嗯,他们搞了这个公关,我才写了一篇文章来批评他们,那这个批评之后,因为我那个时候是观察者网的专栏作家。那这篇文章也发在观察者网的专栏,他们就犯了第二个错误,是他们去找上观察者网的总编,然后跟他说我们不喜欢您这样子乱批评我们。哈哈哈哈哈哈,然后他犯了这个错误,我为什么是说是错误呢?以中国的这个人情社会,他们这样去讲当然是有效的。



从那开之后开始,我写的任何一篇稿在观察者网要发都是很困难。OK。但是我那时候已经知道我持续的替写下去, 2017 年我博客才写了三年。嗯,那已经到现在又有 7 年了,对不对?我那个时候觉得我继续写下去一定会有更重要、更紧要的话题,需要借助观察网那个平台来发表。所以我不能够继续去得罪观察者王的总编,那我也知道中科大要比悟空卫星的团队要邪恶,要强力的很多,他们事实上是不准,完全不准中国国内发任何揭穿量子诈骗的那个文章,你即使是翻译国外的专业文章都不可以。比如说国外的金融时报有一篇文章是从金融的观点说这是诈,金融诈变。那篇文章我建议他们翻译,他们总编就说不行。IEEE,美国的电机专业机构有一篇文章说量子计算的那个所谓的用途非常的可疑,我们找不出实际的用途,那篇文章我也建议他们翻译,总编也不让。为什么?因为不敢得罪中科大那些人。那我那时候,我那个时候经过悟空卫星,那一次被他们抱怨,然后接下来从 2017 年年底到 2018 年,我在观察者网发稿越来越困难,也明明是一些表面上看起来没有得罪什么人的文章都被拖来拖去,然后审查的一塌糊涂。



嗯,那我就觉得我将来可能有机会谈一些更重要的事情,然后观察者网是一个很重要的平台。我如果去,我如果也加入去批判量子通讯的话,那我基本上就会马上被封杀,那这样子就因小失大,而且刚我也提过,如果当时我也觉得量子通讯的骗局已经是到尾声了,所以没有必要再我再跳下去。跟人家掺和的那件事。结果我的预感是正确的,因为我不知道你记不记得 到 2019 年就是两年之后,到 2019 年年初我就注意到所谓的美国陷阱那本书就是法国人被美国人关起来,然后把阿尔斯通被强迫卖给 GE 的那件事情。



唐湘龙 33:07 

非常贱的事情。



王孟源 33:09 

那件事情是我建议给观察者网发表以后他们就继续的推,因为一下子就风靡了整个大陆的舆论界,然后事实上观察者网的相关机构还后来就帮助安排了那本书,再翻译成中文。然后他还把这原作者邀请到上海去。



但是这个内幕我从来没有跟大家讲过。事实上我那篇文章发给观察者网的时候也是被拦下来了。被拦下来以后,我那时候急得不得了,因为我知道那篇文章有多么重要。所以跑去,我还跑去当时我跟我合作的是八方论坛,史东人也很好,我文章发过去第二天我他说这个总编不太喜欢,然后我就赶快打电话给史东,然后史东就说好,那我们在一天我就帮你特别做一个专辑,让他也讲这。所以有关美国陷阱的那件事情,其实是我先做了视频访谈以后,文章才发出去的。嗯,跟了一个,就是这个视频已经先发了,然后他们才了解到这个重要性,要不然那个总编一看是王孟源写的东西,他根本内容懒得看了。哈哈哈,根本就懒得看,就是不想发。



唐湘龙 34:46 

这个,我能理解。



王孟源 34:47 

OK,差一点就被拦了。所以我如果 2017 年的时候跟着去批评潘建伟,跟着去批评量子通讯的话,美国献礼这件事情就可很可能就没有后来。那你说美国陷阱这件事情在舆论上传开对政策跟对现实世界有没有影响?对华为其实是没有影响。



因为当时华为要挽救孟晚舟,基本上就只有两条路,一个是在加拿大打官司,这个是司法的。另外一个是跟美国他们的外交交涉,这两条路跟你能不能看穿美国人的真正目标跟手段没有什么实际关系。OK,所以事后任正非也没有多么感谢我,我也觉得合理的,因为我事实上没有对这件事情影响很大,但是影响的不是华为的那件官司。那影响的是什么呢?影响的是我从 2017 年年底到 2018 年,讲了半天,说美国当时Trump的贸易战,中国的对应太过软弱了。



那时候还有一大堆所谓的中美夫妻论。这些中美夫妻论在美国陷阱那本书风靡之后就被扫清了,就没有人再会傻到会去相信那些事情。所以中国对中美贸易战的应对,从一开始的基本上是,就是转过头来把屁股让人家随便玩的那种态度,转成愿意直面对手,愿意维持自己的尊严,把脊梁骨挺直的是美国陷阱那本书的。



所以那也是我这一次,我最近希望能够重复的,我后来,在那之后我一直希望能够在通过舆论来影响政策。但是我也感觉到这以前的那两个经验给我感觉到第一个他很随机,就是要等适合的话题,然后还要有适合的管道。这主要的原因就是因为现代的社会人口基数太多了。你是一个工业社会啊。大家都有受过基础教育,都能够识字,都能够上互联网,噪音太多了。



然后正因为噪音太多,所以那些政策的执行者跟制定者他们根本不想听你的这些普通媒体的,我们这些,我们现在这些公共媒体 99.9999\% 的新闻评论员都是娱乐性的讨论。你讲的东西其实都是为了满足读者的爽感,读者跟听众的爽感。



跟追求事实真相没有什么没有什么关系。那更不要提说去影响政策,那些政策的制定者跟执行人,他们才没有那个功夫来听这些 99.9999\% 的噪音。所以我在那其后又 2019 年之后。其实在 2019 年下半年我就有一个又有一个很重要的建议,那个我就发现用公共媒体这条路,用舆论这条路走不通,这件事是什么呢?就是我预测当时预测美国的有一场金融风暴会到,而且这张金融风暴是50 年来第一次通胀性的金融风暴。后我说这个风暴应该是 2- 3 年之内会到,然后来了以后是纯粹的通胀性的,那因为它是通胀,所以中国必须要特别注意,要以不能够以普通经济衰退的时候我要维持出口,所以要把那个汇率压低,那种反应要反过来把汇率抬高,然后对美国的通胀落井下石,这样可以摆脱美国打击你的那个打击你的力度。



但是那在那之后舆论的反响很少,为什么?因为这个东西全世界就我一个人在说,所以专业性这么强,所以大家一看就觉得王孟源是什么人的?你这个道理说的对不对?我没有办法判断,反正你的名气不够大,我就不相信。那就完全没有影响,所以我就转而去走体制内的管道。那这体制内管道照规定我是不能够详细说的。但是,但是那个时候刚好是啊,李克强,他李克强这个人很可惜,我觉得他。我对他不是很熟悉,因为我没有见过面,也没有什么第一手或第二手的资料,但是我所能看到的东西,我觉得他应该是个好人,但是绝对不是一个好总理。这个这里的差别在什么呢?清朝最好的皇帝是哪一个?我认为是雍正。是好人吗?



唐湘龙 41:05 

一般认为并不是。



王孟源 41:09 

是清朝所有皇帝里面杀人最多的,但是清朝皇帝里面最好的人是谁?是嘉庆。嘉庆是好皇帝吗?



唐湘龙 41:20 

嗯,事后来看,并不是。由盛转衰的关键就在。



王孟源 41:24 

嘉庆。由盛转衰的关键在乾隆那种浮华的风气之下。真的需要有一个人来扭转那种慢慢的腐败的风气。来重新整顿,来重新做改革。那嘉庆什么都没办法做,所以你做国家的领导最重要的是你的那个决策的方向。你改革的魄力等等。而不是你人好不好,做一个私人的私德好不好?我觉得。我为什么说我怀疑李克强是一个好人呢。因为他还在位的时候,体制内的邀稿很多,而且层级很高,而且奖金很高,当然这些奖金对我是没有什么用的,因为我的原则是我是美国人,我如果再拿了中国官方的钱的话,我自己找麻烦了。FBI要来查水表,所以我不要说是中国官方的钱了,连那个观察者网的稿费我都是一毛钱都不拿的。



唐湘龙 42:35 

哈哈哈,真的?这你没有跟我讲过你是不是在当义工啊?真是因为。



王孟源 42:41 

这都是因为,人家笑我说我的博客,还有我的微博都是自由,为什么?我真的除了这种个人的自由乐坚以外,反正我,第一个我不在乎对这点钱,说真的,我能不能退休,我不在乎这些钱。第二个我不方便拿任何那种正式的机构,什么薪水、什么奖金我都不方便的。因为我是美国公民。



因为我只知道王孟源的生活是非常简单的,甚至于是非常单调的,然后能够过这种生活的人,然后很专注在思考问题,而且每一次都敢去捅马蜂窝。你刚刚讲的这几件事情有一些断断续续,我也听你谈过。可是在这个每一个都是一个很大的马蜂窝,就是在一般的世俗性的讨论,或者说这种新闻平台当中,是不会有人去做这种吃力不讨好的事,你做的真吃力不讨好的事。是我的人生意义嘛、



唐湘龙 43:45 

对,所以我说我,所以我才会说。



王孟源 43:48 

我不需要再赚多钱了,我也不在乎这个名,而且你在私下也嘲笑过我不知道多少次,我是一个宅男。我也公开承认。



唐湘龙 43:59 

对,所以常常有人问我说,你认识王孟源,王孟源是一个怎么样人?我讲不出来,我只能说这个人是个怪人。好,但是最早刚刚讲的这件事,我不管大家能不能听得懂,王孟源刚刚在他的这十年博客里面,他简单的提到的几件事情。每一件事情老实讲,就算是对绝大部分人,包括我在内,我在看这些的讯息的时候,我是没有能力在里面的看出里面有什么玄机或者破绽,要能够有sense,能够看得出来就不容易,看得出来之后敢发出声音更不容易,因为这都是捅马蜂窝的事,吃力不讨好的事情。好



王孟源 44:40 

对啊,第一个你要有那个专业能力。你必须要知道真相。OK,第二个你要愿意讲的话,你必须要把公益为重。你必须要把公益放在你自己的私人的利益为重。因为你如果是一个聪明人的话,尤其是有专业知识的话,你就有信息不对称。那现在中国流行的是什么?事实上美国也是一样,你有了信息不对称,就拿信息不对称去骗人,对不对?



这个实际上是骗人,但是他们可以说是赚钱啊。然后所以第一个,你必须要有信息不对称,第二个有了信息不对称以后你还不愿意拿这个去兑现,反而是主动的把这个信息公开化。解释给大家听,第三个你必须要有沟通的能力。OK。那大概 10 年前,英国的心理协会有一篇很重要的报告,但是一般人不知道。他说你智商只要超过智商的差别,只要超过 30 点,就有绝对的沟通恐困难。我发现,我为什么坚持要……我如果就用普通的沟通的话,那我这一辈子都不用跟别人讲话了。因为我不想说我智商多少,但是基本上我就不会有什么朋友了,没有什么好沟通的。但是我为什么博文坚持要用那种辩证,是很详细的一步步的逻辑解析出来,因为我发现这样的写法可以把这个智商的沟通范围扩大一倍,到60,就是你如果把它详细的一步一步的解释出来写,用那种教科书式的写好的话,那只要对方是也有基本的逻辑思辨能力,那可以在智商相差到 60 点的时候还可以有沟通,那所以我说这个沟通第一个你必须要有那个专业知识,能够有信息不对称。第二个你不想要把它拿来兑现,你愿意把它公开,然后做成功。第三个你要有你已经是智商那么高了,你跟大众来沟通本身就有困难,但是你愿意来做那个努力,是跨越这个智商的鸿沟,智商差异的鸿沟来做这个沟通。好,这是第三,第四点你愿意做宅男。



唐湘龙 47:32 

这个大概符合你个性。



王孟源 47:35 

因为你必须要愿意孤独,因为你到处得罪人的话,你就没有朋友。你如果在乎是到处交朋友的话,你就不可以,不可能会习惯得罪人。你就不可能习惯说真话去得罪人。所以为什么这个世界上没有几个人能够像我这样子?



好,我现在要把你拉回到耶伦,就是有关于我们一开始在谈的就是说你为什么认为现在就是说国际 的金融问题不但没有解决,而且国际金融的危机正在长期化。耶伦在短短的十个月的时间里面两度到中国。在新闻当中看起来就是吃吃喝喝,然后丢一些,看起来好像在经济学 ABC 上面来讲,意在言外的概念,跟他现在特别谈产能过剩的问题。那但是你又认为其实耶伦的两次到访其实反映的是一些很深层很重要的问题,为什么?



这其实就是刚好我们已经快要谈到的事情?就是我警告了这个通胀,而且我给出了中国在金融货币政策上面的正确对应,但是那时候没有效果,然后我觉得这个专业性太高,所以我在体制内的文章里面也只是原则性的去讨论。OK,那原则性的讨论就是什么啊?比如说社会主义国家应该如何管理资本?这样的问题。嗯,OK,作为定调的问题,就是说你必须要以实体经济为主,金融是越简单越好,刚好足够勉强,足够服务,为实体经济服务,为实体经济融资就可以了。那这个都已经被认可,这都已经被中国的最高层认可了,批示过了。他,我两三个月前还, 3 个多月前,还是 4 个多月前,我在你节目上讲过,但他的内部讲话,习近平的内部讲话里面就很明确的遵循我的那种建议,然后在中央党校做了原则性的讲话,那照理说应该都没有问题,但是实际上执行的时候出了很大的问题。嗯,这很大的问题是什么呢?就是当时中央银行的行长易刚奋不顾身的把在美国通胀最严重的时候,也就是 2021 年年底,它让人民币贬值了百分之十几 14\% 的样。照理说当时人民币应该是升值的,因为你的出口货不供不应求的时候,全世界的金融教科书都会跟你讲,这时候你的那个货币应该是升值的。但是因为美国控制通胀的需要,所以要求欧盟的欧元跟人民币都贬值,他们两个银行就乖乖的都贬值了。所以美黎明这场通胀危机是美国乱映美超而引起的,结果最严重的地方反而是在欧盟,为什么美国的通胀这一次没有把它弄死?就是因为他们在通胀最高峰的那半年,他们的进口的工业成品的价格不升反降。那么。



这就可以让他们把通胀的压力局限在两个地方,第一个是他们的服务业,他们的服务业有 30 多年实职收入没有提高,而且服务业所占的人口比例在这 30 多年里面提升了一倍有余。它的重要性之大,那你一旦通胀开始了以后,这些底层的服务业员工,他们也就是底层的中产阶级,下中产阶级,他们终于有机会让看他们的实质收入能够往上提。但是问题是你在一个通胀危机里面,你把你的工资提高的时候,所有的价格都跟着一起提高,所以你的名义工资上去了,实质工资上去的很少,甚至会倒退,这就是通胀最可怕的地方。。



那另外一个是因为美国金融化以后,他们那些私募基金基本上把一些行业,很基本上所有的行业都寡头化了,寡头化以后他们的定价权就很高。所以你通胀上去以后,大家消费者最弱势,他们最倒霉。那这些厂商无论如何他们的利润成长率要高,他们可以维持高于通胀率,那这样一来他们又变成维持通胀长期化的另外一个推手。



美国的金融危机也就是这个通胀危机为长期化,基本上就是这两个因素,而且这两个因素都是我从两三年前就不知道讲多少遍,但是我一直没有批评易刚他 2021 年的时候,人民币应该升值 50\% 的时候,他把人民币贬值14\%,我没有批评他。我又等了一年,等到 2022 年8月金砖会议应该宣布金砖货币的时候,事实上我在那之前也写了一大堆文章,说这个货币不应该是金砖货币,而应该是一个亚元,应该是中俄,顶多再加上沙乌地两三国搞的一个核心的国际货币。但是你如果非要搞成金砖货币,不是最优的解,但是至少还是一个次优解,我愿意看到。



结果,2022 年8月什么都没有,我是从2022 年8月我才开始批评易刚的。这是为什么,很多这种网民,傻逼,他们有两句口头禅。第一句是实践是检验真理的唯一标准,这句话是绝对错误的。第二个是没有调查就没有发言权,这句话也是很容易被扭曲的。为什么?检验真理的唯一标准叫做科学。OK?科学它的真正定义就是求真的一切原则跟方法。OK,实践只是科学里面的一个选项,而且是一个很小很小的选项。我举个例子,当初我们中世纪以前有多少人调查过或者实践过,是地球绕着太阳转还是太阳绕着地球转?



嗯嗯,对,那我想请问你到最后解释了,为什么地球是绕着太阳转的人是谁?是牛顿,牛顿写出他的重力方程式,然后得出了行星绕太阳的这个轨道的解。请问牛顿有没有把眼睛拿下来看太阳?没有。对不对?因为这个所谓的调查跟实践并不是无脑的用手、用眼、用脚的调查跟实践,而是用脑子的调查跟实践。别人的观察一样可以变成你的调查,别人的实践一样可以变成你的实践。真正的关键是在于你的逻辑思考,在于你的逻辑思辨,这也才是科学的核心。



那我博文里面认定哪一个人是错误的,我都批判地不留脸面,但是我判断这个人是蠢是坏,这个判断其实都是很谨慎。尤其是中共的官员。我自己在投资银行当主管,当了十几年。投资银行也是一个官僚机构,它的规模还没有远远没有像一个政府那么那么大,但是我也知道里面就有很多复杂的人事跟政治等议题,所以我也知道你不能够从浮面的角度来妄自批评他们所做的选择。有的时候他们被迫不选择最优解,而选择次优解,这些你都必须要容忍。



所以我刚刚举的他的例子,他那个没有创立亚元,也就是我博客在 2020 年跟 2021 年所说的最优解,我都可以容忍,我认为他可能是因为政治需要而去追求次优点,但是他在 2021 年人民币贬值, 2022 年金砖会议什么什么国际货币都没有搞出来。这就不是最优解跟次优解的问题了,而是他选择了最劣解。对中国的最劣解,而且是对美国的最优解,那这个就变成一个很大的问题,因为美国其实在过去这 40 年变成一个金融帝国,他们的这个殖民体系演化成一个金融殖民体系。



那你在金融方面的出手基本上是这种斗争跟殖民跟反殖民之间斗争的关键战场。你在这个关键战场上的一个主驾,主动的撤手,然后投降,这是一个非常严重的卖国行为。严重到什么地步呢?我认为中国整个中华民族,自汪精卫以来 80 多年最恶劣的卖国行为就是易纲。所以我把他称为应该入嗣美联储的忠犬祠,我其实是心里有点觉得不妥,这个不妥不是因为过分,而是因为这么严重的卖国行为不应该牵扯到玩笑的成分。就是这种事情他是万死莫辞的罪行,罪大恶极的罪行。那就把他开玩笑说成是应该入嗣忠犬祠,这个是其实有点不太妥当。但是不论如何,我说李克强可能是一个好人,就是他很开明。



但是他任用易刚这种人是谁的责任?是他的责任,对不对?他追求的所谓的金融创新、金融开放这个策略责任在谁也是他的责任。所以他不是一个好的总理。他差点把中国带上一条不归路。我为什么说差一点?就是你换刚好及时在 2023 年初。换了新人。这个新内阁目前的金融政策走向还不能够确定,我不能够说他们会遵循李克强易纲的那条路线,或者他们能够有明确的看出事实真相跟对中国的最优的策略选择,而我们刚刚你提到的耶伦在 2023 年新内阁上来不到半年就急急忙忙的来一趟,然后隔了半年又急急忙忙的又来一趟,这代表着什么?她公开的讲的那些话,那些什么威胁,什么产能过剩,那个都是对美国国内的公关需要,对美国的国会议员的堂面上的说辞。她实际上这些话,她不需要来中国讲,她实际上来中国,我认为她的目的就是要摸底,要影响这些新的新内阁里面的经济跟金融管理,



尤其是我刚刚提到美国的这个通胀事实上是长期化的,当然我们现在过去这两个月,他们的所谓的 Service sector Core inflation 就是服务业的核心通胀了,涨到7\%。那 2021 年年底, 2022 年年初最高峰的时候,这个数值是到什么呢?是12\%。OK,那你不能够说这个 7\% 超过 12\% 的一半,事实上它只有它的一半,我跟大家解释一下通胀的平衡点,它的中线并不是零,通胀大家公认它的这个不高不低、不愠不火,刚刚好的那个点是大约2\%。



所以那个三年前是12\%,就是超过这个中线10\%,现在是7\%,就是超过这个中线5\%。所以目前美国的这个通胀热度大约是 3 年前的一半。那为什么我会专门看这个 service sector core inflation?节目开始之前我跟你解释了一下,就是通胀通常是有外来的暂时性因素,就是导火线。那导火线 50 年前是能源危机,我们这一次的这个导火线是,新冠危机,然后又接到俄乌战争。那导火线要碰到一些燃料,这个燃料 50 年前是美元超发,其次 50 年后还是美元超发。OK,那你有了燃料,然后又有导火线,它就开始烧起来了,烧起来以后它要这个助燃的,持续的燃烧的就是要氧气。这个氧气是什么呢啊? 50 年前是美国的那个工会,美国的他们强势的工会,就是你那个通胀上去了,我的那个薪水的涨幅必须要比通胀还要高,那么你通胀原本是5\%,我薪水涨6\%,那么明年的通胀就变成6\%,然后它的薪水再涨7\%,那通胀就变成 7\% 这样地节节上升。



美国的这个通胀长期化,这一次它的动力就是两个,一个是它的那个行业的寡头化,所以它的那个公司的商业利润要必须要紧追着通胀,另外一个就是服务业的薪水,他也必须要紧追着通胀,所以这两个都认为他们必须要比通胀还要高一点,所以你通胀就降不下去,但是问题是你没办法去看那个利润,没办法看行业独占性的利润,因为这个所有的那些相关统一数字,除了行业,他们那些行业因为独占性所以能够主动调高利润以外,事实上也都包含了。我刚刚提的新冠所造成的 supply disruption 就是供给面阻塞,那些供给电阻塞使他们的定价大幅调高,而且调高的程度是 50\% 60\% 十,所以你这些维持通胀的这种效应大概 6\% 7\% 反而看不出来,那所以你维持通胀的两个效应里面,那个寡头化的利润没有统计数据能够简单地、干净地看出来。但是另外那个服务业的工资有一个很干净的数据,所以我就很喜欢看这个数据。这个数据呢,我刚刚说从三年前的 12\% 降到今年年初,去年年底大约是5\%,然后过去这两个月又涨到7\%,那这个就代表着,这就让耶伦跟Powell有点紧张了,对不对?因为尤其是Powell已经承诺说今年要降级三次,现在追下来就很尴尬了啊。然后耶伦想一想,三年前我们那一场通胀怎么度过的?是因为欧盟跟中国牺牲自己的利益,不但不提升他们货品价格,反而贬值来帮助压低我们的通胀。那么最大的问题是什么?现在最大的问题是什么?中国有了新内阁?美国对欧盟,欧元货币政策的控制没有任何的改变,在过去三年没有欧盟的改变,但是对人民银行的控制却改变了,因为人民银行的行长换人。所以我如果是耶伦的,我也会急急忙忙的来看看这个新任的人民银行行长会不会像上一任一样那么乖。那但是反过来看,我认为这也有正面的意义。正面的意义是什么?就是这个新任的行长,他叫潘功胜,是不是?他如果是已经像易纲那样子决心要卖国的话,耶伦就不需要来了,对不对?耶伦会来的意思就是耶伦本身也不确定潘功胜会怎么做,嗯,对不对?所以我们现在积极的用舆论去谈这件事情是很有价值的。



因为不止我一个人,而且我刚刚提过我以前博客成功的影响政策,成功的利用舆论影响政策,都是有很大的偶然性,但是这一次我觉得是可以经过刻意的努力,然后在时间压力下能够有些成效,为什么?因为我也提到以前我要谈的主要是科技管理的问题,对不对?还有那个贸易战这种是国际认知的问题,这个不一定都会有很多成见,必须要克服。我刚提到像刘慈欣的那个假未来科技的那些风潮,但是有关金融误国,有关资本主义怎么压榨,有关殖民主义怎么祸害全球,这些事情刚好是大陆的基础教育,应该都有教过的事情。所以大陆的网民也许在其他方面会有一些反噬的景象,在这一方面,在我谈的这些金融,还有这个货币斗争这件事情上,反而是有很好的基础的,所以它传播地很快。你上个礼拜送给我一个大陆的做的视频,对,他也是开始跟着谈这些方向,其实那个人纯粹是所谓的 Retail 零售的,就是他也是跟风。



唐湘龙 01:09:18 

转发观点了。



王孟源 01:09:20 

转发观点,那上个月有人在发问的时候提到卢麒元,像卢先生,他也是跟我同样的看法,我们其实有交换过一些,间接的交换过一些意见,就是其实读者会把他的意见拿来我博客发文。



那所以我觉得在这件事情上是事有可为。那我的我从去年底写了一系列金融史观的文章,应该把所有的基础的学术根基都已经打好了。论证都完成了,现在的问题就是传播的问题,所以,而且传播你就像你刚才连那些基本上跟风的那些大 v 还有网红都已经开始做这些,而且有些人做讲得很好。嗯,你给我的那个视频我去看了一下,我觉得 so so,就是内涵,不是特别的impress,但是看完之后 YouTube 就建议了另外一个类似的网红视频,我觉得反而是做的蛮不错的。



那个然后我又看了,我就再多看了一集。我基本上只看了两集。我觉得两集的话不能够说确定他的水准很高,但是至少我看到了两集水准都不错。那个人叫做,你的充气小白兔,一个 20 多岁的小女生,小丫头。讲的头头是道的。我想这才给我希望。就是说你连这种 20 多岁的年轻人OK,都能够把这些道理讲得这么通透。那么我们用舆论的压力来希望潘功胜在未来面对抉择的时候能够选择正确的路线,就有点希望。OK。



唐湘龙 01:11:26 

嗯,因为时间的关系,可是刚刚有一条突发的即时新闻。那你现在如果在美国,你打开电视, ABC news 在这个,刚刚,因为我也,我在海外的朋友,刚刚在中东的朋友也发东西给我。他说以色列对伊朗发动了反击行动,那其中的攻击的地点包括伊朗境内的核设施。所以你现在看到的全球的股市的、国际的金融部分,第一个就是说油价大的大涨,瞬间亚洲盘的暴涨3\%,金价也涨破每英两的两千四百美元。好,那现在股汇市现在几乎都在非常混乱的状态,我刚刚看了一下,台股现在暴跌 700 多。



好,所以这个就以色列反击之后,大家就看到伊朗会不会再反击,所以中东的情况似乎又进到了一个非常的相互报复的恶性循环里面。拜登即使出来劝以色列说不要,不要进行报复,看着看着没有用。难道第一个就是说难道之前以色列,难道之前伊朗对以色列的报复是有效的吗?第二个这种的反报复的行动,中东的情况是不是就没完没了?这件事情我不知道,孟源。因为不在,我们今天的准备范围里面,你有没有什么家乡吧?



王孟源 01:12:43 

我上个礼拜在我的博客上回复过了,所以我简单的,一个礼拜前的回复还是有效。那我就把在这边简单的复述一下。就是基本上你如果回顾一下这件事很简单,是以色列先动手,他先去炸在叙利亚的大使馆。



那伊朗在过去这两年俄乌战争的期间有一个很坚定的原则,就是他不愿意站到第一线去跟美国直接对抗。美国也有一个很坚定的原则,他也不愿意站到第一线跟俄国或者伊朗对抗,嗯,对不对?那在这两个原则之下谁最不乐意?以色列最不乐意,对不对?嗯,因为除了以色列以外还有一批人不乐意,就是 CIA 很不乐意。因为你这个,如果大家世界和平了,以后还有你 CIA 什么事情?你要钱没有钱,要人没有人。



所以 CIA 才会去安排莫斯科恐袭,对不对?所以以色列才会去主动打击伊朗来挑衅。那你看,这次伊朗的反击其实是拿捏的刚刚好,他知道到什么程度,美国不会参与,OK?然后到这个题到了这个地步以后,又变成过去这几天基本上是美国跟以色列之间的内部交涉,就是美国跪着要求以色列不要再升级,以色列可以随便开价。那以色列就真的随便开价,然后美国付不出来。你如果因为现在这样打,再继续升级,就代表美国付不出与色列内塔尼亚胡要的价钱。内塔尼亚胡要的价钱是什么?很明显吗?就是你不要这边,你就我希望你出兵,你又不想出兵,你不想出兵可以,你一定要在宣传上、外交上跟金融上跟经济上,全力的支持。OK,那事实上拜登已经左顾右盼,因为他自己的大选情已经很不利了。Trump的一个选举的那个宣传的重点就是他反战了,对,对不对?那所以Biden其实对内塔尼亚胡颇有微词,那这一次以色列如果升级的话,以色列升级的话就代表内塔尼亚胡没有拿到足够的代价。这代价是什么?拜登的无条件支持就是以色列跟美国人讲的。一定是我需要你的无条件支持,除了军事。直接参与,你如果不想直接参与,你就必须要在军援还有金援,还有外交声援上面无限制。可是你看过去这一两个月是什么?他们美国甚至在安理会跟联合国大会有一次反过来投战争票,站到另外那一边了,站到巴勒斯坦那边有一次,那这个就是以色列要拜登承诺绝对不可以,要完全反转。



而且上个礼拜内塔尼亚胡国内最大的政敌,到美国去转了一圈,不晓得你有没有注意到,还到国会去拜访了一道。内塔尼亚胡的要求也一定是你们不可以跟我的政敌这样子私下地眉来眼去,对不对?这个从头他没有拿到以后他就再升级,因为他不在乎你,即使升级到第三次世界大战,以色列也是唯恐天下不乱,同样的 CIA 也是唯恐天下不乱了,对不对?你这个现在眼看着那个俄国在乌克兰打的越来越顺。那现在美国跟欧洲都有点要退的后退的感觉。昨天 Fox news,美国的最强硬的国家主义的那种共和党媒体,站出来说,乌克兰已经要输了,我们是不是该未雨绸缪?



唐湘龙 01:17:26 

终于有人愿意把这句话讲出来了,哈哈哈哈哈哈。



王孟源 01:17:31 

对。那在这种情形下,你这个 CIA 在不赶快中间调一调的话,这个下面玩不下去了,所以他们才会安排那恐袭,安排恐袭这个所谓的阿富汗的ISIS,这个一个 ISIS 在阿富汗的分部,一个所谓的伊斯兰恐怖组织,你看看它成立这几年以来打击的对象都是什么?塔利班,叙利亚,伊朗没错,伊朗的那个大爆炸,没错,游行的时候的那个大爆炸,然后现在莫斯科空袭。好奇怪,一个伊斯兰恐怖组织从来不攻击美国,也不攻击美国的盟友,他专门打美国的对手。对,那不是很明显吗?这个就是 CIA 的白手套。所以我们现在看到的其实是美国顾此失彼,因为美国的这个霸权,对世界的掌控,在中国的工业跟经济能力跟技术能力兴起以后有了替代能力,然后又遭受俄国在硬实力上面的直接挑战。给第三世界一连串的机会来钻漏洞,比如说我们去年有谈过Niger的那个政变也是这样被鼓励出来,然后接下来巴勒斯坦的攻击也是这样被鼓励出来,然后Houthis封锁了,红海也是这样被鼓励出来,这样鼓励出来以后你想想看以色列跟CIA,它名义上是在美国那一边,但是他们也是希望能够有动作,希望在这里面摸鱼的,对不对?那既然你的美国老大的控制已经松开了,那对方都已经,在我对面的那些人都这样动手了,我自己为什么不也进去摸摸鱼呢?我在看到的就是这样子。



唐湘龙 01:19:37 

好。因为今天,今天耽误了孟源的非常长的时间,那尤其是中东的情势,现在是 breaking news,我们我还没有办法拿到全部的一些比较可靠的新闻的细节,以色列到底供给了哪一些的地点?里面中东的讯息已经说包括了伊朗的核设施,但是到底动用了哪一些的武器,然后供给哪一些地点?然后有没有造成伤亡?以及在国际社会当中对这一起的以色列的再攻击的事件的反应,以及后续这些都还有待观察。



好,非常感谢孟源的今天的一个多小时的分享,孟源,孟源把他的这十年,他对于他对整个的一个比较的长线的一个架构的思考做了一个把它捋捋清楚,一个比较清晰的脉络,当重点是在当下要继续往前推进的时候,该怎么做一些很正确的政治选择?这个是孟源想要做的,那我们感谢我们一些的观众朋友,那很专注的参与,近万位的观众在我们的线上。



唐湘龙 01:20:41 

bro K 0011,感谢那杜拉 8 肉,感谢,欢迎王孟源永远在这里发言 90K 谢谢你再来苏战孙那听王博士上课非常重要。那谢谢王博士再来 GM 组王,感谢王博士的贡献,赵星云感谢他说请教王博士两个问题,欧洲大型强子对撞机和大陆大型强子对撞机的有什么不同? Google 的量子计算机和大陆目前量子计算研究方向有什么不同?赵俊,我把这个问题保留到下一次,好吗?因为这个问题比 w 大一点。



王孟源 01:21:20 

我可以简单的回答一下。



唐湘龙 01:21:22 

你请说。





王孟源 01:21:26 

我刚刚提到那个悟空卫星,那个几千万嗯的预算,他们造出一个卫星来,这是几百个人花了好几年建的一个卫星。你平均的话每人每年还不到 10 万块钱。OK,你基本上那个钱刚好是足够维持那个团队,我认为那是正确的。大对撞机那个高能所用的团队是不是应该每年拿个每人,每年每人拿个几万块的研究经费,或者 10 万块左右的经费继续去做他们的研究?我是说他们不能够几万亿的去诈骗公费,因为中国没有那么多研究经费,使劲地达打到一个小行业里面去。没有错,我并不是说他们不能够继续做他们的研究,我并不是说国家应该把他们通通开除,OK,这个维持这个团队这个标准是很低的。 40 年前中国的财政非常糟糕的时候,邓小平面临一个选择,他必须要在运10大客机跟北京的对撞机,当时要建一个小型的对撞机,但这个对中国来说一个非常强大的负担。事实上那个对撞机的价钱比维持运10团队的那个价钱要高。结果邓小平选择了建那个对撞机,是我认为他在科技方向、政策方面最大的错误,他一辈子最大的科技政策的错误。就是当时如果维持那个大客机的话,不是说……运10本身没有什么经济性,你不需要量产,但是你用比那个建当时那个对撞更少的钱就可以维持运10的团队,那个上海团队是比现在高能所那个团队的价值要高几万倍,OK,而且花费要更少。



同样的,欧洲建那个强捕对撞机。雇佣了 6, 000 个博士级的研究员。你说这有没有影响到他们电动车的发展?如果他们有 6, 000 个博士投入发动发展电动车,有那 2, 000 亿欧元投入电动车的发展,你说他们现在会不会面临面对Tesla跟中国电动车浪潮的全面溃退?不会这么惨。嗯,对不对?那现在他们这个强子对撞机除了拿了五篇那种顶尖期刊的论文之外,头条论文,拿到什么? 2, 000 亿美元, 2, 000 亿欧元。



去买了三四篇论文,这个是我反对的。同样的,量子通讯跟量子计算,你每个人每年花个几十万来继续做研究,维持那个团队。绝对赞成,OK。这里的问题是,你不能够拿这种完全没有经济产出可能的东西去股市搜刮股民的民脂民膏几千亿的这样搜刮。因为中国的股民已经够惨了。你的监管单位还这样子放纵这种泡沫公关,无中生有的未来科技。你这是影响中国未来金融资源配置一个很糟糕的事情,更不要提对科技的那个风气影响有多大?而且我自己亲身的经历就是。袁岚峰这些人他们都控制了中国国内的所有媒体,官营的、民营的,包括像观察者网,你都没办法引用国内国外专业非专业的任何的批评。我差点就被封杀了。我如果 7 年前, 8 年前敢站出来批评量子通讯,就会被封杀。我那时候如果被封杀的时候美国陷阱。那本书就不会被介绍到中国。你想看这种年代的影响是确实的。这种恶劣的行为,这种恶劣的风气,就是有很多隐性的、间接的恶劣影响。



唐湘龙 01:26:00 

没错,对,好,来,我们再来把一些,因为时间的关系,我没有办法让就是我们的听众朋友的问题都得到了清楚的回复,有一些可能要留到下回,那 summer 一样感谢。他说王博士你好,你的博客及节目中的论述都非常有道理,不是故意为难,而是心中有个疑问,如果美国状况那么差,为什么包含你在内,还是有这么多人想去美国求学、工作跟定居呢?好, 孟源你要简单回答吗?



王孟源 01:26:32 

我来美国的时候是将近 40 年前,那个时候美国的腐化才刚刚开始,它的这个社会,第一个,你做同样的工作,你的薪水收入要高一个数量级。那你的选择是什么样子?第二个对我来说更重要的是中国是一个人情社会。台湾也是一个裙带社会。台湾的那些小中小企业都是家庭性的。你如果一个外人去帮他们,纯粹就是打工。在美国,当时还没有这种深刻的普遍的人情主义裙带主义。我可以凭自己的本领,就是所谓的 Meritocracy。靠我自己的努力来创造所有的美国梦,就是当时的美国梦还有破没有破灭。所以我那时候选择没有回台湾是主要原因是我不想去一个人情滋味。我说过我是一个宅男。我不喜欢讲人情。



唐湘龙 01:27:35 

OK,好,来,感谢。再来我们看孙情者,孙情者提到说中医怎么会是迷信呢?迷信中医和迷信科学宗教虔诚有什么差别?他说我是中医的受益者。



王孟源 01:27:47 

好,我讲给你听,中医这件事情,有人说我既然不用中医,我就没有调查过。我有调查过啊,对于中医的功效刚好是世界史上最大的科学实验。为什么这么说呢?中国历史上 2, 000 年的王朝历史,用了 2, 000 年的中医,几百亿的人口就用过了,当时的人民的平均寿命是什么?我们不要管那个底层的劳动人民,他们是有营养不良的问题,所以他们的人均寿命不算。



一个普通的中产阶级开店的店老板能够每个礼拜吃肉的那个不愁温饱。平均寿命 50 岁出头,是,你现在中国的平均寿命是多少?将近80,提升了25。不止是这里面,如果说中医是那么了不起,你为什么人均寿命会少25\%?这个实验的取样数目百亿,你说世界上还有比这个更大型的实验吗?然后我再问你一点,当时能够享受中医的,最低级就是中产阶级。然后在上一期贵族阶级,最高层是皇室,那么皇帝想用的中医资源一定是最高的。皇帝的平均寿命有没有比中产阶级高?没有,反而更低。而且你直接去,而且我是说你那种因为兵变而被砍头的政变,被砍头的人不要算那个,因为疾病死亡的,而且这些皇帝有很多是二十几岁、三十几岁就死的。而且历史上明明的注载是吃药死是,我再说一次,中国上明文记载吃药吃死的皇帝几十个,OK。你即使到明清两代,明朝最有名的明臣贤臣张居正,吃药吃死的。刚刚提到清朝最好的皇帝雍正。你不要去相信什么野史,什么吕四娘什么的。他也是吃药吃死的。你只要到历史系里面去找一个熟悉清史的人问一问,就知道他死之前一个礼拜他刚刚进了一批炼丹的材料,几斤的铅汞什么的是,他刚刚练了一批汞,然后练完几天之后他就死了。



唐湘龙 01:30:41 

好吧?那个那那在一个相对愚昧的环境里面。好,再来了, grand 怕谎。感谢零零张,感谢他说这个世界清醒人不多,清醒了又有良知又有勇气,大成几乎的。嫂子又嫂那个感谢孟源好 on so 哈,感谢他说谈到就是有关于有关于华人的社会的一些的陋习,人情贿赂等等,造成社会腐败如何?好感谢。他说中国需要建立一套新的哲学和面对过去的这个社会体制的疗伤。



唐湘龙 01:31:15 

好,那 to s k 感谢。他说不明白为什么有这么多人会冷嘲热讽,有意见你可以提出来,不喜欢你可以无视,不喜欢又非要炒,那是犯贱嘛。我想大概这个 two s k 平常在潜水,但它是针对一些的留言板的,一些的流化。 Johnson 无感谢,他说如果人民币开始升值,立马落井下石的效果大概是没了。其他方面还有积极的影响吗?好的。



王孟源 01:31:45 

OK,这个我要,我想评价。中国的人民币不能够自由兑换。因为它不能够自由兑换,所以它的汇率跟利率是可以脱节的。你的中国的目前的国内就业有点问题。有点通缩的问题。你完全可以调低利率,然后调高汇率。嗯,同时调高汇率。为什么呢?因为这个所谓的利率,而汇率取决于利率,这是信用货币自由流通的信用货币的特,对不对?好,我问你一件事情,湘龙,我问你一件事情。请问黄金,你黄金的利息是多少?



唐湘龙 01:32:31 

没有利息,黄金没有负的。



王孟源 01:32:33 

因为你还要付保管费。你如果买一吨的黄金的话,你还要付很多的保管费。为什么在这种通胀的环境之下,负利率的黄金它的那个价值上涨了20\%,利率跟汇率完全都是两回事。为什么黄金会涨?因为美金实在是太弱,它的超发的实在是太厉害。没错,大家买黄金不是因为它有利,而是因为认为十年之后它的价值会更坚挺。同样,人民币根本不必管说你的利率要高还是低,你要调,国内的经济需要调低就调低,但是你同时可以对外宣布,我在 2030 年之前,我的人民币会稳定的调升到 5: 1。对美元调到,你只要斩钉截铁地承诺会做到这一点,市场就会保证自然地会把你推到这一点。



唐湘龙 01:33:39 

嗯,好,来我们再来。看到这个利欧玛是第一次在 YT 里面送出超级留言的,感谢,那他说第一次的 FIR first name 的 to MR to doctor Wang,像王博士多上节目,他说每个月一次看王孟源是不够的。好,再来这个林 space 感谢,然后如米路感谢。再来这个的黄露露,感谢给王教授,我给王博士打call。对,感谢。那么丽娜张,感谢这两位金 p 的分析,再来云霞有感谢,好,感谢我们所有的听众跟观众朋友,今天,今天孟源竟然被我呢绑了一个半小时的时间,他那边都已经是半夜了。好,不过刚刚除了我们谈的两类两大段的重要的主题之外,另外请大家留意一下中东的形势有一些的新变化,那这个是刚刚最及时的新闻,感谢在周末之前,星期五的时间,在美国美东的时间的深夜的时候,线上为大家上课的往孟源感谢。好,下回见。拜拜,周末快乐。好。



王孟源 01:34:48 

谢谢,拜拜,下周毕见。拜。



\twocolumn[\begin{@twocolumnfalse}
\section{解讀美國「深層政府」如何重傷這個世界!}
\subsection{20240524}
\end{@twocolumnfalse}]唐湘龙 00:28 

龙行天下.好,欢迎来到龙行天下来,今天星期五的时间 9 点半钟的时间,今天是王孟源的时间,来在我们线上的人在美东那几乎是日夜颠倒。但是每次我听他讲话的时候,我是,我这不是客套,我是真的洗耳恭听,那尤其今天我们在关注今天的主题的时候,他有机会让我们对今天的全球政治的基本的问题,就是美国是如何走到今天这个地步的,那有关于新保守主义的 Neocon 的这个概念在过去的王孟源的为大家的讲座——对我来讲是一个很好的讲座课程——在这讲过课程里面的时候,王孟源在提到的时候,仿佛把它当做是一个人尽皆知的概念,好像大家都知道 Neocon 是什么东西,然后就带过去了,但是其实大部分人对于对今天美国为什么会成为今天,感觉上面是一个有一种全球干预,企图急于把自己的价值观、价值系统铺设到全世界,顺我者生,逆我者亡的这样子一种的意识形态到底是怎么形成的?其实是没有什么概念的。好,今天有机会可以让大家回到问题的本职去,好好认识这个问题,那包括美国的在总统选举的过程当中,你看到的川普那所代表的右翼不断地在指控,就是说美国其实被一个 deep state 深层政府所控制,那这个指控到底又意味着什么?这几个概念之间其实是有连通性的,而真正的连通的是你如何去理解今天乌克兰以及在以色列以哈的冲突当中,中东跟俄乌战场这两个战场,这两战的根源其实跟Neocon的意识形态是有高度的关联性的。



唐湘龙 02:27 

如果你不理解这件事情,你就没有办法理解今天的国际的冲突看起来是地缘政治,但是本质上面是美国政治,甚至于包括以色列。那今天在我们线上的王孟源欢迎,(王孟源:大家好)。孟源,今天如果有时间,最后面的时候我们再过来分享一下他对他的减肥经,这个其实对我来讲可能是很有吸引力的,不过我们要严肃一点。我们先来关注一下。今天的话,我们从两个主题开始,第一个孟源,孟源在昨天我们在讨论的时候,孟源提到就是说其中第一个就是美国的共和党,美国共和党明明在选举之后,虽然他上次的其中选举也不算选的好,因为没有预期中好,以至于但是他终究是掌控了众议院。可是众议院,我们看到他众议院的议员一个一个辞,尤其那叫盖拉格的在请辞了之后引起了很多的讨论。那这个讨论就使得今天,今天在美国的国会里面,在处理有关于乌克兰议题的时候,最后你会发现共和党好像守不住城池,终究对乌克兰的这 600 多亿的援助法案似乎抵抗不住拜登的这种所形成的舆论压力。放水让它过了。为什么?



王孟源 03:44 

好,这件事情其实要从去年 10 月开始。问题那时候发生了两件事,第一个是拜登政权准备援助乌克兰的最新一批 610 亿美元的法案,这个呈交国会,然后因为美国的大选也开始热闹起来了,当时的共和党的民粹派就抵制他们的,当时的众议院院长McCarthy是建制派的。所以最后就把他给罢免掉了。罢免掉以后闹了一阵,到最后第四个候选人才脱颖而出,也就是后来的Mike Johnson,虽然不是一直高调的民粹派,其实他是倾向同情民粹派的,所以最后他才能够胜出啊。就是正因为民粹派的人对他有信心,而他历史上又没有得罪建制派太过分,就是他没有太高调,所以就变成一个妥协的候选人,就胜出,变成众议院议长。但是我们要就像你刚刚讲的,美国在2022 年期中选举的时候,共和党选的并不好,他原本大家以为他会在众议院大获全胜,结果只是选胜的那个比例是多少呢? 222 席对 213 席。OK,那美国的。这个众议院它的过半数的标准是217,就是这一届,每一届都不一样,但是这一届是217,所以到了去年 11 月闹了这两件事之后,国会发生了:进一步在众议院,这个民粹派就真正掌权了,在那之前是由建制派的McCarthy当议长,对不对?,后来换成了Johnson,就变成民粹派的人掌权了。那民粹派掌权以后就把这个援乌法案给拦下来了。



王孟源 06:00 

那因为这个是大选议题,所以对他们支持Trump的来选举是有用的。是有一致性的,可以鼓舞他们的选民的士气。那这时候连带的发生了几个巧合,第一个巧合就是什么?McCarthy就辞职了,McCarthy灰头土脸,那建制派的幕后就是美国的那些深层政府。那些大资本财团,那这些大资本财团,你们这些政客其实都是为他们打工的,打完工以后退休以后当然都是要安排好出路,当然这个都是美国的这种制度安排是完全合法的,你只要没有签订合约,是大家心知肚明的,这种非正式的惯例的话就是完全合法。你必须要签了合约,而且签了合约以后违反的法律可能是税法,就是你,你要报税,美国的这个法律的概念跟我们中国人这种政府的治理,还有那种士大夫的对朝堂的责任,他这个是完全不一样。那不论如何了,那个时候刚好McCarthy就决定去拿高薪,跟建制派的背后的财团拿高薪,他就退休了,退休以后刚好还有另外一个丑闻,叫做Santos,一个共和党的国会议员,他闹了丑闻,所以就被开除了。然后又刚好又有一个人退休离职,这是 12 月底的时候离职,所以到了今年的1月 1 号的时候,美国的共和党跟民主党的那个众议院席位的对比是 219 对213,那然后这个援乌法案就越炒越厉害。在我想大家如果回顾一下,今年1月到3月的时候,可以说是大新闻,已经变成拜登政权他们主推的最大的一个政治project,一个他们的那个政策的方向,执政的重点,那结果就卡在众议院上,那这时候很有趣,你如果回去看的话,就在今年3月,一连串有两个众议员,共和党的众议员就一下子辞职。就是没有什么真正的理由就辞职,而且你想想看,美国的众议员任期只有两年。所以一般来说很少有中途提早辞职的。对,反正你做完了,两年一下就到了。而且当你的党的多数是这样岌岌可危的时候,你即使是要急着退休,你也可以说我宣布大选后退休,不必要说大选还有半年的时候就急着退休嘛,对不对?所以这很明显的就是建制派收买了这两个人,就是从McCarthy退休受到启发,然后看到只需要再两票就可以把它压到217,所以3月初退休了一个,然后3月 22 号又退休了第二个。那这样子一来,共和党的众议员就只剩下 217 位。这是最低过半数的最低标准了。他再掉一个他就不是多数党了,就变成两个少数党了。那在这种情形下,你想想看。



王孟源 10:04 

我在看到新闻的第二天,刚好我去大陆的一个学术座谈会,我就在这个座谈会讲了,说大家不要看这个援乌法案,炒了半年多哈,看似那个共和党说什么都不让他过,但是你看到昨天他们又有一个人莫名其妙的忽然辞职,让他们的多数已经掉到没有任何缓冲的余地,随时都可以变成少数。那这个时候你应该可以看出这个这里面很可疑,这个这些离职的众议员很明显是被建制派收买,来被民粹派施压,共党民粹派施压。



王孟源 10:50 

那所以你必须要考虑有可能这个援乌法案会有变。这是3月 23 号我上学术座谈会的时候讲的。那果然到了4月 20 号Mike Johnson就妥协了。那从 4 月 20 号他妥协援乌法案过关之后,在那之后他再也没有提任何民粹派的阻挠行动,都是非常的配合,而甚至上跑的比建制派还远。就是你看现在以巴的这个冲突,美国政坛的几个大咖里面最支持Netanyahu的是谁?就是Mike Johnson,那所以很简单的你就可以看出。这是他们建字派,也就是深层政府的那些大财团,他们对共和党民粹派这一届的众议员釜底抽薪,把他们的所有的权利都阉割掉了。也就是说去年 10 月的时候民粹派做了一个反扑,就是他们做了一个革命,把McCarthy,把他们的那个总督,把建制派来的总督也就是McCarthy给革掉了。但是革掉之后五个月,建制派就把这些这派又打服了。那打服以后一直到我们就基本上现在到大选之间,美国的。这些。这众议院的这些民粹派就没有什么搞投,大家可以,所以热闹,也就是热闹了五个月。



唐湘龙 12:36 

人家所以等于前后,前后等于在共和党的众议员,一共有 5 个离开了众议院。



王孟源 12:44 

对,其中有四个是辞职的,一个是因为贪污案的被解职的。



唐湘龙 12:52 

这算是这种的政治运作。听起来很丑陋,很肮脏,不过他就是这样子公然的,然后也就也是堂而皇之地运作出他要的东西,而这政治人物,尤其是盖拉格,他在离开的时候,我那时候就很好奇说他到底要干嘛呢?他是要参选吗?还是要怎么样?就看起来不过就只是所谓的建制派的收买的对象而已。



王孟源 13:17 

对,其实他们都是打工人,对不对?建制派的这些政客都是打工人。那他们……这个我觉得。这一次最重要的是所谓的自由媒体。没有一个人分析这一点,我在英文媒体也没有看到分析。



唐湘龙 13:42 

没有错,所以说你刚那个分析是非常好的一个分析,我在听完之后我觉得OK。



王孟源 13:48 

很奇怪的就是。你这些主流媒体,他们只报某一个共和党议员离职,可是明明这一离职,共和党的多数党低位就岌岌可危了,但是没有人做进一步分析,这个就是他们美宣掌控这个掌控话语权的一种话语术。很明显的道理,但他不讲,一般的民众就不会自己想到。对不对?它很明显的事情,路上有一只死猪,但是他只要不讲,大家都看不到。我觉得这就是现代社会,现代的工业社会,那个掌握媒体的话语权太重要了。你在这件事里面我觉得最吃惊的就是最重要的一个观察点是他们的宣传点,他们并没有说掩饰,并没有说这个共和党的众议员辞职,所以我们不能报辞职,他慢慢报了。他只是不跟你讲这个辞职实质上的政策影响,他也不跟你把这件事跟援乌法案联系起来。



王孟源 15:09

所以,他靠的就是一般的民众不会想到这些事情。那他只要记者编辑不跟你讲这件事情,他甚至不需要说谎,在实质上就已经说谎了。这也是为什么Trump在上一个任期的时候一直讲Fake news,很多这些费用是直接撒谎没有错,但是有很多时候他是根本不用撒谎。他是跟你明面上,报道了,但是深层的联系他不跟你讲,以后就等同没有发生。为什么?因为民众自己想不到。



王孟源 15:52 

那其实这是资本主义社会的一个自然现象,因为所谓的资本主义衍生出民选体制,这个它里面的基本逻辑脉络就是因为资本原本就不是唯一的。你如果是封建主义的话,大家都知道你的老板是谁,但是资本的话就是有一个好,有很多财团,你每一个工业都有一个大财团或者两个大财团,对不对?那你这样一来以后。就有财团本身有自己的利益,然后行业本身所有财团又有公共的利益,对不对?美国演化到 20 世纪以后,它最重要的几个利益集团是什么呢?最大的就是金融,其次是军工,还有能源,这三个是最大的,然后第四大就是医药业,OK,然后等而次之。然后其实我们最常看到的那些很有名的创业者,像是Elon Musk,他们的政治力量非常的薄弱。Amazon的老板 Bezos。是美国前十大富豪之一,而且已经有钱了三十几年了。但是他为什么去买华盛顿邮报?因为你这种新福的人,没有那些,你不是光有钱就行了,你必须要把这个钱收买,足够的打手,足够的势力,这个都需要时间、需要努力、需要运作、需要投资。像很有名的那个Koch 集团,Koch Brother他们是从 60 年代就开始花大钱,一直到现在,六十几年,他们才算是一个势力。那当然你要成为一个政治势力。一,你现在忽然你手头上有几百亿美元,那你要参与这个政治,美国的这个政治分赃进入这个政治平台,成为深层政府的一部分。



王孟源 18:17 

深层政府的来源有投资者,就是资本。那这个资本,新的资本形成之后要参与这个俱乐部最简单的办法就是收买媒体,我突然刚刚到这个媒体掌控选票是非常影响民众的意向,是一个非常强大的力量,这是为什么? Bezos 去买 Washington post,他买了 Washington post 不是把他自己手下或者他自己去经营,而是雇了一个建制派的。既有的政治打手来帮他经营,你就可以看得出他这个是它要加入建制派的一个投名状。那另外一个你可以自己成为深层政府的人,就是你当他们的打手,然后慢慢的在这个打手之中脱颖而出,成为有最多人脉的。



王孟源 19:24 

因为毕竟是人的社会,对不对?美国虽然不是像中国这样的人情社会,但是你到最后还是Rolodex,讲政治的话还是要看你的那个Rolodex 中文怎么讲(编注:对应的中文词实际上是名片盒)。就是地址部那个电话号码薄,你那个总有的交情的那个人脉的宽广。



唐湘龙 19:50 

就你的通讯录了,你自己手上的通讯录,那通讯录里面到底是哪些人?大概就知道你这个人的社会地位了。



王孟源 19:58 

你能够拉的人,你要拉;不能够拉的人,你要能够吓得住,这个在过去这 50 年,这一套玩的最利落的就是Clinton 家族。OK,所以 Clinton 才从一个政治打手起家,然后做到总统,然后退休之后他跟他老婆都同样成为深层政府的一部分。,就是你要是能够真的做的好的话,然后有足够的人脉的话,那你就是影响力,这个影响力就会——因为毕竟你这个资本他们是很松散的,而且甚至是有冲突的,能源跟一些生产业的财团是有冲突的,能源价格升上去以后,比如说民航及产业就不利。所以他们并不一定都有共同利益,所以中间有很多的折冲。所谓的民选政府其实是一个让他们对外搜刮分赃,但是对内从解决冲突的一个谈判的平台。那在这里老百姓的地位是什么?老百姓的地位其实就是牛羊的地位。那驱使牛羊的牧羊犬就是你们这些媒体人。



王孟源 21:34 

但是当美国是世界霸主的时候,他可以对全世界搜刮,这个时候他不需要搜刮自己的选民,甚至可以分一杯羹。比如说当年波音独占全球的民航机事业的时候,你这些民航机都可以用很高的价钱卖到全世界去。那你这些员工都可以,你都可以舒服的领高薪。那你只要能够确定说全世界只有我一家能够造得出这种飞机来,我就能够继续的以高价垄断这个市场。那你就可以看出为什么啊?在过去 20 年,美国对中国的态度急转直下,因为中国最大的影响就是它崛起之后提供了一个对美国的替代,替代什么呢?替代美国的资金,替代美国的市场,替代美国的技术。替代美国的武力,等等



唐湘龙 22:45 

没错,让美国赚不到钱了。



王孟源 22:48 

那美国没办法垄断高端市场,然后从中用简单的收购,那这样一来他们的很多的内部矛盾以前是可以用简单的撒钱来平抚的,我这个说的是比较直白一点,你东西的好坏其实跟价钱是非常重要的一个因素。比如说你买一台电动车,现在美国人不想买电动车了,过去这半年这个美国电动卖不动。为什么?美国电动车一台动辄5万、6万,没对不对?但是同样的品质的车在中国卖起来相当于十几万,二十万美金,(编注:此处应为口误,应该是指一万多,两万多美元)就是只有 2/ 5 的价钱。那中国的消费者当然认为这是很划算,但是美国的消费者当然认为这很不划算,对不对?你这个车同样一辆车。好不好?卖 60 万的时候非常的不好,卖 20 万的时候就很好,对不对啊?那现在的美国人就是以前他们社会里面的那些矛盾,体制上的一些很不合理、很大的缺陷通通呈现出来,并不是说他们是新的,而在于他们没办法给所有的一般的民众也分赃,所以没办法满收买,平抚他们的情绪。



王孟源 24:34 

那这个的问题你如果溯源一直溯回去,可以说是在是美国罗斯福新政建立一个社会主义政策之后的一个反扑。从大概 50 年代开始,他们先从经济系美国芝加哥大学的经济系开始酝酿,开始要发展一套理论说辞,要来把他们抹黑,把这个所谓的罗斯福的新政的那一套抹黑掉。那就说那这个就是新自由主义的萌芽。



唐湘龙 25:17 

好,接下去呢。对,因为你刚,你刚,你刚刚讲这段是非常非常重要的一段,就是把一些从一开始的时候,当我,对于为什么美国最后众议院会放手让拜登所包裹的这套的法案竟然能过关,因为川普的态度很清楚,我一上任之后,我马上结束俄乌战争。所以基本上面一直在反对,就是说对乌克兰继续在砸钱,可是第一个美国如果不砸钱,那你可以想见欧洲的,欧洲体系也不会跟着砸,因为你美国都不进来了,那我还玩什么呢?最后这笔钱终究下去了。



唐湘龙 25:56 

下去我就很好奇说这共和党为什么放手了?好,但是孟源刚的那个分析,我就让我们看到背后的story。好,但是你刚刚其实已经反反复复谈到了,就是说深层这政府跟所谓的新保守派的这样的一个概念,这个概念其实延续了二三十年之后,并没有人真的去厘清美国的这个 Neocon为什么今天会成为一个深层政府当中的支配性的意识形态?它是怎么出来的?它是怎么控制的?但是今天我觉得孟源的准备是有机会让大家一次理解,因为我常听到我们的听众朋友在问这个问题,我们去如何去解?去理解Neocon到底是怎么出来的,为什么今天对美国的政治有支配力量。



王孟源 26:41 

对,就是这种财团,人不为己天打雷劈,这种自私自利的态度玩到最后一定是自我毁灭。那他们上一次自我毁灭是什么时候呢?是 1929 年。那个大萧条,OK、那大萧条之后他们自我毁灭,但是最倒霉的还是底层的人民,但是呢,这时候美国因为还是一个新兴国家,里面还是很健康,有很多反思的力量。尤其是罗斯福刚好掌权,掌权以后,他是一个很伟大的政治家,他就以美国人能够接受的方式导入了社会主义,然后又不把它叫做社会主义,偷天换日地导入了社会主义,这就叫做所谓的新政。那这个新政不断的演化,继承之后演化、从 30 年代开始一直演化,它的顶峰在什么时候呢?顶峰是在 60 年代后期Johnson的时候,Johnson最后一个所谓的那个Great Society 大社会政策。就是我们现在的所谓社会福利保险金都是那个时候立法的。



王孟源 28:01 

但是其实在 60 年代末期大社会政策出台之后就盛极而衰了,那盛极而衰的原因是什么呢?是那些财团当然不甘心被推到一边去,脱离权力的中心、所以他从 50 年代开始收买芝加哥大学的经济系。然后到了 1959 年,有两个很大的财团,我想大家都听说过,一个叫做福特,一个叫做Rockefeller。他们花了大钱投资了美国的几个商学院啊。大家现在觉得商学院,这个是大赚大钱的,要进去很难,这个很重要,能够进去的都很了不起。可是我先跟大家讲一下,在 1959 年之前,美国的商学院就跟现在的教育学院一样,它其实是给大家出去,给一些那个从业人员想要。



唐湘龙 29:09 

混个学位吗?



王孟源 29:11 

混个学位,去休息两三年,打打屁的,一点用,对社会上一点认知都没有,但是这个是什么时候改变呢?就是 1959 年福特跟Rockefeller开始投资,投资之后顺便开始搞公关,这时候搞到 1980 年之后,搞了 20 年人之后就变成你要有 MBA 才能够当企业主管,OK,那当然他们投资的目的不是光光要提升商学院的社会地位,更重要的是要商学院教导那种唯利是图,为私利可以完全不顾公益。他们这个所谓市场万能这个理论在于什么?它的核心就是你不必有意的去照顾公益。,你只要每个人都自私自利,市场就会自然把贪图私利的功利态度转换成功利的最大化。这当然是胡说八道。



王孟源 30:25 

对,你这历史上没有一个正面的实证,但是有很多负面的证据。但是他们如果不这样讲的话,你不先洗脑的话,你怎么能够让选民接受呢?因为你这等于是要把政治权利,然后最终是把要财富从中产阶级手里解放出来,中产阶级的钱包里面解放出来,然后回归到这些财团巨富的口袋里面。所以第一个要先建立一套成套的谎言,言之成理,然后在媒体上反复的放几千、几万次,直到大家都接受为止,这个就是新自由主义的滥伤。然后到了 70 年代的时候,你如果看美国现在知名的这些智库,全都是 70 年代增资或者创立的,OK,那这个就是已经,你可以看到他们一步一步来,先经济系,然后商学院,然后再下一步是智库。到了 80 年他们就把Reagan——台湾叫做雷根,大陆叫做里根——他们送上了白宫,那他这个时候就开始搞这种市场万能这个所谓的 supply side Economics



唐湘龙 31:56 

供给侧经济学。



王孟源 31:59 

它这个就是简单的就是图利财团的政策,那时候还不叫新自由主义。,因为那个时候大家还有很多选民,左派的选民从 60 年代、 70 年代的传统,他们认为这个左派的任务是抵抗资本的侵略,这个也就是正确的认知。所以那老一派的左派还没有死光那,所以他们美国就有一票人,他们看出这里面有一个洞可以钻,可以创一套新的谎言来欺瞒新的一代,这个就是在 1980 年代,这个所谓的新自由主义才真正的心情。



王孟源 32:53 

在这之前有人叫做新自由主义,也有人谈后来新自由主义说的那些东西,但是他们都不重要,也谈这些事的人也不会自称是新自由主义的人,也不是谈这些事情,是 1986 年之后才创立了新自由主义。这个新自由主义是什么?就是政治上讲白左,社会自由,这个什么同性恋、堕胎这些东西,然后什么动物,什么人权、什么女权,但是这个其实都是障眼法。OK,这个它这些指这些的目的,这些主张是哪里来的?这些其实都是左派在 60 年代、 70 年代的次要主张。在 60 年代跟 70 年代的时候,左派的主要主张是跟资本做斗争,OK,但是它有一些次要主张,比如说反核,比如说女权、人权这些东西。那他们就想出一个很妙的办法,你反客为主,把这些次要东西搬到台面上来作为核心,这样就可以偷天换日。把你核心为了争取中产阶级跟资本斗争的那个精神,把它偷偷的扔到一边,都到扔到了市场上去。



王孟源 34:28 

那这样子,雷根当时那种一切为财团利益、为资本利益的政策就可以证明。那刚好 1989 年苏联垮了,就是靠这一套说辞把他说垮的。因为雷根虽然没有走这些左派的社会主义,但是他用的是所谓的自由民主,所以这些雷根的右派这个自由民主这些说辞跟左派这些假左派,所谓白左的这些所谓人权、女权这些东西融合在一起啊。在它融合之后的四年的六年,就 1992 年就很快有一个总统出来,就是Clinton。



王孟源 35:26

Clinton做了两件事情,第一个他把新自由主义实践到彻底。第二个,他实践的时候是把以前——我刚刚讲了半天一个重点是,这一套其实只不过是图利财团的政策、经济管理政策加上一些社会议题的障研法。但是图利财团在20世纪一向是共和党的政策,因为罗斯福是民主党人OK,所以民主党从 30 年代一直到 70 年代一直都是左派,在经济上是左派。这照顾底层民众跟中产阶级——但是Clinton借着这个机会,借着把新自由主义抬上Pedesta,这个祭坛的这个机会,把它也引进了,不只是腐化了整个美国,而且也腐化了民主党。然后与此同时这件事情,同样的事情也在英国发生,就是Blair,也是这样。想想看在 Reagan 搞这个图利资本主义财团的时候, 80 年代英国也是搞同样东西,就是Thatcher 他们也搞同样,所以等于是隔着大西洋两岸一直搞。



王孟源 37:10 

那我在博客有一系列的文章讨论美国是怎么样在这个过程中怎么样发展出新的一套金融收刮的手段。就是从以往靠的纯粹的国力、工业、军事的这种硬实力转化成用金融手段去搜刮全球。,这一共花了 60 年,那我刚刚讲的这个新自由主义就是跟这个转换的这种经济收刮手段配套的政治跟社会理论。



王孟源 37:50 

至于 Neocon ,就是新保守主义,同样的也是,在 20 世纪很早就有一些人自称为新保守主义,但是他们所讲的新保守主义跟我们现在所认知的新保守完全不是一回事,我们现在所讲的新保守主义来源是 1996 年的一篇论文,而那篇论文是一个美国的犹太人写的,那他写的觉得那个目标不是拿去给当时的美国看,不是要给克林顿看,而是要给当时的以色列总理,新总理,Netanyahu那个时候才刚刚当选。



王孟源 38:35 

严格来说 Neocon 新保守主义啊。你说保守跟自由主义照理说应该是对立的。但是其实它很快就结合了。结合以后又结合得不是很好啊。这里的问题就在于,新自由主义是完全是为美国的金融搜刮那一套而配套用的,但是新保守主义却是针对以色列而发展,那个以色列跟周围的阿拉伯国家根本没有忽悠的空间,所以新自由主义基本上是一种软实力忽悠的策略。



王孟源 39:20 

但是新保守主义就是我的拳头大,所以我就去把你打服了就行。那同样的也是 1996 年提出,四年之后就有一个总统上任,然后要采用这个新包子主义,这个就是小布希,他就任用了一大票这些所谓的新保守主义者,那他们的所以一年的主动的挑动了两个战争,一个是伊拉克,一个是阿富汗。那大家都知道这两个战争都拖了将近 20 年,总共的花费有说是5万亿,有说是8万亿。



王孟源 40:03 

OK,这很惨的损失,美国立国以来一直喜欢躲在后面,这也是因为它的这个地缘战略地位太好了,他可以躲在大西洋后面,其他的列强通通是在另外一边打得落化流水。他可以躲在大西洋的后面。所以他其实从来 200 多年从来没有主动跟一级强拳打,但是他倒是有少数的几次。跟二三流的势力打,结果陷入泥潭,这一次在 20 世纪就有好几次,第一次是韩战,第二次是越战。然后他们其实在越战之后郑重的发誓说再也不干这种傻事了,但是他因为新保守主义,这一套其实很不适合美国的东西,一看就知道不适合美国的东西,但是因为他在以色列里面转了一手,那以色列支持以色列的那些犹太人又刚好是美国的深层政府的重要成员,所以他绕了一圈,把小布什给忽悠了,小布什就犯了这个战略错误。那你看这一次的这个俄乌战争,其实已经是改良式的 Neocon,改良式的新保守主义,就是你因为美军自己没有下场,对不对?连北约都没有下。这已经是你如果跟 20 年前的阿富汗跟伊拉克战争相比,你就可以看出他们的确是学乖了一些。



王孟源 41:55 

OK,然后他们也的确是完成圆满的完成了他们的任务啊。在战争的第一年,我曾经估算过,光美国的能源界就多赚了 5, 000 亿美元。美国的军工界没有赚那么多,可能我估计是大约后来的这两年半才赚了 5, 000 亿。那然后你再加上金融界所赚的也基本上就是……然后还有除了欧盟,除了欧元挡在美元前面承受通胀之外,德国还因为切断了跟俄国天然气的这种贸易关系而丧失了很多工业产能,这些很多大部分的工业产能都转移到美国去了。像化工之类。



王孟源 42:55 

所以美国这一次的确是赚得盆满钵满。所以他们经过 20 年的实验,理解到要怎么样把新保守主义融合到他们这一套金融搜刮,软实力赚钱的体系里面啊。这个是我想要跟大家讲清楚,以后我再讲新自由主义跟新保守主义,大家也想了解一下,就是基本上美国从 1970 年代打破布雷顿森林协议体系,然后让美元开始浮动之后,它就转为金融收刮,到目前为止一共有一二三四五,五次大的潮汐作用,基本上是每十年一次,那跟他跟这个搜刮的国际搜刮的配套的社会跟政治理论就叫做新自由主义。



王孟源 44:06 

那新保守主义是外加的军事手段,这个军事手段他们在第一次尝试的时候太粗暴、太直接了,所以他们吃了亏。就是 20 年前在阿富汗跟尼亚克泽国在 2022 年的俄乌战争,他们就把它修正了,根据着这个金融收刮的这种间接原则把它改良了,那所以我们现在看到的是这种新殖民主义的一种终极态势,那我想讲到历史的终极,我想大家都还记得福山曾经在二十年前写过……



唐湘龙 44:50 

法兰西斯福山



王孟源 44:51 

历史的终结,他所指的历史的终结就是新自由主义,因为那时候新保守主义还是刚刚一篇论文还没有被见过,所以今天有关新自由主义跟新保守主义就聊到这里。那没事。



唐湘龙 45:08 

我要再问个问题,就是我们如何去理解所谓的新保守主义的这个论文?它对于今天,因为那里面其实是有Netanyahu本身的一个参与或者说实践的作用,他是如何影响到今天的以色列?他跟以色列的关系是一个怎么样的关系?



王孟源 45:29 

那篇论文其实就是,原本……他们虽然是美国的智库,但是他是写给当时的以色列新总理的,他的就是说,但是以色列当然没有力量自己独自去打服阿拉伯,他至少在后勤跟财政上要依靠美国的事情。所以他的意思就是说你追寻这种军事冒险策略,但是当然是要把美国拉在后面负责擦屁股。



王孟源 46:04 

那这个他其实是当初写的时候,他大概也没有想到四年之后小布什上任,他的官员就会把这个概念,这个战略概念更推进一步,变成美国自己直接上场,就是以色列根本不用出面,我美国自己直接帮你去把伊拉克跟阿富汗给打服。你要大家回忆一下 20 多年前美国人在侵略伊拉克之前,他们的那个纽约时报里面的那些社论充斥的论调就是我们这一次去打可以在伊拉克建立一个民主的天堂。这个民主的天堂会完全的转化中东,变成一个西方化崇美的社会,没错,从而为以色列建立一个良好的环境。



唐湘龙 47:03 

就解决了文明冲突。



王孟源 47:05 

对,所以你就可以从这件事情上看到它其实就是 1996 年那篇论文的一脉相承。略微的改进的地方在什么呢?就是以色列不用直接出面,美国自己替你,一切替你代劳,以色列可以躲在后面装好人。这个真的是,你如果说有舔狗,那大陆有一句话叫舔狗,就是你去追美女的时候,各种手段,再所不辞,我那时候想到的就是美国可真是一个舔狗,我不想说以色列对他来说,是怎么样一个美女。



唐湘龙 47:47 

对,这个问题就是说就是孟源,孟源在讲到新保守主义,以及我们要理解今天的美国跟以色列的关系,乃至于今天Netanyahu所代表的以色列的保守的这个联合政府,为什么从 11973 年之后绝大部分时间都在执政?因为Netanyahu为什么成为一只九命怪猫,可能都得要从这个新保守主义的论述里面去理解,那才能够看到。



王孟源 48:12 

它有一个内政的问题,以色列本身也是一个很严重的内部分裂,Netanyahu对内的权力稳定,是因为它代表了多数,就是他们内部有一个分裂。知识精英对非知识精英,那这个里面有很大的那个体制性的裂痕,Netanyahu 是代表那个非知识精英,就是那些农民,还有新移民这些人,所以他能这么稳固。对外套的是这个新保守主义,但是对内是因为它代表了这个多数,代表了非知识精英、反知识精英的多数。



唐湘龙 49:05 

好,今天我们前面讲的其实都是政治的意识形态,然后尤其从事一些广泛但是不太容易被一般人所意识到的思想工作。对于实际上表面的政治跟商业的运作影响有多么的巨大,我们其实都是不知不觉当中被工具化的,即使刚孟云提到说,譬如说我在做媒体工作的时候,我们很容易不知不觉成为打手,对了,我也有意识到这一点,所以一直努力保持自己一个独立思考运作,跟尽可能的维持一个简单的人际关系的做法,可它终究不足以去对这个世界的本质面的问题产生影响或者对抗。



唐湘龙 49:50 

因为这就是为什么当我听到赖清德说他要洗涤我们的心灵的时候,我听到觉得非常的恐怖,那个那真太恐怖的一句话了,你知道吗?就是我为什么对孟源今天开的菜单呢?会非常的渴望,就是我听到那句话我非常的惊恐,我知道他想要干什么。



唐湘龙 50:11

好,那接下去我们来最后剩下剩下一点点时间,接下去就要来关注一下减肥的历史,减肥看起来是一个全球运动,这个问题其实我已经好奇很久了,不过孟云今天,今天看来准备不止他个人呢?在这经过的这个过程是整个全球减肥运动风潮当中的一员,可是不只在个人实践,而是减肥在人类的文化史当中所代表的意义。好,这个怎么看?



——————————



王孟源 50:39 

我们首先全球有一个当前世界的公共卫生问题,里面一个很严重的问题,就是肥胖症的泛滥。OK,而这个肥胖症的泛滥并不只是因为我们摄取的考虑,卡路里上升,事实上我们目前的这个卡路里越来越廉价,越来越普及。这其实只是一个次要的因素,我们可以从生理上去研究我们为什么会胖,当然这个你吃多了会胖。但是,是不是所有的卡路里都是一样的呢?绝对不是。有的卡路里增胖的那个效应是其他卡路里的 10 倍。OK,为什么?因为我们是猿类,我们是猴子演化出来的。那在对猴子来说。每年秋天的吃水果是刚好是适合冬天增肥过冬的一个巧合,所以我们演化出这个不是推测,而是你真的可以指出哪一个基因是这样,是负责做这件事情,然后是 1, 400 万年前演化出来的,你连那个基因演化出来的时间都可以讲出来。 1, 400 万,那人类什么时候跟黑猩猩分支呢? 700 万,就 700 万年前,人类跟黑猩猩是一家人,然后在网上还有更原始的人猿类,然后在 1, 400 万年前的时候演化出一个基因,就是我们的新陈代谢的机制对果糖特别敏感,我们一旦吃了果糖以后就会进入增肥状态。注意。你如果要减肥的话,第一个你的卡路里的总摄取量不能太离谱。那这里的均衡的问题就是你如果要持续减肥的话,又不想挨饿,至少我自己不想挨饿了啊。我减肥,我是不想挨饿。



唐湘龙 53:12 

挨饿太痛苦了,这是减肥人物失败的原因啊。



王孟源 53:15 

对对对,所以你要真正说,要说合理的减肥的话,你第一个不能够日常的挨饿,不能说日常挨饿就代表你要摄取足够的卡路里,那这个时候真正重要的就是选择不会让高速增肥的卡路里最糟糕的是什么?最糟糕的有两样,一个是酒精,一个就是果糖。



唐湘龙 53:40 

可是果糖不是普遍在我们的像水果这类东西里面吗?



王孟源 53:45 

水果里面的果糖其实不多。



唐湘龙 53:47 

OK,好。



王孟源 53:48 

我们演化的时候是因为水果里面的果糖,所以引发出这种对果糖的这个机制,但是现在这个肥胖症的果糖摄取过量,其实是 1960 年代有一个日本人发明了从玉米里面分离出,用酵素把玉米的淀粉转化成果糖,这个时候你所制成的这个玉米糖浆,它的同样甜度比砂糖要便宜 3 倍。然后。刚好。70 年代的时候有一大堆心血管疾病,那时候大家就说,哦,这是因为胆固醇,所以从 70 年代、 80 年代的时候,他们大家就说,好,不能够吃太多的胆固醇,不能吃胆固醇就是不能够吃高脂肪。其实脂肪增肥的效应不大,它有,但是不大。像我在吃的话牛排我高兴在吃,我每天都要吃,每天的早餐都是Brioche, Brioche是什么东西?奶油面包,发式的奶油面包。脂肪你只要适量,真正的你要减肥,最不痛苦和效果最好的就这个性价比最好的,就是戒糖。戒是什么?戒玉米糖浆,还有砂糖。为什么呢?砂糖是那个所谓的蔗糖,虽然好像不是果糖,但是它一你一放到嘴巴里面,你的那个唾液里面就会有酵素,它转化成果糖。所以你虽然吃的是蔗糖,实际上还是摄取的还是果糖。所以 2019 年我被我妈妈骂,因为那时候我离婚了,那几年就没有再照顾自己的饮食什么的。



王孟源 55:51 

然后 2019 年回去省亲的时候被妈妈骂说你怎么那么胖?刚好我的博客上面有一篇文章,就是讨论我刚刚讲的这些道理为肥胖症的来源。而且你这样一肥胖,蔗糖不止造成肥胖,而且蔗糖跟果糖不但造成肥胖,而且马上就变成糖尿病。OK,糖尿病以后,然后就会有一大堆并发症,这是非常严重的公共健康问题。



王孟源 56:21 

那我刚刚讲到。除了砂糖本身就已经很糟糕了,就是蔗糖的本身很糟糕了,你果糖又比他还要便宜 3 倍。那刚好在 70 年代, 80 年代又有减脂的这个问题,那我们现在很多东西都是预制的食物,比如说像冷冻食品。或者是包装好的,预先做好的食品,那这些食品工厂,他们作为一个商人,他们要生意好就要口感好,最基本的是什么?加脂肪我们就会喜欢吃,因为这是里面的考虑多,那 70 年代、 80 年代不让你加脂肪,那你就要加什么?还有什么东西我们会自然的觉得口感好甜味。所以他们就开始加果糖。你现在的那些预制的食品,尤其在美国那里的食品里面,那个果糖加的一塌糊涂。这还不算什么,最糟糕的是糖水,全世界肥胖症最严重的地方都是喝可乐喝的最多的地方。



王孟源 57:36 

所以,哈哈哈哈哈。最严重的地方比如说美国,已经是美国人已经很肥了,但是墨西哥人比他还要肥。为什么?墨西哥人他天气更热,所以他喝的可乐更多。我小孩子还年纪还小的时候,我带他去麦当劳,他就问我说这个会不会不健康?我说汉堡根本就没有什么不好的地方,其实很平衡的,里面有淀粉,里面有蛋白质,还有Cheese乳制品什么都有,很好,还有青菜。这个比较不好的是你那个油炸的那个薯条,但是即使那个你偶尔吃一下也没有什么不好。麦当劳吃了会胖、会肥,会不健康,最大的问题是那里的那个糖水。可乐你如果到麦当劳去吃,不喝那些加糖的水的话,一点问题都没有。其实就是一个很正常的餐饮。所以你如果去看现在世界上肥胖症最严重的地方,除了美国,美国已经很严重了,但是比美国更严重的还有这些地方,像什么?像我刚刚讲的墨西哥,沙乌地阿拉伯,中东国家,为什么他们那地方不但热,而且还不让你喝酒。



唐湘龙 58:56 

所以只好喝糖水。冰的糖水。



王孟源 59:00 

喝糖水。



唐湘龙 59:04 

对,假装不喝酒,只好去喝糖水。



王孟源 59:07 

我2019 年暑假回来以后,我第一件事情就是戒糖,戒糖以后半年就掉了 20 磅。 20 磅是多少?



唐湘龙 59:17 

快 5 公斤,不止,到 10 公斤。



王孟源 59:22 

将近 10 公斤了。对,就是我体重的12\%,我原本我的体脂肪率超过30\%,一下就掉到20\%。对,而且不必饿肚子。很简单。



唐湘龙 59:35 

戒糖就好了。但是。



王孟源 59:37 

我真正放弃的就是糕饼,还有那个冰淇淋。



唐湘龙 59:42 

而且这个可是这个很美味,我在台湾,我很难抗拒的是糕饼,比如冰淇淋我可以抗拒,可是我对台湾的糕饼我还真的很难抗拒。



王孟源 59:53 

对,这个就看人了,像我原本就不喜欢吃巧克力蛋糕之类的东西,所以对我来说没有什么太大的



唐湘龙 01:00:01 

我特别喜欢你们台南人的那个Sang dia的时候的那种的大饼,Dua Biang,你知道吗?那个饼都有超有吸引力的。



王孟源 01:00:12 

甜的蛋糕是很糟糕的,还有糖水,你要是这两样能够戒掉,基本上你减肥已经做到 50\% 了。,然后前天我在煮饭的时候。刚好我去年回台湾的时候,从台湾带回来一罐味精,我就加了一点味精,然后我忽然想起来这个全球。肥胖症。上去的过程之中,哪一些国家是例外?原本大家以为法国是例外,但是法国最近这 10 年肥胖症也出来了。OK,那为什么是这样子?因为法国人。他们,原本他们吃饭是喝酒,他们不喝可乐,OK。但是慢慢的也美国化了他,而且他们吃甜点吃的很多,所以他们对肥胖症的抵抗力还没有太高。真正有抵抗力的大家很佩服的就是东亚这些国家。,OK,尤其是日本。那我在用味精炒菜的时候,忽然想起来这个东亚还有日本的饮食的一个特征是什么?就是我们特别讲究这个鲜味里面,尤其喜欢放味精。OK。我刚刚。提到。



王孟源 01:01:36 

你要让口感好,最简单的加脂肪、加糖,但其实还有一样东西是什么?味精,,对不对?那我们为什么东亚的饮食文化对全球肥胖症比较有抵抗力?其实就是因为味精的关系。因为我们可以加味精,不需要加果糖,所以我们是果糖的摄取量的上升速度就没有像美国、欧美的社会那么厉害。所以我们才能够。但是有抵抗力并不代表免疫。我觉得这种是一种公共卫生的议题,这个政府应该要首先多做一点公共教育。所以我那个时候,当时我写那篇智库的文章,其实是想要献给台湾政府或者高雄政府。



唐湘龙 01:02:38 

哈哈哈,想太多了。



王孟源 01:02:41 

2019 年的时候要献给高雄市政府。结果没有机会,但是其实是很合理的,你应该要教导群众不要喝糖水。你即使要喝甜的东西的话,也有零卡路里的,没有果糖的那种人工甜味剂,那个只要不要喝太多,其实要比果糖或者蔗糖要健康很多。你把这个公共教育做下去以后就可以大幅消减糖尿病。那糖尿病真的不只是影响寿命,而且是影响生活品质,对不对?它的这个并发症太多了,真的很糟糕。那所以我觉得我自己已经尝试过了,发现实在是很有效,而且一个人只是一个资料点。这个是做过双盲实验,大规模的双盲实验反复验证过的,而且它这个增肥的机制,我刚刚说过它的整个这个生物化学的生医化学的这个回环路,这个新陈代谢的环路都已经是搞清楚了。连你控制那个回路的那个基因都已经早就定出来了,所以这个东西是完全没有异议的。



王孟源 01:04:08 

它的整个的那个生理机制是完全你被证实的,所以现在只等的是政府为了公共卫生的公共利益呢。,然后做教育,然后甚至做强硬的措施,比如说你规定贩卖的、公开贩卖的饮料里面的果糖哈量不能够超过多少的,多少能,那你的糕饼里面能够用甜味剂去替代砂糖的,就部分替代,在不影响它那个味道的前提之下,能够尽量减少一般人的蔗糖摄取量,因为我知道有些人他们就是所谓的CT,这个美国人所说的那个甜牙齿,他们就是美饭,没有那个甜甜的糕点,那就吃不下去。,那你其实都是有替代剂的,像我现在我喝咖啡里面我会放一点点的那个甜味剂。是,那个叫做什么?我的化学不太好。



(编注,根据博客内容,推测为三氯蔗糖Sucralose,商业化产品为Splenda)



唐湘龙 01:05:21 

当然像是阿斯巴甜这种东西啊。



王孟源 01:05:24 

阿斯巴甜这种是比较早期的,我现在用的是比较新的这些,反正他们基本上都是零卡路里的替代,而且你只要不要用的过量都是没有什么危害的。反而是我说过有人去算那个对新陈代谢的危害,一克的果糖它的危害跟一克的酒精是等同的。酒精其实已经是一种毒了,你喝下去以后脸会潮红,其实你可以看得出它是一种毒药。那你可以想象果糖其实是对新陈代谢的危害,是等同酒精的。



王孟源 01:06:07 

好,你会知道。



唐湘龙 01:06:08 

不过不过不管是酒精或者是糖,它自主的存在是因为它让你的肉体产生一种快乐感,甜味非常快乐,因为这是情节,情绪问题,跟你和喝酒一样,酒基本上面不管你借酒浇愁等等是什么理由,他总是给你带来一种的松弛感跟快乐感。糖也是一样,虽然今天刚跟孟源开玩笑讲说我在听他的减肥经,不过我们在讲的是肥胖症在整个的人类历史当中,尤其现在全球公共卫生课题的时候,它是一个很严肃的问题,许多的公共资源最后都耗在你看不见的,因为糖摄取过量所导致的公共的公共的这种卫生当中的风险。



唐湘龙 01:06:54 

好,当然虽然我们平常讲说减肥是一个全球运动,不论大部分都是基于美观,基于基于个人的这种外观的审美学上的考量。可是肥胖本身它更大的麻烦在于它其实是个严肃的公共卫生的课题,是跟你健康有关的,而不是美观的这么的单纯而已。可是我看到全球的糖的摄取量还在节节高升,全球的糖并没有减量,你从消费总认知它一定用在哪某些地方,我们现在可怕就在于糖在你的食物里面,它的渗透率太高了,你根本看不见,而像果糖这种东西你根本不容易注意,像台湾的街头巷尾,你是台湾人,台南最厉害是什么?台南最厉害就是卖糖水,那台南台的迈腾是这样子,里面的每一个下的都是果果糖,那个一个下的下去的时候。



唐湘龙 01:07:49 

好,这些当然有重要的参考价值了,希望大家在注意到自己的健康的时候,特别如果你有这种减肥,上面考量说减肥如何减的健康,不会觉得是自虐,然后又能够有效的减下来。你必须从刚刚讲的从一个什么果糖、酒精的这个时候处理跟对待看起,特别是果糖酒,你会有防范,因为你起码知道你喝的是酒吗?可是果糖就不是果糖呢?在你的食物当中的渗透率真的太高了,蔗糖也一样。好。今天。



王孟源 01:08:21 

讲一件有趣的事情。



唐湘龙 01:08:22 

你说。



王孟源 01:08:25 

我很轻松的减肥有成之后自己也很得意,然后就跟我的儿子吹牛,然后他就说我们来比一比,谁能够先把体脂率减到 20\% 以下,然后我们就开始比,结果比了几个月以后我们就放弃了,你知道为什么吗?因为我看到一篇报道说现在的这个美国的种猪,至少在吃了瘦肉精之后,它的体脂率只有16\%。我估计了一下,我如果不绝食,就是饿肚子的话,我不可能降到16\%以下。



唐湘龙 01:09:10 

不是,你只要继续吃,在吃美国的猪肉,吃美国的牛肉里面的瘦肉精会让你减得下来的。就是我在台湾上跟大家讲你,其实我们都在不知不觉把自己当做是种猪在打败种猪。



王孟源 01:09:27 

而且现在连猪的那个体脂率都只有16\%,大家想一想,那个我们现在最肥的其实是人,说什么像个肥猪,其实是不对的。



唐湘龙 01:09:38 

所以说猪想很肥的人那才可怕。



好使时间的关系,感谢孟源。来,我感谢一些我们的观众朋友,看看有没有观众朋友的提问可以请教梦雨来普拉德福 K 001,谢谢你FL,感谢。他说孟源老师上次提到皇帝用中医反而短命的证据是不充分的,他说这些皇帝是死于正统中医治疗,还是道家的丹汞,他们是否完全遵照医嘱?中医包括针灸、中药,然后这个把脉正骨都是伪科学嘛。他说我和亲朋好友通过中医所治疗的疾病,他相信中医并不是假的。相比于西医,他认为中医的诊断比较主观,缺少医疗仪器的辅助,医师的随意性比较大,容易误诊。可是好的中医很难遇到,他们大部分都有大量的临床。因此中医虽然有一些不好的中医,或者这个坑蒙拐骗不过,但是副作用比较小,对增强人体的抵抗力。他认为中西医各有所长,他不能够接受王孟源片面的否定中医。



王孟源 01:10:41 

我解释一下。第一个在中国古代这个所谓的铅汞丹道跟中医是没有分割的。OK,你如果所以他们自己都不分割,我没有必要去替他分割啊。你如果非要分割的话,小孩子总不会去炼丹嘛,对不对?当然我的博客上面也举过例子了,乾隆的 17 个皇子,而且当时的皇子是要活过一岁,一周岁才算是皇子,否则就是夭折不算。就是乾隆一共有 17 个活过一岁的皇子,他的平均寿命是 27 岁。他有十个活过一周岁的皇女,她的平均寿命是 18 岁。那我跟你讲,这里面大概没有人是因为吃丹汞,没有炼丹死的,对不对?没有一个是丹毒而死的,但是我跟你保证每一个都是用了好几个中医来照顾,结果他的平均寿命只有十几二十岁。



王孟源 01:11:50 

这就是很简单的事实。而且这个是我们的老祖宗过去两三千年身体力行。,实践了几百亿次的综合结果,OK,所以,要是要忽视,要是要曲解,都是对祖宗祖先的不尊重,也是对事实的不尊重,OK。



唐湘龙 01:12:16 

好,来,再来破论,感谢。然后冯一伟,他说英国的毒血事件再次证明了市场的局限性。好在了90K,谢谢你。然后彭龙英,谢谢他说那个王孟源老师今天讲的美国的故事很精彩。对了,我们就从美国的政治实际运作的那个表象,我们去看底层。不过就是最少美国的舆论圈子里面在做政治斗争的时候,不断的在质疑在政治的,在美国的就是说华盛顿特区的底下有一个 deep state,那个理解是没有问题的,只是如何让那个理解,让更多人可以知道那个 deep state 的它的运作的逻辑,以及它是如何掌控这个世界的?那个那个是有困难。



王孟源 01:13:02 

这就是为什么我们会有一个很好像是矛盾。就是Trump这个人张口就来,他基本上撒谎成性,可是为什么。大家还会受到他所谓的Fake news 这种指控而鼓舞呢?就是因为美国的主流媒体他们好像在小事上没有说谎,但是它的主旨是说谎。没没,就像我今天所举的那些例子,对不对?他明明他有跟你讲这个众议院辞职的,但是他没有跟你讲这个众议院辞职是为了要通过援乌法案。



唐湘龙 01:13:41 

没有错,可怕,这是一种非常的可怕。



王孟源 01:13:44 

他小事情没有撒谎,但是他们背后的要讲的那个大的picture、大局观跟那个脉络,他撒谎了。



唐湘龙 01:13:54 

对,所以我们在做新闻工作的时候,就除了避免就是要能够过滤谎言之外,能够让内幕出来是非常重要的。就事实跟内幕,在新闻工作理解一个事件的时候同等重要。可大部分的时候要同时理解这两件事情非常困难,你要理解事实已经很困难,你要去掌握内幕会变得更困难,有很多的说谎是用内幕去包装的。好,再来,小明同学,来,小明同学里面提到想请教王老师怎么看大陆最新的房地产政策,会有效吗?有学者的文章说这关乎到了中美博弈的走势,您怎么看?



王孟源 01:14:39 

我对国际金融战略有 100\% 绝对的论断,OK,我对中国国内的金融管理的原则也有 100\% 绝对的论断。但是。对于就业跟经济管理的细节我是遵从,有一分确定说一分话,因为我自己不是专门研究这个的,他们中国政府内部决策的考虑事实上都是保密的。你任何一个网络上公开讨论的人都不知道,OK?不知道就不应该谈。所以我曾经说过韩非子有一句话我很喜欢,不知而言不智,知而不言不忠。OK,我知道的事情我一定会讲好,只要是对国家有益的、公益有益的事情一定会讲,但我不知道的事情我不会妄作评论。就是当初那个易纲,他在人民币汇率上面卖国的时候。,我也是一样,先假设他有内部的考虑,一直到所有的借口都被排除了,也就是到事后,我才敢公开批评,对不对?我事先只讲说,我认为应该这样做。是,但是他为什么不这样做?我不知道,可能有他自己的原因。在事后。完全排除了任何借口之后,我才公开的批评呢。



唐湘龙 01:16:20 

好,我们再来看这个 giant peanut 这个大花生 giant peanut,他说很好奇 WSAAP 瓦斯跟犹太人的利益也是共通的吗?那看爱尔兰还是非常反游的,因为爱尔兰毕竟才刚承认了巴勒斯坦,那发展到现在,到底谁是主人谁是狗?好,再来玉婷永婷婷好,他说为了这个我王孟源的借堂的工位的战争了而 Dolly 感谢好再来,绝感谢。他说世卫组织估计 2050 年地球上大约有 13 亿的人口会罹患糖尿病,检糖会是各个国家重要的国安政策。我推荐一种糖叫做阿拉伯糖,一种在日本蛋糕店常用的控血糖的糖。你知道这玩意吗?



王孟源 01:17:12 

我不知道,我有我自己用的,哈哈哈。



唐湘龙 01:17:15 

然后我下次碰到你的时候你再推荐给我好了。好,今天的时间的关系,非常感谢孟源,因为他跟我们的时区是相反的,所以他总是利用晚上的时间,而且我知道他每次在准备个准备工作的时候,他的自我要求非常高,所以我坦白说,我之所以每次期待他,就是洗耳恭听的时候,对我来讲是非常有收获的。好,感谢孟源的连连线,感谢。



王孟源 01:17:43 

很高兴跟大家聊天。



唐湘龙 01:17:44 

好,下个月再下个月再见,再麻烦,拜。




 \end{document}